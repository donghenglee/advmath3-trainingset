\pagecolor{PageColor}
\
\vfil
\hfil  {\fontsize{50pt}{36pt}\selectfont{组合数学}} \hfil
\vfil
\begin{tikzpicture}[remember picture,overlay,every node/.style={inner sep=0pt}]
        \node [shift={(1cm,-1cm)},brown,scale=2,anchor=north west] (CNW)
        at (current page.north west) {\pgfornament[height=1cm,width=1cm]{61}};
        \node [shift={(-1cm,-1cm)},brown,scale=2,anchor=north east] (CNE)
        at (current page.north east) {\pgfornament[height=1cm,width=1cm,symmetry=v]{61}};
        \node [shift={(1cm,1cm)},brown,scale=2,anchor=south west] (CSW)
        at (current page.south west) {\pgfornament[height=1cm,width=1cm,symmetry=h]{61}};
        \node [shift={(-1cm,1cm)},brown,scale=2,anchor=south east] (CSE)
        at (current page.south east) {\pgfornament[height=1cm,width=1cm,symmetry=c]{61}};
        \pgfornamentline[color=brown]{current page.north west}{current page.north east}{2}{87}
        \pgfornamentline{current page.south west}{current page.south east}{2}{87}
        \pgfornamentline{current page.north west}{current page.south west}{3}{87}
        \pgfornamentline{current page.north east}{current page.south east}{3}{87}
        \end{tikzpicture}%
\thispagestyle{empty}
\pagebreak

\begin{center}
  {\fontsize{30pt}{26pt}\selectfont
    \hypertarget{排列与组合}{排列与组合} \label{排列与组合}
  }
\end{center}
\separator
\vspace{1pt}
\nopagecolor
\begin{questions}
    \question 求满足以下条件的三位数的个数:  
    \begin{itemize}
        \item 三个数字各不相同
        \item 数字按递减顺序排列
        \item 其中一个数字是5
    \end{itemize}  
    \begin{solution}
        即先选数字$5$,再从$9$个数字中任选$2$个,所以有 
        \[
        \comb{9}{2}=36
        \]
        个这样的三位数。注意到每组都能唯一地按递减顺序排成一个三位数,而且首位不会是 $0$。
    \end{solution}

    \question 已知 $a,b,c$ 为相异正整数且满足 $abc=2310$,求所有可能相异集合 $\{a,b,c\}$的个数。
    \begin{solution}
        发现$2310 = 2\times 3\times 5\times 7\times 11$,从 $5$ 个因数中,
        \begin{itemize}
            \item 挑 3 个连乘积作为 $a$,剩下 2 个作为 $b,c$,有
            $\comb{5}{3} = 10$种可能。
            \item 挑 2 个连乘积作为 $a$,剩下 3 个再挑 2 个乘积作为 $b$,有 $\comb{5}{2} \cdot \comb{3}{2} = 30$
            种可能。
        \end{itemize}
        故共有$10 + 30 = 40$种可能。
    \end{solution}

    \question $A$ 有一本共有 $2017$ 页的书,页码从 $1$ 到 $2017$。问有多少个页码同时包含至少一个数字 $1$ 和至少一个数字 $9$?例如 $91,1921,191$ 都符合条件。
    \begin{solution}
        情况一:页码为二位数。只有$19,91$满足要求,共2个页码符合要求。

        情况二:页码为三位数。页码中必有数字$1,9$,排列数为$3\cdot 2\cdot 10$,但此时多算了页码$119,191,199,911,919,991,019,091$,故符合要求的页码共有$3\cdot 2\cdot 10-6=52$。

        情况二:页码为三位数。千位必为数字1,而数字9可位于百位、十位或各位,排列数为$3\cdot 10\cdot 10$,但此时多考虑了含两个数字9的页码,共$3\cdot 10$。减去其之后,发现又少算了页码1999。故符合要求的页码共有$3\cdot 10\cdot 10-3\cdot 10+1=271$。

        故共有
        \[
        2+52+271=325
        \]
        个页码符合条件。
    \end{solution}

    \question 设 $N$ 为满足 $x < y < z$ 且
    \[
    xyz = 2^2 \cdot 3^2 \cdot 5^2 \cdot 7^2 \cdot 11^2 \cdot 13^2 \cdot 17^2 \cdot 19^2
    \]
    的正整数三元组 $(x, y, z)$ 的个数。求 $N$。
    \begin{solution}
        忽略$x < y < z$,先计算无序三元组 $(a,b,c)$的个数,使得
        \[
        abc = 2^2 \cdot 3^2 \cdot 5^2 \cdot 7^2 \cdot 11^2 \cdot 13^2 \cdot 17^2 \cdot 19^2
        \]
        每个平方的质因数可以分配给 $a,b,c$ 的方式有 6 种:两个都给 $a$,两个都给 $b$,两个都给 $c$,或者每个数各取一个。共有 8 个平方质因数,因此共有 $6^8$ 种分配方式。

        现考虑$a,b,c$中两个数相等的情况。观察到积$abc$不是完全立方数,所以$a=b=c$不成立。计算恰有一对相等的三元组:以 $a=b$ 为例,每个平方质因数分配给 $c$,亦或分配给 $a$ 和 $b$ 各一个,共 $2^8$ 种三元组需要排除,同理$a=c,b=c$ 也各有 $2^8$ 种,所以相异的三元组数为:
        \[
        6^8 - 3\cdot 2^8
        \]
        将无序三元组 $(a,b,c)$ 转换为有序三元组 $(x,y,z)$,其中$x<y<z$,每个三元组对应 6 个无序三元组,因此
        \[
        N = \frac{1}{6}(6^8 - 3\cdot 2^8) = 279808
        \]
    \end{solution}

    \question 已知 $a_0, a_1, a_2, a_3, a_4$ 的值正好都是 $-1, 0, 1$ 中的数, 则 $a_0 + 3a_1 + 3^2a_2 + 3^3a_3 + 3^4a_4$ 的值是正整数共有多少个?
    \begin{solution}
        形如
        \[
        1\triangle \triangle \triangle \triangle,01\triangle \triangle \triangle,
        001\triangle \triangle,
        0001\triangle,
        00001
        \]
        的数共有
        \[
        3^4,3^3,3^2,3^1,1
        \]
        个,总计
        \[
        81 + 27 + 9 + 3 + 1 = 121
        \]
    \end{solution}

    \question 求正整数有序三元组 $(x,y,z)$ 的个数,使得
    \[
    xyz = 4000.
    \]
    \begin{solution}
        质因数分解得$4000 = 2^5 \cdot 5^3$,设
        \[
        x = 2^a 5^d, \quad y = 2^b 5^e, \quad z = 2^c 5^f,
        \]
        其中 $a,b,c,d,e,f$ 为非负整数,则需满足
        \[
        a+b+c = 5, \quad d+e+f = 3
        \]
        非负整数解的个数为
        \[
        \#(a,b,c) = \comb{5+3-1}{3-1} = 21, \quad \#(d,e,f) = \comb{3+3-1}{3-1} = 10
        \]
        因此有序三元组 $(x,y,z)$ 的个数为
        \[
        21 \cdot 10 = 210
        \]
    \end{solution}

    \question 将 $1,2,3,4,5,6,7$ 排成一列,若规定排列后不得出现 $12,23,34,45,56,67$(如:1273546 不合题意,7362154 符合题意),则有多少种排法?
    \begin{solution}
    7 个相异数字任意排列共有 $7!$ 种。

    从$12,23,34,45,56,67$选$12$后有6个板块,出现$12$的排列数为$\comb{6}{1}6!$。

    依此类推,由容斥原理,总排列数为
    \[
    7!-\comb{6}{1}6!+\comb{6}{2}5!-\comb{6}{3}4!+\comb{6}{4}3!-\comb{6}{5}2!+1=2119
    \]
    \end{solution}

    \question 如果一个数的每一位都大于前一位,则称它为上升数。例如 457 是上升数,但 447 不是。问 400 到 5000 之间共有多少上升数?
    \begin{solution}
        情况一:上升数为三位数。三位数必须在 400 到 999 之间,百位可以是 4,5,6,7,8,9。4 到 9 的 6 个数字中任选 3 个,按升序排列即可得到上升数。排列数为$\comb{6}{3}$。

        情况二:上升数为四位数。四位数小于 5000,千位必须为 1,2,3,4。1 到 9 的数字中任选 4 个,再扣除首位 $\ge 5$ 的数。排列数为$\comb{9}{4} - \comb{5}{4}$。

        因此共有 
        \[
        \comb{6}{3} + (\comb{9}{4} - \comb{5}{4}) = 141
        \]
        个这样的上升数。
    \end{solution}

    \question Ricardo 想要将三个 1、三个 2、两个 3 和一个 4 排成一个九位正整数,且满足以下条件:
    \begin{itemize}
        \item 从左到右,至少有一个 1 在第一个 2 之前,至少有一个 2 在第一个 3 之前,至少有一个 3 在 4 之前;
        \item 任意两个数字 2 不能相邻。
    \end{itemize}
    求总共有多少种符合条件的九位数。
    \begin{solution}
        设 $N$ 为满足条件的整数。$N$ 的首位必须是 1,因此 $N$ 可以以 1、11 或 111 开头。又因为第一个非 1 的数字必须是 2,所以 $N$ 只能以 12、112 或 1112 开头。
        
        情况 1:$N$ 以 12 开头。剩余两个 2 不能相邻,可放在下列位置组合(从左数第几位):
        \[
        (4,6),(4,7),(4,8),(4,9),(5,7),(5,8),(5,9),(6,8),(6,9),(7,9)
        \]
        共有 10 种可能。剩余两个 1 可以放在剩下的 5 个空位中的任意两位,有 $\comb{5}{2}=10$ 种方法。剩余的两个 3 和一个 4 需放在最后 3 个位置中。为保证至少有一个 3 在 4 之前,第一个空位必须放 3,剩下两个位置放另一个 3 和一个 4,有 2 种排列。因此本情况共有:
        \[
        10 \cdot 10 \cdot 2 = 200.
        \]
        情况 2:$N$ 以 112 开头。剩余两个 2 不能相邻,可放在:
        \[
        (5,7),(5,8),(5,9),(6,8),(6,9),(7,9)
        \]
        共有 6 种可能。剩余一个 1 可放在 4 个空位中的任意一位,有 4 种方法。剩余两个 3 和一个 4 需放在最后 3 个位置中。第一个空位放 3,剩下两个位置放另一个 3 和一个 4,有 2 种排列。本情况共有:
        \[
        6 \cdot 4 \cdot 2 = 48.
        \]
        情况 3:$N$ 以 1112 开头。剩余两个 2 不能相邻,可放在
        \[
        (6,8),(6,9),(7,9)
        \]
        共有 3 种可能。剩余的两个 3 和一个 4 需放在最后 3 个位置中。第一个空位放 3,剩下两个位置放另一个 3 和一个 4,有 2 种排列。本情况共有:
        \[
        3 \cdot 2 = 6.
        \]
        综上,共有 $200 + 48 + 6 = 254$ 种符合条件的九位数。
    \end{solution}

    \question
3) 从数字1至9中选出7位数,要求每个数字不重复,并且5和6不能连续出现,求共有多少种排列方式。

\begin{solution}
考虑不同情况:

(a) 5和6都出现:

- 情况一:第1位是5,第2位不是6:
\[
5 \times \perm{7}{5} = 12600
\]

- 情况二:第7位是5,第6位不是6:
\[
5 \times \perm{7}{5} = 12600
\]

- 情况三:5出现在首尾之外的其他位置:
\[
5 \times 4 \times \perm{7}{5} = 50400
\]

因此,5和6都出现的排列数:
\[
12600 + 12600 + 50400 = 75600
\]

(b) 5和6都不出现:
\[
\perm{7}{7} = 7! = 5040
\]

(c) 5出现但6不出现:
\[
7 \times \perm{7}{6} = 7 \times 7! = 35280
\]

(d) 5不出现但6可以出现:
\[
7 \times \perm{7}{6} = 7 \times 7! = 35280
\]

根据加法原理,总排列数为:
\[
75600 + 5040 + 2 \times 35280 = 151200
\]
\end{solution}


    \question 令 $a_n$ 为第 $n$ 小的各位数字之和为 $3$ 的正整数。例如 $a_1=3, a_2=12, a_3=21, a_4=30$。求 $a_{2012}$ 有多少位数。  
    \begin{solution}
        若一个数最多有 $d$ 位,那么其各位和为 $3$ 的这样的数的个数为
        \[
        \comb{d+2}{3}
        \]
        这是因为我们可以用隔板法来理解:将 $3$ 个相同的球(代表数字和为 3)分配到 $d$ 个位置(代表 $d$ 位数),等价于在 $d+2$ 个位置中选择 $3$ 个位置放置隔板。具体地,我们在一排 $d+2$ 个位置中放置恰好 $3$ 个 O(其余为 X),则第 $i$ 位的数字等于第 $i-1$ 个 X 与第 $i$ 个 X 之间的 O 的个数。因此需找最小的 $d$ 使得
        \[
        \comb{d+2}{3}\ge 2012.
        \]
        计算得
        \[
        \comb{23}{3}=1771,\quad \comb{24}{3}=2024.
        \]
        因此当 $d=22$ 时,$\comb{24}{3}=2024\ge2012$,且 $d=21$ 时不足。故 $a_{2012}$ 恰有 $22$ 位。
    \end{solution}

    \question 将$A,B,C,D,E,F,G,H$八个字母排成一列,使得$B$在$A$之右方,$E$在$C$与$D$之间,且$F,G$不相邻,试问符合条件的排法有多少种?
    \begin{solution}
        在所有的排列中,$B$ 在 $A$ 的右方与 $A$ 在 $B$ 的右方各占一半,因此满足"$B$ 在 $A$ 的右方"的排列占全部排列的 $\dfrac{1}{2}$。
        
        同理,在 $C,D,E$ 三个字母的相对位置中,$E$ 在 $C$ 与 $D$ 之间、$C$ 在 $E$ 与 $D$ 之间、$D$ 在 $E$ 与 $C$ 之间各占三分之一,因此满足"$E$ 在 $C$ 与 $D$ 之间"的排列占全部排列的 $\dfrac{1}{3}$。
        
        现计算满足"$F,G$ 不相邻"的排列数:
        
        八个字母的全排列有 $8!$ 种。若 $F,G$ 相邻,可将它们视为一个整体,有 $7!$ 种排法,而 $F,G$ 内部有 $2$ 种排列,因此 $F,G$ 相邻的排法有 $7! \cdot 2$ 种。
        
        所以 $F,G$ 不相邻的排法有 $8! - 7! \cdot 2$ 种。
        
        由于这三个条件相互独立,符合所有条件的排法数为:
        \[
        \frac{8! - 7! \cdot 2}{2 \cdot 3} = 5040
        \]
    \end{solution}

    \question 有 6 个人有网络账号,已知每个人都恰好与自己以外的 2 个人互为好友,则共有种不同的组成方法?
    \begin{solution}
        用六个顶点表示六个人,两顶点有连线表示两人互为好友。

        情况一:一个六边形。共有$\dfrac{5!}{2}$种。

        情况二:两个三角形。共有$\dfrac{\comb{6}{3}}{2}$种。

        因此共有
        \[
        \frac{5!}{2}+\frac{\comb{6}{3}}{2}=70
        \]
        种组合方式。
    \end{solution}

\question 求不大于 $2018$ 的正整数中,二进制中 1 出现比 0 多的个数。
\begin{solution}
    考虑不大于 $2047=2^{11}-1$且满足条件的正整数个数,再减去 2019 至 2047 的 29 个数,因为这29 个数的二进制中至少有6个1。

    对于偶数位数(去掉开头的 1 后有偶数位),一半的数满足 1 的个数多于 0。具体公式为:
    \[
    \frac{1}{2}\left(2^{2k} + \comb{2k}{k}\right)
    \]
    因为恰好 $k$ 个 1 的情况算一半,其余一半满足条件,计算:
    \[
    \frac{1}{2}\Big(2047 + \comb{0}{0} + \comb{2}{1} + \comb{4}{2} + \comb{6}{3} + \comb{8}{4} + \comb{10}{5}\Big) = 1199
    \]
    故答案为$1199-29=1170$。
    \textcolor{red}{Half of the numbers with an even number of bits have this property, since they have an odd number of bits after their initial 1. Of the $2^{2k}$ numbers with $2k$ bits following the initial 1, the number with this property is $\frac{1}{2}(2^{2k} + \binom{2k}{k})$ since those with $k$ 1's will be included, and half of the others will.}
\end{solution}

    \question 小王有八个编号为 1 到 8 的盒子和八个编号为 1 到 8 的球。问他把球放入盒子,使每个盒子恰好有一个球,且球 1 不在盒子 1,球 2 不在盒子 2,球 3 不在盒子 3,有多少种方法?
    \begin{solution}
        设
        \[
        A_1 = \{\text{球 1 在盒子 1}\}, \quad
        A_2 = \{\text{球 2 在盒子 2}\}, \quad
        A_3 = \{\text{球 3 在盒子 3}\}.
        \]
        根据容斥原理,不允许的排列数为
        \begin{align*}
        &|A_1 \cup A_2 \cup A_3| \\
        &= |A_1| + |A_2| + |A_3| - |A_1 \cap A_2| - |A_1 \cap A_3| - |A_2 \cap A_3| + |A_1 \cap A_2 \cap A_3| \\
        &=3\cdot 7! - 3\cdot 6! + 5! = 13080           
        \end{align*}
        因此满足条件的排列数为
        \[
        S = 8! - |A_1 \cup A_2 \cup A_3| = 27240
        \]
    \end{solution}

    \question 在一排有 20 张椅子的座位区中, 要安排甲、乙、丙、丁、戊 5 人入坐, 一人坐一张椅子, 要求第一张与最后一张椅子不能安排人入坐, 且每相邻的 5 张椅子至少要有一人入坐, 任两人不能坐在相邻的椅子上。试问 5 人入坐的方法有多少种可能?
    \begin{solution}
    20 张椅子 5 人先入坐, 有 $5!$ 排法, 剩下 15 张椅子。依规定头尾各放一张, 剩下 13 张; 两人中间各摆一张, 剩下 9 张。

    9 张椅子可以在 5 人的六个间隔中摆放, 有 $\comb{9+6-1}{6-1}$ 摆法。

    为符合「相邻 5 张椅子至少要有一人坐」, 需扣除某个间隔塞入四张或以上的情形,因此实际摆法有
    \[
    5!(\comb{9+6-1}{6-1} - \comb{6}{1} \cdot \comb{6+5-1}{5-1} + \comb{6}{2} \cdot \comb{6+1-1}{1-1})= 69600
    \]
    \end{solution}

    \question 由字母 $A$ 和 $B$ 组成的 9 个字母的字母串中,有多少个不包含连续字母串 $ABBA$?
    \begin{solution}
        总字母串数为 $2^9 = 512$,我们先计算至少包含一个 $ABBA$ 的字符串数量,再用总数减去它。

        $ABBA$ 可以出现在前 6 个位置中的任意一个位置,剩余 5 个位置可以随意填充 $A$ 或 $B$,因此每个起始位置有 $2^5 = 32$ 种方法,共 $6 \cdot 32 = 192$ 种字母串。  

        这 192 种字母串中有些被重复计算,因为 $ABBA$ 可以重叠如下:
        \[
        xyABBABBA, \quad xABBABBAy, \quad ABBABBAxy
        \]
        其中 $x, y$ 可以是 $A$ 或 $B$,共有 $12$ 种。还有 6 种形为
        \[
        xABBAABBA, \quad ABBAxABBA, \quad ABBAABBAx
        \]
        这些被重复计算了 2 次。  

        根据容斥原理,因此至少包含一个 $ABBA$ 的字母串数为 $192-12-6 = 174$。故不包含 $ABBA$ 的字母串数为:
        \[
        512 - 174 = 338
        \]
    \end{solution}

    \question 设 Bauman 字串满足以下条件:
    \begin{itemize}
        \item 每个字母只能是 $A, B, C, D, E$ 之一;
        \item 相邻两个字母不能相同。
    \end{itemize}
    例如 $AECD, BDCEC$ 是 Bauman 字串,而 $ABBC, DAEEE$ 不是。
    \begin{parts}
    \part 长度为 5 的 Bauman 字串中,首尾字母都是 $A$ 的有多少个?
    \begin{solution}
        若首尾均为 $A$,则第二位和第四位不能是 $A$。按第三位分类讨论:

        情形 1:第三位是 $A$。字串形如 $A\_A\_A$,第二位有 4 种选择($B,C,D,E$),第四位有 4 种选择,共 $4 \cdot 4 = 16$ 种。

        情形 2:第三位不是 $A$。第三位有 4 种选择($B,C,D,E$),第二位需与第三位不同且不能为 $A$,共 3 种选择;第四位同理有 3 种选择,共 $4 \cdot 3 \cdot 3 = 36$ 种。

        总数为 $16 + 36 = 52$。
    \end{solution}
    \begin{solution}
        按第二位和第四位是否相同分类:

        情形 1:第二位和第四位相同。字串形如 $Ax\_xA$(其中 $x$ 表示第二位和第四位的字母),第二位有 4 种选择,第三位需与第二位不同,有 4 种选择,共 $4 \cdot 4 = 16$ 种。

        情形 2:第二位和第四位不同。字串形如 $Axy\_A$(其中 $x$, $y$ 分别表示第二位和第四位的字母),第二位有 4 种选择,第四位需与第二位不同,有 3 种选择,第三位需与第二位和第四位都不同,有 3 种选择,共 $4 \cdot 3 \cdot 3 = 36$ 种。

        总数为 $16 + 36 = 52$。
    \end{solution}
    \part 长度为 6 的 Bauman 字串中,包含超过一个 B 的有多少个?
    \begin{solution}
        总字串数为 $5 \cdot 4^5 = 5120$(首位 5 种选择,其余每位 4 种选择)。

        无 $B$ 的字串:首位 4 种选择($A,C,D,E$),其余每位 3 种选择(需与前一位不同且不能是 $B$),共 $4 \cdot 3^5 = 972$ 种。

        恰有 1 个 $B$ 的字串:
        \begin{itemize}
            \item 若 $B$ 在首位或末位:
            \begin{itemize}
                \item $B$ 在首位:第二位 4 种,其余每位 3 种,共 $1 \cdot 4 \cdot 3^4 = 324$ 种
                \item $B$ 在末位:首位 4 种,第二到第五位中第二位 3 种,其余每位 3 种,共 $4 \cdot 3^4 = 324$ 种
            \end{itemize}
            \item 若 $B$ 在第 2 到第 5 位(共 4 个位置):每个位置,$B$ 前一位 4 种,$B$ 后一位 4 种,其余 3 个位置各 3 种,每个位置贡献 $4 \cdot 1 \cdot 4 \cdot 3^3 = 432$ 种,共 $4 \cdot 432 = 1728$ 种
        \end{itemize}
        恰有 1 个 $B$ 的总数为 $324+ 324 + 1728 = 2376$。

        超过 1 个 $B$ 的数目为:
        \[
        5120 - 972 - 2376 = 1772.
        \]
    \end{solution}
    \begin{solution}
       长度为 6 的 Bauman 字串中,由于相邻字母不能相同,$B$ 最多出现 3 次。故只需计算恰有 2 个 $B$ 或恰有 3 个 $B$ 的字串数。

        情形 1:恰有 3 个 $B$。三个 $B$ 必须间隔放置。可能的位置模式有:
        \begin{itemize}
            \item $B\_B\_B\_$(第 1, 3, 5 位):第 2, 4, 6 位各 4 种选择,共 $4^3 = 64$ 种
            \item $\_B\_B\_B$(第 2, 4, 6 位):第 1, 3, 5 位各 4 种选择,共 $4^3 = 64$ 种
            \item $B\_B\_B$(第 1, 3, 5 位):第 2 位 4 种,第 4 位 4 种,第 6 位 3 种,共 $4 \cdot 4 \cdot 3 = 48$ 种
            \item $\_B\_B\_$(第 2, 4, 6 位):第 1 位 4 种,第 3 位 4 种,第 5 位 3 种,共 $4 \cdot 4 \cdot 3 = 48$ 种
        \end{itemize}
        合计 $64 + 64 + 48 + 48 = 224$ 种。

        情形 2:恰有 2 个 $B$。从 6 个位置中选 2 个放 $B$,有 $\comb{6}{2} = 15$ 种选择,但需排除相邻的 5 种,剩余 10 种位置模式。

        逐一计数(用下划线表示非 $B$ 位置):
        \begin{itemize}
            \item $B\_\_\_B\_$:第 2 位 4 种,第 4 位 4 种,第 5, 6 位各 3 种,共 $4 \cdot 4 \cdot 3 \cdot 3 = 144$ 种
            \item $\_\_B\_\_B$:对称,同样 $144$ 种
            \item $B\_\_B\_\_$:第 2, 4 位各 4 种,第 5 位需与第 4 位不同有 3 种,第 6 位 3 种,共 $4 \cdot 4 \cdot 3 \cdot 3 = 144$ 种
            \item $\_\_B\_B\_$:对称,同样 $144$ 种
            \item $B\_B\_\_\_$:第 2 位 4 种,第 4, 5, 6 位各 3 种,共 $4 \cdot 3 \cdot 3 \cdot 3 = 108$ 种
            \item $\_\_\_B\_B$:对称,第 1 位 4 种,第 2, 3 位各 3 种,第 5 位 3 种,共 $4 \cdot 3 \cdot 3 \cdot 3 = 108$ 种
            \item $B\_B\_\_\_$:第 2 位 4 种,第 4 位需与第 3 位不同有 4 种,第 5, 6 位各 3 种,共 $4 \cdot 4 \cdot 3 \cdot 3 = 144$ 种
            \item $\_\_B\_B$:对称,同样 $144$ 种
            \item $B\_\_\_\_B$:第 2 位 4 种,第 3, 4, 5 位各 3 种,共 $4 \cdot 3^3 = 108$ 种
            \item $\_B\_B\_\_$:第 1, 3, 5 位各 4 种,第 6 位 3 种,共 $4^3 \cdot 3 = 192$ 种
        \end{itemize}
        合计 $4 \cdot 144 + 2 \cdot 108 + 2 \cdot 144 + 108 + 192 = 1548$ 种。

        恰 2 个 $B$ 的有 1548 种,恰 3 个 $B$ 的有 224 种,总和为 $1548 + 224 = 1772$。
    \end{solution}
    \end{parts}


    \question 将字母串 $AAAABBBCCC$ 排成一列,且相同字母不相邻的排法有多少种?
    \begin{solution}
        按 $B,C$ 的排列分类,再插入 $A$。先排列 3 个 $B$ 和 3 个 $C$ 的相对次序,记为「$BC$ 串」。共有
        \[
        \comb{6}{3}=\frac{6!}{3!3!}=20
        \]
        种 $BC$ 串。按 $BC$ 串中出现相邻同字母的"块数"分类,列举如下:
        \begin{center}
        \begin{tabular}{c|l}
        编号 & 字串 \\ \hline 
        1 & $BCBCBC$ \\ 
        2 & $CBCBCB$ \\ \hline 
        3 & $BCCBCB$ \\ 
        4 & $CBCBBC$ \\ 
        5 & $BCBCCB$ \\ 
        6 & $CBBCBC$ \\ \hline 
        7 & $BCCBBC$ \\ 
        8 & $CCBBCB$ \\ 
        9 & $BCBBCC$ \\ 
        10 & $CCBCBB$ \\ 
        11 & $BBCCBC$ \\ 
        12 & $CBBCCB$ \\ 
        13 & $BBCBCC$ \\ 
        14 & $CBCCBB$ \\ \hline 
        15 & $CBBBCC$ \\ 
        16 & $CCBBBC$ \\ 
        17 & $BBCCCB$ \\ 
        18 & $BCCCBB$ \\ \hline 
        19 & $CCCBBB$ \\ 
        20 & $BBBCCC$ \\ \hline
        \end{tabular}
        \end{center}
        现在将 4 个 $A$ 插入 $BC$ 串的空位中,使得任何相同字母都不相邻:
        \begin{enumerate}
            \item $BC$ 串中没有相邻同字母(编号 1--2):有 7 个空位插入 4 个 A,方式数 $\comb{7}{4}=35$,对应 2 种 $BC$ 串,共 $2\cdot 35=70$。
            \item 有 1 处相邻同字母(编号 3--6):此时剩 6 个空位插入 4 个 A(等价插入 3 个 A 以避免产生相邻相同字母),方式数 $\comb{6}{3}=20$,对应 4 种 $BC$ 串,共 $4\cdot20=80$。
            \item 有 2 处相邻同字母(编号 7--14):剩 5 个空位插入 2 个 A,方式数 $\comb{5}{2}=10$,对应 8 种 $BC$ 串,共 $8\cdot10=80$。
            \item 有 3 处相邻同字母(编号 15--18):剩 4 个空位插入 1 个 A,方式数 $\comb{4}{1}=4$,对应 4 种 $BC$ 串,共 $4\cdot4=16$。
            \item 有 4 处相邻同字母(编号 19--20):A 恰好插满一组,计 $2$ 种。
        \end{enumerate}
        合计:
        \[
        70+80+80+16+2=248
        \]
    \end{solution}
    \begin{solution}
        根据容斥原理,相同字母不相邻的排法数
        \[
        = \text{4个A不相邻} - (\text{至少2个B相邻} \cup \text{至少2个C相邻})
        \]
        4个A不相邻:先排 3 个 $B$ 和 3 个 $C$,有 $\comb{6}{3}=20$ 种方式,产生 7 个空位,在其中插入 4 个 $A$,有 $\comb{7}{4}=35$ 种方式,共 $20 \cdot 35=700$ 种。

        至少2个B相邻:将 2 个 $B$ 看作一个整体 $BB$,与剩余 1 个 $B$ 和 3 个 $C$ 排列,有 $\frac{5!}{3!}=20$ 种方式,产生 6 个空位插入 4 个 $A$,有 $\comb{6}{4}=15$ 种方式,共 $20 \cdot 15=300$ 种。

        但这样会重复计算 3 个 $B$ 都相邻的情况。当 3 个 $B$ 都相邻时,将它们看作 $BBB$,与 3 个 $C$ 排列有 4 种方式,产生 5 个空位插入 4 个 $A$,有 $\comb{5}{4}=5$ 种方式,共 $4 \cdot 5=20$ 种。

        因此至少 2 个 $B$ 相邻为 $300-20=280$ 种。

        由对称性,至少 2 个 $C$ 相邻也是 $280$ 种。

        至少2个B相邻且至少2个C相邻:将 2 个 $B$ 看作 $BB$,2 个 $C$ 看作 $CC$,与剩余 1 个 $B$ 和 1 个 $C$ 排列,有 $4!=24$ 种方式,产生 5 个空位插入 4 个 $A$,有 $\comb{5}{4}=5$ 种方式,共 $24 \cdot 5=120$ 种。

        需扣除 3 个 $B$ 都相邻的情况:$BBB,CC,C$ 排列有 $3!=6$ 种,产生 4 个空位插入 4 个 $A$,有 $\comb{4}{4}=1$ 种,共 $6 \cdot 1=6$ 种。

        同样扣除 3 个 $C$ 都相邻的情况:$6$ 种。

        因此至少 2 个 $B$ 相邻且至少 2 个 $C$ 相邻为 $120-6-6=108$ 种。

        最终答案为 
        \[
        \comb{6}{3}\comb{7}{4} - \left[2 \times \left(\frac{5!}{3!}\comb{6}{4}-4\comb{5}{4}\right) - \left(4!\comb{5}{4}-2 \times 3!\comb{4}{4}\right)\right]=248
        \]
    \end{solution}

    \question 要爬 12 级台阶,但每步只能上 1 级或 2 级。第 8 级有一条蛇,所以不能踩。问共有多少种爬法?
    \begin{solution}
        由于第 8 级不能踩,必须从第 7 级迈 2 级直接到第 9 级。于是问题可以拆成两段:
        \begin{itemize}
            \item 爬前 7 级的方式数
            \item 从第 9 级到第 12 级的方式数
        \end{itemize}
        总数为
        \[
        (\comb{7}{0}+\comb{6}{1}+\comb{5}{2}+\comb{4}{3})\cdot 3=63
        \]
    \end{solution}

    \question 在集合 $\{1,2,3,4,5,6,7\}$ 的所有非空子集 S 中,有多少个子集不包含数 $|S|$($|S|$ 表示子集元素个数)?例如 $\{3,4\}$ 是这样的子集,因为它不包含数字 2。
    \begin{solution}
        非空子集总数为 $2^7-1=127$。若子集大小为 $k$ 且包含数字 $k$,则剩下的 $k-1$ 个元素必须从另外的 6 个数中选取,共有
        \[
        \comb{6}{k-1}
        \]
        种。对 $k=1,2,\dots,7$ 求和得包含其大小的子集总数为
        \[
        \sum_{k=1}^7 \comb{6}{k-1} = \sum_{j=0}^6 \comb{6}{j} = 2^6 = 64.
        \]
        因此不包含其大小的子集个数为
        \[
        127-64=63.
        \]
    \end{solution}

    \question 有多少个非负整数有序四元组 $(a,b,c,d)$ 满足 $a+b+c+d\le15$?
    \begin{solution}
        将不等式转化为等式
        \[
        a+b+c+d+e=15,
        \]
        其中 $e\ge0$,则原题等价于求此方程的非负整数解个数。根据隔板法,有
        \[
        \comb{15+5-1}{5-1}=\comb{19}{4}=3876
        \]
        个非负整数有序四元组。
    \end{solution}

    \question 以一个正方体的顶点为顶点的四面体共有多少个?
    \begin{solution}
        从正方体的 8 个顶点中任取 4 点:
        \[
        \comb{8}{4} = 70
        \]
        需扣除 4 点共平面的情形,包括正方体的六个面,共 6 种;斜平面,共 6 种。所以四面体个数为
        \[
        70 - 12 = 58
        \]
    \end{solution}

    \question 在 5 只蜗牛进行的比赛中,最多只会出现一次打平,但可以接受任意数量的蜗牛打平。例如,比赛结果可能是 Dazzler 获得第一名;Abby、Cyrus 和 Elroy 并列第二名,而 Bruna 获得第五名。问这种比赛共有多少种不同的结果? 
    \begin{solution}
        考虑没有、两只、三只、四只、五只蜗牛打平,总共有
        \[
        5! + \comb{5}{2} \cdot 4! + \comb{5}{3} \cdot 3! + \comb{5}{4} \cdot 2! + 1 = 431
        \]
        个不同的结果。
    \end{solution}

    \question 在一个 $3 \times 3$ 的网格中,放置 $3$ 个红色棋子、$3$ 个蓝色棋子和 $3$ 个绿色棋子,所有棋子同色不可相邻(横或竖方向),且棋子同色不可区分。请问满足条件的放置方法有多少种?
        \ifprintanswers
        \begin{figure}[H]
        \centering
        \includegraphics[width=0.4\textwidth]{images/image13.png}
        \end{figure}
        \fi
    \begin{solution}
        观察发现,只存在两种基本的放置方法(如上图所示),其他放置方法只能通过这两种结构旋转或颜色变化得到,所以总合法方案数为
        \[
        (4 + 2) \cdot 3! = 36
        \]
    \end{solution}

    \question 小明在注册账号时可以使用字符$1,2,3,a,b,c,A,B,C$来组成五位密码,但要求必须包含数字、小写字母和大写字母,且不可以出现两个相同的字符相邻,例如密码可以设置为$123aA$或$laA12$,但不能设置为$123ab$或$112aA$,试求可以设置不同的密码的个数。
    \begin{solution}
        先考虑相邻字符不同的密码,共有$9 \cdot 8^4 = 36864$种,这里面不满足密码要求的有两类:
        \begin{itemize}
            \item 仅包含单一字符类型(如全数字),这类共有$3 \cdot (3 \cdot 2^4) = 144$种
            \item 仅包含两种字符类型(如数字和小写字母),只满足相邻不同的密码有$6 \cdot 5^4 = 3750$种,但此时我们多算了$2 \cdot (3 \cdot 2^4) = 96$种单一字符类型, 故第二类共有$3 \cdot (3750 - 96) = 10962$种
        \end{itemize}
        不同密码的总个数为$36864 - 144 - 10962 = 25758$.
    \end{solution}

    \question 阿绿想在她的表演服装上缝 6 颗相同的红色钮扣、3 颗相同的绿色钮扣和 3 颗相同的黄色钮扣。若所有钮扣需竖直地排成一直线,且相邻钮扣不同色,则阿绿有多少种排列方法?
    \begin{solution}
        假设红、绿、黄钮扣为$R,G,Y$,先将 $6R$ 排成一列,共 $7$ 个间隔。由于同色不相邻,必须将中间 $5$ 个间隔放入 $3G$ 与 $3Y$ 中的五个。

        情况一: 5 个间隔放 $3G2Y$ 且剩下 $1Y$ 放头或尾端。排列数有$\comb{5}{3}\cdot 2$。

        情况二: 5 个间隔放 $2G3Y$ 且剩下 $1G$ 放头或尾端。排列数有$\comb{5}{3}\cdot 2$。

        情况三: 5 个间隔放 $3G2Y$ 且剩下 $1Y$ 不放头尾。可能情况有$G\cdot G\cdot Y\cdot Y\cdot \boxed{GY},G\cdot G\cdot Y\cdot Y\cdot \boxed{YG}$,排列数各有 $\dfrac{5!}{2!2!}$。

        故排列方法共有
        \[
        \comb{5}{3}\cdot 2+\comb{5}{3}\cdot 2+\frac{5!}{2!2!}+\frac{5!}{2!2!}=100
        \]
    \end{solution}

    
    \question 在一个 $3\times3$ 的方格中,每个小方格被涂成红、白、蓝、绿四种颜色中的一种,要求任何 $2\times2$ 的小方格块都恰好包含四种不同的颜色。共有多少种不同的涂色方案?
    \begin{solution}
        先固定中心格:中心格有 $4$ 种颜色可选。再考虑中心行的左右两格:
        \begin{itemize}
        \item 两格同色。共有 $3$ 种颜色可供选择,此时上下行个别有$2$种填色方法,共 $2^2=4$ 种合法填法。  
        \item 两格异色。左格 $3$ 种选,右格再 $2$ 种选,共 $3\cdot2=6$ 种。在此情况下,其余 4 个角格的颜色被唯一地确定。
        \end{itemize}
        故方法数共有 
        \[
        4 \cdot (3 \cdot 4 + 6) =72
        \]
    \end{solution}

    \question Rita 正在给一朵花涂色。她已经涂好了花心和花茎。接下来,她将用红色、橙色、黄色和蓝色给六片花瓣涂色,每片花瓣只用一种颜色。规则如下:
    \begin{itemize}
        \item 相邻的花瓣不能涂相同的颜色;
        \item 不一定要用到所有四种颜色。
    \end{itemize}
    求共有多少种不同的涂法数?
    \begin{figure}[H]
    \centering
    \includegraphics[width=0.3\textwidth]{images/image170.png}
    \end{figure}
    \begin{solution}
        不失一般性,假设顶端花瓣为红色。

        情况 1:红色出现三次。由于相邻花瓣不能同色,红色只能出现在特定的三片花瓣上,其余三片花瓣可以任意用其他3种颜色涂色,因此这种情况下共有
        \[
        3^3 = 27 \text{ 种涂法。}
        \]
        情况 2:红色出现两次。分情况讨论,
        \begin{itemize}
            \item 两红色花瓣之间相隔两个花瓣。与顶端红色花瓣相邻的两片花瓣可以各自用3种颜色涂色,剩余两片花瓣可以各自用2种颜色涂色,总共有
            \[
            3 \cdot 3 \cdot 2 \cdot 2 = 36 \text{ 种方式。}
            \]
            \item 两红色花瓣之间相隔一个花瓣。两红色花瓣可能在顶端及左侧或右侧,红色花瓣之间的花瓣可以用3种颜色,其他三片花瓣根据两种或三种颜色组合有 $6+6=12$ 种方式,因此该配置共有
            \[
            2 \cdot 3 \cdot 12 = 36 \text{ 种方式。}
            \]
        \end{itemize}
        情况 3:红色出现一次。按顺时针给花瓣涂色(从红色顶端开始):第一片花瓣有 3 种选择,接下来的四片花瓣每片有 2 种选择(不能和相邻花瓣相同,也不能用红色),因此共有
        \[
        3 \cdot 2 \cdot 2 \cdot 2 \cdot 2 = 48 \text{ 种方式。}
        \]
        顶端花瓣可以选择 4 种颜色,故
        \[
        N = 4 \cdot (27 + (36+36) + 48) = 732
        \]
    \end{solution}

    \question 一位农夫拥有一个矩形农田,它被划分成 $2\times2$ 的四块矩形区域。在每块区域中,农夫将种植一种作物:玉米、小麦、大豆或土豆。农夫不希望玉米和小麦出现在相邻的两块地中,也不希望大豆和土豆出现在相邻的两块地中。在这些限制下,农夫有多少种不同的方法为这四块地选择作物?
    \begin{solution}
        注意到:对于每种作物,有恰好一种作物是不能与其相邻的。不失一般性,设左上角种的是小麦。

        情况一:左上角右边和下边两个相邻地块种的是相同的作物。这两块地不能种玉米(因为玉米不能与小麦相邻),因此它们可以是小麦、大豆或土豆中的一种,共有 $3$ 种选择,对于右下角地块,只要与周围的两块地都不冲突即可,有 $3$ 种选择。
        
        情况二:右边和下边两块地种的是不同的作物。可从 $3$ 种合法作物中选择两个不同的,并安排在右和下两个位置,共有 $3 \cdot 2 = 6$ 种方式,此时右下角地块的选择只有 $2$ 种。
        
        综合两种情况相加并乘以左上角作物的 $4$ 种选择,总方法数为
        \[
        4 \cdot (3 \cdot 3 + 6 \cdot 2) = 84 
        \]
    \end{solution}
            
\question 三双不同的鞋子被排成一列,要求不能有一只左脚鞋与另一双的右脚鞋相邻。问共有多少种排法?
\begin{solution}
    设鞋子编号如下:右脚鞋:$1,\ 2,\ 3$,左脚鞋:$4,\ 5,\ 6$,其中 $n$ 与 $n+3$ 配对(例如 $1$ 和 $4$ 是一对)。

    不妨设右脚鞋 $1$ 出现在排列中的第二个位置。此时我们枚举前三个位置可能的情况:
    \begin{itemize}
        \item 剩下的三个鞋子只能是
            \[
            654 \quad \text{或} \quad 645
            \]
            因为 $5$ 或 $4$ 不可以和 $3$(另一双的右脚鞋)相邻。
        \item 剩下的三个鞋子只能是
            \[
            563
            \]
            (不能选 $365$,因为 $6$ 和 $5$ 是不同双鞋的左、右脚,会相邻)
        \item 剩下的三个鞋子可以是
            \[
            256 \quad \text{或} \quad 652
            \]
    \end{itemize}
    所以总方法数为
    \[
    (2 + 1 + 2) \times 12 = \boxed{60}
    \]
    其中乘上 $12$ 是因为我们一开始假设了鞋子 $1$ 出现在第二位,实际有 $2$ 种选哪个右脚鞋先出现,$6$ 种选配对顺序,所以总共乘 $12$。
    \textcolor{red}{实际有 $2$ 种选哪个右脚鞋先出现?}
\end{solution}

    \question 竹东高中的多元选修课程共开设了六门选修课:$A,B,C $为第一类选修课,$D,E,F$ 为第二类选修课,要求每名同学须从中选修三门课,第一类选修课至少要选两门。现有甲、乙、丙三位同学选课,则任意一位同学与其他两位同学均至少有两门相同选修课的选法共有几种?
    \begin{solution}
        每个同学可能的选法有$$ABC, ABD, ABE, ABF, ACD, ACE, ACF, BCD, BCE, BCF,$$共 10 种;甲、乙、丙三人共有 $10^3=1000$ 种选法。

        不符规定的选法分析如下:
        \begin{itemize}
            \item 第一人选了 $ABC$:第二人在第一类有 $\comb{3}{2}=3$ 种选法,第三人在第一类只能与第二人重复一门课,因此有 2 种选法;第二人在第二类的课程有 3 种选择,第三人有 2 种选择,共有 $3\cdot2 \cdot 3\cdot2 =108$ 种。
            \item 第一人、第二人在第一类完全相同:第三人在第一类只有 2 种选择,在第二类课程中三人任选再扣除三人完全相同,即 $3^3-3$;共有 $\comb{3}{2} \cdot 3 \cdot 2 \cdot (3^3-3) =432$ 种。
            \item 三人在第一类彼此最多重复一门课:第一类的选择有 $3!=6$,第二类与情况(2)相同,即 $3^3-3$;共有 $6 \cdot (3^3-3) =144$ 种。
        \end{itemize}
        因此符合要求的选法共有
        \[
        1000-108-432-144 = 316
        \]
    \end{solution}

    \question 下图显示了一个宽为3格、高为3格的点阵,共包含9个小正方形。Carl 在这些正方形的边上放置长为 \(\dfrac{1}{2}\) 英寸的牙签,要求形成一个不相交的闭合环。图中某些格子内标注了一个数字,表示该格子被牙签覆盖的边数。若无数字,则该格子可以被任意数量的牙签覆盖。求 Carl 放置牙签的可能方法数。  
    \begin{figure}[H]
        \centering
        \includegraphics[width=0.4\textwidth]{images/image7.png}
    \end{figure}
    \begin{solution}
        由于牙签回路不能穿过中间的五列,所以整个回路只有两种大致布局:

        情况一:不穿过第二行。只剩下 $2$ 种回路(对称)。
        
        情况一:必须穿过第二行(左、右各穿一次)。则
        \begin{itemize}
          \item 中间四个标 $1$ 的小方格,各有两种放牙签方式,共 $2^4$;
          \item 回路在左右两侧穿过第二行,各有 $3$ 种走法,共 $3^2$。
        \end{itemize}
        因此所有合法回路总数为
        \[
        2^4 \cdot 3^2 + 2 = 146
        \]
    \end{solution}

    \question 有一张由 $5\times6$ 个正方形组成的格线纸,如右图。小强想沿着实线以向左、向右及向下的方向将格线纸剪成两张面积相等的纸张,并且先由 $A$ 点向下剪一格,最后从 $B,C,D,E,F$ 中某一点剪断纸张。问有多少种不同的剪法?
    \begin{figure}[H]
        \centering
        \includegraphics[width=0.3\textwidth]{images/image95.png}
    \end{figure}
    \ifprintanswers
    \begin{figure}[H]
        \centering
        \includegraphics[width=0.4\textwidth]{images/image96.jpg}
    \end{figure}
    \fi
    \begin{solution}
        格线纸共有 $5 \times 6 = 30$ 个方格。设剪线左侧从上往下第 $1,2,3,4,5$ 行的方格数分别为 $2,x,y,z,w$,需满足
        \[
        2 + x + y + z + w = 15,\quad 1 \le x,y,z,w \le 5, \quad x,y,z,w \in \mathbb{N}
        \]
        令 $x' = x - 1, y' = y - 1, z' = z - 1, w' = w - 1$,则
        \[
        x' + y' + z' + w' = 9, \quad 0 \le x',y',z',w' \le 4
        \]
        总方法数即不考虑$0 \le x',y',z',w' \le 4$的情况扣除 $x',y',z',w'$ 中至少有一个大于 $4$ 的情况:
        \[
        \comb{9+4-1}{4-1} - 4 \cdot \comb{4+4-1}{4-1} = 80
        \]
    \end{solution}

    \question 考虑集合 $\{1,2,3,\dots,2024\}$ 的所有恰有 1000 个元素的子集。对于每个这样的子集 $S$,记 $m(S)$ 为 $S$ 中的最小元素。求所有 $m(S)$ 的算术平均数。
    \begin{solution}
        设最小元素为 $a$,则有选择余下 $999$ 个元素的方法数为
        \[
        \comb{2024 - a}{999}, \quad a=1,2,\dots,1025 \quad (1025=2024-999).
        \]
        算术平均数为
        \[
        \frac{\displaystyle \sum_{a=1}^{1025} a \cdot \comb{2024 - a}{999}}{\displaystyle\sum_{a=1}^{1025} \comb{2024 - a}{999}}=\frac{\comb{2025}{1001}}{\comb{2024}{1000}} = \frac{2025}{1001}
        \]
        利用组合恒等式 $\sum_{k=r}^{n} \comb{k}{r} = \comb{n+1}{r+1}$,令 $k = 2024-a$,当 $a$ 从 1 到 1025 时,$k$ 从 2023 到 999:
        \[
        \sum_{a=1}^{1025} \comb{2024 - a}{999} = \sum_{k=999}^{2023} \comb{k}{999} = \comb{2024}{1000}
        \]
        利用恒等式:
\[
\sum_{i=r}^{n} \comb{i}{r} = \comb{n+1}{r+1}
\]

以及权重求和恒等式:
\[
\sum_{k=0}^{n} (k+1) \cdot \comb{n-k}{r} = \comb{n+1}{r+2}
\]

令 $k = a-1$,则 $a = k+1$,当 $a$ 从 1 到 1025 时,$k$ 从 0 到 1024:
\[
\sum_{a=1}^{1025} a \cdot \comb{2024-a}{999} = \sum_{k=0}^{1024} (k+1) \cdot \comb{2024-(k+1)}{999} = \sum_{k=0}^{1024} (k+1) \cdot \comb{2023-k}{999}
\]

利用恒等式 $\sum_{k=0}^{n-r} (k+1) \cdot \comb{n-k}{r} = \comb{n+2}{r+2}$:

令 $n = 2023$, $r = 999$:
\[
\sum_{k=0}^{1024} (k+1) \cdot \comb{2023-k}{999} = \comb{2025}{1001}
\]
        \textcolor{red}{(待解)}
    \end{solution}

    \question 用数字 1、2、3 组成 10 位数,要求其中数字 1 出现次数为偶数。求这样的 10 位数的个数。
    \begin{solution}
        设 \(a_n\) 表示长度为 \(n\) 的、由 1、2、3 组成且数字 1 出现偶数次的序列个数。
        \begin{itemize}
            \item 若首位为 2 或 3(共 2 种选择),则余下 \(n-1\) 位仍需 1 出现偶数次,方案数为 \(2a_{n-1}\)。
            \item 若首位为 1(1 种选择),则已有一个 1,余下 \(n-1\) 位中 1 需出现奇数次。余下 \(n-1\) 位的总序列数为 \(3^{n-1}\),其中 1 出现偶数次的为 \(a_{n-1}\),故出现奇数次的为 \(3^{n-1} - a_{n-1}\)。
        \end{itemize}
        因此,当 \(n > 1\) 时,有递推关系:
        \[
        a_n = 2a_{n-1} + (3^{n-1} - a_{n-1}) = a_{n-1} + 3^{n-1}, \quad a_1 = 2
        \]
        求通项
        \[
        a_n = a_1 + \sum_{k=1}^{n-1} 3^k =2 + \frac{3(3^{n-1} - 1)}{2}=\frac{3^n + 1}{2}
        \]
        代入 \(n=10\)得
        \[
        a_{10} = \frac{3^{10} + 1}{2} = 29525
        \]
    \end{solution}

    \question 某语言只使用字母 $A, B, C, D, E$,其中 $A$ 和 $E$ 为元音,$B, C, D$ 为辅音。一个字母序列称为单词,当且仅当它不含有相邻两个相同字母,且不含有相邻两个元音。问该语言中长度为 10 且以元音开头的单词有多少个?
    \begin{solution}
        设 $v_n$ 为长度为 $n$ 且以元音开头的单词数,$c_n$ 为长度为 $n$ 且以辅音开头的单词数。据题意,
        \[
        v_1=2,\ c_1=3
        \]
        若 $n\ge2$,则  
        \[
        v_n=2c_{n-1},
        \]
        因为在辅音开头的 $(n-1)$ 字母单词前可加 $A$ 或 $E$。且有  
        \[
        c_n=3v_{n-1}+2c_{n-1},
        \]
        其中 $3v_{n-1}$ 意味可在元音开头的 $(n-1)$ 字母单词前加 $B, C, D,2c_{n-1}$ 意味可在辅音开头的 $(n-1)$ 字母单词前加不同于首字母的两个辅音。递推如下表:
        \begin{center}
        \begin{tabular}{|c|c|c|}
        \hline
        $n$ & $v_n$ & $c_n$ \\
        \hline
        1 & 2 & 3 \\
        2 & 6 & 12 \\
        3 & 24 & 42 \\
        4 & 84 & 156 \\
        5 & 312 & 564 \\
        6 & 1128 & 2064 \\
        7 & 4128 & 7512 \\
        8 & 15024 & 27408 \\
        9 & 54816 & 99888 \\
        10 & 199776 & 364224 \\
        \hline
        \end{tabular}
        \end{center}
        因此,长度为 10 且以元音开头的单词共有 $199776$ 个。
    \end{solution}

    \question 有六个格子,初始时三个硬币从左到右依次占据前三个格子(记为 $X,Y,Z$)。一次移动是将某一枚硬币向右移动一格,前提是目标格子为空,且硬币之间不能相互跳跃(因此三枚硬币的相对顺序始终保持 $X$ 在 $Y$ 左边,$Y$ 在 $Z$ 左边)。问:有多少种不同的移动序列,能使三枚硬币最终全部移动到最右边的三个格子?
    \begin{figure}[H]
    \centering
    \includegraphics[width=0.3\textwidth]{images/image162.png}
    \end{figure}
    \begin{solution}
        我们将每个允许的移动序列看作由 $X,Y,Z$ 组成的字符串。例如,字符串 $ZZYXZ$ 表示先移动 $Z$ 一格,再移动 $Z,Y,X,Z$。  

        对于每个整数三元组 $(x,y,z)$,其中 $0 \le x,y,z \le 3$,定义 $S(x,y,z)$ 为使 X 移动 $x$ 格,Y 移动 $y$ 格,Z 移动 $z$ 格的移动序列数。欲求 $S(3,3,3)$。  

        由于硬币不能交叉且只能向右移动,当 $x > y$ 或 $y > z$ 或 $x > z$,有$S(x, y, z) = 0$。因此仅需考虑 $0 \le x \le y \le z \le 3$ 的情况。此时有
        \[
        S(x,y,z) = S(x-1,y,z) + S(x,y-1,z) + S(x,y,z-1)
        \]
        其中当 $x=0$ 时 $S(x-1,y,z)=0,y=0$ 时 $S(x,y-1,z)=0,z=0$ 时 $S(x,y,z-1)=0$,且  
        \[
        S(0,0,0) = 1, \quad S(1,0,0) = 0, \quad S(0,1,0) = 0, \quad S(0,0,1) = 1
        \]
        接下来通过表格逐步填入 $S(x, y, z)$,其中 $z = 1,2,3$,纵轴为 $y$,横轴为 $x$。
        \[  
        \begin{tabular}{c c c}
            \begin{tabular}{|c|cccc|}
            \multicolumn{1}{c}{} & \multicolumn{4}{c}{$x$} \\
            \cline{2-5}
            \multicolumn{1}{c|}{$y$} & 0 & 1 & 2 & 3 \\
            \cline{1-5}
            0 & 1 & 0 & 0 & 0 \\
            1 & 1 & 1 & 0 & 0 \\
            2 & 0 & 0 & 0 & 0 \\
            3 & 0 & 0 & 0 & 0 \\
            \cline{1-5}
            \end{tabular} &
            \begin{tabular}{|c|cccc|}
            \multicolumn{1}{c}{} & \multicolumn{4}{c}{$x$} \\
            \cline{2-5}
            \multicolumn{1}{c|}{$y$} & 0 & 1 & 2 & 3 \\
            \cline{1-5}
            0 & 1 & 0 & 0 & 0 \\
            1 & 2 & 3 & 0 & 0 \\
            2 & 2 & 5 & 5 & 0 \\
            3 & 0 & 0 & 0 & 0 \\
            \cline{1-5}
            \end{tabular} &
            \begin{tabular}{|c|cccc|}
            \multicolumn{1}{c}{} & \multicolumn{4}{c}{$x$} \\
            \cline{2-5}
            \multicolumn{1}{c|}{$y$} & 0 & 1 & 2 & 3 \\
            \cline{1-5}
            0 & 1 & 0 & 0 & 0 \\
            1 & 3 & 6 & 0 & 0 \\
            2 & 5 & 16 & 21 & 0 \\
            3 & 5 & 21 & 42 & 42 \\
            \cline{1-5}
            \end{tabular} \\
            $z=1$ & $z=2$ & $z=3$
        \end{tabular}
        \]
        由表得 
        \[
        S(0,0,1)=1(Z),\quad S(0,1,1)=1(ZY),\quad S(1,1,1)=1
        (ZYX)
        \]
        因此序列的数量为$S(3,3,3)=42$。
    \end{solution}

    \question 给定数字集合 $\{1, 2, 4, 8, 16, 32, 64, 128\}$,每个数字最多可使用 $3$ 次,问:用这些数字之和恰好等于 $530$ 的方案数有多少种?
    \begin{solution}
        每个数字 $2^k$ 可以出现 $0,1,2,3$ 次,对应多项式
        \[
        1 + x^{2^k} + x^{2\cdot 2^k} + x^{3\cdot 2^k}.
        \]
        考虑生成函数
        \[
        (1+x+x^2+x^3)(1+x^2+x^4+x^6)(1+x^4+x^8+x^{12}) \cdots (1+x^{128}+x^{256}+x^{384}).
        \]
        因式分解得
        \[
        \frac{x^4-1}{x-1}\cdot\frac{x^8-1}{x^2-1}\cdots \frac{x^{512}-1}{x^{128}-1}
        = (1+x+x^2+\cdots+x^{255})(1+x^2+x^4+\cdots+x^{510}).
        \]
        要得到 $x^{530}$,记作 $x^{2i}\cdot x^{530-2i}$,其中 $10 \le i \le 127$,共有 118 个 $i$ 的取值,因此方法数为 $118$。
    \end{solution}

    \question 求正整数有序三元组 $(a_1, a_2, a_3)$ 的个数,使得
    \[
    a_1 + a_2 + a_3 = 2020, \quad a_1 \not\equiv 0 \pmod 2, \quad a_2 \not\equiv 0 \pmod 3, \quad a_3 \not\equiv 0 \pmod 4。
    \]
    \begin{solution}
        $a_1$ 为奇数,生成函数为
        \[
        x + x^3 + x^5 + \cdots = \frac{1}{1-x} - \frac{1}{1-x^2} = \frac{x}{1-x^2}
        \]
        $a_2$ 不可被 3 整除,生成函数为
        \[
        x + x^2 + x^4 + x^5 + \cdots = \frac{1}{1-x} - \frac{1}{1-x^3} = \frac{x(1-x^2)}{(1-x)(1-x^3)}
        \]
        $a_3$ 不可被 4 整除,生成函数为
        \[
        x + x^2 + x^3 + x^5 + x^6 + x^7 + \cdots = \frac{1}{1-x} - \frac{1}{1-x^4} = \frac{x(1-x^3)}{(1-x)(1-x^4)}
        \]
        三式相乘得总生成函数
        \[
        \frac{x}{1-x^2} \cdot \frac{x(1-x^2)}{(1-x)(1-x^3)} \cdot \frac{x(1-x^3)}{(1-x)(1-x^4)} = \frac{x^3}{(1-x)^2(1-x^4)}
        \]
        因此所求为 $x^{2020}$ 的系数,即 $x^{2017}$ 在
        \[
        (1 + 2x + 3x^2 + \cdots)(1 + x^4 + x^8 + \cdots)
        \]
        中的系数,即
        \[
        2 + 6 + 10 + \cdots + 2018 = 505 \cdot 1010 = 510050
        \]
    \end{solution}

    \question 给定正立方体 $ABCD-EFGH$,其中 $ABCD$ 为底面,$EFGH$ 为顶面,$A$ 与 $E$ 对应,$B$ 与 $F$ 对应,$C$ 与 $G$ 对应,$D$ 与 $H$ 对应。一只小虫从顶点 $A$ 出发,每分钟沿一条棱移动到相邻顶点(每次移动一条棱)。问:经过恰好 $9$ 分钟后,小虫到达顶点 $G$ 的不同路径数为多少?
    \begin{solution}
        转换矩阵为
        \[
        A=
        \begin{bmatrix}
        0 & 1 & 0 & 1 & 1 & 0 & 0 & 0 \\
        1 & 0 & 1 & 0 & 0 & 1 & 0 & 0 \\
        0 & 1 & 0 & 1 & 0 & 0 & 1 & 0 \\
        1 & 0 & 1 & 0 & 0 & 0 & 0 & 1 \\
        1 & 0 & 0 & 0 & 0 & 1 & 0 & 1 \\
        0 & 1 & 0 & 0 & 1 & 0 & 1 & 0 \\
        0 & 0 & 1 & 0 & 0 & 1 & 0 & 1 \\
        0 & 0 & 0 & 1 & 1 & 0 & 1 & 0
        \end{bmatrix}
        \]
        发现
        \[
        A^9=
        \begin{bmatrix}
        0 & 4921 & 0 & 4921 & 4921 & 0 & 4920 & 0 \\
        4921 & 0 & 4921 & 0 & 0 & 4921 & 0 & 4920 \\
        0 & 4921 & 0 & 4921 & 4920 & 0 & 4921 & 0 \\
        4921 & 0 & 4921 & 0 & 0 & 4920 & 0 & 4921 \\
        4921 & 0 & 4920 & 0 & 0 & 4921 & 0 & 4921 \\
        0 & 4921 & 0 & 4920 & 4921 & 0 & 4921 & 0 \\
        4920 & 0 & 4921 & 0 & 0 & 4921 & 0 & 4921 \\
        0 & 4920 & 0 & 4921 & 4921 & 0 & 4921 & 0
        \end{bmatrix}
        \]
        故由
        \[
        A^9
        \begin{bmatrix}
        1 \\ 0 \\ 0 \\ 0 \\ 0 \\ 0 \\ 0 \\ 0
        \end{bmatrix}
        =
        \begin{bmatrix}
        0 \\ 4921 \\ 0 \\ 4921 \\ 4921 \\ 0 \\ 4920 \\ 0
        \end{bmatrix}
        \]
        可知$A \to G$ 有 $4920$ 种路线。
        \textcolor{red}{(待递推解)}
    \end{solution}

\end{questions}

\pagebreak

\begin{center}
  {\fontsize{30pt}{26pt}\selectfont
    \hypertarget{概率、期望值}{概率、期望值} \label{概率、期望值}
  }
\end{center}
\separator
\vspace{1pt}

\begin{questions}
    \question 正八面体有 $8$ 个全等的正三角形面。任取 $3$ 个面,其面心组成的三角形是正三角形的概率是多少?
    \begin{solution}
        正八面体的对偶多面体是正方体,其面心对应正方体的 $8$ 个顶点。  
        等边三角形对应正方体中形如正四面体侧面的三角形,共 $8$ 个,因此概率为
        \[
        \frac{8}{\comb83} = \frac{1}{7}
        \]
    \end{solution}

    \question 给定正八边形 $PQRSTUVW$。从中随机选择 $4$ 条边并将其延长为直线,问:这 $4$ 条直线相交所形成的四边形恰好包含原八边形的概率是多少?
    \begin{solution}
        选择的四条边相邻,或未选的三条相邻边及一条不相邻的边,皆不能形成包含原八边形的四边形,这样的组合数有
        \[
        8 + 8 \cdot 3 = 32
        \]
        所求概率为
        \[
        \frac{\comb{8}{4}-32}{\comb{8}{4}}=\frac{19}{35}
        \]
    \end{solution}
    
    \question $X$是从$\{1,2,\dots,9\}$中随机抽取$3$个不同的数排列出的最大的三位数,
        $Y$是从$\{1,2,\dots,8\}$中随机抽取$3$个不同的数排列出的最大的三位数,求$X>Y$的概率。
    \begin{solution}
        若 $X$ 的三个数字中包含 $9$,则由其组成的最大三位数一定大于由 $1$ 到 $8$ 中任意三个数字组成的最大三位数,此时 $X > Y$ 恒成立,这样的$X$排列数有$\comb82$,因此
        \[
        P(X > Y) = \frac{\comb82 \cdot \comb83}{\comb93 \cdot \comb83} = \frac{1}{3}
        \]
    \end{solution}
    
    \question 在数字 $1,2,3,4,5,6,7$ 的所有排列中,求至少有两个相邻数字不互质的概率。
    \begin{solution}
        先对奇数 $1,3,5,7$ 排列,有 $\perm44$ 种排法。
        
        数字 $6$ 不能与 $3$ 相邻,$4$ 个奇数间有 $5$ 个空位,去除 $3$ 的左右两个空位,剩 $3$ 个空位可放 $6$,有 $3$ 种放法。
        
        数字 $2$ 和 $4$ 放在剩余 $4$ 个空位,共有 $\perm42 $ 种排法。
        
        故至少有两个相邻数字不互质的概率为
        \[
        1 - \frac{\perm{4}{4} \cdot 3 \cdot \perm{4}{2}}{7!} = \frac{29}{35}
        \]
    \end{solution}

    \question 设 $A$ 和 $B$ 是从集合 $\{1,2,3,4,5,6\}$ 中随机选出的子集,允许 $A=B$。求事件“$A$ 包含于 $B$ 或 $B$ 包含于 $A$”的概率。
    \begin{solution}
        $A \subseteq B$ 当且仅当不存在元素 $x$ 满足 $x\in A$ 且 $x\notin B$,即概率为
        \[
        P(A\subseteq B) = \left(1-\left(\frac{1}{2}\right)^2\right)^6 = \left(\dfrac{3}{4}\right)^6
        \]
        同理 $P(B\subseteq A) = \left(\dfrac{3}{4}\right)^6$,且
        \[
        P(A=B) = \left(\frac{1}{2}\right)^6
        \]
        根据容斥原理,所求概率为
        \[
        2\left(\frac{3}{4}\right)^6 - \left(\frac{1}{2}\right)^6 = \frac{697}{2048}
        \]
    \end{solution}

\question Alice 和 Bob 各自有一副相同的牌,包含 3 张红牌、3 张白牌和 3 张蓝牌。他们轮流从自己的牌中随机抽取一张牌,且不放回。Alice 先抽。求 Alice 在 Bob 抽到任何红牌之前抽完她的所有红牌的概率。

\begin{solution}
每个人都将 9 张牌随机排列,考虑每个人的 3 张红牌的位置。共有
\[
\comb{9}{3}^2
\]
种可能。

考虑一条长度为 10 的序列,由 Alice 抽牌直到她抽到第 3 张红牌为止,再加上 Bob 在 Alice 抽到第 3 张红牌后接下来抽的牌。为了满足条件,这条序列中必须有 6 张红牌,我们在枚举这些情况。

因此概率为
\[
\frac{\comb{10}{6}}{\comb{9}{3}^2} = \frac{5}{168}.
\]
        \textcolor{red}{(待验证, lehigh 2025 q15)}
\end{solution}

    \question 掷一个公平的六面骰子连续十次。求出现恰好三个 6 连在一起的概率。(四个或更多连续的 6 不计。)
    \begin{solution}
        情况一:三个 6 在序列的一端,且相邻数字不是 6:有 2 种方式,对应概率为
        \[
        2 \cdot \left(\frac{1}{6}\right)^3 \cdot \frac{5}{6}
        \]
        情况二:三个 6 在序列中间,两边相邻数字都不是 6:有 6 种方式,对应概率为
        \[
        6 \cdot \left(\frac{1}{6}\right)^3 \cdot \left(\frac{5}{6}\right)^2
        \]
        总概率为
        \[
        \left(\frac{1}{6}\right)^3 \left(2 \cdot \frac{5}{6} + 6 \cdot \frac{25}{36}\right) = \frac{35}{1296}
        \]
    \end{solution}

    \question 投掷一枚均匀硬币 8 次。已知在前 3 次投掷中至少出现一次正面,求 8 次投掷中恰好出现 4 次正面的概率。
    \begin{solution} 
        前 3 次投掷中至少出现一次正面且8次投掷中恰好有4次正面的情形,即前3次及后5次有1正3正、2正2正、3正1正,方法数为
        \[
        \comb{3}{1}\comb{5}{3},\comb{3}{2}\comb{5}{2},\comb{3}{3}\comb{5}{1}
        \]
        8 次投掷中前3次都是反面的方法数为$2^5$,前3次至少1次正面的方法数为$2^8 - 2^5$,因此所求概率为
        \[
        \frac{\comb{3}{1}\comb{5}{3} + \comb{3}{2}\comb{5}{2} + \comb{3}{3}\comb{5}{1}}{2^8 - 2^5}
        = \frac{65}{224}
        \]
    \end{solution}

    \question 有三个碗,每个碗里各有 6 个球。现随机选择一个碗,再选择一个不同的碗,把第一个碗中的一个球移动到第二个碗。经过 5 次这样的移动后,三个碗再次各有 6 个球的概率是多少?
    \begin{solution}
        设碗标为 $A, B, C$。为了最终三个碗各有 6 个球,必须有两个碗各被拿走 2 个球,第三个碗只被拿走 1 个球。选择只拿走 1 个球的碗有 3 种方法,假设为碗 $A$。

        这个移球的回合有 5 种选择,并且球要移到的碗有 2 种选择。

        若从 $A$ 移到 $B$,则必须从 $C$ 移回 $A$,否则无法使 $B$ 和 $C$ 最终数量相等。这个移动的回合有 4 种选择。

        剩下 3 次移动必须是 2 次从 $B$ 到 $C$,1 次从 $C$ 到 $B$,这三次的顺序有 3 种。

        综上,有 $3 \cdot 5 \cdot 2 \cdot 4 \cdot 3$ 种序列使三碗最终各有 6 个球,每次移动有 $3 \cdot 2 = 6$ 种可能,因此总共有 $6^5$ 种可能的移动序列,因此所求概率为
        \[
        \frac{3 \cdot 5 \cdot 2 \cdot 4 \cdot 3}{6^5} = \frac{5}{108}
        \]
    \end{solution}

    \question 一个正立方体骰子六个面上分别有 $2,3,4,5,6,7$ 个点。随机移除一个点(每个点被移除的概率相等)后投掷这个骰子,求朝上的面的点数为奇数的概率。
    \begin{solution}
        移除一个点后,如果从偶数点的面移除一个点,该面变为奇数点。如果从奇数点的面移除一个点,该面变为偶数点。

        情况 1:从偶数点的面移除。则有 4 个奇数点面和 2 个偶数点面,掷出奇数点的概率为 $\dfrac{4}{6}$。

        情况 2:从奇数点的面移除。则有 2 个奇数点面和 4 个偶数点面,掷出奇数点的概率为 $\dfrac{2}{6}$。

        移除某个面的点的概率为该面点数除以 $2+3+4+5+6+7=27$,所求概率为
        \[
        \frac{2}{27}\cdot \frac{4}{6} + \frac{3}{27}\cdot \frac{2}{6} + \frac{4}{27}\cdot \frac{4}{6} + \frac{5}{27}\cdot \frac{2}{6} + \frac{6}{27}\cdot \frac{4}{6} + \frac{7}{27}\cdot \frac{2}{6} = \frac{13}{27}
        \]
    \end{solution}

    \question $A$ 和 $B$ 玩一场卡牌游戏。$A$ 有 6 张牌:2 张红色、2 张黄色、2 张绿色。$B$ 有 4 张牌:2 张紫色、2 张白色。两人轮流出牌,$A$ 先出。每回合,玩家随机选择一张自己手中的牌放到桌上。若某玩家在桌上放出两张同色牌,则该玩家获胜。求 $A$ 获胜的概率。
    \begin{solution}
        游戏一定在 $B$ 的第三回合之前结束,故$A$ 只能在自己的第二回合或第三回合获胜,且不能在第一回合直接获胜。

        情况 1:$A$ 第二回合获胜。$A$ 第二张牌必须和第一张同色。此时 $A$ 还剩 5 张牌,其中 1 张与第一张同色,所以概率为$\dfrac{1}{5}$。$A$ 和 $B$ 的第一张牌没有限制。

        情况 2:$A$ 第三回合获胜。需满足$A$ 第二张牌颜色与第一张不同,$B$ 第二张牌颜色与她的第一张不同且$A$ 第三张牌与他前两张牌之一同色,概率为
        \[
        \frac{4}{5} \cdot \frac{2}{3} \cdot \frac{1}{2} = \frac{4}{15}
        \]
        综上,$A$ 获胜的概率为
        \[
        \frac{1}{5} + \frac{4}{15} = \frac{7}{15}
        \]
    \end{solution}

    \question $C$参加一个比赛,比赛中没有平局,她会一直比赛直到输掉 2 场比赛为止,此时她将被淘汰,不再继续比赛。已知:
    \begin{itemize}
        \item 第一场比赛获胜的概率为 $\dfrac{1}{2}$
        \item 若上一场获胜,则下一场获胜的概率为 $\dfrac{3}{4}$
        \item 若上一场失败,则下一场获胜的概率为 $\dfrac{1}{3}$
    \end{itemize}
    求 $C$ 在被淘汰之前(即输掉第 2 场之前)恰好赢得 3 场比赛的概率。
    \begin{solution}
        $C$ 赢 3 场比赛且输少于 2 场比赛的概率,可能的情况有赢 3场,输 0场或赢 3场,输 1场。以 $W$ 表示胜利,$L$ 表示失败,比赛结果可能的序列为
        \[
        WWW, \quad LWWW, \quad WLWW, \quad WWLW.
        \]
        故所求概率为
        \[
        \frac{1}{2}\cdot\frac{3}{4}\cdot\frac{3}{4} + \frac{1}{2}\cdot\frac{1}{3}\cdot\frac{3}{4}\cdot\frac{3}{4} + \frac{1}{2}\cdot\frac{1}{4}\cdot\frac{1}{3}\cdot\frac{3}{4} + \frac{1}{2}\cdot\frac{3}{4}\cdot\frac{1}{4}\cdot\frac{1}{3} = \frac{7}{16}
        \]
    \end{solution}

    \question $A$ 有 12 个桶:3 个绿色、3 个红色、3 个蓝色和 3 个黄色。他将球随机放入桶中:
    \begin{itemize}
        \item 将 4 个球随机放入 3 个绿色桶(每个球独立地等概率选择一个绿色桶)
        \item 将 3 个球随机放入 3 个红色桶
        \item 将 2 个球随机放入 3 个蓝色桶
        \item 将 1 个球随机放入 3 个黄色桶
    \end{itemize}
    求存在一个绿色桶,其中的球数皆多于其他 11 个桶中任何一个桶的球数的概率。
    \begin{solution}
        以$(a,b,c)$ 表示 3 个桶中球的无序分布。黄色桶分布只能是 $(1,0,0)$。

        蓝色桶:将 2 个球放入 3 个桶,总共有 $3^2=9$ 种方法。
        \begin{itemize}
            \item $(2,0,0)$: 2 个球在同一个桶,有 3 种方式,
            \item $(1,1,0)$: 2 个球分在不同桶,有 $9-3=6$ 种方式。
        \end{itemize}
        红色桶:将 3 个球放入 3 个桶,总共有 $3^3=27$ 种方法。
        \begin{itemize}
            \item $(3,0,0)$: 3 个球在同一个桶,有 3 种方式,
            \item $(1,1,1)$: 每个桶 1 个球,有 $\comb{3}{2}=6$ 种方式,
            \item $(2,1,0)$: 其余 18 种方式。
        \end{itemize}
        绿色桶:将 4 个球放入 3 个桶,总共有 $3^4=81$ 种方法。
        \begin{itemize}
            \item $(4,0,0)$: 3 种方式,
            \item $(3,1,0)$: $4\cdot 3!=24$ 种方式,
            \item $(2,1,1)$: $4\cdot 3 \cdot \dfrac{3!}{2!}=36$ 种方式,
            \item $(2,2,0)$: 其余 18 种方式。
        \end{itemize}
        要使绿色桶中的球比其他 11 个桶都多,可能的情况及概率如下:
        \begin{center}
        \begin{tabular}{|c|c|c|c|c|}
        \hline
        \text{绿色} & \text{红色} & \text{蓝色} & \text{黄色} & \text{概率} \\
        \hline
        $(4,0,0)$ & 任意 & 任意 & 任意 & $\frac{3}{81}$ \\
        \hline
        $(3,1,0)$ & 不是 $(3,0,0)$ & 任意 & 任意 & $\frac{24}{81} \cdot \frac{24}{27}$ \\
        \hline
        $(2,1,1)$ & $(1,1,1)$ & $(1,1,0)$ & 任意 & $\frac{36}{81} \cdot \frac{6}{27} \cdot \frac{6}{9}$ \\
        \hline
        \end{tabular}
        \end{center}
        其中分布 $2/2/0$ 不符合条件,因为没有单个绿色桶的球数比其他桶多,故所求概率为
        \[
        \frac{3}{81} + \frac{24}{81} \cdot \frac{24}{27} + \frac{36}{81} \cdot \frac{6}{27} \cdot \frac{6}{9} = \frac{89}{243}
        \]
    \end{solution}

    \question 有三个盒子:
    \begin{itemize}
        \item 盒子 1 里有 1 个金球和 1 个黑球;
        \item 盒子 2 里有 1 个金球和 2 个黑球;
        \item 盒子 3 里有 1 个金球和 3 个黑球。
    \end{itemize}
    每次从一个盒子里随机抽取一个球,盒子里每个球被选中的概率相等。过程如下:
    \begin{enumerate}
        \item 从盒子 1 中随机抽一个球,放入盒子 2;
        \item 然后从盒子 2 中随机抽一个球,放入盒子 3;
        \item 最后从盒子 3 中随机抽一个球。
    \end{enumerate}
    求从盒子 3 中抽到金球的概率。
        \ifprintanswers
        \begin{figure}[H]
            \centering        
            \includegraphics[width=0.5\textwidth]{images/image171.png}
        \end{figure}
        \fi
    \begin{solution}
        从盒子 1 抽到金球及黑球的概率皆为$\dfrac{1}{2}$。
        
        若从盒子 1 抽到金球,盒子 2 将有 2 个金球和 2 个黑球;若从盒子 1 抽到黑球,盒子 2 将有 1 个金球和 3 个黑球。

        若盒子 2 中有 2 金 2 黑,抽到金球的概率为 $\dfrac{1}{2}$。若盒子 2 中有 1 金 3 黑,抽到金球的概率为 $\dfrac{1}{4}$。
        
        放入盒子 3后,盒子 3 有 2 金 3 黑的概率为
        \[
        \frac{1}{2}\cdot \frac{1}{2} + \frac{1}{2}\cdot \frac{1}{4} = \frac{3}{8}, 
        \]
        此时从盒子 3 抽到金球的概率为
        \[
        \frac{3}{8} \cdot \frac{2}{5} + \frac{5}{8} \cdot \frac{1}{5} =\frac{11}{40}
        \]
    \end{solution}

    \question 八个人分属三个家庭,围坐在圆桌旁。其中两个家庭各有 3 人,另一个家庭有 2 人。求每个人都有至少一个来自不同家庭的邻座的概率。
    \begin{solution}
        对座位编号,并将同一家庭的成员视为不可区分。8 个人的座位排列总数为:
        \[
        \frac{8!}{3! 3!  2!} = 560
        \]
        现探讨至少有一人使得其邻座都来自自己家庭,即3 人家庭坐在一起的排列数。设家庭 $A$ 和 $B$ 各有 3 人,家庭 $C$ 有 2 人。要求家庭 $A$ 的 3 人坐在一起,视他们为一个整体,有 8 种方式选择这个整体的起始位置,剩余 5 个人的排列数为 $\dfrac{5!}{3! 2!}$。因此家庭 $A$ 坐在一起的排列数为
        \[
        8 \cdot \frac{5!}{3! 2!} = 80
        \]
        同理,家庭 $B$ 坐在一起的排列数也是 80。若家庭 $A$ 和 $B$ 都坐在一起,则将两个家庭各看作一个整体,排列数有
        \[
        8 \cdot \frac{3!}{2!} = 24
        \]
        由容斥原理,至少有一人使得其邻座都来自自己家庭的排列数为
        \[
        80 + 80 - 24 = 136
        \]
        所求概率为
        \[
        \frac{560 - 136}{560} = \frac{53}{70}
        \]
    \end{solution}

    \question 八支队伍参加一项比赛,采用单循环赛制(即任意两支队伍之间恰好进行一场比赛)。每场比赛没有平局,两支队伍各有 $\dfrac{1}{2}$ 的获胜概率。求每支队伍都至少输一场且至少赢一场的概率。
    \begin{solution}
        共有 $\comb{8}{2}= 28$场比赛,因此总共有$2^{28}$种可能的比赛结果。

        当一支队伍全胜,同时另一支队伍全败时,选择全胜队有 8 种,全败队有 7 种,它们之间的比赛及与其他队伍的比赛结果已经确定,剩下的 $28-7-6=15$ 场比赛未确定,因此至少有一支队伍全胜或全败的结果数为
        \[
        8 \cdot 2^{21} + 8 \cdot 2^{21} - 8 \cdot 7 \cdot 2^{15}=2^{15}\cdot 968
        \]
        则每支队伍都至少输一场且至少赢一场的概率为
        \[
        1 - \frac{2^{15}\cdot 968}{2^{28}}  = \frac{903}{1024}
        \]
    \end{solution}

    \question 盒子中有大小,形状完全相同的3个红球,3个白球,现抛掷一枚质地均匀的骰子, 掷出几点就从盒子中取出几个球,求取出的球中红球个数大于白球个数的概率。
    \begin{solution}
        记从盒子中取出 $i$ 个球,取出的球中红球个数大于白球个数的概率为 $P_i$ ($i=1, 2, ..., 6$),则 
        \[
        \begin{aligned}
        P_1 &= \frac{\comb{3}{1}\comb{3}{0}}{\comb{6}{1}} = \frac{1}{2}, \quad
        P_2 = \frac{\comb{3}{2} \comb{3}{0}}{\comb{6}{2}} = \frac{1}{5}, \quad
        P_3 = \frac{\comb{3}{2}\comb{3}{1} + \comb{3}{3}\comb{3}{0}}{\comb{6}{3}}  = \frac{1}{2}, \\
        P_4 &= \frac{\comb{3}{3} \comb{3}{1}}{\comb{6}{4}} = \frac{1}{5}, \quad
        P_5 = \frac{\comb{3}{3} \comb{3}{2}}{\comb{6}{5}} = \frac{1}{2}, \quad
        P_6 = 0.
        \end{aligned}
        \]
        所以取出的球中红球个数大于白球个数的概率为 $$\frac{1}{6}\left(\frac{1}{2}+\frac{1}{5}+\frac{1}{2}+\frac{1}{5}+\frac{1}{2}+0\right)=\frac{19}{60}$$
    \end{solution}

    \question $A,B,C$ 独立参加考试。已知:
    \begin{itemize}
        \item $A$ 及格且 $B$ 不及格的概率为 $\dfrac{3}{20}$;
        \item $B$ 及格且 $B$ 不及格的概率为 $\dfrac{1}{4}$;
        \item $A$ 和 $C$ 都及格的概率为 $\dfrac{2}{5}$。
    \end{itemize}
    求至少有一人不及格的概率。
    \begin{solution}
        设$A,B,C$及格的概率为$a,b,c$,由于及格与否为独立事件,据题意有
        \[
        a(1-b) = \frac{3}{20}, \quad b(1-c) = \frac{1}{4}, \quad ac = \frac{2}{5}
        \]
        解得
        \[
        (8b+1)(4b-3) = 0 \Rightarrow b = \frac{3}{4} >0
        \]
        因此至少有一人不及格的概率为
        \[
        1 - abc  = 1 - \frac{3}{5} \cdot \frac{3}{4} \cdot \frac{2}{3}= \frac{7}{10}
        \]
    \end{solution}

    \question 将一块正方形厚纸板划分成 $5 \times 5 = 25$ 个面积相等的小正方格。现将三枚不同颜色的棋子随机放置在小正方格的中心,每个小正方格最多放一枚棋子。求这三枚棋子的位置不共线(即构成三角形)的概率。
    \begin{solution}
        三点共线的方法数有
        \[
        16 \cdot \comb{3}{3} + 4 \cdot \comb{4}{3} + 12 \cdot \comb{5}{3} =152
        \]
        因此三点不共线(即能构成三角形)的概率为
        \[
        \frac{\comb{25}{3} - 152}{\comb{25}{3}} = \frac{537}{575}
        \]
    \end{solution}

\question 从圆内随机选择四个不同的点,这四个点确定四个三角形(包括可能出现的退化三角形)。随机选择其中两个三角形,求圆心位于所选两个三角形的并集内的概率。
\begin{solution}
首先计算圆心位于四个三角形并集内的概率。圆心不在四个三角形的并集内,当且仅当四点位于某条直径的同一侧,即它们在圆上的径向投影也在直径同侧。此时恰好存在一个点,其余三个点位于该点顺时针方向 180 度的弧内。对于给定点,其余三个点落在该弧内的概率为
\[
\left(\frac{1}{2}\right)^3 = \frac{1}{8},
\]
因此四个点中任一点作为该点的概率为 $4 \cdot \frac{1}{8} = \frac{1}{2}$。因此圆心不在凸包内的概率为 $\frac{1}{2}$,所以圆心在凸包内的概率也为 $\frac{1}{2}$。

接下来考虑随机选择两三角形。四点确定四个三角形,它们的交集互不重叠。共有 $\comb{4}{2} = 6$ 对三角形。圆心在随机选择的一对三角形交集内的概率为 $\frac{1}{6} = \frac{2}{12}$。因此圆心在所选两三角形并集内的概率为
\[
\frac{1}{2} - \frac{1}{12} = \frac{5}{12}.
\]
We first calculate the probability that the center is in the union of the four triangles. The center does not lie in the union of the four triangles iff the four points lie on the same side of some diameter of the circle iff this is true of their radial projections onto the circle iff for one of these, the other three are within 180 degrees clockwise of the point, and such a point is essentially unique. Given a point, the probability that the other three lie in this arc is $\left(\frac{1}{2}\right)^3=\frac{1}{8}$, so the probability that the center does not lie in the convex hull is $4 \cdot \frac{1}{8}=\frac{1}{2}$.Ignoring possibilities with probability 0, the center lies in the union of the four triangles iff it lies in the intersection of two of the triangles, and these possibilities are disjoint. Since there are six pairs, the probability of the center lying in a randomly selected intersection is $\frac{1}{6} = \frac{2}{12}$. The center is in the union of the two selected triangles iff it is in the union of the four but not in the intersection of the two which we did not select. This probability is $\frac{1}{2} - \frac{1}{12}$.
\end{solution}

    \question $A$ 和 $B$ 轮流进行游戏,$A$ 先开始。每轮中:
    \begin{itemize}
        \item $A$ 投掷一枚公平硬币,若出现正面则 $A$ 获胜,游戏结束
        \item 若 $A$ 未获胜,则 $B$ 投掷一枚公平骰子,若出现 3 则 $B$ 获胜,游戏结束
        \item 若 $B$ 也未获胜,则进入下一轮,$A$ 再次投掷
    \end{itemize}
    游戏持续进行直到某人获胜。求 $B$ 获胜的概率。
    \begin{solution}
        设事件 $N$ 表示 $B$ 掷出的数不是 3,则$B$ 获胜的可能顺序为
        \[
        T3, \quad TNT3, \quad TNTNT3, \;\ldots
        \]
        每种顺序的概率分别为:
        \[
        \frac{1}{2}\cdot \frac{1}{6} = \frac{1}{12},\quad
        \frac{1}{2}\cdot\frac{5}{6}\cdot\frac{1}{2}\cdot \frac{1}{6} = \frac{5}{144}, \;\ldots
        \]
        这是一个等比数列,公比为 $\displaystyle \frac{1}{2}\cdot \frac{5}{6}=\frac{5}{12}$,因此 $B$ 获胜的概率为
        \[
        \frac{\frac{1}{12}}{1 - \frac{5}{12}} = \frac{1}{7}
        \]
    \end{solution}

    \question 圆圈中有 25 人,随机选出 3 人,求所选三人互不相邻的概率。
    \begin{solution}
        先固定被选中的一人 $A$,其余两人必须从不与 $A$ 相邻的 $22$ 人中选出,且这 $22$ 人在圆上形成一条线段,因此其中相邻的成对共有 $21$ 对,于是可行对数为
        \[
        \comb{22}{2}-21
        \]
        由于三人中谁充当 $A$ 都可以,上式被重复计数 $3$ 次,故满足条件的三人组数为
        \[
        \frac{25}{3}(\comb{22}{2}-21)
        \]
        所选三人互不相邻的概率为
        \[
        \frac{\frac{25}{3}(\comb{22}{2}-21)}{\comb{25}{3}}=\frac{35}{46}
        \]
    \end{solution}

    \question 三对已婚夫妇随机围坐在一张圆桌旁,求没有夫妻相邻的概率。
    \begin{solution}
        先固定 $A$ 先生的座位,其他五人的排列共有 $5!$ 种。

        情况 1:$A$ 女士坐在 $A$ 先生对面。此时$B$ 先生 有 4 种选择,$B$ 女士不坐他旁边有 2 种选择,剩下两人顺序 2 种选择,共 $4\cdot2\cdot2$ 种。

        情况 2:$A$ 女士与 $A$ 先生之间间隔一人。这个人有 4 种选择,其配偶必须坐在其他三个座位的中间位置,$C$夫妇有 2 种选择,且$A$ 女士与 $A$ 先生可互换位置,共 $4\cdot2\cdot2$ 种。

        因此概率为
        \[
        \frac{4\cdot2\cdot2+4\cdot2\cdot2}{5!} = \frac{4}{15}
        \]
    \end{solution}

    \question 伯克利市每天的天气要么是雨天,要么是雾天,其中天气与前一天相同的概率为 \(\dfrac{3}{4}\)。如果我们只知道今天是雨天,求 7 天后仍是雨天的概率。
    \begin{solution}
        7 天后仍是雨天的概率为
        \begin{align*}
        &P(\text{天气完全没变}) + P(\text{天气变恰好2次})+ P(\text{天气变恰好4次})+ P(\text{天气变恰好6次}) \\
        &= \comb 70 \left(\frac14\right)^0\left(\frac34\right)^7
        + \comb 72 \left(\frac14\right)^2\left(\frac34\right)^5
        + \comb 74 \left(\frac14\right)^4\left(\frac34\right)^3 
        + \comb 76 \left(\frac14\right)^6\left(\frac34\right)^1 \\
        &= \frac{129}{256}
        \end{align*}
    \end{solution}

\question 袋中有红球 5 个、白球 3 个、黑球 4 个,若每球被选取的机会均等,每次由袋中取一球,取后不放回,取完为止,则黑球最先取完的概率为?

\begin{solution}
\[
{5\over 5+4}+{3\over 3+4}-{5+3\over 5+3+4}= \frac{20}{63}
\]
%https://math.ntnu.edu.tw/~horng/letter/hpm17010.pdf
\end{solution}

    \question 一对实数 $(a,b)$ 在闭单位圆盘内均匀随机选取,即满足
    \[
    a^{2} + b^{2} \leq \frac{1}{4}.
    \]
    求二次函数
    \[
    y = ax^{2} + 2bx - a
    \]
    与曲线
    \[
    y = x^{2}
    \]
    相交的概率。
    \ifprintanswers
    \begin{figure}[H]
        \centering        
        \includegraphics[width=0.6\textwidth]{images/image174.png}
    \end{figure}
    \fi
    \begin{solution}
        曲线相交当且仅当方程
        \[
        (a-1)x^2 + 2bx - a = 0
        \]
        有实根,此时判别式为非负:
        \[
        (2b)^2 - 4(a-1)(-a) \ge 0 \Rightarrow \left(a - \frac{1}{2}\right)^2 + b^2 \ge \left(\frac{1}{2}\right)^2
        \]
        实数 $(a,b)$满足在圆 $a^2+b^2 = \left(\dfrac{1}{2}\right)^2$ 内但在圆$\left(a - \dfrac{1}{2}\right)^2 + b^2 = \left(\dfrac{1}{2}\right)^2$外,以阴影部分表示,面积为
        \[
        \pi\left(\frac{1}{2}\right)^2-2\left(2\cdot \frac{1}{2}\left(\frac{1}{2}\right)^2\left(\frac{\pi}{3}\right)-\frac{\sqrt{3}}{4}\left(\frac{1}{2}\right)^2\right) = \frac{\pi}{12}+\frac{\sqrt{3}}{8}
        \]
        所求概率为
        \[
        \frac{\frac{\pi}{12}+\frac{\sqrt{3}}{8}}{\frac{\pi}{4}}
        = \frac{1}{3} + \frac{\sqrt{3}}{2\pi} \approx 0.609
        \]
    \end{solution}

    \question 一只蚂蚁从立方体的一个顶点出发,每次可以走到与当前位置相邻的三个顶点中的任意一个。求四步后,蚂蚁回到出发点的概率。
        \begin{solution}
        设蚂蚁起点为 $(0,0,0)$,三步后,蚂蚁抵达顶点 $(1,1,1)$的概率为
        \[
        \frac{3!}{3^3} = \frac{2}{9},
        \]
        否则,蚂蚁会停在 $(1,0,0),(0,1,0)$ 或 $(0,0,1)$。

        若三步后位于 $(1,1,1)$,蚂蚁下一步无法回到 $(0,0,0)$;但如果三步后在 $(1,0,0),(0,1,0)$ 或 $(0,0,1)$,则第四步回到 $(0,0,0)$ 的概率均为 $\dfrac{1}{3}$。

        故四步后回到出发点的概率为
        \[
        \frac{7}{9} \cdot \frac{1}{3} = \frac{7}{27}
        \]
    \end{solution}

    \question 一只蚂蚁在正方体 $ABCD-EFGH$ 的顶点 $A$ 处, 每次等概率地爬行到相邻三个顶点中的一个, 那么六次爬行之后回到顶点 $A$ 处的概率为
\begin{solution}
将立方体顶点坐标设为 $(0,0,0)$ 或 $1$,蚂蚁从 $(0,0,0)$ 出发。每步蚂蚁改变一个坐标的值(0 $\leftrightarrow$ 1)。要在六步后回到原点,每个坐标必须被改变偶数次。设三个坐标变化次数分别为 $x,y,z\ge 0$,且 $x+y+z=6$,且 $x,y,z$ 都是偶数。可行的组合为
\[
(0,0,6),\ (0,6,0),\ (6,0,0),\ (0,2,4),\ (0,4,2),\ (2,0,4),\ (2,4,0),\ (4,0,2),\ (4,2,0),\ (2,2,2)。
\]

每个组合对应的步序列数由多项式系数计算:
\[
(2,2,2) \rightarrow \frac{6!}{2!2!2!}=90,\quad
(0,2,4)\ \text{及类似组合} \rightarrow \frac{6!}{0!2!4!}=15。
\]

前六种 $(0,2,4)$ 类型组合共有 6 种,贡献步序列总数 $6\cdot 15=90$。加上 $(2,2,2)$ 的 90,总共有 $180$ 种有效序列。总步数序列数为 $3^6=729$。因此概率为
\[
\frac{180}{729}=\frac{20}{81}.
\]
\textcolor{red}{(chatgpt,待解)}
\end{solution}

    \question 一只蚂蚁从正八面体的一个顶点出发,每一步随机走到相邻顶点。求 10 步后它回到出发点的概率。  
    \begin{solution}
        设 $a_n,b_n,c_n$ 分别为走 $n$ 步后蚂蚁在 (a) 出发点、(b) 与出发点相邻的四个顶点、(c) 对面顶点 的概率。有递推关系
        \[
        a_n=\frac{1}{4}b_{n-1}, \quad c_n=\frac{1}{4}b_{n-1}, \quad b_n=a_{n-1}+\frac{1}{2}b_{n-1}+c_{n-1}
        \]
        消去 $a_n,c_n$ 得
        \[
        b_n=\frac{1}{2}b_{n-1}+\frac{1}{2}b_{n-2}
        \]
        特征方程为 $x^2=\frac{1}{2}x+\frac{1}{2}$,解得 $x=1,-\dfrac{1}{2}$,故
        \[
        b_n=\alpha\cdot 1^n+\beta\left(-\frac{1}{2}\right)^n
        \]
        初值$a_0=1,b_0=c_0=0$,所以 $b_1=1$,可得 $\alpha=\dfrac{2}{3}, \beta=-\dfrac{2}{3}$,即
        \[
        b_n=\frac{2}{3}\left(1-\left(-\frac{1}{2}\right)^n\right)
        \]
        于是
        \[
        a_{10}=\frac{1}{4}b_{9}=\frac{1}{4}\cdot \frac{2}{3}\left(1-\left(-\frac{1}{2}\right)^9\right)
        =\frac{513}{3072}
        \]
    \end{solution}

    \question 一只蜜蜂在三维空间中移动。掷一个公平的六面骰子,骰子的六个面分别标记为 \( \pm x, \pm y, \pm z \)。假设蜜蜂当前所在点为 \( (x,y,z) \)。若骰子显示 \( +x \),则蜜蜂移动到点 \( (x+1, y, z) \);若骰子显示 \( -x \),则蜜蜂移动到点 \( (x-1, y, z) \)。其他四个面以此类推。已知蜜蜂从点 \( (0,0,0) \) 出发,掷骰子四次。求蜜蜂经过的四条边恰好是某单位立方体上的四条不同边的概率。 
    \begin{solution}
        由乘法原理及想象力,概率为
        \[
        1\cdot \dfrac{4}{6}\cdot \left(\dfrac{1}{6}\cdot \dfrac{3}{6}+\dfrac{2}{6}\cdot \dfrac{2}{6}\right)=\frac{7}{54}
        \]
    \end{solution}

    \question 两支队伍正在进行“三局两胜”的系列赛:最多进行 3 场比赛,率先赢下 2 场的队伍获胜。第一场比赛在 A 队的主场进行,后两场在 B 队的主场进行。A 队在主场赢的概率是 $\dfrac{2}{3}$,在客场赢的概率是 $p$。各场比赛的结果是独立的。若 A 队赢得整个系列赛的概率是 $\dfrac{1}{2}$,求 $p$ 的值。
    \begin{solution}
        已知 A 队主场胜率为 $\dfrac23$,客场胜率为 $p$。三局两胜赛制下,A 队赢下系列赛的情形只有  
        \[
        \text{AA},\; \text{ABA},\; \text{BAA}
        \]
        于是有
        \[
        \frac23 p+\frac23(1-p)p+\frac13p^{2}=\frac12
        \]
        解得  
        \[
        p=\dfrac12\bigl(4-\sqrt{10}\bigr)\le1
        \]
    \end{solution}
    
    \question  一个不均匀的骰子,掷出 $1,2,3,4,5,6$ 点的概率依次成等差数列. 独立地先后掷该骰子两次,所得的点数分别记为 $a,b$. 若事件 “$a+b=7$” 发生的概率为 $\dfrac{1}{7}$,求事件 “$a=b$” 发生的概率。 
    \begin{solution}     
        设掷出$1,2,\dots,6$点的概率分别为$p_1,p_2,\dots,p_6$。由于$p_1,p_2,\dots,p_6$成等差数列,且$p_1+p_2+\dots+p_6=1$,故
        \[
        p_1+p_6=p_2+p_5=p_3+p_4=\frac{1}{3}
        \]
        事件“$a+b=7$”发生的概率为
        \[
        P_1=p_1p_6+p_2p_5+\dots+p_6p_1
        \]
        事件“$a=b$”发生的概率为
        \[
        P_2=p_1^2+p_2^2+\dots+p_6^2
        \]
        于是
        \[
        P_1+P_2=(p_1+p_6)^2+(p_2+p_5)^2+(p_3+p_4)^2=3\cdot\left(\frac{1}{3}\right)^2=\frac{1}{3}
        \]
        由于$P_1=\dfrac{1}{7}$,所以
        \[
        P_2=\frac{1}{3}-\frac{1}{7}=\frac{4}{21}
        \]
    \end{solution}
    \begin{solution}
        设掷出点数$1,2,3,4,5,6$的概率依次为$p-5d,p-3d,p-d,p+d,p+3d,p+5d$,据题意有
        \begin{align}
        (p-5d)+(p-3d)+(p-d)+(p+d)+(p+3d)+(p+5d)&=1 \tag{1}\\
        2[(p-5d)(p+5d)+(p-3d)(p+3d)+(p-d)(p+d)]&=\dfrac{1}{7} \tag{2} 
        \end{align} 
        由$(1)$得$p=\dfrac{1}{6}$,由$(2)$化简得
        \[
        6p^2 - 70 d^2 = \frac{1}{7} \Rightarrow 70d^2=\frac{1}{42}
        \] 
        于是
        \begin{align*}
            P(a=b)&=(p-5d)^2+(p-3d)^2+(p-d)^2+(p+d)^2+(p+3d)^2+(p+5d)^2 \\
            &=6p^2+70d^2 =\frac{1}{6}+\frac{1}{42}=\frac{4}{21}
        \end{align*}
    \end{solution}

    \question 设甲、乙、丙三位射手之命中率依次为 $p,q,r$,其中 $p\ge q\ge r$。今三人同打一靶且互不影响,各发一弹时,此靶不中弹之概率为 $\dfrac{1}{4}$,恰中一弹之概率为 $\dfrac{11}{24}$,恰中二弹之概率为 $\dfrac{1}{4}$,求序组 $(p,q,r)$。
    \begin{solution} 
        据题意,
        \[
        \begin{cases}
        pqr=1-\left(\dfrac{1}{4}+\dfrac{11}{24}+\dfrac{1}{4}\right)=\dfrac{1}{24}, \\[2pt]
        p(1-q)(1-r)+q(1-r)(1-p)+r(1-p)(1-q)=\dfrac{11}{24}, \\[2pt]
        pq(1-r)+qr(1-p)+rp(1-q)=\dfrac{1}{4},
        \end{cases}
        \]
        其中 $p\ge q\ge r$,解得
        \[
        pqr=\frac{1}{24},\quad pq+qr+rp=\frac{3}{8},\quad p+q+r=\frac{13}{12},
        \]
        因此 $p,q,r$ 为三次方程
        \[
        t^3-\frac{13}{12}t^2+\frac{3}{8}t-\frac{1}{24}=0 \Rightarrow (2t-1)(3t-1)(4t-1)=0
        \]
        的三个根,解得
        \[
        (p,q,r)=\left(\frac{1}{2},\frac{1}{3},\frac{1}{4}\right)
        \]
    \end{solution}

    \question 在区间 $[0,10]$ 上等概率地随机选取一个数 $y$。以点 $(0,y)$ 为斜边一端点,斜边另一端点位于非负 $x$ 轴上,并与原点 $(0,0)$ 共同构成一个斜边长度为 10 的直角三角形。求该三角形面积大于 15 的概率。
    \begin{solution}
        由斜边长度 10 可得面积条件为
        \[
        \frac{1}{2} xy = \frac{1}{2} y\sqrt{100-y^2} > 15 
        \]
        解得
        \[
        y < \sqrt{10} \quad \text{或} \quad y> \sqrt{90}
        \]
        所求概率为
        \[
        \frac{\sqrt{90}-\sqrt{10}}{10} = \frac{\sqrt{10}}{5}
        \]
    \end{solution}

    \question 设甲袋中有 5 颗白球、2 颗黑球,乙袋中有 3 颗白球。先自甲袋中任取 4 颗球放入乙袋,再从乙袋中任取 5 颗球放入甲袋,完成一次操作称为一局。已知每颗球被抽到的机会均等,若第一局结束时甲袋中有黑球,求过程中过甲袋取得 2 颗白球、2 颗黑球放入乙袋的概率。
    \begin{solution}
        情况一:甲取 4 白球入乙,再从乙任取 5 球入甲。  
        \[
        P_1 = \frac{\comb{5}{4}}{\comb{7}{4}} = \frac{1}{7}
        \]
        情况二:甲取 3 白 1 黑入乙,再从乙任取 5 球入甲。  
        \[
        P_2 = \frac{\comb{5}{3}\comb{2}{1}}{\comb{7}{4}} = \frac{4}{7}
        \]
        情况三:甲取 2 白 2 黑入乙,再从乙任取 4 白 1 黑球入甲。  
        \[
        P_3 = \frac{\comb{5}{2}\comb{2}{2}}{\comb{7}{4}} \cdot \frac{\comb{5}{4}\comb{2}{1}}{\comb{7}{5}} = \frac{20}{147}
        \]
        情况四:甲取 2 白 2 黑入乙,再从乙任取 3 白 2 黑球入甲。  
        \[
        P_4 = \frac{\comb{5}{2}\comb{2}{2}}{\comb{7}{4}} \cdot \frac{\comb{5}{3}\comb{2}{2}}{\comb{7}{5}} = \frac{20}{147}
        \]
        已知第一局结束时甲袋中有黑球,因此条件概率为
        \[
        \frac{P_3+P_4}{P_1+P_2+P_3+P_4} = \frac{8}{29}
        \]
    \end{solution}

    \question 三个人玩一个游戏,轮流掷一枚公平的六面骰子。若掷出 5 或 6 则该玩家立即获胜,游戏结束;若掷出 1 或 2 则该玩家被淘汰。游戏继续进行,直到有人获胜,或者只剩下一人,则该人获胜。问第一个掷骰子的人获胜的概率是多少?
    \begin{solution}
        设第一个掷骰子的人为 $A$,任意一次掷出 1,2,5,6 的概率是 $\dfrac{2}{3}$,故$A$ 第一个掷出 1,2,5,6 的概率为
        \[
        \frac{2}{3}\left(1+\frac{1}{3}+\frac{1}{3^2}+\cdots\right)=\frac{\frac{2}{3}}{1-\frac{1}{9}}=\frac{9}{13}
        \]
        而此时$A$有$\dfrac{1}{2}$的概率掷出 5 或 6 从而立即获胜。

        若 $A$ 不是第一个掷出 1,2,5,6 的人(概率为 $\dfrac{4}{13}$),那么第一个掷出 1,2,5,6 的人有$\dfrac{1}{2}$的概率掷出 1 或 2 被淘汰。此时只剩下两人,$A$ 与另一人各有 $\dfrac{1}{2}$ 的获胜机会。

        综上,第一个掷骰子的人获胜的概率为
        \[
        \frac{9}{13}\cdot \frac{1}{2}+\frac{4}{13}\cdot \frac{1}{2}\cdot \frac{1}{2}=\frac{11}{26}
        \]
    \end{solution}

    \question 反复抛掷一枚公平硬币,直到第一次出现序列 $HTH$ 为止。求在此过程中,序列 $THTH$ 从未出现过的概率。(例如,事件包含 $HHTH$,但不包含 $TTHTH$。)
    \begin{solution}
        设事件 $E$ 表示在序列 $THTH$ 出现之前先出现 $HTH$。若当前已出现的前缀为 $s$,则记 $P(E\mid s)$ 为在以 $s$ 开始的条件下事件 $E$ 发生的概率。定义
        \[
        x = P(E\mid H), \quad y = P(E\mid T)
        \]
        根据一步分析法,有
        \[
        x = \frac{1}{2} P(E\mid HH) + \frac{1}{4} P(E\mid HTT) + \frac{1}{4} P(E\mid HTH) = \frac{1}{2}x + \frac{1}{4}y + \frac{1}{4},
        \]
        \[
        y = \frac{1}{2} P(E\mid TT) + \frac{1}{4} P(E\mid THH) + \frac{1}{8} P(E\mid THTT) + \frac{1}{8} P(E\mid THTH) = \frac{1}{2}y + \frac{1}{4}x + \frac{1}{8}y
        \]
        解得
        \[
        x = \frac{3}{4}, \quad y = \frac{1}{2}
        \]
        因此所求概率为
        \[
        \frac{1}{2}x + \frac{1}{2}y = \frac{5}{8}
        \]
    \end{solution}

    \question 反复抛掷一枚公平硬币,直到出现连续四个正面 $HHHH$ 或连续六个反面 $TTTTTT$ 为止。求事件“$HHHH$ 先于 $TTTTTT$ 出现”的概率。
    \begin{solution}
        设事件 $E$ 表示在序列 $TTTTTT$ 出现之前先出现 $HHHH$。对任意当前末尾已出现的序列 $s$,记 $P(E\mid s)$ 为在条件 $s$ 下事件 $E$ 发生的概率。定义
        \[
        x = P(E\mid H),\quad y = P(E\mid T).
        \]
        根据后续可能的状态转移,可得到方程组(按下一次或若干次投掷展开):
        \[
        x = \frac{7}{8}y + \frac{1}{8}, \quad
        y = \frac{31}{32}x
        \]
        因此
        \[
        x = \frac{32}{39}, \quad y = \frac{31}{39}
        \]
        由于初始时无既定前缀(在第一次投掷前),所求概率为 
        \[
        \frac12(x+y)= \frac{21}{26}
        \]
    \end{solution}

    \question 四位玩家各自掷一个标准六面骰子,点数最大者获胜。如果出现平手,则平手者继续掷骰,直到一人胜出。$H$ 是其中之一。已知他最终获胜,求他第一轮掷出点数为 $5$ 的条件概率。
    \begin{solution}
        设事件 $A$ 为“$H$ 首掷为 $5$”,事件 $B$ 为“$H$ 获胜”,根据贝叶斯公式,
        \[
        P(A \mid B) = \frac{P(B \mid A) \cdot P(A)}{P(B)}.
        \]
        已知 $P(A) = \dfrac{1}{6}, P(B) = \dfrac{1}{4}$, 而$P(B \mid A)$ 如下分情况:
        \begin{itemize}
            \item 其他人都掷出 $\le 4$,概率为 $\left( \dfrac{2}{3} \right)^3 = \dfrac{8}{27}$
            \item 恰一人也掷 $5$,概率 $3 \cdot \dfrac{1}{6} \cdot \left( \dfrac{2}{3} \right)^2 \cdot \dfrac{1}{2} = \dfrac{1}{9}$
            \item 恰两人也掷 $5$,概率 $3 \cdot \left( \dfrac{1}{6} \right)^2 \cdot \dfrac{2}{3} \cdot \dfrac{1}{3} = \dfrac{1}{54}$
            \item 三人都掷 $5$,概率 $\left( \dfrac{1}{6} \right)^3 \cdot \dfrac{1}{4} = \dfrac{1}{864}$
        \end{itemize}
        故
        \[
        P(B \mid A) = \frac{8}{27} + \frac{1}{9} + \frac{1}{54} + \frac{1}{864} = \frac{41}{96} ,\quad
        P(A \mid B) = \frac{\frac{41}{96} \cdot \frac{1}{6}}{\frac{1}{4}} = \frac{41}{144}
        \]
    \end{solution}

    \question 某种生物不会与其他个体产生互动,但能够自行繁殖。若单独放置一小时,它会以相等的概率变成 $0,1,2,3$ 个个体,分别对应死亡或不同的繁殖结果。其后新产生的个体在每个后续小时中也以同样的方式独立行动。若初始时只有一个个体,设 $p$ 为这一族群能够持续存在(即永不灭绝)的概率,求$p$。
    \begin{solution}
        设 $q$ 为族群最终灭绝的概率。若当前有 $n$ 个个体,则最终灭绝的概率为 $q^n$。题意给出
        \[
        q = \frac14 \left(1 + q + q^2 + q^3\right)
        \]
        化简得
        \[
        (q-1)(q^2 + 2q - 1) = 0
        \]
        在 $[0,1]$ 区间内唯一满足要求的解为
        \[
        q = \sqrt{2} - 1
        \]
        因此
        \[
        p = 1 - q = 2 - \sqrt{2} \approx 0.586
        \]
    \end{solution}

    \question 设 $P(n)$ 表示在 $n$ 次抛掷一枚公平硬币时,出现至少连续三个正面的概率。求满足 $P(n)\ge \dfrac12$ 的最小整数 $n$。
    \begin{solution}
        设事件 $E_i$ 表示首次出现连续三个正面恰好发生在第 $i$ 次抛掷时,则
        \[
        P(n) = \sum_{i=1}^{n-2} \Pr(E_i),
        \]
        其中
        \[
        \Pr(E_i) = 
        \begin{cases}
        \dfrac{1}{8}, & i=1,\\[2mm]
        \dfrac{1}{16}(1-P(i-2)), & i\ge 2.
        \end{cases}
        \]
        当 $i\ge 2$ 时,$E_i$ 发生的条件是第 $i$ 次抛出 H 前必须是 T,并且之前没有出现过三个连续 H。注意 $P(i-2)=0$ 当 $i<5$,逐步计算:
        \[
        P(3)=\frac{1}{8},\quad P(4)=\frac{3}{16},\quad P(5)=\frac{4}{16},\quad P(6)=\frac{5}{16},
        \]
        \[
        P(7)=\frac{6}{16}-\frac{1}{16}\cdot\frac{1}{8},\quad P(8)=\frac{7}{16}-\frac{1}{16}\left(\frac{1}{8}+\frac{3}{16}\right),
        \]
        \[
        P(9)=\frac{8}{16}-\frac{1}{16}\left(\frac{1}{8}+\frac{3}{16}+\frac{4}{16}\right),\quad 
        P(10)=\frac{9}{16}-\frac{1}{16}\left(\frac{1}{8}+\frac{3}{16}+\frac{4}{16}+\frac{5}{16}\right)=\frac{65}{128} > \frac{1}{2}
        \]
        因此最小的 $n$ 满足 $P(n)\ge \dfrac{1}{2}$ 为 10。
    \end{solution}

    \question 设甲、乙两箱中,甲箱内有 1 白球 1 红球,乙箱内有 1 白球 2 红球。现在每次先从甲箱中随机取一球,放入乙箱内,再从乙箱随机取一球放入甲箱,这样视为一局。试求  
    \begin{parts}
    \part 在第二局结束后,有 2 红球在甲箱内的概率。
    \begin{solution}
        设三状态如下表所示:
        \[
        \begin{array}{c|c|c}
        \hline
        \text{状态} & \text{甲箱} & \text{乙箱} \\ \hline
        S_1 & 1\text{白},\ 1\text{红} & 1\text{白},\ 2\text{红} \\ \hline
        S_2 & 2\text{红} & 2\text{白},\ 1\text{红} \\ \hline
        S_3 & 2\text{白} & 3\text{红} \\ \hline
        \end{array}
        \]
        因此转移矩阵为
        \begin{align*}
        A&=\begin{bmatrix}
        P(S_1\to S_1) & P(S_2\to S_1) & P(S_3\to S_1) \\
        P(S_1\to S_2) & P(S_2\to S_2) & P(S_3\to S_2) \\
        P(S_1\to S_3) & P(S_2\to S_3) & P(S_3\to S_3)
        \end{bmatrix} \\
        &=\begin{bmatrix}
        \frac{1}{2}\cdot \frac{1}{2}+ \frac{1}{2}\cdot \frac{3}{4} & 1\cdot\frac{1}{2} & 1\cdot\frac{3}{4} \\
        \frac{1}{2}\cdot\frac{1}{2} & 1\cdot\frac{1}{2} & 0 \\
        \frac{1}{2}\cdot\frac{1}{4} & 0 & 1\cdot\frac{1}{4}
        \end{bmatrix}
        =\begin{bmatrix}
        \frac{5}{8} & \frac{1}{2} & \frac{3}{4} \\
        \frac{1}{4} & \frac{1}{2} & 0 \\
        \frac{1}{8} & 0 & \frac{1}{4}
        \end{bmatrix}
        \end{align*}
        于是
        \[
        A^2
        \begin{bmatrix} 1\\ 0 \\0 \end{bmatrix}
        =
        \left[\begin{matrix}
        \frac{39}{64} & \frac{9}{16} & \frac{21}{32} \\
        \frac{9}{32} & \frac{3}{8} & \frac{3}{16} \\
        \frac{7}{64} & \frac{1}{16} & \frac{5}{32}
        \end{matrix}\right]
        \begin{bmatrix} 1\\ 0 \\0 \end{bmatrix}
        =
        \begin{bmatrix}
        \frac{39}{64} \\
        \frac{9}{32} \\
        \frac{7}{64}
        \end{bmatrix}
        \]
        即第二局结束后,有 2 红球在甲箱内的概率为$\dfrac{9}{32}$。
    \end{solution}
    \part 经过长时间交换后,有 2 红球在甲箱内的概率。
    \begin{solution}
        对角化转移矩阵得
        \[
        A=
        \left[\begin{matrix}
        -2 & 1 & 6 \\
        1 & -2 & 3 \\
        1 & 1 & 1
        \end{matrix}\right]
        \left[\begin{matrix}
        0 & 0 & 0 \\
        0 & \frac{3}{8} & 0 \\
        0 & 0 & 1
        \end{matrix}\right]
        \left[\begin{matrix}
        -\frac{1}{6} & \frac{1}{6} & \frac{1}{2} \\
        \frac{1}{15} & -\frac{4}{15} & \frac{2}{5} \\
        \frac{1}{10} & \frac{1}{10} & \frac{1}{10}
        \end{matrix}\right]
        \]
        于是
        \[
        A^\infty
        =
        \left[\begin{matrix}
        -2 & 1 & 6 \\
        1 & -2 & 3 \\
        1 & 1 & 1
        \end{matrix}\right]
        \left[\begin{matrix}
        0 & 0 & 0 \\
        0 & 0 & 0 \\
        0 & 0 & 1
        \end{matrix}\right]
        \left[\begin{matrix}
        -\frac{1}{6} & \frac{1}{6} & \frac{1}{2} \\
        \frac{1}{15} & -\frac{4}{15} & \frac{2}{5} \\
        \frac{1}{10} & \frac{1}{10} & \frac{1}{10}
        \end{matrix}\right]
        =
        \left[\begin{matrix}
        \frac{3}{5} & \frac{3}{5} & \frac{3}{5} \\
        \frac{3}{10} & \frac{3}{10} & \frac{3}{10} \\
        \frac{1}{10} & \frac{1}{10} & \frac{1}{10}
        \end{matrix}\right]
        \]
        且
        \[
        A^\infty
        \begin{bmatrix} 1\\ 0\\ 0\end{bmatrix}
        =\begin{bmatrix}
        \frac{3}{5} \\
        \frac{3}{10} \\
        \frac{1}{10}
        \end{bmatrix}
        \]
        经过长时间交换后,有 2 红球在甲箱内的概率为
        \[
        P(S_2)=\frac{3}{10}
        \]
    \end{solution}
    \end{parts}

    \question 小明给阿花 7 颗巧克力当礼物,然后跟阿花说:「你每天至少要吃一颗巧克力,表示你有想念我,当然你要一次吃好多颗也可以喔。」

    设随机变量 $X$ 代表阿花吃完巧克力的天数,求期望值 $E(X)$。
    \begin{solution}
        考虑将 7 颗巧克力在 $X$ 天内吃完,且每天至少吃一颗。这等价于将 7 个相同的球放入 $X$ 个不同的盒子,每个盒子至少一个球。由间隔法,
        \[
        P(X=k)=\frac{\comb{6}{k-1}}{2^6}
        \]
        故
        \begin{align*}
        E(X)&=\sum_{k=1}^{7}k\cdot P(X=k) \\
        &=\frac{1}{64}\left(1\cdot \comb{6}{0}+2\cdot \comb{6}{1}+3\cdot \comb{6}{2}+4\cdot \comb{6}{3}+5\cdot \comb{6}{4}+6\cdot \comb{6}{5}+7\cdot \comb{6}{6}\right)
        =4
        \end{align*}
    \end{solution}

    \question 在下图所示的大正方形中随机选择一个点。图中嵌套的小正方形无限向内延续。求该随机点所处正方形数量的期望值。例如,图中所示的样例点位于其中的 2 个正方形内。
    \begin{figure}[H]
        \centering
        \includegraphics[width=0.5\textwidth]{images/image154.png}
    \end{figure} 
    \begin{solution}
        由于四个三角形面积和皆占正方形的一半,所求为
        \[
        1 \cdot \frac{1}{2} + 2 \cdot \frac{1}{2^2} + 3 \cdot \frac{1}{2^3} + \cdots
        \]
        即
        \[
        \left(\frac{1}{2} + \frac{1}{4} + \cdots \right)\left(1 + \frac{1}{2} + \frac{1}{4} + \cdots \right)
        = 1 \cdot 2 = 2
        \]
    \end{solution}

    \question 一个人出生在一周七天中的任意一天且概率相同。一大群人依次报出自己的出生星期,直到七个星期几都至少出现一次为止。试求第一次凑齐全部七个星期几时,对应的那个人的期望序号。
    \begin{solution}
        这是经典的“优惠券收集(Coupon Collector)”问题。设当前已经出现了 $i-1$ 个不同的星期,则下一位出现一个新星期的概率为
        \[
        p=\frac{7-(i-1)}{7},
        \]
        因此获得一个新星期所需的期望人数为
        \[
        \frac{1}{p}=\frac{7}{7-(i-1)}.
        \]
        于是凑齐全部七天所需的期望总人数为
        \[
        1+\frac{7}{6}+\frac{7}{5}+\frac{7}{4}+\frac{7}{3}+\frac{7}{2}+\frac{7}{1}=\frac{363}{20}
        \]
        其中,第 $i$ 项表示在已有 $i-1$ 种不同星期之后,等到第 $i$ 个新星期出现的期望等待人数。
    \end{solution}

    \question 假设投掷某金币一枚出现正面与反面的概率皆为 $\dfrac{1}{2}$,试问平均要连续掷几次金币,才会出现连续三次正面?
    \begin{solution}
        设 $E(n)$ 为出现连续 $n$ 次正面的掷币次数。由一步分析法,
        \[
        E(n) = p(E(n-1) + 1) + (1-p)(E(n-1) + 1 + E(n))
        \]
        代入 $p = \dfrac{1}{2}$,
        \[
        E(1) = 1 + \frac{1}{2} E(1), \quad E(2) = E(1) + 1 + \frac{1}{2} E(2), \quad E(3) = E(2) + 1 + \frac{1}{2} E(3)
        \]
        得
        \[
        E(1) = 2, \quad E(2) = 6, \quad E(3) = 14
        \]
    \end{solution}

    \question 已知一个非公正硬币掷出正面机率为 $\dfrac{1}{3}$,反面机率为 $\dfrac{2}{3}$,今连续掷此硬币,记录每次掷出的结果,每次结果互不影响,令随机变量 $X$表示第一次看到正面、反面、正面依序出现所需的投掷次数,求 $X$的期望值。
    \begin{solution}
        假设期望值为 $E(X)$。若
        \begin{itemize}
        \item 出现反面,期望值变为$E(X)+1$
        \item 出现正面、正面,期望值变为$E(X)+1$
        \item 出现正面、反面、反面,期望值变为$E(X)+3$
        \end{itemize}
        由一步分析法,
        \[
        E(X)=\frac{2}{3}(E(X)+1)+\left(\frac{1}{3}\right)^2(E(X)+1)+\left(\frac{1}{3}\right)\left(\frac{2}{3}\right)^2(E(X)+3)
        \]
        解得
        \[
        E(X)=\frac{33}{2}
        \]
    \end{solution}

    \question 袋子里有 $n+3$ 颗球,其中红球有 3 颗,黑球有 $n$ 颗。每次取一球,取后不放回,直到取到两颗红球为止。设随机变量 $X$ 表示取到两颗红球停止时的次数,求:
    \begin{parts}
    \part 概率 $P(X=k)$,其中 $k=2,3,4,\dots,n+2$。
    \begin{solution}
        三颗红球、$n$ 颗黑球任意排列,共有
        \[
        \comb{n+3}{3}=\frac{(n+3)(n+2)(n+1)}{6}
        \]
        种排列方法。第 $k$ 次取到第 2 颗红球:第 $k$ 次为红球,前 $k-1$ 次中有 1 红球和 $k-2$ 黑球,共 $k-1$ 种排列;后 $n+3-k$ 颗球中有 1 红球和 $n+2-k$ 黑球,共 $n+3-k$ 种排列。因此
        \[
        P(X=k)=\frac{(k-1)(n+3-k)}{(n+3)(n+2)(n+1)/6}
        = \frac{6(k-1)(n-k+3)}{(n+3)(n+2)(n+1)}, \; k=2,3,4,\dots,n+2
        \]
    \end{solution}
    \part 随机变量 $X$ 的期望值 $E(X)$。
    \begin{solution}
        期望值为
        \begin{align*}
        E(X)
        &=\sum_{k=2}^{n+2} k P(X=k) \\
        &= \frac{6}{(n+3)(n+2)(n+1)} \sum_{k=2}^{n+2} k(k-1)(n-k+3)\\
        &= \frac{6}{(n+3)(n+2)(n+1)} \sum_{k=2}^{n+2} \left[-k^3 + (n+4)k^2 - (n+3)k \right] \\
        &= \frac{6}{(n+3)(n+2) (n+1)}\left[(n+3)(n+4) \left(\frac{1}{6}(n+4)(2n+7)-\frac{1}{4}(n+3)(n+6) \right) \right] \\
        &=\frac{n+4}{2}
        \end{align*}
    \end{solution}
    \end{parts}

    \question 在一个由 0 和 1 组成的序列中,"段"指的是由连续相同数字组成的一串,包括长度为 1 的段。例如序列 00100011 中共有四段。对于一个包含 15 个 0 和 9 个 1 的随机排列,求其段数的期望值。
    \begin{solution}
        令随机变量 $X_i$ 表示第 $i$ 个位置是否为某一段的起点。若是,则 $X_i=1$;否则 $X_i=0$。

        第一个位置一定是一段的起点,因此 $E(X_1)=1$。当 $i>1$ 时,第 $i$ 个位置成为段起点当且仅当前一位与当前位不同。其概率为
        \[
        \frac{15}{24}\cdot \frac{9}{23} + \frac{9}{24}\cdot \frac{15}{23}= \frac{45}{92}
        \]
        因此段数的期望值为
        \[
        E(X_1)+E(X_2)+\cdots+E(X_{24})= 1 + 23 \cdot \frac{45}{92}= \frac{49}{4}
        \]
    \end{solution}

    \question $A$ 从帕斯卡三角形的顶端开始。每次移动,她都会向下一层走,并且以相等概率选择向左或向右。完成 6 次移动后,$A$ 经过的数字(包括起点和终点)的期望和是多少?
    
    例如,在下图所示的路径中,$\;A$ 访问的数值之和为 $1 + 1 + 1 + 3 + 6 + 10 + 20 = 42$。 
    \begin{figure}[H]
        \centering
        \includegraphics[width=0.5\textwidth]{images/image8.png}
    \end{figure} 
    \begin{solution}
        设走到第 $n$ 层向右走了 $k$ 步,则 $k\sim\mathrm{Binom}\left(n,\dfrac12\right)$,该层数值为 $\dbinom{n}{k}$,则第 $n$ 层的期望值
        \[
        \mathbb E_n
        =\sum_{k=0}^{n}\binom{n}{k}^2\left(\frac12\right)^n
        \]
        由性质
        \[
        \sum_{k=0}^{n}\binom{n}{k}^2=\binom{2n}{n}
        \]
        总期望值为
        \[
        \sum_{n=0}^{6}\left(\frac12\right)^n\binom{2n}{n}
        =1 + 1 + \frac{3}{2} + \frac{5}{2} + \frac{35}{8} + \frac{63}{8} + \frac{231}{16}
        =\frac{523}{16}
        \]
    \end{solution}

\end{questions}
\pagebreak

\begin{center}
  {\fontsize{30pt}{26pt}\selectfont
    \hypertarget{统计}{统计} \label{统计}
  }
\end{center}
\separator
\vspace{1pt}

\begin{questions}
    \question 连续投掷一枚公平硬币,直到首次出现连续两个反面为止。设随机变量 $X$ 为所需的投掷次数,求 $X$ 的期望与方差。
    \begin{solution}
        令 $p(n)$ 表示投掷 $n$ 次铜板才出现连续两次反面的概率。若第一次出现正面,后 $n-1$ 次才出现连续两次反面。若第一次出现反面,第二次出现正面,后 $n-2$ 次才出现连续两次反面。有递推式:
        \[
        p(n) = \frac{1}{2} p(n-1) + \frac{1}{4} p(n-2),\quad n\ge 3,\quad p(1)=0,\quad p(2)=\frac{1}{4}
        \]
        于是
        \begin{align*}
        E(X) &= \sum_{k=1}^\infty k p(k) = 1\cdot 0 + 2\cdot \frac{1}{4} + \sum_{k=3}^\infty k\, p(k) \\
        &= \frac{1}{2} + \sum_{k=3}^\infty \left(\frac{1}{2} k p(k-1) + \frac{1}{4} k p(k-2)\right) \\
        &= \frac{1}{2} + \frac{1}{2} (1+E(X)) + \frac{1}{4} (2+E(X)) \Rightarrow E(X) = 6
        \end{align*}
        且
        \begin{align*}
        E(X^2) &= \sum_{k=1}^\infty k^2 p(k) 
        = 1^2 \cdot 0 + 2^2 \cdot \frac{1}{4} + \sum_{k=3}^\infty k^2 p(k) \\
        &= 1 + \sum_{k=3}^\infty \left(\frac{1}{2} k^2 p(k-1) + \frac{1}{4} k^2 p(k-2)\right) \\
        &= 1 + \frac{1}{2} (E(X^2) + 2 E(X) + 1) + \frac{1}{4} (E(X^2) + 4 E(X) + 4) \Rightarrow E(X^2) = 58
        \end{align*}
        故方差为
        \[
        Var(X) = E(X^2) - (E(X))^2 = 22
        \]
    \end{solution}

    \question 一有 $n$ 项的等差数列 $\{a_n\}$,公差为 $d=\dfrac{\sqrt{13}}{2}$,此数列的方差为 $260$,求 $n$。
    \begin{solution}
        设该等差数列为 $a,a+d,\dots,a+(n-1)d$。平均数为
        \[
        \overline{X}=a+\frac{n-1}{2}d
        \]
        且
        \[
        E(X^2)=\frac{1}{n}\sum_{k=0}^{n-1}(a+kd)^2
        =a^2+ad(n-1)+d^2\frac{(n-1)(2n-1)}{6}
        \]
        因此方差为
        \[
        \operatorname{Var}(X)=E(X^2)-\overline{X}^2
        =d^2\left(\frac{(n-1)(2n-1)}{6}-\frac{(n-1)^2}{4}\right)
        =\frac{d^2(n^2-1)}{12}
        \]
        代入 $d^2=\dfrac{13}{4}$ 且 $\operatorname{Var}(X)=260$得
        \[
        \frac{13}{4}\cdot\frac{n^2-1}{12}=260 \Rightarrow n=31
        \]
    \end{solution}

    \question 重复操作一个成功概率为 $p$ 的伯努利试验,且 $$S_n = \sum_{k=1}^{n} k^2 (1-p)^k,$$若 $\displaystyle\lim_{n \to \infty} S_n = 180$,试求第三次才出现第一次成功的概率。
    \begin{solution}
        令 $X \sim \text{Geometric}(p)$表示第一次成功出现所需的试验次数,则有
        \[
        E(X^2) = \text{Var}(X) + (E(X))^2 = \dfrac{1-p}{p^2} + \left(\dfrac{1}{p}\right)^2 = \dfrac{2 - p}{p^2}
        \]
        且据题意有
        \[
        E(X^2) = \dfrac{p}{1 - p} \cdot S_\infty
        \Rightarrow \dfrac{2 - p}{p^2} = \dfrac{p}{1 - p} \cdot 180
        \]
        解得
        \[
        180p^3 - p^2 + 3p - 2 =(5p - 1)(36p^2 + 7p + 2) = 0
        \Rightarrow p = \dfrac{1}{5} \quad (\text{舍去虚根})
        \]
        因此
        \[
        P(X = 3) = (1 - p)^2p  = \dfrac{16}{125}
        \]
    \end{solution}

    \question 某实验测得20组样本点 $(x_1,y_1), (x_2,y_2), \dots, (x_{20},y_{20})$,已知
    \[
    \sum_{i=1}^{20} x_i = 400, \quad \sum_{i=1}^{20} y_i = 900.
    \]
    利用最小平方法求得 $y$ 对 $x$ 的回归直线方程为 $y = a x + b$。若
    \[
    \sum_{i=1}^{20} (y_i - a x_i - b)^2 = 0,
    \]
    且 $(x_1,y_1) = (30,40)$,设
    \[
    x' = 2x - 4, \quad y' = -3y + 5, \quad i=1,2,\dots,20,
    \]
    求数据 $(x', y')$ 的回归直线方程。
    \begin{solution}
        有
        \[
        E(X) = \frac{400}{20} = 20, \quad E(Y) = \frac{900}{20} = 45,
        \]
        \[
        E(X') = E(2X - 4) = 2 \cdot 20 - 4 = 36, \quad E(Y') = E(-3Y + 5) = -3 \cdot 45 + 5 = -130.
        \]
        原回归直线经过 $(30,40)$ 与 $(20,45)$,斜率与截距为
        \[
        y = -\frac{1}{2} x + 55
        \]
        故
        \[
        m = \frac{\mathrm{Cov}(2X - 4, -3Y + 5)}{\mathrm{Var}(2X - 4)} = \frac{-6\,\mathrm{Cov}(X,Y)}{4\,\mathrm{Var}(X)} = -\frac{6}{4} \cdot \left(-\frac{1}{2}\right) = \frac{3}{4}.
        \]
        回归直线过点 $(E(X'), E(Y')) = (36, -130)$,故方程为
        \[
        y' - (-130) = \frac{3}{4}(x' - 36) \Rightarrow y' = \frac{3}{4} x' - 157.
        \]
    \end{solution}
    
    \question 已知5组数据如下,
    \[
    \begin{tabular}{c|ccccc}
    \hline
    $X$ & 1 & 2 & 5 & 3 & 4 \\
    \hline
    $Y$ & 3 & $a$ & 5 & $b$ & 6 \\
    \hline
    \end{tabular}
    \]
    且 $Y$ 对 $X$ 的回归直线为 
    \[
    Y = \frac{3}{5}X + \frac{11}{5},
    \]
    求数对 $(a,b)$。
    \begin{solution}
        有
        \[
        \bar x = \frac{1+2+5+3+4}{5} = 3.
        \]
        由于$\bar x,\bar y$在回归直线上,
        \[
        \bar y = \frac{3}{5} \bar x + \frac{11}{5} =\frac{3 + a + 5 + b + 6}{5} \Rightarrow a + b = 6 \tag{1}
        \]
        由回归直线斜率公式,
        \[
        \frac{3}{5} = \frac{\sum x_i y_i - n \bar x \bar y}{\sum x_i^2 - n \bar x ^2} = \frac{3+2a+25+3b+24-5 \cdot 3 \cdot 4}{55-5 \cdot 3^2} \tag{2}
        \]
        由$(1),(2)$解得
        \[
        (a,b)=(4,2)
        \]
    \end{solution}

    \question 有一笔资料包含 $n$ 个数 $x_1, x_2, x_3, \dots, x_n$,其平均数为 $\overline{x}$,试证
    \[
    \sqrt{\frac{1}{n}\sum_{i=1}^{n}x_{i}^{2}-\overline{x}^{2}}\ge\frac{1}{n}\sum_{i=1}^{n}|x_{i}-\overline{x}|
    \]
    \begin{solution}
        由柯西不等式,
        \[
        \left(\sum_{i=1}^n(x_i-\bar x)^2 \right) \left(\sum_{i=1}^n 1^2\right) \ge \left(\sum_{i=1}^n |x_i-\bar x| \right)^2
        \]
        得
        \[
        \frac{1}{n} \sum_{i=1}^n (x_i - \bar x)^2 \ge \left( \frac{1}{n} \sum_{i=1}^n |x_i - \bar x| \right)^2
        \]
        又
        \begin{align*}
        \frac{1}{n} \sum_{i=1}^n (x_i - \bar{x})^2 &= \frac{1}{n} \left( \sum_{i=1}^n x_i^2 - 2\bar{x} \sum_{i=1}^n x_i + \sum_{i=1}^n \bar{x}^2 \right)\\
        &= \frac{1}{n} \left( \sum_{i=1}^n x_i^2 - 2n\bar{x}^2 + n\bar{x}^2 \right)\\
        &= \frac{1}{n} \sum_{i=1}^n x_i^2 - \bar{x}^2
        \end{align*}
        即得证
        \[
        \sqrt{\frac{1}{n}\sum_{i=1}^{n}x_{i}^{2}-\overline{x}^{2}}\ge\frac{1}{n}\sum_{i=1}^{n}|x_{i}-\overline{x}|
        \]
    \end{solution}
\end{questions}