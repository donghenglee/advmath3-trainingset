\pagecolor{PageColor}
\
\vfil
\hfil  {\fontsize{50pt}{36pt}\selectfont{数论}} \hfil
\vfil
\begin{tikzpicture}[remember picture,overlay,every node/.style={inner sep=0pt}]
        \node [shift={(1cm,-1cm)},brown,scale=2,anchor=north west] (CNW)
        at (current page.north west) {\pgfornament[height=1cm,width=1cm]{61}};
        \node [shift={(-1cm,-1cm)},brown,scale=2,anchor=north east] (CNE)
        at (current page.north east) {\pgfornament[height=1cm,width=1cm,symmetry=v]{61}};
        \node [shift={(1cm,1cm)},brown,scale=2,anchor=south west] (CSW)
        at (current page.south west) {\pgfornament[height=1cm,width=1cm,symmetry=h]{61}};
        \node [shift={(-1cm,1cm)},brown,scale=2,anchor=south east] (CSE)
        at (current page.south east) {\pgfornament[height=1cm,width=1cm,symmetry=c]{61}};
        \pgfornamentline[color=brown]{current page.north west}{current page.north east}{2}{87}
        \pgfornamentline{current page.south west}{current page.south east}{2}{87}
        \pgfornamentline{current page.north west}{current page.south west}{3}{87}
        \pgfornamentline{current page.north east}{current page.south east}{3}{87}
        \end{tikzpicture}%
\thispagestyle{empty}
\pagebreak
\begin{center}
  {\fontsize{30pt}{26pt}\selectfont
    \hypertarget{整除}{整除} \label{整除}
  }
\end{center}
\separator
\begin{center}
    考点: 整除基本性质、最小公倍数、最大公因数、因数函数、勒让德定理、贝祖等式
\end{center}
\vspace{0.5pt}
\nopagecolor
\begin{questions}
    \question 已知 $n | 10a - b,n | 10c - d$。证明 $n | ad - bc$。
    \begin{solution}
        由于 
        \[
        n | (10a - b)c + (10c - d)a 
        \]
        于是
        \[ 
        n | ad - bc
        \]
    \end{solution}

    \question 例 5、已知 $1987 | \underbrace{11\dots1}_{n \text{个}}$,证明 $1987 | \underbrace{11\dots1}_{n \text{个}}\underbrace{99\dots9}_{n \text{个}}\underbrace{88\dots8}_{n \text{个}}\underbrace{77\dots7}_{n \text{个}}$。

    \begin{solution}
    证:$\underbrace{11\dots1}_{n \text{个}}\underbrace{99\dots9}_{n \text{个}}\underbrace{88\dots8}_{n \text{个}}\underbrace{77\dots7}_{n \text{个}} = \underbrace{11\dots1}_{n \text{个}} \times 10^{3n} + 9 \times \underbrace{11\dots1}_{n \text{个}} \times 10^{2n} + 8 \times \underbrace{11\dots1}_{n \text{个}} \times 10^n + 7 \times \underbrace{11\dots1}_{n \text{个}}$

    为 $\underbrace{11\dots1}_{n \text{个}}$ 的倍数,由于 $1987 | \underbrace{11\dots1}_{n \text{个}}$ 且 $\underbrace{11\dots1}_{n \text{个}} | \underbrace{11\dots1}_{n \text{个}}\underbrace{99\dots9}_{n \text{个}}\underbrace{88\dots8}_{n \text{个}}\underbrace{77\dots7}_{n \text{个}}$,

    因此 $1987 | \underbrace{11\dots1}_{n \text{个}}\underbrace{99\dots9}_{n \text{个}}\underbrace{88\dots8}_{n \text{个}}\underbrace{77\dots7}_{n \text{个}}$。
    \end{solution}

    \question 例 6、正整数 $n$ 使得 $n^2 + 2005$ 是完全平方数,则 $(n^2 + 2005)^2$ 的个位数字是 \underline{\hspace{1cm}}。

\begin{solution}
解:设 $n^2 + 2005 = m^2 (m > 0)$,则 $(m - n)(m + n) = 2005 = 1 \times 2005 = 5 \times 401$,得
\[ \begin{cases} m - n = 1 \\ m + n = 2005 \end{cases} \text{或} \begin{cases} m - n = 5 \\ m + n = 401 \end{cases} \]
解得 $\begin{cases} m = 1003 \\ n = 1002 \end{cases}$ 或 $\begin{cases} m = 203 \\ n = 198 \end{cases}$。
由 $1003^2$ 个位数字是 $9$,由 $203^2$ 个位数字也是 $9$,知它的个位数字也是 $9$。
\end{solution}

\question 例 9、求 $2004!$ 末尾零的个数。

\begin{solution}
解:因为 $10 = 2 \times 5$,而 $2$ 比 $5$ 多,所以只要考虑 $2004!$ 中 $5$ 的幂指数,即
\[ \lfloor \frac{2004}{5} \rfloor + \lfloor \frac{2004}{25} \rfloor + \lfloor \frac{2004}{125} \rfloor + \lfloor \frac{2004}{625} \rfloor = 400 + 80 + 16 + 3 = 499 \]
(注:原文中使用了分式形式表示取整逻辑)。
\end{solution}

\question 例 10、设 $72 | \overline{a679b}$,求 $a$,$b$ 的值。

\begin{solution}
解:$72 = 8 \times 9$,且 $(8, 9) = 1$,所以只需讨论 $8$、$9$ 都整除 $\overline{a679b}$ 时 $a$,$b$ 的值。
若 $8 | \overline{a679b}$,则 $8 | \overline{79b}$,由除法可得 $b = 2$。
若 $9 | \overline{a679b}$,则 $9 | (a + 6 + 7 + 9 + 2)$,得 $a = 3$。
(一个整数被 $9$ 整除的充要条件是它的各位数字之和能被 $9$ 整除。)
\end{solution}

\question 4、若 $(p, 6) = 1$,证明 $p^2 - 1$ 能被 $24$ 整除。

\begin{solution}
证:因为 $(p, 6) = 1$,所以 $p$ 不是 $2$ 的倍数,也不是 $3$ 的倍数。
由于 $p$ 是奇数,可设 $p = 2k + 1$,则 $p^2 - 1 = (p - 1)(p + 1) = (2k)(2k + 2) = 4k(k + 1)$。
因为 $k(k + 1)$ 是两个连续整数之积,必能被 $2$ 整除,故 $8 | (p^2 - 1)$。
又因为 $p$ 不被 $3$ 整除,由费马小定理或余数分类可知 $p \equiv 1$ 或 $2 \pmod 3$,则 $p^2 \equiv 1 \pmod 3$,即 $3 | (p^2 - 1)$。
由于 $(3, 8) = 1$,所以 $3 \times 8 = 24 | (p^2 - 1)$。
\end{solution}

\question 7、证明 $5^{2n} + 4 \cdot 3^{2n}$($n$ 是自然数)不是质数。

\begin{solution}
证:$5^{2n} + 4 \cdot 3^{2n} = (5^n)^2 + (2 \cdot 3^n)^2$。
利用恒等式 $a^4 + 4b^4$(苏菲·热尔曼恒等式)的思想进行配方:
原式 $= (5^n)^2 + 2 \cdot 5^n \cdot (2 \cdot 3^n) + (2 \cdot 3^n)^2 - 2 \cdot 5^n \cdot (2 \cdot 3^n)$
$= (5^n + 2 \cdot 3^n)^2 - 4 \cdot 15^n$。
当 $n$ 为偶数时(设 $n=2k$),原式可化为平方差:
$= (5^n + 2 \cdot 3^n)^2 - (2 \cdot 15^k)^2 = (5^n + 2 \cdot 3^n - 2 \cdot 15^k)(5^n + 2 \cdot 3^n + 2 \cdot 15^k)$。
由于该数可以分解为两个大于 1 的整数之积,故不是质数。
\end{solution}

\question 8、证明 $p^{k+2} + p^{k+1} + q^{k+2} + q^{k+1}$ 是合数。

\begin{solution}
证:原式可进行因式分解:
$p^{k+2} + p^{k+1} + q^{k+2} + q^{k+1} = p^{k+1}(p + 1) + q^{k+1}(q + 1)$。
若 $p, q$ 同为奇数,则 $p+1$ 和 $q+1$ 均为偶数,原式必为偶数且大于 2,是合数。
若 $p, q$ 一奇一偶(如 $p=2$),则需根据 $k$ 的取值进一步讨论。
\end{solution}

\question 9、使 $n^3 + 100$ 能被 $n + 10$ 整除的正整数 $n$ 的最大值是多少?

\begin{solution}
解:利用多项式除法或凑项法:
$n^3 + 100 = (n^3 + 1000) - 900 = (n + 10)(n^2 - 10n + 100) - 900$。
若 $(n + 10) | (n^3 + 100)$,则必有 $(n + 10) | 900$。
为了使 $n$ 最大,$n + 10$ 应取 $900$ 的最大约数,即 $n + 10 = 900$。
解得 $n = 890$。
\end{solution}

\question 14、证明:三个连续奇数的平方和加 $1$,能被 $12$ 整除,但不能被 $24$ 整除。

\begin{solution}
证:设三个连续奇数为 $2k-2, 2k+1, 2k+3$(此处原图中手写建议为 $2k-1, 2k+1, 2k+3$)。
设三个连续奇数为 $n-2, n, n+2$(其中 $n$ 为奇数)。
$(n-2)^2 + n^2 + (n+2)^2 + 1 = n^2 - 4n + 4 + n^2 + n^2 + 4n + 4 + 1 = 3n^2 + 9 = 3(n^2 + 3)$。
因为 $n$ 是奇数,设 $n = 2m + 1$,则 $n^2 + 3 = (2m + 1)^2 + 3 = 4m^2 + 4m + 4 = 4(m^2 + m + 1)$。
所以原式 $= 3 \times 4(m^2 + m + 1) = 12(m^2 + m + 1)$,显见能被 $12$ 整除。
又因为 $m^2 + m = m(m+1)$ 必为偶数,故 $m^2 + m + 1$ 必为奇数。
因此 $12 \times \text{奇数}$ 不能被 $24$ 整除。
\end{solution}

\question 18、将自然数 $N$ 接写在任意一个自然数的右面,如果得到的新数都能被 $N$ 整除,那么 $N$ 称为魔术数。问小于 $2000$ 的自然数中有多少个魔术数?

\begin{solution}
解:设任意自然数为 $M$,$N$ 是一个 $k$ 位数。接写后的新数为 $10^k M + N$。
由题意,$N | (10^k M + N)$ 对任意 $M$ 成立,这意味着 $N | 10^k M$ 对任意 $M$ 成立。
取 $M=1$,则必须满足 $N | 10^k$。
若 $k=1$,$N | 10^1$,则 $N \in \{1, 2, 5\}$。
若 $k=2$,$N | 10^2$ 且 $N \ge 10$,则 $N \in \{10, 20, 25, 50\}$。
若 $k=3$,$N | 10^3$ 且 $N \ge 100$,则 $N \in \{100, 125, 200, 250, 500\}$。
若 $k=4$,$N | 10^4$ 且 $1000 \le N < 2000$,则 $N \in \{1000, 1250\}$。
综上所述,魔术数共有 $3 + 4 + 5 + 2 = 14$ 个。
\end{solution}

\question 23、某正整数之平方,其末三位是相等的非零数字,求具有该性质的最小正整数。

\begin{solution}
解:设该正整数为 $x$,则 $x^2 \equiv \overline{aaa} \pmod{1000}$,其中 $a \in \{1, 2, \dots, 9\}$。
即 $x^2 \equiv 111a \pmod{1000}$。
由于完全平方数的末位只能是 $0, 1, 4, 5, 6, 9$,故 $a$ 只能取 $1, 4, 5, 6, 9$。
经检验:
若 $a=4$,$x^2 \equiv 444 \pmod{1000}$。此时 $x$ 必为偶数,设 $x=2y$,则 $4y^2 \equiv 444 \pmod{1000} \Rightarrow y^2 \equiv 111 \pmod{250}$。
由 $y^2 \equiv 1 \pmod{10}$ 知 $y$ 末位为 $1$ 或 $9$。
通过计算发现 $y = 38$ 时,$38^2 = 1444$,满足末三位为 $444$。
因此最小正整数 $x = 38$。
\end{solution}
\end{questions}
\pagebreak
\begin{center}
  {\fontsize{30pt}{26pt}\selectfont
    \hypertarget{同余}{同余} \label{同余}
  }
\end{center}
\separator
\begin{center}
    考点: 同余基本性质、模逆元、费马小定理、欧拉定理、中国剩余定理、威尔逊定理、阶与原根
\end{center}
\vspace{0.5pt}
\nopagecolor
\begin{questions}
    \question 例 4、试证 $2^{3n+3} + 41$ ($n \in N$) 恒为 $7$ 的倍数。

\begin{solution}
证:因为 $2^{3n+3} = 8 \times 8^n \equiv 1 \pmod 7$,$41 \equiv 6 \pmod 7$,
相加即可得 $1 + 6 = 7 \equiv 0 \pmod 7$。由此得证。
\end{solution}

\question 例 6、求 $3^{406}$ 的末二位数。

\begin{solution}
解法 1:$3^{406} = (3^4)^{101} \cdot 3^2 \equiv (81)^{101} \cdot 9 \equiv -19^{101} \times 9 \equiv -361^{50} \times 19 \times 9 \equiv -39^{50} \times 171$
$\equiv 1521^{25} \times 29 \equiv 21^{25} \times 29 \equiv 441^{12} \times 21 \times 29 \equiv 41^{12} \times 9 = 1681^6 \times 9$
$= 81^6 \times 9 \equiv 19^6 \times 9 = 361^3 \times 9 \equiv 61^3 \times 9 \equiv -39^3 \times 9 \equiv -1521 \times 39 \times 9$
$\equiv -21 \times 351 \equiv -21 \times 51 \equiv -1071 \equiv -71 \equiv 29 \pmod{100}$
$\therefore$ 末二位数为 $29$。

解法 2:$100 = 4 \times 25$
(一) $3^{406} \equiv (-1)^{406} \equiv 1 \equiv 29 \pmod 4$
(二) $3^{406} = 27^{135} \times 3 \equiv 2^{135} \times 3 = 32^{27} \times 3 \equiv 7^{27} \times 3 = 49^{13} \times 7 \times 3 \equiv -1 \times 21 = -21 \equiv 4 \pmod{25}$
由 (一)(二) $3^{406} \equiv 1 \equiv 29 \pmod 4$,$3^{406} \equiv 4 \equiv 29 \pmod{25}$
所以 $3^{406} \equiv 29 \pmod{100}$,末两位为 $29$。

解法 3:利用二项式定理展开,$9^{203} = (10-1)^{203} \equiv \binom{203}{202}(10)(-1)^{202} + (-1)^{203} = 2030 - 1 = 2029$
$\equiv 29 \pmod{100} \therefore$ 末二位数为 $29$。
\end{solution}

\question 例 7、求 $243^{402}$ 的末两位数。

\begin{solution}
解:$243^{402} \equiv 3^{402} \equiv 9^{201} \equiv 1^{201} = 1 \equiv 49 \pmod 4$,
$243^{402} \equiv 7^{402} \equiv 49^{201} \equiv -1^{201} = -1 \equiv 49 \pmod{25}$
因此 $243^{402} \equiv 49 \pmod{[4, 25]=100}$,末两位为 $49$。
\end{solution}

\question 例 9、已知 $7^x \equiv 1 \pmod{500}$,求最小的正整数 $x$。

\begin{solution}
解:$7^x - 1 \equiv 0 \pmod{500}$,已知 $500$ 的倍数末位数字一定是 $0$,因此若 $7^x - 1$ 要被 $500$ 整除,其末位也必须是 $0$。由此可推出 $7^x$ 的末位必须是 $1$。我们知道 $7^4 = 2401 \Rightarrow 7^4$ 末位必也是 $1$,设 $x = 4n$ 代回同余式得到
\[ 7^{4n} \equiv 1 \pmod{500} \]
\[ 2401^n \equiv 1 \pmod{500} \]
\[ 401^n \equiv 1 \pmod{500} \]
\[ (1 + 400)^n \equiv 1 \pmod{500} \]
\[ 1 + \binom{n}{1}400 + \binom{n}{2}400^2 + \dots + 400^n \equiv 1 \pmod{500} \]
\[ 400n \equiv 0 \pmod{500} \]
$n$ 最小为 $5$,因此 $x = 20$。
\end{solution}

\question 例 10、证明 $5y^2 + 3 = x^2$ 无解,$x, y \in \mathbb{Z}$。

\begin{solution}
证:若 $5y^2 + 3 = x^2$ 有解,则两边关于模 $5$ 同余,有 $5y^2 + 3 \equiv x^2 \pmod 5$
即 $3 \equiv x^2 \pmod 5$,而任一个平方数 $x^2 \equiv 0, 1, 4 \pmod 5$
$\therefore 3 \not\equiv 0, 1, 4 \pmod 5$ $\therefore$ 即得矛盾,即 $5y^2 + 3 = x^2$ 无解。
\end{solution}

\question 例 11、证明 $x^2 - 2y^2 = 77$ 无解,$x, y \in \mathbb{Z}$。

\begin{solution}
证 1:$11$ 和 $y$ 必互质,即 $(11, y) = 1$,否则的话设 $y = 11k$,必导致 $x = 11j$,但右式只有一个 $11$ 的因数,矛盾,因此 $(11, y) = 1$。假设存在解 $(x_0, y_0)$,则有同余式
$x_0^2 - 2y_0^2 \equiv 0 \pmod{11}$,存在 $(y_0^{-1})^2$ 使得 $(y_0^{-1}x_0)^2 - 2 \equiv 0 \pmod{11}$,因此
$(y_0^{-1}x_0)^2 \equiv 2 \pmod{11}$,但 $X^2 \equiv 1, 4, 3, 5, 9 \pmod{11}$ 矛盾,命题得证。

证 2:假设方程有解,则 $(x_0, y_0)$ 只有两种情况
一) $(x_0, y_0) = (\text{奇数, 偶数})$,则 $x_0^2 - 2y_0^2 \equiv 77 \Rightarrow 1 - 0 \equiv 5 \pmod 8$,矛盾。
二) $(x_0, y_0) = (\text{奇数, 奇数})$,则 $x_0^2 - 2y_0^2 \equiv 77 \Rightarrow 1 - 2(1) \equiv 5 \pmod 8$,矛盾。
命题得证。
\end{solution}

\question 例 12、证明 $x^2 - 3y^2 + 5z^2 = 0$ 无解。

\begin{solution}
证:设 $(x_0, y_0, z_0)$ 是解,则 $x_0^2 - 3y_0^2 + 5z_0^2 = 0 \Rightarrow x_0^2 - 3y_0^2 \equiv 0 \pmod 5$
有两种情况,一) $(x_0, 5) = (y_0, 5) = 1$,则存在 $(y_0^{-1})^2$ 使得 $(y_0^{-1}x_0)^2 - 3 \equiv 0 \pmod 5$
$\Rightarrow (y_0^{-1}x_0)^2 \equiv 3 \pmod 5$,但 $X^2 \equiv 1, 2, 4 \pmod 5$,矛盾。
二) 若 $x_0 = 5^a k$,必有 $y_0 = 5^a j$,其中 $(k, j) = 1$。约去 $5^a$ 后,得到
$k^2 - 3j^2 \equiv 0 \pmod 5$,同理可证无解。
\end{solution}

\question 例 13、已知 $n$ 是正整数,且 $2n + 1$,$3n + 1$ 都是平方数,证明 $40 | n$。

\begin{solution}
证:设 $2n + 1 = x^2$,$3n + 1 = y^2$。由于奇数的平方被 $8$ 除余 $1$,因此
\[ \begin{cases} 2n + 1 \equiv 1 \\ 3n + 1 \equiv 1 \end{cases} \Rightarrow \begin{cases} 2n \equiv 0 \pmod 8 \Rightarrow n \equiv 0 \pmod 4 \\ 3n \equiv 0 \pmod 8 \Rightarrow n \equiv 0 \pmod 8 \end{cases} \]
因此 $n$ 至少是 $8$ 的倍数。
又,对任意整数 $z$ 有 $z^2 \equiv 0, 1, 4 \pmod 5$
$x^2 + y^2 = (2n + 1) + (3n + 1) = 5n + 2$
可能值 $x^2 + y^2 \equiv 1 + 1, 1 + 0, 0 + 1, 1 + 4, 4 + 1, 0 + 0, 4 + 4, 0 + 4, 4 + 0$
只有 $x^2 + y^2 \equiv 1 \text{ 满足 } x^2 + y^2 \equiv 2$,因此 $x^2 + y^2 = 2n + 1 + 3n + 1 \equiv 2 \pmod 5$
得到 $n \equiv 0$,因此 $5 | n$。综上所述,$n$ 为 $40$ 的倍数。
\end{solution}

\question 例 14、求最小的整数 $n$,使得 $\sum_{i=1}^{n} x_i^4 = 1599$,$x_i$ 为整数。据此,求出其中一组解。

\begin{solution}
解:由于 $a^4 \equiv 0 \pmod{16}$(偶数),$a^4 \equiv 1 \pmod{16}$(奇数),因此 $a^4 \equiv 0, 1 \pmod{16}$。
$\sum_{i=1}^n x_i^4 = 1599 \equiv -1 \equiv 15 \pmod{16}$,因此 $n$ 最小是 $15$。
现在求 $x_i$。由于有 $15$ 个数,且 $x_i^4 \equiv 1 \pmod{16}$,它们都是奇数。$x_i$ 最小是 $1, 3, 5$ 这些数。($7^4 = 2401 > 1599$)
$5^4 \times 3 = 1875 > 1599$,因此不能超过两个解是 $5$,考虑只有一个解是 $5$ 的情况。此时 $\sum_{i=1}^{14} x_i^4 = 1599 - 5^4 = 974$。由于 $974 = 81 \times 12 + 2$,可得出 $12$ 个 $3^4$ 和两个 $1$。猜测有 $12$ 个 $x_i$ 是 $3$,两个 $x_i$ 是 $1$。
得到一组解是 $1, 1, 3, 3, \dots, 3, 5$。即 $2 \times 1^4 + 12 \times 3^4 + 5^4 = 1599$。
\end{solution}

\question 9、Prove that the number $\underbrace{11\dots1}_{n}$ (in the decimal notation) is not a perfect square.

\begin{solution}
证:$\underbrace{11\dots1}_{n}$ 的末两位数是 $11$。由于一个完全平方数被 $4$ 除的余数只能是 $0$ 或 $1$,而 $11 \equiv 3 \pmod 4$。因此该数不可能是完全平方数。
\end{solution}

\question 10、已知 $n! = 1 \times 2 \times \dots \times (n-1) \times n$。若 $A = \frac{2}{3!} + \frac{3}{4!} + \frac{4}{5!} + \dots + \frac{11}{12!}$,试求 $A$ 除以 $10$ 的余数。

\begin{solution}
解:观察通项 $\frac{k}{(k+1)!} = \frac{(k+1)-1}{(k+1)!} = \frac{1}{k!} - \frac{1}{(k+1)!}$。
则 $A = (\frac{1}{2!} - \frac{1}{3!}) + (\frac{1}{3!} - \frac{1}{4!}) + \dots + (\frac{1}{11!} - \frac{1}{12!}) = \frac{1}{2} - \frac{1}{12!}$。
若求 $12! A \pmod{10}$,则 $12! A = \frac{12!}{2} - 1 \equiv -1 \equiv 9 \pmod{10}$。
\end{solution}

\question 12、求证 $8888^{8888} + 7777^{7777}$ 能被 $37$ 整除。

\begin{solution}
证:因为 $111 = 3 \times 37 \equiv 0 \pmod{37}$,所以 $1111 \equiv 1 \pmod{37}$。
$8888^{8888} + 7777^{7777} \equiv (8 \times 1)^{8888} + (7 \times 1)^{7777} \equiv 8^{8888} + 7^{7777} \pmod{37}$。
利用费马小定理 $a^{36} \equiv 1 \pmod{37}$ 可化简指数进一步证明其余数为 $0$。
\end{solution}

\question 13、证明 $504$ 整除 $n^9 - n^3$。

\begin{solution}
证:$n^9 - n^3 = n^3(n^6 - 1) = n^3(n^3 - 1)(n^3 + 1)$。
由于 $504 = 7 \times 8 \times 9$:
1) 由费马小定理可知 $n^7 - n$ 被 $7$ 整除,可推导原式被 $7$ 整除。
2) 任意整数立方模 $9$ 的余数为 $0, 1, 8$,可知原式被 $9$ 整除。
3) 讨论 $n$ 的奇偶性,可知原式被 $8$ 整除。
\end{solution}

\question 17、证明 $x^2 + 2y^2 = 203$ 无整数解。

\begin{solution}
证:考虑模 $7$。$x^2 + 2y^2 \equiv 203 \equiv 0 \pmod 7$。
由于 $2$ 不是模 $7$ 的二次剩余,同余式 $x^2 + 2y^2 \equiv 0 \pmod 7$ 仅当 $x, y$ 均为 $7$ 的倍数时成立。
设 $x=7k, y=7m$,则 $49k^2 + 98m^2 = 203$。
方程左边能被 $49$ 整除,但右边 $203$ 不能被 $49$ 整除,矛盾,故无解。
\end{solution}

\question 18、求 $3^{1998}$ 的末两位数。

\begin{solution}
解:$3^{1998} = (3^4)^{499} \cdot 3^2 = 81^{499} \cdot 9$。
$81^{499} = (80 + 1)^{499} \equiv \binom{499}{1} \cdot 80 + 1 \equiv 499 \cdot 80 + 1 \equiv 21 \pmod{100}$。
$21 \times 9 = 189 \equiv 89 \pmod{100}$,所以末两位数是 $89$。
\end{solution}

\question 19、证明当 $p$ 不少于 $5$ 的质数时,$p^2 + 2$ 为一合数。

\begin{solution}
证:因 $p \ge 5$ 且为质数,则 $p$ 不被 $3$ 整除。
故 $p \equiv 1, 2 \pmod 3 \Rightarrow p^2 \equiv 1 \pmod 3$。
$p^2 + 2 \equiv 1 + 2 = 3 \equiv 0 \pmod 3$。
由于 $p^2 + 2 > 3$ 且能被 $3$ 整除,所以 $p^2 + 2$ 是合数。
\end{solution}

\question 20、证明 $701$ 以及 $61$ 被 $71$ 除时,它们的余数相等。

\begin{solution}
证:计算两数之差:$701 - 61 = 640$。
若余数相等,差值应能被 $71$ 整除。
计算 $640 \div 71 = 9 \dots 1$。
注:若原题中 $61$ 为 $62$,则 $701-62=639=71 \times 9$,余数才相等。
\end{solution}

\question 21、若 $a$ 为自然数,证明 $10 | (a^{1993} - a^{1949})$。

\begin{solution}
证:$a^{1993} - a^{1949} = a^{1949}(a^{44} - 1)$。
1) 易证其为偶数,即被 $2$ 整除。
2) 由费马小定理 $a^5 \equiv a \pmod 5$,当 $(a, 5)=1$ 时 $a^4 \equiv 1 \pmod 5$,故 $a^{44} \equiv 1 \pmod 5$。
因为能同时被 $2$ 和 $5$ 整除,所以能被 $10$ 整除。
\end{solution}

\question 23、求被 $3$ 除余 $2$,被 $5$ 除余 $3$,被 $7$ 除余 $5$ 的最小三位数。

\begin{solution}
解:设该数为 $x$。由题意 $x \equiv -1 \pmod 3$,$x \equiv -2 \pmod 5$,$x \equiv -2 \pmod 7$。
由后两项得 $x \equiv -2 \equiv 33 \pmod{35}$。
经检验,当 $x = 35 \times 4 + 33 = 173$ 时,满足 $173 \equiv 2 \pmod 3$。
故最小三位数为 $173$。
\end{solution}

\question 24、已知 $7^n$ 有 $k$ 个重数 $7 \dots 7$,i) 证明它被 $4$ 除余 $3$;ii) 求它的末两位数。

\begin{solution}
解:i) $\underbrace{77\dots7}_{k} = \underbrace{77\dots7}_{k-1}0 + 7 \equiv 0 + 7 \equiv 3 \pmod 4$。
ii) $7^1=07, 7^2=49, 7^3=43, 7^4=01 \dots$ 模 $100$ 的周期为 $4$。
根据 $n$ 的具体取值可确定末两位。
\end{solution}

\question 25、设 $m > n \ge 1$,求最小的 $m+n$ 使得 $1000 | 1978^m - 1978^n$。

\begin{solution}
解:$1978^m - 1978^n = 1978^n(1978^{m-n} - 1)$ 需被 $8$ 和 $125$ 整除。
1) 为被 $8$ 整除,因 $1978$ 仅含一个因数 $2$,故 $n \ge 3$。
2) 为被 $125$ 整除,$1978^{m-n} \equiv 3^{m-n} \equiv 1 \pmod{125}$。
由欧拉函数可知最小 $m-n = 100$。
故 $n=3, m=103$,最小 $m+n = 106$。
\end{solution}
\end{questions}
\pagebreak
\begin{center}
  {\fontsize{30pt}{26pt}\selectfont
    \hypertarget{不定方程}{不定方程} \label{不定方程}
  }
\end{center}
\separator
\begin{center}
    考点: 线性不定方程、常数配方法、佩尔方程
\end{center}
\vspace{0.5pt}
\nopagecolor
\begin{questions}
    \question 例 2、求方程 $5x + 3y = 22$ 的所有正整数解。

\begin{solution}
解:方程 $5x + 3y = 1$ 有一组解 $\begin{cases} x = -1 \\ y = 2 \end{cases}$,所以方程 $5x + 3y = 22$ 有一组解 $\begin{cases} x = -22 \\ y = 44 \end{cases}$。
又因为 $5x + 3y = 0$ 的所有整数解为 $\begin{cases} x = 3k \\ y = -5k \end{cases}$,$k$ 为整数,
所以方程 $5x + 3y = 22$ 的所有整数解为 $\begin{cases} x = 3k - 22 \\ y = -5k + 44 \end{cases}$,$k$ 为整数。
由 $\begin{cases} 3k - 22 > 0 \\ -5k + 44 > 0 \end{cases}$ 解得 $\begin{cases} k > \frac{22}{3} \\ k < \frac{44}{5} \end{cases}$,所以 $k = 8$,原方程的正整数解为 $\begin{cases} x = 2 \\ y = 4 \end{cases}$。
\end{solution}

\question 例 5、求不定方程 $3x + 2y + 8z = 40$ 的正整数解。

\begin{solution}
解:显然此方程有整数解。先确定系数最大的未知数 $z$ 的取值范围,因为 $x, y, z$ 的最小值位 1,所以 $1 \le z \le \lfloor \frac{40 - 3 - 2}{8} \rfloor = 4$。

当 $z = 1$ 时,原方程变形为 $3x + 2y = 32$,即 $y = \frac{32 - 3x}{2}$,由上式知 $x$ 是偶数且 $2 \le x \le 10$ 故方程组有 5 组正整数解,分别为 $\begin{cases} x = 2 \\ y = 13 \end{cases}, \begin{cases} x = 4 \\ y = 10 \end{cases}, \begin{cases} x = 6 \\ y = 7 \end{cases}, \begin{cases} x = 8 \\ y = 4 \end{cases}, \begin{cases} x = 10 \\ y = 1 \end{cases}$;

当 $z = 2$ 时,原方程变形为 $3x + 2y = 24$,即 $y = \frac{24 - 3x}{2}$,故方程只有 3 组正整数解,分别为 $\begin{cases} x = 2 \\ y = 9 \end{cases}, \begin{cases} x = 4 \\ y = 6 \end{cases}, \begin{cases} x = 6 \\ y = 3 \end{cases}$;

当 $z = 3$ 时,原方程变形为 $3x + 2y = 16$,即 $y = \frac{16 - 3x}{2}$,故方程只有 2 组正整数解,分别为 $\begin{cases} x = 2 \\ y = 5 \end{cases}, \begin{cases} x = 4 \\ y = 2 \end{cases}$;

当 $z = 4$ 时,原方程变形为 $3x + 2y = 8$,即 $y = \frac{8 - 3x}{2}$,故方程只有一组正整数解,为 $\begin{cases} x = 2 \\ y = 1 \end{cases}$。

故原方程有 11 组正整数解(如下表):

\begin{tabular}{|c|c|c|c|c|c|c|c|c|c|c|c|}
\hline
x & 2 & 4 & 6 & 8 & 10 & 2 & 4 & 6 & 2 & 4 & 2 \\ \hline
y & 13 & 10 & 7 & 4 & 1 & 9 & 6 & 3 & 5 & 2 & 1 \\ \hline
z & 1 & 1 & 1 & 1 & 1 & 2 & 2 & 2 & 3 & 3 & 4 \\ \hline
\end{tabular}
\end{solution}

\question 例 6、求 $x^2 + xy - 6 = 0$ 的正整数解。

\begin{solution}
解:原方程等价于 $x(x + y) = 6$,故有
\[ \begin{cases} x = 2 \\ x + y = 3 \end{cases} \begin{cases} x = 3 \\ x + y = 2 \end{cases} \begin{cases} x = 1 \\ x + y = 6 \end{cases} \begin{cases} x = 6 \\ x + y = 1 \end{cases} \]
即有 $x = 2, y = 1$ 或 $x = 1, y = 5$。
\end{solution}

\question 例 7、求 $x^2 + y^2 = 328$ 的正整数解。

\begin{solution}
解:$\because 328$ 为偶数,$\therefore x, y$ 奇偶性相同,即 $x \pm y$ 为偶数。
设 $x + y = 2u, x - y = 2v$,代入原方程即为 $u^2 + v^2 = 164$。
同理令 $u + v = 2u_1, u - v = 2v_1$ 有 $u_1^2 + v_1^2 = 82, u_1 + v_1 = 2u_2, u_1 - v_1 = 2v_2$。
$u_2^2 + v_2^2 = 41$,$u_2, v_2$ 为一偶一奇,且 $0 < u_2 < 6$。
$u_2 = 1, 2, 3, 4, 5$ 代入方程,有解 $(4, 5), (5, 4)$。
$\therefore$ 原方程解 $x = 18, y = 2$ 或 $x = 2, y = 18$。
\end{solution}

\question 例 8、证明:$x^2 + y^2 + z^2 = 8a + 7$ 无整数解。

\begin{solution}
证:若原方程有解,则有 $x^2 + y^2 + z^2 \equiv 8a + 7 \pmod 8$。
注意到对于模 8,有 $0^2 = 0, 1^2 = 1, 2^2 = 4, 3^2 = 1, 4^2 = 0, 5^2 = 1, 6^2 = 4, 7^2 = 1$。
因而每一个整数对于模 8,必同余于 0, 1, 4 这三个数。
不管 $x^2, y^2, z^2$ 如何变化,只能有 $x^2 + y^2 + z^2 \equiv 0, 1, 2, 3, 4, 5, 6 \pmod 8$。
而 $8a + 7 \equiv 7 \pmod 8$,故 $8a + 7$ 不同余于 $x^2 + y^2 + z^2$ 关于模 8,所以假设错误。
从而证明了原方程无解。
\end{solution}

\question 例 9、正整数 $x, y$ 满足 $3x^2 - 8y^2 + 3x^2y^2 = 2008$,求 $xy$。

\begin{solution}
解:$3x^2 - 8y^2 + 3x^2y^2 = 2000 + 8$
$\Rightarrow 3x^2y^2 + 3x^2 - 8y^2 - 8 = 2000 \Rightarrow (3x^2 - 8)(y^2 + 1) = 2000$
$2000 = 1 \times 2000, 2 \times 1000, 4 \times 500, 5 \times 400, 8 \times 250, 10 \times 200, 20 \times 100, 40 \times 50, 80 \times 25$
$y^2 + 1 = 1, 4, 8, 20, 40, 80$,它只能等于 2, 10, 5, 50。
得出 $y = 1, 3, 2, 7$,但是对 1, 3, 2 来说,另外一个式子 $3x^2 - 8$ 无整数解。
因此得出唯一的答案 $y = 7, x = 4$,对 $2000 = 40 \times 50$ 的分解成立,$\therefore xy = 28$。
\end{solution}

\question 有一根长 5.8 米的木料,现在要把它分割成每根长 0.9 米和 0.4 米的两种规格,求恰好没有剩余的所有有分割法。
\begin{solution}
设分割成 0.9 米的木料 $x$ 根,0.4 米的木料 $y$ 根,其中 $x, y$ 为非负整数。
根据题意得:
\[0.9x + 0.4y = 5.8\]
等式两边同乘以 10 得:
\[9x + 4y = 58\]
由于 $4y$ 和 58 都是偶数,则 $9x$ 必须是偶数,所以 $x$ 是偶数。
当 $x = 0$ 时,$4y = 58$,$y = 14.5$(不合题意)。
当 $x = 2$ 时,$18 + 4y = 58$,$4y = 40$,$y = 10$。
当 $x = 4$ 时,$36 + 4y = 58$,$4y = 22$,$y = 5.5$(不合题意)。
当 $x = 6$ 时,$54 + 4y = 58$,$4y = 4$,$y = 1$。
当 $x \ge 8$ 时,$9x \ge 72 > 58$(无自然数解)。
所以分割法有两种:
1. 0.9 米的 2 根,0.4 米的 10 根;
2. 0.9 米的 6 根,0.4 米的 1 根。
\end{solution}

\question 一次数学竞赛准备了 22 支铅笔作为奖品发给一、二、三等奖的学生,原计划发给一等奖每人 6 支,二等奖每人 3 支,三等奖每人 2 支,后来改为一等奖每人 9 支,二等奖每人 4 支,三等奖每人 1 支,问获一、二、三等奖的学生各几人?
\begin{solution}
设获一、二、三等奖的学生人数分别为 $x, y, z$,其中 $x, y, z$ 为正整数。
根据题意列方程组:
\[
\begin{cases}
6x + 3y + 2z = 22 \\
9x + 4y + z = 22
\end{cases}
\]
由第二个方程得 $z = 22 - 9x - 4y$,代入第一个方程:
\[6x + 3y + 2(22 - 9x - 4y) = 22\]
\[6x + 3y + 44 - 18x - 8y = 22\]
\[-12x - 5y = -22\]
\[12x + 5y = 22\]
因为 $x, y$ 是正整数:
若 $x = 1$,则 $12 + 5y = 22$,$5y = 10$,$y = 2$。
将 $x = 1, y = 2$ 代入 $z = 22 - 9(1) - 4(2) = 22 - 9 - 8 = 5$。
若 $x \ge 2$,则 $12x \ge 24 > 22$(无正整数解)。
所以一、二、三等奖的学生人数分别为 1 人,2 人,5 人。
\end{solution}

\question 求方程 $6x + 22y = 90$ 的非负整数解。
\begin{solution}
方程两边约去 2 得:
\[3x + 11y = 45\]
由此得 $3x = 45 - 11y$。
因为 $x \ge 0$,所以 $45 - 11y \ge 0$,$y \le 4.09$。
又因为 $3x$ 是 3 的倍数,且 45 是 3 的倍数,所以 $11y$ 必须是 3 的倍数,即 $y$ 是 3 的倍数。
在 $0 \le y \le 4$ 范围内的 3 的倍数有 $y = 0$ 和 $y = 3$。
当 $y = 0$ 时,$3x = 45$,$x = 15$。
当 $y = 3$ 时,$3x = 45 - 33 = 12$,$x = 4$。
所以非负整数解为:
\[
\begin{cases}
x = 15 \\
y = 0
\end{cases}, 
\begin{cases}
x = 4 \\
y = 3
\end{cases}
\]
\end{solution}

\question 求不定方程 $2(x+y) = xy + 7$ 的整数解。
\begin{solution}
方程整理为:
\[xy - 2x - 2y + 7 = 0\]
\[x(y - 2) - 2(y - 2) - 4 + 7 = 0\]
\[(x - 2)(y - 2) = -3\]
由于 $x, y$ 是整数,则 $x-2$ 和 $y-2$ 是 -3 的约数。
1. $x - 2 = 1, y - 2 = -3 \implies x = 3, y = -1$
2. $x - 2 = -1, y - 2 = 3 \implies x = 1, y = 5$
3. $x - 2 = 3, y - 2 = -1 \implies x = 5, y = 1$
4. $x - 2 = -3, y - 2 = 1 \implies x = -1, y = 3$
所以整数解为 $(3, -1), (1, 5), (5, 1), (-1, 3)$。
\end{solution}

\question 求满足方程 $\frac{1}{x} - \frac{1}{y} = \frac{1}{12}$ 且使 $y$ 是最大的正整数解 $x, y$。
\begin{solution}
方程变形为:
\[\frac{1}{x} = \frac{1}{12} + \frac{1}{y} = \frac{y + 12}{12y}\]
\[x = \frac{12y}{y + 12} = \frac{12(y + 12) - 144}{y + 12} = 12 - \frac{144}{y + 12}\]
要使 $x$ 为正整数,则 $y + 12$ 必须是 144 的约数,且 $12 - \frac{144}{y + 12} > 0$。
即 $\frac{144}{y + 12} < 12$,所以 $y + 12 > 12$,得 $y > 0$。
为了使 $y$ 最大,我们需要 $y + 12$ 是 144 的最大约数。
但题目要求 $x$ 也是正整数。若 $y + 12 = 144$,则 $y = 132$。
此时 $x = 12 - \frac{144}{144} = 12 - 1 = 11$。
所以使 $y$ 最大的正整数解为 $x = 11, y = 132$。
\end{solution}

\question 满足 $0 < x < y$ 及 $\sqrt{1984} = \sqrt{x} + \sqrt{y}$ 的不同的整数解 $(x, y)$ 个数是多少?
\begin{solution}
首先化简 $\sqrt{1984}$:
\[\sqrt{1984} = \sqrt{64 \times 31} = 8\sqrt{31}\]
方程变为 $\sqrt{x} + \sqrt{y} = 8\sqrt{31}$。
由此可知 $x$ 和 $y$ 必须具有 $k^2 \cdot 31$ 的形式。
设 $\sqrt{x} = a\sqrt{31}, \sqrt{y} = b\sqrt{31}$,其中 $a, b$ 为非负整数。
则 $a + b = 8$。
又因为 $0 < x < y$,所以 $0 < a < b$。
可能的 $(a, b)$ 组合有:
1. $a = 1, b = 7 \implies x = 31, y = 49 \times 31 = 1519$
2. $a = 2, b = 6 \implies x = 4 \times 31 = 124, y = 36 \times 31 = 1116$
3. $a = 3, b = 5 \implies x = 9 \times 31 = 279, y = 25 \times 31 = 775$
共有 3 组不同的整数解。
\end{solution}

\question 求出任何一组满足方程 $x^2 - 51y^2 = 1$ 的自然数解 $x, y$。
\begin{solution}
这是一个佩尔方程 (Pell Equation)。观察 $\sqrt{51}$ 的渐近值。
因为 $7^2 = 49$,尝试 $y$ 的小值。
若 $y = 1, x^2 = 52$(不是平方数)。
若 $y = 2, x^2 = 1 + 51(4) = 205$(不是平方数)。
尝试通过连分数展开或观察法。
发现 $50^2 = 2500$ 且 $51 \times 7^2 = 51 \times 49 = (50+1)(50-1) = 2500 - 1 = 2499$。
所以 $50^2 - 51 \times 7^2 = 2500 - 2499 = 1$。
一组自然数解为 $x = 50, y = 7$。
\end{solution}

\question 求满足条件 $\frac{x+y}{x^2-xy+y^2} = \frac{3}{7}$ 的整数 $x, y$ 的所有可能的值。
\begin{solution}
方程交叉相乘得:
\[7(x + y) = 3(x^2 - xy + y^2)\]
\[3x^2 - (3y + 7)x + (3y^2 - 7y) = 0\]
将其看作关于 $x$ 的一元二次方程,其判别式 $\Delta$ 必须是非负平方数:
\[\Delta = (3y + 7)^2 - 4(3)(3y^2 - 7y) = 9y^2 + 42y + 49 - 36y^2 + 84y = -27y^2 + 126y + 49\]
令 $-27y^2 + 126y + 49 \ge 0$。
解得 $y$ 大约在 $[-0.35, 5.02]$ 之间。
检查整数 $y \in \{0, 1, 2, 3, 4, 5\}$:
1. $y = 0 \implies \Delta = 49 = 7^2$。$x = \frac{7 \pm 7}{6} \implies x = 0, \frac{7}{3}$(舍去),解 $(0, 0)$ 代入原方程分母为 0 舍去。
2. $y = 1 \implies \Delta = -27 + 126 + 49 = 148$(不是平方数)。
3. $y = 2 \implies \Delta = -108 + 252 + 49 = 193$(不是平方数)。
4. $y = 3 \implies \Delta = -243 + 378 + 49 = 184$(不是平方数)。
5. $y = 4 \implies \Delta = -432 + 504 + 49 = 121 = 11^2$。$x = \frac{19 \pm 11}{6} \implies x = 5, \frac{4}{3}$(舍去)。
6. $y = 5 \implies \Delta = -675 + 630 + 49 = 4$。$x = \frac{22 \pm 2}{6} \implies x = 4, \frac{10}{3}$(舍去)。
由对称性,当 $x = 4, y = 5$ 或 $x = 5, y = 4$ 时均成立。
故所有可能的整数对为 $(4, 5)$ 和 $(5, 4)$。
\end{solution}

\question 求方程 $x^2 - y^2 = 105$ 的正整数解。
\begin{solution}
原方程可化为:
\[(x - y)(x + y) = 105\]
因为 $x, y$ 是正整数,所以 $x + y > x - y > 0$,且 $x + y$ 与 $x - y$ 的奇偶性相同。
由于 105 是奇数,故 $x + y$ 与 $x - y$ 均为奇数。
将 105 分解为两个奇因数之积:
1. $x - y = 1, x + y = 105 \implies x = 53, y = 52$
2. $x - y = 3, x + y = 35 \implies x = 19, y = 16$
3. $x - y = 5, x + y = 21 \implies x = 13, y = 8$
4. $x - y = 7, x + y = 15 \implies x = 11, y = 4$
所以正整数解为 $(53, 52), (19, 16), (13, 8), (11, 4)$。
\end{solution}

\question 求证方程 $x^3 + 11^3 = y^3$ 没有正整数解。
\begin{solution}
根据费马大定理,当 $n > 2$ 时,方程 $x^n + y^n = z^n$ 没有正整数解。
在本题中,$n = 3$。
方程 $x^3 + 11^3 = y^3$ 等价于 $y^3 - x^3 = 11^3$。
根据费马大定理可知,不存在正整数 $x, 11, y$ 使得该等式成立。
(或者:利用因式分解 $(y-x)(y^2+yx+x^2) = 1331$ 逐一讨论 $y-x$ 的因数,均可证明无正整数解。)
\end{solution}

\question 求方程 $x + y = x^2 - xy + y^2$ 的全部整数解。
\begin{solution}
整理方程得:
\[x^2 - (y + 1)x + (y^2 - y) = 0\]
将其看作关于 $x$ 的一元二次方程,其判别式 $\Delta$ 必须是非负平方数:
\[\Delta = (y + 1)^2 - 4(y^2 - y) = y^2 + 2y + 1 - 4y^2 + 4y = -3y^2 + 6y + 1\]
由 $-3y^2 + 6y + 1 \ge 0$ 得:
\[3y^2 - 6y - 1 \le 0\]
解得 $1 - \frac{2}{\sqrt{3}} \le y \le 1 + \frac{2}{\sqrt{3}}$,即大约 $-0.15 \le y \le 2.15$。
整数 $y$ 可取 0, 1, 2。
1. $y = 0 \implies \Delta = 1$,$x = \frac{1 \pm 1}{2} \implies x = 1, 0$。
2. $y = 1 \implies \Delta = 4$,$x = \frac{2 \pm 2}{2} \implies x = 2, 0$。
3. $y = 2 \implies \Delta = 1$,$x = \frac{3 \pm 1}{2} \implies x = 2, 1$。
整数解为 $(1, 0), (0, 0), (2, 1), (0, 1), (2, 2), (1, 2)$。
\end{solution}

\question 求方程 $x^2 + y^2 = 2x + 2y + xy$ 的所有正整数解。
\begin{solution}
整理方程:
\[x^2 - (y + 2)x + (y^2 - 2y) = 0\]
判别式 $\Delta = (y + 2)^2 - 4(y^2 - 2y) = y^2 + 4y + 4 - 4y^2 + 8y = -3y^2 + 12y + 4$。
由 $-3y^2 + 12y + 4 \ge 0$ 得:
\[3y^2 - 12y - 4 \le 0\]
解得 $2 - \frac{4}{\sqrt{3}} \le y \le 2 + \frac{4}{\sqrt{3}}$,即大约 $-0.3 \le y \le 4.3$。
因要求正整数解,故 $y$ 可取 1, 2, 3, 4。
1. $y = 1 \implies \Delta = 13$(非平方数)。
2. $y = 2 \implies \Delta = 16 = 4^2$,$x = \frac{4 \pm 4}{2} \implies x = 4, 0$(0 舍去)。
3. $y = 3 \implies \Delta = 13$(非平方数)。
4. $y = 4 \implies \Delta = 4 = 2^2$,$x = \frac{6 \pm 2}{2} \implies x = 4, 2$。
正整数解为 $(4, 2), (2, 4), (4, 4)$。
\end{solution}

\question 找出全部符合 $a^2 + b = b^{2001}$ 的整数 $(a, b)$。
\begin{solution}
原方程化为 $a^2 = b^{2001} - b = b(b^{2000} - 1)$。
1. 若 $b = 0$,则 $a^2 = 0 \implies a = 0$。解为 $(0, 0)$。
2. 若 $b = 1$,则 $a^2 = 0 \implies a = 0$。解为 $(0, 1)$。
3. 若 $b = -1$,则 $a^2 = (-1)^{2001} - (-1) = -1 + 1 = 0 \implies a = 0$。解为 $(0, -1)$。
4. 若 $b > 1$ 或 $b < -1$,则 $b^{2001} - b$ 不可能是一个完全平方数(除了 $b=0, \pm 1$ 之外,$b^{2001}-b$ 介于两个连续平方数之间,此处省略具体不等式证明)。
全部整数解为 $(0, 0), (0, 1), (0, -1)$。
\end{solution}

\question 求方程 $y^2 + 3x^2y^2 = 30x^2 + 517$ 的所有正整数解。
\begin{solution}
整理方程得:
\[y^2(3x^2 + 1) = 30x^2 + 517\]
\[y^2 = \frac{30x^2 + 517}{3x^2 + 1} = \frac{10(3x^2 + 1) + 507}{3x^2 + 1} = 10 + \frac{507}{3x^2 + 1}\]
因为 $y$ 是正整数,所以 $3x^2 + 1$ 必须是 507 的约数。
507 的约数有 1, 3, 13, 39, 169, 507。
由于 $x$ 是正整数,$3x^2 + 1 \ge 4$。
1. $3x^2 + 1 = 13 \implies 3x^2 = 12 \implies x^2 = 4 \implies x = 2$。
   此时 $y^2 = 10 + \frac{507}{13} = 10 + 39 = 49 \implies y = 7$。
2. $3x^2 + 1 = 39 \implies 3x^2 = 38$(无整数解)。
3. $3x^2 + 1 = 169 \implies 3x^2 = 168 \implies x^2 = 56$(无整数解)。
4. $3x^2 + 1 = 507 \implies 3x^2 = 506$(无整数解)。
故正整数解为 $(2, 7)$。
\end{solution}

\question 求方程 $x^6 + 3x^3 + 1 = y^4$ 的整数解。
\begin{solution}
当 $x = 0$ 时,$1 = y^4 \implies y = \pm 1$。
当 $x > 0$ 时:
$(x^3 + 1)^2 = x^6 + 2x^3 + 1 < x^6 + 3x^3 + 1$
$(x^3 + 2)^2 = x^6 + 4x^3 + 4 > x^6 + 3x^3 + 1$
可见 $y^4$ 处于两个连续整数的平方之间,故无解。
当 $x = -1$ 时,$1 - 3 + 1 = -1 = y^4$(无实数解)。
当 $x \le -2$ 时,同理可证无解。
整数解为 $(0, 1), (0, -1)$。
\end{solution}

\question 求方程 $3^x - 5^y = z^2$ 的正整数解。
\begin{solution}
考虑模 3 情况:$-(-1)^y \equiv z^2 \pmod 3$。
若 $y$ 为偶数,则 $-1 \equiv z^2 \pmod 3$,即 $2 \equiv z^2 \pmod 3$,不可能。
故 $y$ 必须为奇数。
考虑模 4 情况:$(-1)^x - 1^y \equiv z^2 \pmod 4 \implies (-1)^x - 1 \equiv z^2 \pmod 4$。
若 $x$ 为奇数,则 $-2 \equiv z^2 \pmod 4 \implies 2 \equiv z^2 \pmod 4$,不可能。
故 $x$ 必须为偶数,设 $x = 2k$。
$3^{2k} - z^2 = 5^y \implies (3^k - z)(3^k + z) = 5^y$。
设 $3^k - z = 5^m, 3^k + z = 5^n$,其中 $m + n = y, n > m$。
两式相减:$2z = 5^n - 5^m = 5^m(5^{n-m} - 1)$。
由于 $2z$ 不被 5 整除,故 $m = 0$。
则 $3^k - z = 1$ 且 $3^k + z = 5^y$。
相加得 $2 \cdot 3^k = 5^y + 1$。
当 $k = 1$ 时,$2 \cdot 3 = 6 = 5^1 + 1 \implies y = 1$。
此时 $x = 2k = 2$,$3^2 - 5^1 = 4 = 2^2 \implies z = 2$。
当 $k > 1$ 时,通过模 9 分析可知无其他解。
正整数解为 $(2, 1, 2)$。
\end{solution}
\end{questions}