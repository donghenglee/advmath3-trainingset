\pagecolor{PageColor}
\
\vfil
\hfil  {\fontsize{50pt}{36pt}\selectfont{几何}} \hfil
\vfil
\begin{tikzpicture}[remember picture,overlay,every node/.style={inner sep=0pt}]
        \node [shift={(1cm,-1cm)},brown,scale=2,anchor=north west] (CNW)
        at (current page.north west) {\pgfornament[height=1cm,width=1cm]{61}};
        \node [shift={(-1cm,-1cm)},brown,scale=2,anchor=north east] (CNE)
        at (current page.north east) {\pgfornament[height=1cm,width=1cm,symmetry=v]{61}};
        \node [shift={(1cm,1cm)},brown,scale=2,anchor=south west] (CSW)
        at (current page.south west) {\pgfornament[height=1cm,width=1cm,symmetry=h]{61}};
        \node [shift={(-1cm,1cm)},brown,scale=2,anchor=south east] (CSE)
        at (current page.south east) {\pgfornament[height=1cm,width=1cm,symmetry=c]{61}};
        \pgfornamentline[color=brown]{current page.north west}{current page.north east}{2}{87}
        \pgfornamentline{current page.south west}{current page.south east}{2}{87}
        \pgfornamentline{current page.north west}{current page.south west}{3}{87}
        \pgfornamentline{current page.north east}{current page.south east}{3}{87}
        \end{tikzpicture}%
\thispagestyle{empty}
\pagebreak

\begin{center}
  {\fontsize{30pt}{26pt}\selectfont
    \hypertarget{解三角形}{解三角形} \label{解三角形}
  }
\end{center}
\separator
\vspace{1pt}
\nopagecolor
\begin{questions}
    \question 已知钝角三角形的边长为 \(10,\,17,\,x\),求 \(x\) 的所有正整数解之和。  
    \begin{solution}
        钝角三角形需满足某边的平方大于其余两边平方和:
        
        \textbf{情况一:} 若 $x < 17$,最大边是 $17$,成立条件为:
        \[
        \begin{cases}
        17^2 > 10^2 + x^2 \Rightarrow x <\sqrt{189}\approx 13.75\\
        x + 10 > 17 \Rightarrow x > 7
        \end{cases}
        \]
        得 $x \in \{8,9,10,11,12,13\}$
        
        \textbf{情况二:} 若 $x > 17$,最大边是 $x$,成立条件为:
        \[
        \begin{cases}
        x^2 > 10^2 + 17^2 = 389 \Rightarrow x > \sqrt{389} \approx 19.7\\
        10 + 17 > x \Rightarrow x < 27
        \end{cases}
        \]
        得 $x \in \{20,21,22,23,24,25,26\}$
        
        故 $$\sum x=224$$
    \end{solution}
    
    \question 若$\triangle ABC$ 的三个高分别为 $\dfrac{1}{2},\ \dfrac{1}{3},\ \dfrac{1}{4}$,求$\triangle ABC$ 的周长。
    \begin{solution}
        设三角形三边长为 $a,\ b,\ c$,对应的高分别为 $\dfrac{1}{2},\ \dfrac{1}{3},\ \dfrac{1}{4}$,则$\triangle ABC$
        面积为\[
        \frac{1}{2}ah_a = \frac{1}{2}bh_b = \frac{1}{2}ch_c 
        \Rightarrow \frac{a}{2} = \frac{b}{3} = \frac{c}{4}
        \]
        现设$a = 2t, b = 3t, c = 4t$,则半周长
        \[
        s = \frac{a + b + c}{2} = \frac{9t}{2}
        \]
        由海伦公式,
        \[
        [\triangle ABC] 
        = \sqrt{\frac{9t}{2} \cdot \frac{5t}{2} \cdot \frac{3t}{2} \cdot \frac{t}{2}}
        =\frac{1}{2} \cdot 2t \cdot \frac{1}{2}
        \]
        解得$t=\dfrac{2\sqrt{15}}{45}$,因此周长为
        \[
        2s = 9t = \frac{2\sqrt{15}}{5}
        \]
    \end{solution}

    \question 设 \(\triangle ABC\) 的内角 \(A,B,C\) 对应边长为 \(a,b,c\)。若 \(b+c=2a,\; 3\sin A = 5\sin B\),求 \( C\)。  
    \begin{solution}
        由正弦定理,
        \[
        3\sin A = 5\sin B \Rightarrow 3a = 5b \Rightarrow b=\frac{3}{5}a
        \]
        又\[
        b+c=2a \Rightarrow c=\frac{8}{5}a
        \]
        由余弦定理,
        \[
        \left(\frac{8}{5}a\right)^2=a^2+\left(\frac{3}{5}a\right)^2-2a\left(\frac{3}{5}a\right)\cos C
        \]
        给出 
        \[
        C=120^\circ
        \]
    \end{solution}
    
    \question 若 \(\triangle ABC\) 的面积为 \(\dfrac{3\sqrt{15}}{4}\),且 \(\cos C = -\dfrac14\),且 \(\sin^2A + \sin^2B = \dfrac{13}{16}\sin^2C\),求边长 \(AB\)。  
    \begin{solution}
        已知面积
        \[
        \frac{1}{2} ab \sin C = \frac{3\sqrt{15}}{4}
        \]
        又由 \[
        \cos C = -\frac{1}{4} \Rightarrow \sin C = \frac{\sqrt{15}}{4}
        \]
        代入上式得 $ab = 6$, 且由正弦定理
        \[
        \sin^2 A + \sin^2 B = \frac{13}{16} \sin^2 C  \Rightarrow a^2 + b^2 = \frac{13}{16} c^2 
        \]
        又由余弦定理:
        \[
        c^2=a^2 + b^2-2ab\cos C =\frac{13}{16} c^2 -2 \cdot 6 \cdot \left(-\dfrac14\right) 
        \]
        得 $c=4$
    \end{solution}
    
    \question 已知 \(\triangle ABC\) 的周长为 \(14\,\text{cm}\),面积为 \(3\sqrt7\,\text{cm}^2\),且 \(\cos B = -\dfrac18\)。求 \(a,b,c\) 的边长。  
    \begin{solution}
        解法同上,已知周长\(a + b + c = 14\),面积\[
         \frac{1}{2} ac \sin B = 3\sqrt7
        \] 
        由\[
        \cos B =  -\dfrac18 \Rightarrow \sin B = \frac{3\sqrt{7}}{8}
        \]
        有 $ac=16$, 且由余弦定理, \[
        b^2=a^2+c^2-2ac \cos B =(a+c)^2-2ac-2ac \cos B =(14-b)^2-2\cdot 16-2\cdot 16 \cdot  \left(-\dfrac18\right)
        \]
        解得 $b=6$,即有$a+c=8$,所以$a,c$是关于$x$的方程式
        \[
        x^2-8x+16=0
        \]
        的两根,于是解得$a=4,\;c=4$,故解为$ a=4,\; b=6,\; c=4$
    \end{solution}
    
    \question 在锐角 \(\triangle ABC\) 中,已知 \(b=5,\; \sin A = \dfrac{\sqrt7}{4}\),面积为 \(\dfrac{15\sqrt7}{4}\),求 \(c\) 及 \(\sin C\)。  
    \begin{solution}
        由面积公式:
        \[
        S = \frac{1}{2}bc\sin A = \frac{1}{2} \cdot 5 \cdot c \cdot \frac{\sqrt{7}}{4} = \frac{15\sqrt{7}}{4}
        \]
        得 $c = 6$,由余弦定理,\[
            a^2=b^2+c^2-2bc \cos A
        \]
        其中$\sin A = \dfrac{\sqrt7}{4} \Rightarrow \cos A = \dfrac{3}{4}$(注意是锐角),上式变成\[
        a^2=5^2+6^2-2\cdot 5 \cdot 6 \cdot \dfrac{3}{4} \Rightarrow a=4
        \]再由正弦定理,\[
        \frac{a}{\sin A}= \frac{c}{\sin C} \Rightarrow \frac{4}{\frac{\sqrt7}{4}}= \frac{6}{\sin C} \Rightarrow \sin C=\dfrac{3\sqrt7}{8}
        \]
    \end{solution}

    \question 已知 $\triangle ABC$ 中,$\;C = 60^\circ$,求
    \[
    \frac{a}{b+c} + \frac{b}{a+c}
    \]
    的值。
    \begin{solution}
        由余弦定理,
        \[
        a^2 + b^2 = c^2+ 2ab \cos C = c^2+2ab \cdot \frac{1}{2} = c^2+ ab
        \]
        于是
        \[
        \frac{a}{b+c} + \frac{b}{a+c}
        = \frac{a^2 + ac + b^2 + bc}{ab+bc+ac+c^2} 
        = \frac{a^2 + b^2 + ac + bc}{bc+ac+a^2+b^2} 
        = 1
        \]
    \end{solution}

    \question 已知 $\triangle ABC$ 中,
    \[
    2a \cos B = c, \quad \sin A \sin B (2-\cos C) = \sin^2 \frac{C}{2} + \frac{1}{2},
    \]
    证明$\triangle ABC$ 为等腰直角三角形。
    \begin{solution}
        据题意,
        \[
        2a \cos B = c \Rightarrow 2 \sin A \cos B = \sin C = \sin(A+B) = \sin A \cos B + \cos A \sin B
        \]
        得
        \[
        \sin(A-B)= 0 \Rightarrow A=B \in (0,\pi)
        \]
        于是
        \[
        \sin A \sin B (2 - \cos C) = \sin^2 \frac{C}{2} + \frac{1}{2}
        \]
        化为
        \begin{align*}
            \sin^2 A (2 + \cos 2A) &= \cos^2 A + \frac{1}{2}, \\
            \sin^2 A (3 - 2 \sin^2 A) &= 1 - \sin^2 A + \frac{1}{2}, \\
            (2 \sin^2 A - 1)(2 \sin^2 A - 3) &= 0,
        \end{align*}
        舍去$\sin^2 A=\dfrac{3}{2}$,解得
        \[
        \sin A = \frac{\sqrt{2}}{2} >0\Rightarrow 
        A = B = \frac{\pi}{4}, C = \frac{\pi}{2}
        \]
        因此,$\triangle ABC$ 为等腰直角三角形。
    \end{solution}

    \question 单位圆上有三点 $A,B,C$,且
    \[
    AB^2+BC^2+CA^2=8
    \]
    证明:$\triangle ABC$ 为直角三角形。
    \begin{solution}
        三点$A,B,C$在单位圆上,外接圆半径为 $1$,由正弦定理,
        \[
        AB=2\sin C,\quad BC=2\sin A,\quad CA=2\sin B
        \]
        代入已知等式得
        \begin{align*}
        \sin^2 A + \sin^2 B + \sin^2 C = 2 &\Rightarrow \frac{1-\cos 2A}{2} + \frac{1-\cos 2B}{2} + \frac{1-\cos 2C}{2} = 2 \\ 
        &\Rightarrow \cos 2A + \cos 2B + \cos 2C = -1
        \end{align*}
        和差化积得
        \[
        -2\cos C\cos(A-B)+2\cos^2C-1=-1 \Rightarrow 2\cos C(\cos C-\cos(A-B))=0
        \]
        若 $\cos C=0$,则 $C=\dfrac{\pi}{2}$。若 $\cos C=\cos(A-B)$,则 $C=A-B$ 或 $C=B-A$,从而 $A=\dfrac{\pi}{2}$ 或 $B=\dfrac{\pi}{2}$。于是得证$\triangle ABC$ 为直角三角形。
    \end{solution}

    \question 设某三角形三边长成等差数列,公差为 $d$,若 $r,R$ 分别表示此三角形的内切圆半径与外接圆半径,试证公差
    \[
    d = \sqrt{2Rr - 4r^2}
    \]
    \begin{solution}
        设三角形三边为 $a<b<c$,三边长成等差,故设
        \[
        a = b-d, c = b+d \quad \Rightarrow s = \frac{a+b+c}{2}= \frac{3b}{2},
        \]
        其中$d>0$,则三角形面积为
        \[
        \triangle = r s = \sqrt{s(s-a)(s-b)(s-c)} = \frac{abc}{4R}
        \]
        即
        \[
        \frac{3br}{2} = \sqrt{\frac{3b}{2} \cdot \frac{b}{2}  \cdot \left(\frac{b^2}{4} - d^2\right)}= \frac{b(b^2 - d^2)}{4R} \Rightarrow 
        \begin{cases}
            b^2 = 4(d^2 + 3 r^2)\\ b^2 - d^2 = 6 R r
        \end{cases}
        \]
        因此,
        \[
        d^2 = 2Rr - 4 r^2 \Rightarrow d = \sqrt{2Rr - 4 r^2}
        \]
    \end{solution}

    \question 已知三角形 $ABC$ 及其 $BC$ 边上的任意一点 $P$ (其中 $P$ 异于 $B,C$ 两点),试证斯德瓦特定理: 
    \[
    (AP^2 + BP\cdot CP)BC= AB^2\cdot CP + AC^2\cdot BP.
    \]
    \begin{solution}
        在$\triangle APB$及$\triangle APC$中,由余弦定理,
        \[
        \begin{cases}
        \cos \angle APB = \dfrac{AP^2 + BP^2 - AB^2}{2AP\cdot BP}, \\[2mm]
        \cos \angle APC = \dfrac{AP^2 + CP^2 - AC^2}{2AP\cdot CP}.
        \end{cases}
        \]
        由于 $\angle APB + \angle APC = \pi \Rightarrow \cos \angle APB = -\cos \angle APC$,于是  
        \[
        CP(AP^2 + BP^2 - AB^2) = -BP(AP^2 + CP^2 - AC^2)
        \]
        整理得  
        \[
        AP^2(CP+BP) + CP\cdot BP(CP+BP) = AB^2\cdot CP + AC^2\cdot BP
        \]
        又 $CP+BP=BC$,所以  
        \[
        (AP^2 + BP\cdot CP)BC = AB^2\cdot CP + AC^2\cdot BP
        \]
    \end{solution}

    \question 如图,\;\(\triangle ABC\) 中 \(AB=10,\,AC=14,\,B=\dfrac{\pi}{3}\)。\(D\) 在 \(BC\) 上且 \(DC=6\)。  
    \begin{figure}[H]
        \centering
        \includegraphics[width=0.3\textwidth]{images/image1.png}
    \end{figure}
    \begin{parts}
        \part 求 \(\angle ADB\)。 
        \begin{solution}
            $\triangle ABC$中,由余弦定理,\[
            14^2=10^2+BC^2-2\cdot10\cdot BC \cdot \cos \frac{\pi}{3} \Rightarrow BC=16
            \]又$BD=16-6 = 10$,则$\triangle ABD$ 为等边三角形,故\(\angle ADB=60^\circ\)
        \end{solution}
        \part 求 \(\sin\angle DAC\)。
        \begin{solution}
            $\triangle ADC$中,由余弦定理,\[
            6^2=10^2+14^2-2\cdot10\cdot14\cdot \cos \angle DAC \Rightarrow \cos \angle DAC = \frac{13}{14}
            \]给出\[
                \sin \angle DAC = \dfrac{3\sqrt3}{14} 
            \]
        \end{solution}
    \end{parts}
    
    \question 在 \(\triangle ABC\) 中,点 \(D\) 在 \(BC\) 上且 \(AD\) 为 \(\angle BAC\) 的角平分线。  
        已知 \(AB=12,\;AD=9,\;AC=15\),令$\theta=\angle BAD=\angle DAC$, 求 $\cos \theta$。 
    \begin{figure}[H]
        \centering
        \includegraphics[width=0.3\textwidth]{images/image2.png}
    \end{figure}
    \begin{solution}
        考虑面积法,由正弦定理, 
        \[
        \frac{1}{2}\cdot12\cdot9\cdot \sin \theta + \frac{1}{2}\cdot9\cdot15\cdot \sin \theta=\frac{1}{2}\cdot12\cdot15\cdot \sin 2\theta
        \]
        化简得
        \[
        243\sin\theta = 360\sin\theta\cos\theta \Rightarrow \cos\theta = \dfrac{27}{40}
        \]
    \end{solution}

    \question 在$\triangle ABC$ 中,过 $A$ 与过 $B$ 的中线互相垂直。已知 $AC=6,BC=7$,求 $AB$。
    \ifprintanswers
    \begin{figure}[H]
        \centering        
        \includegraphics[width=0.4\textwidth]{images/image146.png}
    \end{figure}
    \fi
    \begin{solution}
        设$D,E$在$BC,AC$上使得中线$AD,BE$交于重心$G$,在$\triangle BDG$ 及$\triangle AEG$中,由毕氏定理,
        \[
        x^{2}+4y^{2}=\frac{49}{4},
        \quad
        4x^{2}+y^{2}=9.
        \]
        因此
        \[
        AB^{2}=4x^{2}+4y^{2}=\frac{4}{5}\Bigl(\frac{49}{4}+9\Bigr)=17 \Rightarrow AB=\sqrt{17}
        \]
    \end{solution}

    \question $\triangle ABC$ 中,$AB = AC = 3$, $AD = 1$, $\angle BCD = 60^\circ$, 求 $\cos A$。  
    \ifprintanswers
    \begin{figure}[H]
        \centering        
        \includegraphics[width=0.3\textwidth]{images/image35.png}
    \end{figure}
    \fi
    \begin{solution}
        设 $\angle ACD = \theta$, 则
        \[
        \angle ADC = 120^\circ + \theta, \angle B = 60^\circ + \theta
        \]
        在 $\triangle ADC$ 中,由正弦定理,
        \[
        \frac{1}{\sin \theta} = \frac{3}{\sin(120^\circ + \theta)} \Rightarrow \frac{\sqrt3}{2}\cos \theta - \frac{1}{2}\sin\theta=3\sin \theta \Rightarrow \tan \theta = \frac{\sqrt{3}}{7}
        \]
        由此可得
        \[
        \sin \theta = \frac{\sqrt{3}}{2\sqrt{13}},\quad \cos \theta = \frac{7}{2\sqrt{13}}
        \]
        在 $\triangle BCD$ 中,
        \[
        \frac{BC}{\sin(60^\circ - \theta)} = \frac{2}{\sin 60^\circ} \Rightarrow BC = \frac{4}{\sqrt{3}} \sin(60^\circ - \theta) = \frac{6}{\sqrt{13}}
        \]
        在 $\triangle ABC$ 中,由余弦定理,
        \[
        \cos A = \frac{9 + 9 - \frac{36}{13}}{2 \cdot 9} = \frac{11}{13}
        \]
    \end{solution}

    \question 在$\triangle ABC$中,在 $BC$ 边上取一点 $D$ 使得 $BD = AC$。若 $\angle DAC=60^\circ,\angle ACD=40^\circ$,求 $\angle BAD$。
    \ifprintanswers
    \begin{figure}[H]
        \centering
        \includegraphics[width=0.5\linewidth]{images/image44.png}
    \end{figure}
    \fi
    \begin{solution}
        设$BD = AC = a,AD = b, \angle BAD = \theta$,则
        \[
        \angle ADC = 80^\circ \Rightarrow \angle B = 80^\circ - \theta
        \]
        在$\triangle ADC$及$\triangle ABD$中,由正弦定理,
        \[
        \frac{a}{\sin 80^\circ} = \frac{b}{\sin 40^\circ}, \quad \frac{a}{\sin \theta} = \frac{b}{\sin (80^\circ - \theta)}
        \]
        消去 \( a \)得
        \[
        \frac{\sin \theta}{\sin(80^\circ - \theta)} = \frac{\sin 80^\circ}{\sin 40^\circ} = 2\cos 40^\circ
        \]
        整理得
        \[
        \sin \theta = 2 \cos 40^\circ \sin(80^\circ - \theta) 
        = \sin(120^\circ - \theta) + \sin(40^\circ - \theta)
        \]
        \[
        \sin \theta + \sin(\theta - 120^\circ) =2 \sin(\theta - 60^\circ) \cos 60^\circ = \sin(40^\circ - \theta)
        \]
        \[
        \sin(\theta - 60^\circ) = \sin(40^\circ - \theta)
        \Rightarrow \angle BAD=\theta = 50^\circ 
        \]
    \end{solution}

    \question 已知 $\triangle ABC$ 中, $AB=AC$, 且 $\angle B$ 的角平分线交 $AC$ 于 $D$, 且 $BC=BD+AD$, 求$\angle A$ 。
    \ifprintanswers
    \begin{figure}[H]
        \centering
        \includegraphics[width=0.35\linewidth]{images/image226.png}
    \end{figure}
    \fi
    \begin{solution}
        设 $\angle B=\angle C=2\theta$,则  
        \[
        \angle ABD=\angle DBC=\theta, \quad
        \angle A=\pi-4\theta,\quad
        \angle BDC=\pi-3\theta
        \]
        在$\triangle BDC$及$\triangle ABD$中,由正弦定理,
        \[
        \frac{BC}{BD}
        =\frac{\sin(\pi-3\theta)}{\sin 2\theta}
        =\frac{\sin 3\theta}{\sin 2\theta}, \quad
        \frac{AD}{BD}
        =\frac{\sin\theta}{\sin(\pi-4\theta)}
        =\frac{\sin\theta}{\sin 4\theta}
        \]
        由 $BC=BD+AD$,有
        \begin{align*}
        \frac{BC}{BD} &=1+\frac{AD}{BD} \\
        \frac{\sin 3\theta}{\sin 2\theta}&=1+\frac{\sin\theta}{\sin 4\theta}=1+\frac{\sin\theta}{2\sin 2\theta \cos 2\theta}\\
        2\cos 2\theta \sin 3\theta&=2\sin 2\theta \cos 2\theta+\sin\theta\\
        \sin 5\theta+\sin\theta&=\sin 4\theta+\sin\theta \\
        \sin 5\theta&=\sin 4\theta
        \end{align*}
        故
        \[
        \theta=\frac{\pi}{9} \Rightarrow \angle A=\frac{5\pi}{9}
        \]
    \end{solution}

    \question 设 $D$ 为 $\triangle ABC$ 的 $BC$ 上之一点,且 $BD=AC=1$,若 $\angle BAD=30^\circ$、$\angle CAD=90^\circ$,求 $CD$ 之长。 
    \ifprintanswers
    \begin{figure}[H]
        \centering        
        \includegraphics[width=0.4\textwidth]{images/image121.jpg}
    \end{figure}
    \fi
    \begin{solution}
        设 $CD=a$,由等高性质,
        \[
        \frac{[ABD]}{[ACD]}=\frac{BD}{CD}=\frac{1}{a} \tag{1}
        \]
        故
        \[
        \frac{[ABD]}{[ACD]}=\frac{\frac{1}{2}\,AD\cdot AB\sin30^\circ}{\frac{1}{2}\,AD\cdot AC}=\frac{AB}{2} \tag{2}
        \]
        由 $(1)=(2)$ 得 $AB=\dfrac{2}{a}$,由余弦定理,
        \[
        (a+1)^2=1+\left(\dfrac{2}{a}\right)^2-2\cdot \dfrac{2}{a}\cdot 1 \cdot \cos120^\circ
        \]
        化简得
        \[
        (a^3-2)(a+2)=0
        \]
        取正根 $a^3=2$,故 $a=\sqrt[3]{2}$。
    \end{solution}

    \question 已知一个半径为 1 的圆中,有两条相等的弦互相垂直,且互相分割成 $1:4$的线段。求弦长。
    \begin{solution}
        构造坐标系,使得圆心在原点上,两弦为 $x=-a$及 $y=-a$,且 $a>0$。由题意,
        \[
        4(\sqrt{1-a^{2}}-a)=\sqrt{1-a^{2}}+a
        \]
        解得
        \[
        a^{2}=\frac{9}{34}
        \]
        因此弦长为
        \[
        2\sqrt{1-a^{2}}=\frac{5\sqrt{34}}{17}
        \]
    \end{solution}

    \question 圆内接四边形 \(ABCD\) 中,\(\;AB=BC=3,\;CD=5,\;DA=8\)。求对角线 \(BD\)。  
    \begin{solution}
        圆内接四边形对角互补,有
        \[
        BD^2=3^2+8^2-2\cdot3\cdot8\cdot \cos A = 3^2+5^2-2\cdot3\cdot5\cdot \cos (\pi- A)
        \]
        解得 
        \[
        \cos A = \frac{1}{2} \Rightarrow BD=7
        \]
    \end{solution}
    
    \question 已知四边形 $ABCD$ 内接于圆,且 $BA=AD=1,\cos\angle BAD=-\dfrac{1}{3}$。求证 $BC$为外接圆的直径
    \ifprintanswers
    \begin{figure}[H]
        \centering        
        \includegraphics[width=0.4\textwidth]{images/image196.png}
    \end{figure}
    \fi
\begin{solution}
在$\triangle BAD$中,由余弦定理,
\[
BD^2 = 1^2 + 1^2 - 2\cdot 1 \cdot 1 \cdot \left(-\frac{1}{3}\right) \Rightarrow BD= \sqrt{\frac{8}{3}}
\]
设 \textcolor{red}{$x=\cos\angle ABC$,则 $DC=x$}。由于 $ABCD$ 为圆内接四边形,$\angle ADC=180^\circ-\angle ABC$,故
\[
\cos\angle ADC=-\cos\angle ABC=-x.
\]
同理,$\cos\angle BCD=-\cos\angle BAD=\dfrac{1}{3}$。

在 $\triangle ADC$ 与 $\triangle ABC$ 中应用余弦定理,因 $AC$ 公共,得:
\begin{align*}
1^2+x^2-2(1)(x)\cos\angle ADC &= 1^2+BC^2-2(1)(BC)\cos\angle ABC,\\
1+x^2-2x(-x) &= 1+BC^2-2BCx,\\
0 &= BC^2-2BCx-3x^2,\\
0 &= (BC-3x)(BC+x).
\end{align*}

因 $x>0$,故 $BC=3x$。

又因 $\cos\angle BCD=\tfrac{1}{3}$,且 $DC:BC=1:3$,由$\angle BDC$ 为直角。于是 $BC$ 为圆的直径。
\end{solution}

    \question 已知$\triangle ABC$ 中,$\angle B=90^\circ,\angle A=30^\circ$。点 $P, Q, R$ 分别在 $AB, BC, CA$ 上使得$\triangle PQR$ 为正三角形。若$BC=4$,且 $Q$ 是 $BC$ 的中点,求 $PR$。
    \begin{solution}
        已知$BQ=QC=2$,设 $PB=x,PQ=y,\angle PQB=\angle CRQ=\theta$,在$\triangle PBQ$及$\triangle QRC$中, 由正弦定理,
        \[
        \frac{x}{\sin\theta}=y, \quad \frac{2}{\sin\theta}=\frac{y}{\frac{\sqrt{3}}{2}}
        \]
        解得 $x=\sqrt{3}$,在 $\triangle PBQ$ 中,由毕氏定理
        \[
        y^2 = 3 + 2^2 =7 \Rightarrow PR=y=\sqrt{7}
        \]
    \end{solution}

    \question 已知在$\triangle ABC$中,三边 $AB,BC,CA$ 长度比为 $4:6:5$,三角形内一点 $P$ 满足 $PA=8,PB=4,PC=13$,试求 $\cos\angle APB$ 的值。
    \ifprintanswers
    \begin{figure}[H]
        \centering        
        \includegraphics[width=0.5\textwidth]{images/image25.png}
    \end{figure}
    \fi
    \begin{solution}
        在 ${BC}$ 外取一点 $Q$使得 ${BQ}=6$ 且 $\angle ABC=\angle PBQ$,由相似三角形:
        \[
        \frac{{AB}}{{BC}} = \frac{{BP}}{{BQ}} = \frac{4}{6} \Rightarrow \triangle ABC \backsim \triangle PBQ \  (SAS)\Rightarrow {PQ}=5
        \]
        且有
        \[
        \angle ABP=\angle CBQ \Rightarrow \triangle PBA \backsim \triangle QBC \ (SAS) \Rightarrow {QC}=12
        \]
        由余弦定理,
        \[
        \cos \angle BQP = \frac{5^2 + 6^2 - 4^2}{2 \cdot 5 \cdot 6} = \frac{3}{4} \Rightarrow \sin \angle BQP = \frac{\sqrt{7}}{4}
        \]
        又 $\triangle PQC$ 三边长为 $5,12,13$,故为直角三角形,$\angle PQC=90^\circ$,于是
        \[
        \cos \angle APB = \cos \angle BQC = \cos(\angle BQP + 90^\circ) = -\sin \angle BQP = -\frac{\sqrt{7}}{4}
        \]
    \end{solution}

    \question 在 $\triangle ABC$ 中,$AB=4,AC=6,\cos(B-C)=\dfrac{2}{3}$,求 $BC$。
    \ifprintanswers
    \begin{figure}[H]
        \centering        
        \includegraphics[width=0.35\textwidth]{images/image86.jpg}
    \end{figure}
    \fi
    \begin{solution}
        $AC$ 上找一点 $P$使得 $\angle PBC = \angle C = \theta$,令 $PB = PC = a$,则 $AP = 6 - a$, 由余弦定理,
        \[
        (6-a)^2=4^2 + a^2 - 2 \cdot 4 \cdot a \cdot \frac{2}{3} \Rightarrow a = 3
        \]
        设 $BC = b$,则
        \[
        \cos \angle PBC = \cos \theta = \frac{a^2 + b^2 - a^2}{2ab} ,\quad
        \cos \angle ACB = \cos \theta = \frac{b^2 + 6^2 - 4^2}{12b}
        \]
        联立得
        \[
        \frac{b}{6} = \frac{b^2 + 20}{12b} \Rightarrow b = 2\sqrt{5}
        \]
    \end{solution}

    \question 在 $\triangle ABC$ 中, $AB=1$, $AC=2$, $B-C=\dfrac{2\pi}{3}$ ,求 $\triangle ABC$ 的面积。
    \begin{solution}
        由正弦定理知 
        \[
        \frac{\sin B}{\sin C}=\frac{AC}{AB}=2
        \]
        又 $B-C=\dfrac{2\pi}{3}$,故
        \[
        2\sin C=\sin B=\sin(C+\frac{2}{3}\pi)=-\frac{1}{2}\sin C+\frac{\sqrt{3}}{2}\cos C
        \]
        即 
        \[
        \frac{5}{2}\sin C=\frac{\sqrt{3}}{2}\cos C\Rightarrow\tan C=\frac{\sqrt{3}}{5}
        \]
        注意到 $A=\pi-B-C=\dfrac{\pi}{3}-2C$ ,故$\triangle ABC$ 的面积为
        \[
        [\triangle ABC]=\frac{1}{2}AB\cdot AC\cdot \sin A=\sin \left(\frac{\pi}{3}-2C\right)=\frac{\sqrt{3}}{2}\cos 2C-\frac{1}{2}\sin 2C
        \]
        由 $\tan C=\dfrac{\sqrt{3}}{5}$ 知 
        \[
        \cos 2C=\frac{1-\tan^2 C}{1+\tan^2 C}=\frac{11}{14},\ \sin 2C=\frac{2\tan C}{1+\tan^2 C}=\frac{5\sqrt{3}}{14}
        \]
        从而
        \[
        [\triangle ABC]=\frac{\sqrt{3}}{2}\cdot\frac{11}{14}-\frac{1}{2}\cdot\frac{5\sqrt{3}}{14}=\frac{3\sqrt{3}}{14}\]
    \end{solution}

    \question 设 $\triangle ABC$ 的三边长为 $AB=4, BC=5, CA=6,D,E,F$分别在边$BC,CA,AB$上使得三高为 $AD, BE, CF$,试求三面积比 $[\triangle AEF] : [\triangle BDF] : [\triangle CDE]$。
    \begin{solution}
        由
        \[
        \frac{[\triangle AEF]}{[\triangle ABC]} 
        =\frac{\frac12 AE \cdot AF \sin  A}{\frac12 AB \cdot AC \sin  A}
        =\frac{AB \cos  A \cdot AC \cos  A}{AB \cdot AC} 
        =\cos^2  A
        \]
        同理可得
        \[
        \frac{[\triangle BDF]}{[\triangle ABC]} = \cos^2  B,\quad \frac{[\triangle CDE]}{[\triangle ABC]} = \cos^2 C
        \]
        因此
        \[
        [\triangle AEF] : [\triangle BDF] : [\triangle CDE] = \cos^2  A : \cos^2  B : \cos^2  C
        \]
        由余弦定理,
        \[
        \cos  A = \frac{4^2+6^2-5^2}{2\cdot4\cdot6} = \frac{9}{16},\;
        \cos  B = \frac{4^2+5^2-6^2}{2\cdot4\cdot5} = \frac{1}{8},\;
        \cos  C = \frac{5^2+6^2-4^2}{2\cdot5\cdot6} = \frac{3}{4}
        \]
        所以
        \[
        [\triangle AEF] : [\triangle BDF] : [\triangle CDE] = \frac{81}{256} : \frac{1}{64} : \frac{9}{16} = 81 : 4 : 144
        \]
    \end{solution}

    \question $\triangle ABC$ 中,在 $BC$ 边上取 $D,E$,使得  
    \[
    \angle BAD = \angle DAE = \angle EAC.
    \] 
    已知 $BD=3,DE=4,EC=8$,求 $\triangle ABC$ 的面积。
    \begin{solution}
        设 $\angle BAD = \angle DAE = \angle DAC = \theta, AB=b, AC=c, AD=d, AE=e$,由等高性质,
        \[
        [\triangle ABD] : [\triangle ADE] : [\triangle AEC] = BD : DE : EC = 3:4:8
        \]
        又有
        \[
        [\triangle ABD] = \frac12 bd \sin \theta, \quad 
        [\triangle ADE] = \frac12 de \sin \theta, \quad
        [\triangle AEC] = \frac12 ec \sin \theta.
        \]
        故$b:e = 3:4, d:c = 1:2$,令 $b = 3q,\; c = 2p, \; d = p, \; e = 4q$,由余弦定理,
        \[
        \cos \theta = \frac{b^2+d^2-9}{2bd} = \frac{d^2+e^2-16}{2de} = \frac{e^2+c^2-64}{2ec}.
        \]
        解得
        \[
        \frac{9q^2+p^2-9}{6pq} = \frac{p^2+16q^2-16}{8pq} = \frac{16q^2+4p^2-64}{16pq} \Rightarrow p=6\sqrt 2, \; q=\sqrt 7
        \]
        于是
        \[
        \cos \theta = \frac{\sqrt{14}}{4} \Rightarrow \sin \theta = \frac{\sqrt 2}{4}.
        \]
        $\triangle ABC$ 面积为
        \[
        [\triangle ABC] = \frac12 bc \sin 3\theta = \frac12 (3\sqrt 7)(12\sqrt 2) \bigl(3\sin \theta - 4\sin^3\theta \bigr) 
        = \frac{45\sqrt 7}{2} 
        \]
    \end{solution}

    \question 设 \(a,b,c\) 分别为 \(\triangle ABC\) 的 \(A, B, C\) 对边长,且 \(\angle B = 60^\circ\),证明
    \[
    (a+b+c)\left(\frac{1}{a+b}+\frac{1}{b+c}\right)=3
    \]
    \begin{solution}
        展开左式得
        \[
        (a+b+c)\left(\frac{1}{a+b} + \frac{1}{b+c}\right) = 1 + \frac{c}{a+b} + 1 + \frac{a}{b+c} = 2 + \frac{a^2 + c^2 + ab + bc}{(a+b)(b+c)}
        \]
        由余弦定理,
        \[
        b^2=a^2+c^2-2ac\left(\frac{1}{2}\right) \Rightarrow a^2 + c^2 = b^2 + ac 
        \]
        因此左式为 
        \[
        2 + \frac{b^2 + ac + ab + bc}{(a+b)(b+c)} = 2 + \frac{(a+b)(b+c)}{(a+b)(b+c)} = 3
        \]
    \end{solution}

    \question 在$\triangle ABC$中,其内角$A,B,C$所对应的边分别为$a,b,c$,且
    \[
    \dfrac{b}{a}+\dfrac{a}{b}=4\cos C
    \]
    求$\tan C(\cot A+\cot B)$的值。
    \begin{solution}
        由已知$$\frac{b}{a}+\frac{a}{b}=4\cos C$$及余弦定理,
        \[
        \cos C= \frac{a^2+b^2}{4ab} = \frac{a^2+b^2-c^2}{2ab} \Rightarrow a^2+b^2 = 2c^2
        \]
        于是由余弦、正弦定理,
        \begin{align*}
        &\tan C(\cot A+\cot B) \\
        &= \frac{\sin C}{\cos C}\left( \frac{\cos A}{\sin A} + \frac{\cos B}{\sin B} \right)\\
        &= \frac{\cos A}{\cos C} \cdot \frac{\sin C}{\sin A} + \frac{\cos B}{\cos C} \cdot \frac{\sin C}{\sin B}\\
        &= \left( \frac{b^2+c^2-a^2}{2bc} \cdot \frac{2ab}{a^2+b^2-c^2} \right) \cdot \frac{c}{a} + \left( \frac{a^2+c^2-b^2}{2ac} \cdot \frac{2ab}{a^2+b^2-c^2} \right) \cdot \frac{c}{b} \\
        &= \frac{b^2+c^2-a^2}{a^2+b^2-c^2} + \frac{a^2+c^2-b^2}{a^2+b^2-c^2}
        = \frac{2c^2}{2c^2-c^2} = 2
        \end{align*}
    \end{solution}

    \question 设 $\triangle ABC$ 中 $\angle A,\angle B,\angle C$ 的对边分别为 $a,b,c$,若 $a^2+b^2=8c^2$,求
    \[
    \frac{\tan C}{\tan A} + \frac{\tan C}{\tan B}
    \]
    的值。
    \begin{solution}
        设 $\triangle ABC$ 面积为 $S$,则
        \[
        S = \frac{1}{2}bc \sin A \Rightarrow \sin A = \frac{2S}{bc}, \quad \cos A = \frac{b^2+c^2-a^2}{2bc}
        \]
        因此
        \[
        \tan A = \frac{4S}{b^2+c^2-a^2}, \quad \tan B = \frac{4S}{c^2+a^2-b^2}, \quad \tan C = \frac{4S}{a^2+b^2-c^2}
        \]
        所以
        \[
        \frac{\tan C}{\tan A} + \frac{\tan C}{\tan B} = \frac{b^2+c^2-a^2}{a^2+b^2-c^2} + \frac{c^2+a^2-b^2}{a^2+b^2-c^2} = \frac{2c^2}{a^2+b^2-c^2}
        \]
        代入 $a^2+b^2 = 8c^2$得
        \[
        \frac{2c^2}{8c^2 - c^2} = \frac{2}{7}
        \]
    \end{solution}

    \question 设 $\triangle ABC$ 的三边长 $AB=c,\,BC=a,\,CA=b$, 且 
    \[
    |b-c|\cos\frac{A}{2}=5,\,(b+c)\sin\frac{A}{2}=10,
    \]求 $a$ 之值。 
    \begin{solution}
        两式平方得
        \[
        (b-c)^2 \cos^2 \frac{A}{2} + (b+c)^2 \sin^2 \frac{A}{2} = 125
        \]
        整理得
        \[
        (b^2+c^2)\cos^2 \frac{A}{2} + (b^2+c^2)\sin^2 \frac{A}{2} - 2bc(\cos^2 \frac{A}{2} - \sin^2 \frac{A}{2}) = 125
        \]
        故
        \[
        b^2+c^2-2bc\cos A = a^2 = 125 \Rightarrow a = 5\sqrt{5}
        \]
    \end{solution}

    \question 设 $\triangle ABC$ 中,$A,B,C$ 所对应的边长分别为 $a,b,c$,已知 \[\frac{b}{a}=\frac{|b^2+c^2-a^2|}{bc},\ \frac{c}{b}=\frac{|c^2+a^2-b^2|}{ca},\ \frac{a}{c}=\frac{|a^2+b^2-c^2|}{ab}\] 且 $a<b<c$,求 $A$。
    \begin{solution}
        首先由$a<b<c \Rightarrow b^2+c^2>a^2$,则
        \[
        \frac{b}{a} = \frac{|b^2+c^2-a^2|}{bc} = \frac{b^2+c^2-a^2}{bc} 
        \]
        于是由余弦、正弦定理,
        \[
        \frac{b}{2a} = \frac{b^2+c^2-a^2}{2bc} \Rightarrow \frac{\sin B}{2\sin A} = \cos A \Rightarrow \sin B = \sin2A \Rightarrow B = 2 A
        \]
        同理,
        \[
        \frac{c}{b} = \frac{|c^2+a^2-b^2|}{ca} \Rightarrow C = 2 B
        \]
        因此
        \[
        A +  B +  C =  7 A = \pi \Rightarrow A = \frac{\pi}{7}
        \]
        \textcolor{red}{(待验证$c^2+a^2-b^2>0$)}
    \end{solution}

    \question 在 $\triangle ABC$ 中,角 $A,B,C$ 所对的边分别为 $a,b,c$,其中 $b \ge a$。若 
    \[
    2a\cos(B+C)+c\cos B+b\cos C=0,
    \] 
    且 $\triangle ABC$ 的外接圆半径为 $2$,求 $2b - c$ 的取值范围。
    \begin{solution}
        由条件得
        \[
        2a\cos(B+C)+c\cos B+b\cos C = -2a\cos A + a = 0 \Rightarrow A = 60^\circ
        \]
        由正弦定理,
        \[
        \frac{a}{\sin A} = \frac{a}{\frac{\sqrt{3}}{2}} = 2 \cdot 2 \Rightarrow a = 2\sqrt{3}
        \]
        又因 $b \ge a$,所以
        \[
        4\sin B \ge 2\sqrt{3} \Rightarrow \sin B \ge \frac{\sqrt{3}}{2}
        \]
        且有
        \[
        2b - c = 2 \cdot 4\sin B - 4\sin C = 8\sin B - 4\sin(120^\circ - B)
        \]
        \[
        = 8\sin B - 4\left(\frac{\sqrt{3}}{2}\cos B + \frac{1}{2}\sin B\right)
        = 6\sin B - 2\sqrt{3}\cos B
        = 4\sqrt{3}\sin(B - 30^\circ)
        \]
        因为$60^\circ \le B < 120^\circ$故
        \[2b - c \in {[2\sqrt{3},\; 4\sqrt{3})}
        \]
    \end{solution}

    \question 在$\triangle ABC$中, 若$a^{2}+b^{2}+c^{2}=2\sqrt{3}ab\sin C$, 求 $\cos\dfrac{A}{2}+\cos\dfrac{B}{2}+\cos\dfrac{C}{2}$。
    \begin{solution}
        由余弦定理$\ c^2 = a^2 + b^2 - 2ab\cos C$,代入原式得:
        \[
        a^2 + b^2 + (a^2 + b^2 - 2ab\cos C) =2a^2 + 2b^2 - 2ab\cos C= 2\sqrt{3}ab\sin C
        \]
        整理得
        \[
        \frac{a}{b} + \frac{b}{a} = \cos C + \sqrt{3}\sin C =2\sin\left(C + \frac{\pi}{6}\right)
        \]
        由 AM-GM 不等式可知
        \[
        \frac{a}{b} + \frac{b}{a} \ge 2
        \]
        当且仅当 $a = b$ 时等号成立,此时
        \[
        \sin\left(C + \frac{\pi}{6}\right) = 1 \Rightarrow C = \frac{\pi}{3}
        \]
        因为 $a = b$ 且 $C = \dfrac{\pi}{3}$,三角形 $\triangle ABC$ 为等边三角形,故 $A = B = C = \dfrac{\pi}{3}$,于是
        \[
        \cos\frac{A}{2} + \cos\frac{B}{2} + \cos\frac{C}{2} = 3\cos\frac{\pi}{6} = \frac{3\sqrt{3}}{2}
        \]
    \end{solution}

    \question 设$\triangle ABC$的三个内角$A,B,C$所对应的边分别为$a,b,c$.若$a^4+b^4+c^4=2c^2(a^2+b^2)$且$A=63^\circ$,求$B$。
    \begin{solution}
        由展开式
        $$(a^2+b^2-c^2)^2 = a^4+b^4+c^4+2a^2b^2-2a^2c^2-2b^2c^2 = 2a^2b^2$$
        得$$a^2+b^2-c^2 = \pm\sqrt{2}ab$$
        由余弦定理得$$ \cos C = \pm\frac{\sqrt{2}}{2}$$
        当$\cos C=\frac{\sqrt{2}}{2}$,即$C=45^\circ$,故$B=72^\circ$
        
        当$\cos C=-\frac{\sqrt{2}}{2}$,即$C=135^\circ,A+C=198^\circ>180^\circ$,不合题意.
        
        $\therefore \;B=72^\circ$
    \end{solution}

    \question 设$a,b,c$为一直角三角形的三条边,$\;\theta$是该三角形最小的角。若$\dfrac{1}{a},\;\dfrac{1}{b},\;\dfrac{1}{c}$也构成另一直角三角形,证明$\sin\theta=\dfrac{\sqrt{5}-1}{2}$。
    \begin{solution}
        不妨设$a<b<c$,则$\dfrac{1}{c}<\dfrac{1}{b}<\dfrac{1}{a}$。因此有
        \[
        a^2+b^2=c^2 \quad \text{以及} \quad \frac{1}{a^2}=\frac{1}{b^2}+\frac{1}{c^2}
        \]
        于是有
        \[
        \frac{1}{a^2}-\frac{1}{c^2}=\frac{1}{c^2-a^2}
        \]
        进一步整理可得
        \[
        c^{4}+a^{4}-3c^2a^2=0
        \]
        求解得到
        \[
        a^2=\frac{3c^2-\sqrt{5}c^2}{2}
        \]
        由于$\theta$是最小角,$\ \sin\theta=\dfrac{a}{c}$,故
        \[
        \frac{a}{c}=\sqrt{\frac{6-2\sqrt{5}}{4}}=\frac{\sqrt{5}-1}{2}
        \]
        证毕。
    \end{solution}
    
    \question 已知三角形 $\triangle ABC$ 满足 $\sin A = \cos B = \dfrac{16}{21}\tan C$,且 ${AC} = 1$,求三角形面积。
    \begin{solution}
        因为 $\cos B = \sin\left(\dfrac{\pi}{2} \pm B\right)$,故
        \[
        \sin A = \sin\left(\frac{\pi}{2} \pm B\right)\Rightarrow A=\frac{\pi}{2} \pm B
        \]
        
        若 $ A = \dfrac{\pi}{2} - B$,则 $ C = \dfrac{\pi}{2}$,使 $\tan C$ 无定义,不合题意;于是
        \[
         A = \frac{\pi}{2} + B ,C = \pi -  A -  B = \frac{\pi}{2} - 2B
        \]
        所以
        \[
        \cos B = \frac{16}{21} \tan\left(\frac{\pi}{2} - 2B\right) = \frac{16}{21} \cot 2B = \frac{16}{21} \frac{\cos 2B}{\sin 2B}=\frac{16}{21}\cdot\frac{1-2\sin^{2}B}{2\sin B\cos B}
        \]
        解得
        $$(3\sin B-1)(7\sin^{2}B-3\sin B-8)=0,\sin B=\frac{1}{3}\ \text{或}\ \frac{3\pm\sqrt{233}}{14}$$ 
        注意到 $\sin B=\frac{3+\sqrt{233}}{14}>1$(不合题意) 及 $\sin B=\frac{3-\sqrt{233}}{14}<0\Rightarrow B>\frac{\pi}{2}$(不合题意), 故
        \[
        \sin B=\frac{1}{3}\Rightarrow \cos B = \sin A = \frac{2\sqrt{2}}{3}, \ \tan C = \frac{7\sqrt{2}}{8}
        \]
        由正弦定理
        \[
        \frac{{AC}}{\sin B} = \frac{{BC}}{\sin A} \Rightarrow {BC} = 2\sqrt{2}
        \]
        三角形面积为
        \[
        \frac{1}{2} \cdot {AC} \cdot {BC} \cdot \sin C = \frac{1}{2} \cdot 1 \cdot 2\sqrt{2} \cdot \frac{7}{9} = \frac{7\sqrt{2}}{9}
        \]
    \end{solution}
    
    \question 已知$\triangle ABC$内角$A,B,C$的对边分别为$a,\;b,\;c$,满足$b^2=ac$,点$D$在边$AC$上使得$AD = 2DC$,且$BD \sin\angle ABC = a \sin C$,求$\cos\angle ABC$。
    \begin{solution}
        \textbf{解法一}
        
        由题意知$\;BD = b,AD = \dfrac{2b}{3},DC = \dfrac{b}{3}$,且$\angle ADB,\angle CDB$互补,由余弦定理,
        \[
        \cos\angle ADB = -\cos\angle CDB\Rightarrow \frac{b^2 + \frac{4b^2}{9} - c^2}{2b \cdot \frac{2b}{3}}=-\frac{b^2 + \frac{b^2}{9} - a^2}{2b \cdot \frac{b}{3}}\Rightarrow2a^2 + c^2 = \dfrac{11b^2}{3}
        \]
        又$b^2 = ac$,所以$$6a^2 - 11ac + 3c^2 = 0\Rightarrow c = 3a\ \text{或}\ c = \dfrac{2}{3}a$$
        当$c = 3a$时,$\;b^2 = 3a^2$,由余弦定理得$\cos\angle ABC = \dfrac{7}{6}>1$,不合题意。
        
        当$c = \dfrac{2}{3}a$时,$\;b^2 = \dfrac{2}{3}a^2$,由余弦定理,
        \[
        \cos\angle ABC = \frac{a^2 + c^2 - b^2}{2ac} = \frac{a^2 + \frac{4}{9}a^2 - \frac{2}{3}a^2}{2a \cdot \frac{2}{3}a} = \frac{7}{12}
        \]
    \end{solution}
    \begin{solution}
        \textbf{解法二}
        
        已知 $AD = 2DC$,则 $[\triangle ABD] = \dfrac{2}{3} [\triangle ABC]$,即
        \[
        \frac{1}{2} \cdot \frac{2}{3} b^2 \sin\angle ADB = \frac{2}{3} \cdot \frac{1}{2} ac \sin\angle ABC,
        \]
        而 $b^2 = ac$,即 $\sin\angle ADB = \sin\angle ABC$,故有 $\angle ADB = \angle ABC$,从而 $\angle ABD = \angle C$。
        
        由 $b^2 = ac$,即 $\dfrac{b}{a} = \dfrac{c}{b}$,即 $\dfrac{CA}{CB} = \dfrac{BA}{BD}$,即 $\triangle ACB \backsim \triangle ABD \ (SAS)$,故
        \[
        \frac{AD}{AB} = \frac{AB}{AC} \implies \frac{2b}{c} = \frac{c}{b},
        \]
        又 $b^2 = ac$,所以 $c = \dfrac{2}{3}a$,则
        \[
        \cos\angle ABC = \frac{c^2 + a^2 - b^2}{2ac} = \frac{7}{12}.
        \]
    \end{solution}

    \question 已知三角形 $ABC$ 内切圆切$BC,CA,AB$于 $D,E,F$,若内切圆半径为 $r$,外接圆半径为 $R$,试证 
    \[
    \frac{[\triangle DEF]}{[\triangle ABC]} = \frac{r}{2R}.\]
    \begin{figure}[H]
        \centering
        \includegraphics[width=0.5\linewidth]{images/image43.png}
    \end{figure}
    \begin{solution}
        设三角形 $ABC$ 的边长为 $a, b, c$,内心 $I$,有 $ID = IE = IF = r,\triangle DEF$ 面积为
        \begin{align*}
        [\triangle DEF] 
        &= [\triangle IDE]+[\triangle IEF]+ [\triangle IFD] \\
        &= \frac{1}{2} r^2 (\sin (\pi-A) + \sin (\pi-B) + \sin (\pi-C)) \\
        &= \frac{1}{2} r^2 (\sin A + \sin B + \sin C)
        \end{align*}
        由正弦定理,
        \[
        \sin A = \frac{a}{2R},\quad \sin B = \frac{b}{2R},\quad \sin C = \frac{c}{2R}
        \]
        代入得
        \[
        [\triangle DEF] = \frac{1}{2} r^2 \left( \frac{a+b+c}{2R} \right) = \frac{r^2 s}{2R}
        \]
        因此:
        \[
        \frac{[\triangle DEF]}{[\triangle ABC]} = \frac{\frac{r^2 s}{2R}}{rs} = \frac{r}{2R}
        \]
    \end{solution}

    \question 设 $\triangle ABC$ 的三边长为 $a,b,c$,面积为 $S$,试证明
    \[
    \frac{a^{2} + b^{2} + c^{2}}{S} \ge 4\sqrt{3}.
    \]
    \begin{solution}
        已知三角形的三边为 $a,b,c$,其对角分别为 $A,B,C$。利用余弦定理与面积公式可得
        \[
        c^{2}=a^{2}+b^{2}-2ab\cos C, \quad S=\frac{1}{2}ab\sin C
        \]
        因此
        \begin{align*}
        a^{2}+b^{2}+c^{2}-4S\sqrt{3}&=2[a^{2}+b^{2}-ab(\cos C+\sqrt{3}\sin C)] \\
        &=2\left[a^{2}+b^{2}-2ab\sin\left(\frac{\pi}{6}+C\right)\right] \\
        &\ge 2(a^{2}+b^{2}-2ab) \\
        &=2(a-b)^{2} \\
        &\ge 0
        \end{align*}
        等号成立当且仅当 $a=b$ 且 $\sin\left(\dfrac{\pi}{6}+C\right)=1$即 $a=b,C=\dfrac{\pi}{3}$,亦即 $a=b=c$。
    \end{solution}

    \question 一个梯形的两底比为 $2:1$,两腰比为 $2:1$,两对角线比为 $2:1$。求短底与短腰的比。
    \ifprintanswers
    \begin{figure}[H]
        \centering
        \includegraphics[width=0.4\textwidth]{images/image147.png}
    \end{figure}
    \fi
    \begin{solution}
        设短腰为 $1$,短底为 $x$,高度为 $h$,与短腰相连的底边段长为 $y$,则
        \[
        1-y^{2}=h^{2}=4-(x-y)^{2}
        \]
        且对角线比给出
        \[
        4\bigl(h^{2}+(x+y)^{2}\bigr)=h^{2}+(2x-y)^{2}.
        \]
        解得
        \[
        x=\frac{\sqrt{10}}{2}.
        \]
        因此短底与短腰的比为
        \[
        \sqrt{10}:2
        \]
    \end{solution}

    \question 梯形 \(ABCD\) 中,\(AD\parallels BC,\;BC=3,\;CD=4\),且  
    \(\cos\angle ADC=\dfrac13,\;\angle ABC = 2\angle ADC\)。求 \(AC\)。  
    \begin{solution}
        \textcolor{red}{(待解)}
    \end{solution}
    
    \question 一匀速转动的摩天轮直径 $18$米,底端距地 $1$米。  
        康康在 \(P\) 处距地 $16$米,用 $4$ 秒到达顶端 \(T\),再用 $8$ 秒到达 \(Q\)。 求他抵达 \(Q\) 时离地面的距离。 
    \begin{figure}[H]
        \centering
        \includegraphics[width=0.3\textwidth]{images/image3.png}
    \end{figure}
    \begin{solution}
        设摩天轮圆心为$O$,\;$\angle POT=\theta$,则$\angle TOQ=2\theta$,发现到\[
        \cos \theta=\frac{6}{9}=\frac{2}{3}
        \]与此同时,作$QR$垂直于过圆心$O$的水平线于点$R$,有\[
            \sin (2\theta - 90^\circ)=\frac{QR}{9}
        \]又\[
            \sin (2\theta - 90^\circ)=-\cos2\theta =1-2\cos^2 \theta 
        \]则$QR=1$,故Q距地$(9-1)+1=9$米。
    \end{solution}
    
    \question 设圆心 \(O\)、半径 \(r\) 的圆的切线 \(AB\),点 \(C\) 在线 \(BD\) 上,且  
        \(AB=p,\;BC=CD=DO=q\)。 证明  
        \[
          p^{2}=q^{2}+r^{2}.
        \]
    \begin{figure}[H]
        \centering
        \includegraphics[width=0.25\textwidth]{images/image5.png}
    \end{figure}
    \begin{solution}
        在$\triangle ODC$中,由余弦定理
        \[
        r^2=q^2+q^2-2q^2\cos \angle ODC \Rightarrow \cos \angle ODC=\dfrac{2q^2-r^2}{2q^2}
        \]
        在$\triangle ODB$中,由余弦定理
        \[
        OB^2=q^2+(2q)^2-2 \cdot q \cdot 2q \cdot \dfrac{2q^2-r^2}{2q^2}=q^2+2r^2
        \]
        来到尾声,在$\triangle OAB$中,由毕氏定理
        \[
        OA^2+AB^2=OB^2 \Rightarrow r^2+p^2=q^2+2r^2
        \]
        于是得证 $p^{2}=q^{2}+r^{2}$。
    \end{solution}

    \question 一平行四边形$ABCD$面积为 $36$,对角线$AC,BD$长度为 $10,12$。求$AD$的长度。
    \begin{solution}
        设$M$在$BD$上使得$AM \perp BD$,于是
        \[
        \frac{1}{2}\cdot AM \cdot BD = 36 \Rightarrow AM = 3
        \]
        设对角线$AC,BD$交于点$O$,则 $AO = \dfrac{BD}{2} = 6$,在$\triangle AOM$ 中,由毕氏定理,
        \[
        OM = \sqrt{6^2 - 3^2} = 3\sqrt{3}
        \]
        在$\triangle AOM$ 中,由毕氏定理,
        \[
        AD = \sqrt{3^2 + (3\sqrt{3}+5)^2} = \sqrt{61 + 30\sqrt{3}}
        \]
    \end{solution}

    \question 在平行四边形 $ABCD$ 中,将边 $AD$ 延长至点 $F$,线段 $FB$ 与边 $CD$ 相交于 $G$,与对角线 $AC$ 相交于 $E$,且满足 $FG=4,GE=1$。求 $EB$ 的长度。
    \begin{figure}[H]
        \centering
        \includegraphics[width=0.4\textwidth]{images/image151.png}
    \end{figure}
    \begin{solution}
        设 $EB = x,DG = y,DC = b$,由$\triangle EBA \backsim \triangle EGC \; \text{(AAA)}$,
        \[
        \frac{x}{1} = \frac{b}{b-y}
        \]
        又由 $\triangle GFD \backsim \triangle GBC\; \text{(AAA)}$ 得
        \[
        \frac{4}{y} = \frac{1+x}{b-y} 
        \]
        解得
        \[
        EB =x= \sqrt{5}
        \]
    \end{solution}

    \question 平面上 $AC=AD,\angle ABC=90^\circ, \angle CAD=\alpha,\angle CBD=\beta, \angle CAB=\gamma$, 若 $\cos\alpha=\dfrac{4}{5}, \cos\beta=\dfrac{8}{17}$, 求 $\tan\gamma$。
    \ifprintanswers
    \begin{figure}[H]
        \centering
        \includegraphics[width=0.4\textwidth]{images/image91.jpg}
    \end{figure}
    \fi
    \begin{solution}
        在 $\triangle ABD$ 中, $\angle DAB = \gamma - \alpha, \angle ABD = 90^\circ + \beta,$则
        \[
        \angle ADB = 90^\circ + \alpha - \beta - \gamma
        \]
        由正弦定理,
        \[
        \frac{AB}{AD} = \frac{\sin(90^\circ+\alpha-\beta-\gamma)}{\sin(90^\circ+\beta)} = \frac{\cos(\alpha-\beta-\gamma)}{\cos\beta} = \frac{\cos(\alpha-\beta-\gamma)}{\frac{8}{17}}
        \]
        由于 $AD=AC=AB \cos \gamma$, 所以
        \begin{align*}
        \frac{8}{17}\cos \gamma 
        &= \cos(\alpha-\beta-\gamma)= \cos(\alpha-\beta)\cos\gamma + \sin(\alpha-\beta)\sin\gamma\\
        &= (\cos\alpha \cos\beta + \sin\alpha \sin\beta)\cos\gamma + (\sin\alpha \cos\beta - \sin\beta \cos\alpha)\sin\gamma\\
        &= \left(\frac{4}{5}\cdot \frac{8}{17} + \frac{3}{5}\cdot \frac{15}{17}\right)\cos\gamma + \left(\frac{3}{5}\cdot \frac{8}{17} - \frac{15}{17}\cdot \frac{4}{5}\right)\sin\gamma\\
        &= \frac{77}{85}\cos\gamma - \frac{36}{85}\sin\gamma
        \end{align*}
        得到
        \[
        \tan \gamma = \frac{37}{36}
        \]
    \end{solution}

    \question 四边形 $ABCD$ 的对角线 $AC,BD$ 交于 $E$ 点,且 $$BE=DE=\dfrac{1}{2}AD=3,\angle ADB=90^\circ,\angle ACD=45^\circ,$$求 $\triangle BCD$ 的面积。
    \ifprintanswers
    \begin{figure}[H]
        \centering        
        \includegraphics[width=0.5\textwidth]{images/image123.jpg}
    \end{figure}
    \fi
    \begin{solution}
        在直角 $\triangle ADE$ 中,
        \[
        AE^2=6^2+3^2=45 \Rightarrow AE=3\sqrt{5}
        \]
        在直角$\triangle ADB$ 中,
        \[
        AB^2=6^2+6^2=72 \Rightarrow AB=6\sqrt{2}
        \]
        在 $\triangle AEB$ 中,由余弦定理,
        \[
        \cos \angle AEB = \frac{45+9-72}{18\sqrt{5}} = -\frac{1}{\sqrt{5}} \Rightarrow \sin \angle AEB = \frac{2}{\sqrt{5}}
        \]
        在 $\triangle CDE$ 中,由正弦定理,
        \[
        \frac{3}{\frac{1}{\sqrt{2}}} = \frac{CD}{\frac{2}{\sqrt{5}}} \Rightarrow CD = 6\sqrt{\frac{2}{5}}
        \]
        又
        \[
        \sin \angle BDC = \sin(45^\circ+\angle DEC) = \frac{1}{\sqrt{2}}\cdot \left(-\frac{1}{\sqrt{5}}\right) + \frac{2}{\sqrt{5}}\cdot \frac{1}{\sqrt{2}} = \frac{1}{\sqrt{10}}
        \]
        因此$\triangle BCD$ 面积为
        \[
        \frac{1}{2}\cdot 6\sqrt{\frac{2}{5}}\cdot 6\cdot \frac{1}{\sqrt{10}} = \frac{18}{5}
        \]
    \end{solution}

    \question 有一正三角形 $\triangle ABC$ 的艺术品,边长为 6 公尺,以 $AB$ 为边斜靠在墙上,墙角为 $O$ 点,形成直角 $\triangle OAB,\angle AOB=90^\circ,A$ 点在地面上,$B$ 点在墙上。过 $B$ 点作与地面平行的直线交 $AC$ 于点 $D$,已知 $CD=2$ 公尺,求此艺术品的最高点离地面的高度。
    \ifprintanswers
    \begin{figure}[H]
        \centering        
        \includegraphics[width=0.5\textwidth]{images/image85.jpg}
    \end{figure}
    \fi
    \begin{solution}
        由余弦定理,
        \[
        BD^2=6^2+2^2-2 \cdot 2 \cdot 6 \cdot \frac{1}{2} \Rightarrow BD = 2\sqrt{7}
        \]
        在 $\triangle CBD$ 中,
        \[
        \frac{2}{\sin \angle CBD} = \frac{2\sqrt{7}}{\frac{\sqrt{3}}{2}} \Rightarrow \sin \angle CBD = \frac{\sqrt{3}}{2\sqrt{7}}
        \]
        于是
        \[
        CP = BC \sin \angle CBD = 6 \cdot \frac{\sqrt{3}}{2\sqrt{7}} = 3\sqrt{\frac{3}{7}}.
        \]
        又 $DP \parallels AQ$,由相似三角形,
        \[
        \frac{2}{6} = \frac{3\sqrt{\frac{3}{7}}}{CQ} \Rightarrow CQ = \frac{9}{7}\sqrt{21}\text{ 公尺}
        \]
    \end{solution}

    \question 某人在地面 $A$ 点,测得山峰的仰角为 $\theta$。此人向山脚前进 $110$ 公尺到达 $B$ 点,测得山峰仰角为 $2\theta$,再向山脚前进 $30$ 公尺到达 $C$ 点,又测得山峰仰角为 $90^\circ - \theta$。求山高 $h$。
    \ifprintanswers
    \begin{figure}[H]
        \centering        
        \includegraphics[width=0.6\textwidth]{images/image127.jpg}
    \end{figure}
    \fi
    \begin{solution}
        设从 $C$ 点到山脚的水平距离为 $a$ 公尺,则:
        \[
        h = a \tan (90^\circ - \theta) = \frac{a}{\tan \theta} \tag{1}
        \]
        \[
        h = (30+a) \tan 2\theta \tag{2}
        \]
        \[ 
        h = (140+a) \tan \theta \tag{3}
        \]
        由 $(1)=(3)$ 和 $(1)=(2)$,
        \[
        \frac{a}{140+a} = \tan^2 \theta \tag{4}
        \]
        \[
        \frac{a}{30+a} = \frac{2 \tan^2 \theta}{1 - \tan^2 \theta} = -2 + \frac{2}{1 - \tan^2 \theta} \tag{5}
        \]
        将 (4) 代入 (5),
        \[
        \frac{a}{30+a} = -2 + \frac{2}{1 - \frac{a}{140+a}} \Rightarrow a = 40
        \]
        于是
        \[
        \tan^2 \theta = \frac{40}{180} \Rightarrow \tan \theta = \frac{\sqrt{2}}{3}
        \]
        则山高为
        \[
        h = (140 + 40) \tan \theta = 180 \cdot \frac{\sqrt{2}}{3} = 60 \sqrt{2}
        \]
    \end{solution}

    \question 已知 \(\triangle ABC\) 满足 $2\angle BAC=3\angle ABC$,且 \(K\) 为 \(BC\) 上一点使得 \(\angle KAC = 2\,\angle KAB\),证明
    \begin{figure}[H]
        \centering
        \includegraphics[width=0.4\textwidth]{images/image4.png}
    \end{figure}
    \begin{parts}
    \part $AK=\dfrac{bc}{a},\;BK=\dfrac{a^{2}-b^{2}}{a}$
    \begin{solution}
        记$\angle KAB=\theta$,则$\angle KAC=2\theta,\angle ABC=2\theta,\angle CKA=\theta+2\theta=3\theta$,
        所以
        \[
        \triangle CAK \backsim \triangle CBA \ (\text{AAA}) \Rightarrow AK=\dfrac{bc}{a}
        \]
        由余弦定理,
        \[
        CK^2 =b^2+d^2-2bd \cos 2\theta 
        = b^2+\left(\dfrac{bc}{a}\right)^2-2b\left(\dfrac{bc}{a}\right) \left(\dfrac{a^2+c^2-b^2}{2ac}\right)
        = \dfrac{b^4}{a^2}
        \]
        于是 $CK= \dfrac{b^2}{a}$,且 
        \[
        BK=a-CK=\dfrac{a^{2}-b^{2}}{a}
        \]
    \end{solution}
    \part $(a^{2}-b^{2})(a^{2}-b^{2}+ac)=b^{2}c^{2}$
    \begin{solution}
        记$AK=x,BK=y$,由正弦定理,\[
        \frac{x}{\sin 2\theta}=\frac{y}{\sin \theta} \Rightarrow x=2y \cos \theta
        \]
        又由余弦定理,
        \[
        \cos \theta=\dfrac{c^2+x^2-y^2}{2cx}
        \]
        于是
        \[
        cx^2=y(x^2+c^2-y^2) \Rightarrow x^2=y(y+c)
        \]
        代入 $x=\dfrac{bc}{a}, y=\dfrac{a^{2}-b^{2}}{a}$可得\[
        (a^{2}-b^{2})(a^{2}-b^{2}+ac)=b^{2}c^{2}
        \]
    \end{solution}
    \end{parts}

    \question 已知正方形 $DEFG$ 内接于 $\triangle ABC,EF$ 在 $BC$ 上且 $D,G$ 分别在 $AB,AC$ 上,已知 $\triangle ADG,\triangle BDE,\triangle CGF$ 的面积分别为 $1,3,1$,求 $\triangle ABC$ 的面积。 
    \ifprintanswers
    \begin{figure}[H]
        \centering
        \includegraphics[width=0.5\textwidth]{images/image93.jpg}
    \end{figure}
    \fi
    \begin{solution}
        作 $AQ \perp BC$交 $DG$ 于 $P$,因此$\triangle APG \backsim \triangle GFC \ (\text{AAA}),\triangle APD \backsim \triangle DEB \ (\text{AAA})$,且有
        \[
        \frac{[\triangle APG]}{[\triangle APD]} = \frac{[\triangle GFC]}{[\triangle DEB]} = \frac{1}{3}
        \Rightarrow [\triangle APD]=\frac{3}{4},\quad [\triangle APG]=\frac{1}{4}
        \]
        又
        \[
        \frac{[\triangle BDE]}{[\triangle APD]}=\frac{3}{\frac{3}{4}} = \frac{DE^2}{AP^2}\Rightarrow DE = 2AP
        \]
        因此
        \[
        [\triangle ADG] = 1 = \frac12 \cdot DG \cdot AP = \frac12 \cdot 2(AP)^2 \Rightarrow AP = 1 , DG = 2 
        \]
        $\triangle ABC$面积为
        \[
        1+3+1+2^2 = 9
        \]
    \end{solution}

    \question 设 $P$ 为正 $\triangle ABC$ 内部一点,若 $P$ 点依序到三边 $BC,AC,AB$ 之距离比为 $1:3:2$,求 $PA^2 : PB^2 : PC^2$。
    \ifprintanswers
    \begin{figure}[H]
        \centering
        \includegraphics[width=0.4\textwidth]{images/image79.jpg}
    \end{figure}
    \fi
    \begin{solution}
        作
        \[
        MN \parallels AB, \quad ST \parallels BC, \quad UV \parallels AC,
        \]
        则 $\triangle PUM, \triangle PSV, \triangle PTN$ 均为正三角形,且 $1, 2, 3$ 分别为正三角形的高,因此
        \[
        PU = PM = UM = \frac{2}{\sqrt{3}}, \quad
        PV = PS = VS = \frac{4}{\sqrt{3}}, \quad
        PN = PT = NT = \frac{6}{\sqrt{3}}.
        \]
        由毕氏定理,
        \[
        PA^2 = 2^2 + \left(\frac{6}{\sqrt{3}} + \frac{2}{\sqrt{3}}\right)^2 = \frac{76}{3}, \quad
        PB^2 = 1^2 + \left(\frac{4}{\sqrt{3}} + \frac{1}{\sqrt{3}}\right)^2 = \frac{28}{3},
        \]
        \[
        PC^2 = 3^2 + \left(\frac{2}{\sqrt{3}} + \frac{3}{\sqrt{3}}\right)^2 = \frac{52}{3}.
        \]
        故
        \[
        PA^2 : PB^2 : PC^2 = 19 : 7 : 13
        \]
    \end{solution}

\question 圆内接四边形 $ABCD$ 满足 $AB=4,BC=1,CD=2,DA=3$。设 $P,Q$ 分别为 $BC$ 与 $DA$ 的中点,求 $PQ^2$。
\ifprintanswers
\begin{figure}[H]
    \centering
    \includegraphics[width=0.4\textwidth]{images/image157.png}
\end{figure}
\fi
\begin{solution}
    设对角线交点为 $R$,令 $CR=x$。由$\triangle BCR \backsim \triangle ADR \; \text{(AAA)}$ 及 $\triangle CDR \backsim \triangle BAR \; \text{(AAA)}$可得
    \[
    BR=2x,\quad DR=3x,\quad AR=6x,
    \]
    因此
    \[
    AC=7x,\quad BD=5x
    \]
    在四边形 $ABCD$中,由托勒密定理,
    \[
    AC\cdot BD=AB\cdot CD+BC\cdot DA 
    \]
    得
    \[
    x^2=\frac{11}{35}
    \]
    在$\triangle ABD$及$\triangle ACD$中,由中线定理,
    \[
    AB^2+BD^2=\frac{1}{2}DA^2+2BQ^2, \quad AC^2+CD^2=\frac{1}{2}DA^2 + 2CQ^2
    \]
    得
    \[
    \textcolor{red}{BQ^2=\frac{271}{28},\;CQ^2=\frac{571}{20}}
    \]
    故在$\triangle BCQ$中,由中线定理,
    \[
    BQ^2+CQ^2 = \frac{1}{2}BC^2+2PQ^2 \Rightarrow PQ^2=\frac{291}{35}
    \]
    \textcolor{red}{(待验证)}
\end{solution}

    \question 已知三角形 $ABC$ 和点 $X_1$ 在 $AC$ 上,满足 $AB=AC=91,BC=70,X_1C=65$。设 $X_0 = B$,点 $X_2, X_4, X_6, \ldots$ 在 $AB$ 上,$X_3, X_5, X_7, \ldots$ 在 $AC$ 上,且 $X_{n+1}X_n \perp X_nX_{n-1}$ 对 $n \ge 1$ 成立。求 $\displaystyle \sum_{n=1}^{\infty} X_{n-1}X_n$。
    \ifprintanswers
    \begin{figure}[H]
        \centering
        \includegraphics[width=0.4\textwidth]{images/image153.png}
    \end{figure}
    \fi
    \begin{solution}
        设 $M,N$ 分别为 $A,X_1$ 在 $BC$ 上的垂足,由$\triangle AMC \backsim \triangle X_1NC \; \text{(AAA)}$
        \[
        \frac{NC}{MC} = \frac{X_1C}{AC} \Rightarrow NC = \frac{65}{91} \cdot 35 = 25.
        \]
        由毕氏定理,
        \[
        X_1N = \sqrt{65^2 - 25^2} = 60, \quad X_0X_1 = \sqrt{60^2 + 45^2} = 75.
        \]
        则
        \[
        \tan \angle X_1BA = \tan(\angle ABC - \angle X_1BC) = \frac{\frac{12}{5} - \frac{4}{3}}{1 + \frac{12}{5} \cdot \frac{4}{3}} = \frac{16}{63}
        \]
        因此
        \[
        X_1X_2 = X_0X_1 \cdot \frac{16}{63} = \frac{400}{21}, \quad X_0X_2 = \sqrt{75^2 + \left(\frac{400}{21}\right)^2} = 25 \cdot \frac{65}{21}
        \]
        且$AX_n,AX_{n+2},\dots$成等比数列,公比为
        \[
        r = \frac{AX_2}{AX_0} = \frac{91 - \frac{25 \cdot 65}{21}}{91} = 1 - \frac{125}{147}
        \]
        因此无穷和为
        \[
        \sum_{n=1}^{\infty} X_{n-1}X_n = \frac{X_0X_1 + X_1X_2}{1-r} = \left(75 + \frac{400}{21}\right) \cdot \frac{147}{125} = \frac{553}{5}
        \]
    \end{solution}

    \question 一个等边三角形中有 $n$ 排全等的小圆(如下图 $n=4$ 所示)。求当 $n\to\infty$ 时,圆的面积与三角形面积的比值极限。
    \ifprintanswers
    \begin{figure}[H]
        \centering
        \includegraphics[width=0.4\textwidth]{images/image143.png}
    \end{figure}
    \fi
    \begin{solution}
        设三角形边长为 $1$,小圆半径为 $r$,由 $30$-$60$-$90$ 直角三角形可得
        \[
        2(n-1)r + 2 r\sqrt{3} = 1 \Rightarrow r = \frac{1}{2(n+\sqrt{3}-1)}.
        \]
        共有 $\frac{n(n+1)}{2}$ 个圆,因此圆的总面积为
        \[
        \frac{n(n+1)}{2} \cdot \frac{\pi}{4(n+\sqrt{3}-1)^2}= \frac{\pi}{8} \frac{n(n+1)}{(n+\sqrt{3}-1)^2}.
        \]
        当 $n\to\infty$ 时,$\dfrac{n(n+1)}{(n+\sqrt{3}-1)^2}\to 1$,因此面积比极限为
        \[
        \frac{\dfrac{\pi}{8}}{\dfrac{\sqrt{3}}{4}} = \frac{\pi\sqrt{3}}{6}
        \]
    \end{solution}

    \question 已知 $C_1$ 是半径为 1 的圆,$P_1$ 是 $C_1$ 的外接正方形。对于 $n \ge 2$,令 $C_n$ 为 $P_{n-1}$ 的外接圆,$P_n$ 为 $C_n$ 的外接正 $2^{n+1}$ 边形。设 $r_n$ 为 $C_n$ 的半径。求 $\lim_{n \to \infty} r_n$。
    \ifprintanswers
    \begin{figure}[H]
        \centering
        \includegraphics[width=0.4\textwidth]{images/image155.png}
    \end{figure}
    \fi
    \begin{solution}
        由图可知,
        \[
        r_2 = \frac{r_1}{\cos\frac{\pi}{4}}, \quad r_n = \frac{r_{n-1}}{\cos\frac{\pi}{2^n}}, \quad n\ge 2
        \]
        于是
        \[
        r_n = \frac{1}{\displaystyle \prod_{k=2}^{n} \cos \frac{\pi}{2^k}}
        \]
        且由
        \[
        \sin x = 2 \sin \frac{x}{2} \cos \frac{x}{2} = 2^2 \sin \frac{x}{4} \cos \frac{x}{4} \cos \frac{x}{2} = \cdots = 2^n \sin \frac{x}{2^n} \prod_{k=1}^{n} \cos \frac{x}{2^k}
        \]
        两边除以 $x$ 并取极限 $n\to\infty$ 得
        \[
        \frac{\sin x}{x} = \prod_{k=1}^{\infty} \cos \frac{x}{2^k}
        \]
        令 $x = \dfrac{\pi}{2}$,得到
        \[
        \prod_{k=1}^{\infty} \cos \frac{\pi}{2^{k+1}} = \frac{2}{\pi}
        \]
        因此
        \[
        \lim_{n\to\infty} r_n = \frac{\pi}{2}
        \]
    \end{solution}

    \question 圆内有两条平行弦,其长分别为 12,16,且它们之间的距离为 7。另一条弦位于两条弦中间且与两线等距,求该弦的长度。
    \begin{solution}
        设圆半径为 $r$,两条弦到圆心的距离分别为 $d_1,d_2$,由毕氏定理,
        \[
        \left(\frac{12}{2}\right)^2 + d_1^2 = r^2, \quad \left(\frac{16}{2}\right)^2 + d_2^2 = r^2 \implies 36 + d_1^2 = 64 + d_2^2.
        \]
        假设两弦在圆心的两侧,则
        \[
        d_1 + d_2 = 7, \quad d_1 - d_2 = 4 \implies d_1 = \frac{11}{2}>0, \quad d_2 = \frac{3}{2}>0
        \]
        于是
        \[
        r^2 = 36 + d_1^2 = \frac{265}{4}
        \]
        中间弦到圆心的距离为
        \[
        d = \frac{d_1 - d_2}{2} = 2
        \]
        设中间弦长度为 $x$,则由毕氏定理,
        \[
        \left(\frac{x}{2}\right)^2 + 2^2 = r^2 \Rightarrow x = \sqrt{249}
        \]
    \end{solution}

    \question 在图中,两圆在点 $P$ 相切。$QP$ 和 $SU$ 是大圆的垂直直径,交于 $O$。点 $V$ 在 $QP$ 上使得$VP$ 是小圆的直径。小圆与 $SU$ 相交于 $T$。已知 $QV = 9,ST = 5$,求两圆直径之和。
    \begin{figure}[H]
        \centering
        \includegraphics[width=0.4\textwidth]{images/image173.png}
    \end{figure}
    \begin{solution}
        设大圆直径为 $D$,小圆直径为 $d$,小圆圆心为 $C$,在$\triangle TOC$中,由毕氏定理,
        \[
        TO^2 + OC^2 = CT^2 \Rightarrow \left(\frac{D}{2} - 5\right)^2 + \left(\frac{D}{2} - \frac{d}{2}\right)^2 = \left(\frac{d}{2}\right)^2
        \]
        由 $QV=D-d = 9$,得
        \[
        (D-10)^2 + 9^2 = d^2 
        \]
        于是
        \[
        81 = (d-(D-10))(d+(D-10)) = 1\cdot(d+D-10)
        \]
        所以
        \[
        d + D = 91
        \]
    \end{solution}

    \question 如图所示,三个圆的圆心分别为 $X,Y,Z$,每个圆都与另外两个圆相切。圆心为 $X$ 的圆与矩形 $PQRS$ 相切于三条边,圆心为 $Z$ 的圆与矩形 $PQRS$ 相切于两条边。如果 $XY=30,YZ=20,XZ=40$,求矩形 $PQRS$ 的面积。  
    \begin{figure}[H]
        \centering
        \includegraphics[width=0.4\textwidth]{images/image165.png}
    \end{figure}
    \ifprintanswers
    \begin{figure}[H]
        \centering
        \includegraphics[width=0.4\textwidth]{images/image166.png}
    \end{figure}
    \fi
    \begin{solution}
        圆心之间的距离等于两个相切圆半径之和。设圆心 $X,Y,Z$ 的半径分别为 $x,y,z$,则有:
        \[
        XY = x + y = 30, \quad XZ = x + z = 40, \quad YZ = y + z = 20.
        \]
        解得
        \[
        x=25,\;y=5,\;z=15
        \]
        矩形高度为$2x = 50$,宽度为
        \[
        SR = SU + UV + VR = 25 + \sqrt{40^2 - 10^2} + 15 = 40 + 10\sqrt{15}
        \]
        故矩形面积为
        \[
        [PQRS]=50(40 + 10\sqrt{15}) = 2000 + 500\sqrt{15}
        \]
    \end{solution}

    \question 两个圆 $C_1$ 和 $C_2$ 外切,直线 $l$ 为它们的公切线。直线 $m$ 平行于 $l$ 并与 $C_1$ 和 $C_3$ 相切。三个圆互相切。已知 $C_2$ 的半径为 $9$,$C_3$ 的半径为 $4$,求 $C_1$ 的半径。
    \begin{figure}[H]
        \centering
        \includegraphics[width=0.4\textwidth]{images/image201.png}
    \end{figure}
    \ifprintanswers
    \begin{figure}[H]
        \centering
        \includegraphics[width=0.4\textwidth]{images/image202.png}
    \end{figure}
    \fi
    \begin{solution}
        设 $C_1$ 的半径为 $r$,有
        \[
        C_1C_2 = r+9, \quad C_1C_3 = r+4, \quad C_2C_3 = 13.
        \]
        如图所示,考虑矩形$ABC_2D$,由毕氏定理,在 $\triangle AC_3C_1$ 中,
        \[
        C_3A^2 = (r+4)^2-(r-4)^2 = 16r \Rightarrow C_3A = 4\sqrt{r}.
        \]
        在 $\triangle DC_2C_1$ 中,
        \[
        DC_2^2 = (r+9)^2-(r-9)^2 = 36r \Rightarrow DC_2 = 6\sqrt{r}
        \]
        在 $\triangle BC_3C_2$ 中,
        \[
        CB^2 = 13^2-(2r-13)^2 = -4r^2 + 52r \Rightarrow C_3B = \sqrt{-4r^2 + 52r}
        \]
        由$DC_2 = C_3A + C_3B$得
        \[
        6\sqrt{r} = 4\sqrt{r} + \sqrt{-4r^2+52r} 
        \]
        解得
        \[
        r=12>0
        \]
    \end{solution}

    \question 已知$\triangle ABC$中,$\;AB=13,BC=14,AC=15$,设 $D,E$ 分别为从 $A,B$ 作的高的垂足,求 $\triangle CDE$ 的外接圆直径。
    \begin{solution}
        在$\triangle ABC$中,由余弦定理,
        \[
        13^2 = 14^2 + 15^2 - 2\cdot 14 \cdot 15 \cos C \Rightarrow \cos C = \frac{3}{5}
        \]
        设 $AD$ 与 $BE$ 交于 $X$,由于 $\angle CDX = \angle CEX = 90^\circ$,有 
        \[
        \triangle CDA \backsim \triangle CEB \backsim \triangle XDB \ \text{(AAA)},
        \]
        得到
        \[
        CD = 9, \quad BD = 5, \quad DX = \frac{15}{4}
        \]
        在$\triangle CDX$中,由毕氏定理,
        \[
        CX = \sqrt{CD^2+DX^2}=\frac{39}{4}
        \]
        发现 $CDXE$ 是一圆内接四边形,因此 $\triangle CDE$ 的外接圆直径即为
        \[
        CX=\frac{39}{4}
        \]
    \end{solution}

    \question 已知点 $A,B,C,D$共圆, $AC$ 是直径,且 $\angle CBD=\angle DBA$。若 $BC=2, AB=4$,求 $BD$ 的长度。
    \begin{solution}
        $AC$ 是直径,所以 $\angle ABC=90^\circ$,因此 
        \[
        \angle CBD=\angle DBA=45^\circ
        \]
        又 $\angle CAD$ 与 $\angle CBD$ 为同弧所对的角,所以 
        \[
        \angle CAD=\angle CBD=45^\circ
        \]
        在$\triangle ABC$中,由毕氏定理,
        \[
        AC=\sqrt{AB^2+BC^2}=2\sqrt{5}
        \]
        故
        \[
        \sin \angle BAC=\frac{BC}{AC}=\frac{1}{\sqrt{5}}
        \]
        在$\triangle BAD$中,由正弦定理,
        \begin{align*}
        BD=AC \sin \angle BAD &= AC \sin \left(\angle BAC+\frac{\pi}{4}\right) \\
        &= 2\sqrt{5}\cdot \frac{\sqrt{2}}{2}(\sin\angle BAC+\cos\angle BAC)=3\sqrt{2}            
        \end{align*}
    \end{solution}

    \question 在 $\triangle ABC$ 中,已知 $AB=9,BC=10,CA=11$,且内切圆切 $BC,AB,AC$ 于 $D,E,F$,求 $AD$。
    \ifprintanswers
    \begin{figure}[H]
        \centering
        \includegraphics[width=0.4\textwidth]{images/image124.jpg}
    \end{figure}
    \fi
    \begin{solution}
        设$AE=AF=a, BD=BE=b, CF=CD=c,$则
        \[
        c = s - AB = \frac{9+10+11}{2} - 9 = 6
        \]
        在 $\triangle CAD$ 与 $\triangle CAB$中,由余弦定理,
        \[
        \cos C 
        = \frac{11^2 + 6^2 - AD^2}{2\cdot 11 \cdot 6}
        = \frac{11^2 + 10^2 - 9^2}{2\cdot 11 \cdot 10}
        \]
        解得
        \[
        AD = \sqrt{73}
        \]
    \end{solution}

    \question 在 $\triangle ABC$ 中,$AB=10$, $M$ 为 $AB$ 中点,$\;\triangle ABC$ 内切圆恰将线段 $CM$ 三等分,求 $\triangle ABC$ 面积。
    \ifprintanswers
    \begin{figure}[H]
        \centering
        \includegraphics[width=0.6\textwidth]{images/image110.jpg}
    \end{figure}
    \fi
    \begin{solution}
        设三切点为 $D,E,F$,中线 $CM$ 与内切圆交于 $P,Q$ 两点,据题意有
        \[
        CP = PQ = QA = a.
        \]
        由$\triangle MQD \backsim \triangle MDP \ (\text{AAA})$,
        \[
        \frac{MQ}{MD} = \frac{MD}{MP} \Rightarrow  MD^2 = MP \cdot MQ = 2a^2,
        \]
        同理由$\triangle CPF \backsim \triangle CFQ \ (\text{AAA})$得$CF^2 = 2a^2$,则
        \[
        CE = CF = MD = b  \Rightarrow b = \sqrt{2}a.
        \]
        由中线定理,
        \[
        CA^2 + CB^2 = 2(CM^2 + AM^2) \Rightarrow  (5+2b)^2 + 5^2 = 2(9a^2 + 5^2)
        \]
        代入 $b = \sqrt{2}a$ 解得
        \[
        a = 2\sqrt{2}, \quad b = 4.
        \]
        因此三边长为$AB = 10, BC = 5,  AC = 5 + 2b = 13$,半周长 $s = \dfrac{10+5+13}{2} = 14$,面积为
        \[
        \sqrt{14 \cdot 4 \cdot 9 \cdot 1} = 6\sqrt{14}
        \]
    \end{solution}

    \question 已知$PQRS$ 是边长为 $4$ 的正方形,点 $T$ 在 $QR$ 上,点 $U$ 在 $RS$ 上,且 $\angle UPT = 45^\circ$,求 $\triangle RUT$ 的最大周长。
    \ifprintanswers
    \begin{figure}[H]
        \centering
        \includegraphics[width=0.3\textwidth]{images/image200.png}
    \end{figure}
    \fi
    \begin{solution}
        将 $\triangle PSU$ 绕点 $P$ 逆时针旋转 $90^\circ$,得到全等的 $\triangle PQV$,此时 $V$ 位于 $RQ$ 的延长线上,且有
        \[
        \triangle PTU \congr \triangle PTV \ \text{(SAS)}
        \]
        于是$\triangle RUT$ 的周长
        \[
        UR + RT + UT = UR + RT + TV = UR + RQ + SU = SR + RQ = 8
        \]
        故最大周长为 $8$。
    \end{solution}
    \begin{solution}
        设 $\angle SPU = \theta,0^\circ < \theta < 45^\circ$,则
        \[
        SU = 4 \tan \theta, \quad UR = SR - SU = 4 - 4\tan \theta.
        \]
        由 $\angle UPT = 45^\circ$ 可得 $\angle QPT = 45^\circ - \theta$,于是
        \[
        QT = PQ \tan(45^\circ - \theta) = 4 \cdot \frac{1 - \tan \theta}{1 + \tan \theta}, \quad RT = QR - QT = \frac{8 \tan \theta}{1 + \tan \theta}.
        \]
        由毕氏定理,
        \[
        UT = \sqrt{UR^2 + RT^2} = 4 \cdot \frac{1 + \tan^2 \theta}{1 + \tan \theta}
        \]
        因此,$\triangle RUT$ 的周长为
        \[
        UR + RT + UT = 4 - 4 \tan \theta + \frac{8 \tan \theta}{1 + \tan \theta} + 4 \cdot \frac{1 + \tan^2 \theta}{1 + \tan \theta} 
        = 4 \frac{2 + 2 \tan \theta}{1 + \tan \theta} = 8
        \]
        无论 $\theta$ 取何值周长恒为 $8$,故最大可能为 $8$。
    \end{solution}

    \question 如图,$PRTY$ 和 $WRSU$ 是正方形,点 $Q,X$ 在 $PR,TY$ 上使得 $PQXY$ 是矩形,点 $T,W$ 在 $SU,QX$ 上,$UW$ 与 $TY$ 交于$V$。若矩形 $PQXY$ 面积为 $30$,求 $ST$ 的长度。
    \begin{figure}[H]
        \centering        
        \includegraphics[width=0.35\textwidth]{images/image175.png}
    \end{figure}
    \begin{solution}
        设 $ST = a,\angle STR = \theta$。则
        \[
        \angle TRS = 90^\circ - \theta ,\quad \angle QWR = \theta
        \]
        由 $\triangle RST$,
        \[
        PY = PR = RT = \frac{a}{\cos \theta}, \quad RW = RS = a \tan \theta
        \]
        且由 $\triangle QRW$,
        \[
        QR = RW \sin \theta = a \tan \theta \sin \theta, \quad
        PQ = PR - QR = \frac{a}{\cos \theta} - a \tan \theta \sin \theta = a \cos \theta
        \]
        矩形面积为 $PQ \cdot PY = 30$,即
        \[
        a \cos \theta \cdot \frac{a}{\cos \theta} = a^2 = 30 \Rightarrow a = \sqrt{30} 
        \]
    \end{solution}

    \question 如图,在直角 $\triangle ABC$ 中,$\angle ACB = 90^\circ, AC=6$,以 $AB$ 为一边向三角形外作正方形 $ABEF$。已知正方形的中心为 $O$,且 $OC = 8\sqrt{2}$,求 $BC$ 的长度。
    \begin{figure}[H]
        \centering        
        \includegraphics[width=0.35\textwidth]{images/image134.png}
    \end{figure}
    \ifprintanswers
    \begin{figure}[H]
        \centering        
        \includegraphics[width=0.5\textwidth]{images/image135.jpg}
    \end{figure}
    \fi
    \begin{solution}
        $\triangle OAB$ 为等腰直角三角形,设$G$ 为 $AB$ 中点,则 $OG$ 为中垂线,设$$GA = GB = OG = a,$$同弧所对的圆周角相等,有$\angle ACO = \angle ABO = 45^\circ$,由余弦定理,
        \[
        (\sqrt{2} a)^2=6^2 + (8\sqrt{2})^2-2 \cdot 6 \cdot 8\sqrt{2}\cdot \cos \angle 45^\circ
        \Rightarrow a^2 = 34
        \]
        在直角 $\triangle ABC$ 中,由毕氏定理,
        \[
        (2a)^2 = 6^2 + CB^2 \Rightarrow CB = 10
        \]
    \end{solution}

    \question 在直角$\triangle ABC$中,$D,E,F$分别为三边的中点,$G$为重心,且
    \[
    GD^2+GE^2+GF^2=\frac{50}{3}
    \]
    求斜边$AC$的长度。
    \begin{solution}
        由于$G$为重心,
        \[
        \frac{50}{3}=GD^2+GE^2+GF^2
        =\left(\frac13 BD\right)^2+\left(\frac13 CE\right)^2+\left(\frac13 AF\right)^2
        \]
        从而
        \[
        BD^2+CE^2+AF^2=150
        \]
        在直角$\triangle ABC$中,由毕氏定理,
        \[
        BD=\frac12 AC, \quad CE^2=BC^2+\left(\frac12 AB\right)^2, \quad AF^2=AB^2+\left(\frac12 BC\right)^2
        \]
        于是
        \[
        BD^2+CE^2+AF^2
        =\frac14 AC^2+\frac14 AB^2+BC^2+AB^2+\frac14 BC^2
        \]
        又由毕氏定理$AB^2+BC^2=AC^2$,
        \[
        150=BD^2+CE^2+AF^2=\frac14 AC^2+\frac14 AC^2+AC^2=\frac32 AC^2 \Rightarrow AC=10
        \]
    \end{solution}
    \begin{solution}
        由重心的性质可得
        \[
        GA^2+GB^2+GC^2=\frac13(AB^2+BC^2+AC^2)
        \]
        又$D,E,F$分别为三边的中点,$G$为重心,有
        \[
        GD^2+GE^2+GF^2=\frac14(GA^2+GB^2+GC^2)=\frac1{12}(AB^2+BC^2+AC^2)
        \]
        且由毕氏定理$AB^2+BC^2=AC^2$,因此
        \[
        GD^2+GE^2+GF^2=\frac16 AC^2=\frac{50}{3}
        \]
        解得
        \[
        AC=10
        \]
    \end{solution}

    \question 已知 $\triangle ABC$ 满足 $AB=9,BC=10,CA=17,B'$ 为 $B$ 关于 $AC$ 的对称点,$G,G'$为 $\triangle ABC,\triangle AB'C$的重心,求 $GG'$ 的长度。
    \ifprintanswers
    \begin{figure}[H]
        \centering        
        \includegraphics[width=0.5\textwidth]{images/image158.png}
    \end{figure}
    \fi
    \begin{solution}
        设 $M$ 为 $AC$ 的中点,由$GMG' \backsim BMB' \ \text{(SAS)}$,可得
        \[
        GG' = \frac{1}{3}BB'
        \]
        三角形面积为
        \[
        [\triangle ABC] = \frac{1}{4} BB' \cdot AC =\sqrt{18(18-9)(18-10)(18-17)} = 36 \Rightarrow BB' = \frac{144}{17}
        \]
        于是
        \[
        GG' = \frac{48}{17}
        \]
    \end{solution}

    \question 已知 $\triangle ABC$ 满足 $AB=4,BC=5, CA=6,\triangle ABC$ 的外接圆同时是 $\triangle A'B'C'$ 的内切圆,且 $A,B,C$ 分别在 $B'C',A'C',A'B'$ 上,求 $B'C'$ 的长度。
    \begin{figure}[H]
    \centering        
    \includegraphics[width=0.5\textwidth]{images/image159.png}
    \end{figure}
    \begin{solution}
        由弦切角定理,
        \[
        \angle C'AB = \angle C, \quad \angle B'AC = \angle B
        \]
        由直角三角形关系得
        \[
        C'A = \frac{2}{\cos\angle C'AB} = \frac{2}{\cos C}, \quad B'A = \frac{3}{\cos\angle B'AC} = \frac{3}{\cos B}
        \]
        由余弦定理,得
        \[
        \cos B = \frac{4^2+5^2-6^2}{2\cdot 4\cdot 5}=\frac{1}{8}, \quad \cos C = \frac{5^2+6+2-4^2}{2 \cdot 5 \cdot 6}=\frac{3}{4}
        \] 
        因此
        \[
        B'C' = C'A + B'A = 2\cdot \frac{4}{3} + 3\cdot 8 = \frac{80}{3}
        \]
    \end{solution}

    \question 在 $\triangle ABC$ 中,点 $D, E$ 分别在 $AB, AC$ 上,且 $AD=3, DB=1, AE=2, EC=4$。$BE$ 和 $CD$ 相交于 $P$ 点,若 $AP \perp BC$,求 $\cos \angle BAC$。
    \ifprintanswers
    \begin{figure}[H]
        \centering        
        \includegraphics[width=0.4\textwidth]{images/image128.jpg}
    \end{figure}
    \fi
    \begin{solution}
        设 $Q$ 为 $BC$ 上的垂足,由梅涅劳斯定理,
        \[
        \frac{CE}{EA} \cdot \frac{AD}{DB} \cdot \frac{BQ}{QC}= \frac{4}{2} \cdot \frac{3}{1} \cdot \frac{BQ}{QC} = 1 \Rightarrow \frac{BQ}{QC} = \frac{1}{6}
        \]
        设 $BQ = a,QC = 6a$,在直角$\triangle AQB$ 及 $\triangle AQC$中,由毕氏定理,
        \[
        QA^2 = 16 - a^2 = 36 - (6a)^2 \Rightarrow a^2 = \frac{4}{7}
        \]
        在$\triangle ABC$ 中,由余弦定理,
        \[
        \cos \angle BAC = \frac{4^2 + 6^2 - (7a)^2}{2 \cdot 4 \cdot 6} = \frac{1}{2}
        \]
    \end{solution}

    \question 如下图,三角形 $ABC$ 中,三线段 $AD,BE,CF$ 交于一点 $O$,若 $OD=OE=OF=4$ 且 $OA+OB+OC=37$,求 $OA \cdot OB \cdot OC$ 的值。 
    \begin{figure}[H]
        \centering        
        \includegraphics[width=0.4\textwidth]{images/image82.png}
    \end{figure}
    \begin{solution}
        由等高性质,
        \[
        \frac{S_{\triangle OBC}}{[\triangle ABC]}+ \frac{S_{\triangle OAC}}{[\triangle ABC]}+ \frac{S_{\triangle OAB}}{[\triangle ABC]}= \frac{OD}{AD} +\frac{OE}{BE} +\frac{OF}{CF} =1
        \]
        即
        \[
        \frac{4}{a}+\frac{4}{b}+ \frac{4}{c}=1
        \]
        其中 $a=AD=OA+4$, $b=BE=OB+4$, $c=CF=OC+4$,令
        \[
        ab+bc+ca=t,abc=4(ab+bc+ca)=4t,a+b+c=OA+OB+OC+12=49
        \]
        则设三次多项式 $f(x)$三根为 $a,b,c$,则
        \[
        f(x)=(x-a)(x-b)(x-c)=x^3-49x^2+tx-4t
        \]
        令 $x=4$ 得
        \[
        -(OA \cdot OB \cdot OC)=(4-a)(4-b)(4-c)=64-784+4t-4t=-720
        \]
        即
        \[
        OA \cdot OB \cdot OC=720
        \]
    \end{solution}

    \question 已知 $\triangle ABC$ 为等边三角形,$D,E,F$在 $BC,CA,AB$上使得$AF=BD=CE=\dfrac{1}{3}AB$。若$AD,BE,CF$分别交$CF,AD,BE$于$G,H,I$,求 
    \[
    \frac{[\triangle GHI]}{[\triangle ABC]}
    \]
    \ifprintanswers
    \begin{figure}[H]
        \centering        
        \includegraphics[width=0.4\textwidth]{images/image142.png}
    \end{figure}
    \fi
    \begin{solution}
        设
        \[
        \alpha = [\triangle AFG]=[\triangle BDH]=[\triangle CEI], \beta = [AEIG]=[BHGF]=[CIHD], \gamma = [\triangle GHI]
        \]
        由等高性质,得
        \[
        2(2\alpha+\beta) = 2\beta + \alpha + \gamma \Rightarrow \gamma = 3\alpha
        \]
        在$\triangle ACF$中,由余弦定理,
        \[
        CF^2 = AF^2 + (3AF)^2 - 2 \cdot AF \cdot (3AF)\cos 60^\circ = 7 AF^2.
        \]
        由于 $\triangle ACF \backsim \triangle ICE \ \text{(AAA)}$,
        \[
        \frac{[\triangle ACF]}{[\triangle ICE]}=\frac{2\alpha + \beta}{\alpha} = \frac{CF^2}{AF^2} = 7 \Rightarrow \beta = 5\alpha
        \]
        因此
        \[
        \frac{[\triangle GHI]}{[\triangle ABC]} = \frac{\gamma}{\gamma + 3\alpha + 3\beta} = \frac{1}{7}
        \]
    \end{solution}

    \question $\triangle ABC$ 中, $AB=6, BC=8, CA=12, D, E$ 在 $CA$ 上, 且 $AD = DE = EC,$ 求 $BD,BE$的长。
    \ifprintanswers
    \begin{figure}[H]
        \centering        
        \includegraphics[width=0.55\textwidth]{images/image89.jpg}
    \end{figure}
    \fi
    \begin{solution}
        令 $BD = a, BE = b,$由余弦定理,
        \[
        \cos \angle BDA = \frac{a^2 + 16 - 36}{8a}, \quad \cos \angle BDE = \frac{a^2 + 16 - b^2}{8a}
        \]
        由 $\cos \angle BDA = -\cos \angle BDE$得
        \[
        2a^2 - b^2 = 4 \tag{1}
        \]
        同理,由余弦定理,
        \[
        \cos \angle BED = \frac{b^2 + 16 - a^2}{8b}, \quad \cos \angle BEC = \frac{b^2 + 16 - 64}{8b}
        \]
        由 $\cos \angle BED = -\cos \angle BEC$可得
        \[
        a^2 - 2b^2 = -32 \tag{2}
        \]
        联立解得 
        \[
        a = \frac{2\sqrt{30}}{3},b=\frac{2\sqrt{17}}{3}
        \]
    \end{solution}

    \question 在 $\triangle ABC$ 中,$D,E$ 在 $AB$ 上使得 $AD:DE:EB=1:2:1,M$在$BC$上使得 $AM$为中线,$CD,CE$交$AM$于$G,H$,求 $AG:GH:HM$。
    \ifprintanswers
    \begin{figure}[H]
        \centering        
        \includegraphics[width=0.4\textwidth]{images/image150.png}
    \end{figure}
    \fi
    \begin{solution}
        设 $AG:GH:HM = x:y:z$,在$\triangle ABM$中,以 $CE$ 为横截线,由梅涅劳斯定理,
        \[
        \frac{AH}{HM} \cdot \frac{MC}{CB} \cdot \frac{BE}{EA} =
        \frac{x+y}{z} \cdot \frac{1}{2} \cdot \frac{1}{3} = 1 \Rightarrow x+y = 6z
        \]
        同理,以 $CD$ 为横截线,由梅涅劳斯定理,
        \[
        \frac{AG}{GM} \cdot \frac{MC}{CB} \cdot \frac{BD}{DA} =\frac{x}{y+z} \cdot \frac{1}{2} \cdot \frac{3}{1} = 1 \Rightarrow x = \frac{2}{3}(y+z)
        \]
        设 $z=1$,解得 $x=\dfrac{14}{5},\ y=\dfrac{16}{5}$,于是
        \[
        AG:GH:HM = 14:16:5
        \]
    \end{solution}

    \question $\triangle ABC$ 满足 $AB=20,AC=12$, 若在 $BC$ 边上取两点 $P$ 和 $Q$, 使得 $AQ$ 为 $\angle BAC$ 的角平分线且 $BP=QC$, 求 $\sqrt{AP^2-AQ^2}$ 的值。
    \ifprintanswers
    \begin{figure}[H]
        \centering        
        \includegraphics[width=0.5\textwidth]{images/image88.jpg}
    \end{figure}
    \fi
    \begin{solution}
        令 $BP = CQ = a, PQ = b$,在 $AB$ 上取点 $R$使得 $AR = AC = 12$, 则 
        \[
        \triangle AQR \congr \triangle AQC \; (SSS) \Rightarrow QR = QC = a
        \]
        又 $AQ$ 为 $\angle A$ 的角平分线, 因此 
        \[
        \frac{20}{12} = \frac{a+b}{a} \Rightarrow a+b = \frac{5}{3}a
        \]
        在$\triangle BQR$及$\triangle BAP$中,由余弦定理,
        \[
        \cos \angle B = \frac{(\frac{5a}{3})^2 + 8^2 - a^2}{2\cdot8 \cdot \frac{5a}{3}} = \frac{20^2 + a^2 - AP^2}{2\cdot 20 \cdot a}
        \]
        且在$\triangle BAQ$中,由余弦定理,
        \[
        \cos \angle B = \frac{20^2 + (\frac{5a}{3})^2 - AQ^2}{2\cdot 20 \cdot \frac{5a}{3}} 
        \]
        解得
        \[
        h^2 = 240 - \frac{5}{3}a^2, w^2 = 304 - \frac{5}{3}a^2 \Rightarrow \sqrt{w^2 - h^2} = 8
        \]
    \end{solution}

    \question 如图,$\triangle ABC$ 中,$\angle C=90^\circ$ 且 $AD = DE = EB$。已知 $\angle ACD = \alpha,\angle DCE = \beta,\angle ECB = \gamma$,求 
    \[
    \frac{\sin \alpha \cdot \sin \gamma}{\sin \beta}.
    \]
    \begin{figure}[H]
        \centering
        \includegraphics[width=0.4\linewidth]{images/image137.png}
    \end{figure}
    \ifprintanswers
    \begin{figure}[H]
        \centering
        \includegraphics[width=0.6\linewidth]{images/image138.jpg}
    \end{figure}
    \fi
    \begin{solution}
        过 $A$ 作水平线且过 $B$ 作垂直线,两线交于 $F$,则 $ACBF$ 为矩形。延长 $CE$ 交 $BF$ 于 $H$,延长 $CD$ 交 $AF$ 于 $G$,则有
        \[
        BH = HF,\quad AG = GF.
        \]
        设$BH = HF = a, \quad AG = GF = b$,
        在 $\triangle CAG$及 $\triangle CBH$中,由毕氏定理,
        \[
        \sin \alpha = \frac{b}{\sqrt{4a^2+b^2}}, \quad \sin \gamma = \frac{a}{\sqrt{a^2+4b^2}}.
        \]
        在 $\triangle CGH$ 中,由余弦定理,
        \[
        \cos \beta 
        = \frac{(4a^2+b^2)+(a^2+4b^2)-(a^2+b^2)}{2\sqrt{(4a^2+b^2)(a^2+4b^2)}}
        = \frac{2(a^2+b^2)}{\sqrt{(4a^2+b^2)(a^2+4b^2)}},
        \]
        于是
        \[
        \sin \beta = \sqrt{1-\cos^2 \beta} = \frac{3ab}{\sqrt{(4a^2+b^2)(a^2+4b^2)}}
        \]
        因此
        \[
        \frac{\sin \alpha \cdot \sin \gamma}{\sin \beta}= \frac{1}{3}
        \]
    \end{solution}

    \question 如下图,直角三角形 $ABC$ 中,$\angle BAC=90^\circ$,点 $D, E, F$ 分别为 $BC, CA, AB$ 之中点,$BE$ 与 $CF$ 交于点 $G,BC=18,\angle BGC=150^\circ$,求 $\triangle GBC$ 的面积。
    \begin{figure}[H]
        \centering
        \includegraphics[width=0.5\linewidth]{images/image74.png}
    \end{figure}
    \begin{solution}
        设$AF = FB = a,AE = EC = b,$则
        \[
        a^2 + b^2 = \frac{1}{4}\cdot 18^2=81
        \]
        设直线$BE,CF$斜率为
        \[
        m_1 = -\frac{b}{2a},\quad
        m_2 = -\frac{2b}{a}
        \]
        由
        \[
        \tan 150^\circ = \frac{m_2 - m_1}{1 + m_1 m_2}\Rightarrow
        -\frac{1}{\sqrt{3}} = \frac{-\frac{2b}{a} + \frac{b}{2a}}{1 + \left(-\frac{b}{2a}\right)\left(-\frac{2b}{a}\right)}
        = -\frac{3ab}{2(a^2 + b^2)}
        \]
        代入 $a^2 + b^2 = 81$, 有
        \[
        ab = 18 \sqrt{3}
        \]
        故$\triangle ABC$面积为
        \[
        \frac{1}{2}\cdot 2a \cdot 2b = 36 \sqrt{3}
        \]
        发现点$G$是重心,所以
        \[
        [\triangle GBC] = \frac{1}{3} [\triangle ABC]= 12 \sqrt{3}
        \]
    \end{solution}

    \question 在 $\triangle ABC$中,$\angle C=90^{\circ},G$为$\triangle ABC$的重心,且$G$到$BC,CA$的距离和为$6$。若 $AB=15$,试求 $\triangle ABC$ 的内切圆面积。
    \ifprintanswers
    \begin{figure}[H]
        \centering
        \includegraphics[width=0.5\linewidth]{images/image122.jpg}
    \end{figure}
    \fi
    \begin{solution}
        设$G$在$AC,BC$上的垂足分别为$H,I$,且$GI=a,GH=6-a$,由毕氏定理,
        \[
        GC^2 = a^2+(6-a)^2
        \]
        又$G$为重心,则 
        \[
        GC=\frac{2}{3}CD=\frac{2}{3}\cdot\frac{1}{2}AB=5
        \]
        解得
        \[
        a= \frac{6+\sqrt{14}}{2} \Rightarrow 6-a= \frac{6- \sqrt{14}}{2}
        \]
        又$\triangle AGH \backsim \triangle AFC \ \text{(AAA)}$,
        \[ 
        \frac{2}{3}=\frac{6-a}{FC}\Rightarrow FC= \frac{3}{2}(6-a) \Rightarrow BC =3(6-a)=\frac{18- 3\sqrt{14}}{2}
        \]
        同理,
        \[
        AC= 3a = \frac{18+3\sqrt{14}}{2}
        \]  
        所以 $\triangle ABC$面积为
        \[
        [\triangle ABC]=\frac{1}{2}\cdot \frac{18- 3\sqrt{14}}{2} \cdot \frac{18+ 3\sqrt{14}}{2}
        =\frac{1}{2}r\left(15+\frac{18- 3\sqrt{14}}{2} + \frac{18+ 3\sqrt{14}}{2}\right)  
        \]
        得$r=\dfrac{3}{2}$,故内切圆面积为$\dfrac{9\pi}{4}$。
    \end{solution}

    \question 已知$\triangle ABC$ 满足 $AB=AC=3,BC=1$,点 $D$ 在 $AB$ 上使得 $\triangle ACD$ 与 $\triangle BCD$ 的内切圆半径相等,求该内切圆半径。
    \begin{solution}
        设 $BD = x,CD = y$,三角形面积为内切圆半径乘半周长,于是
        \[
        \frac{[\triangle ACD]}{[\triangle BCD]} = \frac{s_{ACD}}{s_{BCD}} = \frac{6 - x + y}{y + 1 + x}.
        \]
        由等高性质,
        \[
        \frac{[\triangle ACD]}{[\triangle BCD]} = \frac{AD}{BD} = \frac{3-x}{x}.
        \]
        联立两式可得
        \[
        y = \frac{4x - 3}{3 - 2x} \tag{1}
        \]
        在$\triangle BCD$ 中,由余弦定理,
        \[
        y^2 = x^2 + 1^2 - 2 \cdot x \cdot 1 \cdot \frac{1}{6} = x^2 + 1 - \frac{1}{3}x \tag{2}
        \]
        联立$(1),(2)$得
        \[
        x(x-3)(12x^2 - 4x - 9) = 0 \Rightarrow x = \frac{1 + 2\sqrt{7}}{6}, y = \frac{\sqrt{7}}{2}.
        \]
        故 $\triangle ABC$ 面积为
        \[
        [\triangle ABC] = \frac{1}{2} r \cdot \left( \frac{6 - x + y}{2} + \frac{y + 1 + x}{2} \right) = \frac{1}{2} \sqrt{3^2 - \left(\frac{1}{2}\right)^2} = \frac{\sqrt{35}}{4}
        \]
        解得
        \[
        r = \frac{\sqrt{35} - \sqrt{5}}{12}
        \]
    \end{solution}

    \question $\triangle ABC$ 中, $AB=\sqrt{3},\ AC=3\sqrt{3},\ BC=\sqrt{21}$, 由各边分别向外作正三角形。已知此三个向外所作正三角形的重心会形成另一个新的正三角形, 求此新的三角形的面积。 
    \ifprintanswers
    \begin{figure}[H]
        \centering
        \includegraphics[width=0.5\linewidth]{images/image98.jpg}
    \end{figure}
    \fi
    \begin{solution}
        在$\triangle ABC$ 中,由余弦定理,
        \[
        (\sqrt{21})^2=(\sqrt 3)^2 + (3\sqrt 3)^2-2 \cdot \sqrt 3 \cdot 3\sqrt 3 \cdot \cos \angle CAB
        \Rightarrow \angle CAB = 60^\circ
        \]
        由正三角形$\triangle ACD$及$\triangle ABE$,
        \[
        AG_1=1,AG_3=3,\ \angle G_1AB=\angle G_3AC=30^\circ,
        \]
        在$\triangle AG_1G_3$ 中,由余弦定理,
        \[
        (G_1G_3)^2 = 1^2 + 3^2 - 2 \cdot 1 \cdot 3 \cdot \cos \angle G_1AG_3 \Rightarrow G_1G_3 = \sqrt{13}
        \]
        故
        \[
        [\triangle G_1G_2G_3]=\frac{13}{4}\sqrt 3
        \]
    \end{solution}

    \question 已知等腰三角形 $ABC$ 的腰长为 $AB=AC=13$,底边为 $BC=10$。设三角形的内心为 $I$,外心为 $O$,求 $OI$ 的长度。
    \ifprintanswers
    \begin{figure}[H]
        \centering
        \includegraphics[width=0.4\linewidth]{images/image133.jpg}
    \end{figure}
    \fi
    \begin{solution}
        设 $D$ 为$BC$ 中点,则 $AD \perp BC$,且 $BD = DC = 5$,在直角 $\triangle ADC$ 中,由毕氏定理,
        \[
        AD^2 + 5^2 = 13^2 \Rightarrow AD = 12.
        \]
        设 $G$ 为$AC$上的垂足,则
        \[
        \frac12 \cdot 12 \cdot 10 = \frac12 \cdot 13 \cdot DG \Rightarrow DG = \frac{60}{13}
        \]
        在直角 $\triangle ADG$ 中,由毕氏定理,
        \[
        12^2 = AG^2 + \left(\frac{60}{13}\right)^2 \Rightarrow AG = \frac{144}{13}
        \]
        由$\triangle AEO \backsim \triangle AGD \ (AAA)$,
        \[
        \frac{AO}{12} = \frac{\frac{13}{2}}{\frac{144}{13}} \Rightarrow AO = \frac{169}{24}
        \]
        且由$\triangle AFI \backsim \triangle AGD \ \text{(AAA)}$,
        \[
        \frac{ID}{\frac{60}{13}} = \frac{12-ID}{12} \Rightarrow ID = \frac{10}{3}
        \]
        所以
        \[
        OI = AD - AO - ID = 12 - \frac{169}{24} - \frac{10}{3} = \frac{13}{8}
        \]
    \end{solution}

%\question 已知$\triangle ABC$中, $\angle BAC=90^{\circ}$ , $AB=1$ , $AC=\sqrt{3}$ ,若$P$点在$\triangle ABC$内部且满足 $\frac{\vec{PA}}{|\vec{PA}|}+\frac{\vec{PB}}{|\vec{PB}|}+\frac{\vec{PC}}{|\vec{PC}|} = \vec{0}$ , 求序对$(PA,PB,PC)=$ \underline{\hspace{2cm}} . 
%\begin{solution}
% 因为 ${\vec u\over |\vec u|}$ 为单位向量,所以
% \[
% {\overrightarrow{PA} \over |\overrightarrow{PA}|} +{\overrightarrow{PB} \over |\overrightarrow{PB}|} +{\overrightarrow{PC} \over |\overrightarrow{PC}|} =\vec 0
% \]
% 可得
% \[
% \angle APB=\angle BPC=\angle APC = \tfrac{360^\circ}{3} =120^\circ .
% \]

% 令
% \[
% PA=a,\quad PB=b,\quad PC=c ,
% \]
% 则
% \[
% \cos \angle APB= \frac{a^2+b^2-1}{2ab},\quad 
% \cos \angle BPC= \frac{b^2+c^2-4}{2bc},\quad 
% \cos \angle APC= \frac{a^2+c^2-3}{2ac}.
% \]
% 由 $\cos 120^\circ=-\tfrac12$ 得
% \[
% \begin{cases}
% a^2+b^2+ab=1 \quad (1)\\
% b^2+c^2+bc=4 \quad (2)\\
% c^2+a^2+ca=3 \quad (3)
% \end{cases}
% \]

% 将(1)+(2)+(3)得
% \[
% 2(a^2+b^2+c^2)+ab+bc+ca=8 \quad (4)
% \]

% 另一方面,
% \[
% \triangle ABC 面积 = \triangle PAB+\triangle PBC+\triangle PCA
% \]
% \[
% = \tfrac12 ab\sin120^\circ + \tfrac12 bc\sin120^\circ + \tfrac12 ac\sin120^\circ
% = \tfrac{\sqrt{3}}{4}(ab+bc+ca).
% \]
% 而
% \[
% \triangle ABC 面积 = \tfrac12 AB\times AC = \tfrac{\sqrt{3}}{2}.
% \]
% 所以
% \[
% ab+bc+ca=2.
% \]

% 代入(4)得
% \[
% a^2+b^2+c^2=3.
% \]
% 因此
% \[
% (a+b+c)^2 = a^2+b^2+c^2+2(ab+bc+ca) = 3+4=7 \Rightarrow a+b+c=\sqrt{7}.
% \]

% 又由
% \[
% (1)+(2)\Rightarrow a^2+2b^2+c^2+ab+bc=5,
% \]
% \[
% (2)+(3)\Rightarrow a^2+b^2+2c^2+bc+ca=7,
% \]
% \[
% (1)+(3)\Rightarrow 2a^2+b^2+c^2+ab+ac=4,
% \]
% 可得
% \[
% \begin{cases}
% b^2+ab+bc=2 \\
% c^2+bc+ca=4 \\
% a^2+ab+ac=1
% \end{cases}
% \]

% 所以
% \[
% \begin{cases}
% b(a+b+c)=2 \\
% c(a+b+c)=4 \\
% a(a+b+c)=1
% \end{cases}
% \Rightarrow
% \begin{cases}
% a= \tfrac{1}{\sqrt{7}} \\
% b= \tfrac{2}{\sqrt{7}} \\
% c= \tfrac{4}{\sqrt{7}}
% \end{cases}
% \]

% 故答案为
% \[
% (PA,PB,PC)=\left( \tfrac{1}{\sqrt{7}}, \tfrac{2}{\sqrt{7}}, \tfrac{4}{\sqrt{7}} \right).
% \]
%\end{solution}

\question 在 $\triangle ABC$ 中,$AB=6,BC=4,CA=5$,圆 $O_1, O_2, O_3$ 为 $\triangle ABC$ 的三旁切圆,$O_1$ 与 $BC$ 相切于 $D$,$O_2$ 与 $CA$ 相切于 $E$,$O_3$ 与 $AB$ 相切于 $F$。求面积比 
\[
\frac{[\triangle DEF]}{[\triangle ABC]}.
\]
\begin{figure}[H]
    \centering
    \includegraphics[width=0.35\linewidth]{images/image108.png}
\end{figure}
\ifprintanswers
\begin{figure}[H]
    \centering
    \includegraphics[width=0.5\linewidth]{images/image109.jpg}
\end{figure}
\fi
\begin{solution}
    设$a = BC = 4, b = CA = 5, c = AB = 6, s = \dfrac{a+b+c}{2} = \dfrac{15}{2}$,由旁切圆性质,
    \[
    AF = CD = s - b = \frac{5}{2}, \quad
    BF = CE = s - a = \frac{7}{2}, \quad
    AE = BD = s - c = \frac{3}{2}
    \]
    由$\triangle ABC \backsim \triangle AEF \backsim \triangle BDF \backsim \triangle CDE$,
    \begin{align*}
    \frac{S_{\triangle DEF}}{[\triangle ABC]}
    &= 1- \frac{S_{\triangle AEF}}{[\triangle ABC]} - \frac{S_{\triangle BDF}}{[\triangle ABC]} - \frac{S_{\triangle CDE}}{[\triangle ABC]}\\[2mm]
    &= 1 - \left(
    \frac{AE \cdot AF}{AC \cdot AB} +
    \frac{BD \cdot BF}{BC \cdot AB} +
    \frac{CD \cdot CE}{AC \cdot BC}
    \right)\\[2mm]
    &= 1 - \left(
    \frac{\frac{3}{2} \cdot \frac{5}{2}}{5 \cdot 6} +
    \frac{\frac{3}{2} \cdot \frac{7}{2}}{4 \cdot 6} +
    \frac{\frac{5}{2} \cdot \frac{7}{2}}{5 \cdot 4}
    \right)= \frac{7}{32}
    \end{align*}
    \textcolor{red}{(待验证相似三角形,顺序,旁切圆性质)}
\end{solution}

    \question 设钝角三角形 $ABC$ 的内切圆半径为 $\sqrt{3}$,已知内切圆圆心 $O$ 到顶点 $A$ 的距离为 $2$,而 $O$ 到顶点 $B$ 的距离为 $2\sqrt{7}$,求 $\triangle ABC$ 的面积。
    \ifprintanswers
    \begin{figure}[H]
        \centering
        \includegraphics[width=0.3\linewidth]{images/image129.jpg}
    \end{figure}
    \fi
    \begin{solution}
        设 $D, E, F$ 分别为在$AB,BC,AC$上的内切圆切点,在$\triangle ADO,\triangle OBD$中,由毕氏定理,
        \[
        2^2 = (\sqrt{3})^2 + AD^2, (2\sqrt{7})^2 = (\sqrt{3})^2 + BD^2
        \]
        则
        \[
        AD = AF = 1, \quad BD = BE = 5, \quad CE = CF = a
        \]
        $\triangle ABC$半周长为$s=a+6$,则面积为
        \[
        [\triangle ABC]= \sqrt{3} (a + 6) =\sqrt{a+6\cdot a\cdot1\cdot5} \Rightarrow a = 9
        \]
        故
        \[
        [\triangle ABC]= 15 \sqrt{3}
        \]
    \end{solution}

    \question 在 $\triangle ABC$ 中,已知 $\cos\angle BAC = \dfrac{4\sqrt{3}}{7},\angle B = 30^\circ,O$ 为 $\triangle ABC$ 的内心,直线 $AO$ 与 $BC$ 相交于 $D$ 点,求 $\dfrac{BD}{CD}$。
    \ifprintanswers
    \begin{figure}[H]
        \centering
        \includegraphics[width=0.6\linewidth]{images/image116.jpg}
    \end{figure}
    \fi
    \begin{solution}
        设$\angle CAD = \angle DAB = \theta, CD = 1, BD = k$,  由 $\cos \angle A = \dfrac{4\sqrt{3}}{7}$,得 
        \[
        \sin \angle A = \dfrac{1}{7}
        \]
        在 $\triangle ABD$ 及$\triangle ACD$中,由正弦定理,
        \[
        \frac{k}{\sin \theta} = \frac{AD}{\sin 30^\circ}, \quad \frac{1}{\sin \theta} = \frac{AD}{\sin \angle C} 
        \]
        于是
        \[
        AD=\frac{k}{2\sin \theta} = \frac{\sin(150^\circ - \angle A)}{\sin \theta} \Rightarrow k = 2\sin(150^\circ - \angle A)
        \]
        即
        \[
        \frac{BD}{CD}=k = 2\sin(150^\circ - \angle A) = 2\left( \frac{1}{2} \cdot \frac{4\sqrt{3}}{7} - \left(-\frac{\sqrt{3}}{2}\right) \cdot \frac{1}{7} \right) = \frac{5\sqrt{3}}{7}
        \]
    \end{solution}

    \question 若等腰三角形$\triangle ABC$的外接圆半径 $R$ 与内切圆半径 $r$ 之比
    \[
    \frac{R}{r} = 1+\sqrt{2},
    \]
    且顶角$A$ 不为直角,求 $\sin\dfrac{A}{2}$。
    \ifprintanswers
    \begin{figure}[H]
        \centering
        \includegraphics[width=0.5\linewidth]{images/image126.jpg}
        \end{figure}
    \fi
    \begin{solution}
        设 $O$ 为外接圆圆心, $P$ 为内切圆圆心,$D,E$为$P$在$AC,AB$上的垂足,$G$为$AB$中点,则
        \[
        AB = 2R \cos \frac{A}{2}, \quad
        AF = 2R \cos^2 \frac{A}{2}, \quad
        OP = 2R \cos^2 \frac{A}{2} - R - r
        \]
        又由$\triangle AGO\backsim \triangle AEP \ (AAA)$,
        \[
        \frac{GO}{EP} = \frac{AO}{AP} \Rightarrow \frac{R \sin \frac{A}{2}}{r} = \frac{R}{2R\cos^2 \frac{A}{2} - r}
        \]
        令$x = \sin \dfrac{A}{2}$,将$\dfrac{R}{r} = 1+\sqrt{2}$代入得
        \[
        (x+1)\big((2+2\sqrt{2}) x^2 - (2+2\sqrt{2}) x + 1 \big) = 0
        \]
        舍去 $x=-1$,得
        \[
        \sin \frac{A}{2} = \frac{\sqrt{2}}{2} \quad \text{或} \quad \sin \frac{A}{2} = \frac{2-\sqrt{2}}{2}
        \]
    \end{solution}

    \question 已知四边形 $ABCD$ 中,$\; AB=BC=CD,\angle A=54^\circ,\angle B=108^\circ,\angle C$ 为钝角,求 $\angle C$。
    \ifprintanswers
    \begin{figure}[H]
        \centering
        \includegraphics[width=0.5\linewidth]{images/image92.jpg}
    \end{figure}
    \fi
    \begin{solution}
        作 $BD$,令 $\angle CBD=\angle CDB=\theta$, $AB=BC=CD=a$, 则
        \[
        \angle C=180^\circ-2\theta, \quad
        \angle ABD=108^\circ-\theta, \quad
        \angle ADB=18^\circ+\theta,  
        \] 
        在$\triangle ABD$及$\triangle BCD$中,由正弦定理,
        \[
        \frac{a}{\sin(18^\circ+\theta)}=\frac{BD}{\sin54^\circ},\quad
        \frac{a}{\sin\theta}=\frac{BD}{\sin(180^\circ-2\theta)}.
        \]
        故
        \[
        BD=\frac{\sin54^\circ}{\sin(18^\circ+\theta)}=\frac{\sin(180^\circ-2\theta)}{\sin\theta}= \frac{\sin2\theta}{\sin\theta}=2\cos\theta.
        \]
        于是
        \[
        \sin54^\circ=2\cos\theta\,\sin(18^\circ+\theta)=\sin18^\circ\cos2\theta+\cos18^\circ\sin2\theta+\sin18^\circ,
        \]
        即
        \[
        \sin(2\theta+18^\circ)=\sin54^\circ-\sin18^\circ=\frac{\sqrt5+1}{4}-\frac{\sqrt5-1}{4}=\frac{1}{2}
        \]
        故 
        \[
        2\theta+18^\circ=30^\circ \Rightarrow \angle C=168^\circ>90^\circ
        \]
    \end{solution}

    \question 已知边长为 $2$ 的正方形 $ABCD,E,F$ 分别为 $AD,CD$ 的中点,$AF$ 与 $EB,BD$ 相交于 $G,H$,求四边形 $DEGH$ 的面积。
    \ifprintanswers
    \begin{figure}[H]
        \centering
        \includegraphics[width=0.5\linewidth]{images/image145.png}
    \end{figure}
    \fi
    \begin{solution}
        首先有
        \[
        1=[\triangle ADF]=[\triangle ADH]+[\triangle DFH] =\frac{1}{2} \cdot DH \cdot 1 \cdot \sin 45^\circ +  \frac{1}{2} DH \cdot \frac{1}{2} \cdot \sin 45^\circ
        \]
        解得
        \[
        DH = \frac{2\sqrt{2}}{3}, \quad [\triangle DFH] = \frac{1}{3}
        \]
        设 $\alpha = \angle DAF$,由$\triangle ADF$得
        \[
        \sin \alpha = \frac{1}{\sqrt{5}}, \quad \sin(90^\circ-\alpha) = \frac{2}{\sqrt{5}}.
        \]
        同理有
        \[
        1=[\triangle ABE]=[\triangle ABG]+[\triangle AEG] 
        = \frac{1}{2} \cdot AG \cdot 1 \cdot \sin (90^\circ-\alpha) + \frac{1}{2} AG \cdot \frac{1}{2} \cdot \alpha
        \]
        解得
        \[
        AG = \frac{2\sqrt{5}}{5}, \quad [\triangle AEG] = \frac{1}{5}
        \]
        故四边形 $DEGH$ 的面积为
        \[
        [DEGH] = [\triangle ADF] - [\triangle DFH] - [\triangle AEG] = 1 - \frac{1}{3} - \frac{1}{5} = \frac{7}{15}
        \]
    \end{solution}

    \question 长方形 $ABCD$ 边长为 $2,3$,以 $A$ 为旋转中心逆时针旋转 $30^\circ$。求重叠部分面积。
    \begin{figure}[H]
        \centering
        \includegraphics[width=0.5\linewidth]{images/image67.png}
    \end{figure}
    \ifprintanswers
    \begin{figure}[H]
        \centering
        \includegraphics[width=0.5\linewidth]{images/image69.png}
    \end{figure}
    \fi
    \begin{solution}
        如图, $AB' = 3, \angle B'AB = 30^\circ$,则
        \[
        B'R = \frac{3}{2}, \quad AR = \frac{3\sqrt{3}}{2}
        \]
        又$B'Q = 2 - B'R = \dfrac{1}{2},\angle QB'C' = 30^\circ$,则
        \[ 
        QS = \frac{\sqrt{3}}{6} 
        \]
        重叠面积为
        \[
        [ABCD]- [B'PCS] - [ABPB'] =6 - \left(\frac{3}{2} - \frac{17\sqrt{3}}{24} \right) - \left(\frac{9}{2} - \frac{9\sqrt{3}}{8} \right) = \frac{11\sqrt{3}}{6} 
        \]
    \end{solution}

    \question 一边长为 $1$ 的正方形 $\Gamma_1$ 绕一顶点逆时针旋转 $\theta$ 得正方形 $\Gamma_2$,若两正方形重叠部分的面积为 $\dfrac{2}{3}$,求 $\tan\theta$。
    \ifprintanswers
    \begin{figure}[H]
        \centering
        \includegraphics[width=0.4\linewidth]{images/image72.png}
    \end{figure}
    \fi
    \begin{solution}
        令 $\angle AOP = \theta, \angle AOC = 90^\circ - \theta$,又 $\triangle OAB \congr \triangle OCB \ \text{(RHS)}$,  
        \[
        AB = OA \tan \angle AOB 
        = \tan\left( 45^\circ - \frac{\theta}{2} \right) 
        = \frac{1 - \tan\frac{\theta}{2}}{1 + \tan\frac{\theta}{2}}
        \]
        重叠面积为
        \[ 
        2[\triangle OAB] = OA \cdot AB = AB = \frac{1 - \tan\frac{\theta}{2}}{1 + \tan\frac{\theta}{2}} = \frac{2}{3}
        \]
        解得
        \[
        \tan\frac{\theta}{2} = \frac{1}{5} \Rightarrow \tan\theta = \frac{2\tan\frac{\theta}{2}}{1 - \tan^2\frac{\theta}{2}} = \frac{5}{12}
        \]
    \end{solution}

    \question $ABCD$为平行四边形,点$E,F$分别在边$AB,BC$上。已知
    \[
    [\triangle AED] = 7, \quad [\triangle EBF] = 3, \quad [\triangle CDF] = 6,
    \]
    求\(\triangle DEF\) 的面积。
    \ifprintanswers
    \begin{figure}[H]
        \centering
        \includegraphics[width=0.4\linewidth]{images/image66.png}
    \end{figure}
    \fi
    \begin{solution}
        不妨设$ABCD$为矩形,且设$AD = a, BE = b$,则
        \[
        AE = \frac{14}{a}, \quad BF = \frac{6}{b}, \quad CF = a - \frac{6}{b}, \quad CD = \frac{12}{a - \frac{6}{b}} = \frac{12b}{ab - 6}
        \]
        解 $AB = CD$得,
        \[
        b + \frac{14}{a} = \frac{12b}{ab - 6} \Rightarrow ab = 2 + 2\sqrt{22}
        \]
        \(\triangle DEF\) 的面积为
        \[
        [ABCD] - [\triangle AED] - [\triangle EBF] - [\triangle CDF] = (ab + 14) - 16 = 2\sqrt{22}
        \]
    \end{solution}
    \ifprintanswers
    \begin{figure}[H]
        \centering
        \includegraphics[width=0.5\linewidth]{images/image68.png}
    \end{figure}
    \fi
    \begin{solution}
        如果不假设为矩形,如上图,步骤相同,只是算出来的是 \(ab \sin \theta = 2 + 2\sqrt{22}\),结果是一样的。
    \end{solution}

    \question 如图,已知 $DE=10, EF=8, BF=6$,求正方形 $ABCD$ 面积。
    \begin{figure}[H]
        \centering
        \includegraphics[width=0.4\linewidth]{images/image60.png}
    \end{figure}
    \begin{solution}
        在$\triangle BFE$中,由毕氏定理,
        \[
        BE=\sqrt{6^2+8^2}=10
        \]
        设 $\angle BEF=\theta,\sin\theta=\dfrac{3}{5}$,则
        \[
        \cos \angle BED=\cos(90^\circ+\theta)=-\sin \theta=-\frac{3}{5}
        \]
        在$\triangle BED$中,由余弦定理,
        \[
        BD^2 = 10^2+10^2 - 2\cdot 10\cdot 10 \cdot \left(-\frac{3}{5}\right) \Rightarrow BD=\sqrt{320}
        \]
        正方形面积为
        \[
        AD^2=160
        \]
    \end{solution}

    \question 已知$P$为$\triangle ABC$内一点,满足$\angle PBA=80^{\circ},\angle PBC=20^{\circ},\angle PCB=10^{\circ}$,且$\angle PCA=30^{\circ}$,求$\angle PAC$。
    \ifprintanswers
    \begin{figure}[H]
        \centering
        \includegraphics[width=0.5\linewidth]{images/image76.png}
    \end{figure}
    \fi
    \begin{solution}
        延长$BP$交$AC$于$D$,设$\angle PAD=\theta$, $PD=a$,则  
        \[
        \angle PDA=60^\circ, \angle DPC=30^\circ \Rightarrow DC=DP=a
        \]  
        在$\triangle PDC$中,由余弦定理,
        \[
        CP^2=a^2+a^2-2a^2\cos120^\circ
        \Rightarrow CP=\sqrt3\,a
        \]
        在$\triangle ADP$中,由正弦定理,
        \[
        \frac{a}{\sin\theta}=\frac{AP}{\sin60^\circ} \Rightarrow AP=\frac{\sqrt3\,a}{2\sin\theta} \tag{1}
        \]
        在$\triangle ABP$中,由正弦定理,
        \[
        \frac{AP}{\sin80^\circ}=\frac{BP}{\sin(40^\circ-\theta)} \Rightarrow AP=\frac{\sin80^\circ}{\sin(40^\circ-\theta)}BP \tag{2}
        \]
        在$\triangle BCP$中,由正弦定理,
        \[
        \frac{\sqrt3\,a}{\sin20^\circ}=\frac{BP}{\sin10^\circ} \Rightarrow BP=\frac{\sin10^\circ}{\sin20^\circ}\sqrt3\,a \tag{3}
        \]
        由式$(1),(2),(3)$,
        \[
        AP=\frac{\sin80^\circ}{\sin(40^\circ-\theta)}\cdot\frac{\sin10^\circ}{\sin20^\circ}\sqrt3\,a=\frac{\sqrt3\,a}{2\sin\theta}
        \] 
        整理得
        \[
        \sin\theta=\frac{\sin20^\circ}{2\sin10^\circ\sin80^\circ}\sin(40^\circ-\theta)
        =\frac{\sin20^\circ}{\cos70^\circ-\cos90^\circ}\sin(40^\circ-\theta)
        \]
        故
        \[
        \theta=20^\circ
        \]
    \end{solution}

    \question 已知 $\triangle ABC$ 中, $\angle ACB=90^{\circ}$, $AB=2AC$, 又此三角形内部有一点 $P$,满足 $PA=\sqrt{2}, PB=\sqrt{10},PC=1$,求 $\triangle ABC$ 的面积。
    \ifprintanswers
    \begin{figure}[H]
        \centering
        \includegraphics[width=0.5\linewidth]{images/image56.png}
    \end{figure}
    \fi
    \begin{solution}
        设$\angle PCB = \theta, A(0,a), B(\sqrt{3}a, 0), C(0,0), P(\cos\theta, \sin\theta)$,则
        \[
        PA^2 = \cos^2\theta + (\sin\theta - a)^2 = 2,       PB^2 = (\cos\theta - \sqrt{3}a)^2 + \sin^2\theta = 10,
        \]
        展开得
        \[
        \sin\theta = \frac{a^2 - 1}{2a}, \quad \cos\theta = \frac{3a^2 - 9}{2\sqrt{3}a}
        \]
        解得
        \[
        \frac{(a^2 - 1)^2}{4a^2} + \frac{(3a^2 - 9)^2}{12 a^2} = 1 \Rightarrow a^2 = 3 + \sqrt{2}
        \]
        故$\triangle ABC$面积为
        \[
        \frac{1}{2} \cdot a \cdot \sqrt{3}a = \frac{3 \sqrt{3} + \sqrt{6}}{2}
        \]
    \end{solution}

    \question 设 $P$ 为正 $\triangle ABC$ 内一点,满足 $AP=2, BP=3, CP=\sqrt{7}$,求正 $\triangle ABC$ 的边长。
    \ifprintanswers
    \begin{figure}[H]
        \centering
        \includegraphics[width=0.5\linewidth]{images/image94.jpg}
    \end{figure}
    \fi
    \begin{solution}
        将 $\triangle CPB$ 绕顶点 $B$ 逆时针旋转 $60^\circ$,此时点 $C$ 与点 $A$ 重合,且有
        \[
        P'B = PB =3,\quad P'A = PC = \sqrt{7},\quad \angle PBP' =60^\circ
        \]
        所以 $\triangle PP'B$ 是正三角形,得 $PP'=3$,在$\triangle APP'$中,由余弦定理,
        \[
        7=2^2+3^2-2\cdot 2 \cdot 3 \cdot \cos \angle APP' \Rightarrow \angle APP' =60^\circ
        \]
        因此 $\angle APB = 60^\circ + 60^\circ =120^\circ$,在$\triangle APB$中,由余弦定理,
        \[
        AB^2 = 2^2 + 3^2 - 2\cdot 2\cdot 3 \cdot \cos \angle APB \Rightarrow AB = \sqrt{19}
        \]
    \end{solution}

    \question 如图,一点$P$在 $\triangle ABC$ 内使得$\angle PAB=\angle PBC=\angle PCA=\alpha$且 $AB=7,BC=8,CA=9$。求 $\tan\alpha$。
    \begin{figure}[H]
        \centering
        \includegraphics[width=0.4\linewidth]{images/image148.png}
    \end{figure}
    \begin{solution}
        设 $AP=x,BP=y,CP=z$,在 $\triangle ABP,\triangle BCP,\triangle CAP$中,由余弦定理,
        \[
        y^{2}=7^{2}+x^{2}-14x\cos\alpha,
        \quad
        z^{2}=8^{2}+y^{2}-16y\cos\alpha,
        \quad
        x^{2}=9^{2}+z^{2}-18z\cos\alpha.
        \]
        三式相加得
        \[
        (14x+16y+18z)\cos\alpha=194 \tag{1}
        \]
        又
        \[
        [\triangle ABC]=[\triangle ABP]+[\triangle BCP]+[\triangle CAP]=\frac12(7x+8y+9z)
        \]
        由海伦公式,
        \[
        [\triangle ABC]=\sqrt{12\cdot3\cdot4\cdot5}=12\sqrt{5}
        \]
        因此
        \[
        (7x+8y+9z)\sin\alpha=24\sqrt{5}\tag{2}
        \]
        $\dfrac{(2)}{(1)}$得
        \[
        \tan\alpha=\frac{24\sqrt{5}}{97}
        \]
    \end{solution}

    \question 在正方形 $ABCD$ 中,点 $E,F$ 在对角线 $AC$ 上,且 $\angle EDF=45^\circ$。若 $AE=x,EF=y,FC=z$,证明 $y^2=x^2+z^2$。
    \ifprintanswers
    \begin{figure}[H]
        \centering
        \includegraphics[width=0.4\linewidth]{images/image178.png}
    \end{figure}
    \fi
    \begin{solution}
        将 $\triangle DFC$ 绕点 $D$ 逆时针旋转 $90^\circ$得 $\triangle DF'A$,使得 $DF=DF'$。由$\angle F'AD=\angle FCD=45^\circ$,
        \[
        \angle F'AE = \angle EAD + \angle F'AD = \angle EAD +\angle FCD=90^\circ
        \]
        在 $\triangle F'AE$ 中,由毕氏定理,
        \[
        F'E^2 = F'A^2 + AE^2 =  x^2+z^2
        \]
        且由于 $\triangle F'DE \congr \triangle FDE \ \text{(SAS)},$有$F'E=FE=y$,故得证
        \[
        y^2 = x^2+z^2
        \]
    \end{solution}
    \ifprintanswers
    \begin{figure}[H]
        \centering
        \includegraphics[width=0.4\linewidth]{images/image179.png}
    \end{figure}
    \fi
    \begin{solution}
        设 $\angle ADE=\theta$,则
        \[
        \angle AED = 135^\circ-\theta,\quad \angle DEF=45^\circ+\theta,\quad \angle EFD=90^\circ-\theta
        \]
        在 $\triangle AED,\triangle DEF,\triangle DFC$ 中,由正弦定理,
        \[
        \frac{x}{\sin\theta}=\frac{ED}{\sin 45^\circ}, \quad
        \frac{y}{\sin 45^\circ} = \frac{FD}{\sin(45^\circ+\theta)},\quad
        \frac{z}{\sin(45^\circ-\theta)} = \frac{FD}{\sin 45^\circ}
        \]
        化简得
        \[
        \frac{x}{y}=\sin 2\theta,\qquad \frac{z}{y}=\cos 2\theta
        \]
        于是
        \[
        \frac{x^2}{y^2}+\frac{z^2}{y^2}=\sin^2 2\theta+\cos^2 2\theta=1 \Rightarrow x^2+z^2=y^2
        \]
        证毕。
        \textcolor{red}{怎样化简得?}
    \end{solution}

    \question 设 $P$ 是正方形 $ABCD$ 内部一点,且 $P$ 到 $A,B,C$ 三顶点的距离分别为 $1,2,3$,求此正方形的面积。
    \ifprintanswers
    \begin{figure}[H]
        \centering
        \includegraphics[width=0.4\linewidth]{images/image51.png}
    \end{figure}
    \fi
    \begin{solution}
        设正方形$ABCD$边长为 $a$,在$\triangle APB$及$\triangle BPC$中,由余弦定理,
        \[
        \cos \angle PBA = \frac{a^{2} + 2^2-1^2}{2\cdot 2 \cdot a}, \quad \cos \angle PBC = \frac{a^{2} +2^2 - 3^2}{2\cdot 2 \cdot a}
        \]
        由$\cos \angle PBC= \cos (90^\circ - \angle PBA)=\sin \angle PBA$,有
        \[
        \cos^{2} \angle PBA + \sin^2 \angle PBA = \left(\frac{a^{2} + 3}{4a}\right)^{2} + \left(\frac{a^{2} - 5}{4a}\right)^{2} = 1
        \]
        解得
        \[
        [ABCD]=a^{2} = 5 + 2 \sqrt{2}.
        \]
    \end{solution}

    \question 平面上,$P$为$AB$上一点满足$AP=5,BP=3,Q$为平面上一点满足$AQ=7,BQ=3$。若以$AB$为直径作一圆$C$,自$P$向$Q$作射线$PQ$交圆$C$于点$R$,试求$PR$的长度。
    \ifprintanswers
    \begin{figure}[H]
        \centering
        \includegraphics[width=0.5\linewidth]{images/image61.png}
    \end{figure}
    \fi
    \begin{solution}
        在$\triangle ABQ$中,由余弦定理,
        \[
        7^2 = 8^2+3^2 -2\cdot 3 \cdot 8 \cdot \cos \angle B
        \Rightarrow \angle B=60^\circ
        \]
        又$BP=BQ=3$,故$\triangle PQB$为等边三角形,于是$PQ=3$;在$\triangle OPR$中,由余弦定理,
        \[
        4^2=1^2 + PR^2-2 \cdot 1 \cdot PR \cdot \cos 60^\circ 
        \]
        解得
        \[
        PR = \frac{-1 + \sqrt{61}}{2}
        \]
    \end{solution}

    \question 两圆 $O_1,O_2$ 都经过 $\triangle ABC$ 的顶点 $A$,且分别与 $BC$ 边相切于点 $B,C$ 。已知圆 $O_1,O_2$ 的面积分别为 $m,n$,试以 $m,n$ 表示 $\triangle ABC$ 外接圆的面积。
    \ifprintanswers
    \begin{figure}[H]
        \centering
        \includegraphics[width=0.5\linewidth]{images/image224.png}
    \end{figure}
    \fi
    \begin{solution}
        设$O_1,O_2,\triangle ABC$外接圆的半径为$R_1,R_2,R_3;\angle ABC = \beta, \angle ACB = \gamma$。在 $\wideparen{AB},\wideparen{AC}$ 分别取点 $P,Q$,由弦切角定理,
        \[
        \angle APB = \angle ABC = \beta, \quad \angle AQC = \angle ACB = \gamma.
        \]
        由正弦定理,
        \[
        4 R_3^2 = \frac{\overline{AC}}{\sin \beta} \cdot \frac{\overline{AB}}{\sin \gamma} = \frac{2 R_2 \sin \gamma}{\sin \beta} \cdot \frac{2 R_1 \sin \beta}{\sin \gamma} = 4 R_1 R_2 \Rightarrow R_3^2 = R_1 R_2
        \]
        所以 $\triangle ABC$ 外接圆的面积为
        \[
        \pi R_3^2 = \pi R_1 R_2 = \sqrt{\frac{m}{\pi}} \cdot \sqrt{\frac{n}{\pi}} \cdot \pi = \sqrt{mn}
        \]
    \end{solution}

    \question 一半径为 4 的大圆里面有三个两两外切,且皆与大圆内切的小圆,其中两小圆半径分别为 1 和 3,且它们的圆心在大圆的一条直径上,求第三个小圆的半径。
    \ifprintanswers
    \begin{figure}[H]
        \centering
        \includegraphics[width=0.5\linewidth]{images/image140.png}
    \end{figure}
    \fi
    \begin{solution}
        设$B=(0,0)$为大圆圆心 ,$A=(-1,0)$为半径为 3 的小圆圆心,$C=(3,0)$为半径为 1 的小圆圆心,$D$为所求半径为 $r$ 的小圆圆心,由海伦公式及等高性质,
        \[
        3=\frac{[\triangle BDC]}{[\triangle ABD]}=
        \frac{\sqrt{4(3-r)\cdot r \cdot1}}{\sqrt{4(1-r)\cdot r\cdot3}}
        \]
        其中$\triangle ABD,\triangle BDC$半周长为
        \[
        [\triangle ABD]=\frac{1}{2}(3+r+(4-r)+1)=4, \quad s_{\triangle BDC}=\frac{1}{2}(1+r+(4-r)+3)=4
        \]
        解得第三个小圆的半径为
        \[
        r=\frac{12}{13}
        \]
    \end{solution}

    \question 在边长为 4 的正三角形 $ABC$ 内接圆中,点 $P$ 在弧 $AC$ 上满足 $AP \cdot PC = 5$。求 $BP$ 的长度。
    \begin{figure}[H]
        \centering
        \includegraphics[width=0.4\linewidth]{images/image152.png}
    \end{figure}
    \begin{solution}
        在四边形 $ABPC$中,由托勒密定理,
        \[
        AB \cdot PC + AP \cdot BC = AC \cdot BP 
        \]
        由$AB=AC=BC$,得
        \[
        BP = AP + PC
        \]
        易知$\angle APC = 120^\circ$,在$\triangle APC$中,由余弦定理,
        \[
        4^2 = AP^2 + PC^2 - 2 \cdot AP \cdot PC \cos 120^\circ \Rightarrow AP^2 + PC^2 = 11
        \]
        于是
        \[
        BP^2 = (AP+PC)^2 = AP^2 + PC^2 + 2 \cdot AP \cdot PC = 21 \Rightarrow BP = \sqrt{21}
        \]
    \end{solution}

    \question 已知三角形 $ABC$ 的外心为 $O$,且 $\angle B$ 为钝角。延长$BC$与外接圆在点 $A$处的切线交于 $D$使得$AO = AD = 2$。$F$ 在 $BC$ 上满足 $AF \perp OD$且交于点 $E$。过点 $A,D,O$ 的圆与 $BC$ 的交于点$H$。已知 $FH = \dfrac{\sqrt{3}}{3}$,求 $\triangle OEF$ 的面积。
    % 23. [6 pts] [7] Let $ABC$ be a triangle with circumcenter $O$ and $\angle B$ obtuse. Suppose the tangent to its circumcircle at $A$ intersects ray $CB$ at $D$, and that $AO=AD=2$. Suppose that $F$ lies on $BC$ with $AF \perp OD$, and let $E$ be the point where $AF$ meets $OD$. Let $H$ be the point where the circle passing through $ADO$ meets $BC$. What is the area of triangle $OEF$ if $FH=\frac{\sqrt{3}}{3}$?23. $\frac{1}{2}\sqrt{2}$. Let $J$ be the point where the extension of $AF$ meets circle $ADO$. Let $x = DF$ and $y = EF$. Power of a point using chords $DH$ and $AJ$ yields $x\frac{\sqrt{3}}{3}=(\sqrt{2}+y)(\sqrt{2}-y)$ while Pythagoras on triangle $DEF$ yields $x^{2}=y^{2}+2$. Combining, we get $x\frac{\sqrt{3}}{3}=2-(x^{2}-2)$ so $x^{2}+x\frac{\sqrt{3}}{3}-4=0$. The solution is $x=\sqrt{3}$ and $y=1$, so the area of the right triangle $OEF$ is $\frac{1}{2}\sqrt{2}\cdot1$.
    \ifprintanswers
    \begin{figure}[H]
        \centering
        \includegraphics[width=0.6\linewidth]{images/image160.png}
    \end{figure}
    \fi
    \begin{solution}
延长 $AF$ 与过点 $A,D,O$ 的圆交于点$J$,设 $x = DF,y = EF$。由幂点定理($DH$ 与 $AJ$)有
\[
x \cdot \frac{\sqrt{3}}{3} = (\sqrt{2}+y)(\sqrt{2}-y) = 2 - y^2.
\]
由直角三角形 $DEF$ 得
\[
x^2 = y^2 + 2.
\]
联立得
\[
x \cdot \frac{\sqrt{3}}{3} = 2 - (x^2 - 2) \implies x^2 + x\frac{\sqrt{3}}{3} - 4 = 0.
\]
解得 $x = \sqrt{3}$,$y = 1$。

因此直角三角形 $OEF$ 的面积为
\[
\frac{1}{2} \cdot OE \cdot EF = \frac{1}{2} \cdot \sqrt{2} \cdot 1 = \frac{1}{2}\sqrt{2}.
\]
\textcolor{red}{$AE=\sqrt{2}$?怎样证明$\angle DAE = 45^\circ$?}
\end{solution}

    \question 已知$\triangle ABC$中,$AB=5,BC=7,AC=6$。点$D$在$BC$上移动,从$D$向$AB,AC$作垂线,垂足分别为$E,F$,求$EF$的最小值。
    \ifprintanswers
    \begin{figure}[H]
        \centering
        \includegraphics[width=0.45\linewidth]{images/image54.png}
    \end{figure}
    \fi
    \begin{solution}
        由$\angle DFA = \angle DEA = 90^\circ$,故点$A, E, D, F$共圆,设该圆半径为$R$,在$\triangle AEF$中,由正弦定理,
        \[
        EF = 2R \sin \angle A = AD \sin \angle A
        \]
        当$AD \perp BC$时,$AD$取得最小值;在$\triangle ABC$中,由余弦定理,
        \[
        \cos \angle A = \frac{5^2 + 6^2 - 7^2}{2 \cdot 5 \cdot 6} = \frac{1}{5} \Rightarrow \sin \angle A = \frac{2\sqrt{6}}{5}
        \]
        $\triangle ABC$面积为
        \[
        [\triangle ABC] = \frac{1}{2} \cdot 5 \cdot 6 \cdot \sin \angle A = \frac{1}{2} AD_{\min} \cdot BC \Rightarrow AD_{\min}=\frac{12 \sqrt{6}}{7}
        \]
        此时
        \[
        EF_{\min} = \frac{12 \sqrt{6}}{7} \cdot \frac{2 \sqrt{6}}{5} = \frac{144}{35}.
        \]
    \end{solution}

    \question 已知三角形 $ABC$ 的重心为 $G$,且 $GB=7,GC=3,G$ 到直线 $BC$ 的距离为 $2$, 求 $GA$ 的长度。
    \ifprintanswers
    \begin{figure}[H]
        \centering
        \includegraphics[width=0.5\linewidth]{images/image57.png}
    \end{figure}
    \fi
    \begin{solution}
        设$GH \perp BC, BC = a,$则半周长为$5+\dfrac{a}{2}$,则
        \[
        [\triangle GBC]=\sqrt{\left(5+\frac{a}{2}\right)\left(\frac{a}{2}-2\right)\left(\frac{a}{2}+2\right)\left(5-\frac{a}{2}\right)} = \frac{1}{2} \cdot a \cdot 2
        \]
        解得
        \[
        a = 4 \sqrt{5} \quad \text{或} \quad a = 2 \sqrt{5}
        \]
        设$P$为$BC$中点,由中线定理,
        \[
        7^2+3^2=2\left(GP^2+\left(\frac{a}{2}\right)^2\right) \Rightarrow GP = 3\ \text{或} \ 2 \sqrt{6}
        \]
        故
        \[
        GA=2GP=6 \ \text{或} \ 4 \sqrt{6}
        \]
    \end{solution}

    \question 已知 $G$ 为 $\triangle ABC$ 的重心, $BC=10,AG=4,\angle BGC=\dfrac{3\pi}{4}$,求 $\triangle ABC$ 的面积。
    \ifprintanswers
    \begin{figure}[H]
        \centering        
        \includegraphics[width=0.5\textwidth]{images/image225.png}
    \end{figure}
    \fi
    \begin{solution}
        设直线 $AG$ 与 $BC$ 相交于 $D$。由于 $G$ 为重心,
        \[
        GD=\frac{AG}{2}=2,\quad BD=DC=\frac{10}{2}=5.
        \]
        在 $\triangle GBC$ 中,由中线定理,
        \[
        GB^2+GC^2=2(2^2+5^2)=58
        \]
        又由余弦定理,
        \[
        10^2=GB^2+GC^2-2\,GB\cdot GC \cos\frac{3\pi}{4} \Rightarrow GB\cdot GC=21\sqrt{2}
        \]
        于是$\triangle GBC$ 面积为
        \[
        [\triangle GBC]
        =\frac12\cdot 21\sqrt{2}\cdot\sin\frac{3\pi}{4}
        =\frac{21}{2}
        \]
        由等高性质,$\ \triangle ABC$ 面积为
        \[
        [\triangle ABC]=3[\triangle GBC]
        =\frac{63}{2}
        \]
    \end{solution}

    \question $I$ 为 $\triangle ABC$ 内心,$D$ 在 $AB$ 上,$E$ 在 $AC$ 上,且 $D,I,E$ 三点共线。已知 $AD=DE=5,AE=6$, 求 $AI$。
    \ifprintanswers
    \begin{figure}[H]
        \centering
        \includegraphics[width=0.4\linewidth]{images/image47.png}
    \end{figure}
    \fi
    \begin{solution}
        设 $\triangle ABC$ 内切圆半径为 $h$,由 $AD=DE$ 可知 
        \[
        \angle AED=\angle DAE=2\angle IAE=2\theta
        \]
        $\triangle ADE$面积为
        \[
        [\triangle ADE]=\sqrt{8\cdot3\cdot3\cdot2}=12=\frac{1}{2}\cdot 5h+\frac{1}{2}\cdot 6h \Rightarrow h=\frac{24}{11}
        \]
        在$\triangle ADE$ 中,由余弦定理,
        \[
        \cos 2\theta=\frac{6^2+5^2-5^2}{2\cdot 6\cdot 5}=\frac{3}{5} \Rightarrow \sin\theta=\sqrt{\frac{1-\cos 2\theta}{2}}=\sqrt{\frac{1}{5}}
        \]
        故
        \[
        AI=\frac{h}{\sin\theta}=\frac{24\sqrt{5}}{11}
        \]
    \end{solution}

    \question $\triangle ABC$满足$AB:AC:BC=3:5:7$。令$I$为$\triangle ABC$的内心, 直线$AI$交$\triangle ABC$外接圆于另一点$D$, 试求$\dfrac{ID}{AD}$的值。
    \ifprintanswers
    \begin{figure}[H]
        \centering
        \includegraphics[width=0.4\textwidth]{images/image33.png}
    \end{figure}
    \fi
    \begin{solution}
        设$E$为$AD$与$BC$的交点,因为$AD$为$\angle A$角平分线,由角平分线定理,
        \[
        \frac{BE}{EC} = \frac{AB}{AC} = \frac{3}{5} \Rightarrow 
        BE = \frac{21}{8}, CE= \frac{35}{8}
        \]
        在$\triangle ABC$ 中,由余弦定理,
        \[
        (7k)^2=(3k)^2 + (5k)^2 - 2 \cdot 3k \cdot 5k \cdot 
        \cos \angle A, k>0 \Rightarrow \angle A = 120^\circ, \angle BAE = 60^\circ
        \]
        在$\triangle ABE$ 中,由余弦定理,
        \[
        \left(\frac{21}{8}\right)^2=3^2 + AE^2-2 \cdot 3 \cdot AE \cdot \cos \angle BAE \Rightarrow AE = \frac{15}{8}
        \]
        又$\angle ABE = \angle ADC$,于是有$\triangle ABE \backsim \triangle CDE \ (AAA)$,则
        \[
        \frac{\frac{15}{8}}{\frac{21}{8}} = \frac{\frac{35}{8}}{DE} \Rightarrow DE = \frac{49}{8}, AD = \frac{15}{8} + \frac{49}{8} = 8
        \]
        又由角平分线定理,
        \[
        \frac{\frac{21}{8}}{3} = \frac{EI}{AI} \Rightarrow AI = \frac{8}{15} \cdot AE = 1
        \]
        故
        \[
        \frac{ID}{AD} = \frac{8 - 1}{8} = \frac{7}{8}
        \]
    \end{solution}

    \question 在 $\triangle ABC$ 的边 $AB$ 与 $AC$ 的外侧分别作正三角形 $\triangle ABE$ 及 $\triangle ACF$。已知 $AC = 1$ 且 $EF = 2$,求 $\triangle ABC$ 面积的最大可能值。
    \begin{solution}
        设 $\angle BAC = \theta,AB = c$,则 $\angle EAF = 240^\circ - \theta$,在$\triangle AEF$中,由余弦定理,
        \[
        2^2 = c^2 + 1^2 - 2 c \cdot 1 \cdot \cos(240^\circ - \theta)
        \Rightarrow c^2 + c \cos \theta + \sqrt{3} c \sin \theta =3
        \]
        设 $x = c \cos \theta$, $y = c \sin \theta$,上式变为
        \[
        x^2 + y^2 + x + \sqrt{3} y = 3 \Rightarrow \left( x + \frac12 \right)^2 + \left( y + \frac{\sqrt{3}}{2} \right)^2 = 4
        \]
        $\triangle ABC$的面积为
        \[
        \frac12 \cdot 1 \cdot c \sin \theta = \frac{y}{2} \le \frac12 \left( 2 - \frac{\sqrt{3}}{2} \right) = 1 - \frac{\sqrt{3}}{4}
        \]
        等号成立当且仅当$x=-\dfrac{1}{2},y=2-\dfrac{\sqrt{3}}{2}$,即
        \[
        c=\sqrt{x^2+y^2}=\sqrt{\frac{17}{4}-2\sqrt{3}},
        \]
        \[
        \theta=\arctan\frac{y}{x}=\pi-\arctan(4-\sqrt{3})
        \]
    \end{solution}

    \question 如图,点 $M$ 是长方形 $ABCD$ 的边 $AD$ 的中点,$\triangle BME$ 是正三角形,$N$ 为线段 $EC$ 的中点。已知 $\angle AMB = 48^\circ$,${BM} = 1$,求 ${DN}$ 的长。 
        \begin{figure}[H]
            \centering
            \includegraphics[width=0.4\textwidth]{images/image26.png}
        \end{figure}
    \begin{solution}
        设$AB = a = \sin 48^\circ,MA = MB = b = \sin 42^\circ=\cos 48^\circ,$取
        \[A(0,a), B(0,0), C(2b,0),
        D(2b,a), E(\cos 12^\circ, -\sin 12^\circ),
        \]
        则$CE$中点 $N$ 为
        \[
        N\left( b + \frac{1}{2}\cos 12^\circ,\ -\frac{1}{2}\sin 12^\circ \right)
        \]
        故
        \[
        \begin{aligned}
        {DN}^2 &= \left(b - \frac{1}{2} \cos 12^\circ\right)^2 + \left(a + \frac{1}{2} \sin 12^\circ\right)^2 \\
        &= a \sin 12^\circ - b \cos 12^\circ + \frac{1}{4} (\cos^2 12^\circ + \sin^2 12^\circ) + a^2 + b^2 \\
        &= -\cos 60^\circ + \frac{1}{4} + 1  = \frac{3}{4} \Rightarrow DN = \frac{\sqrt{3}}{2}
        \end{aligned}
        \]
    \end{solution}

    \question 已知 $AC$ 为半圆的直径,若将弧 $AB$ 沿弦 $AB$ 向下折使其与 $AC$ 相交于 $D$ 点,且 $AD=9,DC=7$,则 $AB$ 的长度为多少?  
    \ifprintanswers
    \begin{figure}[H]
        \centering        
        \includegraphics[width=0.5\textwidth]{images/image80.jpg}
    \end{figure}
    \fi
    \begin{solution}
        作 $D'D'' \parallels BC$ 交圆于 $D',D''$ 及直线 $AB$ 于 $P$ 点,
        $AB$ 为 $DD'$ 的中垂线,
        \[
        AD' = AD = 9,
        \]
        在$\triangle AOD'$ 中,由余弦定理,
        \[
        \cos \angle AOD' = \frac{8^2 + 8^2 - 9^2}{2 \cdot 8 \cdot 8} = \frac{47}{128}
        \] 
        在$\triangle DOD'$ 中,由余弦定理,
        \[ 
        \cos \angle D'OD = -\frac{47}{128} = \frac{8^2 + 1^2 - DD'^2}{2 \cdot 8 \cdot 1} \Rightarrow DD' = \frac{9\sqrt{14}}{4}  ,PD = PD' = \frac{9\sqrt{14}}{8} 
        \]
        由圆幂定理,
        \[
        9 \cdot 7 = \frac{9\sqrt{14}}{4}  \cdot DD'' \Rightarrow DD'' = 2\sqrt{14}
        \]
        又 $\triangle APD \backsim \triangle ABC \ (AAA)$,
        \[
        \frac{AP}{PB} = \frac{AD}{DC} = \frac{9}{7}
        \]
        令 $AP = 9k,PB = 7k,k > 0$,由圆幂定理,
        \[
        \frac{9\sqrt{14}}{8}  \cdot \left( \frac{9\sqrt{14}}{8}  + 2\sqrt{14} \right) = 9k \cdot 7k \Rightarrow k = \frac{5\sqrt{2}}{8}
        \] 
        故
        \[
        AB = 16k = 10\sqrt{2}
        \]
    \end{solution}

    \question 如图,在正$\triangle ABC$中,将顶点$A$折至$D$,使得$D$在$BC$上,若$BD=1,CD=2$,求$EF$之长。
    \begin{figure}[H]
        \centering
        \includegraphics[width=0.4\linewidth]{images/image104.png}
    \end{figure}
    \ifprintanswers
    \begin{figure}[H]
        \centering
        \includegraphics[width=0.4\linewidth]{images/image105.jpg}
    \end{figure}
    \fi
    \begin{solution}
        设
        \[
        EA = ED = a, FA = FD = b, \angle EDA = \angle A = 60^\circ,
        \]
        在$\triangle BDE$ 及$\triangle CDF$中,由余弦定理,
        \[
        a^2=(3-a)^2 + 1-2(3-a)\cos 60^\circ ,\quad
        b^2=(3-b)^2 + 4-4(3-b)\cos 60^\circ
        \]
        解得
        \[
        a = \frac75,\quad b = \frac74
        \]
        同理,在$\triangle EDF$中,由余弦定理,
        \[
        EF^2 = a^2 + b^2 -2ab \cos 60^\circ \Rightarrow EF = \frac{7\sqrt{21}}{20}
        \]
    \end{solution}

    \question 如下图,将长 $AB=240$、宽 $BC=288$ 的长方形纸张对折,让顶点 $C$ 刚好落在 $AB$ 的中点 $M$ 上。若 $EF$ 是折线,求折线 $EF$ 的长度。
    \begin{figure}[H]
        \centering
        \includegraphics[width=0.35\linewidth]{images/image48.png}
    \end{figure}
    \ifprintanswers
    \begin{figure}[H]
        \centering
        \includegraphics[width=0.45\linewidth]{images/image139.png}
    \end{figure}
    \fi
    \begin{solution}
        如上图,设 $\angle EFC=\angle EFM=\theta$,则 
        \[
        \angle MFB=180^{\circ}-2\theta,\angle FMB=2\theta-90^{\circ}
        \]
        过 $E$ 点作水平线交 $BC$ 于 $G$ 点,由直角三角形$EFG$得 $EF = \dfrac{240}{\sin\theta}$,又  
        \[
        288 = BC = BF + FC = BF + FM = \frac{120}{\cos(2\theta-90^{\circ})} + 120\tan(2\theta-90^{\circ})
        \]
        化简得
        \[
        288 = \frac{120}{\sin 2\theta} - 120\cot 2\theta
        = 120\cdot \frac{1-\cos 2\theta}{\sin 2\theta}
        = 120\cdot \frac{2\sin^{2}\theta}{2\sin\theta\cos\theta}
        = 120\tan\theta \Rightarrow \tan\theta = \frac{12}{5}
        \]
        故 
        \[
        \sin\theta = \frac{12}{13} \Rightarrow EF = 240 \cdot \frac{13}{12} = 260
        \]
    \end{solution}
    \begin{solution}
        设 $MF = FC = a$,在直角 $\triangle BMF$ 中,由毕氏定理,
        \[
        a^2 = 120^2 + (288 - a)^2 \Rightarrow a = 169
        \]
        设 $DE = D'E = b$,在直角 $\triangle AEM$ 及$\triangle CDE$中,由毕氏定理,
        \[
        ME^2 = 120^2 + (288 - b)^2, \quad EC^2 = b^2 + 240^2
        \]
        由 $EM = EC$ 解得
        \[
        b = 69
        \]
        在直角 $\triangle EFG$ 中,由毕氏定理,
        \[
        EF^2 = 240^2 + (a - b)^2 = 260^2,
        \]
        故折线长度为 $EF = 260$。
    \end{solution}

    \question 如图,一个平行四边形 $PQRS$ 沿着线段 $MN$ 将平行四边形对折,恰好让 $P$ 点与 $R$ 点重合,且让 $Q$ 点坐落于 $Q'$ 的位置,而 $M,N$ 分别在 $PS,QR$ 上。若 $PQ=6,PM = 4\sqrt{3},\cos Q = -\dfrac{\sqrt{11}}{4}$,求折痕长 $MN$。
    \begin{figure}[H]
        \centering
        \includegraphics[width=0.6\textwidth]{images/image29.png}
    \end{figure}
    \begin{solution}
        设$\angle SMR = \angle MRQ = \alpha, SR = PQ = 6, \angle PMN = \angle MNR = \theta,$

        又 $MN$ 为折线,所以有 $MR = MP = 4\sqrt{3},\angle RMN = \angle PMN = \theta,$

        在 $\triangle MSR$ 中,由正弦定理,
        \[
        \frac{6}{\sin \alpha} = \frac{4\sqrt{3}}{1-\cos^2 Q} \Rightarrow \sin \alpha = \frac{\sqrt{15}}{8}
        \]
        于是
        \[
        \cos \alpha = \frac{7}{8}, \cos \frac{\alpha}{2} = \frac{\sqrt{15}}{4}
        \]
        又 $2\theta + \alpha = 180^\circ$,所以
        \[
        \sin \theta = \sin\left(90^\circ - \frac{\alpha}{2}\right) = \cos \frac{\alpha}{2} = \frac{\sqrt{15}}{4}
        \]
        在 $\triangle RMN$ 中,由正弦定理,
        \[
        \frac{MN}{\frac{\sqrt{15}}{8}} = \frac{4\sqrt{3}}{\frac{\sqrt{15}}{4}} \Rightarrow 
        MN = 2\sqrt{3}
        \]
    \end{solution}
    
    \question 以两种方式将矩形 $PQRS$放置于矩形 $ABCD$内:方式一为 $Q,R$ 分别在 $B,C$上,方式二为 $P,Q,R,S$ 分别在 $AB,BC,CD,DA$ 上。若 $AB=718,PQ=250$,求 $BC$ 的长度。
    \begin{figure}[H]
        \centering        
        \includegraphics[width=0.6\textwidth]{images/image186.png}
    \end{figure}
    \begin{solution}
        已知$AB=718,PQ=250$,在图二中,设 $BC=x,BQ=a,PB=b$,则
        \[
        AD=PS=QR=x,\quad QC=x-a,\quad AP=718-b
        \]
        由于$\triangle RDS \congr \triangle PBQ \ \text{(ASA)},\triangle SAP \congr \triangle QCR \ \text{(ASA)}$,因此
        \[
        \dfrac{SA}{PB}=\dfrac{AP}{BQ}=\dfrac{SP}{PQ} \Rightarrow
        \dfrac{x-a}{b}=\dfrac{718-b}{a}=\dfrac{x}{250} \tag{1}
        \]
        在 $\triangle PBQ$ 及 $\triangle SAP$ 中,由毕氏定理,
        \[
        a^2+b^2=250^2, \quad (x-a)^2+(718-b)^2=x^2 \tag{2,3}
        \]
        由$(1),(2),(3)$解得
        \[
        a=88,b=234,BC=x=1375
        \]
    \end{solution}

    \question 正方形 $PQRS$ 的边长为 $4$。点 $U$ 在对角线 $PR$ 上使得$PR=4UR$。以 $U$ 为圆心作圆,使得圆与正方形的两条边相切。$PW$ 是圆的切线,且 $W$ 在 $QR$ 上。求 $PW$ 的长度。
    \ifprintanswers
    \begin{figure}[H]
        \centering        
        \includegraphics[width=0.5\textwidth]{images/image214.png}
    \end{figure}
    \fi
    \begin{solution}
        由 $PR=4UR=4\sqrt{2}$,得
        \[
        PU=\frac{3}{4}PR=3\sqrt{2}, \quad UR=\sqrt{2}.
        \]
        设圆分别与 $PW,WR, RS$ 相切于点 $X,Y,Z$,由$\triangle UZR$知圆的半径为 $1$,在$\triangle PXU$中,由毕氏定理,
        \[
        PX=\sqrt{(3\sqrt{2})^2-1^2}=\sqrt{17}
        \]
        设$x=PW$,在$\triangle PQW$中,由毕氏定理,
        \[
        4^2+(3-x)^2=(\sqrt{17}+x)^2 \Rightarrow x=\frac{4}{\sqrt{17}+1}
        \]
        故
        \[
        PW=\sqrt{17}+\frac{4}{\sqrt{17}+1}=\frac{5\sqrt{17}-1}{4}
        \]
    \end{solution}

    \question 如下图所示, $AB=8$, 以 $AB$ 为直径的半圆上有 $C, D$ 两点, 且 $AC=2,BD=7$, 求 $CD$ 的长度。 
    \begin{figure}[H]
        \centering        
        \includegraphics[width=0.5\textwidth]{images/image99.png}
    \end{figure}
    \ifprintanswers
    \begin{figure}[H]
        \centering        
        \includegraphics[width=0.5\textwidth]{images/image100.jpg}
    \end{figure}
    \fi
    \begin{solution}
        $AB$ 为直径,在 $\triangle ABC$ 及 $\triangle ABD$中,由毕氏定理,
        \[
        BC=\sqrt{8^2-2^2}=2\sqrt{15}, AD=\sqrt{8^2-7^2}=\sqrt{15}
        \]
        设$\theta_1=\angle ABC,\theta_2=\angle ABD,$则
        \[
        \tan \theta_1 =\frac{1}{\sqrt{15}},\quad
        \tan (\theta_1+\theta_2) = \frac{\frac{1}{\sqrt{15}} + \tan \theta_2}{1 - \frac{1}{\sqrt{15}} \tan \theta_2} = \frac{\sqrt{15}}{7} \Rightarrow \tan \theta_2 = \frac{1}{\sqrt{15}}
        \]
        在 $\triangle BCD$中,由余弦定理,
        \[
        \cos \theta_2 = {\sqrt{15}\over 4} 
        = {(2\sqrt{15})^2 + 7^2 - CD^2 \over 2\cdot 7 \cdot 2\sqrt{15}} \Rightarrow CD = 2
        \]
    \end{solution}

    \question 三角形 $ABC$ 内接于一直径为 $5\sqrt{13}$ 的圆, $D$ 点在 $AC$ 上, 且 $\angle ABD = 90^\circ$. 已知 $BC = 10,AD = 13$, 求 $CD$。
    \ifprintanswers
    \begin{figure}[H]
        \centering        
        \includegraphics[width=0.5\textwidth]{images/image87.jpg}
    \end{figure}
    \fi
    \begin{solution}
        由于 $\angle ABD$ 为直角, 延长 $BD$ 交圆周于 $B'$, 则 $AB'$ 为直径,由$\triangle DBC \backsim \triangle DAB' \ (\text{AAA})$,
        \[
        \frac{DB}{DA} = \frac{BC}{AB'} \Rightarrow \frac{DB}{13} = \frac{10}{5\sqrt{13}} 
        \Rightarrow DB = 2\sqrt{13}
        \]
        在 $\triangle ABD$ 中,由毕氏定理,
        \[
        AB^2 + (2\sqrt{13})^2 = 13^2 \Rightarrow AB = 3\sqrt{13}
        \]
        在 $\triangle ABB'$ 中,由毕氏定理,
        \[
        BB' = \sqrt{(5\sqrt{13})^2 - (3\sqrt{13})^2} = 4\sqrt{13} \Rightarrow DB' = BB' - BD = 2\sqrt{13}
        \]
        再由 $\triangle DBC \backsim \triangle DAB'$,
        \[
        \frac{BC}{AB'} = \frac{CD}{DB'} \Rightarrow \frac{10}{5\sqrt{13}} = \frac{CD}{2\sqrt{13}} \Rightarrow CD = 4
        \]
    \end{solution}

    \question 已知圆内接六边形 $ABCDEF$ 的边长依次为 $5,5,7,7,5,7$,求该六边形的面积。
    \ifprintanswers
    \begin{figure}[H]
        \centering
        \includegraphics[width=0.5\linewidth]{images/image78.png}
    \end{figure}
    \fi
    \begin{solution}
        设圆心为 $O$,则
        \[
        \angle AOB : \angle BOC : \angle COD : \angle DOE : \angle EOF : \angle FOA = 5:5:7:7:5:7
        \]
        即
        \[
        \angle AOB = \angle BOC = \angle EOF = 50^\circ,\quad \angle COD = \angle DOE = \angle FOA = 70^\circ \Rightarrow \angle BAF = 120^\circ
        \]
        在$\triangle BAF$ 中,由余弦定理,
        \[
        BF^2 =5^2 + 7^2 - 2\cdot 5\cdot 7 \cdot \cos 120^\circ= 109
        \]
        六边形面积为
        \[
        [ABCDEF]= [\triangle BDF] + 3 [\triangle ABF]
        = \frac{\sqrt{3}}{4} \cdot 109 + \frac{3}{2} \cdot 5 \cdot 7 \cdot \sin 120^\circ
        = \frac{107\sqrt{3}}{2}
        \]
    \end{solution}

    \question 已知$\triangle ABC$中,$AB=25,AC=30,\angle B=2\angle C$, 且$\angle B$的内角平分线交$\triangle ABC$的外接圆于$D$点,且$B,D$为相异点。试求$\triangle BCD$的面积。 
    \ifprintanswers
    \begin{figure}[H]
        \centering
        \includegraphics[width=0.5\linewidth]{images/image70.png}
    \end{figure}
    \fi
    \begin{solution}
        设$\angle C = \theta$,则$\angle B = 2\theta$,$\angle A = \pi - 3\theta$, 在$\triangle ABC$ 中,由正弦定理,
        \[
        \frac{25}{\sin \theta} = \frac{30}{\sin 2\theta} \implies \cos \theta = \frac{3}{5}.
        \]
        因同弧所对圆周角相等,$\angle D = \angle A,\angle B = \angle DCB,$故$\triangle ABC \cong \triangle DCB$\ (ASA),因此
        \[
        [\triangle BCD] = [\triangle ABC] = \frac{1}{2} \cdot 25 \cdot 30 \cdot \sin(\pi - 3\theta) = 375 \sin 3\theta.
        \]
        由$\sin \theta = \dfrac{4}{5}$,
        \[
        [\triangle BCD] = 375 \left(3 \cdot \frac{4}{5} - 4 \cdot \left(\frac{4}{5}\right)^3 \right) = 132
        \]
    \end{solution}

    \question 半圆 $\Gamma$ 的直径 $AB$ 之长为 $14$, 圆 $\Omega$ 切 $AB$ 于点 $P$ 且交 $\Gamma$ 于点 $Q$ 与点 $R$。若 $QR=3\sqrt{3}$ 且 $\angle QPR=60^{\circ}$, 求 $\triangle PQR$ 的面积。
    \ifprintanswers
    \begin{figure}[H]
        \centering
        \includegraphics[width=0.5\linewidth]{images/image59.png}
    \end{figure}
    \fi
    \begin{solution}
        设半圆心为 $C$, 小圆心为 $O$, 小圆半径为 $r$。由$\angle QPR=60^\circ$ 得 $\angle QOR=120^\circ$,在$\triangle ABC$ 中,由正弦定理,
        \[
        \frac{QR}{\sin\angle QOR}=\frac{3\sqrt3}{\sin 120^\circ}=2r \Rightarrow OQ=OR=r=3
        \]
        且发现$\triangle COR \congr \triangle COQ$ \ (SSS),故 $\angle COQ=\angle COR=120^\circ$。在$\triangle COQ$中,由余弦定理,
        \[
        7^2 = CO^2+3^2-2\cdot CO\cdot 3 \cdot \cos 120^\circ \Rightarrow CO=5
        \]
        在$\triangle COP$中,由毕氏定理,$CP=\sqrt{5^2-3^2}=4$,设 $\angle COP=\theta$, 有
        \[
        \sin\theta=\frac{4}{5},\quad \cos\theta=\frac{3}{5}
        \]
        因此
        \begin{align*}  
        [\triangle PQR]
        &=[\triangle POQ]+[\triangle POR]+[\triangle OQR]\\
        &=\frac12\cdot 3^2\bigl(\sin(120^\circ-\theta)+\sin(120^\circ+\theta)\bigr)+\frac12\cdot 3^2\sin 120^\circ
        =\frac{99\sqrt3}{20}
        \end{align*}
    \end{solution}

    \question 已知圆内接四边形 $ABCD$ 中可以作一个内切圆,且 $E,F,G,H$ 分别在$AB,BC,CD,DA$上,且为此四边形 $ABCD$ 与其内切圆相切的四个切点,若 $FG=6,EF=7,EH=8$,求 $GH$ 的长度。
    \ifprintanswers
    \begin{figure}[H]
        \centering
        \includegraphics[width=0.4\textwidth]{images/image23.png}
    \end{figure}
    \fi
    \begin{solution}
        由圆内接四边形性质有
        \[
        \angle A + \angle C =\angle GOF + \angle C = 180^\circ 
        \Rightarrow \angle A = \angle GOF
        \]
        所以
        \[
        \triangle AHE \backsim \triangle OGF \ \text{(AAA)} \Rightarrow {AH} = {AE} = \dfrac{4}{3}r
        \]
        在$\triangle AHE$及$\triangle OHE$中,由余弦定理,
        \[
        \cos \angle A = -\cos \angle HOE \Rightarrow \frac{2r^2 - 8^2}{2r^2}= -\frac{\left(\frac{4r}{3}\right)^2 + \left(\frac{4r}{3}\right)^2 - 8^2}{2 \cdot \left(\frac{4r}{3}\right)^2}\Rightarrow r = 5
        \]
        同理可得$\angle D = \angle EOF$,在$\triangle OEF$中,由余弦定理,
        \[
        \cos \angle D = \cos \angle EOF = \frac{2r^2 - 7^2}{2r^2}  = \frac{1}{50}
        \]
        又 $\cos \angle GOH = -\cos \angle D = -\dfrac{1}{50}$
        ,在$\triangle OGH$中,由余弦定理,
        \[
        \cos \angle GOH = \frac{2r^2 - {GH}^2}{2r^2}
        \Rightarrow {GH} =\sqrt{51}
        \]
    \end{solution}

    \question 若圆 $O_1$ 与圆 $O_2$ 的半径比为 $2:3$,且圆 $O_1$ 与圆 $O_2$ 交于 $A,B$ 两点。过 $B$ 点作一直线分别交圆 $O_1$ 及圆 $O_2$ 于 $C,D$ 两点,且 $\angle CAD=\dfrac{2\pi}{3}$,求 $\tan \angle ACD$。 
    \ifprintanswers
    \begin{figure}[H]
    \centering        
    \includegraphics[width=0.7\textwidth]{images/image83.png}
    \end{figure}
    \fi
    \begin{solution}
        调整两圆心距使得  $C$ 在 $O_1O_2$ 上,令 $\angle ACD=\theta$,则
        \[
        \angle AO_2O_1 = \frac{1}{2}\angle AO_2B = \angle ADC = 60^\circ-\theta
        \]
        又 $\angle ACB=180^\circ-\theta$,所以
        \[
        \angle AO_1B = 360^\circ-2(180^\circ-\theta)=2\theta \Rightarrow \angle AO_1O_2=\theta
        \]
        在$\triangle APO_1$及$\triangle APO_2$中,设$r_1=2k,r_2=3k,$则
        \[
        \sin \angle AO_1O_2 = \sin \theta = \frac{AP}{2k}, \sin \angle AO_2O_1 = \sin (60^\circ-\theta) = \frac{AP}{3k}
        \]
        解得
        \[
        2\sin \theta = 3\left(\frac{\sqrt{3}}{2}\cos\theta - \frac{1}{2}\sin\theta\right)
        \Rightarrow \tan \theta = \frac{3\sqrt{3}}{7}
        \]
    \end{solution}
    
    \question 已知 $\triangle ABC$ 内接于半径为 $1$ 的圆。对于$\triangle ABC$的每一条边,作一圆使得与该边切于其中点,且该圆与大圆相切。已知其中两个小圆的半径分别为 $\dfrac{2}{3},\dfrac{2}{11}$,求第三个小圆的半径。
    \ifprintanswers
    \begin{figure}[H]
        \centering
        \includegraphics[width=0.5\textwidth]{images/image156.png}
    \end{figure}
    \fi
    \begin{solution}
        设大圆圆心为 $O,BC,AC,AB$中点为 $D,E,F$,对应小圆半径为 \[
        r_1=\frac{2}{3},r_2=\frac{2}{11},r_3
        \]  
        在$\triangle BDO$中,由毕氏定理,
        \[
        BC^2=4(OB^2-OD^2)
        =4\Bigl(1-(1-r_1)^2\Bigr)=\frac{32}{9}
        \]
        在$\triangle CEO$中,由毕氏定理,
        \[
        AC^2=4(OC^2-OE^2)
        =4\Bigl(1-(1-r_2)^2\Bigr)=\frac{288}{121}
        \]
        设 $\angle BOC=\beta,\ \angle COA=\gamma$,在$\triangle BOC$及$\triangle AOC$中,由余弦定理,
        \[
        \cos\beta=\frac{OB^2+OC^2-BC^2}{2OB\cdot OC}=-\frac{7}{9},\quad
        \cos\gamma=\frac{OC^2+OA^2-AC^2}{2OC\cdot OA}=-\frac{23}{121}.
        \]
        由此得
        \[
        \sin\beta=\frac{4\sqrt2}{9},\quad \sin\gamma=\frac{84\sqrt2}{121}.
        \]
        于是
        \[
        \cos\angle AOB=\cos(\beta-\gamma)=\cos\beta\cos\gamma+\sin\beta\sin\gamma
        =\frac{833}{1089}.
        \]
        在$\triangle AOB$中,由余弦定理,
        \[
        AB^2=OA^2+OB^2-2OA\cdot OB\cos\angle AOB=\frac{512}{1089}
        \]
        因此
        \[
        OF=\sqrt{OA^2-\frac14 AB^2}=\frac{31}{33} \Rightarrow r_3=\frac12(1-OF)=\frac{1}{33}
        \]
    \end{solution}

    \question $P,Q$ 为正方形 $ABCD$ 内两点使得 $DP \parallels BQ$ 且 $DP = PQ = BQ$,求 $\angle ADP$ 的最小可能值。
    \begin{figure}[H]
        \centering
        \includegraphics[width=0.3\textwidth]{images/image39.png}
    \end{figure}
    \ifprintanswers
    \begin{figure}[H]
        \centering
        \includegraphics[width=0.35\textwidth]{images/image40.png}
    \end{figure}
    \fi
    \begin{solution}
        设$DP = PQ = QB = a,\angle DPR = \angle QBC = \theta, \angle RPQ = \alpha$,则:
        \[
        CD = DR + MQ + QN = 2a \sin \theta + a \sin \alpha
        \]
        \[
        BC = BN + BN - PM = 2a \cos \theta - a \cos \alpha
        \]
        由$CD = BC$得
        \[
        2(\sin \theta - \cos \theta) = -(\sin \alpha + \cos \alpha)
        \]
        两边平方得
        \[
        4 - 4 \sin 2\theta = 1 + \sin 2\alpha
        \Rightarrow \sin 2\theta = \frac{3 - \sin 2\alpha}{4}
        \]
        由 $\sin 2\alpha \in [-1, 1]$,
        \[
        \frac{1}{2} \le \sin 2\theta \le 1
        \Rightarrow 15^\circ \le \theta \le 45^\circ
        \]
        因此$\theta$ 的最小值为 $15^\circ$。
    \end{solution}
    \ifprintanswers
    \begin{figure}[H]
        \centering
        \includegraphics[width=0.35\textwidth]{images/image190.png}
    \end{figure}
    \begin{figure}[H]
        \centering
        \includegraphics[width=0.35\textwidth]{images/image191.png}
    \end{figure}
    \fi
    \begin{solution}
        以$D$为原点构造坐标系,设 $P(a,b),Q(2-a,2-b)$,由 $PD=PQ$ 得
        \[
        a^2+b^2=(2-2a)^2+(2-2b)^2 \Rightarrow \left(a-\frac{4}{3}\right)^2+\left(b-\frac{4}{3}\right)^2=\frac{8}{9}
        \]
        因此 $P$ 在圆心 $\left(\dfrac{4}{3},\dfrac{4}{3}\right)$,半径 $\dfrac{2\sqrt{2}}{3}$ 的圆上。当 $DP$ 与此圆相切时,$\theta=\angle ADP$ 取最小值, 由$\triangle DOP$,
        \[
        \sin(45^\circ-\theta)=\frac{OP}{OD}=\frac{1}{2}
        \]
        此时 $45^\circ-\theta=30^\circ$,即$\theta=15^\circ$。
    \end{solution}
    \ifprintanswers
    \begin{figure}[H]
        \centering
        \includegraphics[width=0.35\textwidth]{images/image192.png}
    \end{figure}
    \fi
    \begin{solution}
        设正方形$ABCD$边长为 $1$,令 $\angle ADP=\theta,PD=a,PB=b$。在 $\triangle PDB$ 中,由余弦定理,
        \[
        \cos(45^\circ-\theta)=\frac{a^2+2-b^2}{2\sqrt{2}a} \tag{1}
        \]
        在 $\triangle PDB$ 中,由余弦定理,
        \[
        \cos(45^\circ-\theta)=\frac{a^2+\left(\frac{\sqrt2}{2}\right)^2-\left(\frac{a}{2}\right)^2}{2a\left(\frac{\sqrt2}{2}\right)}=\frac{\frac{3}{4}a^2+\frac{1}{2}}{\sqrt{2}a} \tag{2}
        \]
        由$(1),(2)$得
        \[
        b^2=\frac{1}{2}(2-a^2)
        \]
        代入$(1)$得
        \[
        \cos(45^\circ-\theta)=\frac{a^2+2-\frac{1}{2}(2-a^2)}{2\sqrt{2}a} = \frac{1}{4\sqrt{2}}\left(3a+\frac{2}{a}\right) \ge \frac{1}{4\sqrt{2}} \cdot 2\sqrt{6}=\frac{\sqrt{3}}{2},
        \]
        等号成立当且仅当$a^2=\frac{2}{3}$,当$\cos(45^\circ-\theta)$为最小时,$\theta=15^\circ$为最小。
    \end{solution}
    \ifprintanswers
    \begin{figure}[H]
        \centering
        \includegraphics[width=0.35\textwidth]{images/image193.png}
    \end{figure}
    \fi
    \begin{solution}
        连接 $BD$交 $PQ$ 于 $M$,由于$\triangle PDM \congr \triangle QBM \ \text{(ASA)}$,有$DM=BM$,即$M$为$BD$中点。不失一般性,设 $PM=1,PD=2$,记 $\angle ADP=\theta,\angle PDM=\alpha$,则由于 $\theta+ \alpha=45^\circ$,当$\alpha$取最大时$\theta$取最小。
        在$\triangle PMD$中,由正弦定理,
        \[
        \frac{\sin \alpha}{1}=\frac{\sin \angle PMD}{2}
        \]
        欲使$\sin \alpha$最大,取$\sin \angle PMD=1$,此时$\sin \alpha = \dfrac{1}{2}$,即 $\alpha = 30^\circ$,所以 $\theta$ 的最小值为 $15^\circ$。  
    \end{solution}
    \ifprintanswers
    \begin{figure}[H]
        \centering
        \includegraphics[width=0.35\textwidth]{images/image194.png}
    \end{figure}
    \fi
    \begin{solution}
        以$D$为原点构造坐标系,设 $PD=PQ=QB=a,\angle ADP=\theta$,则$P(a\sin\theta,a\cos\theta),Q(1-a\sin\theta,1-a\cos\theta)$,且有
        \[
        PQ^2=(1-2a\sin\theta)^2+(1-2a\cos\theta)^2=a^2,
        \]
        化简得
        \[
        \frac{2+3a^2}{4a}=\sin\theta+\cos\theta=\sqrt{2}\sin(\theta+45^\circ)
        \]
        从而
        \[
        \sin(\theta+45^\circ)=\frac{2+3a^2}{4\sqrt{2}a}=\frac{1}{2\sqrt{2}a}+\frac{3a}{4\sqrt{2}}\ge \frac{\sqrt{3}}{2},
        \]
        由于$0^\circ \le \theta \le 90^\circ,\sin(\theta+45^\circ)$在$\theta=15^\circ$有最小,故$\theta=15^\circ$为最小。
    \end{solution}

    \question 右图为月偏食的示意图,满月被地球的影锥遮蔽一部分。假设满月和地球影锥截面都是正圆,在图中标记圆弧 $\wideparen{AP} = \wideparen{PB}$ 均为满月的圆周一部分,圆弧 $\wideparen{AQ} = \wideparen{QB}$ 均为地球影锥截圆的圆周一部分。若 $AB$ 与 $PQ$ 分别是满月圆半径的 $\sqrt{3}$ 与 $2-\sqrt{3}$ 倍,求 
    \[
    \frac{\text{月偏食亮面面积}}{\text{满月圆面积}}.
    \]
    \begin{figure}[H]
        \centering
        \includegraphics[width=0.35\textwidth]{images/image36.png}
    \end{figure}
    \ifprintanswers 
    \begin{figure}[H]
        \centering
        \includegraphics[width=0.5\textwidth]{images/image37.png}
    \end{figure}
    \fi
    \begin{solution}
        设小圆(月亮)圆心为 $O_1$,半径$r = 1$ ,大圆(太阳)圆心为$O_2$,半径$R$,$C$为$O_1Q$ 与$AB$的交点,则
        \[
        BC = \frac{1}{2} AB = \frac{1}{2} \sqrt{3} \Rightarrow \angle CO_1B = 60^\circ,\angle O_1BC = 30^\circ
        \]
        又 $CQ = O_1P - PQ - O_1C = \sqrt{3} - \frac{3}{2}$,
        \[
        \tan \angle QBC = \frac{CQ}{BC} = 2 - \sqrt{3} \Rightarrow \angle QBC = 15^\circ,\angle CQB = 75^\circ
        \]
        由于 $O_2B = O_2Q = R$,因此
        \[
        \angle BO_2C = 180^\circ - 75^\circ \cdot 2 = 30^\circ \Rightarrow R = 2 BC = \sqrt{3}
        \]
        故弓形$AQB$面积为
        \[
        [\text{弓形}AQB] 
        = [\text{扇形}AQBO_2]-[\triangle ABO_2] 
        = \frac{1}{6} (\sqrt{3})^2 \pi - \frac{1}{2} (\sqrt{3})^2\sin 60^\circ
        =\frac{\pi}{2} - \frac{3\sqrt{3}}{4}
        \]
        同理,
        \[
        [\text{弓形}APB]
        = [\text{扇形}APBO_1]-[\triangle ABO_1] 
        = \frac{1}{3} (1)^2 \pi - \frac{1}{2} (1)^2 \sin 120^\circ
        =\frac{\pi}{3} - \frac{\sqrt{3}}{4}
        \]
        故
        \[
        \frac{\text{月偏食亮面面积}}{\text{满月圆面积}} = \frac{\left( \frac{\pi}{3} - \frac{\sqrt{3}}{4} \right) - \left( \frac{\pi}{2} - \frac{3\sqrt{3}}{4} \right)}{\pi(1)^2}=\frac{\sqrt{3}}{2\pi} - \frac{1}{6}
        \]
    \end{solution}

    \question 下图中,已知 $\triangle ABC$ 为正三角形,$DEFGHIJK$ 为正八边形,且 $E$ 为 $BC$ 上一点,$CE=2,A,C,D$ 三点共线,$A,B,F,G$ 四点共线,求 $AF$。
    \begin{figure}[H]
        \centering
        \includegraphics[width=0.5\textwidth]{images/image125.png}
    \end{figure}
    \begin{solution}
        正八边形内角为$(8-2)\times 180^\circ \div 8=135^\circ$,则
        \[
        \angle EFB = 45^\circ, \angle BEF = 15^\circ, \angle CED = 30^\circ, \angle CDE = 30^\circ \Rightarrow CD = CE = 2
        \]
        在 $\triangle CDE$ 中,由余弦定理,
        \[
        DE^2=2^2 + 2^2 - 2\cdot2\cdot2\cdot\cos 120^\circ \Rightarrow DE = 2\sqrt{3}
        \]
        在 $\triangle BEF$ 中,由正弦定理,
        \[
        \frac{2\sqrt{3}}{\sin 120^\circ} = \frac{EB}{\sin 45^\circ} = \frac{BF}{\sin 15^\circ} \Rightarrow EB = 2\sqrt{2}, BF = \sqrt{6} - \sqrt{2},
        \]
        其中$\sin 15^\circ=\dfrac{\sqrt{6}-\sqrt{2}}{4}$,因此
        \[
        AF = AB + BF = CE + EB + BF = 2 + \sqrt{6} + \sqrt{2}
        \]
    \end{solution}

    \question 某圆内接六边形 $ABCDEF$,其中 $AB=BC=CD=1,DE=EF=FA=2$,求此六边形的面积。
    \ifprintanswers
    \begin{figure}[H]
        \centering
        \includegraphics[width=0.5\textwidth]{images/image136.jpg}
    \end{figure}
    \fi
    \begin{solution}
        设圆心为 $O$,半径为$r$,且
        \[
        \angle AOB = \angle BOC = \angle COD = b, \quad \angle DOE = \angle EOF = \angle FOA = a, 
        \]
        由圆周角和
        \[
        3a + 3b = 360^\circ \Rightarrow \angle EDC = a+b=120^\circ 
        \]
        在 $\triangle EDC$ 中,由余弦定理,
        \[
        EC^2 = 1^2 + 2^2 - 2\cdot 1 \cdot 2 \cdot \cos 120^\circ \Rightarrow EC = \sqrt{7}
        \]
        在 $\triangle EOC$ 中,由正弦定理,
        \[
        \frac{\sqrt{7}}{\sin 120^\circ} = 2r \Rightarrow r = \sqrt{\frac{7}{3}}
        \]
        在 $\triangle EOD$ 及$\triangle DOC$中,由余弦定理,
        \[
        \cos a = \frac{r^2 + r^2 - 1^2}{2 r^2} = \frac{11}{14}, \cos b = \frac{r^2 + r^2 - 2^2}{2 r^2} = \frac{1}{7} \Rightarrow \sin a = \frac{5\sqrt{3}}{14}, \ \sin b = \frac{4\sqrt{3}}{7}.
        \]
        故六边形$ABCDEF$面积为
        \[
        [ABCDEF] = \frac{3}{2} r^2 \sin a + \frac{3}{2} r^2 \sin b = \frac{7}{2} \left( \frac{5\sqrt{3}}{14} + \frac{4\sqrt{3}}{7} \right) = \frac{13\sqrt{3}}{4}
        \]
    \end{solution}

    \question 边长为 $1$ 的正五边形内部,去掉同时与五个顶点距离都小于 $1$ 的点后,求剩余部分的面积。
    \ifprintanswers
    \begin{figure}[H]
        \centering
        \includegraphics[width=0.5\textwidth]{images/image132.jpg}
    \end{figure}
    \fi
    \begin{solution}
        如图,此题相当于求正五边形$ABCDE$扣除青色部分的面积,该部分由一个小正五边形和五个弓形组成,首先求小正五边形边长 $PQ$,易知
        \[
        \angle PAQ = \angle QAE + \angle PAB - \angle A = 60^\circ + 60^\circ - 108^\circ = 12^\circ, \angle QAB = 60^\circ - 12^\circ = 48^\circ.
        \]
        在 $\triangle APQ$ 中,由余弦定理,
        \[
        PQ^2 = 1^2+1^2-2 \cdot 1 \cdot 1 \cdot \cos 12^\circ = 2 - 2\cos 12^\circ
        \]
        弓形面积为
        \[
        [\text{弓形}]=[\text{扇形}APQ]-[\triangle APQ] = \frac{12}{360} \cdot 1^2 \cdot \pi \cdot-\frac{1}{2} \cdot 1^2 \cdot\sin 12^\circ=\frac{\pi}{30} - \frac{1}{2}\sin 12^\circ.
        \]
        小正五边形与五个弓形组成区域 $R$ 面积
        \[
        R = \frac{5}{4} (2 - 2\cos 12^\circ) \tan 54^\circ + 5\left(\frac{\pi}{30} - \frac{1}{2}\sin 12^\circ \right) 
        \]
        所以剩余部分面积为
        \begin{align*}
        [\text{剩余部分}] &= [ABCDE]-R \\
        &= \frac{5}{4}\tan 54^\circ - \frac{5}{4} (2 - 2\cos 12^\circ) \tan 54^\circ - \frac{\pi}{6} + \frac{5}{2}\sin 12^\circ \\
        &= \frac{5}{4} \tan 54^\circ (2\cos 12^\circ - 1) + \frac{5}{2}\sin 12^\circ - \frac{\pi}{6} \\
        &={5\over 4}{\cos 36^\circ \over \sin 36^\circ}(2\cos 12^\circ-1)+{5\over 2}\sin 12^\circ -{\pi \over 6} \\
        &={5\over 4} \times {2\cos 24^\circ-\cos 36^\circ \over \sin 36^\circ}-{\pi \over 6} \\
        &= \frac{5\sqrt{3}}{4} - \frac{\pi}{6}
        \end{align*}
        其中
        \[
        \cos24^\circ=\cos60^\circ\cos36^\circ+\sin60^\circ\sin36^\circ
        =\frac12\cos36^\circ+\frac{\sqrt3}{2}\sin36^\circ
        \]
    \end{solution}
\end{questions}

\pagebreak

\begin{center}
  {\fontsize{30pt}{26pt}\selectfont
    \hypertarget{三角函数}{三角函数} \label{三角函数}
  }
\end{center}
\separator
\vspace{1pt}

\begin{questions}
    \question 已知$\cos x+\cos^2 x+\cos^3x=1$,求$\cos^3 x+\sec x$的值。
    \begin{solution}
    反复操弄得 
    \begin{align*}
        \cos^3x+\dfrac{1}{\cos x}&=\dfrac{\cos^4x+1}{\cos x}\\
        &=\dfrac{\cos^4x+\cos x+\cos^2x+\cos^3x}{\cos x}\\
        &=\cos^3x+\cos^2x+\cos x+1\\
        &=1+1\\
        &=2
    \end{align*}
    \end{solution}
    \question 已知 \(\cos\theta + 8\sin\theta = 4\)。求 $\dfrac{1+\sin\theta}{\cos\theta}$ 的值。  
    \begin{solution}
        两边平方$\cos\theta =4- 8\sin\theta$有
        \[
        1-\sin^2 \theta=16-64\sin\theta+64\sin^2\theta \Rightarrow \sin \theta= \frac{3}{5}\; \text{或} \;\frac{5}{13}
        \]
        当$\sin \theta= \dfrac{3}{5},\cos \theta= \dfrac{3}{5},\dfrac{1+\sin\theta}{\cos\theta}=-2$;
        当$\sin \theta= \dfrac{5}{13},\cos \theta= \dfrac{12}{13},\dfrac{1+\sin\theta}{\cos\theta}=\dfrac{3}{2}$,故
        \[
        \dfrac{1+\sin\theta}{\cos\theta}=-2\;\text{或}\;\dfrac{3}{2}
        \]
    \end{solution}
    
    \question 若 \(\sec x + \tan x = 2018\),求 \(\csc x + \cot x\)。  
    \begin{solution}
        已知 \(\sec x + \tan x = 2018\),则 \(\sec x - \tan x =\dfrac{\sec^2 x - \tan^2 x}{\sec x + \tan x}= \dfrac{1}{2018}\),解得
        \[
        \sec x = \frac{2018^2+1}{2\cdot 2018},\; \tan x = \frac{2018^2-1}{2\cdot 2018}
        \]
        于是
        \[
        \csc x + \cot x=\frac{\sec x}{\tan x}+\frac{1}{\tan x}=\frac{2018^2+1+2\cdot2018}{2018^2-1}=\frac{2019^2}{2018^2-1}
        \]
    \end{solution}

    \question 已知 $\sin (1+\cos^2 x+\sin^4 x)=\dfrac{13}{14}$,求 $\cos (1+\sin^2 x+\cos^4 x)$的值。
    \begin{solution}
        记$\alpha=1+\cos^2 x+\sin^4 x, \beta=1+\sin^2 x+\cos^4 x$,发现
        \begin{align*}
        \alpha-\beta&=(1+\cos^2 x+\sin^4 x)-(1+\sin^2 x+\cos^4 x)\\
        &=\cos^2 x(1-\cos^2 x)+\sin^2 x(\sin^2 x-1)\\
        &=\cos^2 x\sin^2 x-\sin^2 x\cos^2 x\\
        &=0
        \end{align*}
        于是
        \[
        \cos^2 \beta = 1-\sin^2 \beta=1-\sin^2 \alpha= 1- \left(\frac{13}{14}\right)^2=\frac{27}{196} \Rightarrow \cos \beta=\pm \frac{3\sqrt3}{14}
        \]
    \end{solution}  

    \question 设 $a,b$ 为实数,且满足 
    \[
    \sin a+\sin b=\frac{\sqrt{2}}{2},\cos a+\cos b=\frac{\sqrt{6}}{2},
    \]
    求 $\sin(a+b)$ 的值。
    \begin{solution}
        由
        \[
        \frac{\sin a+\sin b}{\cos a+\cos b}
        =\frac{2\sin\frac{a+b}{2}\cos\frac{a-b}{2}}{2\cos\frac{a+b}{2}\cos\frac{a-b}{2}}
        =\tan\frac{a+b}{2}=\frac{1}{\sqrt{3}}
        \]
        因此
        \[
        \sin(a+b)=\frac{2\tan\frac{a+b}{2}}{1+\tan^2\frac{a+b}{2}}=\frac{\sqrt{3}}{2}
        \]
    \end{solution}
    \begin{solution}
        设复数$z_1=\cos a+i\sin a,\quad z_2=\cos b+i\sin b$,则
        \[
        z=z_1+z_2
        =\frac{\sqrt{6}}{2}+i\frac{\sqrt{2}}{2}
        =\sqrt{2}\left(\cos 30^\circ+i\sin 30^\circ\right)
        \Rightarrow \frac{a+b}{2}=30^\circ \Rightarrow a+b=60^\circ
        \]
        所以
        \[
        \sin(a+b)=\frac{\sqrt{3}}{2}
        \]
    \end{solution}

    \question 已知 $x\neq y,x\neq 0,y\neq 0$, 且$\tan(x+y)=2\tan(x-y)$,求
    \[
    \frac{\sin 2x}{\sin 2y}
    \]
    的值。
    \begin{solution}
        展开$\tan(x+y)=2\tan(x-y)$得
        \[
        \frac{\tan x+\tan y}{1-\tan x\tan y}
        =
        2\left(\frac{\tan x-\tan y}{1+\tan x\tan y}\right)\Rightarrow \tan x(1+\tan^2 y)=3\tan y(1+\tan^2 x)
        \]
        利用恒等式 $1+\tan^2\theta=\sec^2\theta$,得
        \[
        \frac{\tan x\sec^2 y}{\tan y\sec^2 x}=\frac{\sin x\cos x}{\sin y\cos y}=\frac{\sin 2x}{\sin 2y}=3
        \]
    \end{solution}

    \question
    已知三角方程
    \[
    \frac{\sin(x-\alpha)}{\cos(x-\alpha)-2\tan\alpha\sin(x-\alpha)}=\tan\alpha
    \]
    证明
    \[
    \tan x=2\tan\alpha
    \]
    \begin{solution}
        据题意有
        \[
        \frac{\sin(x-\alpha)}{\cos(x-\alpha)-2\tan\alpha\sin(x-\alpha)}=\tan\alpha
        \]
        \[
        \sin(x-\alpha)+2\tan^2\alpha\sin(x-\alpha)=\tan\alpha\cos(x-\alpha)
        \]
        \[
        \sin(x-\alpha)\left(1+2\tan^2\alpha\right)=\tan\alpha\cos(x-\alpha)
        \]
        \[
        \tan(x-\alpha)\left(1+2\tan^2\alpha\right)=\tan\alpha
        \]
        \[
        \frac{\tan x-\tan\alpha}{1+\tan x\tan\alpha}\left(1+2\tan^2\alpha\right)=\tan\alpha
        \]
        \[
        (\tan x-\tan\alpha)(1+2\tan^2\alpha)=\tan\alpha(1+\tan x\tan\alpha)
        \]
        \[
        \tan x+2\tan^2\alpha\tan x-\tan\alpha-2\tan^3\alpha
        =\tan\alpha+\tan^2\alpha\tan x
        \]
        \[
        (\tan x-2\tan\alpha)(\tan^2\alpha+1)=0
        \]
        由于 $\tan^2\alpha+1\neq0$,故
        \[
        \tan x=2\tan\alpha
        \]
        证毕。
    \end{solution}

    \question 已知 $0<x<\dfrac{\pi}{2},0<y<\dfrac{\pi}{2}$,且
    \[
    \sin(x+y)\sin(x-y)=\frac{5}{36}, \quad \cos x+\cos y=\frac{5}{6}.
    \]
    求 $\cos(x-y)$的值。
    \begin{solution}
        由
        \begin{align*}
        \frac{5}{36}=\sin(x+y)\sin(x-y) &= (\sin x\cos y + \cos x \sin y)(\sin x\cos y - \cos x \sin y) \\
        &= \sin^2 x \cos^2 y - \cos^2 x \sin^2 y \\
        &= (1-\cos^2 x)\cos^2 y - \cos^2 x (1-\cos^2 y) \\
        &= \cos^2 y - \cos^2 x 
        \end{align*}
        再由平方差公式,得
        \[
        \cos y - \cos x = \frac{1}{6}.
        \]
        联立可得
        \[
        \cos x = \frac{1}{3}, \quad \cos y = \frac{1}{2}
        \]
        由于 $x, y$ 皆为锐角,
        \[
        \sin x = \frac{2\sqrt{2}}{3}, \quad \sin y = \frac{\sqrt{3}}{2}
        \]
        使用差角公式:
        \[
        \cos(x-y) = \cos x \cos y + \sin x \sin y = \frac{1}{3}\cdot \frac{1}{2} + \frac{2\sqrt{2}}{3} \cdot \frac{\sqrt{3}}{2} = \frac{1+\sqrt{6}}{6}
        \]
    \end{solution}

    \question 已知方程
    \[
    \frac{\sin x}{\sin y} = \frac{1}{2}, \quad 
    \frac{\cos x}{\cos y} = \frac{3}{2},
    \]
    求 $\tan^2(x+y)$。
    \begin{solution}
        由
        \[
        \frac{\sin x}{\sin y} = \frac{1}{2}, \quad 
        \frac{\cos x}{\cos y} = \frac{3}{2},
        \]
        可得
        \[
        \frac{\sin x + \sin y}{\sin x - \sin y }=\frac{2\sin \frac{x+y}{2}\cos \frac{x-y}{2}}{2\sin \frac{x-y}{2}\cos \frac{x+y}{2}}=-3,\quad \frac{\cos x + \cos y}{\cos x - \cos y }=\frac{2\cos \frac{x+y}{2}\cos \frac{x-y}{2}}{-2\sin \frac{x-y}{2}\sin \frac{x+y}{2}}=5 ,
        \]
        故
        \[
        \tan^2 \frac{x+y}{2}=\frac{3}{5}
        \]
        又
        \[
        \tan (x+y)=\frac{2\tan \frac{x+y}{2}}{1-\frac{3}{5}}=5\tan\frac{x+y}{2}
        \]
        所以
        \[
        \tan^2 (x+y)=25\cdot \frac{3}{5}=15
        \]
    \end{solution}

    \question 设 $\beta \in \left[\dfrac{3\pi}{4}, \pi\right]$ 且满足
    \[
    \cos\alpha + \sin(\alpha+\beta) + \cos(\alpha+\beta) = \sqrt{3}
    \]
    求 $\beta$ 的值。
    \begin{solution}
        \begin{align*}
        \sqrt{3} &= \cos\alpha + \sin(\alpha+\beta) + \cos(\alpha+\beta)\\
        &= \cos\alpha + \sin\alpha \cos\beta + \cos\alpha \sin\beta + \cos\alpha \cos\beta - \sin\alpha \sin\beta \\
        &= \cos\alpha(1 + \sin\beta + \cos\beta) + \sin\alpha(\cos\beta - \sin\beta) \\
        &= \sqrt{(1 + \sin\beta + \cos\beta)^2 + (\cos\beta - \sin\beta)^2} \sin(\alpha + \gamma)\\
        &= \sqrt{3 + 2\sin\beta + 2\cos\beta}\sin(\alpha + \gamma)
        \end{align*}
        由于 $\sin(\alpha + \gamma) \le 1$,因此
        \[
        \sqrt{3 + 2\sin\beta + 2\cos\beta} \ge \sqrt3 \Rightarrow \sin\beta + \cos\beta \ge 0 \Rightarrow \sin\left(\beta + \frac{\pi}{4}\right) \ge 0
        \]
        而 $\beta + \frac{\pi}{4} \in [\pi, \frac{5\pi}{4}]\Rightarrow \sin\left(\beta + \frac{\pi}{4}\right) \le 0$,故\[\sin\left(\beta + \frac{\pi}{4}\right)=0 \Rightarrow \beta=\dfrac{3\pi}{4}\]
    \end{solution}

    \question 求解三角方程
    \[
    \sin x\sin 2x+\sin 2x\sin 3x+\sin 3x\sin 4x=0
    \]
    \begin{solution}
        原方程化为
        \[
        \sin 2x(\sin x+\sin 3x)+\sin 3x\sin 4x=0
        \]
        \[
        \sin 2x(2\sin 2x\cos x)+\sin 3x(2\sin 2x\cos 2x)=0
        \]
        \[
        2\sin 2x(\sin 2x\cos x+\sin 3x\cos 2x)=0
        \]
        \[
        2\sin 2x \sin x \left[2\cos^2 x + (3-4\sin^2 x)(1-2\sin^2 x)\right]=0
        \]
        \[
        \sin 2x \sin x \left[1+(3-4\sin^2 x)^2\right]=0
        \]
        此时$1+(3-4\sin^2 x)^2>0$,解$\sin 2x=0$即可,通解为
        \[
        x=\frac{n\pi}{2},\quad n\in\mathbb{Z}
        \]
    \end{solution}

    \question 解三角方程
    \[
    2\cos x\sin^2 x-2\cos^2 x\sin x+\cos^2 x-4\sin^2 x+3\cos x\sin x+2\sin x-2\cos x=0
    \]
    \begin{solution}
        原方程化为
        \[
        2\cos x\sin x(\sin x -\cos x)+(\cos x -\sin x)(\cos x +4\sin x)+2(\sin x-\cos x)=0
        \]
        \[
        (\sin x-\cos x)\left[2\cos x\sin x-(\cos x +4\sin x)+2\right]=0
        \]
        \[
        (\sin x-\cos x)(2\sin x-1)(\cos x-2)=0
        \]
        注意到方程$\cos x=2$无实数解,于是
        \[
        \tan x = 1\quad \text{或} \quad \sin x=\frac{1}{2}
        \]
        通解为
        \[
        x=n\pi+\frac{\pi}{4} \quad \text{或} \quad x=n\pi +(-1)^n \left(\frac{\pi}{6}\right),\quad n\in\mathbb{Z}
        \]
    \end{solution}

    \question 解方程 
    \[
    \sin^2 x + \sin^2 2x = \sin^2 3x.
    \]
    \begin{solution}
        首先有
        \begin{align*}
        \sin^2 3x - \sin^2 x 
        &= (\sin 3x + \sin x)(\sin 3x - \sin x)\\
        &=(2 \sin 2x \cos x)(2 \cos 2x \sin x)\\
        &=2 \sin^2 2x \cos 2x
        \end{align*}
        故原方程式变为
        \[
        \sin^2 2x (2 \cos 2x - 1) = 0
        \]
        解得
        \[
        x = \frac{n\pi}{2} \quad \text{或} \quad x = n\pi \pm \frac{\pi}{6},\quad n \in \mathbb{Z}
        \]
    \end{solution}

    \question 解方程 
    \[
    \cos^2 x + 3 \cos^2 2x = \cos^2 3x.
    \]
    \begin{solution}
        同上题,有
        \begin{align*}
        \cos^2 3x - \cos^2 x 
        &= (\cos 3x + \cos x)(\cos 3x - \cos x) \\
        &= (2 \cos 2x \cos x)(-2 \sin 2x \sin x) \\
        &= -2 \cos 2x \sin^2 2x \\
        &= -2 \cos 2x (1-\cos^2 2x)
        \end{align*}
        故原方程式变为
        \[
        \cos 2x (\cos 2x-2)(2\cos 2x+1) =0
        \]
        其中$\cos 2x=2$无实数解,故通解为
        \[
        x = \frac{n\pi}{2} + \frac{\pi}{4} \quad \text{或} \quad x = n\pi \pm \frac{\pi}{3},\quad n \in \mathbb{Z}
        \]
    \end{solution}
    
    \question 求解
    \[
    \sin\left(3x-\frac{\pi}{3}\right)+\sin\left(x-\frac{\pi}{3}\right)=\sin x
    \]
    \begin{solution}
        和差化积得
        \[
        2\sin\left(2x-\frac{\pi}{3}\right)\cos x=\sin x 
        \]
        \[
        \cos x (\sin2x-\sqrt3\cos2x)=\sin x
        \]
        \[ 
        \sin x(2\cos^2 x-1)=\sqrt{3}\cos 2x\cos x 
        \]
        \[
        \cos2x(\tan x-\sqrt3)=0
        \]
        故通解为
        \[
        x=n\pi \pm \frac{\pi}{4} \quad \text{或} \quad x=n\pi+\frac{\pi}{3}, \quad n\in\mathbb{Z}
        \]
    \end{solution}

    \question 解方程
    \[
    (\cos 4x + \cos x)^2 + (\sin 4x + \sin x)^2 = 2\sqrt{3}\sin 3x, \quad 0 \le x < \pi
    \]
    \begin{solution}
        首先展开左式,
        \[
        (\cos 4x+\cos x)^2+(\sin 4x+\sin x)^2 = 1 + 1 + 2\cos(4x-x) = 2 + 2\cos 3x
        \]
        原方程即
        \[
        2 + 2\cos 3x = 2\sqrt{3}\sin 3x \Rightarrow \sqrt{3}\sin 3x - \cos 3x = 1 \Rightarrow \sin\left(3x-\frac{\pi}{6}\right) = \frac{1}{2}
        \]
        故通解为
        \[
        3x - \frac{\pi}{6} = n\pi + (-1)^n\left(\frac{\pi}{6}\right) \Rightarrow x= \frac{n\pi}{3} + \frac{1+(-1)^n}{18}\pi, \quad n \in \mathbb{Z}
        \]
    \end{solution}

    \question 解三角方程式 
    \[
    16\sin^{2}\theta + \tan^{2}\theta + 9(\csc^{2}\theta + \cot^{2}\theta) = 30, 
    \]
    其中 $0^\circ \leq \theta \leq 360^\circ$。
    \begin{solution}
        原方程式即
        \[
        (4 \sin\theta - 3 \csc\theta)^2 + (\tan\theta - 3 \cot\theta)^2 = 0
        \]
        于是
        \[
        4 \sin\theta - 3 \csc\theta = 0 \Rightarrow \sin\theta = \pm\frac{\sqrt{3}}{2} \Rightarrow \theta =60^\circ, 120^\circ, 240^\circ, 300^\circ
        \]
        且
        \[
        \tan\theta - 3 \cot\theta = 0 \Rightarrow \tan\theta = \pm\sqrt{3} \Rightarrow \theta = 60^\circ, 120^\circ, 240^\circ, 300^\circ
        \]
        故原方程式的解为
        \[
        \theta = 60^\circ, 120^\circ, 240^\circ, 300^\circ
        \]
    \end{solution}

    \question 求解方程
    \[
    (x+y)^2 + 4(x+y)\cos(x-y) + 4 = 0
    \]
    \begin{solution}
        注意到
        \begin{align*}
        (x+y)^2 + 4(x+y)\cos(x-y) + 4 &= \left[(x+y) + 2\cos(x-y)\right]^2 - (2\cos(x-y))^2 + 4 \\
        &= \left[(x+y) + 2\cos(x-y)\right]^2 + \left[2\sin(x-y)\right]^2
        \end{align*}
        由于两个平方项均为实数,原方程等价于
        \[
        (x+y) + 2\cos(x-y) = 0 \tag{1}
        \]
        且
        \[
        \sin(x-y) = 0 \Rightarrow x-y = n\pi, \quad n \in \mathbb{Z}
        \]
        若 $n$ 为偶数,则$x-y = 2k\pi \Rightarrow \cos(x-y) = 1$,代入 (1) 得
        \[
        x+y = -2
        \]
        若 $n$ 为奇数,则$x-y = (2k+1)\pi \Rightarrow \cos(x-y) = -1$,代入 (1) 得
        \[
        x+y = 2
        \]
        故可解得
        \[
        x = -1+k\pi,  y = -1-k\pi, \quad \text{或} \quad
        x = 1+\frac{(2k+1)\pi}{2}, \, y = 1-\frac{(2k+1)\pi}{2}, \quad k \in \mathbb{Z}
        \]
    \end{solution}

    \question 解方程
    \[
    4\cos x\cos 2x\cos 5x+1=0.
    \]
    \begin{solution}
        原方程等价于
        \[
        4\cos x\cos 2x\cos 5x=-1
        \]
        注意到 $x=n\pi,n\in\mathbb{Z}$ 不是原方程的解,两边乘 $\sin x$得
        \[
        4\sin x\cos x\cos 2x\cos 5x=-\sin x
        \]
        即
        \[
        2\sin 2x\cos 2x\cos 5x=-\sin x \Rightarrow \sin 4x\cos 5x=-\sin x
        \]
        积化和差得
        \[
        \frac12\sin 9x-\frac12\sin x=-\sin x \Rightarrow
        \sin 9x=-\sin x
        \]
        即
        \[
        \sin 9x=\sin(-x)
        \]
        通解为
        \[
        9x=-x+2n\pi \Rightarrow x=\frac{n\pi}{5}, \quad n\in\mathbb{Z}, \quad n \not\equiv 0 \pmod 5
        \]
        或
        \[
        9x=\pi-(-x)+2n\pi \Rightarrow x=\frac{n\pi}{4}+\frac{\pi}{8}, \quad n\in\mathbb{Z}
        \]
    \end{solution}

    \question 解 
    \[
    \sin^{8}\theta + \cos^{8}\theta = \frac{17}{32}.
    \] 
    \begin{solution}
        化简左式得
        \begin{align*}
        \sin^8 \theta + \cos^8 \theta
        &=(\sin^4 \theta + \cos^4 \theta)^2 - 2 \sin^4 \theta \cos^4 \theta \\
        &=\left((\sin^2 \theta + \cos^2 \theta)^2 - 2 \sin^2 \theta \cos^2 \theta\right)^2 - 2 \sin^4 \theta \cos^4 \theta \\
        &=\left(1 - \frac{1}{2} \sin^2 2\theta\right)^2 - \frac{1}{8} \sin^4 2\theta 
        \end{align*}
        故原方程即
        \[
        4 \sin^2 2\theta - 32 \sin^2 2\theta + 15 = 0 \Rightarrow (2 \sin^2 2\theta - 1)(2 \sin^2 2\theta - 15) = 0
        \]
        于是由$\sin^2 2\theta = \dfrac{1}{2} \Rightarrow \sin 2\theta = \pm \dfrac{\sqrt{2}}{2}$,解为
        \[
        \theta = \frac{n\pi}{2} + \frac{\pi}{8}, \quad n \in \mathbb{Z}
        \]
        且由于$0 \leq \sin^2 \alpha \leq 1, \alpha \in \mathbb{R}$,方程式$\sin^2 2\theta = \dfrac{15}{2}$无解。
    \end{solution}

    \question 解方程  
    \[
    \cos\frac{4x}{3}=\cos^{2}x.
    \]
    \begin{solution}
        令 \(x = \dfrac{3t}{2}\),则 \(\dfrac{4x}{3} = 2t\),原式变为
        \[
        \cos2t = \cos^2\frac{3t}{2}
        \]
        无中生有,
        \[
        2\cos2t - 1 = 2\cos^2\frac{3t}{2} - 1 = \cos3t
        \]
        于是
        \[
        2(2\cos^2t-1) - 3 = 4\cos^3 t - 3\cos t
        \Rightarrow (4\cos^2 t - 3)(\cos t - 1) = 0
        \]
        若 \(\cos t = 1\),则 \(t = 2k\pi \Rightarrow x  = 3k\pi\);若 \(\cos^2 t = \dfrac{3}{4}\),则$t = \pm \dfrac{\pi}{6} + 2k\pi \Rightarrow x = \pm\dfrac{\pi}{4} + 3k\pi $,通解为
        \[
        x = 3k\pi \quad \text{或} \quad x = \pm\dfrac{\pi}{4} + 3k\pi,\quad k \in \mathbb{Z}
        \]
    \end{solution}

    \question 设 $x$ 满足 $0 < x < \dfrac{\pi}{2}$ 且 
    \[
    \cos \left(\frac{3}{2} \cos x\right) = \sin \left(\frac{3}{2} \sin x\right),
    \]
    求 $\sin 2x$ 的所有可能值。
    \begin{solution}
        由于 $x \in \left(0,\dfrac{\pi}{2}\right)$,所以 
        \[
        \frac{3}{2}\cos x,\frac{3}{2}\sin x \in \left(0,\frac{3}{2}\right) \subset \left(0,\frac{\pi}{2}\right)
        \]
        若 $Y,Z \in \left(0,\dfrac{\pi}{2}\right)$,则 $\cos Y = \sin Z$ 当且仅当 $Y+Z=\dfrac{\pi}{2}$,因此
        \[
        \cos\left(\frac{3}{2}\cos x\right) = \sin\left(\frac{3}{2}\sin x\right)
        \quad \Longleftrightarrow \quad
        \frac{3}{2}\cos x + \frac{3}{2}\sin x = \frac{\pi}{2}
        \]
        即
        \[
        \cos x + \sin x = \frac{\pi}{3}
        \]
        两边平方得
        \[
        \sin 2x = \frac{\pi^2}{9} - 1
        \]
        为唯一可能值。
    \end{solution}
    
    \question 设 $0^\circ \le \alpha \le 45^\circ$,若已知 $\cot 2\alpha - \sqrt{3} = \sec \alpha$,求 $\alpha$。
    \begin{solution}
        化简得
        \[
        \cot(2\alpha) - \sec \alpha 
        = \frac{\cos 2\alpha}{\sin 2\alpha} - \frac{1}{\cos \alpha} 
        = \frac{1-2\sin^2\alpha}{2\sin \alpha\cos \alpha} - \frac{1}{\cos \alpha}
        \]
        原方程即
        \[
        1 - 2\sin^2\alpha - 2\sin\alpha = 2\sqrt{3} \sin \alpha \cos \alpha 
        \]
        两边平方得
        \[
        (1 - 2\sin^2\alpha - 2\sin\alpha)^2 = 12\sin^2\alpha(1-\sin^2\alpha)
        \]
        \[
        16\sin^4\alpha + 8\sin^3\alpha - 12\sin^2\alpha - 4\sin\alpha + 1 = 0
        \]
        \[
        -4\sin\alpha(3\sin\alpha - 4\sin^3\alpha) - (-8\sin^3\alpha + 6\sin\alpha) + 2\sin\alpha + 1 = 0
        \]
        \[
        -4\sin\alpha \sin 3\alpha - 2\sin 3\alpha + 2\sin\alpha + 1 = 0
        \]
        \[
        -2\sin 3\alpha (2\sin\alpha + 1) + 2\sin\alpha + 1 = 0
        \]
        \[
        (1 - 2\sin 3\alpha)(2\sin\alpha + 1) = 0
        \]
        故
        \[
        \sin 3\alpha = \frac{1}{2} \Rightarrow 3\alpha = 30^\circ \Rightarrow \alpha = 10^\circ
        \]
    \end{solution}

    \question 
    \begin{parts}
    \part 已知$\theta \neq 2n\pi-\dfrac{\pi}{2},\ n\in \mathbb{Z}$,证明
    \[
    \tan^2\left(\frac{\pi}{4}-\frac{\theta}{2}\right) \equiv \frac{1-\sin\theta}{1+\sin\theta}
    \]
    \begin{solution}
        有
        \begin{align*}
        \tan^2\left(\frac{\pi}{4}-\frac{\theta}{2}\right) 
        &= \left(\frac{1-\tan\frac{\theta}{2}}{1+\tan\frac{\theta}{2}}\right)^2 \\
        &= \left(\frac{\cos\frac{\theta}{2}-\sin\frac{\theta}{2}}{\cos\frac{\theta}{2}+\sin\frac{\theta}{2}}\right)^2 \\
        &= \frac{1-2\sin\frac{\theta}{2}\cos\frac{\theta}{2}}{1+2\sin\frac{\theta}{2}\cos\frac{\theta}{2}} \\
        &= \frac{1-\sin\theta}{1+\sin\theta}.
        \end{align*}
        故左式=右式,恒等式得证。
    \end{solution}
    \part 解方程
    \[
    \tan\left(\frac{\pi}{4}-2x\right)=\sqrt{7+4\sqrt{3}},\quad 0\le x<\pi.
    \]
    \begin{solution}
        由(a),取 $\theta = 4x$,则
        \[
        \tan^2\left(\frac{\pi}{4}-2x\right) = \frac{1-\sin 4x}{1+\sin 4x}
        \]
        原方程两边平方得
        \[
        \frac{1-\sin 4x}{1+\sin 4x} = 7+4\sqrt{3} \Rightarrow \sin 4x = -\frac{\sqrt{3}}{2}
        \]
        所以通解为
        \[
        4x = n\pi+(-1)^n\left(-\frac{\pi}{3}\right) \Rightarrow x = \frac{n\pi}{4}+(-1)^n\left(-\frac{\pi}{12}\right), \quad n\in\mathbb{Z}
        \]
        且慢,注意到
        \[
        x=\frac{n\pi}{2}+\frac{\pi}{3}, \quad n\in\mathbb{Z}
        \]
        为增根,舍去之,于是只取
        \[
        x=\frac{n\pi}{2}-\frac{\pi}{12}, \quad n\in\mathbb{Z}
        \]
        在区间 $0\le x<\pi$ 内的解有
        \[
        x = \frac{5\pi}{12},\ \frac{11\pi}{12}
        \]
    \end{solution}
    \end{parts}

    \question 
    \begin{parts}
    \part 已知 $\theta \neq 2n\pi+\dfrac{\pi}{2},k\in\mathbb{Z}$,证明三角恒等式
    \[
    \tan\left(\frac{\theta}{2}+\frac{\pi}{4}\right)\equiv\tan\theta+\sec\theta
    \]
    \begin{solution}
        从左式开始,有
        \[
        \tan\left(\frac{\theta}{2}+\frac{\pi}{4}\right)
        =\frac{\tan\frac{\theta}{2}+1}{1-\tan\frac{\theta}{2}}=\frac{\sin\frac{\theta}{2}+\cos\frac{\theta}{2}}{\cos\frac{\theta}{2}-\sin\frac{\theta}{2}}
        \]
        将分子分母同乘 $\cos\dfrac{\theta}{2}+\sin\dfrac{\theta}{2}$,得
        \[
        =\frac{\sin^2\frac{\theta}{2}+2\sin\frac{\theta}{2}\cos\frac{\theta}{2}+\cos^2\frac{\theta}{2}}
        {\cos^2\frac{\theta}{2}-\sin^2\frac{\theta}{2}}
        =\frac{1+\sin\theta}{\cos\theta}=\sec\theta+\tan\theta
        \]
        于是左式等于右式,恒等式得证。
    \end{solution}
    \part 已知
    \[
    \tan x-\tan(x-\alpha)=2\tan\alpha
    \]
    其中 $\alpha$ 为常数。以 $\alpha$ 的三角函数表示 $\tan x$。
    \begin{solution}
        可得
        \[
        \tan x-\frac{\tan x-\tan\alpha}{1+\tan x\tan\alpha}=2\tan\alpha \Rightarrow \tan^2x-2\tan x\tan\alpha-1=0
        \]
        配方得
        \[
        (\tan x-\tan\alpha)^2=1+\tan^2\alpha=\sec^2\alpha
        \]
        因此
        \[
        \tan x=\tan\alpha\pm\sec\alpha
        \]
    \end{solution}
    \part 解三角方程
    \[
    \tan x-\tan\left(x-\frac{3\pi}{5}\right)=2\tan\frac{3\pi}{5},\quad 0\le x<2\pi
    \]
    \begin{solution}
        取 $\alpha=\dfrac{3\pi}{5}$,由 (b) 得
        \[
        \tan x=\tan\frac{3\pi}{5}\pm\sec\frac{3\pi}{5}
        \]
        且由(a),以$\theta-\pi$代替$\theta$有恒等式
        \[
        \tan\left(\frac{\theta}{2}-\frac{\pi}{4}\right)\equiv\tan\theta-\sec\theta
        \]
        因此
        \[
        \tan\frac{3\pi}{5}+\sec\frac{3\pi}{5}
        =\tan\left(\frac{3\pi}{10}+\frac{\pi}{4}\right)
        =\tan\frac{11\pi}{20}
        \]
        \[
        \tan\frac{3\pi}{5}-\sec\frac{3\pi}{5}
        =\tan\left(\frac{3\pi}{10}-\frac{\pi}{4}\right)
        =\tan\frac{\pi}{20}
        \]
        于是
        \[
        \tan x=\tan\frac{11\pi}{20}\quad \text{或}\quad \tan x=\tan\frac{\pi}{20}
        \]
        在 $0\le x<2\pi$ 内解得
        \[
        x=\frac{\pi}{20},\ \frac{11\pi}{20},\ \frac{21\pi}{20},\ \frac{31\pi}{20}
        \]
    \end{solution}
    \end{parts}

    \question 设 $0 \le x \le \pi$,求 
    \[
    1+\sqrt{\sin x} - \sqrt{x} = \cos 2x + 2x^2
    \] 
    的实根个数。
    \begin{solution}
        有
        \begin{align*}
        1+\sqrt{\sin x}- \sqrt x = 1-2\sin^2 x+2x^2& \\
        \sqrt{\sin x}-\sqrt x +2(\sin^2 x-x^2)&=0 \\
        \sqrt{\sin x}-\sqrt x +2(\sin x+x)(\sin x-x) &=0 \\
        \sqrt{\sin x}-\sqrt x +2(\sin x+x)(\sqrt{\sin x}-\sqrt x)(\sqrt{\sin x}+\sqrt x)&=0 \\
        (\sqrt{\sin x}-\sqrt x)\left[2(\sin x+x) (\sqrt{\sin x}+\sqrt x)+1\right]&=0
        \end{align*}
        由 $\sqrt{\sin x}=\sqrt x$ 得 $x=0$;由 $0\le x\le \pi \Rightarrow \sin x \ge 0$,可知 
        \[
        2(\sin x+x) (\sqrt{\sin x}+\sqrt x)+1>0
        \] 恒成立,因此原方程式只有 $1$ 个实根。
    \end{solution}

    \question 已知 $\theta,\alpha,\beta$ 为相异实数,且满足
    \[
    \tan(\theta-\alpha)+\tan(\theta-\beta)=x, \quad 
    \cot(\theta-\alpha)+\cot(\theta-\beta)=y.
    \]
    试以$x,y$表示 $\tan(\alpha-\beta)$。
    \begin{solution}
        不妨设$A = \tan(\theta-\alpha), B = \tan(\theta-\beta)$,由已知得
        \[
        A+B = x, \quad \frac{1}{A}+\frac{1}{B} = y \Rightarrow AB = \frac{x}{y}
        \]
        故
        \begin{align*}
        \tan(\alpha-\beta) 
        &= \tan[(\theta-\beta) - (\theta-\alpha)] \\
        &= \frac{\tan(\theta-\beta)-\tan(\theta-\alpha)}{1+\tan(\theta-\beta)\tan(\theta-\alpha)} \\
        &= \frac{B-A}{1+AB} \\
        &= \pm \frac{\sqrt{(B-A)^2}}{1+AB} \\
        &= \pm \frac{\sqrt{(A+B)^2 - 4AB}}{1+AB} \\
        &= \pm \frac{\sqrt{x^2 - \frac{4x}{y}}}{1 + \frac{x}{y}} \\
        &= \pm \frac{\sqrt{xy(xy - 4)}}{x+y}
        \end{align*}
    \end{solution}

    \question 已知正数 $x, y$ 及角 $\theta \neq \dfrac{n\pi}{2},n \in \mathbb{Z}$满足
    \[
    \begin{cases}
    \displaystyle \frac{\sin \theta}{x} = \frac{\cos \theta}{y} \\[6pt]
    \displaystyle \frac{\cos^4 \theta}{x^4} + \frac{\sin^4 \theta}{y^4} = \frac{7 \sin 2\theta}{x^3y + xy^3}
    \end{cases}
    \]
    求 $\dfrac{x}{y} + \dfrac{y}{x}$。
    \begin{solution}
        令 
        \[
        \frac{\sin \theta}{x} = \frac{\cos \theta}{y}=\frac{1}{k}
        \]
        即$x = k \sin \theta, y = k \cos \theta$,则
        \[
        \frac{\cos^4 \theta}{\sin^4 \theta} + \frac{\sin^4 \theta}{\cos^4 \theta} = \frac{14 \sin \theta \cos \theta}{\sin \theta \cos \theta (\sin^2 \theta + \cos^2 \theta)} = 14
        \]
        设 
        \[
        S = \frac{x}{y} + \frac{y}{x} = \tan \theta + \cot \theta
        \]
        注意到
        \[
        (S^2-2)^2-2=\tan^4 \theta + \cot^4 \theta=14
        \]
        故
        \[
        S = \sqrt{6} >0
        \]
    \end{solution}

    \question 定义 $f(x) = \sin^6 x + \cos^6 x + k (\sin^4 x + \cos^4 x)$,其中 $k \in \mathbb{R}$。
    \begin{parts}
    \part 求所有$k$使得 $f(x)$ 对任意 $x$ 恒为常数。
    \begin{solution}
        利用 $\sin^2x+\cos^2x=1$,设 $u=\sin^2x$,则
        \begin{align*}
        f(x)&=\sin^6x+(1-u)^3+k(\sin^4x+(1-u)^2)\\
        &=(1+k)-(3+2k)u+(3+2k)u^2。
        \end{align*}
        仅当 $3+2k=0$,即 $k=-\dfrac{3}{2}$时,$f(x)=1+k=-\dfrac{1}{2}$ 为常数。
        \end{solution}
    \begin{solution}
        由分解
        \begin{align*}
        f(x)&=(\sin^2x+\cos^2x)((\sin^2x+\cos^2x)^2-3\sin^2x\cos^2x) \\
        &+k((\sin^2x+\cos^2x)^2-2\sin^2x\cos^2x)
        \end{align*}
        整理得
        \[
        f(x)=(1+k)-(3+2k)\sin^2x\cos^2x
        \]
        仅当 $3+2k=0$,即 $k=-\dfrac{3}{2}$,有 $f(x)=-\dfrac{1}{2}, \forall x \in \mathbb{R}$
    \end{solution}
    \begin{solution}
        直接对 $f(x)$ 求导:
        \[
        f'(x)=2\sin x\cos x(\sin^2x-\cos^2x)(3+2k)。
        \]
        欲使 $f'(x)\equiv 0$,必须有 $3+2k=0$,即 $k=-\dfrac{3}{2}$。  
    \end{solution}
    \part 若 $k=-0.7$,求方程 $f(x)=0$ 的所有解。
    \begin{solution}
       当 $k=-0.7$ 时,由(a)解法二,
        \[
        f(x)=(1+k)-(3+2k)\sin^2x\cos^2x=0.3-1.6\sin^2x\cos^2x
        \]
        令 $u=\sin^2x$,则 $\cos^2x=1-u$,方程$f(x)=0$即
        \[
        1.6u(1-u)=0.3,
        \]
        解得
        \[
        u=\sin^2x=\frac{1}{4},\ \frac{3}{4} \Rightarrow \sin x=\pm \frac{1}{2},\ \pm \frac{\sqrt{3}}{2}
        \]
        解集为
        \[
        x=\frac{\pi}{6}+n\pi,\ \frac{5\pi}{6}+n\pi,\ \frac{\pi}{3}+n\pi,\ \frac{2\pi}{3}+n\pi,\quad n\in\mathbb{Z}。
        \]
    \end{solution}
    \part 求所有 $k$使得$f(x)=0$有实数解。
    \begin{solution}
        由 (a) 解法一,
        \[
        f(x)=(3+2k)u^2-(3+2k)u+(1+k),\quad u=\sin^2x \in [0,1]。
        \]
        若 $k=-\tfrac{3}{2}$,则$f(x)\equiv -\tfrac{1}{2}$无解,则由方程
        \[
        u^2-u+\frac{1+k}{3+2k}=0
        \]
        解得
        \[
        u=\frac{1}{2}\pm \frac{1}{2}\sqrt{-\frac{1+2k}{3+2k}}
        \]
        且判别式满足
        \[
        \Delta=(3+2k)^2-4(3+2k)(1+k)=(3+2k)(-1-2k) \ge 0
        \]
        即 
        \[
        -\frac{3}{2}<k\leq -\frac{1}{2}
        \]
        由于 $u\in[0,1]$,经检验得当 
        \[
        -1\leq k\leq -\frac{1}{2}
        \] 
        时方程$f(x)=0$有实数解。 
        \textcolor{red}{感觉有更简洁的解法}
    \end{solution}
    \end{parts}

    \question 证明恒等式
    \[
    \frac{\sin 2\theta}{1+\cos 2\theta} \cdot \frac{\cos \theta}{1+\cos \theta} \equiv \tan \frac{\theta}{2}
    \]
        \begin{solution}
        有
        \begin{align*}
        \frac{\sin 2\theta}{1+\cos 2\theta} \cdot \frac{\cos \theta}{1+\cos \theta} 
        &=\frac{2 \sin \theta \cos \theta}{2\cos^2 \theta} \cdot \frac{\cos \theta}{1+\cos \theta} \\
        &= \frac{\sin \theta}{1+\cos \theta} \\
        &= \frac{2 \sin \frac{\theta}{2} \cos \frac{\theta}{2}}{2\cos^2 \frac{\theta}{2}} \\
        &= \tan \frac{\theta}{2}
        \end{align*}
        故左式=右式,恒等式得证。
    \end{solution}

    \question 证明三角恒等式
    \[
    \frac{1 + \tan \theta \tan 3\theta}{1 + \tan 2\theta \tan 3\theta} \equiv \frac{\cos^2 2\theta}{\cos^2 \theta}.
    \]
    \begin{solution}
        由
        \begin{align*}
        \text{左式} &= \frac{1 + \tan \theta \tan 3\theta}{1 + \tan 2\theta \tan 3\theta} \\
        &= \frac{\cos 2\theta \cos 3\theta}{\cos \theta \cos 3\theta} \cdot \frac{\cos \theta \cos 3\theta+\sin \theta \sin 3\theta}{\cos 2\theta \cos 3\theta+ \sin 2\theta \sin 3\theta} \\
        &= \frac{\cos 2\theta}{\cos \theta} \cdot \frac{\cos(3\theta - \theta)}{\cos(3\theta - 2\theta)}  \\
        &= \frac{\cos^2 2\theta}{\cos^2 \theta} \\
        &= \text{右式}
        \end{align*}
        因此恒等式成立。
    \end{solution}
    
    \question 证明恒等式
    \[
    \frac{1}{\tan 3\theta - \tan \theta} - \frac{1}{\cot 3\theta - \cot \theta} \equiv \cot 2\theta
    \]
    \begin{solution}
        化简得
        \begin{align*}
        &\frac{1}{\tan 3\theta - \tan \theta} - \frac{1}{\cot 3\theta - \cot \theta} \\
        &= \frac{\cos 3\theta \cos \theta}{\sin 3\theta \cos \theta - \sin \theta \cos 3\theta} - \frac{\sin 3\theta \sin \theta}{\sin 3\theta \cos \theta - \cos 3\theta \sin \theta} \\
        &= \frac{\cos 3\theta \cos \theta - \sin 3\theta \sin \theta}{\sin 3\theta \cos \theta - \sin \theta \cos 3\theta} \\
        &= \frac{\cos(3\theta - \theta)}{\sin(3\theta - \theta)} \\
        &= \cot 2\theta
        \end{align*}
        故左式=右式,恒等式得证。
    \end{solution}

    \question 证明恒等式
    \[
    \frac{\sec 16\theta - 1}{\sec 8\theta - 1} \equiv \frac{\tan 16\theta}{\tan 4\theta}.
    \]
    \begin{solution}
        \begin{align*}
        \frac{\sec 16\theta - 1}{\sec 8\theta - 1}
        &= \frac{(1-\cos 16\theta)\cos 8\theta}{(1-\cos 8\theta)\cos 16\theta} \\
        &= \frac{1-\cos^2 16\theta}{1+\cos 16\theta} \cdot \frac{\cos 8\theta}{\cos 16\theta} \cdot \frac{1+\cos 8\theta}{1-\cos^2 8\theta} \\
        &= \frac{\sin^2 16\theta}{2\cos^2 8\theta} \cdot \frac{\cos 8\theta}{\cos 16\theta} \cdot \frac{2\cos^2 4\theta}{\sin^2 8\theta}\\
        &= \frac{\sin^2 16\theta}{\sin 16\theta} \cdot \frac{1}{\sin 8\theta} \cdot \frac{2\cos^2 4\theta}{\cos 16\theta}\\
        &= \tan 16 \theta \cdot \frac{2 \cos^2 4\theta}{2\cos 4 \theta \sin 4 \theta}\\
        &= \frac{\tan16\theta}{\tan 4\theta}
        \end{align*}
        故左式=右式,恒等式得证。
    \end{solution}

    \question 证明恒等式
    \[
    \cot^2 \theta \left( \frac{\sec \theta - 1}{\sin \theta - 1} \right) + \sec^2 \theta \left( \frac{\sin \theta - 1}{\sec \theta - 1} \right) \equiv 0
    \]
    \begin{solution}
        由于
        \begin{align*}
        &\cot^2 \theta \left( \frac{\sec \theta - 1}{\sin \theta - 1} \right) + \sec^2 \theta \left( \frac{\sin \theta - 1}{\sec \theta - 1} \right) \\
        &= -\cot^2 \theta \left( \frac{(\sec \theta - 1)(1+\sin \theta)}{1 - \sin^2 \theta} \right) + \sec^2 \theta \left( \frac{(\sin \theta - 1)(1-\sec \theta)}{\sec^2 \theta-1} \right) \\
        &= -\cot^2 \theta \left( \frac{(1 - \sec \theta)(\sin \theta - 1)}{\cos^2 \theta} \right) + \sec^2 \theta \left( \frac{(1 - \sec \theta)(\sin \theta - 1)}{\tan^2 \theta} \right) \\
        &= \left( -\frac{1}{\sin^2 \theta} + \frac{1}{\sin^2 \theta} \right) (1 - \sec \theta)(\sin \theta - 1) \\
        &= 0
        \end{align*}
        即左式=右式,故恒等式得证。
    \end{solution}

    \question 证明 
    \[
    \sin 5x \equiv 16\sin^5 x - 20 \sin^3 x + 5 \sin x
    \]
    \begin{solution}
        展开得
        \begin{align*}
        \sin 5x
        &= \sin(2x+3x) \\
        &= \sin 2x \cos 3x + \cos 2x \sin 3x \\
        &= 2 \sin x \cos x(4\cos^3 x - 3\cos x) + (1-2\sin^2 x)(3\sin x - 4\sin^3 x) \\
        &= 2 \sin x (1-\sin^2 x)\left(4(1-\sin^2 x)-3\right) + (1-2\sin^2 x)(3\sin x - 4\sin^3 x) \\
        &= 16\sin^5 x - 20\sin^3 x + 5\sin x
        \end{align*}
        于是左式=右式,故得证。
    \end{solution}

    \question 证明
    \[
    4 (\cos 3\theta \sin^3 \theta + \sin 3\theta \cos^3 \theta) \equiv 3 \sin 4\theta
    \]
    \begin{solution}
        \begin{align*}
        4 (\cos 3\theta \sin^3 \theta + \sin 3\theta \cos^3 \theta)
        &= 4 \bigl((4\cos^3\theta - 3\cos\theta)\sin^3\theta + (3\sin\theta - 4\sin^3\theta)\cos^3\theta\bigr) \\
        &= 4 \bigl(-3\cos\theta\sin^3\theta + 3\sin\theta\cos^3\theta\bigr) \\
        &= 12 \sin\theta\cos\theta(\cos^2\theta - \sin^2\theta) \\
        &= 6 \sin 2\theta \cos 2\theta \\
        &= 3 \sin 4\theta
        \end{align*}
        于是左式=右式,恒等式得证。
    \end{solution}

    \question 证明
    \[
    \sin^4\left(\frac{\pi}{4}-\theta\right)+\sin^4\left(\frac{\pi}{4}+\theta\right)
    \equiv \frac{3}{4}-\frac{1}{4}\cos 4\theta
    \]
    \begin{solution}
        有
        \begin{align*}
        \sin^4\left(\frac{\pi}{4}-\theta\right)+\sin^4\left(\frac{\pi}{4}+\theta\right)
        &=\frac{1}{4}\left[(1-\cos\left(\frac{\pi}{2}-2\theta\right))^2+(1+\cos\left(\frac{\pi}{2}+2\theta\right))^2\right] \\
        &=\frac{1}{4}\left[2+2\sin^2 2\theta\right] \\
        &=\frac{1}{2}\left(1+\frac{1-\cos 4\theta}{2}\right) \\
        &=\frac{3}{4}-\frac{1}{4}\cos 4\theta
        \end{align*}
        即左式=右式,恒等式得证。
    \end{solution}

    \question 已知
    \[
    8 \sin^3 x \sin 2x = a \cos 5x + b \cos 3x + c \cos x,
    \]
    求实数 $a, b, c$ 的值。
    \begin{solution}
    \noindent
        对左式化简,
        \begin{align*} 
        8 \sin^3 x \sin 2x 
        &= 4 \sin 2x \sin x \cdot 2 \sin^2 x \\ 
        &= 4 \sin 2x \sin x (1 - \cos 2x) \\ 
        &= 4 \sin 2x \sin x - 4 \sin 2x \cos 2x \sin x \\ 
        &= -2 (-2 \sin 2x \sin x) - 2 \sin 4x \sin x \\ 
        &= -2 (\cos 3x - \cos x) + \cos 5x - \cos 3x \\ 
        &= \cos 5x - 3 \cos 3x + 2 \cos x
        \end{align*}
        于是得到
        \[
        a = 1, \quad b = -3, \quad c = 2
        \]
    \end{solution}

    \question 证明恒等式
    \[
    32 \sin^4 x \cos^2 x \equiv 2 - \cos 2x - 2 \cos 4x + \cos 6x,
    \]
    \begin{solution}
        化简右边得
        \begin{align*}
        2 - \cos 2x - 2 \cos 4x + \cos 6x 
        &= 2(1 - \cos 4x) - 2 \sin 4x \sin 2x \\
        &= 2(2 \sin^2 2x) - 2 (2 \sin 2x \cos 2x) \sin 2x \\
        &= 4 \sin^2 2x (1 - \cos 2x) \\
        &= 4 (2 \sin x \cos x)^2 \cdot 2 \sin^2 x \\
        &= 32 \sin^4 x \cos^2 x
        \end{align*}
        所以恒等式得证。
    \end{solution}

    \question 将$16 \sin^5 \theta $表示成$a \sin \theta + b \sin 3\theta + c \sin 5\theta$的形式,其中 $a, b, c$ 为实数。
    \begin{solution}
        有
        \begin{align*}
        16 \sin^5 \theta 
        &= 16 \sin^3 \theta (1 - \cos^2 \theta) \\
        &= 16 \sin^3 \theta - 16 \sin^3 \theta \cos^2 \theta \\
        &= 16 \sin \theta (1 - \cos^2 \theta) - 4 \sin \theta \sin^2 2\theta \\
        &= 16 \sin \theta - 16 \sin \theta \cos^2 \theta - 4 \sin \theta \sin^2 2\theta \\
        &= 14 \sin \theta - 8 \sin 2\theta \cos \theta + 2 \sin \theta (1 - 2 \sin^2 2\theta) \\
        &= 14 \sin \theta - 4 (\sin 3\theta + \sin \theta) + 2 \sin \theta \cos 2\theta \\
        &= 14 \sin \theta - 4 \sin 3\theta - 4 \sin \theta + \sin 3\theta - \sin \theta \\
        &= 10 \sin \theta - 5 \sin 3\theta + \sin 5\theta
        \end{align*}
    \end{solution}

    \question 证明恒等式
    \[
    \tan \theta + \tan(\theta + 120^\circ) + \tan(\theta + 240^\circ) = 3 \tan 3\theta.
    \]
    \begin{solution}
        由左式化简得
        \begin{align*}
        &\tan \theta + \tan(\theta + 120^\circ) + \tan(\theta + 240^\circ) \\
        &= \tan \theta + \frac{\tan \theta - \sqrt{3}}{1 + \sqrt{3} \tan \theta} + \frac{\tan \theta + \sqrt{3}}{1 - \sqrt{3} \tan \theta} \\
        &= \tan \theta + \frac{(\tan \theta - \sqrt{3})(1 - \sqrt{3} \tan \theta) + (\tan \theta + \sqrt{3})(1 + \sqrt{3} \tan \theta)}{1 - 3 \tan^2 \theta} \\
        &= \tan \theta + \frac{8 \tan \theta}{1 - 3 \tan^2 \theta} \\
        &= \frac{9 \tan \theta - 3 \tan^3 \theta}{1 - 3 \tan^2 \theta} \\
        &= 3 \left( \frac{3 \tan \theta - \tan^3 \theta}{1 - 3 \tan^2 \theta} \right) \\
        &= 3 \tan 3\theta
        \end{align*}
        故恒等式得证。
    \end{solution}

    \question 已知
    \[
    x=\csc\theta-\sin\theta,\quad y=\sec\theta-\cos\theta,\quad 0<\theta<\frac{\pi}{2}
    \]
    利用三角恒等式证明
    \[
    y^2x^2\left(x^{\frac{2}{3}}+y^{\frac{2}{3}}\right)^3=1
    \]
    \begin{solution}
        发现
        \[
        x=\csc\theta-\sin\theta=\frac{1}{\sin\theta}-\sin\theta=
        \frac{1-\sin^2\theta}{\sin\theta}
        =\frac{\cos^2\theta}{\sin\theta}
        \]
        同理有
        \[
        y=\sec\theta-\cos\theta=\frac{1}{\cos\theta}-\cos\theta
        =\frac{1-\cos^2\theta}{\cos\theta}
        =\frac{\sin^2\theta}{\cos\theta}
        \]
        于是
        \[
        \frac{y}{x}
        =\frac{\sin^3\theta}{\cos^3\theta}
        =\tan^3\theta \Rightarrow \tan\theta=\left(\frac{y}{x}\right)^{\frac{1}{3}}
        \]
        再由$y=\tan\theta\sin\theta$两边平方得
        \[
        y^2=\tan^2\theta\sin^2\theta
        =\tan^2\theta\left(1-\frac{1}{\sec^2\theta}\right)
        =\tan^2\theta\left(1-\frac{1}{1+\tan^2\theta}\right)
        =\frac{\tan^4\theta}{1+\tan^2\theta}
        \]
        将 $\tan\theta=\left(\dfrac{y}{x}\right)^{\frac{1}{3}}$代入得,
        \[
        y^2=\frac{\left(\frac{y}{x}\right)^{\frac{4}{3}}}{1+\left(\frac{y}{x}\right)^{\frac{2}{3}}}
        \]
        整理得
        \[
        x^{\frac{4}{3}}y^{\frac{2}{3}}(x^{\frac{2}{3}}+y^{\frac{2}{3}})=1
        \]
        两边立方得
        \[
        y^2x^2\left(x^{\frac{2}{3}}+y^{\frac{2}{3}}\right)^3=1
        \]
        证毕。
    \end{solution}

    \question 证明以下恒等式:
    \begin{parts}
    \part 
    \[
    \sin x + \sin\left(x + \frac{2\pi}{3}\right) + \sin\left(x + \frac{4\pi}{3}\right) \equiv 0
    \]
    \begin{solution}
        使用和角公式,
        \begin{align*}
        \sin x + \sin\left(x + \frac{2\pi}{3}\right) + \sin\left(x + \frac{4\pi}{3}\right) 
        &= \sin x + 2 \sin(x + \pi) \cos\frac{\pi}{3} \\
        &= \sin x - \sin x  \\
        &= 0
        \end{align*}
    \end{solution}
    \part 
    \[
    \sin^3 x + \sin^3\left(x + \frac{2\pi}{3}\right) + \sin^3\left(x + \frac{4\pi}{3}\right) \equiv -\frac{3}{4} \sin 3x
    \]
    \end{parts}
    \begin{solution}
        使用立方正弦公式 
        \[
        \sin^3 \theta = \frac{1}{4}\left(3 \sin \theta - \sin 3\theta\right),
        \]
        \begin{align*}
        & \sin^3 x + \sin^3\left(x + \frac{2\pi}{3}\right) + \sin^3\left(x + \frac{4\pi}{3}\right) \\
        &= \frac{1}{4} \left[3 \left[\sin x + \sin\left(x + \frac{2\pi}{3}\right) + \sin\left(x + \frac{4\pi}{3}\right)\right] - (\sin 3x + \sin (3x+2\pi) + \sin (3x+4\pi)) \right] \\
        &= \frac{1}{4} \left[3 \cdot 0 - 3 \sin 3x \right] \\
        &= -\frac{3}{4} \sin 3x
        \end{align*}
    \end{solution}

    \question 证明下列恒等式:
    \begin{parts}
    \part 
    \[
    \cos \theta + \cos\left(\theta + \frac{2\pi}{3}\right) + \cos\left(\theta + \frac{4\pi}{3}\right) \equiv 0
    \]
    \begin{solution}
        和差化积得
        \begin{align*}
        \cos \theta + \cos\left(\theta + \frac{2\pi}{3}\right) + \cos\left(\theta + \frac{4\pi}{3}\right) 
        &= \cos \theta + 2 \cos\left(\theta + \pi\right)\cos\frac{\pi}{3} \\
        &= \cos \theta - \cos \theta \\
        &= 0
        \end{align*}
        故恒等式得证。
    \end{solution}
    \part 
    \[
    \cos^2 \theta + \cos^2\left(\theta + \frac{2\pi}{3}\right) + \cos^2\left(\theta + \frac{4\pi}{3}\right) \equiv \frac{3}{2}
    \]
    \begin{solution}
        由平方公式 
        \[
        \cos^2 x = \frac{1}{2}(1+\cos 2x),
        \]
        则
        \begin{align*}
        &\cos^2 \theta + \cos^2\left(\theta + \frac{2\pi}{3}\right) + \cos^2\left(\theta + \frac{4\pi}{3}\right)\\
        &= \frac{1}{2} \left[3 + \cos 2\theta + \cos\left(2\theta + \frac{4\pi}{3}\right) + \cos\left(2\theta + \frac{8\pi}{3}\right) \right] \\
        &= \frac{1}{2} \left[3 + \cos 2\theta + 2\cos (2\theta+2\pi)\cos \frac{2\pi}{3} \right] \\
        &= \frac{1}{2} \left[3 + \cos 2\theta - \cos 2\theta \right] \\ 
        &= \frac{3}{2}
        \end{align*}
        因此恒等式得证。
    \end{solution}
    \part 
    \[
    \cos^3 \theta + \cos^3\left(\theta + \frac{2\pi}{3}\right) + \cos^3\left(\theta + \frac{4\pi}{3}\right) \equiv \frac{3}{4} \cos 3\theta
    \]
    \begin{solution}
        由三次余弦公式,
        \[
        \cos^3 x = \frac{1}{4} \left(3\cos x + \cos 3x\right).
        \]
        因此:
        \begin{align*}
        &\cos^3 \theta + \cos^3\left(\theta + \frac{2\pi}{3}\right) + \cos^3\left(\theta + \frac{4\pi}{3}\right) \\ 
        &= \frac{1}{4} \left[3\left[\cos \theta + \cos\left(\theta + \frac{2\pi}{3}\right) + \cos\left(\theta + \frac{4\pi}{3}\right)\right]+\cos 3\theta + \cos (3\theta+2\pi) + \cos (3\theta+4\pi) \right] \\
        &= \frac{1}{4} \left[3 \cdot 0 + 3 \cos 3\theta \right] \\
        &= \frac{3}{4} \cos 3\theta
        \end{align*}
        故恒等式得证。
    \end{solution}
    \end{parts}

    \question 已知$A,B,C$为$\triangle ABC$的内角,证明恒等式
    \begin{parts}
    \part 
    \[
    \sin 2A - \sin 2B + \sin 2C = 4 \cos A \sin B \cos C
    \]
    \begin{solution}
        已知 $A + B + C = 180^\circ$,所以 
        \[
        \sin(A+B)=\sin C, \quad \cos(A + B) = -\cos C
        \]
        于是
        \begin{align*}
        \sin 2A - \sin 2B + \sin 2C
        &= 2\cos(A+B)\sin(A-B) + 2\sin C\cos C \\
        &= -2\cos C\sin(A-B) + 2\sin C\cos C \\
        &= 2\cos C[\sin C - \sin(A-B)] \\
        &= 2\cos C[\sin(A+B) - \sin(A-B)] \\
        &= 2\cos C \cdot 2\cos A\sin B \\
        &= 4\cos A\sin B\cos C
        \end{align*}
        故得证。
    \end{solution}
    \part 
    \[
    \sin A + \sin B + \sin C = 4 \cos \frac{A}{2} \cos \frac{B}{2} \cos \frac{C}{2}
    \]
    \begin{solution}
        已知 $A + B + C = 180^\circ$,所以 
        \[
        \sin\frac{A+B}{2}=\cos \frac{C}{2}, \quad \cos\frac{A+B}{2}=\sin \frac{C}{2}
        \]
        故
        \begin{align*}
        \sin A + \sin B + \sin C &= 2 \sin \frac{A+B}{2} \cos \frac{A-B}{2} + \sin C \\
        &= 2 \cos \frac{C}{2} \cos \frac{A-B}{2} + 2 \sin \frac{C}{2} \cos \frac{C}{2} \\
        &= 2 \cos \frac{C}{2} \left(\cos \frac{A-B}{2} + \sin \frac{C}{2}\right) \\
        &= 2 \cos \frac{C}{2} \left(\cos \frac{A-B}{2} + \cos \frac{A+B}{2}\right) \\
        &= 2 \cos \frac{C}{2} \cdot 2 \cos \frac{A}{2} \cos \frac{B}{2} \\
        &= 4 \cos \frac{A}{2} \cos \frac{B}{2} \cos \frac{C}{2}
        \end{align*}
        于是恒等式得证。
    \end{solution}
    \part 
    \[
    \cos A - \cos B + \cos C = 4 \cos \frac{A}{2} \sin \frac{B}{2} \cos \frac{C}{2} - 1
    \]
    \begin{solution}
        已知 $A + B + C = 180^\circ$,所以 
        \[
        \cos\frac{A+B}{2}=\sin \frac{C}{2}, \quad \sin\frac{A+B}{2}=\cos \frac{C}{2}
        \]
        故
        \begin{align*}
        \cos A - \cos B + \cos C 
        &= -2 \sin \frac{A+B}{2} \sin \frac{A-B}{2} + \cos C \\
        &= -2 \cos \frac{C}{2} \sin \frac{A-B}{2} + 2 \cos^2 \frac{C}{2} - 1 \\
        &= 2 \cos \frac{C}{2} \left(-\sin \frac{A-B}{2} + \cos \frac{C}{2}\right) - 1 \\
        &= 2 \cos \frac{C}{2} \left(-\sin \frac{A-B}{2} + \sin \frac{A+B}{2}\right) - 1 \\
        &= 2 \cos \frac{C}{2} \cdot 2 \cos \frac{A}{2} \sin \frac{B}{2} - 1 \\
        &= 4 \cos \frac{A}{2} \sin \frac{B}{2} \cos \frac{C}{2} - 1
        \end{align*}
        得证。
    \end{solution}
    \part 
    \[
    (\cot B + \cot C)(\cot C + \cot A)(\cot A + \cot B) = \csc A \csc B \csc C
    \]
    \begin{solution}
        由$B+C = 180^\circ - A$,所以 $\sin(B+C) = \sin A$,于是
        \[
        \cot B + \cot C = \frac{\cos B \sin C + \sin B \cos C}{\sin B \sin C} = \frac{\sin(B+C)}{\sin B \sin C}=\frac{\sin A}{\sin B \sin C} 
        \]
        同理可得
        \[
        \cot C + \cot A = \frac{\sin B}{\sin C \sin A},\quad
        \cot A + \cot B = \frac{\sin C}{\sin A \sin B}
        \]
        三项相乘得
        \[
        (\cot B + \cot C)(\cot C + \cot A)(\cot A + \cot B) 
        = \frac{1}{\sin A \sin B \sin C}
        = \csc A \csc B \csc C
        \]
        故得证。
    \end{solution}
    \part 
    \[
    \sin^2 \frac{A}{2} + \sin^2 \frac{B}{2} + \sin^2 \frac{C}{2} = 1 - 2 \sin \frac{A}{2} \sin \frac{B}{2} \sin \frac{C}{2}
    \]
    \begin{solution}
        由半角公式,
        \[
        \sin^2 \frac{A}{2} + \sin^2 \frac{B}{2} + \sin^2 \frac{C}{2} = \frac{1}{2}\left[3 - (\cos A + \cos B + \cos C)\right]
        \]
        而
        \begin{align*}
        \cos A + \cos B + \cos C 
        &= 2 \cos \frac{A+B}{2} \cos \frac{A-B}{2} + \cos C \\
        &= 2 \sin \frac{C}{2} \cos \frac{A-B}{2} + 1 - 2 \sin^2 \frac{C}{2} \\
        &= 1 + 2 \sin \frac{C}{2} (\cos \frac{A-B}{2} - \sin \frac{C}{2}) \\
        &= 1 + 2 \sin \frac{C}{2} (\cos \frac{A-B}{2} - \cos \frac{A+B}{2}) \\
        &= 1 + 2 \sin \frac{C}{2} \cdot 2\sin \frac{A}{2} \sin \frac{B}{2} \\
        &= 1 + 4 \sin \frac{A}{2} \sin \frac{B}{2} \sin \frac{C}{2}
        \end{align*}
        于是得证
        \[
        \sin^2 \frac{A}{2} + \sin^2 \frac{B}{2} + \sin^2 \frac{C}{2} = \frac{1}{2}\left(3 -1 - 4 \sin \frac{A}{2} \sin \frac{B}{2} \sin \frac{C}{2}\right) = 1 - 2 \sin \frac{A}{2} \sin \frac{B}{2} \sin \frac{C}{2}
        \]
    \end{solution}
    \part 
    \[
    \cot A + \cot B + \cot C = \csc A \csc B \csc C + \cot A \cot B \cot C.
    \]
    \begin{solution}
        首先有
        \begin{align*}
        \cot A + \cot B + \cot C 
        &= \frac{\cos A}{\sin A} + \frac{\cos B}{\sin B} + \frac{\cos C}{\sin C} \\
        &= \frac{\cos A \sin B \sin C + \sin A \cos B \sin C + \sin A \sin B \cos C}{\sin A \sin B \sin C}
        \end{align*}
        其中分子为
        \begin{align*}
        & \cos A \sin B \sin C + \sin A \cos B \sin C + \sin A \sin B \cos C \\
        &= \sin C (\cos A \sin B + \sin A \cos B) - \frac{1}{2} (\cos (A+B) - \cos (A-B)) \cos C \\
        &= \sin C \sin (A+B) - \frac{1}{2} (-\cos C - \cos (A-B)) \cos C \\
        &= \sin^2 C + \frac{1}{2} \cos^2 C + \frac{1}{2} \cos (A-B) \cos C \\
        &= \sin^2 C + \cos^2 C - \frac{1}{2} \cos^2 C + \frac{1}{2} \cos (A-B) \cos C \\
        &= 1 + \frac{1}{2} \cos C (\cos (A-B) - \cos C) \\
        &= 1 + \frac{1}{2} \cos C \left(-2 \sin \frac{A-B+C}{2} \sin \frac{A-B-C}{2}\right) \\
        &= 1 - \cos C \sin \frac{A-B+C}{2} \sin \frac{A-B+C}{2} \\
        &= 1 - \cos C \sin \frac{180^\circ-2B}{2} \sin \frac{-180^\circ+2A}{2} \\
        &= 1 + \cos C \sin (90^\circ-B) \sin (90^\circ-A) \\
        &= 1 + \cos A \cos B \cos C
        \end{align*}
        于是
        \[
        \cot A + \cot B + \cot C = \frac{1 + \cos A \cos B \cos C}{\sin A \sin B \sin C} = \csc A \csc B \csc C + \cot A \cot B \cot C.
        \]
        即左式=右式,恒等式得证。\textcolor{red}{有更简洁的证明?}
    \end{solution}
    \end{parts}

    \question 若$A,B,C$为任意角,证明
    \[
    \sin A + \sin B + \sin C - \sin(A+B+C) = 4 \sin \frac{A+B}{2} \sin \frac{B+C}{2} \sin \frac{C+A}{2} 
    \]
    \begin{solution}
        \begin{align*}
        &\sin A + \sin B + \sin C - \sin(A+B+C) \\
        &= 2 \sin \frac{A+B}{2} \cos \frac{A-B}{2} + \sin C - \sin(A+B)\cos C - \cos(A+B)\sin C \\
        &= 2 \sin \frac{A+B}{2} \cos \frac{A-B}{2} - 2 \sin \frac{A+B}{2} \cos \frac{A+B}{2} \cos C + \sin C (1-\cos(A+B)) \\
        &= 2 \sin \frac{A+B}{2} \cos \frac{A-B}{2} - 2 \sin \frac{A+B}{2} \cos \frac{A+B}{2} \cos C + \sin C \left(2\sin^2 \frac{A+B}{2}\right) \\
        &= 2 \sin \frac{A+B}{2} \left( \cos \frac{A-B}{2} - \cos \frac{A+B}{2} \cos C + \sin \frac{A+B}{2} \sin C \right) \\
        &= 2 \sin \frac{A+B}{2} \left( \cos \frac{A-B}{2} - \cos \left(\frac{A+B}{2}+C\right) \right) \\
        &= 2 \sin \frac{A+B}{2} \left( 2 \sin \frac{A+C}{2} \sin \frac{B+C}{2} \right) \\
        &= 4 \sin \frac{A+B}{2} \sin \frac{B+C}{2} \sin \frac{C+A}{2}
        \end{align*}
        故得证。
    \end{solution}

    \question 若$A,B,C$为任意角,证明
    \begin{align*}
    &\cos^2 A + \cos^2 B + \cos^2 C - 2 \cos A \cos B \cos C - 1 \\
    &= -4 \sin \frac{A+B+C}{2} \sin \frac{A+B-C}{2} \sin \frac{B+C-A}{2} \sin \frac{C+A-B}{2}.
    \end{align*}
    \begin{solution}
        \begin{align*}
        &\cos^2 A + \cos^2 B + \cos^2 C - 2 \cos A \cos B \cos C - 1 \\
        &= \frac{1}{2}(2+\cos 2A+\cos 2B) + \cos^2 C - (\cos(A+B) + \cos(A-B)) \cos C - 1 \\
        &= \frac{1}{2}(\cos 2A+\cos 2B) + \cos^2 C - (\cos(A+B) + \cos(A-B)) \cos C \\
        &= \cos(A+B)\cos(A-B) + \cos^2 C - \cos C \cos(A+B) - \cos C \cos(A-B) \\
        &= (\cos(A+B) - \cos C)(\cos(A-B) - \cos C) \\
        &= \left(-2 \sin \frac{A+B+C}{2} \sin \frac{A+B-C}{2}\right) \left(2 \sin \frac{A-B+C}{2} \sin \frac{A-B-C}{2}\right) \\
        &= -4 \sin \frac{A+B+C}{2} \sin \frac{A+B-C}{2} \sin \frac{B+C-A}{2} \sin \frac{C+A-B}{2}
        \end{align*}
        故得证。
    \end{solution}

    \question 证明
    \[
    \sin 50^\circ (1 + \sqrt{3} \tan 10^\circ)=1.
    \]
    \begin{solution}
        有
        \begin{align*}
        \sin 50^\circ (1 + \sqrt{3} \tan 10^\circ) &= \sin 50^\circ \left(\frac{\cos 10^\circ+\sqrt{3} \sin 10^\circ}{\cos 10^\circ}\right) \\
        &= 2 \sin 50^\circ \left(\frac{\cos 60^\circ \cos 10^\circ + \sin 60^\circ \sin 10^\circ}{\cos 10^\circ}\right) \\
        &= 2 \sin 50^\circ \frac{\cos 50^\circ}{\cos 10^\circ} \\
        &= \frac{\sin 100^\circ}{\cos 10^\circ} \\
        &= 1
        \end{align*}
    \end{solution}

    \question 证明
    \[
    4 \sin 40^{\circ} - \tan 40^{\circ} = \sqrt{3}.
    \]
    \begin{solution}
        \begin{align*}
        \tan 40^{\circ} + \sqrt{3} 
        & = \frac{\sin 40^{\circ}}{\cos 40^{\circ}} + \frac{\sin 60^{\circ}}{\cos 60^{\circ}} \\
        & = \frac{\sin 40^{\circ} \cos 60^{\circ} + \cos 40^{\circ} \sin 60^{\circ}}{\cos 40^{\circ} \cos 60^{\circ}} \\
        & = \frac{\sin 100^{\circ}}{\cos 40^{\circ} \cdot \frac{1}{2}} \\
        & = \frac{2 \sin 100^{\circ}}{\cos 40^{\circ}} \\
        & = \frac{2 \sin 80^{\circ}}{\cos 40^{\circ}} \\
        & = 4 \sin 40^{\circ} 
        \end{align*}
        于是得证
        \[
        4 \sin 40^{\circ} - \tan 40^{\circ} = \sqrt{3}
        \]
    \end{solution}
    
    \question 证明
    \[
    \tan 24^\circ (\sqrt{3} \sec 36^\circ + \tan 6^\circ)=1.
    \]
    \begin{solution}
        先证
        \[
        \cos 36^\circ - \cos 72^\circ = \frac{1}{2}
        \]
        有
        \begin{align*}
        2\cos^2 72^\circ - 1 &= \cos 144^\circ = -\cos 36^\circ \tag{1} \\
        2\cos^2 36^\circ - 1 &= \cos 72^\circ \tag{2}
        \end{align*}
        $(2)-(1)$得
        \[
        2(\cos 36^\circ + \cos 72^\circ)(\cos 36^\circ - \cos 72^\circ) = \cos 72^\circ + \cos 36^\circ
        \]
        即
        \begin{align*}
        \cos 36^\circ - \cos 72^\circ &= \frac{1}{2} = \cos 60^\circ 
        \end{align*}
        于是
        \[
        \cos 36^\circ = \cos 72^\circ + \cos 60^\circ = 2\cos 66^\circ \cos 6^\circ
        \]
        故
        \begin{align*}
        \tan 24^\circ (\sqrt{3} \sec 36^\circ + \tan 6^\circ) 
        &= \tan 24^\circ \left( \frac{2\sin 60^\circ}{2\cos 66^\circ \cos 6^\circ} + \tan 6^\circ \right) \\
        &= \tan 24^\circ \left( \frac{\sin 66^\circ \cos 6^\circ - \cos 66^\circ \sin 6^\circ}{\cos 66^\circ \cos 6^\circ} + \tan 6^\circ \right) \\
        &= \tan 24^\circ (\tan 66^\circ - \tan 6^\circ + \tan 6^\circ) \\
        &= 1
        \end{align*}
    \end{solution}
    
    \question 证明 
    \[
    \cos 20^\circ \cos 40^\circ \cos 80^\circ = \frac{1}{8}.
    \]
    \begin{solution}
        积化和差、再和差化积,
        \begin{align*}
        \cos 20^\circ \cos 40^\circ \cos 80^\circ
        &= \frac{1}{2}\cos 20^\circ (\cos 120^\circ + \cos 40^\circ) \\
        &= \frac{1}{2}\left(-\frac{1}{2}\right)\cos 20^\circ + \frac{1}{2}\cos 40^\circ \cos 20^\circ \\
        &= -\frac{1}{4}\cos 20^\circ + \frac{1}{4}\left(\frac{1}{2}+\cos 20^\circ\right) \\
        &= \frac{1}{8}
        \end{align*}
    \end{solution}
    \begin{solution}
        由恒等式$\sin2\theta=2\sin\theta\cos\theta$,
        \begin{align*}
        \cos 20^\circ \cos 40^\circ \cos 80^\circ
        &= \frac{1}{2\sin 20^\circ}\sin 40^\circ \cos 40^\circ \cos 80^\circ \\
        &= \frac{1}{4\sin 20^\circ}\sin 80^\circ \cos 80^\circ \\
        &= \frac{1}{8\sin 20^\circ}\sin 160^\circ \\
        &= \frac{1}{8}
        \end{align*}
    \end{solution}

    \question 证明 
    \[
    \sin 20^\circ \sin 40^\circ \sin 80^\circ = \frac{\sqrt{3}}{8}.
    \]
    \begin{solution}
        积化和差、再和差化积,
        \begin{align*}
        \sin 20^\circ \sin 40^\circ \sin 80^\circ
        &= \frac{1}{2} \sin 20^\circ (\cos 40^\circ+\cos 60^\circ) \\
        &= \frac{1}{4}(\sin 60^\circ-\sin 20^\circ)+\frac{1}{4}\sin 20^\circ \\
        &= \frac{1}{4}\cdot\frac{\sqrt{3}}{2} \\
        &= \frac{\sqrt{3}}{8}
        \end{align*}
    \end{solution}
    \begin{solution}
        由恒等式 
        \[
        \sin x \sin(60^\circ-x)\sin(60^\circ+x)=\frac{1}{4}\sin 3x
        \]
        取 $x=20^\circ$,则
        \[
        \sin 20^\circ \sin 40^\circ \sin 80^\circ
        = \frac{1}{4}\sin 60^\circ
        = \frac{\sqrt{3}}{8}
        \]
    \end{solution}

    \question 计算
    \[
    \sin 10^\circ \sin 30^\circ \sin 50^\circ \sin 70^\circ
    \]
    的值。
    \begin{solution}
        由恒等式$\sin2\theta=2\sin\theta\cos\theta$,
        \begin{align*}
        \sin 10^\circ \sin 30^\circ \sin 50^\circ \sin 70^\circ
        &= \sin 10^\circ \cdot \frac{1}{2} \cdot \cos 40^\circ \cos 20^\circ \\
        &= \frac{1}{4\cos 10^\circ} \sin 20^\circ \cos 20^\circ \cos 40^\circ \\
        &= \frac{1}{8\cos 10^\circ} \sin 40^\circ \cos 40^\circ \\
        &= \frac{1}{16\cos 10^\circ} \sin 80^\circ \\
        &= \frac{1}{16}
        \end{align*}
    \end{solution}

    \question 计算
    \[
    \cos 10^\circ \cos 30^\circ \cos 50^\circ \cos 70^\circ
    \]
    的值。
    \begin{solution}
        积化和差、再和差化积,
        \begin{align*}
        \cos 10^\circ \cos 30^\circ \cos 50^\circ \cos 70^\circ
        &= \frac{1}{2}(\cos60^\circ + \cos40^\circ) \cdot \frac{\sqrt{3}}{2} \cos 70^\circ  \\
        &= \frac{\sqrt{3}}{4}\left(\frac{1}{2}\cos 70^\circ+\frac{1}{2}(\cos110^\circ + \cos30^\circ)\right) \\
        &= \frac{\sqrt{3}}{8}(\cos 70^\circ-\cos70^\circ + \frac{\sqrt{3}}{2}) \\
        &= \frac{3}{16}
        \end{align*}
    \end{solution}

    \question 证明 
    \[
    \tan 50^\circ + \tan 60^\circ + \tan 70^\circ = \tan 80^\circ.
    \]
    \begin{solution}
        若 $A+B+C=180^\circ$,则有
        \[
        \tan A \tan B \tan C = \tan A + \tan B + \tan C。
        \]
        于是
        \begin{align*}
        \tan 50^\circ + \tan 60^\circ + \tan 70^\circ
        &= \tan 50^\circ \tan 60^\circ \tan 70^\circ \\[1pt]
        &= \frac{\sin 50^\circ \sin 60^\circ \sin 70^\circ}{\cos 50^\circ \cos 60^\circ \cos 70^\circ} \\[2pt]
        &= \frac{\sin 60^\circ \left(\cos 20^\circ + \frac{1}{2}\right)}{\cos 60^\circ \left(\cos 20^\circ - \frac{1}{2}\right)} \\[2pt]
        &= \frac{\frac{1}{2}(\sin 80^\circ + \sin 40^\circ) + \frac{1}{2}\sin 60^\circ}{\frac{1}{2}(\cos 80^\circ + \cos 40^\circ) + \frac{1}{2}\cos 60^\circ} \\[2pt]
        &= \frac{\sin 40^\circ + \sin 60^\circ + \sin 80^\circ}{\cos 40^\circ - \cos 60^\circ + \cos 80^\circ} \\[2pt]
        &= \frac{2 \sin 50^\circ \cos 10^\circ + \sin 80^\circ}{2 \sin 50^\circ \sin 10^\circ + \cos 80^\circ} \\[2pt]
        &= \frac{\sin 80^\circ (2 \sin 50^\circ + 1)}{\cos 80^\circ (2 \sin 50^\circ + 1)} \\[2pt]
        &= \tan 80^\circ
        \end{align*}
    \end{solution}

    \question 计算 
    \[
    \tan 9^\circ - \tan 27^\circ - \tan 63^\circ + \tan 81^\circ
    \]
    的值。
    \begin{solution}
        首先有
        \begin{align*}
        S &=\tan 9^\circ - \tan 27^\circ - \tan 63^\circ + \tan 81^\circ \\
        &= \tan 9^\circ + \cot 9^\circ - (\tan 27^\circ + \cot 27^\circ) \\[1pt]
        &= \frac{\sin^2 9^\circ + \cos^2 9^\circ}{\sin 9^\circ \cos 9^\circ}
        - \frac{\sin^2 27^\circ + \cos^2 27^\circ}{\sin 27^\circ \cos 27^\circ} \\[2pt]
        &= \frac{1}{\sin 9^\circ \cos 9^\circ} - \frac{1}{\sin 27^\circ \cos 27^\circ} \\[2pt]
        &= \frac{2}{\sin 18^\circ} - \frac{2}{\sin 54^\circ}
        \end{align*}
        现求$\sin 18^\circ$及$\sin 54^\circ$,设$\theta=18^\circ$,解
        \[
        \sin 2\theta = \sin (90^\circ-3\theta)=\cos 3\theta
        \]
        可得
        \[
        \sin 18^\circ=\frac{\sqrt{5}-1}{4}>0,\quad \sin 54^\circ = 3\sin 18^\circ-4\sin^3 18^\circ = \frac{\sqrt{5}+1}{4}
        \]
        故
        \[
        S = \frac{2}{\dfrac{\sqrt{5}-1}{4}} - \frac{2}{\dfrac{\sqrt{5}+1}{4}}= 4
        \]
    \end{solution}

    \question 试证 
    \[\dfrac{3}{\sin^2 40^\circ} - \dfrac{1}{\cos^2 40^\circ}=32 \sin 10^\circ.
    \]
    \begin{solution}
        有
        \begin{align*}
        \frac{3}{\sin^2 40^\circ} - \frac{1}{\cos^2 40^\circ} 
        &= \frac{3 \cos^2 40^\circ - \sin^2 40^\circ}{\sin^2 40^\circ \cos^2 40^\circ} \\[6pt]
        &= 16 \cdot \frac{\left(\displaystyle\frac{\sqrt{3}}{2} \cos 40^\circ + \frac{1}{2} \sin 40^\circ\right)
        \left(\displaystyle\frac{\sqrt{3}}{2} \cos 40^\circ - \frac{1}{2} \sin 40^\circ\right)}{\sin^2 80^\circ} \\[6pt]
        &= 16 \cdot \frac{\sin(60^\circ + 40^\circ)\,\sin(60^\circ - 40^\circ)}{\sin 80^\circ \cos 10^\circ} \\[6pt]
        &= 32 \sin 10^\circ
        \end{align*}
    \end{solution}

    \question 计算 $$\sin^2 37^\circ + \sin^2 8^\circ + \sqrt{2}\sin 37^\circ \sin 8^\circ$$ 之值。
    \begin{solution}
        由
        \[
        \sin 37^\circ = \sin(45^\circ - 8^\circ) = \frac{\sqrt{2}}{2} \cos 8^\circ - \frac{\sqrt{2}}{2} \sin 8^\circ
        \]
        则
        \[
        \begin{aligned}
        \sin^2 37^\circ &= \frac{1}{2} \cos^2 8^\circ - \sin 8^\circ \cos 8^\circ + \frac{1}{2} \sin^2 8^\circ \\
        \sqrt{2}\sin 37^\circ \sin 8^\circ &= \sin 8^\circ \cos 8^\circ - \sin^2 8^\circ
        \end{aligned}
        \]
        因此
        \[
        \sin^2 37^\circ + \sin^2 8^\circ + \sqrt{2}\sin 37^\circ \sin 8^\circ = \frac{1}{2} \cos^2 8^\circ + \frac{1}{2} \sin^2 8^\circ = \frac{3}{4}
        \]
    \end{solution}
    \begin{solution}
        考虑 $\triangle ABC$ ,其中外接圆半径 $R = \dfrac{1}{2}$,且
        \[
        \angle A = 135^\circ,\quad \angle B = 37^\circ,\quad \angle C = 8^\circ
        \]
        由正弦定理,
        \[
        a = \sin 135^\circ = \frac{\sqrt{2}}{2},\quad b = \sin 37^\circ, \quad c = \sin 8^\circ
        \]
        由余弦定理,
        \[
        \cos \angle A = -\frac{\sqrt{2}}{2} = \frac{\sin^2 37^\circ + \sin^2 8^\circ - \sin^2 135^\circ}{2 \cdot \sin 37^\circ \cdot \sin 8^\circ}
        \]
        即
        \[
        \sin^2 37^\circ + \sin^2 8^\circ + \sqrt{2} \sin 37^\circ \sin 8^\circ = \frac{1}{2}
        \]
    \end{solution}
    
    \question 在 $\triangle ABC$ 中,设 ${BC}=a,\ {AC}=b,\ {AB}=c$;若 $a,b,c$ 成等差数列,试求 $\tan\dfrac{A}{2}\tan\dfrac{C}{2}$ 的值。
    \begin{solution}
        据题意有$a + c = 2b$,由正弦定理,
        \[
        \sin A + \sin C = 2\sin B 
        \]
        又因 $A + B + C = 180^\circ,$
        \[
        2\sin B = 2\sin(A + C) = \sin A + \sin C
        \]
        于是
        \[
        4\sin\frac{A + C}{2}\cos\frac{A + C}{2} = 2\sin\frac{A + C}{2}\cos\frac{A - C}{2}
        \Rightarrow 2\cos\frac{A + C}{2} = \cos\frac{A - C}{2}
        \]
        将余弦展开得
        \[
        2\left( \cos\frac{A}{2}\cos\frac{C}{2} - \sin\frac{A}{2}\sin\frac{C}{2} \right)
        =
        \cos\frac{A}{2}\cos\frac{C}{2} + \sin\frac{A}{2}\sin\frac{C}{2}
        \]
        即
        \[
        3\sin\frac{A}{2}\sin\frac{C}{2} = \cos\frac{A}{2}\cos\frac{C}{2}
        \Rightarrow
        \tan\frac{A}{2} \tan\frac{C}{2} = \frac{1}{3}
        \]
    \end{solution}

    \question 已知$\triangle ABC$中,$AB=5,BC=9$,且$\cot A, \cot B, \cot C$成等差数列,试求$\tan^{2}\dfrac{B}{2}$。
    \begin{solution}
        据题意,有
        \[
        2 \cot B = \cot A + \cot C \Rightarrow \frac{\cos A}{\sin A} + \frac{\cos C}{\sin C} = \frac{\sin (A+C)}{\sin A \sin C} = \frac{\sin B}{\sin A \sin C} = \frac{2 \cos B}{\sin B}
        \]
        由余弦定理及正弦定理,
        \[
        \cos B = \frac{a^2 + c^2 - b^2}{2 a c}, \quad \frac{\sin^2 B}{2 \sin A \sin C} = \frac{b^2}{2 a c}
        \]
        因此
        \[
        2 b^2 = a^2 + c^2 = 9^2+5^2 \Rightarrow b^2 = 53 
        \]
        故
        \[
        \cos B = \frac{53}{90} \Rightarrow \tan^2 \frac{B}{2} = \frac{1 - \cos B}{1 + \cos B}= \frac{37}{143}
        \]
    \end{solution}

    \question 若 $\triangle ABC$ 的内角 $A,B,C$ 满足 $\sin A=\cos B=\tan C$ ,求 $\cos^{3}A+\cos^{2}A-\cos A$ 的值.
    \begin{solution}
        由 $\sin A=\cos B$ 知 $$A=\dfrac{\pi}{2}\pm B$$但 $\tan C$ 有意义,故 $C$ 不为直角,从而只
        能是 $$A=\frac{\pi}{2}+B$$
        进而有 $C=\pi-A-B=\dfrac{3\pi}{2}-2A$,所以 $$\sin A=\tan C=\tan\left(\frac{3\pi}{2}-2A\right)=\cot 2A$$
        于是
        \[1 = \sin A \cdot \tan 2A = \sin A \cdot \frac{\sin 2A}{\cos 2A} = \sin A \cdot \frac{2 \sin A \cos A}{2 \cos^2 A - 1}
        = \frac{2(1 - \cos^2 A) \cdot \cos A}{2 \cos^2 A - 1}\]
        即 
        \[
        2 \cos^2 A - 1 = 2(1 - \cos^2 A) \cdot \cos A \Rightarrow \cos^3 A + \cos^2 A - \cos A = \frac{1}{2}
        \]
    \end{solution}
    
    \question 已知 \(\triangle ABC\) 满足  
        \[
          \cos C = \frac{\sin A + \cos A}{2} = \frac{\sin B + \cos B}{2},
        \]
        求 \(\cos C\)。  
    \begin{solution}
        \textbf{解法一}
        
        由条件知 $$\cos C=\frac{\sqrt{2}}{2}\sin(A+\frac{\pi}{4})=\frac{\sqrt{2}}{2}\sin(B+\frac{\pi}{4})$$
        假如 $$(A+\frac{\pi}{4})+(B+\frac{\pi}{4})=\pi$$
        则 $C=\dfrac{\pi}{2},\cos C=0$, 但 $\sin(A+\dfrac{\pi}{4})>0$, 矛盾.
        所以只可能 $$A+\frac{\pi}{4}=B+\frac{\pi}{4}$$
        此时 $A=B\in(0,\dfrac{\pi}{2}),C=\pi-2A$,且注意到 $$\cos C=\frac{\sqrt{2}}{2}\sin(A+\frac{\pi}{4})>0$$故 $C<\dfrac{\pi}{2}$. 所以 $A=B\in(\dfrac{\pi}{4},\dfrac{\pi}{2})$, 结合条件得
        \begin{align*} 
        \cos C &= -\cos 2A = -\sin\left(2A+\frac{\pi}{2}\right) = -2\sin\left(A+\frac{\pi}{4}\right)\cos\left(A+\frac{\pi}{4}\right) \\ &= -2\sqrt{2}\cos C\cdot\left(-\sqrt{1-(\sqrt{2}\cos C)^{2}}\right)
        \end{align*}
        又$\cos C>0$, 化简得 $$8(1-2\cos^{2}C)=1\Rightarrow\cos C=\dfrac{\sqrt{7}}{4}$$
    \end{solution}  
    \begin{solution}
        \textbf{解法二}
        
        将第二个等式重写为
        \[
        \sin(A+45^{\circ}) = \sin(B+45^{\circ})
        \]
        假设 $A \neq B$,则由于 $A+45^{\circ}$ 和 $B+45^{\circ}$ 均在区间 $(45^{\circ}, 225^{\circ})$ 内, 
        \[
        (A+45^{\circ}) + (B+45^{\circ}) = 180^{\circ} \implies A+B = 90^{\circ}
        \]
        此时 $\cos(C) = 0$,故必须满足 $A=B=135^{\circ}$,矛盾;因此$A=B$,故
        \[
        \cos(C) = -\cos(A+B) = \sin^2 A - \cos^2 A = \frac{\sin A + \cos A}{2}
        \iff 
        \]
        \[
        \sin A - \cos A = \frac{1}{2} \iff (\cos A)^2 + \left(\cos A + \frac{1}{2}\right)^2 = 1
        \iff
        \]
        \[
        \cos A = \frac{-1 + \sqrt{7}}{4} \iff \cos C = \frac{\sqrt{7}}{4}
        \]
    \end{solution}
    \question 已知 $\triangle ABC$ 中, $AB=3$, 且外接圆半径为 $6$, 求
        \[
        \begin{vmatrix} 
        1 & \cos A & \cos B \\ 
        \cos A & -1 & \cos C \\ 
        \cos B & \cos C & -1 
        \end{vmatrix}
        \]
    \begin{solution}
        由正弦定理
        \[
        \frac{3}{\sin C} = 2 \cdot 6 \Rightarrow \sin C = \frac{1}{4}, \cos C = \frac{\sqrt{15}}{4}
        \]
        令
        \[
        \triangle = 
        \begin{vmatrix} 
        1 & \cos A & \cos B\\ 
        \cos A & -1 & \cos C\\ 
        \cos B & \cos C & -1
        \end{vmatrix} 
        = 1 + 2\cos A \cos B \cos C + \cos^2 A + \cos^2 B - \cos^2 C
        \]
        由 $ \cos C = -\cos(A+B)$,
        \begin{align*}
        \cos^2 C &= (\cos A \cos B - \sin A \sin B)^2 \\
        &= \cos^2 A \cos^2 B - 2 \cos A \cos B \sin A \sin B + (1-\cos^2 A)(1-\cos^2 B) \\
        &= 1 + 2 \cos^2 A \cos^2 B - 2 \cos A \cos B \sin A \sin B - \cos^2 A - \cos^2 B \\
        &= 1 + 2 \cos A \cos B (\cos A \cos B - \sin A \sin B) - \cos^2 A - \cos^2 B \\
        &= 1 + 2 \cos A \cos B \cos(A+B) - \cos^2 A - \cos^2 B\\
        \end{align*}
        故
        \[
        \triangle = 2 - 2 \cos^2 C= \frac{1}{8}
        \]
    \end{solution}

   \question 令 $A, B, C$ 为任意三角形 $\triangle ABC$ 的内角,满足
    \[
    \cos^{2}A + \cos^{2}B + 2 \sin A \sin B \cos C = \frac{15}{8},
    \]
    \[
    \cos^{2}B + \cos^{2}C + 2 \sin B \sin C \cos A = \frac{14}{9}.
    \]
    求
    \[
    \cos^{2}C + \cos^{2}A + 2 \sin C \sin A \cos B.
    \]
    \begin{solution}
        由$\sin (A + C)=\sin B$,两式相加得
        \[
        \cos^{2}A + 2 \cos^{2}B + \cos^{2}C + 2 \sin B \sin (A + C) = \cos^{2}A + 2 \cos^{2}B + \cos^{2}C + 2 \sin^{2} B = \frac{247}{72}
        \]
        因此
        \[
        \cos^{2}A + \cos^{2}C = \frac{247}{72} - 2 = \frac{103}{72} \tag{1}
        \]
        同理,两式相减得
        \[
        \cos^{2}A - \cos^{2}C + 2 \sin (A+C) \sin (A - C) = \cos^{2}A - \cos^{2}C + (\cos 2C - \cos 2A) =  \frac{23}{72}
        \]
        因此
        \[
        \cos^{2}A - \cos^{2}C + 2 \cos^{2}C - 1 - (2 \cos^{2}A - 1) = -\cos^{2}A + \cos^{2}C = \frac{23}{72} \tag{2}
        \]
        由 (1),(2) 解得
        \[
        \cos^{2}A = \frac{5}{9}, \quad \cos^{2}C = \frac{7}{8}
        \]
        因此
        \begin{align*}
        &\cos^{2}C + \cos^{2}A + 2 \sin C \sin A \cos B \\
        &= \frac{103}{72} + 2 \sin C \sin A (-\cos (A + C))\\
        &= \frac{103}{72} + 2 \sin C \sin A (\sin A \sin C - \cos A \cos C) \\
        &= \frac{103}{72} + 2 \sin^{2} A \sin^{2} C - 2 \sin C \cos C \sin A \cos A\\
        &= \frac{103}{72} + 2 \cdot \frac{4}{9} \cdot \frac{1}{8} \pm 2 \cdot \frac{1}{\sqrt{8}} \cdot \sqrt{\frac{7}{8}} \cdot \frac{2}{3} \cdot \frac{\sqrt{5}}{3} = \frac{111\pm4 \sqrt{35}}{72}
        \end{align*}
    \end{solution}

    \question 已知方程 $x \in \left(0, \dfrac{\pi}{2}\right)$
    \[
    \left(\tan^2 \frac{x}{2}\right) (\cot^4 x + 1)(\csc^2 x + \tan^2 x) = 1
    \] 
    有唯一实数解。求正整数$k$满足 
    \[
    \cos^{2019} x = \sin^k x
    \]
    \begin{solution}
        由
        \[
        \left(\tan^2 \frac{x}{2}\right) (\cot^4 x + 1)(\csc^2 x + \tan^2 x) = 1
        \]
        化为
        \[
        \frac{\sin^2 x}{(1+\cos x)^2}\left(\frac{\cos^4 x}{\sin^4 x}+1\right)\left(\frac{1}{\sin^2 x}+\frac{\sin^2 x}{\cos^2 x}\right)=1 
        \]
        \[
        (\cos^4 x +\sin^4 x)(\cos^2 x +\sin^4 x)=\sin^4 x \cos^2 x (1+\cos x)^2 
        \]
        展开化简得
        \[
        (\cos^3 x -\sin^4 x)^2=0 \Rightarrow \cos^3 x = \sin^4 x
        \]
        于是
        \[
        k=4 \cdot 673=2692
        \]
    \end{solution}
    \begin{solution}
        由柯西不等式,
        \[
        \cot^2 \frac{x}{2}=(\cot^4 x + 1)( \tan^2 x+\csc^2 x) \ge (\cot x +\csc x)^2 = \left(\frac{\cos x +1}{\sin x}\right)^2=\cot^2 \frac{x}{2}
        \]
        此时等号成立,于是有
        \[
        \frac{\cot^2 x}{\tan x}=\frac{1}{\csc x} \Rightarrow \cos^3 x = \sin^4 x
        \]
        同上,
        \[
        k=4 \cdot 673=2692
        \]
    \end{solution}

    \question 设 $\theta$ 为一锐角,满足
    \[
    \frac{16}{\sin^6 \theta} + \frac{1}{\cos^6 \theta} = 81,
    \]
    求 $\tan \theta$。
    \begin{solution}
        由柯西不等式,
        \[
        \left(\frac{16}{\sin^6 \theta} + \frac{1}{\cos^6 \theta}\right)(\sin^2 \theta + \cos^2 \theta) \ge \left(\frac{4}{\sin^2 \theta} + \frac{1}{\cos^2 \theta}\right)^2 \tag{1}
        \]
        再由柯西不等式,
        \[
        \left(\frac{4}{\sin^2 \theta} + \frac{1}{\cos^2 \theta}\right)(\sin^2 \theta + \cos^2 \theta) \ge (2+1)^2 \tag{2}
        \]
        由(1)与(2)得
        \[
        \frac{16}{\sin^6 \theta} + \frac{1}{\cos^6 \theta} \ge 9^2 = 81.
        \]
        等号成立当且仅当
        \[
        \frac{4}{\sin^2 \theta} = \frac{1}{\cos^2 \theta} \Rightarrow \tan \theta = \sqrt{2}>0 
        \]
    \end{solution}

    \question 设 $\theta,\phi$ 为锐角,且
    \[
    \frac{\sin^{2024}\theta}{\cos^{2022}\phi}+\frac{\cos^{2024}\theta}{\sin^{2022}\phi}=1,
    \]
    则 $\sin^{2023}\theta-\cos^{2023}\phi=?$

    \begin{solution}
        由柯西不等式,
\[
\left(\frac{\sin^{2024}\theta}{\cos^{2022}\phi}+\frac{\cos^{2024}\theta}{\sin^{2022}\phi}\right)\left(\frac{\cos^{2022} \phi}{\sin^{2020} \theta} + \frac{\sin^{2022} \phi}{\cos^{2020} \theta}\right) \ge (\sin^2 \theta + \cos^2 \theta)^2 =1
\]
\[
\Rightarrow 1\cdot \left( \left(\frac{\cos^{1011} \phi}{\sin^{1010} \theta} \right)^2+ \left(\frac{\sin^{1011} \phi}{\cos^{1010} \theta} \right)^2 \right) \ge 1
\]

\[
\Rightarrow \left(\frac{\cos^{1011} \phi}{\sin^{1010} \theta} \right)^2+ \left(\frac{\sin^{1011} \phi}{\cos^{1010} \theta} \right)^2 = 1
\]

此时
\[
\frac{\sin ^{2022}\theta}{\cos^{2022}\phi} = \frac{\cos^{2022} \theta}{\sin^{2022} \phi} \Rightarrow \sin\theta \sin \phi = \cos \theta \cos \phi \Rightarrow \cos(\theta+\phi)=0
\]

\[
\Rightarrow \theta+\phi =\frac{\pi}{2} \Rightarrow \sin \theta= \sin\left(\frac{\pi}{2}-\phi\right)= \cos \phi
\]

\[
\Rightarrow \sin^{2023} \theta-\cos^{2023} \phi = 0
\]
\textcolor{red}{为什么第三行=1?}
\end{solution}

    \question 证明对任意\(\triangle ABC\)均有  
        \[
        a\cos A + b\cos B + c\cos C = 2a\sin B\sin C\]
    \begin{solution}
        由正弦定理,设$a=k\sin A, b=k\sin B, c=k \sin C,$其中$k\in \mathbb{R}$,左式变为\begin{align*}
            a\cos A + b\cos B + c\cos C
            &=a\cos A+ k(\sin B \cos B+\sin C \cos C)\\
            &=a\cos A+ \frac{1}{2}k(\sin 2B+\sin 2C)\\
            &=a\cos A+ k\sin(B+C)\cos(B-C)\\
            &=a\cos A+ k\sin A\cos(B-C)\\
            &=a\cos A+ a\cos(B-C)\\
            &=a(\cos(B-C)-\cos (B+C))\\
            &=2a\sin B\sin C \text{(得证)}
        \end{align*}
    \end{solution}

    \question 已知一锐角三角形 $\triangle ABC$ 的边长分别为 $a,b,c$。设 $r$ 为 $\triangle ABC$ 内切圆的半径,$R$ 为 $\triangle ABC$ 外接圆的半径,试证
    \begin{parts}
    \part $$r = 4R \sin \frac{A}{2} \sin \frac{B}{2} \sin \frac{C}{2}$$
    \begin{solution}
        设 $S$ 为 $\triangle ABC$ 面积,则  
        \[
        S = \dfrac{1}{2}(a + b + c)r = \dfrac{abc}{4R}\Rightarrow r = \dfrac{abc}{2R(a + b + c)}
        \]
        又由于 $a = 2R\sin A,b = 2R\sin B,c = 2R\sin C$,代入得
        \[
        \begin{aligned}
        r &= \dfrac{2R\sin A \cdot 2R\sin B \cdot 2R\sin C}{2R(2R\sin A + 2R\sin B + 2R\sin C)} \\
        &= 2R \cdot \dfrac{\sin A \sin B \sin C}{\sin A + \sin B + \sin C} \\[1mm]
        &= 2R \cdot \dfrac{2\sin\frac{A}{2}\cos\frac{A}{2} \cdot 2\sin\frac{B}{2}\cos\frac{B}{2} \cdot 2\sin\frac{C}{2}\cos\frac{C}{2}}{2\sin\frac{A+B}{2}\cos\frac{A-B}{2} + 2\sin\frac{C}{2}\cos\frac{C}{2}} \\[1mm]
        &= 8R \cdot \dfrac{\sin\frac{A}{2}\sin\frac{B}{2}\sin\frac{C}{2}\cos\frac{A}{2}\cos\frac{B}{2}\cos\frac{C}{2}}{\cos\frac{A-B}{2}\cos\frac{C}{2} + \cos\frac{A+B}{2}\cos\frac{C}{2}} \\[1mm]
        &= 8R \cdot \dfrac{\sin\frac{A}{2}\sin\frac{B}{2}\sin\frac{C}{2}\cos\frac{A}{2}\cos\frac{B}{2}}{\cos\frac{A-B}{2} + \cos\frac{A+B}{2}} \\[1mm]
        &= 8R \cdot \sin\frac{A}{2}\sin\frac{B}{2}\sin\frac{C}{2} \cdot \dfrac{\frac{1}{2}(\cos\frac{A-B}{2} + \cos\frac{A+B}{2})}{\cos\frac{A-B}{2} + \cos\frac{A+B}{2}} \\[1mm]
        &= 4R \sin\frac{A}{2}\sin\frac{B}{2}\sin\frac{C}{2}
        \end{aligned}
        \]
    \end{solution}
    \part $$\frac{abc}{\sqrt{2(a^2 + b^2)(b^2 + c^2)(c^2 + a^2)}} \geq \frac{r}{2R}$$
    \begin{solution}
        由余弦定理,  
        \[
        \cos A = \dfrac{b^2 + c^2 - a^2}{2bc} \ge \dfrac{b^2 + c^2 - a^2}{b^2 + c^2} = 1 - \dfrac{a^2}{b^2 + c^2}
        \]
        而 $\cos A = 1 - 2\sin^2\frac{A}{2}$,所以
        \[
        1 - 2\sin^2\frac{A}{2} \ge 1 - \dfrac{a^2}{b^2 + c^2} \Rightarrow \dfrac{a^2}{b^2 + c^2} \ge 2\sin^2\frac{A}{2}
        \]
        同理可得
        \[
        \dfrac{b^2}{a^2 + c^2} \ge 2\sin^2\frac{B}{2},\quad
        \dfrac{c^2}{a^2 + b^2} \ge 2\sin^2\frac{C}{2}
        \]
        于是
        \[
        \dfrac{abc}{\sqrt{(a^2 + b^2)(b^2 + c^2)(c^2 + a^2)}} \ge \sqrt{2\sin^2\frac{A}{2} \cdot 2\sin^2\frac{B}{2}\cdot 2\sin^2\frac{C}{2}}= 2\sqrt{2} \sin\frac{A}{2} \sin\frac{B}{2} \sin\frac{C}{2}
        \]
        由 $(a)$ 得$$\sin\frac{A}{2} \sin\frac{B}{2} \sin\frac{C}{2} = \dfrac{r}{4R},$$
        故
        \[
        \dfrac{abc}{\sqrt{2(a^2 + b^2)(b^2 + c^2)(c^2 + a^2)}} \ge \frac{1}{\sqrt2}\cdot 2\sqrt{2}\cdot \dfrac{r}{4R}=\dfrac{r}{2R} 
        \]
    \end{solution}
    \end{parts}
    
    \question 已知 $A, B, C$ 是一锐角三角形的内角,证明
    \[
    (\tan A + \tan B + \tan C)^2 \ge (\sec A + 1)^2 + (\sec B + 1)^2 + (\sec C + 1)^2
    \]
    \begin{solution}
        由于 $A, B, C$ 都是锐角,故这些角的三角函数均为正。注意到
        \begin{align*}
        &(\cos A - \cos B)^2 \\
        &= \cos^2 A + \cos^2 B - 2 \cos A \cos B \\
        &= \cos(B+C)\cos A - \cos(A+C)\cos B - 2 \cos A \cos B \\
        &= \cos B \cos C \cos A + \sin B \sin C \cos A - \cos A \cos C \cos B + \sin A \sin C \cos B - 2 \cos A \cos B \\
        &= \cos A \cos B \cos C (\tan B \tan C + \tan A \tan C - 2 \sec C - 2)
        \end{align*}
        因为左边非负,有
        \[
        \tan B \tan C + \tan A \tan C \ge 2 \sec C + 2
        \]
        同理,
        \[
        \tan A \tan B + \tan A \tan C \ge 2 \sec A + 2, \quad
        \tan A \tan B + \tan B \tan C \ge 2 \sec B + 2
        \]
        将三式相加得
        \[
        \tan A \tan B + \tan A \tan C + \tan B \tan C \ge \sec A + \sec B + \sec C + 3
        \]
        因此,
        \begin{align*}
        &(\tan A + \tan B + \tan C)^2 \\
        &= \tan^2 A + \tan^2 B + \tan^2 C + 2(\tan A \tan B + \tan A \tan C + \tan B \tan C) \\
        &\ge \tan^2 A + \tan^2 B + \tan^2 C + 2(\sec A + \sec B + \sec C) + 6 \\
        &= \sec^2 A + \sec^2 B + \sec^2 C + 2(\sec A + \sec B + \sec C) + 3 \\
        &= (\sec A + 1)^2 + (\sec B + 1)^2 + (\sec C + 1)^2
        \end{align*}
    \end{solution}

    \question 已知 $x$ 为一实数, $0 < x < \pi$, 证明: 对于所有的自然数 $n$,
    \[
    \sin x + \frac{\sin 3x}{3} + \frac{\sin 5x}{5} + \dots + \frac{\sin(2n-1)x}{2n-1}
    \]
    的值为正数。
    \begin{solution}
        令 $$f(x) = \sin x + \frac{\sin 3x}{3} + \frac{\sin 5x}{5} + \dots + \frac{\sin(2n-1)x}{2n-1}$$
        由
        $$2\sin x \sin(2k-1)x = \cos(2k-2)x - \cos 2kx$$
        可得
        \begin{align*}
            &2f(x)\sin x\\
            &=1-\cos2x+\frac{\cos2x-\cos4x}{3}+\frac{\cos4x-\cos6x}{5}+\cdots+\frac{\cos(2n-2)x-\cos2nx}{2n-1} \\ 
            &=1-\left(1-\frac{1}{3}\right)\cos2x-\left(\frac{1}{3}-\frac{1}{5}\right)\cos4x-\cdots-\left(\frac{1}{2n-3}-\frac{1}{2n-1}\right)\cos(2n-2)x-\frac{\cos2nx}{2n-1} \\ 
            &\ge 1-\left[\left(1-\frac{1}{3}\right)+\left(\frac{1}{3}-\frac{1}{5}\right)+\cdots+\left(\frac{1}{2n-3}-\frac{1}{2n-1}\right)+\frac{1}{2n-1}\right]=0
        \end{align*}
        若等号成立,则有$\cos2kx=1(k=1,2,\dots,n)$,但因$0<x<\pi$,故$\cos2x\ne1$.于是得 $f(x)\sin x>0$.又因$\sin x>0$,所以$f(x)>0.$
    \end{solution}

    \question 证明  
    \[
    \cot\theta-\cot2\theta=\csc2\theta
    \]
    据此,若已知  
    \[
    \frac1{\sin8^\circ}+ \frac1{\sin16^\circ}+\cdots+\frac1{\sin4096^\circ}+ \frac1{\sin8192^\circ}
    =\frac1{\sin\alpha},
    \]
    其中$\alpha\in (0,90^\circ)$, 求$\alpha$。
    \begin{solution}
        有
        \[
        \cot \theta - \cot 2\theta 
        = \frac{\sin 2\theta \cos \theta - \sin \theta \cos 2\theta}{\sin \theta \sin 2\theta}
        = \frac{\sin(2\theta - \theta)}{\sin \theta \sin 2\theta}
        = \csc 2\theta \quad(\text{得证})
        \]
        故原式是一裂项和
        \[
        \sum_{k=0}^{10}\frac1{\sin\left(2^k8^\circ\right)}=\sum_{k=0}^{10}\left(\cot (2^{k-1}8^\circ) - \cot (2^k8^\circ)\right)=\cot 4^\circ - \cot 8192^\circ
        \]
        而 \(8192 \bmod 90 = 2\),故
        \begin{align*}
        \sum_{k=0}^{10}\frac1{\sin\left(2^k8^\circ\right)}
        &=\frac1{\tan4^\circ}-\frac1{\tan8192^\circ}\\
        &=\frac1{\tan4^\circ}+\tan2^\circ\\
        &=\frac{\cos4^\circ}{\sin4^\circ}+\frac{1-\cos4^\circ}{\sin4^\circ}\\
        &=\frac1{\sin4^\circ} \Rightarrow \alpha=4^\circ
        \end{align*}
        其中利用了恒等式$\tan \theta=\dfrac{1-\cos 2\theta}{\sin 2\theta}$。
    \end{solution}  

    \question 证明 
    \[
    \sum_{k=1}^{90} 2k\sin (2k)^\circ=90\cot 1^\circ
    \]
    \begin{solution}
        由 $k\sin (2k)^\circ+(90-k)\sin (180-2k)^\circ=90\sin (2k)^\circ,k=1,2,\cdots,90$,
        \begin{align*}
        \sum_{k=1}^{90} 2k\sin 2k^\circ
        &= 2(90\sin 2^\circ +90\sin 4^\circ +\cdots +90\sin 88^\circ)+ 2\cdot 45\sin 90^\circ \\
        &= \frac{90}{\sin 1^\circ}(2\sin 2^\circ \sin 1^\circ +2\sin 4^\circ \sin 1^\circ +\cdots +2\sin 88^\circ \sin 1^\circ)+90 \\
        &= \frac{90}{\sin 1^\circ}( \cos 1^\circ-\cos 3^\circ +\cos 3^\circ-\cos 5^\circ +\cdots +\cos 87^\circ-\cos 89^\circ)+90 \\
        &= \frac{90}{\sin 1^\circ} (\cos 1^\circ-\cos 89^\circ)+ 90 \\
        &= \frac{90}{\sin 1^\circ} (\cos 1^\circ-\sin 1^\circ)+ 90 \\
        &= 90\cot 1^\circ
        \end{align*}
    \end{solution}

    \question 计算  
    \[
    \sin^{6}1^\circ+\sin^{6}2^\circ+\cdots+\sin^{6}89^\circ
    \]
    \begin{solution}
        由$\sin x=\cos \left(\dfrac{\pi}{2}-x\right)$,设 
        \[
        S=\sin^{6}1^\circ+\sin^{6}2^\circ+\cdots+\sin^{6}89^\circ=\cos^{6}1^\circ+\cos^{6}2^\circ+\cdots+\cos^{6}89^\circ,
        \]
        则 
        \[
        2S= \sum_{k=1}^{44} (\sin^6 k^\circ+\cos^6 k^\circ) + 2\cdot\left(\frac{\sqrt2}{2}\right)^6
        \]
        其中
        \begin{align*}
        \sin^6 x + \cos^6 x &= (\sin^2 x + \cos^2 x)(\sin^4 x + \cos^4 x-\sin x \cos x) \\
        &=1 \cdot ((\sin^2 x + \cos^2 x)^2-3\sin^2 x \cos^2 x) \\
        &=1-\frac{3}{4}\sin^2 2x 
        \end{align*}且
        \[
        \sum_{k=1}^{44} \sin^2 (2k)^\circ=\sum_{k=1}^{22} (\sin^2 (2k)^\circ+\cos^2 (2k)^\circ)=22
        \]
        故
        \[
        2S=44-\frac{3}{4}\cdot 22+\frac{1}{4} \Rightarrow S=\frac{111}{8}
        \]
    \end{solution}
        
    \question 求 
    \[
    \tan^2 \frac {\pi}{16} \cdot \tan^2 \frac {3\pi}{16} 
    +\tan^2 \frac {\pi}{16} \cdot \tan^2 \frac {5\pi}{16}
    +\tan^2 \frac {3\pi}{16} \cdot \tan^2 \frac {7\pi}{16}
    +\tan^2 \frac {5\pi}{16} \cdot \tan^2 \frac {7\pi}{16}\] 
    的值。
    \begin{solution}
        将原式写成
        \[
        \left(\tan^2 \frac{\pi}{16}+\tan^2\frac{7\pi}{16}\right)\left(\tan^2 \frac{3\pi}{16}+\tan^2 \frac{5\pi}{16}\right) = \left(\tan^2 \frac{\pi}{16}+\cot^2\frac{\pi}{16}\right)\left(\tan^2 \frac{3\pi}{16}+\cot^2 \frac{3\pi}{16}\right).
        \]
        注意到        
        \[
        \tan^2\theta+\cot^2\theta=\frac{\sin^4\theta+\cos^4\theta}{\sin^2\theta \cos^2\theta}
        =\frac{1}{\sin^2\theta \cos^2 \theta}-2
        = \frac{4}{\sin^2 2\theta} -2
        = \frac{8}{1-\cos 4\theta}-2
        \]
        故 
        \[
        \tan^2\dfrac{\pi}{16}+\cot^2\dfrac{\pi}{16} = -2+\dfrac{8}{1-\frac{\sqrt 2}{2}} = 14+8\sqrt 2
        \]
        同理得 
        \[
        \tan^2\dfrac{3\pi}{16}+\cot^2\dfrac{3\pi}{16}=14-8\sqrt 2
        \]
        故原式 $=196-2\cdot 64=68$。
    \end{solution}    

    \question 求表达式 
    \[
    \log_{8}[(\tan 1^\circ+\sqrt{3})(\tan 2^\circ+\sqrt{3})(\tan 3^\circ+\sqrt{3})\cdots(\tan 29^\circ+\sqrt{3})]
    \]
    的值。
    \begin{solution}
        若 $A + B = 30^\circ$,则
        \[
        \tan 30^\circ = \frac{1}{\sqrt{3}} = \frac{\tan A + \tan B}{1 - \tan A \tan B} \Rightarrow \tan A + \tan B + \sqrt{3}(\tan A + \tan B) = 1
        \]
        于是
        \[
        (\tan A + \sqrt{3})(\tan B + \sqrt{3}) = \tan A \tan B + \sqrt{3}(\tan A + \tan B) + 3 = 4
        \]
        因此
        \[
        P=(\tan 1^\circ+\sqrt{3})(\tan 2^\circ+\sqrt{3})(\tan 3^\circ+\sqrt{3})\cdots(\tan 29^\circ+\sqrt{3})=4^{14} \cdot 2 =2^{29}
        \]
        故
        \[
        \log_{8}P= \frac{29}{3}
        \]
    \end{solution}

    \question 已知 $\sin\theta=\dfrac{3}{5}$,求
    \[
    \left( \frac{\sqrt{2}\cos\theta}{\sqrt{3}}+ \frac{\cos 2\theta}{\sqrt{6}}+ \frac{\cos 3\theta}{\sqrt{24}}+\cdots \right)
    \left( \frac{\sqrt{2}\sin\theta}{\sqrt{3}}+ \frac{\sin 2\theta}{\sqrt{6}}+ \frac{\sin 3\theta}{\sqrt{24}}+\cdots \right)
    \]
    \begin{solution}
        发现
        \[
        \frac{\sqrt{2}}{\sqrt{3}} \left( \cos\theta + \frac{1}{2}\cos 2\theta + \frac{1}{4}\cos 3\theta + \cdots \right)
        = \frac{1}{3} \sqrt{6} \cdot \Re \left( \frac{e^{i\theta}}{1 - \frac{1}{2} e^{i\theta}} \right)
        = \frac{2}{3} \sqrt{6} \cdot \Re \left( \frac{e^{i\theta}}{2 - e^{i\theta}} \right)        
        \]
        已知 \(\sin\theta = \dfrac{3}{5}\),则 \(\cos\theta = \dfrac{4}{5},\)代入 \(e^{i\theta} = \cos\theta + i\sin\theta = \dfrac{4}{5} + \dfrac{3}{5}i\) 得
        \[
        \frac{2\sqrt{6}}{3} \cdot \Re \left( \frac{1}{3} + \frac{2}{3}i \right)
        = \frac{2\sqrt{6}}{3} \cdot \frac{1}{3}
        = \frac{2\sqrt{6}}{9} 
        \]
        同理
        \[
        \frac{\sqrt{2}}{\sqrt{3}} \left( \sin\theta + \frac{1}{2}\sin 2\theta + \frac{1}{4}\sin 3\theta + \cdots \right) 
        = \frac{\sqrt{6}}{3}  \cdot \Im \left( \frac{e^{i\theta}}{1 - \frac{1}{2} e^{i\theta}} \right)
        = \frac{2\sqrt{6}}{3}  \cdot \Im \left( \frac{1}{3} + \frac{2}{3}i \right)
        = \frac{4\sqrt{6}}{9} 
        \]
        两式相乘得
        \[
        \left( \frac{\sqrt{2}\cos\theta}{\sqrt{3}}+ \frac{\cos 2\theta}{\sqrt{6}}+ \frac{\cos 3\theta}{\sqrt{24}}+\cdots \right)
        \left( \frac{\sqrt{2}\sin\theta}{\sqrt{3}}+ \frac{\sin 2\theta}{\sqrt{6}}+ \frac{\sin 3\theta}{\sqrt{24}}+\cdots \right)=\frac{2\sqrt{6}}{9}  \cdot \frac{4\sqrt{6}}{9} = \frac{16}{27}
        \]  
    \end{solution}

    \question 证明
    \[
    \cos \frac{\pi}{11} + \cos \frac{3\pi}{11} + \cos \frac{5\pi}{11} + \cos \frac{7\pi}{11} + \cos \frac{9\pi}{11} = \frac{1}{2}.
    \]
    \begin{solution}
        设 $z = \cos \theta + i \sin \theta$,其中 $\theta = \dfrac{\pi}{11}$,则由棣莫弗定理,
        \[
        z^k = \cos k\theta + i \sin k\theta.
        \]
        考虑和
        \[
        S = z + z^3 + z^5 + z^7 + z^9=z \frac{(z^2)^5 - 1}{z^2 - 1}=\frac{z^{11} - z}{z^2 - 1}
        \]
        由于 $z^{11} = \cos 11\theta + i \sin 11\theta = \cos \pi + i \sin \pi = -1$,所以
        \[
        S = \frac{-1 - z}{z^2 - 1} = \frac{1}{1-z}
        \]
        代入 $z = \cos \theta + i \sin \theta$:
        \begin{align*}
        S &= \frac{1}{1-\cos \theta - i \sin \theta} \\
        &= \frac{(1-\cos \theta) + i \sin \theta}{(1-\cos \theta)^2 + \sin^2 \theta} \\
        &= \frac{1}{2} + \frac{i \sin \theta}{2(1-\cos \theta)} \\
        &= \frac{1}{2} + \frac{i \cdot 2 \sin \frac{\theta}{2} \cos \frac{\theta}{2}}{2 \cdot 2 \sin^2 \frac{\theta}{2}} \\
        &= \frac{1}{2} + \frac{i}{2} \cot \frac{\theta}{2}
        \end{align*}
        故所求即
        \[
        \cos \frac{\pi}{11} + \cos \frac{3\pi}{11} + \cos \frac{5\pi}{11} + \cos \frac{7\pi}{11} + \cos \frac{9\pi}{11} = \Re(S)= \frac{1}{2}
        \]
    \end{solution}

    \question 求值 
    \[
    \frac{1}{\tan^2{\dfrac{\pi}{18}}}+\frac{1}{\tan^2{\dfrac{5\pi}{18}}}+\frac{1}{\tan^2{\dfrac{7\pi}{18}}}
    \]
    \begin{solution}
        设 \(\theta = \dfrac{\pi}{18} \Rightarrow 3\theta = \dfrac{\pi}{6}\),两边取$\tan,$
        \[
        \tan(3\theta) = \tan\frac{\pi}{6} \Rightarrow \frac{3\tan\theta - \tan^3\theta}{1 - 3\tan^2\theta} = \frac{1}{\sqrt{3}}
        \]
        整理得
        \[
        \sqrt{3}\tan^3\theta - 3\tan^2\theta - 3\sqrt{3}\tan\theta + 1 = 0
        \]
        发现该三次方程的三个根分别是
        \[
        x_1 = \tan\frac{\pi}{18},\; x_2 = -\tan\frac{5\pi}{18},\; x_3 = \tan\frac{7\pi}{18}
        \]
        由韦达定理
        \[
        x_1 + x_2 + x_3 = \sqrt 3,\;
        x_1x_2 + x_2x_3 + x_3x_1 = -3,\;
        x_1x_2x_3 = -\frac{1}{\sqrt3}
        \]
        则所求为
        \begin{align*}
        \frac{1}{x_1^2} + \frac{1}{x_2^2} + \frac{1}{x_3^2} &= \left( \frac{1}{x_1} + \frac{1}{x_2} + \frac{1}{x_3} \right)^2 - 2\left( \frac{1}{x_1x_2} + \frac{1}{x_2x_3} + \frac{1}{x_3x_1} \right)  \\
        &=\left(\frac{x_1x_2 + x_2x_3 + x_3x_1}{x_1x_2x_3}\right)^2-2\left(\frac{x_1 + x_2 + x_3}{x_1x_2x_3}\right)\\
        &= \left( \frac{-3}{-\frac{1}{\sqrt{3}}} \right)^2 - 2 \cdot \frac{\sqrt{3}}{-\frac{1}{\sqrt{3}}} \\
        &= 33
        \end{align*}
    \end{solution}

    \question 计算
    \[
    \cos^7 \frac{\pi}{9} + \cos^7 \frac{5\pi}{9} + \cos^7 \frac{7\pi}{9}
    \]
    \begin{solution}
        因 \( \theta = \dfrac{\pi}{9},\dfrac{5\pi}{9},\dfrac{7\pi}{9} \)皆满足
        \[
        \cos 3\theta = 4\cos^3 \theta - 3\cos \theta
        \]
        故 \(  \cos \dfrac{\pi}{9},\cos \dfrac{5\pi}{9} , \cos \dfrac{7\pi}{9} \) 是方程 \( 8x^3 - 6x - 1 = 0 \) 的三根。
        
        现设$x=\cos \dfrac{\pi}{9},y=\cos \dfrac{5\pi}{9} , z=\cos \dfrac{7\pi}{9}, S_n = x^n + y^n + z^n$,
        我们将用牛顿恒等式递推计算 \( S_7 \)。根据韦达定理可得
        \[
        \begin{aligned}
        S_1 &= x + y + z = 0 \\
        S_2 &= (x + y + z)^2 - 2(xy + yz + zx) = 0^2 - 2\cdot\left(-\frac{3}{4}\right) = \frac{3}{2} \\
        S_3 &= x^3 + y^3 + z^3 = 3xyz = \frac{3}{8}
        \end{aligned}
        \]
        递推公式为
        \begin{align*}
        S_n &= (x+y+z)S_{n-1} - (xy+yz+zx)S_{n-2} + xyz \cdot S_{n-3} \\
        &=0 \cdot S_{n-1} - (-\dfrac{3}{4}) S_{n-2} + \dfrac{1}{8} \cdot S_{n-3}
        \end{align*}
        于是
        \[
        \begin{aligned}
        S_4 &= 0 \cdot S_3 - (-\dfrac{3}{4}) S_2 + \dfrac{1}{8} \cdot S_1 = \dfrac{3}{4} \cdot \dfrac{3}{2} = \dfrac{9}{8} \\
        S_5 &= 0 \cdot S_4 - (-\dfrac{3}{4}) S_3 + \dfrac{1}{8} \cdot S_2 = \dfrac{3}{4} \cdot \dfrac{3}{8} + \dfrac{1}{8} \cdot \dfrac{3}{2} = \dfrac{9}{32} + \dfrac{3}{16} = \dfrac{15}{32} \\
        S_6 &= 0 \cdot S_5 - (-\dfrac{3}{4}) S_4 + \dfrac{1}{8} \cdot S_3 = \dfrac{3}{4} \cdot \dfrac{9}{8} + \dfrac{1}{8} \cdot \dfrac{3}{8} = \dfrac{27}{32} + \dfrac{3}{64} = \dfrac{57}{64} \\
        S_7 &= 0 \cdot S_6 - (-\dfrac{3}{4}) S_5 + \dfrac{1}{8} \cdot S_4 = \dfrac{3}{4} \cdot \dfrac{15}{32} + \dfrac{1}{8} \cdot \dfrac{9}{8} = \dfrac{45}{128} + \dfrac{9}{64} = \dfrac{63}{128}
        \end{aligned}
        \]
    \end{solution}

    \question 证明
    \[
    \tan \dfrac{2\pi}{15} \tan \dfrac{4\pi}{15} \tan \dfrac{8\pi}{15} \tan \dfrac{16\pi}{15} = -1
    \]
    \begin{solution}
        原式写成
        \[
        V=\frac{\sin 24^\circ \sin 48^\circ \sin 96^\circ \sin 192^\circ}{\cos 24^\circ \cos 48^\circ \cos 96^\circ \cos 192^\circ}
        \]
        记 \(T = \sin 24^\circ \sin 48^\circ \sin 96^\circ \sin 192^\circ,N = \cos 24^\circ \cos 48^\circ \cos 96^\circ \cos 192^\circ\)
        
        利用三角恒等式 \(\sin(60^\circ - \alpha)\sin \alpha \sin(60^\circ + \alpha) = \dfrac{1}{4}\sin 3\alpha\),整理 \(T\) 为
        \begin{align*}
        T &= -\sin 24^\circ \sin 48^\circ \sin 96^\circ \sin 12^\circ\\
        &= -\frac{\sin 12^{\circ}\sin 48^{\circ}\sin 108^{\circ}}{\sin 108^{\circ}} \cdot \sin 24^{\circ}\sin 96^{\circ}\\
        &=-\frac{1}{4}\frac{\sin 144^{\circ}}{\sin 108^{\circ}} \cdot \sin 24^{\circ}\sin 96^{\circ} \\
        &=-\frac{1}{4} \cdot \frac{\sin 24^\circ \sin 36^\circ \sin 96^\circ}{\sin 108^\circ} \\
        &= -\frac{1}{4} \cdot \frac{\frac{1}{4} \sin 108^\circ}{\sin 108^\circ} \\
        &= -\frac{1}{16}
        \end{align*}
        连续使用恒等式 $\sin2\alpha=2\sin \alpha\cos \alpha$有
        \[
        16\sin 24^\circ \cdot N = \sin 384^\circ = \sin 24^\circ \Rightarrow N = \frac{1}{16}
        \]
        故
        \[
        V = \frac{T}{N} = \frac{-\frac{1}{16}}{\frac{1}{16}} = \boxed{-1}
        \]
    \end{solution}
    \begin{solution}
        设 \(a = \dfrac{\pi}{15}\),有
        \begin{align*}
        \tan 2a \tan 4a \tan 8a \tan 16a &= \frac{2\sin a \cos a}{\cos 2a} \cdot \frac{2\sin 2a \cos 2a}{\cos 4a} \cdot \frac{2\sin 4a \cos 4a}{\cos 8a} \cdot \frac{2\sin 8a \cos 8a}{\cos 16a} \\&=
        16 \cdot \frac{\cos a \sin a \sin 2a \sin 4a \sin 8a}{\cos 16a}
        \end{align*}
        因为$\cos 16a=-\cos a,$于是
        \[
        \tan 2a \tan 4a \tan 8a \tan 16a = -16 \sin a \sin 2a \sin 4a \sin 8a
        \]
        我们来计算这个乘积。设 \(x = \cos 3a\),因为
        \[
        4\sin a \sin 4a = 2\cos3a - 2\cos5a=2\cos 3a-1
        \]
        \[
        4\sin 2a \sin 8a = 2\cos6a - 2\cos10a=2\cos 6a+1
        \]
        它们的乘积
        \[
        (2\cos 3a-1)(2\cos 6a+1)=(2x-1)(2(2x^2-1)+1)=2x(4x^2-2x-1)+1
        \]
        其中$x = \cos 3a$,又因为 \(\cos 9a = -\cos 6a\),可以推出
        \[
        4x^3-3x=1-2x^2 \Rightarrow (x + 1)(4x^2 - 2x - 1) = 0 \Rightarrow 2x\cdot 0+1=1
        \]
        故
        \[
        16 \sin a \sin 2a \sin 4a \sin 8a = -1
        \Rightarrow \tan \dfrac{2\pi}{15} \tan \dfrac{4\pi}{15} \tan \dfrac{8\pi}{15} \tan \dfrac{16\pi}{15} = -1
        \]
    \end{solution}

    \question 求  
    \[
    S = \sum_{n=1}^7 \tan^2\left(\frac{n\pi}{16}\right)
    \]
    \begin{solution}
        注意到 $$\tan\left(-\frac{7\pi}{16}\right), \tan\left(-\frac{6\pi}{16}\right), \dots, \tan\left(\frac{7\pi}{16}\right)$$ 是方程 $\tan16x = 0$ 的解,由恒等式
        \[
        \tan(n\theta) = \frac{\comb{n}{1} \tan\theta - \comb{n}{3} \tan^3\theta + \cdots}{1 - \comb{n}{2} \tan^2\theta + \cdots}
        \]也即是
        \[
        \comb{16}{1}x - \comb{16}{3}x^3 + \cdots + \comb{16}{13}x^{13} - \comb{16}{15}x^{15} = 0
        \]
        的根,去掉零根 $\tan0=0$ 后,
        $$\tan\left(\pm \frac{7\pi}{16}\right), \tan\left(\pm \frac{6\pi}{16}\right), \dots, \tan\left(\pm \frac{\pi}{16}\right)$$ 是
        \[
        \comb{16}{1} - \comb{16}{3}x^2 + \comb{16}{5}x^4 - \cdots + \comb{16}{13}x^{12} - \comb{16}{15}x^{14} = 0
        \]
        的根,因此$$\tan^2\left(\frac{7\pi}{16}\right), \tan^2\left(\frac{6\pi}{16}\right), \dots, \tan^2\left(\frac{\pi}{16}\right)$$ 
        是
        \[
        \comb{16}{1} - \comb{16}{3}y + \comb{16}{5}y^2 - \cdots + \comb{16}{13}y^6 - \comb{16}{15}y^7 = 0
        \]
        的根;由韦达定理,根之和为
        \[
        \frac{\comb{16}{13}}{\comb{16}{15}} = 35
        \]
    \end{solution}

    \question 若 $\alpha = \dfrac{2\pi}{1999}$, 试求 
    \[
    \cos\alpha\cos 2\alpha\cos 3\alpha\ldots\cos 998\alpha\cos 999\alpha
    \]
    的值。
    \begin{solution}
        记$S = \cos \alpha \cos 2\alpha \cdots \cos 999\alpha, T = \sin \alpha \sin 2\alpha \cdots \sin 999\alpha$,则
        \[
        ST = \sin \alpha \cos \alpha \sin 2\alpha \cos 2\alpha \cdots \sin 999\alpha \cos 999\alpha 
        \]
        即
        \[
        2^{999} ST = \sin 2\alpha \sin 4\alpha \cdots \sin 1998\alpha
        \]
        且注意到
        \[
        \sin 1998\alpha = - \sin(2\pi - 1998\alpha) = - \sin\frac{2\pi}{1999} = - \sin \alpha
        \]
        同理,
        \[
        \sin 1996\alpha = - \sin 3\alpha, \quad \sin 1994\alpha = - \sin 5\alpha, \cdots
        \]
        因此
        \[
        2^{999} ST = \sin \alpha \sin 4\alpha \cdots (-\sin 3\alpha)(-\sin \alpha) = \sin \alpha \sin 4\alpha \cdots \sin 3\alpha \sin \alpha = T 
        \]
        由于 $T \neq 0$,故
        \[
        S = \cos \alpha \cos 2\alpha \cos 3\alpha \cdots \cos 998\alpha \cos 999\alpha = \frac{1}{2^{999}}
        \]
    \end{solution}

    \question 证明:对于所有大于1的自然数 $n$,有
    \[
    \sin \frac{\pi}{n} \cdot \sin \frac{2\pi}{n} \cdots \sin \frac{(n-1)\pi}{n} = \frac{n}{2^{n-1}}.
    \]
    \begin{solution}
        设
        \[
        \omega = \cos \frac{2\pi}{n} + i \sin \frac{2\pi}{n},
        \]
        则多项式
        \[
        x^{n-1} + x^{n-2} + \cdots + 1 = 0
        \]的 $n-1$ 个根为$\omega^k, k=1,2,\dots,n-1,$且$\omega^n = 1,$令
        \[
        f(x) = x^{n-1} + x^{n-2} + \cdots + 1 = (x-\omega^1)(x-\omega^2)\cdots (x-\omega^{n-1}),
        \]
        则
        \[
        f(1) = n = (1-\omega^1)(1-\omega^2)\cdots (1-\omega^{n-1}).
        \]
        且发现
        \[
        1 - \omega^k = 2 \sin^2 \frac{k\pi}{n} - 2 i \sin \frac{k\pi}{n} \cos \frac{k\pi}{n} = 2 \sin \frac{k\pi}{n} \left(\sin \frac{k\pi}{n} - i \cos \frac{k\pi}{n}\right).
        \]
        故
        \[
        |1 - \omega^k| = 2 \sin \frac{k\pi}{n}.
        \]
        取模可得
        \[
        n = |1 - \omega^1||1 - \omega^2| \cdots |1 - \omega^{n-1}| = 2^{n-1} \sin \frac{\pi}{n} \cdot \sin \frac{2\pi}{n} \cdots \sin \frac{(n-1)\pi}{n}.
        \]
        即得证
        \[
        \sin \frac{\pi}{n} \cdot \sin \frac{2\pi}{n} \cdots \sin \frac{(n-1)\pi}{n} = \frac{n}{2^{n-1}}.
        \]
    \end{solution}
\question
计算下列乘积:

(a) $\sin \frac{\pi}{2n + 1} \sin \frac{2\pi}{2n + 1} \sin \frac{3\pi}{2n + 1} \dots \sin \frac{n\pi}{2n + 1}$
以及
$\sin \frac{\pi}{n} \sin \frac{2\pi}{n} \sin \frac{3\pi}{n} \dots \sin \frac{(n - 1)\pi}{n}$

(b) $\cos \frac{\pi}{2n + 1} \cos \frac{2\pi}{2n + 1} \cos \frac{3\pi}{2n + 1} \dots \cos \frac{n\pi}{2n + 1}$
以及
$\cos \frac{\pi}{2n} \cos \frac{2\pi}{2n} \cos \frac{3\pi}{2n} \dots \cos \frac{(n - 1)\pi}{2n}$

\begin{solution}
    (a) 设 $\omega = \cos \frac{2\pi}{n} + i \sin \frac{2\pi}{n}$ 为单位的 $n$ 次复根。
    考虑方程 $x^{n-1} + x^{n-2} + \cdots + 1 = 0$,其根为 $\omega^k$ ($k=1,2,\dots,n-1$)。
    因此有恒等式:
    \[ f(x) = x^{n-1} + x^{n-2} + \cdots + 1 = (x-\omega^1)(x-\omega^2)\cdots (x-\omega^{n-1}) \]
    令 $x=1$,得:
    \[ f(1) = n = (1-\omega^1)(1-\omega^2)\cdots (1-\omega^{n-1}) \]
    由于 $|1 - \omega^k| = |1 - (\cos \frac{2k\pi}{n} + i \sin \frac{2k\pi}{n})| = |2 \sin^2 \frac{k\pi}{n} - 2i \sin \frac{k\pi}{n} \cos \frac{k\pi}{n}| = 2 \sin \frac{k\pi}{n}$。
    取模可得:
    \[ n = 2^{n-1} \sin \frac{\pi}{n} \sin \frac{2\pi}{n} \dots \sin \frac{(n-1)\pi}{n} \]
    由此得出第一个重要结果:
    \[ \sin \frac{\pi}{n} \sin \frac{2\pi}{n} \dots \sin \frac{(n-1)\pi}{n} = \frac{n}{2^{n-1}} \]
    同理可证
    (通过替换 $n$ 为 $2n+1$ 并利用对称性):
    \[ \sin \frac{\pi}{2n + 1} \sin \frac{2\pi}{2n + 1} \dots \sin \frac{n\pi}{2n + 1} = \frac{\sqrt{2n + 1}}{2^n} \]

    (b) 对于余弦乘积,我们利用倍角公式 $\sin 2\theta = 2 \sin \theta \cos \theta$。
    对于 $2n+1$ 的情形:
    \[ \sin \frac{2k\pi}{2n + 1} = 2 \sin \frac{k\pi}{2n + 1} \cos \frac{k\pi}{2n + 1} \]
    将 $k=1, 2, \dots, n$ 的等式全部相乘:
    \[ \prod_{k=1}^{n} \sin \frac{2k\pi}{2n+1} = 2^n \left( \prod_{k=1}^{n} \sin \frac{k\pi}{2n+1} \right) \left( \prod_{k=1}^{n} \cos \frac{k\pi}{2n+1} \right) \]
    由于 $\sin \frac{2k\pi}{2n+1}$ 的集合(当 $k$ 超过 $n/2$ 时利用 $\sin(\pi-\theta)=\sin\theta$)与 $\sin \frac{k\pi}{2n+1}$ 的集合完全相同,两边的正弦乘积项约去,得:
    \[ \cos \frac{\pi}{2n + 1} \cos \frac{2\pi}{2n + 1} \dots \cos \frac{n\pi}{2n + 1} = \frac{1}{2^n} \]

    对于 $\cos \frac{k\pi}{2n}$ 的乘积:
    利用 $\sin \theta \cos \theta = \frac{1}{2} \sin 2\theta$:
    \[ \left[ \prod_{k=1}^{n-1} \cos \frac{k\pi}{2n} \right] \left[ \prod_{k=1}^{n-1} \sin \frac{k\pi}{2n} \right] = \frac{1}{2^{n-1}} \prod_{k=1}^{n-1} \sin \frac{k\pi}{n} \]
    已知右侧乘积为 $\frac{1}{2^{n-1}} \cdot \frac{n}{2^{n-1}} = \frac{n}{2^{2n-2}}$。
    又因为 $\sin \frac{k\pi}{2n} = \cos \frac{(n-k)\pi}{2n}$,所以左侧两个括号内的数值相等。
    取平方根(项均为正)得:
    \[ \cos \frac{\pi}{2n} \cos \frac{2\pi}{2n} \dots \cos \frac{(n - 1)\pi}{2n} = \frac{\sqrt{n}}{2^{n-1}} \]

    注:若将 (a) 的结果除以 (b) 的结果,可得正切乘积:
    \[ \tan \frac{\pi}{2n + 1} \tan \frac{2\pi}{2n + 1} \dots \tan \frac{n\pi}{2n + 1} = \sqrt{2n + 1} \]
    \[ \tan \frac{\pi}{2n} \tan \frac{2\pi}{2n} \dots \tan \frac{(n - 1)\pi}{2n} = 1 \]
\end{solution}
    \question 证明:对于任意正整数 $n$,有
        \[
        \tan \frac{\pi}{2n+1} \tan \frac{2\pi}{2n+1} \cdots \tan \frac{n\pi}{2n+1} = \sqrt{2n+1}.
        \]
    \begin{solution}
        注意到
        \[
        \left( \cos \frac{k\pi}{2n+1} + i \sin \frac{k\pi}{2n+1} \right)^{2n+1} = \left( e^{i \frac{k\pi}{2n+1}} \right)^{2n+1} = e^{ik\pi} = (-1)^k.
        \]
        由二项式定理,比较虚部可得
        \[
        \sum_{j=0}^n (-1)^j \binom{2n+1}{2j+1} \left( \cos \frac{k\pi}{2n+1} \right)^{2n-2j} \left( \sin \frac{k\pi}{2n+1} \right)^{2j+1} = 0.
        \]
        将两边除以$\left( \cos \dfrac{k\pi}{2n+1} \right)^{2n+1}$得到
        \[
        \sum_{j=0}^n (-1)^j \binom{2n+1}{2j+1} \left( \tan \frac{k\pi}{2n+1} \right)^{2j+1} = 0.
        \]
        即\(\tan \dfrac{k\pi}{2n+1},1 \leq k \leq 2n+1\) 是次数为 \(2n+1\)的多项式
        \[
        \sum_{j=0}^n (-1)^j \binom{2n+1}{2j+1} z^{2j+1} = 0,
        \]
        的根;不考虑根 \(z = 0\) 可知, $\tan\left(\dfrac{k\pi}{2n+1}\right),\; 1 \leq k \leq 2n$是多项式
        \[
        \sum_{j=0}^n (-1)^j \binom{2n+1}{2j+1} z^{2j} = (-1)^n z^{2n} + \cdots + (2n+1) = 0.
        \]
        的根,因此由韦达定理,
        \[
        \prod_{k=1}^{2n} \tan\left(\frac{k\pi}{2n+1}\right) = (-1)^n \cdot (2n+1).
        \]
        且观察到
        \[
        \tan\left(\frac{(2n+1-k)\pi}{2n+1}\right) = -\tan\left(\frac{k\pi}{2n+1}\right),\;1 \leq k \leq n
        \]
        所以
        \[
        \left( \prod_{k=1}^n \tan\left(\frac{k\pi}{2n+1}\right) \right)^2 = 2n+1.
        \]
        又因\(\tan\left(\dfrac{k\pi}{2n+1}\right) > 0,\;1 \leq k \leq n\),故
        \[
        \tan\left(\frac{\pi}{2n+1}\right) \tan\left(\frac{2\pi}{2n+1}\right) \cdots \tan\left(\frac{n\pi}{2n+1}\right) = \sqrt{2n+1}.
        \]
    \end{solution}

    \question 设
    \[
    f_n(\theta)=\sin\theta\sin2\theta\sin4\theta\cdots\sin(2^{\,n-1}\theta), \quad n \in \mathbb{N}
    \]
    证明对任意实数 $\theta$ 及正整数 $n$皆有
    \[
    |f_n(\theta)|\le \frac{2}{\sqrt{3}}\,\left|f_n\left(\frac{\pi}{3}\right)\right|
    \]
    \begin{solution}
        由AM-GM不等式,
        \begin{align*}
        |g(\theta)| &= |\sin \theta| |\sin(2\theta)|^{1/2} = \frac{\sqrt{2}}{\sqrt[4]{3}} \left( \sqrt[4]{|\sin \theta| \cdot |\sin \theta| \cdot |\sin \theta| \cdot |\sqrt{3}\cos \theta|} \right)^2 \\
        &\le \frac{\sqrt{2}}{\sqrt[4]{3}} \cdot \frac{3 \sin^2 \theta + 3 \cos^2 \theta}{4} = \left(\frac{\sqrt{3}}{2}\right)^{3/2}
        \end{align*}
        故函数$g(\theta)$的最大值为 $\left(\dfrac{\sqrt{3}}{2}\right)^{3/2}$,并且在 $\theta=\dfrac{2^k\pi}{3},k \in \mathbb{N}$处取得。注意到
        \[
        f_n(\theta)
        = g(\theta)\cdot g(2\theta)^{\frac{1}{2}}\cdot g(4\theta)^{\frac{3}{4}}\cdots
        g(2^{n-1}\theta)^E,
        \]
        其中
        \[
        E=\frac{2}{3}\left(1-\left(-\frac{1}{2}\right)^n\right)
        \]
        于是
        \begin{align*}
        \left|\frac{f_n(\theta)}{f_n(\frac{\pi}{3})}\right|
        &=\left|\frac{g(\theta)\cdot g(2\theta)^{\frac{1}{2}}\cdots g(2^{n-1}\theta)^E}{g(\frac{\pi}{3})\cdot g(\frac{2\pi}{3})^{\frac{1}{2}}\cdots g(2^{n-1}\frac{\pi}{3})^E}\right|
        \cdot \left|\frac{\sin(2^n\theta)}{\sin(2^n\frac{\pi}{3})}\right|^{1-\frac{E}{2}}\\
        &\le \left|\frac{\sin(2^n\theta)}{\sin(2^n\frac{\pi}{3})}\right|^{1-\frac{E}{2}}
        \le \left(\frac{1}{\sqrt{3}/2}\right)^{1-\frac{E}{2}}
        \le \frac{2}{\sqrt{3}}
        \end{align*}
        因此得证对所有实数 $\theta$ 及任意正整数$n$ 皆有
        \[
        |f_n(\theta)|\le \frac{2}{\sqrt{3}}\,\left|f_n\left(\frac{\pi}{3}\right)\right|
        \]
    \end{solution}
\end{questions}
\pagebreak

\begin{center}
  {\fontsize{30pt}{26pt}\selectfont
    \hypertarget{反三角函数}{反三角函数} \label{反三角函数}
  }
\end{center}
\separator
\vspace{1pt}

\begin{questions}
    \question 若 $|x|\le 1$,证明
    \[
    \tan^{-1}\sqrt{\frac{1-x}{1+x}}=\frac{1}{2}\cos^{-1}x
    \]
    \begin{solution}
        设$\theta=\tan^{-1}\sqrt{\dfrac{1-x}{1+x}}$,于是
        \[
        1+\tan^2\theta=1+\frac{1-x}{1+x}=\frac{2}{1+x}
        \]
        从而
        \[
        \cos^2\theta=\frac{1}{\sec^2\theta}=\frac{1+x}{2} \Rightarrow \cos2\theta=2\cdot\frac{1+x}{2}-1=x.
        \]
        由于 $|x|\le 1$,且 $\theta \in \left[0,\dfrac{\pi}{2}\right]$,可知 $2\theta\in[0,\pi]$,因此
        \[
        2\theta=\cos^{-1}x \Rightarrow \theta=\frac{1}{2}\cos^{-1}x
        \]
        于是得证
        \[
        \tan^{-1}\sqrt{\frac{1-x}{1+x}}=\frac{1}{2}\cos^{-1}x
        \]
    \end{solution}

    \question 已知锐角 $\theta,\phi$ 满足
    \[
    2\cos\theta = \cos\phi, \quad 2\sin\theta = 3\sin\phi,
    \]
    证明
    \[
    \theta + \phi = \pi - \arctan \sqrt{15}.
    \]
    \begin{solution}
        将已知两式平方并相加,可得
        \[
        4(\cos^2\theta + \sin^2\theta) = \cos^2\phi + 9\sin^2\phi \Rightarrow \sin^2\phi = \frac{3}{8} \Rightarrow \sin\phi = \sqrt{\frac{3}{8}}>0
        \]
        同时可得
        \[
        4\sin^2\theta = 9\sin^2\phi = \frac{27}{8} \Rightarrow \sin\theta = \sqrt{\frac{27}{32}}
        \]
        于是$\tan\theta =\sqrt{\dfrac{27}{5}},\tan\phi=\sqrt{\dfrac{3}{5}}$,且
        \[
        \tan(\theta+\phi) = \frac{\tan\theta + \tan\phi}{1 - \tan\theta\tan\phi} = -\sqrt{15}
        \]
        由于 $0<\theta+\phi<\pi$,取
        \[
        \theta+\phi = \pi - \arctan\sqrt{15}
        \]
    \end{solution}

    \question 已知
    \[
    \sin^{-1}(\sin\alpha+\sin\beta)+\sin^{-1}(\sin\alpha-\sin\beta)=\frac{\pi}{2},
    \]
    求 $\sin^2\alpha+\sin^2\beta$ 的值。
    \begin{solution}
        由
        \[
        \sin^{-1}(\sin\alpha+\sin\beta)=\frac{\pi}{2}-\sin^{-1}(\sin\alpha-\sin\beta)=\cos^{-1}(\sin\alpha-\sin\beta)
        \]
        得到
        \[
        \sin\alpha+\sin\beta=\sin (\cos^{-1}(\sin\alpha-\sin\beta))=\sqrt{1-(\sin\alpha-\sin\beta)^2}
        \]
        因此
        \[
        (\sin\alpha+\sin\beta)^2+(\sin\alpha-\sin\beta)^2=1 \Rightarrow \sin^2\alpha+\sin^2\beta=\frac{1}{2}
        \]
    \end{solution}

    \question 已知
    \[
    \sin^{-1}x+\sin^{-1}y=\frac{2\pi}{3},\quad
    \cos^{-1}x-\cos^{-1}y=\frac{\pi}{3},
    \]
    求 $x,y$ 的值。
    \begin{solution}
        由反三角恒等式 
        \[
        \sin^{-1}x+\cos^{-1}x=\frac{\pi}{2}
        \]
        故
        \[
        \cos^{-1}x-\cos^{-1}y
        =\left(\frac{\pi}{2}-\sin^{-1}x\right)-\left(\frac{\pi}{2}-\sin^{-1}y\right)
        =\sin^{-1}y-\sin^{-1}x=\frac{\pi}{3}
        \]
        于是可解得
        \[
        \sin^{-1}x=\frac{\pi}{6},\;\sin^{-1}y=\frac{\pi}{2}\Rightarrow x=\frac{1}{2},\;y=1
        \]
    \end{solution}

    \question 解方程
    \[
    \tan^{-1}\!\left[x\cos\!\left(2\sin^{-1}\frac{1}{x}\right)\right]=\frac{\pi}{4}.
    \]
    \begin{solution}
        两边取正切,得
        \[
        x\cos\left(2\sin^{-1}\frac{1}{x}\right)=1 \Rightarrow
        \cos\left(2\sin^{-1}\frac{1}{x}\right)=\frac{1}{x}
        \]
        于是有
        \[
        2\sin^{-1}\frac{1}{x}=\pm\cos^{-1}\frac{1}{x}\pm 2n\pi,\quad n=0,1,2,3,\dots
        \]
        利用恒等式$\sin^{-1}t+\cos^{-1}t=\dfrac{\pi}{2}$,可得
        \[
        2\sin^{-1}\frac{1}{x}
        =\pm\left(\frac{\pi}{2}-\sin^{-1}\frac{1}{x}\right)\pm 2n\pi.
        \]
        给出
        \[
        \sin^{-1}\frac{1}{x}=\frac{\pi}{6}\pm\frac{2n\pi}{3}
        \quad \text{或} \quad
        \sin^{-1}\frac{1}{x}=-\frac{\pi}{2}\pm 2n\pi
        \]
        由于$-\dfrac{\pi}{2}\leq \sin^{-1}t\leq \dfrac{\pi}{2}$,故取
        \[
        \sin^{-1}\frac{1}{x}=\frac{\pi}{6}\quad\text{或}\quad
        \sin^{-1}\frac{1}{x}=-\frac{\pi}{2}.
        \]
        即
        \[
        x=2 \quad \text{或} \quad x=-1
        \]
        经检验,可知都是原方程式的解。
    \end{solution}

    \question 解方程
    \[
    \tan^{-1}\left(\sqrt{2}(x+1)\right)-\tan^{-1}\left(\frac{x-1}{\sqrt{2}}\right)=\tan^{-1}\!\frac{\sqrt{2}}{3}.
    \]
    \begin{solution}
        原方程式两边取正切,
        \[
        \frac{\sqrt{2}(x+1)-\dfrac{x-1}{\sqrt{2}}}{1+\sqrt{2}(x+1)\cdot\dfrac{x-1}{\sqrt{2}}}=\frac{\sqrt{2}}{3}
        \]
        解得
        \[
        x=3\quad\text{或}\quad x=-\frac{3}{2}
        \]
        经检验都是方程的实数解。
    \end{solution}

    \question 解反三角方程
    \[
    \sin^{-1}\frac{2x}{1+x^2}=\tan^{-1}\frac{3}{x+7}
    \]
    \begin{solution}
        两边取正弦得
        \[
        \frac{2x}{1+x^2}=\sin\left(\tan^{-1}\frac{3}{x+7}\right)=\pm \frac{3}{\sqrt{3^2+(x+7)^2}}
        \]
        解得
        \[
        x=\frac{1}{5},\quad x=7+2\sqrt{13}
        \]
        经检验皆为方程的实数解。
    \end{solution}

    \question 解方程
    \[
    3\sin^{-1}2x+3\sin^{-1}x = 2\pi
    \]
    \begin{solution}
        原方程式即
        \[
        \sin^{-1}2x+\sin^{-1}x = \frac{2\pi}{3}
        \]
        对两边取正弦得
        \[
        2x\cos(\sin^{-1}x)+x\cos(\sin^{-1}2x) = \frac{\sqrt{3}}{2}
        \]
        即
        \[
        2x\sqrt{1-x^2}+x\sqrt{1-(2x)^2}=\frac{\sqrt{3}}{2}
        \]
        解得$x=\pm \dfrac{1}{2}$,经检验,原方程式的解为
        \[
        x=\frac{1}{2}
        \]
    \end{solution}

    \question 解方程
    \[
    \sin(\cot^{-1}(x+1)) = \cos(\tan^{-1} x)
    \]
    \begin{solution}
        由 $\cos A = \sin\left(\frac{\pi}{2} - A\right)$,
        \[ 
        \sin[\cot^{-1}(x+1)] = \sin\left[\frac{\pi}{2} - \tan^{-1} x\right] 
        \]
        于是解为
        \[
        \cot^{-1}(x+1) = \frac{\pi}{2} - \tan^{-1} x \quad \text{或} \quad \cot^{-1}(x+1) = \pi - \left(\frac{\pi}{2} - \tan^{-1} x\right) 
        \]
        又由 $\cot^{-1} A = \tan^{-1}\dfrac{1}{A}$,
        \[ 
        \tan^{-1}\left(\frac{1}{x+1}\right) + \tan^{-1} x = \frac{\pi}{2} \quad \text{或} \quad \tan^{-1}\left(\frac{1}{x+1}\right) - \tan^{-1} x = \frac{\pi}{2} 
        \]
        两边取正切,对于左侧方程
        \begin{align*}
        \tan\left[\tan^{-1}\left(\frac{1}{x+1}\right) + \tan^{-1} x\right] &= \tan \frac{\pi}{2} \\
        \frac{\frac{1}{x+1} + x}{1 - \frac{1}{x+1} \cdot x} &= \infty \\
        x^2 + x + 1 &= \infty
        \end{align*}
        得 $x = \pm \infty$,而对于右侧方程,
        \begin{align*}
        \tan\left[\tan^{-1}\left(\frac{1}{x+1}\right) - \tan^{-1} x\right] &= \tan \frac{\pi}{2} \\
        \frac{\frac{1}{x+1} - x}{1 + \frac{1}{x+1} \cdot x} &= \infty \\
        \frac{1 - x^2 - x}{2x + 1} &= \infty \\
        2x + 1 &= 0 \\
        x &= -\frac{1}{2}
        \end{align*}
        所以解为 $x = -\dfrac{1}{2}$,经检验都是方程的解。
    \end{solution}

    \question 解方程
    \[
    (\tan^{-1} x)^2+(\cot^{-1} x)^2=\frac{5\pi^2}{8}.
    \]
    \begin{solution}
        设$\theta=\tan^{-1}x$,原方程式变为
        \[
        \theta^2+\left(\frac{\pi}{2}-\theta\right)^2=\frac{5\pi^2}{8}.
        \]
        整理得
        \[
        (4\theta-3\pi)(4\theta+\pi)=0 \Rightarrow \theta=\frac{3\pi}{4}\quad \text{或}\quad \theta=-\frac{\pi}{4}.
        \]
        注意到$-\dfrac{\pi}{2}\le \tan^{-1}x\le \dfrac{\pi}{2}$,因此舍去 $\theta=\dfrac{3\pi}{4}$,于是
        \[
        x=\tan\left(-\frac{\pi}{4}\right)=-1
        \]
        经检验是原方程的解。
    \end{solution}

    \question 若 
    \[
    \tan^{-1} x + \tan^{-1} y + \tan^{-1} z = \frac{\pi}{2},
    \] 
    证明 
    \[
    xy + yz + zx = 1
    \]
    \begin{solution}
        设$\alpha = \tan^{-1} x, \beta = \tan^{-1} y, \gamma = \tan^{-1} z$,则
        \[
        \alpha + \beta + \gamma = \frac{\pi}{2}.
        \]
        利用三角恒等式
        \[
        \tan(\alpha+\beta+\gamma)
        =\frac{\tan\alpha + \tan\beta + \tan\gamma - \tan\alpha \tan\beta \tan\gamma}
        {1 - \tan\alpha \tan\beta - \tan\beta \tan\gamma - \tan\gamma \tan\alpha}.
        \]
        代入 $\tan\alpha=x,\ \tan\beta=y,\ \tan\gamma=z$,得
        \[
        \tan(\alpha+\beta+\gamma)
        =\frac{x + y + z - xyz}{1 - xy - yz - zx}.
        \]
        由于$\alpha+\beta+\gamma = \dfrac{\pi}{2},\tan(\alpha+\beta+\gamma)$不存在,则
        \[
        1 - xy - yz - zx = 0 \Rightarrow xy + yz + zx = 1
        \]
        注:其逆命题不成立。
    \end{solution}

    \question 若 $x>1$, 求证
    \[
    \tan^{-1} x + \tan^{-1} \frac{1-x}{1+x} = \frac{\pi}{4}.
    \]
    \begin{solution}
        设 $\alpha = \tan^{-1} x, \beta = \tan^{-1} \dfrac{1-x}{1+x}$,于是
        \[
        \tan(\alpha + \beta) = \frac{\tan \alpha + \tan \beta}{1 - \tan \alpha \tan \beta} = \frac{x + \frac{1-x}{1+x}}{1 - x \cdot \frac{1-x}{1+x}}= \frac{x^2+1}{1+x^2} = 1 
        \]
        由于 $x>1$, 
        \[
        \frac{\pi}{4} < \alpha < \frac{\pi}{2}, \quad -1 < \frac{1-x}{1+x} < 0 \Rightarrow -\frac{\pi}{4} < \beta < 0
        \]
        因此 
        \[
        0 < \alpha + \beta < \frac{\pi}{2}
        \]
        这表示
        \[
        \tan^{-1} x + \tan^{-1} \frac{1-x}{1+x} =\alpha + \beta= \frac{\pi}{4}
        \]
    \end{solution}

    \question 证明
    \[
    \tan^{-1}\frac{1}{2}+\tan^{-1}\frac{1}{5}+\tan^{-1}\frac{1}{8}=\frac{\pi}{4}.
    \]
    \begin{solution}
        设
        \[
        A=\tan^{-1}\frac{1}{2},\quad B=\tan^{-1}\frac{1}{5},\quad C=\tan^{-1}\frac{1}{8}
        \]
        利用正切的三角和公式,
        \[
        \tan(A+B+C)
        =\frac{\tan A+\tan B+\tan C-\tan A\cdot\tan B\cdot\tan C}
        {1-(\tan A\tan B+\tan B\tan C+\tan C\tan A)}
        \]
        代入 $\tan A=\dfrac{1}{2},\ \tan B=\dfrac{1}{5},\ \tan C=\dfrac{1}{8}$,得
        \[
        \tan(A+B+C)=1=\tan\frac{\pi}{4}
        \]
        现证明 $\tan^{-1} t<t$:设 $f(x)=\tan x - x$,则 $f'(x)=\sec^2 x - 1 = \tan^2 x > 0$,且 $f(0)=0$,故对所有$x>0$有$f(x)>0$,即 $\tan x > x$对$x=\tan^{-1}t \in \left(0,\dfrac{\pi}{2}\right)$恒成立,从而 $\tan^{-1} t < t$。因此
        \[
        A+B+C < \frac{1}{2}+\frac{1}{5}+\frac{1}{8}=\frac{33}{40}<\frac{\pi}{2}
        \]
        又 $A,B,C>0$,所以 $A+B+C\in\left(0,\dfrac{\pi}{2}\right)$。在此区间上 $\tan\theta$ 严格递增,且仅在 $\theta=\frac{\pi}{4}$ 时 $\tan\theta=1$,故
        \[
        A+B+C=\frac{\pi}{4}
        \]
        证毕。
    \end{solution}
    \begin{solution}
        由恒等式
        \[
        \tan^{-1} x + \tan^{-1} y = \tan^{-1} \frac{x+y}{1-xy}, \quad x>0,\;y>0,\;xy<1
        \]
        有
        \[
        \tan^{-1}\frac{1}{2}+\tan^{-1}\frac{1}{5} = \tan^{-1}\frac{\frac{1}{2}+\frac{1}{5}}{1-\frac{1}{2}\cdot\frac{1}{5}} = \tan^{-1}\frac{7}{9}
        \]
        故原式为
        \[
        \tan^{-1}\frac{7}{9} + \tan^{-1}\frac{1}{8} =\tan^{-1}\frac{\frac{7}{9}+\frac{1}{8}}{1-\frac{7}{9}\cdot\frac{1}{8}} = \tan^{-1}1 = \frac{\pi}{4}
        \]
        即得证。
    \end{solution}

    \question 化简
    \[
    \tan^{-1} \left( \frac{\sqrt{1+\sin x} - \sqrt{1-\sin x}}{\sqrt{1+\sin x} + \sqrt{1-\sin x}} \right), 
    \]
    其中$0 < x < \dfrac{\pi}{4}$。
    \begin{solution}
        有理化分母,可得
        \begin{align*}
        \tan^{-1} \left( \frac{\sqrt{1+\sin x} - \sqrt{1-\sin x}}{\sqrt{1+\sin x} + \sqrt{1-\sin x}} \right) 
        &= \tan^{-1} \left( \frac{(\sqrt{1+\sin x} - \sqrt{1-\sin x})^2}{(1+\sin x) - (1-\sin x)} \right) \\
        &= \tan^{-1} \left( \frac{2 - 2\sqrt{1-\sin^2 x}}{2\sin x} \right) \\
        &= \tan^{-1} \left( \frac{1 - \cos x}{\sin x} \right) \\
        &= \tan^{-1} \left( \frac{2\sin^2 \frac{x}{2}}{2\sin \frac{x}{2} \cos \frac{x}{2}} \right) \\
        &= \tan^{-1} \left(\tan \frac{x}{2}\right) \\
        &= \frac{x}{2}
        \end{align*}
    \end{solution}

    \question 证明
    \[ 
    f(x) = \tan^{-1} (3x) + \sin^{-1} \left[(9x^2+1)^{-\frac{1}{2}}\right] 
    \]
    为一常数,并求此常数。
    \begin{solution}
        对 $x$ 求导,
        \begin{align*}
        f'(x)
        &=\frac{3}{1+(3x)^2}
        +\frac{1}{\sqrt{1-(9x^2+1)^{-1}}}
        \cdot\left(-\frac{1}{2}\right)(18x)(9x^2+1)^{-\frac{3}{2}} \\
        &=\frac{3}{1+9x^2}
        -\frac{9x}{\sqrt{\frac{9x^2}{9x^2+1}}}\cdot\frac{1}{(9x^2+1)^{\frac{3}{2}}} \\
        &=\frac{3}{1+9x^2}
        -3\sqrt{9x^2+1}\cdot\frac{1}{(9x^2+1)^{\frac{3}{2}}} \\
        &=\frac{3}{1+9x^2}-\frac{3}{9x^2+1} \\
        &=0.
        \end{align*}
        因此 $f(x)$ 为常数,取 $x=0$得
        \[
        f(0)=\tan^{-1}0+\sin^{-1}(1)=\frac{\pi}{2}.
        \]
        于是
        \[
        \tan^{-1}(3x)+\sin^{-1}\!\left(\frac{1}{\sqrt{9x^2+1}}\right)=\frac{\pi}{2}
        \]
    \end{solution}

    \question 若 $0 \le x \le 1$, 求证
    \[
    \cos(\sin^{-1} x) < \sin^{-1}(\cos x)
    \]
    \begin{solution}
        设 $\alpha = \sin^{-1} x$,则 $\sin\alpha = x$. 又 $x \in [0, 1],\alpha \in \left[0, \dfrac{\pi}{2}\right]$,从而有 
        \[
        \sqrt{1-x^2} = \cos\alpha
        \]
        故
        \[ x+\sqrt{1-x^2} = \sin\alpha + \cos\alpha = \sqrt{2} \sin\left(\alpha + \frac{\pi}{4}\right) \le \sqrt{2} < \frac{\pi}{2}
        \]
        即 
        \[
        \sqrt{1-x^2} < \frac{\pi}{2} - x
        \]
        由于$x \in [0, 1],x=\cos^{-1}(\cos x) =x$,故得证 
        \[
        \cos(\sin^{-1} x) < \sin^{-1}(\cos x),\quad 0 \le x \le 1
        \]
    \end{solution}

    \question 若 $\alpha, \beta \in (0, 2\pi), \ \alpha \neq \beta$, 且 $\alpha, \beta$ 满足方程
    \[
    \sin x + \sqrt{3} \cos x + a = 0,
    \]
    求实数 $a$ 的取值范围,以及 $\alpha + \beta$ 的值。
    \begin{solution}
        注意到原方程为
        \[
        \sin\left(x + \frac{\pi}{3}\right) = -\frac{a}{2}
        \]
        所以 
        \[
        -\frac{a}{2} \in [-1,1] \Rightarrow a \in [-2,2]
        \]
        设 $\alpha,\beta$ 为原方程的解,有
        \[
        \begin{cases}
        \sin \alpha + \sqrt{3} \cos \alpha + a = 0, \\
        \sin \beta + \sqrt{3} \cos \beta + a = 0.
        \end{cases}
        \]
        两式相减得
        \[
        (\sin\alpha - \sin\beta) + \sqrt{3} (\cos\alpha - \cos\beta) = 0
        \]
        和差化积得
        \[
        2 \cos \frac{\alpha+\beta}{2} \sin \frac{\alpha-\beta}{2} - 2\sqrt{3} \sin \frac{\alpha+\beta}{2} \sin \frac{\alpha-\beta}{2} = 0
        \]
        由于 $\alpha \ne \beta$,有 $\sin \dfrac{\alpha-\beta}{2} \ne 0$,于是
        \[
        \cos \frac{\alpha+\beta}{2} - \sqrt{3} \sin \frac{\alpha+\beta}{2} = 0 \Rightarrow \tan \frac{\alpha+\beta}{2}  = \frac{\sqrt{3}}{3}
        \]
        又 $\alpha+\beta \in (0,4\pi)$,因此
        \[
        \alpha + \beta = \frac{\pi}{3} \quad \text{或} \quad  \frac{7\pi}{3}
        \]
    \end{solution}

    \question 已知方程
    \[
    (\sin^{-1} x)^3 + (\cos^{-1} x)^3 = k\pi^3, |x| \le 1,
    \]
    其中 $k$ 为常数。
    \begin{parts}
    \part 证明该方程有解的一个必要非充分条件是 $k \ge \frac{1}{32}$。
    \begin{solution}
        由恒等式 $\sin^{-1} x + \cos^{-1} x = \dfrac{\pi}{2}$,
        \[
        (\sin^{-1} x)^3 + \left(\frac{\pi}{2} - \sin^{-1} x\right)^3 = k\pi^3 
        \]
        整理得
        \[
        (\sin^{-1} x)^2 - \frac{\pi}{2}(\sin^{-1} x) + \frac{\pi^2}{12}(1-8k) = 0 \tag{1}
        \]
        欲使(1)有实数解,判别式为非负:
        \begin{align*}
        \left(-\frac{\pi}{2}\right)^2 - 4 \cdot 1 \cdot \frac{\pi^2}{12}(1-8k) &\ge 0 \\
        \frac{\pi^2}{4} - \frac{\pi^2}{3}(1-8k) &\ge 0 \\
        \frac{1}{4} - \frac{1}{3} + \frac{8}{3}k &\ge 0 \\
        32k &\ge 1 \\
        k &\ge \frac{1}{32}
        \end{align*}
        由于条件$k \ge \dfrac{1}{32}$可能会生成 $|\sin^{-1} x| > 1$的解即非实数解,故此条件为必要非充分条件。
    \end{solution}
    \part 若方程只有一个解,求该解。
    \begin{solution}
        若只有一个解,则 $k = \dfrac{1}{32}$,方程变为
        \[
        (\sin^{-1} x)^2 - \frac{\pi}{2}(\sin^{-1} x) + \frac{\pi^2}{12}\left(1 - 8 \cdot \frac{1}{32}\right) = 0 
        \]
        即
        \[
        \left(\sin^{-1} x - \frac{\pi}{4}\right)^2 = 0 
        \]
        解得
        \[
        \sin^{-1} x = \frac{\pi}{4} \Rightarrow x = \frac{\sqrt{2}}{2}
        \]
    \end{solution}
    \part 若 $k = \dfrac{7}{96}$,求方程的两个解。
    \begin{solution}
        若 $k = \dfrac{7}{96}$,将方程配方可得
        \[
        \left(\sin^{-1} x - \frac{\pi}{4}\right)^2 = \frac{\pi^2}{36} 
        \]
        于是解为
        \[
        \sin^{-1} x - \frac{\pi}{4} = \pm \frac{\pi}{6} \Rightarrow
        \sin^{-1} x = \frac{5\pi}{12} \quad \text{或} \quad \sin^{-1} x = \frac{\pi}{12} 
        \]
        即
        \[
        x = \sin\frac{5\pi}{12} \quad \text{或} \quad  x = \sin\frac{\pi}{12}
        \]
    \end{solution}
    \end{parts}
\end{questions}

\pagebreak

\begin{center}
  {\fontsize{30pt}{26pt}\selectfont
    \hypertarget{平面向量}{平面向量} \label{平面向量}
  }
\end{center}
\separator
\vspace{1pt}

\begin{questions}
    \question 在坐标平面上的平行四边形 \(ABCD\) 中,若 \(\overrightarrow{AB} = (4,8)\),\(\;\overrightarrow{AD} = (1,4)\),求 $|\overrightarrow{AC}| + |\overrightarrow{BD}|$。
    \begin{solution}
        发现
        \[
        \overrightarrow{AC} = \overrightarrow{AB} + \overrightarrow{AD} = (4 + 1, 8 + 4) = (5,12)  
        \]
        且
        \[
        \overrightarrow{BD} = \overrightarrow{AD} - \overrightarrow{AB} = (1 - 4, 4 - 8) = (-3,-4)  
        \]
        因此
        \[
        |\overrightarrow{AC}| + |\overrightarrow{BD}| =\sqrt{5^2 + 12^2}+\sqrt{(-3)^2 + (-4)^2} = 18
        \]
    \end{solution}

    \question 在 \(\triangle ABC\) 中,已知 \(AB=4,BC=5,AC=6\),求 $\overrightarrow{AB} \cdot \overrightarrow{BC}$ 。
    \begin{solution}
    由余弦定理,
    \[
    6^2 = 4^2 + 5^2 - 2 \cdot 4 \cdot 5 \cdot \cos{\angle ABC}
    \Rightarrow \cos{\angle ABC} = \frac{1}{8}
    \]
    所以
    \begin{align*}
        \overrightarrow{AB} \cdot \overrightarrow{BC} &= |\overrightarrow{AB}| \cdot |\overrightarrow{BC}| \cdot \cos{\theta} \\
        &= 4 \cdot 5 \cdot \cos{(\pi - \angle ABC)} \\
        &= 4 \cdot 5 \cdot (-\cos{\angle ABC}) \\
        &= -\frac{5}{2}
    \end{align*}
    \end{solution}
    
    \question 已知 \(\mathbf{a} + \mathbf{b} + \mathbf{c} = \mathbf{0}\),且 \(|\mathbf{a}|=3\), \(|\mathbf{b}|=5\), \(|\mathbf{c}|=7\),试求 $|2\mathbf{a} + \mathbf{b}| $。
    \begin{solution}
        由已知
        \[
        \mathbf{a} + \mathbf{b} + \mathbf{c} = \mathbf{0}
        \Rightarrow \mathbf{a} + \mathbf{b} = -\mathbf{c} \Rightarrow |\mathbf{a} + \mathbf{b}| = |-\mathbf{c}| = |\mathbf{c}| \Rightarrow |\mathbf{a} + \mathbf{b}|^2 = |\mathbf{c}|^2
        \]
        展开得
        \[ |\mathbf{a}|^2 + 2\mathbf{a} \cdot \mathbf{b} + |\mathbf{b}|^2 = |\mathbf{c}|^2
        \Rightarrow 3^2 + 2\mathbf{a} \cdot \mathbf{b} + 5^2 = 7^2
        \Rightarrow \mathbf{a} \cdot \mathbf{b} = \frac{15}{2}
        \]
        于是
        \[
         |2\mathbf{a} + \mathbf{b}|^2 = 4|\mathbf{a}|^2 + 4\mathbf{a} \cdot \mathbf{b} + |\mathbf{b}|^2
        = 4 \cdot 3^2 + 4 \cdot \frac{15}{2} + 5^2 = 91
        \Rightarrow |2\mathbf{a} + \mathbf{b}| = \sqrt{91}
        \]
    \end{solution}

    \question 已知 $|\mathbf{a}|=1$, 且 $\mathbf{b}\cdot(\mathbf{b}-\mathbf{a})=0$, 求 $|\mathbf{b}-\mathbf{a}||\mathbf{b}+\mathbf{a}|$ 的最大值。
    \begin{solution}
        由 $\mathbf{b}\cdot(\mathbf{b}-\mathbf{a})=0$ 得
        \[
        |\mathbf{b}|^2 - \mathbf{a}\cdot \mathbf{b} = 0 \implies \mathbf{a}\cdot \mathbf{b} = |\mathbf{b}|^2
        \]
        于是
        \begin{align*}
        |\mathbf{b}-\mathbf{a}|^2|\mathbf{b}+\mathbf{a}|^2 
        &= [(\mathbf{b}-\mathbf{a})\cdot(\mathbf{b}-\mathbf{a})][(\mathbf{b}+\mathbf{a})\cdot(\mathbf{b}+\mathbf{a})] \\
        &= (|\mathbf{b}|^2 - 2 \mathbf{a}\cdot \mathbf{b} + |\mathbf{a}|^2)(|\mathbf{b}|^2 + 2 \mathbf{a}\cdot \mathbf{b} + |\mathbf{a}|^2) \\
        &= (|\mathbf{b}|^2 - 2|\mathbf{b}|^2 + 1)(|\mathbf{b}|^2 + 2|\mathbf{b}|^2 + 1) \\
        &= (1 - |\mathbf{b}|^2)(3|\mathbf{b}|^2 + 1)
        \end{align*}
        设 $|\mathbf{b}|^2 = m$, 配方得
        \[
        |\mathbf{b}-\mathbf{a}|^2|\mathbf{b}+\mathbf{a}|^2 = (1 - m)(3m + 1) = -3m^2 + 2m + 1 = -3\left(m - \frac{1}{3}\right)^2 + \frac{4}{3}
        \]
        当 $m = \dfrac{1}{3}$ 时取得最大值$\dfrac{4}{3}$,此时
        \[
        |\mathbf{b}-\mathbf{a}||\mathbf{b}+\mathbf{a}| = \frac{2\sqrt{3}}{3}
        \]
    \end{solution}

    \question 平面上非零向量 $\mathbf{a},\mathbf{b},\mathbf{c}$ 满足 $\mathbf{a}\perp\mathbf{b}$, 且 $\mathbf{b}\cdot\mathbf{c}=20|\mathbf{a}|\cdot|\mathbf{b}|,\mathbf{c}\cdot\mathbf{a}=25|\mathbf{b}|\cdot|\mathbf{c}|$, 求 $|\mathbf{c}|$ 的最小值。
    \begin{solution}
        设向量 $\mathbf{b}$ 与 $\mathbf{c}$ 的夹角为 $\theta$, 又因 $\mathbf{a}\perp\mathbf{b}$, 所以 $\mathbf{a}$ 与 $\mathbf{c}$ 的夹角为 $90^\circ-\theta$。由内积定义,
        \[
        \cos\theta=\frac{\mathbf{b}\cdot\mathbf{c}}{|\mathbf{b}||\mathbf{c}|},\quad
        \cos(90^\circ-\theta)=\frac{\mathbf{c}\cdot\mathbf{a}}{|\mathbf{c}||\mathbf{a}|}.
        \]
        据题意有,
        \[
        |\mathbf{c}|\cos\theta=\frac{20|\mathbf{a}|}{|\mathbf{b}|},\quad
        |\mathbf{c}|\sin\theta=\frac{25|\mathbf{b}|}{|\mathbf{a}|}.
        \]
        由AM-GM不等式,
        \[
        |\mathbf{c}|^2
        =\frac{400|\mathbf{a}|^2}{|\mathbf{b}|^2}
        +\frac{625|\mathbf{b}|^2}{|\mathbf{a}|^2}
        \ge 2\sqrt{400\cdot 625}=1000
        \]
        所以
        \[
        |\mathbf{c}|\ge 10\sqrt{10}
        \]
    \end{solution}
    
    \question 设 $\vec{a},\vec{b}$ 为两个垂直的平面向量, 且 $|\vec{a}|=2|\vec{b}|=10$. 当 $0 \le t \le 1$ 时, 记向量 $t\vec{a}+(1-t)\vec{b}$ 与向量 $(t-\frac{1}{5})\vec{a}+(1-t)\vec{b}$ 最大夹角为 $\theta$, 则 $\cos\theta=$
    \begin{solution}
        \textcolor{red}{(待解)}
    \end{solution}

    \question 已知 $\triangle ABC$ 中,$\ AB=3,AC=6,BC=7,O$ 为外心,$I$ 为内心,求 $\overrightarrow{OI} \cdot \overrightarrow{BC}$。
    \ifprintanswers
    \begin{figure}[H]
        \centering
        \includegraphics[width=0.4\textwidth]{images/image32.png}
    \end{figure}
    \fi
    \begin{solution}
        内心$I$的向量表示为
        \[
        \overrightarrow{OI} 
        = \frac{7\overrightarrow{OA}+6\overrightarrow{OB}+3\overrightarrow{OC}}{7+6+3}
        \]
        于是
        \begin{align*}
        \overrightarrow{OI} \cdot \overrightarrow{BC} 
        &= \frac{7}{16} \overrightarrow{OA} \cdot \left( \overrightarrow{BA} + \overrightarrow{AC} \right)
        + \frac{6}{16} \overrightarrow{OB} \cdot \overrightarrow{BC}
        + \frac{3}{16} \overrightarrow{OC} \cdot \overrightarrow{BC}\\
        &= \frac{7}{16} \left( \frac{1}{2} \cdot 3^2 - \frac{1}{2} \cdot 6^2 \right)
        + \frac{6}{16} \left( -\frac{1}{2} \cdot 7^2 \right)
        + \frac{3}{16} \left( \frac{1}{2} \cdot 7^2 \right)
        = -\frac{21}{2}
        \end{align*}
        其中$O$为外心时,有
        \begin{align*}  
        \overrightarrow{OX} \cdot \overrightarrow{YX}
        &=\overrightarrow{OX} \cdot (\overrightarrow{OX}-\overrightarrow{OY})\\
        &=|\overrightarrow{OX}|^2-\overrightarrow{OX} \cdot \overrightarrow{OY}\\
        &=|\overrightarrow{OX}|^2-\frac{1}{2}(|\overrightarrow{OX}|^2+|\overrightarrow{OY}|^2-|\overrightarrow{XY}|^2)\\
        &=\frac{1}{2}|\overrightarrow{XY}|^2
        \end{align*}
    \end{solution}

\question 若 $H$ 为 $\triangle ABC$ 的垂心,且
\[
\overrightarrow{AH} = \frac{1}{3}\overrightarrow{AB} + \frac{1}{4}\overrightarrow{AC},
\]
求 $\cos \angle BAC$。
\ifprintanswers
\begin{figure}[H]
    \centering
    \includegraphics[width=0.5\linewidth]{images/image53.png}
\end{figure}
\fi
\begin{solution}
因 $A,D,C$ 共线,且
\[
\overrightarrow{AH} = \frac{1}{3}\overrightarrow{AB} + \frac{1}{4}\overrightarrow{AC} = \frac{1}{3}\overrightarrow{AB} + \frac{2}{3}\overrightarrow{AD} \implies \overline{AD} : \overline{AC} = 3 : 8,????
\]
设
\[
A(0,0),\quad D(3k,0),\quad C(8k,0),\quad H(3k,t),\quad B(3k,s),
\]
则
\[
(3k,t) = \frac{1}{3}(3k,s) + \frac{1}{4}(8k,0) \implies s=3t,
\]
所以
\[
B(3k,3t).
\]

由垂心定义,有
\[
\overrightarrow{AH} \perp \overrightarrow{BC} \implies (3k,t) \cdot (5k, -3t) = 15k^2 - 3t^2 = 0 \implies t = \sqrt{5}k.
\]

计算
\[
\cos \angle BAC = \frac{\overrightarrow{AB} \cdot \overrightarrow{AC}}{|\overrightarrow{AB}||\overrightarrow{AC}|} = \frac{24k^2}{\sqrt{9k^2 + 9t^2} \cdot 8k} = \frac{24k^2}{\sqrt{54k^2} \cdot 8k} = \frac{1}{\sqrt{6}} = \frac{\sqrt{6}}{6}.
\]
\textcolor{red}{(待验证)}
\end{solution}

    \question 已知三角形 \(ABC\) 的外心 \(O\) 满足
    \[
    2\overrightarrow{OA}+3\overrightarrow{OB}+4\overrightarrow{OC}=0.
    \]
    求\(\sin A\).
    \begin{solution}
        由题意
        \[        
        3\overrightarrow{OB}+4\overrightarrow{OC}=-2\overrightarrow{OA}.
        \]
        两边平方得
        \[        
        9|\overrightarrow{OB}|^2+24\overrightarrow{OB}\cdot\overrightarrow{OC}+16|\overrightarrow{OC}|^2=4|\overrightarrow{OA}|^2
        \]
        注意到外心到各顶点距离相等,设 \(R=|\overrightarrow{OA}|=|\overrightarrow{OB}|=|\overrightarrow{OC}|\),代入得
        \[        
        9R^2+24\overrightarrow{OB}\cdot\overrightarrow{OC}+16R^2=4R^2.
        \]
        整理得
        \[
        \overrightarrow{OB}\cdot\overrightarrow{OC}=-\frac{7}{8}R^2.
        \]
        $\text{又因圆心角}=2\times\text{圆周角}$,
        \[        
        \overrightarrow{OB}\cdot\overrightarrow{OC}=|\overrightarrow{OB}||\overrightarrow{OC}|\cos\angle BOC=R^2\cos 2A,
        \]
        有
        \[
        \cos 2A=1-2\sin^2A=-\frac{7}{8}
        \Rightarrow \sin A=\frac{\sqrt{15}}{4}\quad(\sin A>0).
        \]
    \end{solution}
    
    \question 设 $\triangle ABC$ 的外心为$O$,重心为$H$,若 $|\overrightarrow{OA}+\overrightarrow{OB}+\overrightarrow{OC}|=\sqrt{3}$,求 $|\overrightarrow{HA}+\overrightarrow{HB}+\overrightarrow{HC}|$。
    \begin{solution}
        设 $O$ 为外心,$G$ 为重心,$H$ 为垂心,由于 $G$ 是重心,有 $\overrightarrow{GA} + \overrightarrow{GB} + \overrightarrow{GC} = \overrightarrow{0}$,于是
        \[
        |\overrightarrow{OA} + \overrightarrow{OB} + \overrightarrow{OC}| 
        = |3\overrightarrow{OG} + (\overrightarrow{GA} + \overrightarrow{GB} + \overrightarrow{GC})|=3|\overrightarrow{OG}| = \sqrt{3} \Rightarrow |\overrightarrow{OG}| = \frac{\sqrt{3}}{3}
        \]
        且
        \[
        |\overrightarrow{HA} + \overrightarrow{HB} + \overrightarrow{HC}|
        = |3\overrightarrow{HG} + (\overrightarrow{GA} + \overrightarrow{GB} + \overrightarrow{GC})|
        = 3|\overrightarrow{HG}|
        \]
        由欧拉线定理,
        \[
        |\overrightarrow{HG}| = 2|\overrightarrow{OG}| = \frac{2\sqrt{3}}{3}
        \]
        所以
        \[
        |\overrightarrow{HA} + \overrightarrow{HB} + \overrightarrow{HC}| = 2\sqrt{3}
        \]
    \end{solution}

    \question 
    \begin{parts}
    \part 在 $\triangle ABC$ 中,有 $\overrightarrow{AB} \cdot \overrightarrow{BC} = \overrightarrow{BC} \cdot \overrightarrow{CA} = \overrightarrow{CA} \cdot \overrightarrow{AB}$, 求$\cos A$。 
    \begin{solution}
        设 $a=|\overrightarrow{BC}|,b=|\overrightarrow{AC}|,c=|\overrightarrow{AB}|$,由余弦定理,
        \[
        a^2=b^2+c^2-2bc\cos A=b^2+c^2-2|\overrightarrow{AC}||\overrightarrow{AB}|\cos A
        \]
        故
        \[
        |\overrightarrow{AC}||\overrightarrow{AB}|\cos A
        =-\overrightarrow{CA}\cdot\overrightarrow{AB}
        =\frac{b^2+c^2-a^2}{2}.
        \]
        同理,
        \[
        |\overrightarrow{AB}||\overrightarrow{BC}|\cos B
        =-\overrightarrow{AB}\cdot\overrightarrow{BC}
        =\frac{a^2+c^2-b^2}{2},
        \]
        \[
        |\overrightarrow{AC}||\overrightarrow{BC}|\cos C
        =-\overrightarrow{BC}\cdot\overrightarrow{CA}
        =\frac{a^2+b^2-c^2}{2}.
        \]
        因此
        \[
        \frac{a^2+b^2-c^2}{2}=\frac{b^2+c^2-a^2}{2}=\frac{a^2+c^2-b^2}{2}
        \Rightarrow a=b=c
        \]
        即$\triangle ABC$ 为等边三角形,有
        \[
        \cos A=\frac{1}{2}
        \]
    \end{solution} 
    \part 在$\triangle ABC$ 中, $2\,\overrightarrow{AB} \cdot \overrightarrow{BC} = \overrightarrow{BC} \cdot \overrightarrow{CA} = 3\,\overrightarrow{CA} \cdot \overrightarrow{AB}$, 求 $\cos A$。
    \begin{solution}
        由上,
        \[
        3\,\overrightarrow{CA}\cdot\overrightarrow{AB}=-\frac{3}{2}(b^2+c^2-a^2)
        \]
        \[
        \overrightarrow{BC}\cdot\overrightarrow{CA}=-\frac12(a^2+b^2-c^2)
        \]
        \[
        2\,\overrightarrow{AB}\cdot\overrightarrow{BC}=-(a^2+c^2-b^2)
        \]
        设
        \[
        \frac{3}{2}(b^2+c^2-a^2)
        =\frac12(a^2+b^2-c^2)
        =a^2+c^2-b^2=3t
        \]
        可解得
        \[
        a^2=\frac{9t}{2},b^2=4t,c^2=\frac{5t}{2}
        \]
        于是
        \[
        \cos A=\frac{b^2+c^2-a^2}{2bc}
        =\frac{2t}{2\sqrt{10}\,t}
        =\frac{\sqrt{10}}{10}
        \]
    \end{solution}
    \end{parts}

    \question 已知 $G$ 为 $\triangle ABC$ 的重心,且 
    \[
    6\overrightarrow{GA}\cdot\overrightarrow{AB} = 3\overrightarrow{GB}\cdot\overrightarrow{BC} = 2\overrightarrow{GC}\cdot\overrightarrow{CA},
    \] 
    求 $\sin A$。
    \begin{solution}
        设 $AB=c,BC=a,CA=b$,由余弦定理
        \[
        \overrightarrow{GA}\cdot \overrightarrow{AB}=-\overrightarrow{AG}\cdot \overrightarrow{AB}
        =-\frac{1}{3}(\overrightarrow{AB}+\overrightarrow{AC})\cdot \overrightarrow{AB}
        =-\frac{1}{3} \left(c^2+\frac{b^2+c^2-a^2}{2}\right)
        =-\frac{b^2+3c^2-a^2}{6},
        \]
        同理
        \[
        \overrightarrow{GB}\cdot\overrightarrow{BC}
        =-\frac{c^2+3a^2-b^2}{6},\quad
        \overrightarrow{GC}\cdot\overrightarrow{CA}
        =-\frac{a^2+3b^2-c^2}{6}.
        \]
        由题意,可解得
        \[
        b^2+3c^2-a^2=\frac{1}{2}\left(c^2+3a^2-b^2\right)
        =\frac{1}{3}\left(a^2+3b^2-c^2\right) \Rightarrow a^2=b^2=\frac{5}{2}c^2
        \]
        由余弦定理,
        \[
        \cos A=\frac{b^2+c^2-a^2}{2bc}=\frac{1}{2}\cdot\frac{c}{b}
        =\frac{1}{2}\cdot\sqrt{\frac{2}{5}}=\frac{\sqrt{10}}{10}
        \]
        所以
        \[
        \sin A=\sqrt{1-\cos^2 A}=\frac{3\sqrt{10}}{10}.
        \]
    \end{solution}

    \question 圆之内接四边形 $ABCD$满足 
    \[
    \overrightarrow{AC} = \frac{3}{2}\overrightarrow{AB} + \frac{5}{2}\overrightarrow{AD}
    \]
    求 $\dfrac{\sin \angle DAB}{\sin \angle ABC}$。
    \ifprintanswers
    \begin{figure}[H]
        \centering
        \includegraphics[width=0.4\linewidth]{images/image45.png}
    \end{figure}
    \fi
    \begin{solution}
        设 $AC$ 与 $BD$ 交于 $P$,并令
        \[
        \overrightarrow{AP} = t\overrightarrow{AC} = \frac{3t}{2}\overrightarrow{AB} + \frac{5t}{2}\overrightarrow{AD} 
        \]
        由于$P$在$BD$上,解得
        \[
        \frac{3t}{2} + \frac{5t}{2} = 1 \Rightarrow t = \frac{1}{4}
        \]
        于是
        \[
        \overrightarrow{AP} = \frac{3}{8}\overrightarrow{AB} + \frac{5}{8}\overrightarrow{AD} \Rightarrow
        DP = 3k,PB = 5k,k>0
        \]
        同理,
        \[
        \overrightarrow{BD} = \overrightarrow{BA} + \overrightarrow{AD} = \overrightarrow{BA} + \frac{2}{5}(\overrightarrow{AC} - \frac{3}{2}\overrightarrow{AB}) = \frac{8}{5}\overrightarrow{BA} + \frac{2}{5}\overrightarrow{AC} = \frac{6}{5}\overrightarrow{BA} + \frac{2}{5}\overrightarrow{BC}
        \]
        解得
        \[
        \overrightarrow{BP} = t\overrightarrow{BD} = \frac{6t}{5}\overrightarrow{BA} + \frac{2t}{5}\overrightarrow{BC} \Rightarrow t = \frac{5}{8}
        \Rightarrow \overrightarrow{BP} = \frac{3}{4}\overrightarrow{BA} + \frac{1}{4}\overrightarrow{BC}
        \Rightarrow AP = m, PC = 3m
        \]
        由圆幂定理,
        \[
        AP \cdot PC = BP \cdot PD \Rightarrow 3m^2 = 15k^2 \Rightarrow \frac{k}{m} = \frac{1}{\sqrt{5}}
        \]
        由正弦定理,
        \[
        \frac{\sin \angle DAB}{\sin \angle ABC} = \frac{AC}{BD} = \frac{8k}{4m} = \frac{2\sqrt{5}}{5}
        \]
    \end{solution}

    \question 设 $\triangle ABC$ 的三边 $BC,CA,AB$ 的长度分别为 $a,b,c$, 在边 $BC,CA,AB$ 上分别取点 $L,M,N$使得 $BL:LC=c:b,CM:MA=a:c,AN:NB=b:a$。若 
    \[
    b\overrightarrow{BM}+c\overrightarrow{CN}+a\overrightarrow{AL}=\vec{0},
    \]
    试证$\triangle ABC$ 为正三角形。
    \begin{solution}
        据题意有
        \begin{align*}
        \overrightarrow{AL}&=\frac{b}{b+c}\overrightarrow{AB}+\frac{c}{b+c}\overrightarrow{AC} \\[1mm]
        \overrightarrow{BM}&=-\overrightarrow{AB}+\overrightarrow{AM}
        =-\overrightarrow{AB}+\frac{c}{a+c}\overrightarrow{AC} \\
        \overrightarrow{CN}
        &=\overrightarrow{CB}+\overrightarrow{BN}
        =\overrightarrow{AB}-\overrightarrow{AC}-\frac{a}{a+b}\overrightarrow{AB}
        =\frac{b}{a+b}\overrightarrow{AB}-\overrightarrow{AC}
        \end{align*}
        故$b\overrightarrow{BM}+c\overrightarrow{CN}+a\overrightarrow{AL}$ 为
        \[
        b\left(-\overrightarrow{AB}+\frac{c}{a+c}\overrightarrow{AC}\right)
        +c\left(\frac{b}{a+b}\overrightarrow{AB}-\overrightarrow{AC}\right)
        +a\left(\frac{b}{b+c}\overrightarrow{AB}+\frac{c}{b+c}\overrightarrow{AC}\right)
        \]
        \[
        =\left(-b+\frac{bc}{a+b}+\frac{ab}{b+c}\right)\overrightarrow{AB}
        +\left(\frac{bc}{a+c}-c+\frac{ac}{b+c}\right)\overrightarrow{AC}=\overrightarrow{0}
        \]
        于是由余弦定理,
        \[
        -b+\frac{bc}{a+b}+\frac{ab}{b+c}=0
        \Rightarrow \frac{c}{a+b}+\frac{a}{b+c}=1
        \Rightarrow b^2=a^2+c^2-ac
        \Rightarrow \angle B=60^\circ
        \]
        同理
        \[
        \frac{bc}{a+c}-c+\frac{ac}{b+c}=0
        \Rightarrow \angle C=60^\circ
        \]
        因此得证 $\triangle ABC$ 为正三角形。
    \end{solution}

    \question 点 \(O\) 在三角形 \(\triangle ABC\) 内,且满足 $\overrightarrow{OA} + 2\overrightarrow{OB} + 3\overrightarrow{OC} = \overrightarrow{0}$,求三角形面积的比值 $\frac{[\triangle ABC]}{[\triangle AOC]} =$ \ans{$3$}
    \begin{solution}
        \textcolor{red}{(待解)}
    \end{solution}

    \question 已知两向量 $|\textbf{a}| = |\textbf{b}| = 1$, 且 $\textbf{a}\cdot\textbf{b}=\dfrac12$, 若向量 $\textbf{c}$ 满足 $\textbf{a}-\textbf{c}$ 与 $\textbf{b}-\textbf{c}$ 夹角为 $120^\circ$, 求 $|\textbf{c}|$ 的最小值。
    \ifprintanswers
    \begin{figure}[H]
        \centering
        \includegraphics[width=0.4\linewidth]{images/image228.png}
    \end{figure}
    \fi
    \begin{solution}
        设$\textbf{a}=\overrightarrow{OA}, \textbf{b}=\overrightarrow{OB}, \textbf{c}=\overrightarrow{OC}$,其中 $O$ 为原点。由
        \[
        |\textbf{a}|=|\textbf{b}|=1,\quad \textbf{a}\cdot\textbf{b}=\frac12
        \]
        可知 $A,B$ 在单位圆上,且 $\angle AOB=60^\circ$。当 $|\textbf{c}|$ 最小时,有$|\textbf{a}-\textbf{c}|=|\textbf{b}-\textbf{c}|$,此时 $C$ 在 $AB$ 的中垂线上,且靠近原点 $O$。设$A(1,0), B\left(\dfrac12,\dfrac{\sqrt3}{2}\right)$, 且$P$ 为 $AB$ 中点,$OP$ 为等边 $\triangle OAB$ 的高,有
        \[
        OP=\frac{\sqrt3}{2}.
        \]
        $\triangle ACP$ 为 $30^\circ-60^\circ-90^\circ$ 三角形,故
        \[
        CP=\frac{1}{2\sqrt3}
        \]
        因此
        \[
        |\textbf{c}|=OC=OP-CP=\frac{\sqrt3}{3}
        \]
    \end{solution}

    \question 已知 $\textbf{a}$, $\textbf{b}$, $\textbf{c}$ 为三个非零向量,其中 $|\textbf{a}|=4$, $|\textbf{b}|=6$, $\textbf{a}$ 在 $\textbf{b}$ 上的正射影长为 1,若 $(\textbf{c}-\textbf{a})\cdot(\textbf{c}-\textbf{b})=0$,试求 $|\textbf{c}|$ 的最大可能值。
    \begin{solution}
        设
        \[
        \begin{cases}
        B(6,0) \\
        A(1,\sqrt{15}) \\
        C(a,b)
        \end{cases}
        \Rightarrow
        \begin{cases}
        \textbf a = \overrightarrow{OA} \\
        \textbf b = \overrightarrow{OB} \\
        \textbf c = \overrightarrow{OC}
        \end{cases}
        \Rightarrow
        \begin{cases}
        \textbf c - \textbf a = (a-1, b - \sqrt{15}) \\
        \textbf c - \textbf b = (a - 6, b)
        \end{cases}
        \]
        则
        \[
        (\textbf c - \textbf a) \cdot (\textbf c - \textbf b) = (a - 1)(a - 6) + b(b - \sqrt{15}) = 0
        \]
        由拉格朗日乘子法,令
        \[
        \begin{cases}
        f(a,b) = a^2 + b^2 \\
        g(a,b) = (a - 1)(a - 6) + b(b - \sqrt{15})
        \end{cases}
        \Rightarrow
        \begin{cases}
        f_a = \lambda g_a \\
        f_b = \lambda g_b
        \end{cases}
        \Rightarrow
        \begin{cases}
        2a = (2a - 7)\lambda \\
        2b = (2b - \sqrt{15})\lambda
        \end{cases}
        \]
        则
        \[
        \frac{a}{b} = \frac{2a - 7}{2b - \sqrt{15}} \Rightarrow b = \frac{\sqrt{15}}{7} a \quad 
        \]
        代入 $g(a,b) = 0$得
        \[
        32a^2 - 224a + 147 = 0 \Rightarrow a = \frac{28 + 7\sqrt{10}}{8}
        \]
        故
        \[
        |\textbf c| = \sqrt{f(a,b)} = \sqrt{a^2 + \frac{15}{49}a^2} = \frac{8}{7} \cdot \frac{28 + 7\sqrt{10}}{8} = 4 + \sqrt{10}
        \]
        \textcolor{red}{(非通解)}
    \end{solution}

    \question 已知 $\triangle ABC$ 为边长为 1 的正三角形,设 $BC$ 边上有 $n-1$ 个等分点,由 $B$ 点到 $C$ 点的顺序为$P_1, P_2, \dots, P_{n-1}$,且令 $B=P_0, C=P_n$。若
    \[
    S_n = \overrightarrow{AB} \cdot \overrightarrow{AP_0} + \overrightarrow{AP_1} \cdot \overrightarrow{AP_2} + \overrightarrow{AP_3} \cdot \overrightarrow{AP_4} + \dots + \overrightarrow{AP_{n-1}} \cdot \overrightarrow{AC},
    \]
    求 $\displaystyle \lim_{n \to \infty} \frac{S_n}{n}$。
    \begin{solution}
        由 $B=P_0, C=P_n$ 可得
        \[
        |\overrightarrow{AP_0}| = |\overrightarrow{AB}| =1, \quad 
        \overrightarrow{P_0P_k} = \frac{k}{n} \overrightarrow{P_0P_n} = \frac{k}{n}\overrightarrow{BC}, \quad k=0,1,\dots,n
        \]
        则
        \[
        S_n = \sum_{k=0}^{n-1} \overrightarrow{AP_k} \cdot \overrightarrow{AP_{k+1}}
        = \sum_{k=0}^{n-1} (\overrightarrow{AP_0} + \overrightarrow{P_0P_k}) \cdot (\overrightarrow{AP_0} + \overrightarrow{P_0P_{k+1}})
        \]
        展开得
        \[
        S_n = \sum_{k=0}^{n-1} \Big( |\overrightarrow{AP_0}|^2 + \overrightarrow{AP_0} \cdot \overrightarrow{P_0P_{k+1}} + \overrightarrow{P_0P_k} \cdot \overrightarrow{AP_0} + \overrightarrow{P_0P_k} \cdot \overrightarrow{P_0P_{k+1}} \Big)
        \]
        代入 $\overrightarrow{P_0P_k} = \dfrac{k}{n} \overrightarrow{BC}$ 及 $\overrightarrow{AP_0} = \overrightarrow{AB}$得
        \[
        S_n = \sum_{k=0}^{n-1} \Big( 1 + \overrightarrow{AB} \cdot \frac{2k+1}{n} \overrightarrow{BC} + \frac{k(k+1)}{n^2} \Big)
        = \sum_{k=0}^{n-1} \Big( 1 + \frac{2k+1}{n}\left(-\frac{1}{2}\right) + \frac{k(k+1)}{n^2} \Big)
        \]
        化简得
        \[
        S_n = \sum_{k=0}^{n-1} \left(1 - \frac{2k+1}{2n} + \frac{k(k+1)}{n^2}\right)
        = n - \frac{n^2 - n +1}{2n} + \frac{n^2 -1}{3n}
        \]
        所以
        \[
        \lim_{n \to \infty} \frac{S_n}{n} = \lim_{n \to \infty} \left( 1 - \frac{n^2 - n +1}{2n^2} + \frac{n^2 -1}{3n^2} \right)
        = 1 - \frac{1}{2} + \frac{1}{3} = \frac{5}{6}
        \]
    \end{solution}

\end{questions}

\pagebreak

\begin{center}
  {\fontsize{30pt}{26pt}\selectfont
    \hypertarget{直角坐标}{直角坐标} \label{直角坐标}
  }
\end{center}
\separator
\vspace{1pt}

\begin{questions}
    \question $\triangle ABC$ 的三边 $BC, AC, AB$, 其中点坐标分别为 $(1,5),(-2,1),(4,3)$,试求 $A, B, C$ 的坐标。
    \begin{solution}
        设$A(a_1,a_2),\; B(b_1,b_2),\; C(c_1,c_2),$据题意有
        \[
        \begin{cases}
        \dfrac{b_1+c_1}{2}=1 \\[6pt] \dfrac{a_1+c_1}{2}=-2 \\[6pt] \dfrac{a_1+b_1}{2}=4
        \end{cases}
        \quad
        \text{及}
        \quad
        \begin{cases}
        \dfrac{b_2+c_2}{2}=5 \\[6pt] \dfrac{a_2+c_2}{2}=1 \\[6pt] \dfrac{a_2+b_2}{2}=3
        \end{cases}
        \]
        解得 $A(1,-1),\; B(7,7),\; C(-5,3)$
    \end{solution}

    \question 已知 $A(1,2),B(3,3)$, 线段 $AB$ 绕某点 $P$ 旋转后变成了线段 $A'B'$, 其中 $A'(3,1),B'(4,3)$。求 $P$ 的坐标。
    \begin{solution}
        发现旋转中心 $P$ 即线段 $AA'$ 和 $BB'$ 垂直平分线的交点。$A(1,2)$,$A'(3,1)$中点为$\left(2,\dfrac{3}{2}\right)$,于是垂直平分线为 
        \[
        y - \frac{3}{2} = 2(x - 2) \Rightarrow y = 2x - \frac{1}{2}
        \] 
        同理可得 $B(3,3),B'(4,3)$垂直平分线为
        \[
        x = \frac{7}{2}
        \] 
        联立方程解得 $P\left( \dfrac{7}{2}, \dfrac{9}{2} \right)$
    \end{solution}
    
    \question 直线 $y = mx + 2$ 与 $|x| + |y| = 1$ 相交, 求 $m$ 的范围。
    \begin{solution}
        即求方程组 
        \[
        \begin{cases}
        y = mx + 2 \\ |x| + |y| = 1
        \end{cases}
        \]
        有实数解时$m$的范围。发现,由作图知即求\[
        \begin{cases}
        y = mx + 2 \\ x + |y| = 1
        \end{cases}
        \]有实数解时$m$的范围。分情况再以判别式的性质可求得 $m \ge 2 , m \le -2$
    \end{solution}
          
    \question 设 $P_1(1,1),\; P_2(-2,-1)$, 直线 $x + y + 1 = 0$ 与 $P_1P_2$ 相交于点 $P$, 求 $\dfrac{P_1P}{PP_2}$。
    \begin{solution}
        是否在考虑作$P_1P_2$的直线方程式$\rightarrow$与直线 $x + y + 1 = 0$联立解得点$P\rightarrow$用距离公式?其实只需这样:由相似三角形,
        \[
        \begin{aligned}
        \dfrac{P_1P}{PP_2} 
        &= \frac{\left|\dfrac{1+1+1}{\sqrt{1^2+1^2}}\right|}{\left|\dfrac{-2-1+1}{\sqrt{1^2+1^2}}\right|}
        &= \dfrac{3}{2}
        \end{aligned}
        \]轻轻松松。
    \end{solution}
    
    \question 平面上有两定点 $A(1,4)$ 与 $B(5,2)$,若点 $P$ 在直线 $L: x+y-6=0$ 上移动,则 $|PA-PB|$ 的最大值是多少?
    \begin{solution}
        设 $P(t, 6 - t)$,则
        \begin{align*}
        f(t) = |PA - PB| 
        &= \left| \sqrt{(t - 1)^2 + (2 - t)^2} - \sqrt{(t - 5)^2 + (4 - t)^2} \right|\\
        &= \left| \frac{((t - 1)^2 + (2 - t)^2) -((t - 5)^2 + (4 - t)^2)}{
        \sqrt{(t - 1)^2 + (2 - t)^2} + \sqrt{(t - 5)^2 + (4 - t)^2}} \right|\\
        &= \left| \frac{12t - 36}{\sqrt{2t^2 - 6t + 5} + \sqrt{2t^2 - 18t + 41}} \right| \\
        &= \left| \frac{12 - \frac{36}{t}}{\sqrt{2 - \frac6t + \frac{5}{t^2}} + \sqrt{2 - \frac{18}{t} + \frac{41}{t^2}}} \right|
        \end{align*}
        故
        \[
        \lim_{t \to \pm\infty} f(t) = \frac{12}{2\sqrt{2}} = 3\sqrt{2}
        \]
    \end{solution}

    \question 已知坐标平面上三条直线 $L, L_1, L_2$, 若 $L$ 为水平线, $L_1, L_2$ 的斜率分别为 $\dfrac{2}{3}$ 与 $-\dfrac{3}{2}$, 且 $L$ 被 $L_1, L_2$ 截出的线段长度为 $26$,求三线所围成的三角形面积。
    \begin{solution}
        观察到$ m_{L_1}\cdot m_{L_2}=\dfrac{2}{3} \cdot \left(-\dfrac{3}{2}\right)=-1$,即$L_1 \perp L_2$,故欲求三角形的高。设$L_1, L_2$分别交$L$于点$(p,q),(p+26,q)$,解
        \[
        \begin{cases}
        y-q=\dfrac{2}{3}(x-p)\\ y-q=-\dfrac{3}{2}(x-p-26)
        \end{cases}
        \]
        可得$L_1, L_2$的交点为$(p+18,q+12)$。故三角形面积为$\dfrac{1}{2} \cdot 26 \cdot 12=156$。
    \end{solution}    
        
    \question 已知直线
    \[
    L_1: 3x + 3y = 2,\qquad
    L_2: 2x - 3y = 3,\qquad
    L_3: x - ay = -2
    \]
    将平面分成六个区域, 求$a$的所有可能值。
    \begin{solution}
        情况一:$L_1 \parallels L_3$,有 $a=-1$。

        情况二:$L_2 \parallels L_3$,有 $a=\dfrac{3}{2}$。

        情况三:$L_1,L_2$与$L_3$有唯一公共点,解$L_1,L_2$得 $\left(1,-\dfrac{1}{3}\right)$,故$a=-9$。

        经检验,$a$的所有可能值为 
        \[
        a = -1, \frac{3}{2}, -9
        \]
    \end{solution} 

    \question 若三直线
    \[
    L_1: 2x-y-3=0, \quad L_2: 2x+ay+3=0, \quad L_3: ax-y-6=0,
    \]
    不能围成一三角形, 求实数$a$之值。
    \begin{solution}
        情况一:$L_1 \parallels L_2$,有 $a=-1$。

        情况二:$L_1 \parallels L_3$,有 $a=2$。

        情况三:$L_2 \parallels L_3$,有 $a^2=-2$,此时无实数解。

        情况四:$L_1,L_2$与$L_3$有唯一公共点,联立$L_1,L_3$得
        \[
        x=\frac{3}{a-2},\quad y=\frac{12-3a}{a-2}, \quad a \neq 2
        \]
        代入$L_2$解得
        \[
        -3a(a-5)=0 \Rightarrow a=0 \ \text{或} \ a=5
        \]
        经检验,实数$a$之可能值为
        \[
        a=-1,0,2,5
        \]
    \end{solution}

    \question 三直线
    \[
    L_1: x - y + 2 = 0, \quad L_2: 2x + 3y + 9 = 0, \quad L_3: 8x + 3y - 27 = 0,
    \]
    围成三角形 \(\triangle ABC\),若点 \(P = (3,k)\) 在三角形内部,求 \(k\) 的范围。
    \begin{solution}
        欲使\(P = (3,k)\) 在三角形内,需满足\[
        \begin{cases}
            3-k+2>0 \\ 2(3)+3k+9>0\\8x+3y-27<0
        \end{cases}\quad \Rightarrow \quad 
        \begin{cases}
            k<5 \\ k>-5\\k<1 
        \end{cases}\quad \Rightarrow \quad 
        -5<k<1\]
    \end{solution}

    \question 已知点 $P(2,1)$,$Q(5,5)$,点 $R(x, 3x)$。若 $\triangle PQR$ 分别以 $P$ 或 $Q$ 为直角顶点,求点 $R$ 的坐标。
\begin{solution}
情况一:若 $\angle RPQ = 90^\circ$,则 $k_{PR} \cdot k_{PQ} = -1$。
已知 $k_{PQ} = \frac{5-1}{5-2} = \frac{4}{3}$,则 $k_{PR} = -\frac{3}{4}$:
\[ \frac{3x - 1}{x - 2} = -\frac{3}{4} \implies 4(3x - 1) = -3(x - 2) \]
\[ 12x - 4 = -3x + 6 \implies 15x = 10 \implies x = \frac{2}{3} \]
此时 $y = 3(\frac{2}{3}) = 2$,点 $R$ 坐标为 $(\frac{2}{3}, 2)$。

情况二:若 $\angle PQR = 90^\circ$,则 $k_{QR} \cdot k_{PQ} = -1$。
则 $k_{QR} = -\frac{3}{4}$:
\[ \frac{3x - 5}{x - 5} = -\frac{3}{4} \implies 4(3x - 5) = -3(x - 5) \]
\[ 12x - 20 = -3x + 15 \implies 15x = 35 \implies x = \frac{7}{3} \]
此时 $y = 3(\frac{7}{3}) = 7$,点 $R$ 坐标为 $(\frac{7}{3}, 7)$。
综上所述,所有可能的点 $R$ 坐标为 $(\frac{2}{3}, 2)$ 或 $(\frac{7}{3}, 7)$。
\end{solution}
    
    \question 设 \(x,y\) 为二元一次联立不等式图形上的任一点,已知
    \[
    \begin{cases}
    x + y \geq -1 \\
    x - y \geq -3 \\
    4x - y \leq 6
    \end{cases}
    \]
    函数 \(P = x + k y\) 在点 \((1,-2)\) 取得唯一最小值,求实数 \(k\) 的范围。
    \begin{solution}
        由作图得知求出可行解之顶点分别为 $(1,-2),(3,6),(-2,1)$, 据题意有
        \[
        \begin{cases}
            P(1,-2)<P(-2,1)\\P(1,-2)<P(3,6)
        \end{cases}\quad \Rightarrow \quad
        \begin{cases}
            1-2k<-2+k\\1-2k<3+6k
        \end{cases}\quad \Rightarrow \quad
            k>1
        \]
    \end{solution}
    
    \question 在坐标平面上,求二元一次联立不等式
    \[
    |x - 2y| \leq 2, \quad |x + 2y| \leq 2,
    \]
    的解所成区域面积。
    \begin{solution}
        由作图得知该区域为一长为$4$,宽为$2$的菱形,故面积为$8$。
    \end{solution}

    \question 已知 $k>0$,直线 $y = 3kx + 4k^2$ 与抛物线 $y = x^2$ 相交于两点 $P$ 和 $Q$。若 $O$ 是原点,且 $\triangle OPQ$ 的面积为 $80$,求该直线的斜率。
    \ifprintanswers
    \begin{figure}[H]
        \centering
        \includegraphics[width=0.5\linewidth]{images/image172.png}
    \end{figure}
    \fi
    \begin{solution}
        联立直线 $y = 3kx + 4k^2$ 与抛物线 $y = x^2$,解得
        \[
        P=(-k,k^2),Q=(4k,16k^2)
        \]
        设 $S,T$ 为 $P,Q$ 在$x$ 轴的垂足,则 
        \[
        SP=k^2,TQ=16k^2,ST=4k-(-k)=5k
        \]
        且由
        \begin{align*}
        80=[\triangle OPQ]
        &=[\text{梯形} PSTQ]-[\triangle PSO]-[\triangle QTO] \\
        &=\frac{1}{2}(k^2+16k^2)\cdot 5k-\frac{1}{2} \cdot k^2 \cdot k-\frac{1}{2} \cdot 16k^2 \cdot 4k=10k^3
        \end{align*}
        可得 $k = 2$,因此直线斜率为 $3k = 6$。
    \end{solution}

    \question 设 $P_1, P_2, P_3$ 为抛物线 $y=x^2$ 上的三点,$l_1, l_2, l_3$ 分别为抛物线在这些点处的切线。切线两两相交,设 $Q_{12}=l_1 \cap l_2, Q_{13}=l_1 \cap l_3, Q_{23}=l_2 \cap l_3$。求 $\triangle P_1P_2P_3$ 与 $\triangle Q_{12}Q_{13}Q_{23}$ 面积之比。
    \begin{solution}
        设 $P_1=(a,a^2), P_2=(b,b^2), P_3=(c,c^2)$ 且 $a \neq b \neq c$,切线方程为
        \[
        l_1: y = 2ax - a^2, \quad l_2: y = 2bx - b^2, \quad l_3: y = 2cx - c^2。
        \]
        可解得
        \[
        Q_{12} = \left(\frac{a+b}{2}, ab\right), \quad Q_{13} = \left(\frac{a+c}{2}, ac\right), \quad Q_{23} = \left(\frac{b+c}{2}, bc\right)。
        \]
        于是
        \[
        [\triangle P_1P_2P_3] = \frac{1}{2}
        \left|\begin{vmatrix}
        1 & 1 & 1 \\
        a & b & c \\
        a^2 & b^2 & c^2
        \end{vmatrix}\right|
        =\frac{1}{2}|(a-b)(b-c)(c-a)|
        \]
        \[
        [\triangle Q_{12}Q_{13}Q_{23}] = \frac{1}{2} \cdot \frac{1}{2}
        \left|\begin{vmatrix}
        1 & 1 & 1 \\
        a+b & a+c & b+c \\
        ab & ac & bc
        \end{vmatrix}\right|
        =\frac{1}{4}|(a-b)(b-c)(c-a)|
        \]
        故
        \[
        \frac{[\triangle P_1P_2P_3]}{[\triangle Q_{12}Q_{13}Q_{23}]} = 2
        \]
    \end{solution}

    \question 已知抛物线 $P_1: f(x) = x^2 + bx + c$ 的顶点为 $P$,抛物线 $P_2: g(x) = -x^2 + dx + e$ 的顶点为 $Q$。$P,Q$ 不重合,且都在$P_1,P_2$上。
    \begin{parts}
    \part 证明 $bd=2(e - c)$。
    \begin{solution}
        抛物线$P_1$顶点 $P$ 在 $x = -\dfrac{1}{2}b$上。由于 $P$ 在两条抛物线上,
        \[
        \frac{1}{4}b^2 + b\Bigl(-\frac{1}{2}b\Bigr) + c = -\frac{1}{4}b^2 + d\Bigl(-\frac{1}{2}b\Bigr) + e 
        \]
        化简得
        \[
        bd = 2(e - c)
        \]
        因此得证。
    \end{solution}
    \part 证明经过 $P$ 与 $Q$ 的直线斜率为 $\dfrac{1}{2}(b+d),y$ 截距为 $\dfrac{1}{2}(c+e)$。
    \begin{solution}
        顶点 $P,Q$ 坐标为 
        \[
        P\left(-\frac{1}{2}b, -\frac{1}{4}b^2 + c\right),Q\left(-\frac{1}{2}d, -\frac{1}{4}d^2 + e\right)
        \]  
        直线 $PQ$ 的斜率为
        \[
        \frac{(-\frac{1}{4}b^2 + c) - (-\frac{1}{4}d^2 + e)}{-\frac{1}{2}b - (-\frac{1}{2}d)} 
        = \frac{1}{2}(b + d)
        \]
        故直线方程为
        \[
        y = \frac{1}{2}(b+d)x + \frac{1}{2}(c+e)
        \]
        因此斜率为 $\dfrac{1}{2}(b+d),y$ 截距为 $\dfrac{1}{2}(c+e)$。
    \end{solution}
    \begin{solution}
        两抛物线方程相加得到 $2y = (b+d)x + (c+e)$,即
        \[
        y = \frac{1}{2}(b+d)x + \frac{1}{2}(c+e)。
        \]
        因此 $P$ 与 $Q$ 在同一条直线上,直线斜率为 $\dfrac{1}{2}(b+d),y$ 截距为 $\dfrac{1}{2}(c+e)$。
    \end{solution}
    \end{parts}

    \question 已知 $a > \dfrac{1}{2}$,抛物线方程为 $y = ax^2 + 2$,顶点为 $V$。抛物线与直线 $y = -x + 4a$ 相交于 $B$ 和 $C$ 两点,若 $\triangle VBC$ 的面积为 $\dfrac{72}{5}$,求 $a$。
    \ifprintanswers
    \begin{figure}[H]
        \centering
        \includegraphics[width=0.5\linewidth]{images/image187.png}
    \end{figure}
    \fi
    \begin{solution}
        抛物线 $y = ax^2 + 2$ 的顶点在 $x=0$上,因此 $V=(0,2)$。联立抛物线与直线解得
        \[
        x = -2 \text{ 或 } \ x = \frac{2a-1}{a}.
        \]
        因此
        \[
        B(-2,4a+2),\quad C\left(2-\frac{1}{a},4a-2 + \frac{1}{a}\right)
        \]
        设 $P,Q$ 在通过 $V$ 的水平线上,使得 $BP,CQ$ 垂直于 $PQ$,则
        \[
        P(-2,2),Q\left(2-\dfrac{1}{a},2\right),
        \]
        且
        \[
        BP = 4a, \quad CQ = 4a -4 + \frac{1}{a}, \quad PQ = 4 - \frac{1}{a}
        \]
        因此 $\triangle VBC$ 面积为
        \begin{align*}
        \frac{72}{5}=[\triangle VBC]
        &=[\text{梯形} PBCQ]-[\triangle BPV]-[\triangle CQV] \\
        &=\frac{1}{2}\left(4a+4a -4 + \frac{1}{a}\right)\left(4 - \frac{1}{a}\right)-
        \frac{1}{2}\cdot 4a \cdot 2 - \frac{1}{2}\left(4a -4 + \frac{1}{a}\right)\left(2-\frac{1}{a}\right)
        \end{align*}
        解方程可得
        \[
        a = \frac{5}{2} > \frac{1}{2}
        \]
    \end{solution}

    \question 一个等边三角形位于第一象限,顶点为 $(0,0),(x_1,4),(x_2,11)$,求序对 $(x_1,x_2)$。
    \begin{solution}
        构造复平面,复数$x_2+11i$逆时针旋转 $60^\circ$后变为复数$x_1+4i$,即
        \[
        x_2+11i=\left(\frac12+\frac{\sqrt3}{2}i\right)(x_1+4i)
        \]
        比较实部与虚部得
        \[
        x_2=\frac12 x_1-2\sqrt3,\quad 11=\frac{\sqrt3}{2}x_1+2
        \]
        解得
        \[
        (x_1,x_2)=(6\sqrt3,\sqrt3)
        \]
    \end{solution}

    \question 已知正方形 $WXYZ$ 的对角线 $WY$ 的斜率为 $2$,求直线 $WX$ 及 $XY$ 的斜率之和。
    \ifprintanswers
    \begin{figure}[H]
        \centering
        \includegraphics[width=0.35\linewidth]{images/image184.png}
    \end{figure}
    \fi
    \begin{solution}
        设 $WY$ 与水平线成夹角 $\theta$, 有$\tan\theta = 2$,因此
        \[
        m_{WX}+m_{XY}
        =m_{WX}+m_{WZ}
        =\tan(\theta+45^\circ)+\tan(\theta-45^\circ)
        =\frac{2+1}{1-2\cdot1}+\frac{2-1}{1+2\cdot1}
        =-\frac{8}{3}
        \]
    \end{solution}
    \ifprintanswers
    \begin{figure}[H]
        \centering
        \includegraphics[width=0.35\linewidth]{images/image185.png}
    \end{figure}
    \fi
    \begin{solution}
        将正方形平移使得 $W$为原点。设 $Y(2a,2b)$,有
        \[
        m_{WY}=\frac{2b-0}{2a-0}=2 \Rightarrow b=2a \Rightarrow Y(2a,4a)
        \]
        对角线 $WY$ 的中点为 $C(a,2a)$,由于$WC \perp CX$,
        \[
        m_{WC}=2 \Rightarrow m_{XC}=-\frac{1}{2}
        \]
        于是由图可得 $X(-a,3a)$,因此
        \[
        m_{WX}=\frac{3a-0}{-a-0}=-3.
        \]
        又因 $XY\perp WX$,所以 $m_{XY}=\dfrac{1}{3}$,斜率之和为
        \[
        -3+\frac{1}{3}=-\frac{8}{3}
        \]
    \end{solution}

    \question 矩形 $ABCD$ 的顶点为 $A(0,0),B(0,12),C(6,12),D(6,0)$。一直线经过点 $U(0,u),V(2,4),W(6,w)$,且将矩形 $ABCD$ 分成两个梯形,使得该两个梯形面积之比为 $5:3$,求 $U,W$。
    \ifprintanswers
    \begin{figure}[H]
        \centering
        \includegraphics[width=0.7\linewidth]{images/image161.png}
    \end{figure}
    \fi
    \begin{solution}
        矩形 $ABCD$ 面积为 $6 \cdot 12 = 72$,因此两个梯形面积分别为
        \[
        \frac{5}{8}\cdot 72 = 45,\quad \frac{3}{8}\cdot 72 = 27.
        \]
        \textbf{情况 1:直线交于 $AB$ 与 $CD$。}  
        直线 $UV$ 与 $VW$ 共线,有
        \[
        \frac{4-u}{2-0} = \frac{w-4}{6-2} \Rightarrow w = 12 - 2u.
        \]
        其中 $0<u,w<12$,梯形 $ADWU$ 的面积为
        \[
        \frac{1}{2}\cdot 6 \cdot (w+u) = 3(w+u)=3(12-u)
        \]
        若 $3(12-u)=27$,解得 $U(0,3), W(6,6)$; 
        若 $3(12-u)=45$,解得 $u=-3$,不合题意。

        \textbf{情况 2:直线交于 $AD$ 与 $BC$}  
        设交点为 $E(e,0), F(f,12)$,此时
        \[
        \frac{4}{2-e} = \frac{8}{f-2} \Rightarrow f = 6 - 2e.
        \]
        梯形 $BFEA$ 的面积为
        \[
        \frac{1}{2}\cdot 12 \cdot (f+e) = 6(f+e) = 6(6-e)
        \]
        若 $6(6-e)=27$,得 $e=\dfrac{3}{2}, f=3$,此时直线斜率为 $8$,由 $V(2,4)$ 得 $U(0,-12), W(6,36)$;  
        若 $6(6-e)=45$,得 $e=-\dfrac{3}{2}$ 不合题意。

        故解为
        \[
        U(0,3), W(6,6) \quad \text{或} \quad U(0,-12), W(6,36).
        \]
    \end{solution}

    \question 已知点 $J(2,7),K(5,3),L(r,t)$ 构成一个三角形,且其面积不超过 $10$。令 $\mathcal{R}$ 为所有满足条件的点 $L$ 构成的区域,且 $0 \le r,t \le 10$。求 $\mathcal{R}$ 的面积。
    \ifprintanswers
    \begin{figure}[H]
        \centering
        \includegraphics[width=0.5\linewidth]{images/image169.png}
    \end{figure}
    \fi
    \begin{solution}
        线段 $JK$ 的长度为
        \[
        JK = \sqrt{(2-5)^2 + (7-3)^2} = 5
        \]  
        以 $JK$ 为底,则 $\triangle JKL$ 的高度 $h$ 满足
        \[
        \frac{1}{2} \cdot 5 \cdot h \le 10 \Rightarrow h \le 4
        \] 
        直线 $JK$ 的方程为:
        \[
        \frac{y-7}{x-2} = \frac{3-7}{5-2}  \Rightarrow 4x+3y=29
        \]  
        设
        \[
        f(x,y)=\frac{|4x+3y-29|}{5},
        \]
        解$f(0,y)=4,f(x,0)=4,f(10,y)=4,f(x,10)=4$可知与直线$JK$距离$4$单位的两条平行线与正方形交于
        \[
        (0,3),\left(\dfrac{9}{4},0\right),(10,3),\left(\dfrac{19}{4},10\right),
        \]
        故$\mathcal{R}$面积为
        \[
        100 - \frac{1}{2}\cdot 3 \cdot \frac{9}{4} - \frac{1}{2}\cdot \frac{21}{4} \cdot 7 = \frac{313}{4}
        \]
    \end{solution}

    \question 已知坐标平面上有一正方形$ABCD$,点 $$E(6,0),F(6,6),G(0,8),H(-5,4)$$ 分别在 ${AB},{BC},{CD},{DA}$ 上,求正方形 $ABCD$ 的面积。
    \begin{solution}
        设直线
        \[
        AB: y = m(x - 6), \quad CD: y = m x + 8.
        \]
        且
        \[
        {BC}: y = -\frac{1}{m}(x - 6) + 6, \quad {AD}: y = -\frac{1}{m}(x + 5) + 4.
        \]
        由$d({AB}, {CD}) = d({BC}, {AD}),$
        \[
        \frac{|8 + 6m|}{\sqrt{m^2 + 1}} = \frac{|2m + 11|}{\sqrt{m^2 + 1}} 
        \]
        解得
        \[
        m = \frac{3}{4} \text{ 或 }\ m = -\frac{19}{8}.
        \]
        经检验,$\ m = -\dfrac{19}{8}$ 时点 $E$ 不在 ${AB}$ 上,舍去;因此正方形面积为$10^2=100$
    \end{solution}

    \question 已知函数 $f(x)=x^{2}-\sqrt{2}x$ 与 $g(x)=-x^{2}-1$ 有两条公切线,求四个切点组成的四边形周长。
    \begin{solution}
        设在 $f(x)=x^{2}-\sqrt{2}x$ 上的切点为 $A(a,a^2-\sqrt{2}a)$,切线斜率为 $f'(a)=2a-\sqrt{2}$,切线方程为
        \[
        y=(2a-\sqrt{2})(x-a)+a^2-\sqrt{2}a.
        \]
        与 $g(x)=-x^{2}-1$ 联立得
        \[
        x^2+(2a-\sqrt{2})x - a^2 +1 =0
        \]
        其中判别式为 $0$,
        \[
        (2a-\sqrt{2})^2 -4(1-a^2)=0 \Rightarrow a=\frac{\sqrt{2}\pm \sqrt{6}}{4}
        \]
        故$f(x)$上的切点为
        \[
        A\left(\frac{\sqrt{2}+\sqrt{6}}{4},-\frac{\sqrt{3}}{4}\right),\quad A'\left(\frac{\sqrt{2}-\sqrt{6}}{4},\frac{\sqrt{3}}{4}\right),
        \]
        $g(x)$ 上的切点为
        \[
        B\left(\frac{\sqrt{2}-\sqrt{6}}{4},-\frac{\sqrt{3}}{4}\right),\quad B'\left(\frac{\sqrt{2}+\sqrt{6}}{4},\frac{\sqrt{3}}{4}\right).
        \]
        因为$AA' = BB' = AB' = A'B = \dfrac{3}{2}$,周长为
        \[
        4\cdot \frac{3}{2} = 6
        \]
    \end{solution}

    \question 已知 $\triangle ABC$ 的外心 $O(-1,2)$,内心 $I(2,2)$,顶点 $A(2,8)$,求直线 $BC$ 的方程式。
    \begin{solution}
        由$O(-1,2), I(2,2), A(2,8)$,可得$\triangle ABC$ 外接圆 $\Gamma_1$ 方程式为
        \[
        (x+1)^2 + (y-2)^2 = 3^2+6^2 = 45
        \]
        过$I(2,2), A(2,8)$的直线方程式为 $x=2$,交 $\Gamma_1$于
        \[
        D = (2, -4)
        \]
        以 $D$ 为圆心,$DI = 6$ 为半径的圆 $\Gamma_2$ 方程式为
        \[
        (x-2)^2 + (y+4)^2 = 36.
        \]
        联立 $\Gamma_1$ 与 $\Gamma_2$,由\href{http://w.mathsgreat.com/SC/geom_th/geom_th_123.pdf}{鸡爪定理},得$BC$ 的方程式
        \[
        x - 2y = 4
        \]
    \end{solution}

    \question 设$\triangle ABC$中,已知$BC$与$x$轴平行,且$A(1,8)$,内切圆圆心为$(0,0)$,半径$4$,求$\triangle ABC$垂心$H$的坐标。
    \ifprintanswers
    \begin{figure}[H]
        \centering
        \includegraphics[width=0.5\linewidth]{images/image62.png}
    \end{figure}
    \fi
    \begin{solution}
        设过$A(1,8)$的圆的切线切点为$P(a,\sqrt{16 - a^2})$,  
        则  
        \[
        m_{PA}\cdot m_{OP}=-1 \Rightarrow \frac{\sqrt{16-a^2}}{a}=\frac{\sqrt{16-a^2}-8}{a-1} =-1 
        \]
        解得切点为
        \[
        a = \frac{48}{13}, -\frac{16}{5} \Rightarrow P\left(-\frac{16}{5}, \frac{12}{5}\right), Q\left(\frac{48}{13}, \frac{20}{13}\right)
        \]
        对圆方程求导知切线斜率$y' = -\dfrac{x}{y}$,则
        \[
        m_P = \frac{4}{3},\quad m_Q = -\frac{12}{5}
        \]
        故切线方程分别为  
        \[
        L_1: 4x - 3y + 20 = 0, \quad L_2: 12x + 5y = 52
        \]
        又 $B = L_1 \cap (y = -4) = (-8, -4), C = L_2 \cap (y = -4) = (6, -4)$,且过$A$的垂线为  
        \[
        L_3: x = 1
        \]
        过$B(-8, -4)$且与$L_2$垂直的直线方程为  
        \[
        L_4: y+4=\frac{5}{12} (x+8)\Rightarrow 5x - 12y = 8 
        \]
        联立 $L_3, L_4$得
        \[
        H\left(1, -\frac{1}{4}\right)
        \]
        \end{solution}

    \question 已知点 $A(3,1),B\left(\dfrac{5}{3},2\right)$ 是平行四边形 $ABCD$ 的顶点,且 $ABCD$ 四个顶点都落在函数 $$f(x)=\log_2 \dfrac{ax-b}{x-1}$$ 的图像上,求平行四边形 $ABCD$ 的面积。
    \begin{solution}
        由
        \[
        f(3) = \log_2 \frac{3a - b}{2} = 1, f\left(\frac{5}{3}\right) \log_2 \frac{5a/3 - b}{2/3} = 2
        \]
        解得
        \[
        a = 1, \; b = -1, f(x) = \log_2 \frac{x+1}{x-1}
        \]
        注意到
        \[
        f(-x) = \log_2 \frac{-x+1}{-x-1} = \log_2 \frac{x-1}{x+1} = -\log_2 \frac{x+1}{x-1} = -f(x),
        \]
        所以$f(x)$关于原点对称,平行四边形的另两个顶点为,
        \[
        C(-3,-1), \quad D\left(-\frac{5}{3},-2\right).
        \]
        故平行四边形 $ABCD$ 面积为
        \[
        \frac{1}{2}\left|6-\frac{5}{3}+6-\frac{5}{3}-\left(\frac{5}{3}-6+\frac{5}{3}-6\right) \right|=\frac{26}{3}
        \]
    \end{solution}

    \question 若方程 $$|x^2 - 4x + 3| - a = x$$ 恰有 $4$ 个实根,求实数 $a$ 的范围。
    \begin{solution}
        考虑
        \[
        \begin{cases}
        \Gamma: y = |x^2 - 4x + 3| \\
        L: y = x + a
        \end{cases}
        \]
        可知 $\Gamma$ 与 $x$ 轴交于 $A(1, 0),B(3, 0)$,过 $A$ 且与 $L$ 平行的直线为
        \[
        L_1: y = x - 1
        \]
        当 $1 \le x \le 3$ 时,有
        \[
        \Gamma: y = -x^2 + 4x - 3
        \Rightarrow y' = -2x + 4 = 1 \Rightarrow x = \frac{3}{2}
        \Rightarrow \text{切点 } P\left(\frac{3}{2}, \frac{3}{4}\right)
        \]
        过 $P$ 且与 $L$ 平行的直线为
        \[
        L_2: y = x - \frac{3}{4}
        \]
        因此当 $L$ 在 $L_1$ 与 $L_2$ 之间(不含切点)时,即当
        \[
        -1 < a < -\frac{3}{4}
        \]
        有恰好 $4$ 个交点。
    \end{solution}

    \question 如图,一抛物线顶点为 $D$,与 $x$ 轴交于 $A,C(4,0)$,与 $y$ 轴交于 $B(0,-4)$。若 $\triangle ABC$ 之面积为 $4$,试求 $\triangle DBC$ 之面积。
    \begin{figure}[H]
    \centering
    \includegraphics[width=0.45\textwidth]{images/image38.png}
    \end{figure}
    \begin{solution}
        假设 $O$ 为原点,则
        \[
        [\triangle ABC] = \frac{1}{2} \cdot OB \cdot AC = 2 \cdot AC = 4 \Rightarrow AC = 2, 
        \]
        故$A(2,0)$,于是$2,4$ 为抛物线 $y = f(x)$ 的两根,设
        \[
        f(x) = k(x - 2)(x - 4)
        \]
        可得
        \[
        f(0) = 8k = -4 \Rightarrow y = f(x) = -\frac{1}{2}(x - 2)(x - 4)
        \]
        于是
        \[
        f(3) = \frac{1}{2} \Rightarrow D\left(3, \frac{1}{2}\right)
        \]
        故
        \[
        [\triangle DBC] = \frac{1}{2} 
        \left|\begin{vmatrix} 
            3 & \frac{1}{2} & 1 \\ 0 & -4 & 1 \\ 4 & 0 & 1 
        \end{vmatrix}\right| = 3
        \]
    \end{solution}

    \question 已知$\triangle ABC$ 为直角三角形,其中 $\angle C=90^{\circ},AB=36$,而 $BC$ 上的中线为:$x+2y=0,AC$ 上的中线为:$x+y=0$,求 $\triangle ABC$ 的面积。
    \begin{solution}
        由 $A$ 在 $L_1: x+2y=0$ 上,$B$ 在 $L_2: x+y=0$ 上,设 
        \[
        A = (-2a, a),B = (b, -b),
        \]
        又$L_1,L_2$交于重心 $G(0,0)$,设 $C = (2a-b, -a+b)$,于是
        \[
        m_{CA}\cdot m_{CB}=\frac{-a+b-a}{2a-b+2a} \cdot \frac{-a+b+b}{2a-b-b}=-1 \Rightarrow 10a^2 - 15ab + 4b^2 = 0 \tag{1}
        \]
        由$AB = 36$,
        \[
        (-2a - b)^2 + (a + b)^2 = 5a^2 + 6ab + 2b^2 = 36^2 \tag{2}
        \]
        由 $(1),(2)$ 解得
        \[
        ab = \frac{2 \cdot 36^2}{27}
        \]
        故$\triangle ABC$ 的面积为
        \[
        [\triangle ABC] = \frac{1}{2} 
        \left|\begin{vmatrix} 
            -2a & a & 1 \\ b & -b & 1 \\ 2a - b & -a + b & 1 
        \end{vmatrix}\right|
        = \frac{3}{2} \cdot \frac{2 \cdot 36^2}{27} = 144
        \]
    \end{solution}

    \question
    \begin{parts}
    \part 已知两条抛物线 $y=x^{2}-8x+17$ 与 $y=-x^{2}+4x+7$。
    \begin{subparts}
    \subpart 求两条抛物线的顶点 $V_1$ 与 $V_2$ 的坐标。
    \begin{solution}
        完全平方得
        \[
        y=x^{2}-8x+17=(x-4)^{2}+1,
        \]
        \[
        y=-x^{2}+4x+7=-(x-2)^{2}+11.
        \]
        所以顶点为 $V_1(4,1), V_2(2,11)$。
    \end{solution}
    \subpart 若两条抛物线相交于 $P,Q$,证明四边形 $V_1PV_2Q$ 是平行四边形。
    \begin{solution}
        联立两抛物线得
        \[
        x^{2}-8x+17=-x^{2}+4x+7 \Rightarrow (x-5)(x-1)=0.
        \]
        得 $P(5,2),Q(1,10)$。发现$V_1V_2,PQ$ 的中点分别为 
        \[
        \left(\frac{4+2}{2},\frac{1+11}{2}\right)=(3,6), \quad
        \left(\frac{5+1}{2},\frac{2+10}{2}\right)=(3,6)
        \]
        即两条对角线互相平分,故得证 $V_1PV_2Q$ 是平行四边形。
    \end{solution}
    \end{subparts}
    \part 已知两条抛物线 $y=-x^{2}+bx+c$ 与 $y=x^{2}$,顶点分别为 $V_3,V_4$。当 $b,c$ 取某些值时,两条抛物线相交于相异点 $R,S$。
    \begin{subparts}
    \subpart 求$(b,c)$的解集,使得 $R,S$ 存在且 $V_3,V_4,R,S$ 为相异的四点。
    \begin{solution}
        完全平方得
        \[
        y=-\left(x-\frac{b}{2}\right)^{2}+\frac{b^{2}}{4}+c,
        \]
        顶点为 $V_3\left(\dfrac{b}{2},\dfrac{b^{2}}{4}+c\right)$,另一抛物线顶点为 $V_4(0,0)$,联立两抛物线得
        \[
        -x^{2}+bx+c=x^{2} \Rightarrow 2x^{2}-bx-c=0
        \]
        其中判别式需满足 $b^{2}+8c>0$,即 $c>-\dfrac{b^{2}}{8}$,且交点横坐标为
        \[
        x=\frac{b\pm \sqrt{b^{2}+8c}}{4}.
        \]
        需有 $c\ne0$,否则$R,S$与 $V_3$ 或 $V_4$ 重合;若 $b=0$,则 $V_3=(0,c)$,与 $V_4$ 重合需满足 $c=0$,所以解集为 
        \[
        \left\{\left.(b,c)\right|c>-\frac{b^{2}}{8}\ \text{且} \  c\ne0\right\}.
        \]
    \end{solution}
    \subpart 求使得 $R,S$ 存在且 $V_3,V_4,R,S$ 为相异的四点,且四边形 $V_3RV_4S$ 为矩形的所有 $(b,c)$。
    \begin{solution}
        已知 $R(x_1,x_1^{2}),S(x_2,x_2^{2})$,其中 $x_1,x_2$ 为方程 $2x^{2}-bx-c=0$ 的根,由韦达定理,
        \[
        x_1+x_2=\frac{b}{2}, \quad x_1x_2=-\frac{c}{2}
        \]  
        发现$V_3V_4,RS$ 的中点分别为 
        \[
        \left(\frac{b}{4},\frac{b^{2}+4c}{8}\right), \quad \left(\frac{x_1+x_2}{2},\frac{x_1^{2}+x_2^{2}}{2}\right)
        =\left(\frac{b}{4},\frac{b^{2}+4c}{8}\right)
        \]
        两条对角线中点相同,所以 $V_3RV_4S$ 是平行四边形,再要求一对相邻边垂直:  
        \[
        m_{V_4R}\cdot m_{V_4S} = x_1x_2=-1 =-\frac{c}{2} \Rightarrow c=2
        \]  
        结合 (i) 的条件,$c=2$ 时满足 $c>-\dfrac{b^{2}}{8}$ 且 $c\ne0$,因此所有解为 $(b,2)$,其中 $b \in \mathbb{R}$。
    \end{solution}
    \end{subparts}
    \end{parts}

    \question 坐标平面上,正六边形 $OABCDE$ 的顶点依逆时针排列为 $O$(原点)$,A,B,C,D,E$,其中直线 $OA$ 的方程为 $y=3x$,直线 $BE$ 的方程为 $y=3x+2$。试求此正六边形的外接圆圆心坐标。
        \ifprintanswers
        \begin{figure}[H]
        \centering
        \includegraphics[width=0.45\textwidth]{images/image24.png}
        \end{figure}
        \fi
    \begin{solution}
        两条平行线间的距离为
        \[
        {AP} = \frac{2}{\sqrt{10}}
        \]
        在$\triangle APB$中,
        \[
        {OA} = {AB} = \frac{\frac{2}{\sqrt{10}}}{\sin 60^\circ} = \frac{4}{\sqrt{30}}.
        \]
        设$A(a, 3a)$ 在 $y = 3x$ 上,则
        \[
        {OA}^2 = a^2 + (3a)^2 = 10a^2 = \left( \frac{4}{\sqrt{30}} \right)^2 = \frac{16}{30} \Rightarrow a = \frac{2}{5\sqrt{3}}\Rightarrow A\left( \frac{2}{5\sqrt{3}}, \frac{6}{5\sqrt{3}} \right)
        \]
        将 $A$ 逆时针旋转 $60^\circ$,得圆心$Q$
        \[
        \begin{bmatrix}
        \frac{1}{2} & -\frac{\sqrt{3}}{2} \\
        \frac{\sqrt{3}}{2} & \frac{1}{2}
        \end{bmatrix}
        \begin{bmatrix}
        \frac{2}{5\sqrt{3}} \\
        \frac{6}{5\sqrt{3}}
        \end{bmatrix}
        =
        \begin{bmatrix}
        \frac{\sqrt{3} - 9}{15} \\
        \frac{\sqrt{3} + 1}{5}
        \end{bmatrix}
        \Rightarrow Q\left( \frac{\sqrt{3} - 9}{15}, \frac{\sqrt{3} + 1}{5} \right).
        \]
    \end{solution}

    \question 已知函数 $y=f(x)$ 的图像既关于点 $(1,1)$ 中心对称, 又关于直线 $x+y=0$ 轴对称. 若 $x \in (0,1)$ 时, $f(x)=\log_{2}(x+1)$, 求 $f (\log_{2} 10)$ 的值。
    \begin{solution}
        记函数 $y=f(x)$ 的图像为$\Gamma$,
        对所有 $x_{0} \in (0,1)$, 令 $y_{0}=\log_{2}(1+x_{0})$, 则 $$(x_{0},y_{0}) \in \Gamma,y_{0} \in (0,1)$$
        利用 $\Gamma$ 的中心对称性与轴对称性, 可依次推得
        $$(2-x_{0}, 2-y_{0}) \in \Gamma, (y_{0}-2, x_{0}-2) \in \Gamma, (4-y_{0}, 4-x_{0}) \in \Gamma$$
        解$4-y_{0}=4-\log_{2}(1+x_{0})=\log_{2}10$得 $x_{0}=\dfrac{3}{5}$,因此 $$f(\log_{2}10)=f(4-y_{0})=4-x_{0}=4-\frac{3}{5}=\frac{17}{5}$$
    \end{solution}

    \question 若过原点可对 $y=x^{3}+ax^{2}+x+1$ 函数图形作三条切线,求 $a$ 的取值范围。
    \begin{solution}
        令切点 $P(t,t^3+at^2+t+1)$,则斜率为 $3t^2+2at+1$,  
        过 $P$ 之切线 
        \[
        L: y=(3t^2+2at+1)(x-t)+t^3+at^2+t+1 
        \]
        又 $L$ 过 $(0,0)$,故
        \[
        -3t^3-2at^2-t+t^3+at^2+t+1=0
        \]
        由$f(t)=2t^3+at^2-1=0$有三相异根,可得
        \[
        f'(t)=6t^2+2at=2t(3t+a)=0 \Rightarrow t=0,-\frac{a}{3} \text{有极值}
        \]
        由
        \[
        f(0)f\left(-\frac{a}{3}\right)= -1\left(-\frac{2a^3}{27}+\frac{a^3}{9}-1\right)< 0
        \]
        解得
        \[
        a > 3
        \]
    \end{solution}

    \question 在坐标平面上,已知$A(8,0),B(0,6)$, $P$ 为圆$x^{2}+y^{2}=16$上的动点,求$3PA + 2PB$的最小值。
    \begin{solution}
        设$P(x,y)$为圆$x^{2}+y^{2}=16$上的动点,$Q(0,a)$使得$PQ = \dfrac{2}{3} PB$,即
        \[
        x^{2} + (y - a)^{2} = \frac{4}{9}\left( x^{2} + (y - 6)^{2} \right)
        \]
        整理得
        \[
        5x^{2} + 5y^{2} - 18 a y + 48 y + 9 a^{2} - 144 = 0
        \]
        代入圆方程 $x^{2} + y^{2} = 16$,得
        \[
        9 a^{2} - 64 - 6 y (3 a - 8) = (3 a - 8)(3 a + 8 - 6 y) = 0
        \]
        解得
        \[
        a = \frac{8}{3} \Rightarrow Q \left( 0, \frac{8}{3} \right)
        \]
        因此
        \[
        3 PA + 2 PB = 3(PA + PQ) \ge 3 AQ = 3 \sqrt{8^{2} + \left(\frac{8}{3}\right)^{2}} =8 \sqrt{10}
        \]
    \end{solution}

    \question 在坐标平面上,将一个过原点且半径为 $r$ 的圆,完全放入 $y \ge x^{4}$ 的区域内, 此时 $r$ 的最大值为何?  
    \begin{solution}
        设圆的方程为 $x^2+(y-r)^2=r^2$,由 $y\ge x^4$ 得 $\sqrt{y}\ge x^2$,代入方程得
        \[
        \sqrt{y}+(y-r)^2\ge r^2,\quad (y>0).
        \]
        即
        \[
        2r \le y+\frac{1}{2\sqrt{y}}+\frac{1}{2\sqrt{y}}.
        \]
        由AM-GM不等式,
        \[
        y+\frac{1}{2\sqrt{y}}+\frac{1}{2\sqrt{y}}\ge \frac{3}{2}\sqrt[3]{2}
        \]
        所以
        \[
        r \le \frac{3}{4}\sqrt[3]{2}
        \]
    \end{solution}

    \question 坐标平面上,若四边形的四个顶点都在函数 $f(x)$ 上,则称此四边形为 $f(x)$ 的内接四边形。已知函数 $f(x) = x^3 + ax$ 的图形有唯一一个内接正方形,求 $a$ 的值。
        \ifprintanswers
        \begin{figure}[H]
        \centering
        \includegraphics[width=0.3\textwidth]{images/image21.png}
        \end{figure}
        \fi
    \begin{solution}
        由 $f(x) = x^3 + ax$ 可得:
        \[
        f''(x) = 6x = 0 
        \]
        即$f(x)$的对称中心为原点$(0,0)$,且为该内接正方形的中心,其对角线分别为
        \[
        L_1: y = mx, \quad L_2: y = -\frac{1}{m}x
        \]
        记 $A,B$ 在 $L_1,L_2$ 上,且都在$f(x)$上,则
        \[
        A: x^3 + ax = mx \Rightarrow x = 0 \text{ 或 } x^2 = m - a
        \]
        \[
        B: x^3 + ax = -\frac{1}{m}x \Rightarrow x = 0 \text{ 或 } x^2 = -\left(a + \frac{1}{m}\right)
        \]
        得
        \[
        A\left(\sqrt{m - a}, \; m\sqrt{m - a}\right), \quad B \left(\sqrt{-\left(a + \frac{1}{m}\right)},\; -\frac{1}{m} \sqrt{-\left(a + \frac{1}{m}\right)}\right)
        \]
        因 $B$ 为 $A$ 逆时针旋转 $90^\circ$ 得来,故
        \[
        \left(\sqrt{-\left(a + \frac{1}{m}\right)},\; -\frac{1}{m} \sqrt{-\left(a + \frac{1}{m}\right)}\right) = \left(-m\sqrt{m - a},\; \sqrt{m - a} \right)
        \]
        两边比较得:
        \[
        -m\sqrt{m - a} = \sqrt{-\left(a + \frac{1}{m}\right)}
        \Rightarrow m^2(m - a) = -\left(a + \frac{1}{m}\right)
        \]
        解得:
        \[
        a = \frac{m^3}{m^2 - 1} + \frac{1}{m(m^2 - 1)} = \frac{(m - \frac{1}{m})^2 + 2}{m - \frac{1}{m}}
        = m - \frac{1}{m} + \frac{2}{m - \frac{1}{m}}
        \]
        设 $t = m - \frac{1}{m}$,则:
        \[
        |a| = \left| t + \frac{2}{t} \right| \ge 2\sqrt{2}
        \Rightarrow a = -2\sqrt{2}
        \]
        等号成立当且仅当$f(x) = x^3 + ax$ 有唯一内接正方形。
    \end{solution}

    \question 设$P(0,a)$是$y$轴上异于原点的任意一点,过点$P$且平行于$x$轴的直线与曲线$y=\frac{1}{a}\ln x$交于点$Q$,曲线$y=\frac{1}{a}\ln x$在点$Q$处的切线交$y$轴于点$R$,则$\triangle PQR$的面积的最小值为\underline{\quad}.
    \begin{solution}
        $\frac{\sqrt{2e}}{2}$\textcolor{red}{(待解)}
    \end{solution}

    \question 已知曲线 $y=x^4+ax^3+ax^2+x+1$ 在点 $(0,1)$ 的切线,不只在 $(0,1)$ 与曲线相切,试求 $a$ 的值。
    \begin{solution}
        令 $f(x)=x^4+ax^3+ax^2+x+1$,则
        \[
        f'(x)=4x^3+3ax^2+2ax+1,\quad f'(0)=1.
        \]
        故过 $(0,1)$ 的切线斜率为 $1$,切线方程为 $L:\;y=x+1$。将 $L$ 代入曲线方程得
        \[
        x^4+ax^3+ax^2+x+1 = x+1
        \]
        即
        \[
        x^2(x^2+ax+a)=0
        \]
        题意要求切线不只在 $(0,1)$ 与曲线相切,因此 $x^2+ax+a=0$ 当中必须有重根,故判别式
        \[
        a^2-4a=a(a-4)=0
        \]
        当 $a=0$ 时,方程仅在 $x=0$ 有切点,不符题意;因此取
        \[
        a=4
        \]
    \end{solution}

    \question 设 $a\in\mathbb{R}$,若 $y=x^3-x$ 与 $y=x^2-a^2+a$ 有公切线,试求 $a$ 的范围。
    \begin{solution}
        令$\Gamma_1:f(x)=x^3-x, \Gamma_2:g(x)=x^2-a^2+a$,则
        \[
        f'(x)=3x^2-1,\quad g'(x)=2x.
        \]
        设公切线 $L=\overleftrightarrow{AB}$,其中$A\in\Gamma_1, B\in\Gamma_2$,记$A=(s,s^3-s), B=(t,t^2-a^2+a)$,公切线斜率为
        \[
        m_L=\frac{t^2-a^2+a-(s^3-s)}{t-s}=3s^2-1=2t
        \]
        消去$t=\dfrac{3s^2-1}{2}$,可得
        \[
        -9s^4+8s^3+6s^2-1=4a^2-4a
        \]
        欲存在实数 $s$ 使等式成立,必须有
        \[
        4a^2-4a \le \max_{s\in\mathbb{R}} h(s)
        \]
        求导得
        \[
        h'(s)=-36s^3+24s^2+12s=-12s(s-1)(3s+1),\quad h''(s)=-108s^2+48s+12
        \]
        由此临界点为 $s=0,1,-\dfrac{1}{3}$。由于$h''(0)> 0, h''(1)< 0, h''\left(-\dfrac{1}{3}\right)< 0,h(s)$的极大值为
        \[
        h(1)=4,\; h\left(-\frac{1}{3}\right)=-\frac{20}{27}
        \]
        因此 $h_{\max}=4$,故需
        \[
        4a^2-4a \le 4
        \]
        解不等式得
        \[
        \frac{1-\sqrt{5}}{2}\le a\le\frac{1+\sqrt{5}}{2}
        \]
    \end{solution}

    \question 在平面直角坐标系中,函数 $y=\dfrac{1}{|x|}$ 的图像为 $\Gamma$.设 $\Gamma$ 上的两点 $P,Q$ 满足: $P$ 在第一象限, $Q$ 在第二象限,且直线 $PQ$ 与 $\Gamma$ 位于第二象限的部分相切于点 $Q$.求 $|PQ|$ 的最小值.
    \begin{solution}
        当 $x<0$ 时, $y=-\dfrac{1}{x}$,导数为 
        \[
        y'=\frac{1}{x^{2}}
        \]
        设 $Q\left(-a,-\dfrac{1}{a}\right)$,其中 $a>0$,由条件知 $PQ$ 的斜率为 $y'|_{x=-a}=\dfrac{1}{a^{2}}$,故直线 $PQ$ 的方程为 
        \[
        y=\frac{1}{a^{2}}(x+a)+\frac{1}{a}=\frac{x+2a}{a^{2}}
        \]
        将上述方程与 $y=\dfrac{1}{x}(x>0)$ 联立,得 
        \[
        x^{2}+2ax-a^{2}=0\Rightarrow x_{p}=(\sqrt{2}-1)a
        \]
        于是
        \[
        |PQ|=\sqrt{1+\left(\frac{1}{a^{2}}\right)^{2}}\cdot|x_{p}-x_{Q}|=\sqrt{1+\frac{1}{a^{4}}}\cdot\sqrt{2}a
        \ge\sqrt{\frac{2}{a^2}}\cdot\sqrt{2}a=2
        \]
        当 $a=1$,即 $Q(-1,1)$ 时,$|PQ|$ 取最小值 $2$。
    \end{solution}

    \question 在 $\Gamma: y = x^3$ 上有一点 $P$,已知 $P$ 在第一象限且其 $x$ 坐标为 $a$。现以 $P$ 为切点作 $\Gamma$ 之切线 $L$,交 $y$ 轴于点 $Q$,且交 $\Gamma$ 于另一点 $S$,试求:
    \begin{parts}
    \part  ${PQ}:{QS}$  
    \begin{solution}
        已知曲线 $\Gamma: y = x^3$,故其导数为:
        \[
        y' = 3x^2
        \]
        在点 $P(a, a^3)$ 处的切线斜率为 $3a^2$,代入点斜式得切线 $L$:
        \[
        y = 3a^2(x - a) + a^3
        \]
        切线交 $y$ 轴于
        \[
        y = -3a^3 + a^3 = -2a^3 \Rightarrow Q(0, -2a^3)
        \]
        将切线方程代入 $\Gamma$ 得:
        \[
        x^3 = 3a^2(x - a) + a^3 \Rightarrow x^3 - 3a^2x + 2a^3 = 0\Rightarrow (x - a)^2(x + 2a) = 0
        \]
        另一个交点 $S$ 对应 $x = -2a$,代入 $\Gamma$ 得
        \[
        S = (-2a, -8a^3)
        \]
        于是
        \[
        {PQ} = \sqrt{(a - 0)^2 + (a^3 + 2a^3)^2} = \sqrt{a^2 + 9a^6}
        \]
        \[
        {QS} = \sqrt{(0 + 2a)^2 + (-2a^3 + 8a^3)^2} = \sqrt{4a^2 + 36a^6}
        \]
        故比值为
        \[
        {PQ}:{QS} = 1:2
        \]
    \end{solution}
    \part  $\Gamma$ 与切线 $L$ 所围成之封闭区域的面积,以 $a$ 表示。
    \begin{solution}
        所围面积为 $\Gamma$ 与切线之间的面积,即:
        \begin{align*}
        \int_{-2a}^{a} \left[x^3 - (3a^2(x - a) + a^3)\right] dx
        &= \int_{-2a}^{a} \left[x^3 - 3a^2x + 2a^3\right] dx\\
        &= \left[ \frac{1}{4}x^4 - \frac{3}{2}a^2x^2 + 2a^3x \right]_{-2a}^{a}
        = \frac{27}{4}a^4
        \end{align*}
    \end{solution}
    \end{parts}

    \question 在平面直角坐标系中,函数 $$y=\frac{x+1}{|x|+1}$$ 的图像上有三个不同的点位于直线 $l$ 上,且这三点的横坐标之和为0.求 $l$ 的斜率的取值范围.
    \begin{solution}
        当 $x\ge0$ 时, $y=1$ ;当 $x<0$ 时, $y=\frac{x+1}{1-x}$ 关于 $x$ 严格递增且小于1.
        设直线 $l:y=kx+b$ ,则条件等价于方程 
        \[
        kx+b=\frac{x+1}{|x|+1} \tag{1}
        \]
        有三个不同的实数解 $x_{1},x_{2},x_{3}(x_{1}<x_{2}<x_{3})$ ,满足 $$x_{1}+x_{2}+x_{3}=0$$
        首先有 $k\ne0$ (否则, $l$ 只能是 $y=1$ ,但此时 $l$ 与函数 $y=\frac{x+1}{|x|+1}$ 图像的任意三个公共点的横坐标之和必大于0,不合题意)
        
        当 $x<0$ 时,方程(1)可整理为 
        \[
        kx^{2}-(k-b-1)x+1-b=0  \tag{2}
        \]
        至多两个负数解;当 $x\ge0$ 时,方程(1)即为 \[kx+b=1 \tag{3}\]
        至多一个非负解,这表明方程(2)有两个不同的负数解 $x_{1},x_{2}$ ,其中 $$x_{1}+x_{2}=\frac{k-b-1}{k}$$方程(3)有非负解 $x_{3}=\dfrac{1-b}{k}$;由 $x_{1}+x_{2}+x_{3}=0$ ,可知 $k=2b$,进而有 $$x_{3}=\frac{1-b}{2b}$$ 由 $x_{3}\ge0$ 得 $0<b\le1$.
        方程(2)变为 $2bx^{2}+(1-b)x+1-b=0$ ,由判别式 $$(1-b)^{2}-4\cdot2b(1-b)=(1-b)(1-9b)>0,$$ 
        并结合 $0<b\le1$ ,可知 $0<b<\frac{1}{9}$;经检验,此时 $x_{1},x_{2}$ 确实为负数,符合题意,故 $l$ 的斜率 $k$ 的取值范围是 $$0<k<\frac{2}{9}.$$
    \end{solution}

    \question 直线 $y = \sqrt{3}x$ 上有一点 $A,x$ 轴上有一点 $B$, 圆 $(x - 12)^2 + (y - 5)^2 = 4$ 上有一点 $C$。试求 $\triangle ABC$ 的最小周长。 
    \ifprintanswers
    \begin{figure}[H]
        \centering
        \includegraphics[width=0.5\linewidth]{images/image55.png}
    \end{figure}
    \fi
    \begin{solution}
        设 $O(0,0)$, 圆心 $P(12,5)$,当$OP$ 与圆的交点为 $C$时$\triangle ABC$ 周长为最小,此时 
        \[
        OC = OP - 2 = 11
        \]  
        设点 $C$ 关于直线 $y = \sqrt{3}x$ 的对称点为 $C'$, 关于 $x$ 轴的对称点为 $C''$,则线段 $C'C''$ 与直线 $y = \sqrt{3}x$ 的交点为 $A$, 与 $x$ 轴的交点为 $B$,因此$AC = AC', BC = BC''$,故三角形周长为
        \[
        AC' + AB + BC'' = C'C''
        \]
        设$\angle AOC = \theta_1, \angle COB = \theta_2,$直线 $y = \sqrt{3}x$ 与 $x$ 轴夹角为 $60^\circ$, 即
        \[
        \theta_1 + \theta_2 = 60^\circ
        \]
        又因$C, C''$ 关于 $x$轴对称, $C, C'$  关于直线$y=\sqrt{3}x$ 对称,所以
        \[
        \angle BOC'' = \theta_1,\angle AOC' = \theta_2 \Rightarrow \angle C'OC'' = 2(\theta_1 + \theta_2) = 120^\circ 
        \]
        且
        \[
        OC'= OC'' = OC = 11
        \]
        在 $\triangle OC'C''$ 中,由余弦定理,
        \[
        C'C''^2=11^2 + 11^2-2 \cdot 11 \cdot 11\cdot \cos 120^\circ \Rightarrow C'C'' = 11\sqrt{3}
        \]
    \end{solution}


\end{questions}

\pagebreak

\begin{center}
  {\fontsize{30pt}{26pt}\selectfont
    \hypertarget{圆锥曲线}{圆锥曲线} \label{圆锥曲线}
  }
\end{center}
\separator
\vspace{1pt}

\begin{questions}
    \question 已知三圆方程为 $C_1: x^2+y^2+2ax+12y+10a+8=0,C_2: x^2+y^2-2x-a^2+2a=0,C_3: x^2+y^2-22x-6ay+8a^2-25a+36=0$,若三圆的圆心共线,求 $a$ 的值。
    \begin{solution}
        设$C_1,C_2,C_3$ 的圆心分别为$(-a, -6),(1, 0),(11, 3a)$,已知三圆心共线,所以斜率满足
        \[
        \frac{0 - (-6)}{1 - (-a)} = \frac{3a - 0}{11 - 1}
        \]
        解得
        \[
        a = -5 \quad \text{ 或 } \quad a = 4
        \]
        验证 $C_3$ 的半径,
        \[
        r_3^2 = 11^2 + (3a)^2 - (8a^2 - 25a + 36) = a^2 + 25a + 85
        \]
        当$a=4, r_3^2 = 201 > 0$,当$a=-5, r_3^2 = -15 < 0$,故可行解只有
        \[
        a = 4
        \]
    \end{solution}

    \question 若方程 $ x^2 + y^2 + 4k x - 6k y + 12k^2 - 4k - 8 = 0 $ 表示一个圆,求面积为最小时圆的方程。
    \begin{solution}
        将原式配方成
        \[
        (x + 2k)^2 + (y - 3k)^2 = k^2 + 4k + 8
        \]
        此为圆的标准方程,圆心为 $(-2k,\; 3k)$,半径平方
        \[
        r^2 = k^2 + 4k + 8 = (k+2)^2+4
        \]
        在 $k = -2$时最小, 因此圆的方程为
        \[
        (x - 4)^2 + (y + 6)^2 = 4
        \]
    \end{solution}
        
    \question 一个圆经过点 \( A(4, -2) \),且与 \(x\) 轴和 \(y\) 轴都相切,求该圆的方程式。
    \begin{solution}
        发现$A(4, -2)$在第四象限,已知圆与 \(x\) 轴和 \(y\) 轴都相切,所以圆只能落在第四象限。又因为圆心到两个坐标轴的距离都等于半径 \(r\),设圆心为 \((r,\; -r)\),半径为 \(r > 0\),方程为
        \[
        (x - r)^2 + (y + r)^2 = r^2
        \]
        将已知点 \(A(4, -2)\) 代入:
        \[
        (4 - r)^2 + (-2 + r)^2 = r^2
        \Rightarrow r^2 - 12r + 20 = 0
        \Rightarrow r = 2 \text{ 或 } 10
        \]
        因此圆方程为 \((x - 2)^2 + (y + 2)^2 = 4\)
        或\((x - 10)^2 + (y + 10)^2 = 100\)
    \end{solution}

    \question 设一个圆与直线 \(L_1: 3x - 4y + 7 = 0\)、\(L_2: 3x - 4y - 3 = 0\) 都相切,且圆心位于直线 \(L: x - 2y + 2 = 0\) 上,求该圆的方程式。  
    \begin{solution}
        首先,直线 \(L_1\) 和 \(L_2\) 是两条平行线,它们之间的距离为圆的直径:
        \[
        \text{直径} = \frac{|-3 - (7)|}{\sqrt{3^2 + (-4)^2}} = 2 \Rightarrow r = 1
        \]
        设圆心为 \((a, b)\),圆心到 \(L_1\) 或 \(L_2\) 的距离都等于 \(1\),即\[
        \begin{cases}
        \dfrac{|3a - 4b + 7|}{5} = 1 \Rightarrow  3a - 4b + 7 = \pm 5 \\
        \dfrac{|3a - 4b - 3|}{5} = 1 \Rightarrow  3a - 4b -3 = \pm 5 
        \end{cases}
        \]
        给出 $3a - 4b +2=0$,又圆心在直线 \(x - 2y + 2 = 0\) 上,即解\[
        \begin{cases}
        3a - 4b + 2 = 0 \\
        a - 2b + 2 = 0 
        \end{cases} \Rightarrow (a,b)=(2,2)
        \]     
        所以圆方程为
        \[
        (x - 2)^2 + (y - 2)^2 = 1
        \]
        \end{solution}

    \question 试求过点 \(P(3, 1)\) 且与圆 \((x - 1)^2 + (y + 2)^2 = 4\) 相切的直线方程。  
        \begin{solution}
        先判断点 \(P(3, 1)\) 是否在圆外:
        \[
        (3 - 1)^2 + (1 + 2)^2 = 13 > 4
        \Rightarrow P \text{ 在圆外}
        \]
        设切线斜率为 \(m\),切线方程为
        \[
        y - 1 = m(x - 3) \Rightarrow mx - y - 3m + 1 = 0
        \]
        圆心为 \((1, -2)\),半径 \(r = 2\),由圆心到切线距离为半径性质得
        \[
        \frac{|m \cdot 1 - (-2) - 3m - 1|}{\sqrt{m^2 + 1}} = 2
        \Rightarrow \frac{|2m - 3|}{\sqrt{m^2 + 1}} = 2
        \]
        两边平方得
        \[
        \frac{(2m - 3)^2}{m^2 + 1} = 4
        \Rightarrow m=\frac{5}{12}
        \]
        只解出一$m$,表示另一切线为铅直线 $x=3$,当 $m=\frac{5}{12}$ ,切线为 $5x-12y-3=0$
    \end{solution}
        
    \question 从点 \(P(3,5)\) 向圆\(C: x^2 + y^2 + 2x - 4y = 0\) 所作的两条切线切点分别为 \(A,B\),试求过 \(A,B,P\) 三点的圆方程。  
    \begin{solution}
        圆 \(C\) 的方程为
        \[
        x^2 + y^2 + 2x - 4y = 0 \Rightarrow (x+1)^2 + (y-2)^2 = 5
        \]
        其中圆心为 \(O(-1, 2)\),半径为 \(\sqrt{5}\)。
        
        发现过 \(A,B,P\) 三点的圆即是以 \(OP\) 为直径的圆(半圆含直角),该圆圆心为 \(OP\) 的中点:
        \[
       \left( \frac{3 + (-1)}{2}, \frac{5 + 2}{2} \right) = (1, \frac{7}{2})
        \]
        计算直径 \(|OP|\):
        \[
        |OP| = \sqrt{(3 + 1)^2 + (5 - 2)^2} = 5
        \Rightarrow \text{半径} = \frac{5}{2}
        \]
        因此所求圆的方程为
        \[
        (x - 1)^2 + \left(y - \frac{7}{2} \right)^2 = \frac{25}{4}
        \]
    \end{solution}

    \question 求两圆
    \[
    C_1: x^2+y^2-2y-4=0, \quad C_2: x^2+y^2-6x-11=0
    \]
    的公切线方程。
    \begin{solution}
        $C_1$圆心为 $A(0,1)$,半径为 $r_1 = \sqrt{5}$。$C_2$圆心为 $B(3,0)$,半径为 $r_2 = 2\sqrt{5}$。圆心距为
        \[
        AB = \sqrt{(3-0)^2 + (0-1)^2} = \sqrt{10}
        \]
        且
        \[
        r_1 + r_2 = 3\sqrt{5}, \quad r_2 - r_1 = \sqrt{5}
        \]
        由于 $r_2-r_1 < AB < r_1+r_2$,两圆相交于两点。设两条所求外公切线的交点为 $P(a,b)$,由相似三角形,$P$的$x,y$-坐标满足
        \[
        \frac{0-a}{3-a}=\frac{1-b}{0-b}=\frac{PA}{PB}=\frac{1}{2}
        \]
        解得$P(-3,2)$,现设切线方程为 
        \[
        L: y-2 = m(x+3) \Rightarrow mx - y + 3m + 2 = 0
        \]
        由于圆心$A(0,1)$至切线的距离等于半径,
        \[
        \frac{|m(0)-1+3m+2|}{\sqrt{m^2+1}} = \sqrt{5} 
        \]
        解得
        \[
        m = \frac{1}{2}, \quad m = -2
        \]
        故对应的公切线方程为
        \[
        y-2 = \frac{1}{2}(x+3) \Rightarrow x - 2y + 7 =0
        \]
        及
        \[
        y-2 = -2(x+3) \Rightarrow 2x + y +4 = 0
        \]
    \end{solution}

    \question
例 10、半径分别为 $1\text{ cm}、2\text{ cm}$ 及 $3\text{ cm}$ 的三个圆互相外切,如图所示。有一个小圆落在它们之间,且与它们都相切。若此小圆的半径为 $\frac{p}{q}\text{ cm}$,其中 $p$ 及 $q$ 为除了 $1$ 以外,并无其它公因数的两个正整数,求 $p+q$。

\begin{solution}
解:如图,建立坐标系得到三个圆方程:
$x^2 + y^2 = 1^2$,$x^2 + (y-3)^2 = 2^2$,$(x-4)^2 + y^2 = 3^2$,
设中间小圆圆心为 $(m, n)$,半径为 $r$。
利用相切性质,圆心距离 $=$ 半径之和:
\[ \sqrt{m^2 + n^2} = r + 1, \quad \sqrt{m^2 + (n-3)^2} = r + 2, \quad \sqrt{(m-4)^2 + n^2} = r + 3 \]
得
\begin{align*}
m^2 + n^2 &= r^2 + 2r + 1 \quad \dots (1) \\
m^2 + (n-3)^2 &= r^2 + 4r + 4 \quad \dots (2) \\
(m-4)^2 + n^2 &= r^2 + 6r + 9 \quad \dots (3)
\end{align*}
$(2) - (1)$ 得到 $n = 1 - \frac{2}{3}r$,
$(3) - (1)$ 得到 $m = 1 - \frac{r}{2}$。
代回 $(1)$:
\[ \left( 1 - \frac{r}{2} \right)^2 + \left( 1 - \frac{2}{3}r \right)^2 = r^2 + 2r + 1 \Rightarrow 25r^2 - 156r + 36 = 0 \]
$r = 6, \frac{6}{25}$。
根据条件 $6 + 25 = 31$。
\end{solution}

    \question 已知有一圆方程式为 $x^2+y^2=37$, 且圆的内部有一点 $P(1,2)$。
    \begin{parts}
    \part 若 $P$ 点为圆的某弦的中点, 试求此弦所在的直线方程式。
    \begin{solution}
        圆方程为 $x^2+y^2=37$,圆心为 $O(0,0)$,半径为 $\sqrt{37}$,点 $P(1,2)$ 在圆内,$OP$ 的斜率为
        \[
        m=\frac{2-0}{1-0}=2
        \]
        弦过 $P$ 且垂直于 $OP$,故斜率为 $-\dfrac{1}{2}$,弦方程为
        \[
        y-2=-\frac{1}{2}(x-1) \Rightarrow x+2y=5
        \]
    \end{solution}
    \part 若 $P$ 点为圆的某弦的一个三等分点, 试求此弦所在的直线方程式。
    \ifprintanswers
    \begin{figure}[H]
        \centering
        \includegraphics[width=0.4\linewidth]{images/image227.png}
    \end{figure}
    \fi
    \begin{solution}
        设弦长为 $3a$,弦的三等分点之一为 $P$,弦中点为 $Q$,则
        \[
        PQ=\frac{a}{2},\quad OP=\sqrt{1^2+2^2}=5
        \]
        设弦所在直线为
        \[
        y=m(x-1)+2.
        \]
        由毕氏定理
        \[
        OQ^2=OA^2-AQ^2=OP^2-PQ^2 \Rightarrow 
        37-\left(\frac{3a}{2}\right)^2=5-\left(\frac{a}{2}\right)^2,
        \]
        解得$a=4$,故由
        \[
        OQ=\frac{|2-m|}{\sqrt{m^2+1}}=1
        \]
        解得$m=\dfrac{3}{4}$或$m=\infty$,于是弦方程为
        \[
        3x-4y+5=0,\quad \text{或}\quad x=1
        \]
    \end{solution}
    \end{parts}
    \question 设一条光线经过点 \(A(-4,5)\),经 \(x\) 轴反射后恰与圆 \((x - 2)^2 + (y - 2)^2 = 5\) 相切,求反射线的斜率。
    \begin{solution}
        设反射线斜率为 \(k\),反射线经过\(A(-4,5)\)关于$x$轴的对称点: \((-4,-5)\),方程式为
        \[
        y+5=k(x+4) \Rightarrow y = kx + 4k - 5
        \]
        圆 \((x - 2)^2 + (y - 2)^2 = 5\)圆心为 \(C(2,2)\),半径为 \(\sqrt{5}\),由圆心到切线的距离等于半径的性质,
        \[
        \frac{|k \cdot 2 - 1 \cdot 2 + (4k - 5)|}{\sqrt{k^2 + 1}} =\frac{|6k - 7|}{\sqrt{k^2 + 1}} = \sqrt{5}
        \]
        即
        \[
        (6k - 7)^2 = 5(k^2 + 1)
        \Rightarrow 31k^2 - 84k + 44 = 0 \Rightarrow (k-2)(31k-22)=0
        \]
        得
        \[
        k = 2 \; \text{或}\; k = \frac{22}{31}\;\text{(不合题意)}
        \]
        $\therefore \;k=2$

        \textbf{又解}:用判别式$=0$求$k$
    \end{solution}

    \question 坐标平面上一圆被直线 \(x - y = 1\) 和 \(x - y = 5\) 所截的弦长均为 14,求该圆的面积。 
        \begin{solution}
        设圆半径为 \(r\),两条直线\(x - y = 1\) 和 \(x - y = 5\)间距离为
        \[
        \frac{|5 - 1|}{\sqrt{1^2 + (-1)^2}} = \frac{4}{\sqrt{2}} = 2\sqrt{2}
        \]
        因两弦等长且平行,圆心到两条直线的距离相等,圆心到任一条直线距离为
        \[
        d =\sqrt{2}
        \]
        由毕氏定理,
        \[
        r^2=\left(\frac{14}{2}\right)^2 + (\sqrt2)^2= 51
        \]
        故圆面积为$51\pi$。
    \end{solution}

    \question 若直线 \(2x - y + a = 0\) 与圆 \(x^2 + y^2 - 6x + by + c = 0\) 在点 \((4,1)\) 相切,求$a,b,c$的值。  
    \begin{solution}
        圆心 $\left(3,\ -\dfrac{b}{2}\right),$ 半径 $r = \sqrt{9 + \dfrac{b^2}{4} - c}$,切点 \((4,1)\)在直线与圆上,解得
        \[
        2(4) - 1 + a = 0 \Rightarrow a=-7
        \]
        \[
        4^2 + 1^2 - 6 \cdot 4 + b \cdot 1 + c = 0  \Rightarrow b + c = 7 \quad \tag{1}
        \]
        圆心到切线的距离等于半径:
        \[
        \frac{|2 \cdot 3 - (-\frac{b}{2}) - 7|}{\sqrt{5}}  = \sqrt{9 + \frac{b^2}{4} - c}
        \Rightarrow
        \left( \frac{b}{2} - 1 \right)^2 = 5\left(9 + \frac{b^2}{4} - c\right)
        \]
        将 (1) 代入得    
        \[
        (b-2)^2=5(8+b^2+4b)
        \Rightarrow b = -3, c=-7
        \]
        故
        \[
        a=-7,b=-3,c=10
        \]
    \end{solution}

    \question 设点 \(O_1\) 为圆 \(C: x^2 + y^2 - 6x + 4y + 9 = 0\) 的圆心。今以另一点 \(O_2\) 为圆心,\(\;O_1O_2\) 为半径作一圆,且该圆与圆 \(C\) 交于 \(A,B\) 两点。若 \(AO_2 = 3\),求$AB$的长度。 
    \begin{solution}
        已知圆 \(C: x^2 + y^2 - 6x + 4y + 9 = 0\),圆心 \(O_1 = (3, -2)\),半径 \(r_1 = 2\)。

        观察到 \(\triangle AO_1O_2\) 的三边为 \(AO_1 = 2, AO_2 = 3, O_1O_2 = 3\),用海伦公式得面积:
        \[
        S = \sqrt{4(4 - 2)(4 - 3)(4 - 3)} = 2\sqrt{2}
        \]
        又
        \[
        S = \frac{1}{2} \cdot O_1O_2 \cdot h = \frac{3h}{2} \Rightarrow h = \frac{4\sqrt{2}}{3}
        \]
        所以 \(AB = 2h = \dfrac{8\sqrt{2}}{3}\)。
    \end{solution}

    \question 平面上有一定点 $A(-3,3)$ 及一圆 $C:x^2+y^2-4x-4y+k=0$,若光源由 $A$ 点射出,碰到 $x$ 轴上 $P,Q$ 两点形成的两条反射光线恰好与圆 $C$ 相切,且 $PQ = \dfrac{7}{4}$,求 $k$ 之值。
    \ifprintanswers
    \begin{figure}[H]
        \centering
        \includegraphics[width=0.5\linewidth]{images/image46.png}
    \end{figure}
    \fi
    \begin{solution}
        圆 $C:(x-2)^2+(y-2)^2=8-k$ 圆心 $M(2,2)$,半径 $r=\sqrt{8-k},A(-3,3)$ 关于 $x$ 轴的对称点为 $A'(-3,-3)$,于是过$A',M$的直线方程式为
        \[
        x=y
        \]
        且此线为 $\angle PA'Q$ 的角平分线,故$L$ 交 $x$轴于 $O(0,0)$  
        现设 $PO = a,QO = \dfrac{7}{4} - a$,则
        \[
        P(-a,0),Q\left(-a+\frac{7}{4},0\right)
        \]
        根据反射光与入射光夹角相等,有
        \[
        \frac{A'P}{A'Q} = \frac{PO}{QO} 
        \Rightarrow \frac{\sqrt{(-3+a)^2+9}}{\sqrt{(a-\frac{19}{4})^2+9}} = \frac{a}{\frac{7}{4}-a}
        \]
        整理得
        \[
        4a^2 - 31a + 21 = 0 \Rightarrow a = \frac{3}{4} \quad (a=7\text{ 不合})
        \]
        则$P\left(-\dfrac{3}{4},0\right)$,过$A',P$的直线方程式为
        \[
        3y = 4x + 3
        \]
        由$ r = d(M, L_2)$,得
        \[
        1 = \sqrt{8-k} \Rightarrow k=7
        \]
    \end{solution}

    \question 在直角坐标平面上,已知圆 $C$ 的半径为 $4\sqrt{13}$ 且圆心在第三象限。从圆 $C$ 外一点 $P$ 对圆 $C$ 作两条切线,切点为 $A, B$,斜率分别为 $\dfrac{2}{3},\dfrac{3}{2}$。若 $\triangle PAB$ 的外接圆圆心为 $(6,4)$,求圆 $C$ 的圆心坐标。 
    \ifprintanswers
    \begin{figure}[H]
        \centering
        \includegraphics[width=0.5\linewidth]{images/image52.png}
    \end{figure}
    \fi
    \begin{solution}
        因为 $A,B$ 为切点,故
        \[
        \angle CAP = \angle CBP = 90^\circ,
        \]
        且外接圆圆心 $Q(6,4)$ 是线段 $CP$ 的中点,$CP$ 是外接圆直径。设圆心 $C(a,b)$,则点
        \[
        P = (12 - a, \; 8 - b).
        \]
        切线方程为
        \[
        L_1: y = \frac{2}{3} (x - (12 - a)) + (8 - b), \quad
        L_2: y = \frac{3}{2} (x - (12 - a)) + (8 - b).
        \]
        整理得
        \[
        L_1: 2x - 3y + 2a - 3b = 0, L_2: 3x - 2y + 3a - 2b - 20 = 0
        \]
        圆心 $C$ 到两切线距离均为半径 $4 \sqrt{13}$,
        \[
        \frac{|2a - 3b|}{\sqrt{13}} = \frac{|3a - 2b - 20|}{\sqrt{13}} = 4 \sqrt{13}
        \]
        由 $a,b < 0$,解得圆心为
        \[
        a = -20, b = -22 \Rightarrow (-20, -22)
        \]
    \end{solution}

    \question 已知圆 $x^{2}+y^{2}=37$ 内一点 $P(1,2)$,若 $P$ 点为某弦的三等分点,求此弦所在的直线方程。
    \ifprintanswers
    \begin{figure}[H]
        \centering
        \includegraphics[width=0.4\linewidth]{images/image106.jpg}
    \end{figure}
    \fi
    \begin{solution}
        设 $P$ 为弦 $AB$ 的三等分点,即$AP = k, PB = 2k$,圆心 $O$ 到 $P$ 的距离 
        \[
        OP = \sqrt{1^2+2^2} = \sqrt{5}
        \]
        令直线 $OP$ 与圆交于 $C, D$ 两点,由$\triangle APC \backsim \triangle DPB \ $(SAS),得
        \[
        \frac{AP}{PC} = \frac{DP}{PB} \Rightarrow \frac{k}{\sqrt{37}-\sqrt{5}} = \frac{\sqrt{37}+\sqrt{5}}{2k} \Rightarrow k = 4
        \]
        因此$AP = 4, PB = 8$,中点 $Q$ 为 $AB$ 的中点,则 
        \[
        PQ = 2,\quad OQ = 1
        \]
        设直线 $AB: y = m(x-1)+2$,由圆心到切线的距离等于半径的性质,解得
        \[
        OQ = 1 = \frac{|2 - m|}{\sqrt{m^2+1}} \Rightarrow m = \frac{3}{4} 
        \]
        或直线$AB$与$x$轴垂直,所以弦所在直线方程为
        \[
        3x - 4y + 5 = 0 \quad \text{或} \quad x = 1
        \]
    \end{solution}

    \question 由圆$x^2+y^2-4x+6y=7$外一点 $P$向该圆引两条切线,切于两点 $A(6,-1),B$。若点 $P$ 到圆心 $C$ 的距离为 $\sqrt{65}$,求 $B,P$ 的坐标。
    \begin{solution}
        圆$x^2+y^2-4x+6y=7$的圆心为$C(2,-3)$,半径为$r=\sqrt{20}$。设 $P(a,b)$,由于 $AP \perp AC$,
        \[
        \frac{-1-(-3)}{6-2}\cdot\frac{b+1}{a-6}=-1 \Rightarrow b=11-2a.
        \]
        由毕氏定理, $PA=\sqrt{65-20}=\sqrt{45}$,即
        \[
        (a-6)^2+(b+1)^2=45
        \]
        代入 $b=11-2a$解得
        \[
        (a-6)^2+(12-2a)^2=45 \Rightarrow a=3 \quad \text{或} \quad a=9
        \]
        故
        \[
        P(3,5)\quad \text{或}\quad P(9,-7).
        \]
        现求对应的 $B$,设 $B(k,h)$,由$BP=\sqrt{45},BC=\sqrt{20}$,当 $P(3,5)$,解
        \[
        \begin{cases}
        (k-3)^2+(h-5)^2=45,\\
        (k-2)^2+(h+3)^2=20.
        \end{cases}
        \]
        可得
        \[
        B\left(-\frac{18}{13},-\frac{1}{13}\right).
        \]
        当 $P(9,-7)$,解
        \[
        \begin{cases}
        (k-9)^2+(h+7)^2=45,\\
        (k-2)^2+(h+3)^2=20,
        \end{cases}
        \]
        得
        \[
        B\left(\frac{30}{13},-\frac{17}{13}\right).
        \]
        故所求为
        \[
        P(3,5),\; B\left(-\frac{18}{13},-\frac{1}{13}\right) \quad \text{或 } \quad P(9,-7),\; B\left(\frac{30}{13},-\frac{17}{13}\right)
        \]
    \end{solution}

    \question 已知两条平行直线 $L_1:y = 2x + 5,L_2:y = 2x - 1$是以点 $C$ 为圆心的圆的切线。若另一直线$L_3$经过点 $(9,0)$且垂直于 $L_1,L_2$,求圆心 $C$。
    \begin{solution}
        据题意,圆的半径为 $r = \dfrac{1}{2}\left|\dfrac{5+1}{\sqrt{2^2+(-1)^2}}\right|= \dfrac{3}{\sqrt{5}}$,且圆心 $C$ 位于中位线 $L:y = 2x + 2$ 上, $L_3$ 的方程为
        \[ 
        y - 0 = -\frac{1}{2}(x - 9) \Rightarrow y = \frac{1}{2}(9 - x) 
        \]
        联立$L,L_3$,
        \[ 
        2x + 2 = \frac{1}{2}(9 - x) \Rightarrow x = 1, y=4
        \]
        可知交点为 $M(1,4)$。设 $D(0,2)$ 为中位线在 $y$ 轴的截距,$L_3$与圆交于两点 $A, B$,考虑相似三角形 $\triangle AMC \backsim \triangle ADC \ \text{(AAA)}$,设$MC = x$,有
        \[
        \frac{MC}{AC} = \frac{AC}{DC} \Rightarrow \frac{x}{\frac{3}{\sqrt{5}}} = \frac{\frac{3}{\sqrt{5}}}{\sqrt{5} + x} ,
        \]
        其中$DM = \sqrt{1^2 + (4-2)^2} = \sqrt{5}$,舍去负根,解得
        \[ 
        x = \frac{-5\sqrt{5} + \sqrt{305}}{10} 
        \]
        由此得 
        \[
        DC = \sqrt{5} + \frac{-5\sqrt{5} + \sqrt{305}}{10} = \frac{5\sqrt{5} + \sqrt{305}}{10}
        \]
        设$CD$与水平轴的夹角为$\theta$,由已知可得$\tan \theta=2$,故圆心$C$的$x,y-$坐标为
        \[ 
        x = DC \cos \theta = \frac{5\sqrt{5} + \sqrt{305}}{10}\cdot \frac{1}{\sqrt{5}} = \frac{5 + \sqrt{61}}{10}
        \]
        \[ 
        y+2 = DC \sin \theta + 2 = \frac{5\sqrt{5} + \sqrt{305}}{10} \cdot \frac{2}{\sqrt{5}} + 2 = \frac{15 + \sqrt{61}}{5} 
        \]
        因此圆心 $C$ 的坐标为
        \[ 
        C\left( \frac{5 + \sqrt{61}}{10}, \frac{15 + \sqrt{61}}{5} \right) 
        \]
    \end{solution}

    \question 直线 $L$ 与圆 $C$ 的方程分别为
    \[
    L: y=\lambda(x-a)+a\sqrt{\lambda^{2}+1},\quad
    C: x^{2}+y^{2}=2ax,
    \]
    其中 $a$ 为正数,$\lambda$ 为参数。证明对任意 $\lambda$,直线 $L$ 恒为圆 $C$ 的切线。
    \begin{solution}
        将圆$C$化为标准形式,
        \[
        (x-a)^{2}+y^{2} = a^{2}
        \]
        可知圆心为 $P(a,0)$,半径为 $a$;直线 $L$ 的斜率为 $\lambda$,因此过点 $P(a,0)$ 且与 $L$ 垂直的直线斜率为 $-\dfrac{1}{\lambda}$,其方程为
        \[
        y-0 = -\frac{1}{\lambda}(x-a) \Rightarrow x = a-y\lambda
        \]
        设该垂线与直线 $L$ 交于点 $Q$,联立两直线方程解得
        \[
        Q\left(a-\frac{a\lambda}{\sqrt{\lambda^{2}+1}},\;\frac{a}{\sqrt{\lambda^{2}+1}}\right)
        \]
        于是圆心 $P(a,0)$ 到直线 $L$ 的距离为
        \[
        PQ=\sqrt{\left(a-\frac{a\lambda}{\sqrt{\lambda^{2}+1}}-a\right)^{2}+\left(\frac{a}{\sqrt{\lambda^{2}+1}}-0\right)^{2}} 
        =\sqrt{\frac{a^{2}(\lambda^{2}+1)}{\lambda^{2}+1}} 
        =a
        \]
        由于$PQ$与 $\lambda$ 无关,因此对任意 $\lambda$,直线 $L$ 始终与圆 $C$ 相切,故直线 $L$ 对所有 $\lambda$ 均为圆 $C$ 的切线。
    \end{solution}

    \question 已知圆 \((x - p)^2 + y^2 = r^2\) 及 \((y - p)^2 + x^2 = r^2\)交于相异点 \(A ,B\),且 \(a ,b\) 分别是 \(A ,B\) 的 \(x\)-坐标。 若\(C ,D\) 分别为两圆的圆心,
    \begin{parts}
    \part 证明 $a + b = p,a^2 + b^2 = r^2$且$p^2<2r^2$。
    \begin{solution}
        两圆方程式联立得
        \[
        (x - p)^2 + y^2 = (y - p)^2 + x^2
        \Rightarrow 2px = 2py  \Rightarrow x = y\; (\because p\neq0)
        \]
        代回第一式,
        \[
        (x - p)^2 + x^2 = r^2 \Rightarrow 2x^2 - 2px + p^2 - r^2 = 0
        \]
        由韦达定理,
        \[
        a + b = p,\; ab = \frac{p^2 - r^2}{2}
        \]
        又
        \[
        a^2 + b^2 = (a + b)^2 - 2ab = p^2 - 2 \cdot \frac{p^2 - r^2}{2} = r^2
        \]
        且判别式必为正,故\[
        \Delta=4p^2-4\cdot2(p^2-r^2)>0 \Rightarrow p^2<2r^2
        \]
    \end{solution}
    \part 若 \(r\) 是常量而 \(p\) 是变量,使得四边形 \(CADB\) 面积有最大值,求证此时 \(A\) 或 \(B\)是原点。
    \begin{solution}
        设 $A(a,a),B(b,b),C(p,0)$, $D(0,p)$。则直线 $CD$ 的斜率为 $-1$。点 $A,B$ 在直线 $y=x$ 上,所以 $AB$ 的斜率为 $1$。因此 $AB \perp CD$, 四边形 $CADB$ 是风筝形,其面积为
        \begin{align*}
        \frac{1}{2}\cdot AB \cdot CD 
        &= \frac{1}{2}\sqrt{2(a-b)^2}\sqrt{2p^2} \\
        &= \frac{1}{2}\sqrt{2r^2-(p^2-r^2)}\sqrt{2p^2} \\
        &= \sqrt{2r^2p^2-p^4}
        \end{align*}
        其中$f(p^2)=2r^2p^2-p^4$是一关于 $p^2$ 的二次函数,开口向下,其顶点在
        \[
        p^2=\frac{-2r^2}{2(-1)}=r^2
        \]
        因此面积在 $p^2=r^2$ 时取最大。又由 $(a+b)^2=p^2$, 所以
        \[
        a^2+2ab+b^2=r^2
        \]
        但 $a^2+b^2=r^2$, 故 $2ab=0$, 即 $a=0$ 或 $b=0$。

        若 $a=0$, 则 $A(0,0)$;若 $b=0$, 则 $B(0,0)$。所以当面积最大时,原点必为 $A$ 或 $B$。
    \end{solution}
    \end{parts}

    \question 求抛物线 \(y^{2}=16x\) 上距直线 \(4x-3y+24=0\) 最近的点坐标。  
    \begin{solution}
        设抛物线上一点为 \((t^2, 4t),\;t\in \mathbb{R}\),则到直线的距离为:
        \[
        d = \frac{|4t^2 - 12t + 24|}{\sqrt{4^2 + (-3)^2}} = \frac{4}{5}|t^2 - 3t + 6|
        \]
        又    
        \[
        t^2 - 3t + 6 = \left(t - \frac{3}{2}\right)^2 + \frac{15}{4}
        \]
        在 \(t = \dfrac{3}{2}\) 时最小,所以最近的点坐标为
        \[
        \left(t^2, 4t\right) = \left(\frac{9}{4}, 6\right)
        \]
    \end{solution}
    
    \question 已知抛物线的焦弦 \(AB\),准线到焦弦两端的垂足满足 \(AM=9,\ BN=4\),求 \(MN\)。
    \ifprintanswers
    \begin{figure}[H]
    \centering
    \includegraphics[width=0.2\textwidth]{images/image15.png}
    \end{figure}
    \fi
    \begin{solution}
        由抛物线定义,
        \[
        AB=AF+BF=AM+BN=9+4=13
        \]
        由毕氏定理,
        \[
        AB^2=MN^2+(AM-BN)^2 \Rightarrow MN^2=13^2-5^2=12^2
        \]
        $\therefore\; MN=12$
    \end{solution}
    
    \question 已知一个抛物线形状的拱桥,拱顶 A 点离水面 2 公尺时,水面宽度 \(BC = 4\) 公尺。若水面再下降 1 公尺,求新的水面宽度 \(DE\)。
    \begin{solution}
        设拱顶 A 为原点,抛物线开口向下,方程为:
        \[
        y = -a x^2,a>0
        \]
        当 \(y = -2\) 时,\(x = \pm 2\),代入得:
        \[
        -2 = -a(2)^2 \Rightarrow a = \frac{1}{2}
        \Rightarrow y = -\frac{1}{2}x^2
        \]
        当 \(y = -3\) 时,
        \[
        -3 = -\frac{1}{2}x^2 \Rightarrow x = \pm\sqrt{6}
        \Rightarrow DE = 2\sqrt{6}
        \]
    \end{solution}
    
    \question 求一边在直线 \(y=2x-17\) 上,另两顶点在抛物线 \(y=x^{2}\) 上的正方形面积。
    \begin{solution}
        设正方形顶点$A(x_1,x_1^2),B(x_2,x_2^2)$在抛物线 \(y=x^{2}\) 上,有
        \[
        m_{AB}=\dfrac{x_2^2-x_1^2}{x_2-x_1}=x_1+x_2=2
        \]
        且
        \[
        \left|\frac{2x_1-x_1^2-17}{\sqrt{2^2+1^2}}\right|=\sqrt{(x_2-x_1)^2+(x_2^2-x_1^2)^2}
        \]
        将$x_2=2-x_1$代入化简得\[
        (2x_1-x_1^2-17)^2=100(1-x_1)^2
        \]
        不失一般性,即解\[
        2x_1-x_1^2-17=10(1-x_1)
        \]
        得$x_1=3$或$x_1=9$,故正方形面积为$16$或$256$.
    \end{solution}

    \question 设一抛物线的顶点坐标为 $V(-1,3)$,且对称轴的方程为 $L: 2x+y-1=0$。若此抛物线经过点 $A(3,3)$,求此抛物线的正焦弦长。
    \begin{solution}
        直线 $L: 2x+y-1=0$ 与 $x$ 轴交于 $P\left(\dfrac{1}{2},0\right)$,以 $P$ 为旋转中心,逆时针旋转 $\tan^{-1} 2$ 将 $L$ 变为 $x$ 轴。

        顶点 $V(-1,3)$ 旋转后得到 $V'\left(\dfrac{1}{2}-\dfrac{15}{2\sqrt 5},0\right)$,点 $A(3,3)$ 旋转后得到 $A'\left(\dfrac{1}{2}-\dfrac{7}{2\sqrt 5},\dfrac{8}{\sqrt 5}\right)$。

        以 $V'$ 为顶点,旋转后的抛物线方程为
        \[
        y^2 = 4c\left(x - \frac{1}{2} + \frac{15}{2\sqrt 5}\right).
        \]
        将 $A'$ 代入方程,解得
        \[
        c = \frac{4\sqrt 5}{5}
        \]
        因此正焦弦长为
        \[
        4c = \frac{16\sqrt 5}{5}
        \]
    \end{solution}

    \question 已知抛物线顶点在原点,焦点在 $x$ 轴上,$\ \triangle ABC$ 三个顶点都在抛物线上,且重心为抛物线焦点 $F$,边 $BC$ 所在直线为 $4x+y-20=0$,求抛物线方程。
    \begin{solution}
        设抛物线方程为
        \[
        y^2 = 4c x
        \]
        $BC$ 与直线 $L: 4x+y-20=0$ 相交,由于直线上动点 $P(t,20-4t)$在抛物线上,
        \[
        (20-4t)^2 = 4 c t \Rightarrow 4 t^2 - (40+c) t + 100 =0.
        \]
        设两交点为 $B(t_1,20-4t_1),C(t_2,20-4t_2),A(x_a,y_a)$ 在抛物线上,且已知重心为焦点 $F(c,0)$,得
        \[
        c = \frac{x_a + t_1 + t_2}{3}, \quad 0 = \frac{y_a + (20-4t_1) + (20-4t_2)}{3}.
        \]
        解得
        \[
        x_a = \frac{11}{4}c -10, \quad y_a = c.
        \]
        又$A(x_a,y_a)$ 在抛物线上,
        \[
        c^2 = 4 c \left(\frac{11}{4}c - 10 \right) \ \Rightarrow c=4.
        \]
        因此抛物线方程为
        \[
        y^2 = 16 x
        \]
    \end{solution}

    \question 已知 $P(a, -a^2),Q(b, -b^2)$在抛物线$y = -x^2$上,其中$a < 0,b > 0$。设 $M$ 为线段 $PQ$ 的中点,$R$ 为过点 $M$ 的垂直直线与抛物线的交点, $l$ 为抛物线在点 $Q$ 处的切线。证明任何一端在 $PQ$ 上、另一端在 $l$ 上的垂直线段均被过 $Q,R$ 的直线平分。
    \begin{solution}
        切线 $l$ 的方程为
        \[ 
        y = -2bx + b^2 
        \]
        包含点 $P,Q$ 的直线方程为
        \[ 
        \frac{y+b^2}{x+a^2}=\frac{b+b^2}{a+a^2} \Rightarrow y = -(b+a)x + ab 
        \]
        $M$ 坐标为 $\left(\dfrac{a+b}{2},-\dfrac{a^2+b^2}{2}\right)$,故 $R$ 坐标为 $\left(\dfrac{a+b}{2}, -\left(\dfrac{a+b}{2}\right)^2\right)$,过
        $Q$ 及 $R$ 的直线方程为
        \[ 
        y = -\left(\frac{a+3b}{2}\right)x + \left(\frac{ab+b^2}{2}\right) 
        \]
        从线段 $PQ$ 到切线 $l$ 的垂直线段的中点 $y$ 坐标,是两条直线纵坐标的平均值
        \[ 
        y_{avg} = \frac{1}{2}(-2bx + b^2 - (b+a)x + ab) = -\left(\frac{a+3b}{2}\right)x + \left(\frac{ab+b^2}{2}\right)
        \]
        与经过 $Q$ 和 $R$ 的直线方程完全一致,故得证。
    \end{solution}

    \question 已知抛物线 $$y^2 = 2px$$ 与双曲线 $$y = -\dfrac{1}{x}$$ 相交于点 $R$,一条同时与抛物线和双曲线相切的公切线分别相切于点$S,T$,求证对于任意正实数 $p,\triangle RST$ 的面积与$p$无关,即为一定值。
    \begin{solution}
        设公切线为 $y=mx+c$,代入抛物线得
        \[
        (mx+c)^2 = 2px \Rightarrow m^2x^2 + 2(mc-p)x + c^2 = 0 \tag{1}
        \]
        令判别式为零:
        \[
        4(mc - p)^2 - 4m^2c^2 = 0 \Rightarrow p=2mc \tag{2}
        \]
        同理,将$y = mx + c$代入双曲线得
        \[
        mx^2 + cx + 1 = 0 \tag{3}
        \]
        且有
        \[
        c^2 = 4m \tag{4}
        \]
        由$(1),(3)$得
        \[
        x_{S}=\frac{p-mc}{m^2}, x_{T}=-\frac{c}{2m} 
        \]
        由$(2),(4)$得
        \[
        x_{S}=\frac{c}{m}=\frac{4}{c}, x_{T}=-\frac{2}{c} \Rightarrow y_{S}=2c, y_{T}=\frac{c}{2}
        \]
        即\[
        S\left(\frac{4}{c},2c\right),T\left(-\frac{2}{c},\frac{c}{2}\right)
        \]
        将抛物线与双曲线联立得
        \[
        \left(-\frac{1}{x}\right)^2 = 2px \Rightarrow
        R\left( \left( \frac{1}{2p} \right)^{\frac13},\ -\left(2p\right)^{\frac13} \right)
        \]
        由$\dfrac{(4)}{(3)}$得$c=(2p)^{\frac13}$,于是$R$坐标为
        \[
        R\left(\frac{1}{c},-c\right)
        \]
        故$\triangle RST$ 的面积为
        \[
        \frac{1}{2} \left| 
        \begin{vmatrix}
        \frac{4}{c} & 2c & 1 \\
        -\frac{2}{c} & \frac{c}{2} & 1 \\
        \frac{1}{c} & -c & 1
        \end{vmatrix}
        \right|=\frac{1}{2} \left|2+2+2-\frac12+4+4\right|=\frac{27}{4},
        \]
        且与 \(p\) 无关。
    \end{solution}

    \question 已知抛物线$\Gamma$的顶点是原点$O$,焦点是$F(0,1)$.过直线$y=-2$上任意一点$A$作抛物线$\Gamma$的两条切线,切点分别为$P$、$Q$,求证:
    \begin{parts}
    \part 直线$PQ$过定点;
    \begin{solution}
        易知抛物线$\Gamma$的方程为 $x^{2}=4y$.设点 $$A(t,-2),P(x_{1},y_{1}),Q(x_{2},y_{2})$$
        则过点$P$的抛物线$\Gamma$的切线 $l_{1}$ 的方程为 $$y-y_{1}=\frac{x_{1}}{2}(x-x_{1})$$
        即 $$x_{1}x-2y-2y_{1}=0$$
        同理,过点$Q$的抛物线$\Gamma$的切线 $l_{2}$ 的方程为: $$x_{2}x-2y-2y_{2}=0$$
        由 $l_{1}$, $l_{2}$ 过点$A$,可得 $$x_{1}t+4-2y_{1}=0, x_{2}t+4-2y_{2}=0$$
        这表明,点 $P(x_{1},y_{1}),Q(x_{2},y_{2})$的坐标满足方程: $$tx-2y+4=0$$
        即直线$PQ$的方程,易得直线$PQ$过定点$(0,2)$.
    \end{solution}
    \part $\angle PFQ=2\angle PAQ$.
    \begin{solution}
        不妨设$P$在$Q$左边,则 $x_{1}<x_{2}$.
        因为 $$\tan\angle PAQ=\frac{\frac{x_{1}}{2}-\frac{x_{2}}{2}}{1+\frac{x_{1}}{2}\cdot\frac{x_{2}}{2}}=\frac{2(x_{1}-x_{2})}{x_{1}x_{2}+4}$$
        所以 $$\tan 2\angle PAQ = \frac{2\tan\angle PAQ}{1-\tan^2\angle PAQ} = \frac{\frac{4(x_1-x_2)}{x_1x_2+4}}{1-\left(\frac{4(x_1-x_2)}{x_1x_2+4}\right)^2} = \frac{4(x_1-x_2)(x_1x_2+4)}{(x_1x_2+4)^2-4(x_1-x_2)^2}$$
        又因为
        $$\tan\angle PFQ = \frac{\frac{y_1-1}{x_1}-\frac{y_2-1}{x_2}}{1+\frac{y_1-1}{x_1}\cdot\frac{y_2-1}{x_2}} = \frac{x_2\left(\frac{x_1^2}{4}-1\right)-x_1\left(\frac{x_2^2}{4}-1\right)}{x_1x_2+\left(\frac{x_1^2}{4}-1\right)\left(\frac{x_2^2}{4}-1\right)} = \frac{4(x_1-x_2)(x_1x_2+4)}{(x_1x_2+4)^2-4(x_1-x_2)^2}$$
        故
        $$\tan 2\angle PAQ = \tan\angle PFQ$$
        又 $0 < \angle PAQ < 90^\circ < \angle PFQ < 180^\circ$,则
        $$\angle PFQ = 2\angle PAQ$$
        \end{solution}
    \end{parts}

    \question 坐标平面上有一圆 $C: x^2 + y^2 = 20$,试回答下列问题:
    \begin{parts}
    \part 抛物线 $\Gamma_1$ 与圆 $C$ 相切于 $A(-\sqrt{10}, \sqrt{10})$ 和 $B(\sqrt{10}, \sqrt{10})$ 两点,求抛物线 $\Gamma_1$ 的方程式为何?
    \begin{solution}
        由切点可知抛物线关于 $y$ 轴对称且开口向下,设
        \[
        \Gamma_1: y = ax^2 + b,\quad a < 0
        \]
        将$A(-\sqrt{10}, \sqrt{10})$ 代入得$10a + b = \sqrt{10}$;将 $\Gamma_1$ 代入圆 $C$:
        \[
        x^2 + (ax^2 + b)^2 = 20 \Rightarrow a^2x^4 + (2ab + 1)x^2 + b^2 - 20 = 0
        \]
        令判别式为 $0$,
        \[
        (2ab + 1)^2 - 4a^2(b^2 - 20) = 0
        \]
        代入 $b = \sqrt{10} - 10a$ 得
        \[
        80a^2 + 4a(\sqrt{10} - 10a) + 1 = 0 \Rightarrow 40a^2 + 4\sqrt{10}a + 1 = 0
        \]
        解得$\displaystyle a = -\frac{\sqrt{10}}{20}, b = \frac{3\sqrt{10}}{2}$,所以抛物线方程为
        \[
        y = -\frac{\sqrt{10}}{20}x^2 + \frac{3\sqrt{10}}{2}
        \]
        \end{solution}
        \part 抛物线 $\Gamma_2$ 与圆 $C$ 相切于 $D(2,4)$ 和 $E(-4,2)$ 两点,求抛物线 $\Gamma_2$ 的顶点坐标为何?
        \begin{solution}
        考虑将 $\Gamma_1$ 旋转使其与 $\Gamma_2$ 对应,设旋转矩阵为
        \[
        T = 
        \begin{bmatrix}
        \cos \theta & -\sin \theta \\
        \sin \theta & \cos \theta
        \end{bmatrix}
        \]
        满足
        \[
        T
        \begin{bmatrix}
        \sqrt{10} \\
        \sqrt{10}
        \end{bmatrix}
        =
        \begin{bmatrix}
        2 \\
        4
        \end{bmatrix}
        \Rightarrow
        \cos \theta = \frac{3}{\sqrt{10}} ,\sin \theta = \frac{1}{\sqrt{10}}
        \]
        又$\Gamma_1$ 顶点为 $P(0, \frac{3\sqrt{10}}{2})$,则 $\Gamma_2$ 顶点为
        \[
        T(P) = 
        \begin{bmatrix}
        \cos \theta & -\sin \theta \\
        \sin \theta & \cos \theta
        \end{bmatrix}
        \begin{bmatrix}
        0 \\
        \frac{3\sqrt{10}}{2}
        \end{bmatrix}
        =
        \begin{bmatrix}
        -\frac{3}{2} \\
        \frac{9}{2}
        \end{bmatrix}
        \Rightarrow
        \Gamma_2 \text{ 顶点 } = \left(-\frac{3}{2}, \frac{9}{2} \right)
        \]
    \end{solution}
    \end{parts}

    \question 在平面直角坐标系$xOy$中,点$P(0,1)$,点$A$为动点,以线段$AP$为直径的圆与$x$轴相切,设$A$点的轨迹为曲线$E$.
    \begin{parts}
    \part 求轨迹$E$的方程.
    \begin{solution}
        设$A(x,y),AP$中点$(\frac{x}{2},\frac{y+1}{2})$,则由题意$$\sqrt{x^2+(y-1)^2}=2\left|\frac{y+1}{2}\right|$$
        化简即得轨迹$E$的方程式$x^2=4y$。
    \end{solution}
    \part 若直线$AP$与$E$交于另一点$B,\triangle AOB$的外接圆交$E$于点$C$(不与$O,A,B$重合),过点$C$作$E$的切线交直线$AP$于点$N$,求$|ON|$的最小值.
    \begin{solution}
        设$l_{AP}: y=kx+1,A(x_1, y_1), B(x_2, y_2), C(x_3,y_3)$,
        $$\begin{cases} y=kx+1 \\ x^2=4y \end{cases} \Rightarrow x^3-4kx-4=0$$
        其中$$x_1+x_2=4k, x_1x_2=-4$$
        又$O,A,B,C$四点共圆,设$\triangle AOB$外接圆方程为$x^2+y^2+dx+ey=0$,解得
        $$\begin{cases} x^2+y^2+dx+ey=0 \\ x^2=4y \end{cases} \Rightarrow x^3+(16+4e)x+16d=0$$
        其中$x_1,x_2,x_3$是方程$x^3+(4e+16)x+16d=0$的三个根,且$$x_1+x_2+x_3=0$$
        即$$x_3=-(x_1+x_2)=-4k$$代入$E$得$y_3=4k^2$得$C(-4k,4k^2)$,过$C$曲线$E$的切线方程为 $$2kx+y+4k^2=0$$
        与$l_{AP}: y=kx+1$联立得
        $$N\left(-\frac{4k^2+1}{3k}, -\frac{4k^2-2}{3}\right)$$
        故
        \begin{align*}
        |ON|^2
        &=\left(-\frac{4k^2+1}{3k}\right)^2+\left(-\frac{4k^2-2}{3}\right)^2 \\
        &= \frac{1}{9}\left(16k^4+12+\frac{1}{k^2}\right)\\
        &\ge \dfrac{1}{9}\left(3\sqrt[3]{16k^4 \cdot \dfrac{1}{2k^2} \cdot \dfrac{1}{2k^2}}+12\right) \\
        &=\dfrac{\sqrt[3]{4}+4}{3}
        \end{align*}
        当且仅当$16k^4 = \dfrac{1}{2k^2}\iff k^6 = \dfrac{1}{32}\iff k=\pm\left(\dfrac{1}{2}\right)^{\frac{5}{6}}$时,$$|ON|_{\min}=\sqrt{\frac{\sqrt[3]{4}+4}{3}}$$
    \end{solution}
    \end{parts}

    \question 已知正$\triangle ABC$ 内接于抛物线$$y=x^2-\frac{71}{36},$$点$P$满足$\overrightarrow{PA}+\overrightarrow{PB}+\overrightarrow{PC}=\textbf{0}$,是否存在定点$Q$,使得点$P$到点$Q$的距离与点$P$到$x$轴的距离相等?若存在,求出点$Q$的坐标;若不存在,请说明理由。
    \begin{solution}
        设$A(x_1,x_1^2-\frac{71}{36})$, $B(x_2,x_2^2-\frac{71}{36})$, $C(x_3,x_3^2-\frac{71}{36})$,易知$x_1,x_2,x_3$互不相等,
    
        取$BC$的中点为$D$,则点$D$的坐标为$$D(\frac{x_2+x_3}{2},\frac{x_2^2+x_3^2}{2}-\frac{71}{36})$$
        因为$\triangle ABC$为正三角形,所以$AD \perp BC$,由$\overrightarrow{AD}\cdot\overrightarrow{BC}=0$,得$$(\frac{x_2+x_3}{2}-x_1,\frac{x_2^2+x_3^2}{2}-x_1^2)\cdot(x_3-x_2,x_3^2-x_2^2)=0$$
        整理得
        \[
        x_2^3+x_3^3+x_2^2x_3+x_2x_3^2-2x_1^2x_2-2x_1^2x_3+x_2+x_3-2x_1=0 \tag{1} 
        \]
        同理得
        \[
        x_1^3+x_2^3+x_1^2x_2+x_1x_2^2-2x_3^2x_1-2x_3^2x_2+x_1+x_2-2x_3=0 \tag{2} 
        \]
        由$(1)-(2)$,得$x_3^3-x_1^3+x_2^2(x_3-x_1)+3x_2(x_3^2-x_1^2)+2x_1x_3(x_3-x_1)+3(x_3-x_1)=0$
        整理得\[
        x_1^2+x_2^2+x_3^2+3(x_1x_2+x_1x_3+x_2x_3)+3=0 \tag{3}
        \]
        设 $P(x_{0},y_{0})$ ,因为 $\overrightarrow{PA}+\overrightarrow{PB}+\overrightarrow{PC}=\textbf{0}$ ,所以 $$x_{0}=\frac{x_{1}+x_{2}+x_{3}}{3},y_{0}=\frac{x_{1}^{2}+x_{2}^{2}+x_{3}^{2}}{3}-\frac{71}{36}$$
        因为 $$(x_{1}+x_{2}+x_{3})^{2}=x_{1}^{2}+x_{2}^{2}+x_{3}^{2}+2x_{1}x_{2}+2x_{1}x_{3}+2x_{2}x_{3}=9x_{0}^{2}$$
        $$x_{1}^{2}+x_{2}^{2}+x_{3}^{2}=3y_{0}+\frac{71}{12}$$
        所以 $$x_{1}x_{2}+x_{1}x_{3}+x_{2}x_{3}=\frac{1}{2}(9x_{0}^{2}-3y_{0}-\frac{71}{12})$$
        由(3)得 $$3y_{0}+\frac{71}{12}+\frac{3}{2}(9x_{0}^{2}-3y_{0}-\frac{71}{12})+3=0$$
        整理得 $$x_{0}^{2}=\frac{1}{9}(y_{0}-\frac{1}{36})$$
        所以点 $P(x_{0},y_{0})$ 是抛物线上一点,其焦点为 $(0,\frac{1}{18})$ ,准线为$x$轴,
        由抛物线的定义得点 $Q$ 的坐标为 $$Q(0,\frac{1}{18}).$$
    \end{solution}

    \question 抛物线 $C$ 的直角坐标方程为
    \[
    y^2 = 4ax,
    \]
    其中 $a>0$。$P(ap^2,2ap),Q(aq^2,2aq)$ 为抛物线 $C$ 上的相异点。过 $P,Q$ 作抛物线的切线,两切线相交于点 $R$。设 $S$ 为抛物线的焦点,证明
    \[
    SR^2 = SP \cdot SQ
    \]
    \begin{solution}
        由$y^2 = 4ax$,求导得
        \[
        \frac{dy}{dx}=\frac{2a}{y} \Rightarrow \left.\frac{dy}{dx}\right|_{x=ap^2,y=2ap} =\frac{1}{p}
        \]
        同理,在$P(ap^2,2ap),Q(aq^2,2aq)$切线方程为
        \[
        y-2ap=\frac{1}{p}(x-ap^2),\quad y-2aq=\frac{1}{q}(x-aq^2)
        \]
        联立方程解得
        \[
        x=apq,y=a(p+q) \Rightarrow R(apq,a(p+q))
        \]
        抛物线 $y^2=4ax$ 的焦点为$S(a,0)$,于是
        \[
        SP=\sqrt{(ap^2-a)^2+(2ap)^2}=\sqrt{a^2(p^2+1)^2}=a(p^2+1)
        \]
        同理
        \[
        SQ=a(q^2+1)
        \]
        而
        \[
        SR^2=(apq-a)^2+(a(p+q))^2 =a^2(p^2q^2+p^2+q^2+1) =a^2(p^2+1)(q^2+1)
        \]
        于是
        \[
        SR^2=[a(p^2+1)][a(q^2+1)]=SP \cdot SQ
        \]
        证毕。
    \end{solution}

    \question 抛物线 $P$ 的焦点为 $S(6,0)$,准线为 $x=0$。
    \begin{parts}
    \part 证明 $P$ 的直角坐标方程为 $y^2 = 12(x-3)$。
    \begin{solution}
        由抛物线定义,
        \[
        \sqrt{(x-6)^2 + y^2} = x
        \]
        化简得
        \[
        y^2 = 12(x-3)
        \]
    \end{solution}
    \part 验证 $P$ 的参数方程为
    \[
    x = 3t^2+3, \quad y = 6t
    \]
    \begin{solution}
        将参数方程 $x = 3t^2+3, y=6t$ 代入 
        \[
        y^2 = 36t^2, \quad 12(x-3) = 36t^2 \Rightarrow y^2 = 12(x-3)
        \]
        验证成立。
    \end{solution}
    \part 证明过点 $Q(3q^2+3,6q)$ 的切线方程为
    \[
    qy + 3 = x + 3q^2
    \]
    \begin{solution}
        由链导法,
        \[
        \frac{dy}{dx} = \frac{\frac{dy}{dt}}{\frac{dx}{dt}} = \frac{6}{6t} = \frac{1}{t}
        \]
        过 $Q(3q^2+3,6q)$的切线方程为
        \[
        y - 6q = \frac{1}{q}(x - (3q^2+3)) \Rightarrow qy + 3 = x + 3q^2
        \]
    \end{solution}
    \part 设 $R$ 为抛物线上一点,使 $QSR$ 在一条直线上,证明过 $Q$ 和 $R$ 的切线交于 $y$ 轴。
    \begin{solution}
        直线 $QSR$ 的方程为
        \[
        y - 0 = \frac{6 - 0}{3q^2+3 - 6}(x-6) \Rightarrow y= \frac{2q}{q^2-1}(x-6).
        \]
        与抛物线 $x = 3t^2+3, y = 6t$ 联立得
        \[
        6t = \frac{6q(t^2-1)}{q^2-1} \Rightarrow t = -\frac{1}{q}
        \]
        其中点 $Q$ 对应$t=q$,于是$R\left(\dfrac{3}{q^2}+3, -\dfrac{6}{q}\right)$,过$R$的切线方程为
        \[
        -\frac{1}{q}y + 3 = x + \frac{3}{q^2}
        \]
        联立
        \[
        qy + 3 = x + 3q^2,\quad -\frac{1}{q}y + 3 = x + \frac{3}{q^2}
        \]
        可得交点为
        \[
        (0,\frac{3(q^2-1)}{q})
        \]
        所以切线交于 $y$ 轴,证毕。
    \end{solution}
    \end{parts}

    \question 已知抛物线的参数方程为
    \[
    x=\frac{1}{3}t^2,\quad y=\frac{2}{3}t,\quad t\in\mathbb R,
    \]
    曲线在抛物线上一点 $P$ 的法线与抛物线相交于异与$P$的点 $Q$。求 $PQ$ 的最小值。
    \begin{solution}
        由参数方程消去参数 $t$, 得
        \[
        y^2=\frac{4}{3}x
        \]
        于是过$P\left(\dfrac{p^2}{3}, \dfrac{2p}{3}\right)$的切线斜率为
        \[
        \frac{dy}{dx}=\frac{2}{3y} \left.\frac{dy}{dx}\right|_{t=p}= \frac{1}{p}
        \]
        故法线斜率为 $-p$,过$P$法线方程为
        \[
        y - \frac{2p}{3} = -p\left(x - \frac{p^2}{3}\right) \Rightarrow 3y + 3px = 2p + p^3
        \]
        与 $x = \frac{3}{4}y^2$ 联立得,
        \[
        3y + 3p\left(\frac{3}{4}y^2\right) = 2p + p^3 \Rightarrow (3y - 2p)(3py + 4 + 2p^2) = 0
        \]
        由于点 $P$对应$y = \dfrac{2p}{3}$,点 $Q$ 的纵坐标为
        \[
        y_Q = -\frac{4 + 2p^2}{3p}
        \]
        此时由$y = \frac{2}{3}t$知$Q$处的参数为
        \[
        q = -p - \frac{2}{p}
        \]
        故
        \begin{align*}
        |PQ|^2 &= \left(\frac{q^2}{3} - \frac{p^2}{3}\right)^2 + \left(\frac{2q}{3} - \frac{2p}{3}\right)^2 \\
        &= \frac{1}{9}(q - p)^2 [(q + p)^2 + 4] \\
        &= \frac{1}{9}\left(-2p - \frac{2}{p}\right)^2 \left[\left(-\frac{2}{p}\right)^2 + 4\right] \\
        &= \frac{16(p^2 + 1)^3}{9p^4}
        \end{align*}
        设 
        \[
        f(p) = \frac{(p^2 + 1)^3}{p^4},
        \]
        求导得
        \[
        f'(p) = \frac{2(p^2 + 1)^2(p^2 - 2)}{p^5}
        \]
        令 $f'(p) = 0$,得 $p^2 = 2$。此时 $|PQ|^2$ 取得最小值$\dfrac{16(2 + 1)^3}{9 \cdot 2^2} = 12$, 因此$PQ$ 的最小值为 $\sqrt{12} = 2\sqrt{3}$。
    \end{solution}
    
    \question 已知抛物线
    \[
    C: y=2x^{2},
    \]
    直线 $y=kx+2$ 交 $C$ 于 $A,B$ 两点,$M$ 为线段 $AB$ 的中点,过 $M$ 作 $x$ 轴的垂线,垂足为 $N$。
    \begin{parts}
    \part 证明:抛物线 $C$ 在点 $N$ 处的切线与 $AB$ 平行;
    \begin{solution}
        设$A(x_1,2x_1^2),B(x_2,2x_2^2)$,将 $y=kx+2$ 代入 $y=2x^2$,得
        \[
        2x^2-kx-2=0
        \]
        由韦达定理,
        \[
        x_1+x_2=\frac{k}{2},\quad x_1x_2=-1
        \]
        由于 $M$ 为 $AB$ 的中点,
        \[
        x_N=x_M=\frac{x_1+x_2}{2}=\frac{k}{4}
        \]
        故
        \[
        N\left(\frac{k}{4},0\right)
        \]
        设抛物线在点 $N$ 处的切线为
        \[
        y-\frac{k^2}{8}=m\left(x-\frac{k}{4}\right)
        \]
        将 $y=2x^2$ 代入上式,得
        \[
        2x^2-mx+\frac{mk}{4}-\frac{k^2}{8}=0
        \]
        由于该直线与抛物线相切,判别式为零:
        \[
        \Delta=m^2-8\left(\frac{mk}{4}-\frac{k^2}{8}\right)=(m-k)^2=0
        \]
        解得 $m=k$,即切线与 $AB$ 平行。
    \end{solution}
    \begin{solution}
        设$A(x_1,2x_1^2), B(x_2,2x_2^2)$,由
        \[
        2x^2-kx-2=0
        \]
        得
        \[
        x_1+x_2=\frac{k}{2},\quad x_1x_2=-1
        \]
        因为
        \[
        x_N=\frac{x_1+x_2}{2}=\frac{k}{4}
        \]
        又 $N$ 在抛物线上,故
        \[
        N\left(\frac{k}{4},\frac{k^2}{8}\right)
        \]
        抛物线 $y=2x^2$ 的导数为 $y'=4x$,因此在点 $N$ 处的切线斜率为
        \[
        4\cdot \frac{k}{4}=k
        \]
        而 $AB$ 的斜率为 $k$,故切线与 $AB$ 平行。
    \end{solution}
    \part 是否存在实数 $k$ 使 $\overrightarrow{NA}\cdot\overrightarrow{NB}=0$?若存在,求 $k$ 的值;若不存在,说明理由。
    \begin{solution}
        假设存在实数 $k$ 使 $\overrightarrow{NA}\cdot\overrightarrow{NB}=0$,则 $NA\perp NB$,由于 $M$ 是 $AB$ 的中点,故
        \[
        MN=\frac{1}{2}AB
        \]
        又
        \[
        y_M=\frac{1}{2}(y_1+y_2)
        =\frac{1}{2}(kx_1+2+kx_2+2)
        =\frac{1}{2}\left[k(x_1+x_2)+4\right]
        =\frac{k^2}{4}+2
        \]
        因为 $MN\perp x$ 轴,
        \[
        MN=|y_M-y_N|
        =\left|\frac{k^2}{4}+2-\frac{k^2}{8}\right|
        =\frac{k^2+16}{8}.
        \]
        而
        \[
        AB=\sqrt{1+k^2}\,|x_1-x_2|
        =\sqrt{1+k^2}\sqrt{(x_1+x_2)^2-4x_1x_2}
        =\frac{1}{2}\sqrt{k^2+10}\sqrt{k^2+16}.
        \]
        由 $MN=\dfrac{1}{2}AB$,解得
        \[
        \frac{k^2+16}{8}=\frac{1}{4}\sqrt{k^2+10}\sqrt{k^2+16} \Rightarrow k=\pm2
        \]
        因此存在 $k=\pm2$,使 $\overrightarrow{NA}\cdot\overrightarrow{NB}=0$。
    \end{solution}
    \begin{solution}
        假设存在实数 $k$ 使 $\overrightarrow{NA}\cdot\overrightarrow{NB}=0$。由
        \[
        \overrightarrow{NA}=\left(x_1-\frac{k}{4},\,2x_1^2-\frac{k^2}{8}\right),
        \quad
        \overrightarrow{NB}=\left(x_2-\frac{k}{4},\,2x_2^2-\frac{k^2}{8}\right),
        \]
        得
        \begin{align*}
        \overrightarrow{NA}\cdot\overrightarrow{NB}
        &=\left(x_1-\frac{k}{4}\right)\left(x_2-\frac{k}{4}\right)
        +\left(2x_1^2-\frac{k^2}{8}\right)\left(2x_2^2-\frac{k^2}{8}\right) \\
        &=\left(x_1x_2-\frac{k}{4}(x_1+x_2)+\frac{k^2}{16}\right)
        \left(1+4x_1x_2+k(x_1+x_2)+\frac{k^2}{4}\right)
        \end{align*}
        代入
        \[
        x_1+x_2=\frac{k}{2},\quad x_1x_2=-1,
        \]
        得
        \[
        \overrightarrow{NA}\cdot\overrightarrow{NB}
        =\left(-1-\frac{k^2}{16}\right)\left(-3+\frac{3}{4}k^2\right).
        \]
        由于 $-1-\dfrac{k^2}{16}\neq0$,故
        \[
        -3+\frac{3}{4}k^2=0 \Rightarrow k=\pm2.
        \]
        因此存在实数 $k=\pm2$,使 $\overrightarrow{NA}\cdot\overrightarrow{NB}=0$。
    \end{solution}
    \end{parts}

    \question 已知
    \[
    \frac{(x+1)^{2}}{25}+\frac{(y-1)^{2}}{9}=1
    \]
    求
    \[
    \sqrt{(x+5)^{2}+(y-1)^{2}}+\sqrt{(x-3)^{2}+(y-1)^{2}}
    \]
    的最小值。
    \begin{solution}
        发现到椭圆方程
        \[
        \dfrac{(x+1)^{2}}{25}+\dfrac{(y-1)^{2}}{9}=1
        \]
        焦点为$F_1(-5,1),F_2(3,1)$,由椭圆定义,椭圆上一点$P(x,y)$满足
        \[
        \sqrt{(x+5)^2 + (y-1)^2} + \sqrt{(x-3)^2 + (y-1)^2} =PF_1 + PF_2 = 2\cdot 5= 10
        \]
    \end{solution}

    \question 试求二次曲线
    \[
    13x^{2} - 10xy + 13y^{2} - 6x - 42y - 27 = 0
    \]
    的正焦弦长。
    \begin{solution}
        将二次项写成矩阵形式,
        \[
        13x^{2} - 10xy + 13y^{2} = 
        \begin{bmatrix} x & y \end{bmatrix}
        \begin{bmatrix} 13 & -5 \\ -5 & 13 \end{bmatrix}
        \begin{bmatrix} x \\ y \end{bmatrix}
        \]
        对角化得
        \[
        A = \begin{bmatrix} 13 & -5 \\ -5 & 13 \end{bmatrix} =
        \begin{bmatrix} \frac{\sqrt{2}}{2} & -\frac{\sqrt{2}}{2} \\[6pt] \frac{\sqrt{2}}{2} & \frac{\sqrt{2}}{2} \end{bmatrix}
        \begin{bmatrix} 8 & 0 \\ 0 & 18 \end{bmatrix}
        \begin{bmatrix} \frac{\sqrt{2}}{2} & \frac{\sqrt{2}}{2} \\[6pt] -\frac{\sqrt{2}}{2} & \frac{\sqrt{2}}{2} \end{bmatrix}
        \]
        对线性项变换,
        \[
        [-6, -42]
        \begin{bmatrix} \frac{\sqrt{2}}{2} & -\frac{\sqrt{2}}{2} \\[6pt] \frac{\sqrt{2}}{2} & \frac{\sqrt{2}}{2} \end{bmatrix}
        = [-24\sqrt{2}, -18\sqrt{2}]
        \]
        将方程化为标准形式:
        \[
        8x'^2 + 18y'^2 - 24\sqrt{2} x' - 18\sqrt{2} y' - 27 = 0
        \]
        即
        \[
        \frac{\left(x' - \frac{3}{\sqrt{2}}\right)^2}{9} + \frac{\left(y' - \frac{1}{\sqrt{2}}\right)^2}{4} = 1
        \]
        由此得$a = 3, b = 2,$正焦弦长为
        \[
        \frac{2b^2}{a} = \frac{8}{3}
        \]
    \end{solution}
    
    \question 设\(A,B\)为椭圆 \(\dfrac{x^{2}}{25}+\dfrac{y^{2}}{16}=1\) 两焦点 ,若$P$为椭圆上任意点,求 \(\triangle PAB\) 面积最大值。  
    \begin{solution}
        椭圆焦点$A(-3,0),\;B(3,0),$
        设 $P(x,y)$ 在椭圆上,  
        \[
        \overrightarrow{PA} = (-3 - x,\ -y),\quad \overrightarrow{PB} = (3 - x,\ -y)
        \]
        面积:
        \[
        S = \frac{1}{2} \left| \overrightarrow{PA} \times \overrightarrow{PB} \right| 
        = \frac{1}{2} |(-3 - x)(-y) - (3 - x)(-y)| = 3|y|
        \]
        故当$|y| = 4,\;\triangle PAB$面积为最大:$12$
    \end{solution}

    \question 椭圆 \(\dfrac{x^{2}}{9}+\dfrac{y^{2}}{4}=1\) 焦点为 \(F_1,F_2\),若椭圆上点 \(P\) 满足 \(|PF_1|:|PF_2|=2:1\),求 \(\triangle PF_1F_2\) 面积。 
    \begin{solution}
        椭圆焦点为 \(F_1 = (-\sqrt{5}, 0), F_2 = (\sqrt{5}, 0)\),由椭圆定义
        \[
        |PF_1| + |PF_2| = 3|PF_2| = 2a = 6 \Rightarrow |PF_1| = 4,\; |PF_2| = 2
        \]
        又$|F_1F_2|=2\sqrt5$,由海伦公式,
        \[
        s = \frac{4 + 2 + 2\sqrt{5}}{2} = 3 + \sqrt{5}
        \]
        \[
        S = \sqrt{ (3+\sqrt{5})(-1+\sqrt{5})(1+\sqrt{5})(3-\sqrt{5})} = 4
        \]
    \end{solution}

    \question 求椭圆 
    \[
    \Gamma: \frac{x^2}{9} + \frac{y^2}{4} = 1
    \] 
    上任一切线在第一象限被 $x$ 轴、$y$ 轴截出之线段长的最小值。
    \begin{solution}
        由$\Gamma$隐微分得,
        \[
        \frac{2x}{9}+\frac{yy'}{2}=0 \Rightarrow y'=-\frac{4x}{9y}
        \]
        设切点为 $P(3\cos\theta,2\sin\theta)$, 则 
        \[
        y'(3\cos\theta,2\sin\theta)=-\frac{2\cos\theta}{3\sin\theta}=-\frac{2}{3}\cot\theta,
        \]
        切线 
        \[
        L: y=-\frac{2}{3}\cot\theta(x-3\cos\theta)+2\sin\theta
        \] 
        与坐标轴的交点
        \[
        A\left(0,\frac{2}{\sin\theta}\right), \quad B\left(\frac{3}{\cos\theta},0\right)
        \]
        由柯西不等式,
        \[
        \left[\left(\frac{2}{\sin\theta}\right)^2+\left(\frac{3}{\cos\theta}\right)^2\right](\sin^2\theta+\cos^2\theta) \ge (2+3)^2
        \]
        故
        \[
        AB^2 = \frac{4}{\sin^2\theta} + \frac{9}{\cos^2\theta} \ge 25 \Rightarrow AB_{\min}=5
        \]
    \end{solution}

    \question 已知 $a>b>0,F$ 是 
    \[
    \Gamma:\frac{x^2}{a^2}+\frac{y^2}{b^2}=1
    \] 
    的一个焦点,$AB$ 是 $\Gamma$ 的弦且 $F$ 在 $AB$ 上。试证明
    \[
    \frac{1}{FA}+\frac{1}{FB}=\frac{2a}{b^2}
    \]
    \begin{solution}
        假设两焦点为 $F,F'$ 且$FA=p, FB=q$,则
        \[
        F'A=2a-p,\quad F'B=2a-q
        \]
        由余弦定理,
        \[
        \cos \angle AFF' = \frac{p^2+(2c)^2-(2a-p)^2}{2p(2c)}, \quad
        \cos \angle BFF' = \frac{q^2+(2c)^2-(2a-q)^2}{2q(2c)}
        \]
        由于 $\angle AFF' + \angle BFF' = 180^\circ$,
        \[
        \frac{p^2+(2c)^2-(2a-p)^2}{2p(2c)}
        = -\frac{q^2+(2c)^2-(2a-q)^2}{2q(2c)}
        \]
        可化简得
        \[
        \frac{1}{FA}+\frac{1}{FB}=\frac{2a}{b^2}
        \]
    \end{solution}

    \question 已知椭圆 $C_1: \frac{x^2}{4} + y^2 = 1$, 椭圆 $C_2: (x-2)^2 + 4y^2 = 1$, $C_1, C_2$ 的公切线与 $x$ 轴交于点 $A$, 则点 $A$ 的坐标为 $(4,0)$
    \begin{solution}
        \textcolor{red}{(待解)}
    \end{solution}

    \question 求椭圆 $E: x^2 + 16y^2 = 16$ 与圆 $C: x^2 + y^2 = 4$ 的公切线方程。
    \begin{solution}
        设切线斜率为 $m$,椭圆与圆
        \[
        E: \frac{x^2}{16} + y^2 = 1, \quad C: x^2 + y^2 = 4
        \]
        上斜率为 $m$ 的切线分别为
        \[
        y = mx \pm \sqrt{16m^2 + 1},\quad y = mx \pm 2\sqrt{1+m^2}
        \]
        联立解得
        \[
        \sqrt{16m^2 + 1} = 2\sqrt{1+m^2} \Rightarrow m = \pm \frac{1}{2}
        \]
        当 $m = \dfrac{1}{2}$,
        \[
        y = \frac{1}{2}x \pm \sqrt{5} \Rightarrow x - 2y + 2\sqrt{5} = 0, \quad x - 2y - 2\sqrt{5} = 0
        \]
        当 $m = -\frac{1}{2}$,
        \[
        y = -\frac{1}{2}x \pm \sqrt{5} \Rightarrow x + 2y - 2\sqrt{5} = 0, \quad x + 2y + 2\sqrt{5} = 0
        \]
        所以四条公切线方程为
        \[
        x - 2y + 2\sqrt{5} = 0, \quad x - 2y - 2\sqrt{5} = 0, \quad x + 2y - 2\sqrt{5} = 0, \quad x + 2y + 2\sqrt{5} = 0
        \]
    \end{solution}

    \question 在椭圆$\Omega$中,$F_1, F_2$为焦点,$A$为长轴的一个端点,$B$为短轴的一个端点,若$\angle F_1BF_2 = \angle FAB$,求$\Omega$的离心率。
    \begin{solution}
        因为$\angle F_1BF_2 = \angle FAB$,所以
        \[
        \triangle BF_1F_2 \backsim \triangle ABF_1\Rightarrow\angle BF_1F_2 = \angle ABF_1\Rightarrow|AB| = |AF_1|,
        \]
        即
        \[
        \sqrt{a^2+b^2}=a+c\Rightarrow2e^2+2e-1=0
        \]
        解得$e=\dfrac{\sqrt{3}-1}{2}$(负解舍去)。
    \end{solution}
    
    \question 已知椭圆 \(\dfrac{x^2}{a^2} + \dfrac{y^2}{b^2} = 1\) 的左焦点为 \(F\),右顶点为 \(A\),上顶点为 \(B\),离心率为 \(\dfrac{\sqrt{5}-1}{2}\),求 \(\angle ABF\)。  
    \begin{solution}
        设椭圆 \(\dfrac{x^2}{a^2} + \dfrac{y^2}{b^2} = 1\)离心率为\(e=\dfrac{\sqrt{5}-1}{2}\),左焦点为 \(F(-ae,0)\)、右顶点为 \(A(a,0)\)、上顶点为 \(B(0,b)\),中心$O(0,0)$,发现\[
        \tan \angle ABO=\frac{a}{b},\;\tan \angle OBF=\frac{ae}{b}\]于是\[
         \tan \angle ABF 
        = \tan (\angle ABO+ \angle OBF) 
        = \frac{\frac{a}{b}+\frac{ae}{b}}{1-\frac{a^2}{b^2}\cdot e} 
        \]
        又$\dfrac{a}{b}=\dfrac{1}{\sqrt{(1-e^2)}}$,代入有\[
        \tan \angle ABF = -\frac{1}{e^2-e+1}\sqrt{\frac{e+1}{e-1}}
        \]
        发现$e = \dfrac{\sqrt{5}-1}{2}\Rightarrow e^2+e-1=0$,由此得\[
        \tan \angle ABF =\infty \Rightarrow \angle ABF = 90^\circ 
        \]
    \end{solution}
    
    \question 设直线 
    \[
    \frac{x}{5}+\frac{y}{4}=1
    \]   
    与椭圆 
    \[
    \frac{x^2}{25}+\frac{y^2}{16}=1 
    \] 相交于两点 $A,B$,点 $P$ 在椭圆上使得\( \triangle PAB \) 的面积为 $9$,求点 $P$ 的坐标。
    \begin{solution}
        设点 $P$ 到直线 $AB$ 的距离为
        \[
        d = \frac{2[\triangle PAB]}{AB} = \frac{18}{AB}
        \]
        欲求AB,联立直线与椭圆:
        \[
        \begin{cases}
        \;\dfrac{x}{5}+\dfrac{y}{4}=1\\
        \;\dfrac{x^2}{25} + \dfrac{y^2}{16} = 1
        \end{cases}
        \]
        解得$A(0,4),B(5,0)$,则$d=\dfrac{18}{\sqrt{41}}$
        
        设过 $P$ 与直线 $AB$ 平行的直线为$4x + 5y = k$,有
        \[
        \frac{|k - 20|}{\sqrt{41}} = \frac{18}{\sqrt{41}} \Rightarrow k = 38 \text{ 或 } \,2
        \]
        若 $k = 38$,则      
        $\begin{cases}
        4x + 5y = 38 \\
        \dfrac{x^2}{25} + \dfrac{y^2}{16} = 1
        \end{cases}$无实数解;
        若 $k = 2$,可解得 $$P\left( \dfrac{1 \pm \sqrt{199}}{4},\ \dfrac{1 \mp \sqrt{199}}{5} \right)$$
    \end{solution}

    \question 已知圆$\Omega$与$x$轴、$y$轴均相切,圆心在椭圆$$\Gamma: \frac{x^2}{a^2} + \frac{y^2}{b^2} = 1 $$内,且$\Omega$与$\Gamma$有唯一的公共点$(8,9)$.求$\Gamma$的焦距。
    \begin{solution}
        设圆$\Omega$的圆心为$P(r,r)$,则有$$(8-r)^2+(9-r)^2=r^2\Rightarrow r=5\ \text{或} \ r=29$$
        因为$P$在$\Gamma$内,故$r=5$;椭圆$\Gamma$在点$A(8,9)$处的切线为$$l: \frac{8x}{a^2} + \frac{9y}{b^2} = 1$$
        且有$$m_l\cdot m_{PA}=-1 \Rightarrow \frac{32}{a^2}=\frac{27}{b^2}$$
        联立$$\frac{64}{a^2}+\frac{81}{b^2}=1$$解得$$a^2=160,b^2=135$$从而$\Gamma$的焦距为$$2\sqrt{a^2-b^2}=10$$
    \end{solution}

    \question 已知 $P$ 是椭圆 
    \[
    \frac{x^2}{a^2}+\frac{y^2}{b^2}=1
    \]
    上任意一点,$PN$ 是过点 $P$ 的法线,$F_1, F_2$ 为椭圆焦点。证明 $PN$ 平分 $\angle F_1PF_2$。
    \begin{solution}
        设椭圆上动点为 $P(a\cos\theta, b\sin\theta),\theta \in \mathbb{R}$,过$P$的切线斜率为
        \[
        \frac{dy}{dx} = -\frac{b^2 x}{a^2 y}
        \]
        则过$P$的法线斜率为
        \[
        m_{PN} = -\frac{1}{m_\text{tangent}} = \frac{a^2 y}{b^2 x} = \frac{a\sin\theta}{b\cos\theta}
        \]
        焦点坐标 $F_1(-ae,0)$, $F_2(ae,0)$,其中$e$为离心率,则
        \[
        m_{PF_1} = \frac{b\sin\theta - 0}{a\cos\theta + ae} = \frac{b\sin\theta}{a(\cos\theta + e)},\quad
        m_{PF_2} = \frac{b\sin\theta - 0}{a\cos\theta - ae} = \frac{b\sin\theta}{a(\cos\theta - e)}
        \]
        代入
        \[
        \tan\theta_1 = \frac{m_{PN} - m_{PF_1}}{1 + m_{PN} m_{PF_1}},\quad
        \tan\theta_2 = \frac{m_{PF_2} - m_{PN}}{1 + m_{PN} m_{PF_2}}
        \]
        整理得
        \[
        \tan\theta_1 = \tan\theta_2= \frac{ae\sin\theta}{b},
        \]
        由于角为锐角,$\theta_1 = \theta_2$,得证$PN$平分$\angle F_1PF_2$。
    \end{solution}

    \question 已知两椭圆 $E_1,E_2$ 的方程分别为  
    \[
    E_1: \frac{x^2}{a^2} + \frac{y^2}{b^2} - \frac{2x}{c} = 0, \quad
    E_2: \frac{x^2}{b^2} + \frac{y^2}{a^2} + \frac{2x}{c} = 0.
    \]  
    设两椭圆的公切线与椭圆$E_1,E_2$分别交于$A,B$,且$O$为原点,证明$\angle AOB$ 为直角。
    \begin{solution}
        对 $E_1,E_2$配方得
        \[
        E_1:\frac{\left(x - \frac{a^2}{c}\right)^2}{\frac{a^4}{c^2}} + \frac{y^2}{\frac{a^2 b^2}{c^2}} = 1, \quad \frac{\left(x - \frac{b^2}{c}\right)^2}{\frac{b^4}{c^2}} + \frac{y^2}{\frac{a^2 b^2}{c^2}} = 1.
        \]
        因此,$E_1$ 与 $E_2$ 的公切线为水平线,则
        \[
        A = \left(\frac{a^2}{c}, \pm \frac{c}{ab}\right), \quad B = \left(\frac{b^2}{c}, \pm \frac{c}{ab}\right),
        \]
        观察到
        \[
        m_{OA}\cdot m_{OB}  = \frac{\frac{c}{ab}}{\frac{a^2}{c}} \cdot \frac{\frac{c}{ab}}{\frac{b^2}{c}} = 1
        \]
        因此$OA \perp OB,\angle AOB$ 为直角。
    \end{solution}

    \question 已知$a,b>0$,
    \begin{parts}
    \part 试证过椭圆 
    \[
    \frac{x^{2}}{a^{2}}+\frac{y^{2}}{b^{2}}=1
    \] 
    上两点$P(a\cos\theta, b\sin\theta),Q(a\cos\phi, b\sin\phi)$的直线方程式为
    \[
    \frac{x}{a}\cos\frac{\theta+\phi}{2}+\frac{y}{b}\sin\frac{\theta+\phi}{2}=\cos\frac{\theta-\phi}{2}
    \]
    \begin{solution}
        $PQ$直线方程斜率为
        \[
        m = \frac{b\sin\phi - b\sin\theta}{a\cos\phi - a\cos\theta} 
        = \frac{b}{a} \frac{2\cos\frac{\theta+\phi}{2}\sin\frac{\phi-\theta}{2}}{-2\sin\frac{\theta+\phi}{2}\sin\frac{\phi-\theta}{2}}
        = -\frac{b}{a} \cot\frac{\theta+\phi}{2}.
        \]
        故$PQ$直线方程为
        \[
        y - b\sin\theta = -\frac{b}{a}\frac{\cos\frac{\theta+\phi}{2}}{\sin\frac{\theta+\phi}{2}}(x - a\cos\theta).
        \]
        整理得
        \[
        \frac{x}{a}\cos\frac{\theta+\phi}{2} + \frac{y}{b}\sin\frac{\theta+\phi}{2} = \cos\left(\theta - \frac{\theta+\phi}{2}\right)
        = \cos\frac{\theta-\phi}{2}.
        \]
    \end{solution}
    \part 若弦 $PQ$ 与圆$x^{2}+y^{2}=b^{2}$相切, 试证: 
    \[
    a^{2}\cos^{2}\frac{\theta-\phi}{2}=b^{2}\cos^{2}\frac{\theta+\phi}{2}+a^{2}\sin^{2}\frac{\theta+\phi}{2}
    \]
    \begin{solution}
        由于弦$PQ$到圆心 $(0,0)$ 的距离等于半径,
        \[
        \frac{|\cos\frac{\theta-\phi}{2}|}{\sqrt{\frac{\cos^2\frac{\theta+\phi}{2}}{a^2} + \frac{\sin^2\frac{\theta+\phi}{2}}{b^2}}} = b
        \]
        整理得
        \[
        a^{2}\cos^{2}\frac{\theta-\phi}{2}=b^{2}\cos^{2}\frac{\theta+\phi}{2}+a^{2}\sin^{2}\frac{\theta+\phi}{2}
        \]
    \end{solution}
    \part 若 $\sin(\theta-\phi)\ge 0$, 证明弦 $PQ$ 之长为 $a\sin(\theta-\phi)$。
    \begin{solution}
        有
        \begin{align*}
        PQ^2 &= (a\cos\theta - a\cos\phi)^2 + (b\sin\theta - b\sin\phi)^2 \\
        &= a^2\left(-2\sin\frac{\theta+\phi}{2}\sin\frac{\theta-\phi}{2}\right)^2 + b^2\left(2\cos\frac{\theta+\phi}{2}\sin\frac{\theta-\phi}{2}\right)^2 \\
        &= 4\sin^2\frac{\theta-\phi}{2} \left(a^2 \sin^2\frac{\theta+\phi}{2} + b^2 \cos^2\frac{\theta+\phi}{2}\right) \\
        &= 4\sin^2\frac{\theta-\phi}{2} \left(a^{2}\cos^{2}\frac{\theta-\phi}{2}\right) \\
        &= a^2 \left(2\sin\frac{\theta-\phi}{2}\cos\frac{\theta-\phi}{2}\right)^2 \\
        &= a^2 \sin^2(\theta-\phi)
        \end{align*}
        由于$a>0,\sin(\theta-\phi)\ge 0$,故
        \[
        PQ = a \sin(\theta-\phi)
        \]
    \end{solution}
    \end{parts}

    \question 设 $A,B,C$ 为椭圆 
    \[
    \Gamma: \frac{x^{2}}{16}+\frac{y^{2}}{32}=1
    \] 
    上三点,且 $\triangle ABC$ 的重心恰为此椭圆的中心,已知 $A(\sqrt{6}+\sqrt{2},2\sqrt{3}-2)$,求 $\triangle ABC$ 面积。
    \begin{solution}
        设变换:
        \[
        x = x', y = \sqrt{2}y'
        \]
        代入椭圆 $\Gamma$,得
        \[
        \frac{x'^2}{16} + \frac{y'^2}{16} = 1
        \]
        为一圆 $\Gamma'$,其中圆心为原点 $(0, 0)$,半径为 $4$,又因为
        \[
        x' = x, y' = \frac{y}{\sqrt{2}}
        \Rightarrow A'(\sqrt{6}+\sqrt{2}, \sqrt{6}-\sqrt{2})
        \Rightarrow OA' = 4
        \]
        由于 $\triangle A'B'C'$ 为正三角形,且其重心为原点,边长为
        \[
        A'B' = 4 \cdot \frac{3}{2} \cdot \frac{2}{\sqrt{3}} = 4\sqrt{3}
        \Rightarrow [\triangle A'B'C'] = \frac{\sqrt{3}}{4}(4\sqrt{3})^2 = 12\sqrt{3}
        \]
        换回原坐标系,雅可比行列式为
        \[
        J = \left| \frac{\partial(x, y)}{\partial(x', y')} \right| = 
        \begin{vmatrix}
        1 & 0\\
        0 & \sqrt{2}
        \end{vmatrix} = \sqrt{2}
        \Rightarrow [\triangle ABC] = J \cdot [\triangle A'B'C'] = 12\sqrt{6}
        \]
    \end{solution}

    \question 已知 $\triangle ABC$ 的顶点 $A,B$ 在椭圆
    \[
    \frac{x^2}{4}+\frac{y^2}{3}=1
    \]
    上,点 $C$ 在直线 $l:y=x+2$ 上,且 $AB \parallels l$。
    \begin{parts}
    \part 若 $AB$ 通过坐标原点 $O$ 时,求 $AB$ 的长及 $\triangle ABC$ 的面积;
    \begin{solution}
        设 $A,B$ 两点坐标分别为 $(x_1,y_1),(x_2,y_2)$,已知 $AB\parallels l$,且 $AB$ 经过原点 $O(0,0)$,
        故 $AB$ 直线方程为$y=x$,与椭圆方程
        \[
        \frac{x^2}{4}+\frac{y^2}{3}=1
        \]
        联立解得$x=\pm1$,于是
        \[
        AB=\sqrt{(x_1-x_2)^2+(y_1-y_2)^2}
        =\sqrt{2}\,|x_1-x_2|=2\sqrt{2}
        \]
        又 $AB$ 上的高等于原点到直线 $l:y=x+2$ 的距离,
        故
        \[
        [\triangle ABC]=\frac{1}{2}\cdot 2\sqrt{2}\cdot \frac{|2|}{\sqrt{2}}=2
        \]
    \end{solution}
    \part 若 $\angle ABC=90^\circ$ 且斜边 $AC$ 的长最大时,求 $AB$ 所在直线的方程。
    \begin{solution}
        设 $AB$ 所在直线的方程为$y = x + c $,将其代入方程 $x^2 + 3y^2 = 4$,整理得
        \[ 
        4x^2 + 6cx + 3c^2 - 4 = 0 
        \]
        因为 $A, B$ 在椭圆上,故判别式满足
        \[ 
        \Delta = (6c)^2 - 4 \cdot 4 \cdot (3c^2 - 4) = 64 - 12c^2 > 0 \Rightarrow c^2 < \frac{16}{3}
        \]
        设 $A, B$ 两点的坐标分别为 $(x_1, y_1), (x_2, y_2)$,由韦达定理得
        \[ 
        x_1 + x_2 = -\frac{3c}{2}, \quad x_1x_2 = \frac{3c^2 - 4}{4} 
        \]
        于是线段 $AB$ 的长度平方为
        \[ 
        AB^2 = 2(x_1 - x_2)^2 = 2\left[(x_1 + x_2)^2 - 4x_1x_2\right] = \frac{32 - 3c^2}{2} 
        \]
        点 $B$ 到直线 $l: y = x + 2$ 的距离 $|BC|$ 为
        \[ 
        |BC| = \frac{|2 - c|}{\sqrt{2}}
        \]
        因为 $\angle ABC = 90^\circ$,由毕氏定理,
        \[ 
        AC^2 = AB^2 + BC^2 = \frac{32 - 3c^2}{2}  + \frac{(2 - c)^2}{2} = -(c + 1)^2 + 19
        \]
        当 $c = -1$ 时,$AC^2$ 取得最大值 $19$,此时 $c^2 = 1 < \frac{16}{3}$,符合判别式条件,因此$AB$ 所在直线的方程为:
        \[ 
        y = x - 1 
        \]
    \end{solution}
    \end{parts}

    \question 设椭圆
    \[
    \frac{x^{2}}{a^{2}}+\frac{y^{2}}{b^{2}}=1\ (a>b>0)
    \]
    的左右焦点分别为 $F_{1},F_{2}$,离心率
    \[
    e=\frac{\sqrt{2}}{2},
    \]
    点 $F_{2}$ 到右准线 $l$ 的距离为 $\sqrt{2}$。
    \begin{parts}
    \part 求 $a,b$ 的值;
    \begin{solution}
        由离心率定义
        \[
        e=\frac{c}{a}=\frac{\sqrt{2}}{2}
        \]
        又点 $F_{2}$ 到右准线 $l$ 的距离为
        \[
        d=\frac{a}{e}-c=\sqrt{2}
        \]
        联立解得
        \[
        a=2,c=\sqrt{2}
        \]
        又
        \[
        b^{2}=\sqrt{a^{2}-c^{2}}=\sqrt{2}
        \]
    \end{solution}
    \part 设 $M,N$ 是 $l$ 上的两个动点,且
    \[
    \overrightarrow{F_{1}M}\cdot\overrightarrow{F_{2}N}=0,
    \]
    证明当 $MN$ 取最小值时,
    \[
    \overrightarrow{F_{2}F_{1}}+\overrightarrow{F_{2}M}+\overrightarrow{F_{2}N}=\vec{0}.
    \]
    \begin{solution}
        由 $c=\sqrt{2},a=2$可得$F_{1}(-\sqrt{2},0), F_{2}(\sqrt{2},0)$,右准线 $l$ 的方程为
        \[
        x=2\sqrt{2}
        \]
        设$M(2\sqrt{2},y_{1}), N(2\sqrt{2},y_{2})$,由$\overrightarrow{F_{1}M}\cdot\overrightarrow{F_{2}N}=0$得
        \[
        (2\sqrt{2}+\sqrt{2},y_{1})\cdot(2\sqrt{2}-\sqrt{2},y_{2})=0 \Rightarrow y_1y_2=-6 \Rightarrow y_{2}=-\frac{6}{y_{1}}
        \]
        于是由AM-GM不等式,
        \[
        MN=|y_{1}-y_{2}|=\left|y_{1}+\frac{6}{y_{1}}\right|\ge 2\sqrt{6}
        \]
        当且仅当 $y_{1}=\pm\sqrt{6}$ 时取等号,即 $y_{2}=-y_{1}$,此时
        \[
        \overrightarrow{F_{2}F_{1}}+\overrightarrow{F_{2}M}+\overrightarrow{F_{2}N}
        =(2\sqrt{2},0)+(-\sqrt{2},y_{1})+(-\sqrt{2},y_{2})
        =(0,y_{1}+y_{2})
        =\vec{0}.
        \]
        证毕。
    \end{solution}
    \end{parts}

    \question 已知抛物线 
    \[
    E: x^{2} = 2 p y \;(p > 0)
    \]
    的焦点为 $F$,点 $P(m,n)\; (m \neq 0)$ 在 $E$ 上,过点 $P$ 作抛物线的切线 $l$,直线 $l$ 与椭圆 
    \[
    C: \dfrac{x^{2}}{a^{2}} + \dfrac{y^{2}}{b^{2}} = 1\; (a > b > 0)
    \]
    交于两不同点 $A, B$,记线段 $AB$ 的中点为 $D$。设$O$ 为原点,直线 $OD$与直线 $y = -\dfrac{p b^{2}}{a^{2}}$ 交于点 $Q$。
    \begin{parts}
    \part 证明直线 $y = -\dfrac{p b^{2}}{a^{2}}$ 与直线 $PQ$ 垂直。
    \begin{solution}
        点 $P$ 在抛物线上,故 $P(m,\dfrac{m^{2}}{2 p})$,抛物线 $E: x^{2} = 2 p y$求导得切线$l$斜率$\dfrac{m}{p}$,故过点 $P$ 作抛物线的切线 $l$方程为
        \[
        y-\frac{m^{2}}{2 p}=\dfrac{m}{p}(x-m)\Rightarrow y = \frac{m}{p} x - \frac{m^{2}}{2 p}.
        \]
        与椭圆方程
        \[
        \frac{x^{2}}{a^{2}} + \frac{y^{2}}{b^{2}} = 1,
        \]
        联立得化为关于 $x$ 的二次方程
        \[
        4(a^{2} m^{2} + b^{2} p^{2}) x^{2} - 4 a^{2} m^{3} x + a^{2} (m^{4} - 4 p^{2} b^{2}) = 0.
        \]
        记两根为 $x_{1}, x_{2}$,则
        \[
        x_{1} + x_{2} = \frac{a^{2} m^{3}}{a^{2} m^{2} + b^{2} p^{2}}, \quad
        y_{1} + y_{2} = \frac{m}{p} (x_{1} + x_{2}) - \frac{m^{2}}{p} = - \frac{m^{2} b^{2} p}{a^{2} m^{2} + b^{2} p^{2}}.
        \]
        中点 $D$ 坐标为
        \[
        D \left( \frac{a^{2} m^{3}}{2 (a^{2} m^{2} + b^{2} p^{2})}, \, -\frac{m^{2} b^{2} p}{2 (a^{2} m^{2} + b^{2} p^{2})} \right)
        \]
        故直线 $OD$ 方程为
        \[
        y=-\frac{b^{2} p}{a^{2} m}x
        \]
        与 $y = -\dfrac{p b^{2}}{a^{2}}$ 联立得点$Q$的方程
        \[
        x = m
        \]
        故 $PQ$ 是垂直于 $x$ 轴的直线,而直线 $y = -\dfrac{p b^{2}}{a^{2}}$ 是水平线,二者垂直。
    \end{solution}
    \part 设 $a = 2 b$,直线 $l$ 与 $F Q$ 交于 $G$,证明
    \[
    1 < \frac{[\triangle F P G]}{[\triangle P G Q]} < 2
    \]
    \begin{solution}
        当 $a = 2 b$,直线 $y = -\dfrac{p b^{2}}{a^{2}}$ 即
        \[
        y = -\frac{p}{4},
        \]
        为介于抛物线准线 $y = -\dfrac{p}{2}$ 和 $x$ 轴之间的平行线。延长 $PQ$ 与抛物线准线交于点 $H \left(m, -\dfrac{p}{2}\right)$,焦点 $F \left(0, \dfrac{p}{2}\right)$,则直线 $HF$ 斜率
        \[
        k_{HF} = \frac{\frac{p}{2} + \frac{p}{2}}{0 - m} = -\frac{p}{m}
        \]
        切线 $l$ 斜率为
        \[
        k = \frac{m}{p},
        \]
        故 $l \perp HF$。由抛物线定义
        \[
        PF = PH = \frac{m^{2}}{2 p} + \frac{p}{2}, \quad
        PQ = PH - \frac{p}{4} = \frac{m^{2}}{2 p} + \frac{p}{4}
        \]
        $\triangle PFH$ 为等腰三角形,且切线 $l$ 平分 $\angle F P H$,由角平分线性质,
        \[
        \frac{[\triangle F P G]}{[\triangle P G Q]} = \frac{F G}{G Q} = \frac{P F}{P Q} = \frac{2 m^{2} + 2 p^{2}}{2 m^{2} + p^{2}}
        \]
        显然,
        \[
        1 < \frac{[\triangle F P G]}{[\triangle P G Q]} < 2
        \]
    \end{solution}
    \end{parts}

    \question 小何在打篮球时发现地面上的篮球在灯光照射下在地面形成的影子是一个椭圆, 并观察到地面与篮球接触点正是椭圆的焦点。如图,已知篮球半径为 $1$,灯源 $P$ 离地高度为 $4$,其正下方为点 $A$, 地面上椭圆右顶点与点 $A$ 的水平距离为 $3$,求该椭圆的离心率 $e$。
    \begin{figure}[H]
        \centering        
        \includegraphics[width=0.3\textwidth]{images/image16.png}
    \end{figure}
    \ifprintanswers
    \begin{figure}[H]
        \centering        
        \includegraphics[width=0.3\textwidth]{images/image17.png}
    \end{figure}
    \fi
    \begin{solution}
        以$A$为原点建立平面直角坐标系,则$P(0,4)$, $R(-3,0)$, 直线$PR$的方程为 $$y=\frac{4}{3}x+4$$
        设$M(n,1)$, $Q(n,0)$,由$M$到直线$PR$的距离为1,$$\left|\frac{\frac{4}{3}n+4-1}{\sqrt{1+(\frac{4}{3})^{2}}}\right|=1$$
        解之得 $n=-\frac{7}{2}$\,(舍 $n=-1$),则 $M(-\frac{7}{2},1)$, $Q(-\frac{7}{2},0)$。
        
        设直线$PN$的方程为 $y=kx+4$,由M到直线$PN$的距离为1,$$\left|\frac{-\frac{7}{2}k+4-1}{\sqrt{1+k^{2}}}\right|=1$$
        整理得 $$45k^{2}-21k+8=0\Rightarrow k_{PN}=\dfrac{8}{15}\;(\text{舍}\,k_{PR}=\dfrac{4}{3}\,)$$ 
        故直线$PN$的方程为 $y=\dfrac{8}{15}x+4$,得$N(-\dfrac{15}{2},0)$,故
        $$\begin{cases} NQ=-\frac{7}{2}+\frac{15}{2}=4=a+c \\ RQ=-3+\frac{7}{2}=\frac{1}{2}=a-c \end{cases}\Rightarrow a=\frac{9}{4},\,c=\frac{7}{4}$$ 
        所以 $$e=\dfrac{c}{a}=\frac{7}{9}$$
    \end{solution}
    
    \question 椭圆焦点为 \(F_1,F_2\),点 \(P\) 在椭圆上。设 \(O\) 为 \(\triangle PF_1F_2\) 外接圆心,且
        \(\overrightarrow{PO}\cdot\overrightarrow{F_1F_2}=2\,\overrightarrow{PF_1}\cdot\overrightarrow{PF_2}\)。求椭圆最小离心率。  
    \begin{solution}
        设$F_1(-c,0),\ F_2(c,0),\ P(x,y)$,椭圆方程为
        \[
        \frac{x^2}{a^2} + \frac{y^2}{b^2} = 1
        \]
        由于 \(O\) 是外接圆心,且 \(F_1F_2\) 是弦,故 \(O\) 在 \(y\) 轴上,设$O(0,y_O)$,由$|OP| = |OF_1| = |OF_2|,$
        \[
        x^2 + (y - y_O)^2 = c^2 + y_O^2 \Rightarrow x^2 + y^2 - 2 y y_O = c^2 \tag{1}
        \]
        由条件$\overrightarrow{PO}\cdot\overrightarrow{F_1F_2}=2\,\overrightarrow{PF_1}\cdot\overrightarrow{PF_2}$得到
        \[
        (-x, y_O - y)\cdot(2c,0)=(-c - x, -y)\cdot(c - x, -y)\Rightarrow -c x = x^2 - c^2 + y^2 \tag{2}
        \]
        由$(1),(2)$得
        \[
        2 y y_O = - c x\Rightarrow y_O = -\frac{c x}{2 y}\;(y \neq 0)
        \]
        代回$(1)$得
        \[
        x^2 + y^2 + c x = c^2
        \]
        又由椭圆方程
        \[
        y^2 = b^2 \left(1 - \frac{x^2}{a^2}\right) = (a^2 - c^2) \left(1 - \frac{x^2}{a^2}\right).
        \]
        代入得
        \[
        x^2 + (a^2 - c^2)\left(1 - \frac{x^2}{a^2}\right) + c x = c^2
        \]
        整理为
        \[
        \frac{c^2}{a^2} x^2 + c x + a^2 - 2 c^2 = 0
        \]
        此为关于 \(x\) 的二次方程,存在实数解则判别式 \(\Delta \geq 0\):
        \[
        \Delta = c^2 - 4 \frac{c^2}{a^2} (a^2 - 2 c^2) \geq 0 \Rightarrow \frac{c^2}{a^2} \geq \frac{3}{8}
        \]
        故最小离心率为
        \[
        e = \frac{c}{a} \geq \sqrt{\frac{3}{8}} = \frac{\sqrt{6}}{4}
        \]
    \end{solution}

    \question 在平面直角坐标系中,两点到点 $(0, -\sqrt{3}),(0, \sqrt{3})$ 的距离之和等于 4,设点 $P$ 的轨迹为 $C$。
    \begin{parts}
    \part 写出 $C$ 的方程;
    \begin{solution}
        据题意,所求轨迹方程式为
        \[
        \sqrt{x^2+(y+\sqrt{3})^2}+\sqrt{x^2+(y-\sqrt{3})^2}=4
        \]
        化简得
        \[
        x^2 + \frac{y^2}{4} = 1
        \]
    \end{solution}
    \part 设直线 $y = kx + 1$ 与 $C$ 交于 $A, B$ 两点,求 $k$ 使得 $OA \perp OB$,并求此时 $AB$ 的长。
    \begin{solution}
        设 $A(x_1, y_1)$, $B(x_2, y_2)$, 满足
        \[
        x^2 + \frac{y^2}{4} = 1,\quad y = kx + 1
        \]
        联立得
        \[
        (k^2 + 4)x^2 + 2kx - 3 = 0
        \]
        由韦达定理,
        \[
        x_1 + x_2 = -\frac{2k}{k^2 + 4},\quad x_1 x_2 = -\frac{3}{k^2 + 4}.
        \]
        由于$OA \perp OB$ 等价于 $x_1 x_2 + y_1 y_2 = 0$,而 
        \[
        y_1 y_2 = (kx_1+1)(kx_2+1) = k^2 x_1 x_2 + k(x_1+x_2) + 1,
        \]
        所以
        \[
        x_1x_2 + y_1y_2 = (k^2+1)x_1 x_2 + k(x_1+x_2) + 1 = \frac{-4k^2 + 1}{k^2 + 4}
        \]
        所以当 $k = \pm \dfrac{1}{2}$时,有$x_1 x_2 + y_1 y_2 = 0$即$OA \perp OB$。且当 $k = \pm \dfrac{1}{2}$,
        \[
        x_1 + x_2 = \mp \frac{4}{17},\quad x_1 x_2 = -\frac{12}{17},
        \]
        于是
        \begin{align*}
        AB &= \sqrt{(x_2-x_1)^2 + (y_2-y_1)^2} \\
        &= \sqrt{(1+k^2)(x_2-x_1)^2} \\
        &= \sqrt{(1+k^2)((x_2+x_1)^2-4x_1x_2)} \\
        &= \sqrt{\frac{5}{4}\cdot\left(\frac{16}{289} + \frac{48}{17}\right)} \\
        &= \frac{8\sqrt{13}}{17}
        \end{align*}   
    \end{solution}
    \end{parts}

    \question 设椭圆中心在坐标原点,$A(2,0),B(0,1)$ 是它的两个顶点,直线 $y=kx,k>0$ 与 $AB$ 相交于点 $D$,与椭圆相交于 $E,F$ 两点。
    \begin{parts}
    \part 若 $\overrightarrow{ED}=6\overrightarrow{DF}$,求 $k$ 的值;
    \begin{solution}
        由题意,椭圆方程为
        \[
        \frac{x^2}{4}+y^2=1.
        \]
        直线 $AB,EF$ 的方程分别为
        \[
        x+2y=2,\quad y=kx\ ,k>0
        \]
        设 $D(x_0,kx_0),E(x_1,kx_1),F(x_2,kx_2),x_1<x_2$,联立椭圆与$y=kx$,解得
        \[
        \frac{x^2}{4}+k^2x^2=1 \Rightarrow x_2=-x_1=\frac{2}{\sqrt{1+4k^2}} 
        \]
        由 $\overrightarrow{ED}=6\overrightarrow{DF}$,得$x_0-x_1=6(x_2-x_0)$,从而
        \[
        x_0=\frac{1}{7}(6x_2+x_1)=\frac{5}{7}x_2=\frac{10}{7\sqrt{1+4k^2}} \tag{1}
        \]
        又因 $D$ 在 $AB$ 上,
        \[
        x_0+2kx_0=2 \Rightarrow x_0=\frac{2}{1+2k} \tag{2}
        \]
        联立$(1),(2)$,解得
        \[
        24k^2-25k+6=0 \Rightarrow k=\frac{2}{3}\ \text{或}\ k=\frac{3}{8}
        \]
    \end{solution}
    \part 求四边形 $AEBF$ 面积的最大值。
    \begin{solution}
        点 $E,F$ 到直线 $AB$ 的距离分别为
        \[
        h_1=\frac{|x_1+2kx_1-2|}{\sqrt{5}}
        =\frac{2(1+2k-\sqrt{1+4k^2})}{\sqrt{5(1+4k^2)}},
        \]
        \[
        h_2=\frac{|x_2+2kx_2-2|}{\sqrt{5}}
        =\frac{2(1+2k+\sqrt{1+4k^2})}{\sqrt{5(1+4k^2)}}.
        \]
        又$AB=\sqrt{2^2+1^2}=\sqrt{5}$,四边形 $AEBF$ 的面积为
        \[
        [AEBF]=\frac{1}{2} AB (h_1+h_2)
        =\frac{1}{2}\cdot \sqrt{5} \cdot \frac{4(1+2k)}{\sqrt{5(1+4k^2)}}
        =\frac{2(1+2k)}{\sqrt{1+4k^2}}
        =2\sqrt{1+\frac{4k}{1+4k^2}}
        \]
        由AM-GM不等式,
        \[
        \frac{1+4k^2}{4k}=\frac{1}{4k}+k \ge 1
        \]
        当 $k=\dfrac{1}{2}$ 时等号成立,此时四边形 $AEBF$ 面积取最大值$2\sqrt{2}$。
    \end{solution}
    \begin{solution}
        已知 $BO=1, AO=2$,设 $y_1 = kx_1, y_2 = kx_2$,由$x_2=\dfrac{2}{\sqrt{1+4k^2}}$ 知 $x_2 > 0, y_2 = -y_1 > 0$,故四边形 $AEBF$ 面积为
        \[ 
        [AEBF] = [\triangle BEF] + [\triangle AEF] = x_2 + 2y_2 = \sqrt{(x_2 + 2y_2)^2}
        \]
        由柯西不等式,
        \[ 
        [AEBF] \le \sqrt{2(x_2^2 + 4y_2^2)} = 2\sqrt{2} 
        \]
        当 $x_2 = 2y_2$ 时,四边形 $AEBF$ 面积取得最大值 $2\sqrt{2}$。
    \end{solution}
    \end{parts}

    \question 已知椭圆 
    \[
    C: \frac{x^2}{a^2} + \frac{y^2}{b^2} = 1,\quad a>b>0,
    \]
    的一个焦点为 $F(1,0)$,且椭圆过点 $(2,0)$。
    \begin{parts}
    \part 求椭圆$C$的方程。
    \begin{solution}
        由题意,椭圆的半长轴 $a=2$,焦距 $c=1$,则$b^2 = a^2 - c^2 = 3$,因此椭圆方程为
        \[
        \frac{x^2}{4} + \frac{y^2}{3} = 1
        \]
    \end{solution}
    \part 若 $AB$ 为垂直于 $x$ 轴的弦,直线 $l: x=4$ 与 $x$ 轴交于点 $N$,直线 $AF$ 与 $BN$ 交于点 $M$,试证点 $M$ 恒在椭圆 $C$ 上。
    \begin{solution}
        设 $A(m,n)$,则 $B(m,-n),n\ne 0$,且满足椭圆方程
        \[
        \frac{m^2}{4} + \frac{n^2}{3} = 1 \Rightarrow 4n^2=12-3m^2
        \]
        直线 $AF,BN$ 方程分别为
        \[
        n(x-1) - (m-1)y = 0, \quad n(x-4) - (m-4)y = 0 
        \]
        联立解得交点 $M(x_0,y_0)$
        \[
        x_0 = \frac{5m-8}{2m-5},\quad y_0 = \frac{3n}{2m-5}.
        \]
        由于
        \[
        \frac{x_0^2}{4} + \frac{y_0^2}{3} 
        = \frac{(5m-8)^2}{4(2m-5)^2} + \frac{3n^2}{(2m-5)^2}
        = \frac{(5m-8)^2 + 3(12-3m^2)}{4(2m-5)^2} 
        = 1
        \]
        因此点 $M$ 恒在椭圆 $C$ 上。
    \end{solution}
    \begin{solution}
        由题意得 $F(1, 0), N(4, 0)$,设 $A(m, n)$,则 $B(m, -n),n \neq 0$,满足椭圆方程
        \[
        \frac{m^2}{4} + \frac{n^2}{3} = 1 \tag{1}
        \]
        直线 $AF$ 与 $BN$ 的方程分别为
        \[
        n(x-1) - (m-1)y = 0, \quad n(x-4) - (m-4)y = 0 \tag{2}
        \]
        联立可得当$x \ne \dfrac{5}{2}$时,
        \[
        m = \frac{5x-8}{2x-5},\quad n = \frac{3y}{2x-5}
        \]
        代入 (1),整理得
        \[
        \frac{x^2}{4} + \frac{y^2}{3} = 1, \quad y \ne 0.
        \]
        而当 $x = \dfrac{5}{2}$ 时,由 (2) 得
        \[
        \begin{cases}
        \dfrac{3}{2} n - (m-1)y = 0 \\[6pt]
        -\dfrac{3}{2} n + (m+4)y = 0
        \end{cases}
        \Rightarrow n=0,\, y=0,
        \]
        与 $n \ne 0$ 矛盾,因此点 $M$ 的轨迹方程为
        \[
        \frac{x^2}{4} + \frac{y^2}{3} = 1, \quad y \ne 0,
        \]
        即点 $M$ 恒在椭圆 $C$ 上。
    \end{solution}
    \part 求 $\triangle AMN$ 面积的最大值。
    \begin{solution}
        设 $AM$ 的方程为 $x = t y + 1$,代入椭圆方程,化简得
        \[
        (3t^2+4)y^2 + 6t y - 9 = 0
        \]
        设 $A(x_1,y_1),M(x_2,y_2)$,由韦达定理,
        \[
        y_1 + y_2 = -\frac{6t}{3t^2+4},\quad y_1y_2 = -\frac{9}{3t^2+4}.
        \]
        于是
        \[
        |y_1 - y_2| = \sqrt{(y_1+y_2)^2 - 4y_1y_2} = \frac{4\sqrt{3}\sqrt{3t^2+3}}{3t^2+4}.
        \]
        令 $3t^2+4 = \lambda \ge 4$,得
        \[
        |y_1-y_2| = 4\sqrt{3} \sqrt{\frac{\lambda-1}{\lambda^2}}
        = 4\sqrt{3} \sqrt{-\left(\frac{1}{\lambda}-\frac{1}{2}\right)^2 + \frac{1}{4}}.
        \]
        因为$\lambda \ge 4,0<\dfrac{1}{\lambda}\le\dfrac{1}{4}$,所以当$\dfrac{1}{\lambda}=\dfrac{1}{4}$即 $\lambda=4,t=0$ 时,$|y_1-y_2|$ 取得最大值 $3$,此时
        \[
        [\triangle AMN] = \frac{1}{2}FN |y_1-y_2| = \frac{3}{2}|4-1| = \frac{9}{2}
        \]
    \end{solution}
    \end{parts}

    \question 设椭圆
    \[
    C: \frac{x^2}{a^2}+\frac{y^2}{b^2}=1,\quad a>b>0,
    \]
    其相应于焦点 $F(2,0)$ 的准线方程为 $x=4$。
    \begin{parts}
    \part 求椭圆 $C$ 的方程。
    \begin{solution}
        由题意得,
        \[
        \begin{cases} 
        c = 2 \\[3pt]
        \dfrac{a^2}{c} = 4 \\[3pt]
        a^2 = b^2 + c^2 
        \end{cases}
        \implies a^2 = 8,\; b^2 = 4 
        \]
        因此,椭圆 $C$ 的方程为 
        \[
        \frac{x^2}{8} + \frac{y^2}{4} = 1
        \]
    \end{solution}
    \part 已知过点 $F_1(-2,0)$、倾斜角为 $\theta$ 的直线交椭圆 $C$ 于 $A,B$ 两点,证明
    \[
    AB=\frac{4\sqrt{2}}{2-\cos^2\theta};
    \]
    \begin{solution}
        椭圆 $C$ 的左焦点为$F_1(-2,0)$,离心率$e=\dfrac{\sqrt{2}}{2}$,左准线方程为 $l:x=-4$。
        
        设直线 $AB$ 的倾斜角为 $\theta$,作 $AA_1\perp l$于$A_1,BB_1\perp l$于$B_1$,且$l$于$x$轴交于点$H$,由于 $A$ 在椭圆上,
        \[
        AF_1=\frac{\sqrt{2}}{2}AA_1=\frac{\sqrt{2}}{2}(F_1H+AF_1\cos\theta)=\sqrt{2}+\frac{\sqrt{2}}{2}AF_1\cos\theta
        \]
        可得
        \[
        AF_1=\frac{2}{\sqrt{2}-\cos\theta}
        \]
        同理
        \[
        BF_1=\frac{2}{\sqrt{2}+\cos\theta}
        \]
        因此
        \[
        AB=AF_1+BF_1=\frac{2}{\sqrt{2}-\cos\theta}+\frac{2}{\sqrt{2}+\cos\theta}=\frac{4\sqrt{2}}{2-\cos^2\theta}
        \]
    \end{solution}
    \begin{solution}
        当 $\theta \neq \dfrac{\pi}{2}$ 时,记 $k = \tan \theta$,则直线 $AB$ 的方程为$y = k(x+2)$,将其代入椭圆方程 $x^2 + 2y^2 = 8$,整理得
        \[ 
        (1+2k^2)x^2 + 8k^2x + 8(k^2-1) = 0 
        \]
        设 $A(x_1, y_1), B(x_2, y_2)$,由韦达定理得
        \[ 
        x_1 + x_2 = -\frac{8k^2}{1+2k^2}, \quad x_1x_2 = \frac{8(k^2-1)}{1+2k^2} 
        \]
        计算弦长$AB$,
        \begin{align*}
        AB &= \sqrt{(x_1-x_2)^2 + (y_1-y_2)^2} \\
        &= \sqrt{(1+k^2)(x_1-x_2)^2} \\
        &= \sqrt{(1+k^2)[(x_1+x_2)^2 - 4x_1x_2]} \\
        &= \sqrt{(1+k^2)\left[\left(\frac{-8k^2}{1+2k^2}\right)^2 - \frac{32(k^2-1)}{1+2k^2}\right]} \\
        &= \frac{4\sqrt{2}(1+k^2)}{1+2k^2}
        \end{align*}
        将 $k^2 = \tan^2 \theta$代入得
        \[ 
        AB = \frac{4\sqrt{2}}{2-\cos^2 \theta} \tag{1}
        \]
        当 $\theta = \dfrac{\pi}{2}$ 时,$AB = 2\sqrt{2}$,仍满足式(1),于是得证
        \[ 
        AB = \frac{4\sqrt{2}}{2-\cos^2 \theta} 
        \]
    \end{solution}
    \part 过点 $F_1(-2,0)$ 作两条互相垂直的直线分别交椭圆 $C$ 于 $A,B$ 和 $D,E$,求 $AB+DE$ 的最小值。
    \begin{solution}
        因为 $AB\perp DE$,其倾斜角为 $\theta+\dfrac{\pi}{2}$,由 $(b)$ 可得
        \[
        AB=\frac{4\sqrt{2}}{2-\cos^2\theta},\quad
        DE=\frac{4\sqrt{2}}{2-\sin^2\theta}.
        \]
        于是
        \[
        AB+DE
        =\frac{4\sqrt{2}}{2-\cos^2\theta}+\frac{4\sqrt{2}}{2-\sin^2\theta} 
        =\frac{12\sqrt{2}}{2+\sin^2\theta\cos^2\theta}
        =\frac{12\sqrt{2}}{2+\frac{1}{4}\sin^2 2\theta}
        \]
        当 $\sin^2 2\theta=1$,即 $\theta=\dfrac{\pi}{4}$ 或 $\dfrac{3\pi}{4}$ 时,$\ AB+DE$ 取得最小值$\dfrac{16\sqrt{2}}{3}$。
    \end{solution}
    \end{parts}

    \question 已知椭圆
    \[
    C: \frac{y^{2}}{a^{2}}+\frac{x^{2}}{b^{2}}=1,\quad a>b>0
    \]
    的离心率为 $\dfrac{1}{2}$,上下焦点为 $F_{1}, F_{2}$,右顶点为 $D$;过 $F_{1}$ 做垂直于 $DF_{2}$ 的直线交椭圆 $C$ 于 $A, B$ 两点,且
    \[
    BD - AF_{1} = \frac{8\sqrt{3}}{39}
    \]
    \begin{parts}
    \part 求 $AD + BD$ 的值。
    \begin{solution}
        由椭圆离心率可知 $a:b:c = 2:\sqrt{3}:1$,则 $F_{1}F_{2} = F_{1}D = F_{2}D = 2c = a$,即 $\triangle F_{1}F_{2}D$ 为等边三角形;因为 $AB \perp DF_{2}$,且直线 $AB$ 过焦点 $F_{1}$,
        故直线 $AB$ 垂直平分线段 $DF_{2}$,由此得
        \[ 
        AD = AF_{2}, \quad BD = BF_{2} 
        \]
        由椭圆定义知 $BF_{1} + BF_{2} = 2a$,则有
        \[ 
        BD - AF_{1} = BF_{2} - (AB - BF_{1}) = 2a - AB = 4c - AB = \frac{8\sqrt{3}}{39} 
        \]
        解法一:设 $A(x_{1}, y_{1}), B(x_{2}, y_{2})$,
        \[ 
        \begin{cases}
        l_{AB}: y = -\sqrt{3}x + c \\ 
        C: \dfrac{y^{2}}{4c^{2}} + \dfrac{x^{2}}{3c^{2}} = 1
        \end{cases} \implies 13x^{2} - 6\sqrt{3}cx - 9c^{2} = 0 
        \]
        由韦达定理,
        \[
        AB = \sqrt{1+(-\sqrt{3})^2} \cdot \sqrt{(x_1+x_2)^2 - 4x_1x_2} 
        = \frac{2}{13}\sqrt{108c^2 + 36 \cdot 13c^2} = \frac{48c}{13}
        \]
        由 
        \[
        4c - AB = \frac{4c}{13} = \frac{8\sqrt{3}}{39},
        \]
        解得 
        \[
        c = \frac{2\sqrt{3}}{3} \Rightarrow a = \frac{4\sqrt{3}}{3}, b = 2
        \],
        故
        \[ 
        AD + BD = AF_2 + BF_2 = (2a - AF_1) + (2a - BF_1) = 4a - AB = \frac{112\sqrt{3}}{39} 
        \]
    \end{solution}
    \begin{solution}
        解法二:由焦半径公式可知 
        \[ 
        AB = \frac{2ab^2}{a^2 - c^2\cos^2\theta} = \frac{4c \cdot 3c^2}{4c^2 - c^2 \cdot \frac{3}{4}} = \frac{48}{13}c 
        \]
        其中 $\theta$ 为直线 $AB$ 与 $y$ 轴的夹角,其余步骤同解法一。
    \end{solution}
    \part 过 $A, B$ 做椭圆 $C$ 的两条切线交于点 $E$,若 $F_{1}E$ 交 $x$ 轴于 $P$,$F_{2}E$ 交 $x$ 轴于 $Q$,求线段 $PQ$ 的长度。
    \begin{solution}
        由 (a) 知直线 $AB$ 的方程为 
        \[
        y = -\sqrt{3}x + \frac{2\sqrt{3}}{3},
        \]
        且椭圆 $C$ 方程为 
        \[
        \frac{3y^2}{16} + \frac{x^2}{4} = 1,
        \]
        且焦点坐标为 $F_1(0, -\frac{2\sqrt{3}}{3}), F_2(0, \frac{2\sqrt{3}}{3})$。设 $E(x_0, y_0)$,则切点弦 $AB$ 的方程为 
        \[
        \frac{3y_0}{16}y + \frac{x_0}{4}x = 1
        \]
        对照直线 $AB$ 的方程,对应项系数成比例,解得 
        \[
        x_0 = 6, y_0 = -\frac{8\sqrt{3}}{3}
        \]
        直线 $EF_1$ 的方程为 
        \[
        y = x + \frac{2\sqrt{3}}{3},
        \]
        令 $y = 0$ 得 $x = -2$,即 $P(-2, 0)$;直线 $EF_2$ 的方程为 
        \[
        y = \frac{5\sqrt{3}}{9}x - \frac{2\sqrt{3}}{3},
        \]
        令 $y = 0$ 得 $x = \frac{6}{5}$,即 $Q(\frac{6}{5}, 0)$,所以线段 $PQ$ 的长度为
        \[ 
        PQ = \frac{6}{5} - (-2) = \frac{16}{5} 
        \]
    \end{solution}
    \end{parts}

    \question 已知 $P(1,1)$ 为椭圆 $$\Gamma: \frac{x^2}{9} + \frac{y^2}{4} = 1$$ 内一点。过 $P$ 作椭圆的弦 ${A_1A_4},{A_2A_5},{A_3A_6}$,任意两相邻弦的夹角之一为 $\dfrac{\pi}{3}$。试求
    \[
    \frac{1}{PA_1 \cdot PA_4} + \frac{1}{PA_2 \cdot PA_5} + \frac{1}{PA_3 \cdot PA_6}
    \]
    的值。
    \begin{solution}
        设过点 $P(1,1)$ 的一条直线为 $L: y = m(x - 1) + 1$。  
        与椭圆相交于两点 $A_1(x_1, y_1)$ 与 $A_4(x_4, y_4)$,其中
        \[
        y_1 = mx_1 - m + 1,\quad y_4 = mx_4 - m + 1
        \]
        点 $P$ 到 $A_1$ 与 $A_4$ 的距离为
        \[
        PA_1 = \sqrt{(x_1 - 1)^2 + (mx_1 - m)^2}
         = |x_1 - 1|\sqrt{m^2 + 1},PA_4 = |x_4 - 1|\sqrt{m^2 + 1}
        \]
        因此
        \[
        PA_1 \cdot PA_4 = |x_1 - 1||x_4 - 1|(m^2 + 1)= |1 - (x_1 + x_4) + x_1 x_4|(m^2 + 1)
        \tag{1}
        \]
        将 $y = m(x - 1) + 1$ 代入椭圆方程
        \[
        \frac{x^2}{9} + \frac{(mx - m + 1)^2}{4} = 1
        \Rightarrow
        (9m^2 + 4)x^2 + 18m(1 - m)x + 9(1 - m)^2 - 36 = 0
        \]
        设该方程两根为 $x_1, x_4$,则
        \[
        x_1 + x_4 = \frac{-18m(1 - m)}{9m^2 + 4},\quad
        x_1 x_4 = \frac{9(1 - m)^2 - 36}{9m^2 + 4}
        \]
        代入 (1)得:
        \[
        PA_1 \cdot PA_4 = \left|1 + \frac{18m(m - 1)}{9m^2 + 4} + \frac{9(1 - m)^2 - 36}{9m^2 + 4}\right|(m^2 + 1)
        = \frac{23(m^2 + 1)}{9m^2 + 4}
        \]
        即
        \[
        \frac{1}{PA_1 \cdot PA_4} = \frac{9m^2 + 4}{23(m^2 + 1)} = \frac{9}{23} - \frac{5}{23} \cdot \frac{1}{m^2 + 1} = \frac{9}{23} - \frac{5}{23} \cos^2 \theta
        \]
        其中$m = \tan \theta,\theta$为${A_1A_4}$与$x$轴的夹角,且由恒等式,
        \[
        \cos^2 \theta + \cos^2\left(\theta + \frac{\pi}{3}\right) + \cos^2\left(\theta + \frac{2\pi}{3}\right) = \frac{3}{2}
        \]
        故
        \[
        \begin{aligned}
        &\frac{1}{PA_1 \cdot PA_4} + \frac{1}{PA_2 \cdot PA_5} + \frac{1}{PA_3 \cdot PA_6} \\
        &= \left( \frac{9}{23} - \frac{5}{23} \cos^2 \theta \right)
        + \left( \frac{9}{23} - \frac{5}{23} \cos^2\left(\theta + \frac{\pi}{3}\right) \right)
        + \left( \frac{9}{23} - \frac{5}{23} \cos^2\left(\theta + \frac{2\pi}{3}\right) \right) \\
        &= \frac{27}{23} - \frac{5}{23} \cdot \frac{3}{2} =\frac{39}{46}
        \end{aligned}
        \]
    \end{solution}

    \question 已知椭圆 $$\frac{x^2}{3} + \frac{y^2}{2} = 1$$ 的左、右焦点分别为 $F_1$ 与 $F_2$,过焦点 $F_1$ 的直线交椭圆于 $B, D$ 两点,过焦点 $F_2$ 的直线交椭圆于 $A, C$ 两点,且 $AC \perp BD$,垂足为点 $P$。求四边形 $ABCD$ 面积的最小值。
    \ifprintanswers
    \begin{figure}[H]
        \centering        
        \includegraphics[width=0.5\textwidth]{images/image107.jpg}
    \end{figure}
    \fi
    \begin{solution}
        椭圆$a = \sqrt{3}, b = \sqrt{2}, c = \sqrt{a^2-b^2}=1$,焦点$F_1(-1,0), F_2(1,0)$,设$BD,AC$直线方程为
        \[
        L_1: y = m(x+1), \quad L_2: y = -\frac{1}{m}(x-1)
        \]
        代入椭圆方程
        \[
        \frac{x^2}{3} + \frac{m^2(x+1)^2}{2} = 1, \quad
        \frac{x^2}{3} + \frac{(x-1)^2}{2 m^2} = 1
        \]
        解得
        \[
        x_b + x_d = -\frac{6 m^2}{3 m^2 + 2},\ x_b x_d = \frac{3 m^2 - 6}{3 m^2 + 2}, \quad
        x_a + x_c = \frac{6}{2 m^2 + 3},\ x_a x_c = \frac{3 - 6 m^2}{2 m^2 + 3}
        \]
        则
        \[
        (x_b - x_d)^2 = (x_b + x_d)^2 - 4 x_b x_d = \frac{48 (m^2 + 1)}{(3 m^2 + 2)^2}, \quad
        (x_a - x_c)^2 = \frac{48 m^2 (m^2 + 1)}{(2 m^2 + 3)^2}
        \]
        \[
        BD = \sqrt{m^2+1} |x_b-x_d| = \frac{4 \sqrt{3} (m^2+1)}{3 m^2 + 2}, \quad
        AC = \sqrt{\frac{m^2+1}{m^2}} |x_a - x_c| = \frac{4 \sqrt{3} (m^2+1)}{2 m^2 + 3}
        \]
        由AM-GM不等式,
        \[
        [ABCD] = \frac{1}{2} AC \cdot BD = \frac{24 (m^2+1)^2}{(3 m^2 + 2)(2 m^2 + 3)} \ge \frac{24 (m^2+1)^2}{\left( \frac{3 m^2 + 2 + 2 m^2 + 3}{2} \right)^2} = \frac{24 (m^2+1)^2}{\frac{25}{4} (m^2+1)^2} = \frac{96}{25}
        \] 
        故四边形$ABCD$ 面积的最小值为$\dfrac{96}{25}$
    \end{solution}

    \question 已知 $B_{1},B_{2}$ 分别为椭圆 $$C: \frac{x^{2}}{8}+\frac{y^{2}}{4}=1$$ 的下顶点、上顶点,过点 $A(0,-2\sqrt{2})$ 的直线 $l$ 交椭圆 $C$ 于 $P,Q$ 两点(异于点 $B_{1},B_{2}$);
    \begin{parts}
    \part 若$\tan \angle B_{1}PB_{2}=2 \tan \angle B_{1}QB_{2}$ ,求直线 $l$ 的方程;
    \begin{solution}
        椭圆下顶点$B_{1}(0,-2)$,上顶点$B_{2}(0,2)$,直线 $l$ 斜率存在,设 $l$ 方程为 
        \[
        y=kx-2\sqrt{2}
        \]
        与椭圆方程
        \[
        \frac{x^{2}}{8}+\frac{y^{2}}{4}=1 
        \]
        联立得 
        \[
        (1+2k^{2})x^{2}-8\sqrt{2}kx+8=0
        \]
        则 
        \[
        \triangle=(8\sqrt{2}k)^{2}-32(1+2k^{2})>0 \Rightarrow k^{2}>\frac{1}{2}
        \]
        设 $P(x_{1},y_{1}),Q(x_{2},y_{2})$,则 
        \[
        x_{1}+x_{2}=\frac{8\sqrt{2}k}{1+2k^{2}}, \ x_{1}x_{2}=\frac{8}{1+2k^{2}}
        \]
        由点 $P,Q$ 在椭圆上知
        \[
        y_{1}^{2}-4=-\frac{1}{2}x_{1}^{2},\ y_{2}^{2}-4=-\frac{1}{2}x_{2}^{2}
        \]
        于是
        \[
        \tan \angle B_{1}PB_{2}=\left|\frac{k_{PB_{1}}-k_{PB_{2}}}{1+k_{PB_{1}}k_{PB_{2}}}\right|=\left|\frac{\frac{y_{1}-(-2)}{x_{1}}-\frac{y_{1}-2}{x_{1}}}{1+\frac{y_{1}-(-2)}{x_{1}}\frac{y_{1}-2}{x_{1}}}\right|=\left|\frac{4x_{1}}{x_{1}^{2}+y_{1}^{2}-4}\right|=\left|\frac{4x_{1}}{x_{1}^{2}-\frac{1}{2}x_{1}^{2}}\right|=\left|\frac{8}{x_{1}}\right|
        \]
        同理$\tan \angle B_{1}QB_{2}=\dfrac{8}{|x_{2}|}$;由 $\tan \angle B_{1}PB_{2}=2 \tan \angle B_{1}QB_{2}$ ,得 $$\frac{8}{|x_{1}|}=2\cdot\frac{8}{|x_{2}|}$$
        又 $x_{1}x_{2}=\dfrac{8}{1+2k^{2}}>0$,于是 
        \[
        x_{2}=2x_{1}\Rightarrow x_{1}+x_{2}=3x_1=\frac{8\sqrt{2}k}{1+2k^{2}}\Rightarrow x_{1}=\frac{8\sqrt{2}k}{3(1+2k^{2})}
        \]
        且
        \[
        x_{2}=2x_{1}\Rightarrow x_{1}x_{2}=2x_1^2=\frac{16\sqrt{2}k}{3(1+2k^{2})}\Rightarrow x_{1}=\frac{8}{1+2k^{2}}
        \]
        故
        \[
        32k^{2}=9(1+2k^{2})\Rightarrow k^{2}=\frac{9}{14}>\frac{1}{2}
        \]
        因此直线 $l$ 方程为 
        \[
        y=\frac{3\sqrt{14}}{14}x-2\sqrt{2}
        \]
    \end{solution}
    \part 设 $R$ 为直线 $B_{1}P$ 、 $B_{2}Q$ 的交点,求 ${AR}\cdot{B_{1}B_{2}}$ 的值.
    \begin{solution}
        直线 $B_{1}P,B_{2}Q$ 方程分别为 
        \[
        y=\frac{y_{1}+2}{x_{1}}x-2,\quad y=\frac{y_{2}-2}{x_{2}}x+2
        \]
        两式联立得 
        \[
        \frac{y+2}{y-2}=\frac{(y_{1}+2)x_{2}}{(y_{2}-2)x_{1}}=\frac{(kx_{1}-2\sqrt{2}+2)x_{2}}{(kx_{2}-2\sqrt{2}-2)x_{1}}
        \]
        因为 
        \begin{align*}
        &(kx_{1}-2\sqrt{2}+2)x_{2}-(-3+2\sqrt{2})(kx_{2}-2\sqrt{2}-2)x_{1}\\
        &=(4-2\sqrt{2})kx_{1}x_{2}+(-2\sqrt{2}+2)(x_{1}+x_{2})\\
        &=(4-2\sqrt{2})k\cdot\frac{8}{1+2k^{2}}+(-2\sqrt{2}+2)\cdot\frac{8\sqrt{2}k}{1+2k^{2}}=0
        \end{align*}
        所以
        \[
        \frac{y+2}{y-2}=-3+2\sqrt{2}\Rightarrow y=-\sqrt{2}
        \]
        即点$R$在定直线 $y=-\sqrt{2}$ 上;设 $R(t,-\sqrt{2}),$则 ${AR}=(t,\sqrt{2})$,又 ${B_{1}B_{2}}=(0,4),$ 因此 $${AR}\cdot{B_{1}B_{2}}=4\sqrt{2}$$
    \end{solution}
    \end{parts}

    \question 双曲线 
    \[
    \frac{x^{2}}{9}-\frac{y^{2}}{16}=1
    \] 
    的焦点为 \(F_1,F_2\)。设第一象限点 \(P\) 满足 \(PF_1:PF_2=1:3\)。求 \(\triangle F_1PF_2\) 周长。  
    \begin{solution}
        双曲线焦点$F_1=(-5,0),\ F_2=(5,0)$,已知$PF_1:PF_2=1:3$,由双曲线定义,
        \[
        PF_2 - PF_1 = 2a \Rightarrow 3PF_1 - PF_1 = 6 \Rightarrow PF_1 = 3
        \]
        \(\triangle F_1PF_2\)周长为
        \[
        PF_1 + PF_2 + F_1F_2 = 3 + 9 + 10 = 22
        \]
    \end{solution}
    
    \question 已知双曲线焦点 \((-2,-2),(8,-2)\),且一条渐近线斜率为 \(-\dfrac{4}{3}\)。求其标准方程。  
    \begin{solution}
        双曲线中心$C(3, -2)$,焦距$c = 5$,又已知渐近线斜率为 $\pm \dfrac{b}{a} = \pm \dfrac{4}{3}$, 有 
        $\dfrac{b^2}{a^2} = \dfrac{16}{9}$,
        
        由 $c^2 = a^2 + b^2$ 得
        \[
        25 = a^2 + b^2 = a^2 + \frac{16}{9}a^2 = \frac{25}{9}a^2
        \Rightarrow a^2 = 9,\ b^2 = 16
        \]
        所以标准方程为
        \[
        \dfrac{(x - 3)^2}{9} - \dfrac{(y + 2)^2}{16} = 1
        \]
    \end{solution}

    \question 已知双曲线 $$\Gamma: \dfrac{x^2}{4}-\dfrac{y^2}{5}=1,$$过焦点 $F$ 作一焦弦 $PQ$,已知 $PQ$ 的斜率为 $1$,求 $PF' + QF'$。 
    \begin{solution}
        双曲线$c=\sqrt{4+5}=3$,焦点$F(3,0), F'(-3,0)$,过 $F$ 且斜率为 $1$ 的直线为
        \[
        L: y=x-3,
        \]
        设交点为 $P(x_1,x_1-3),\;Q(x_2,x_2-3)$,代入双曲线方程解得
        \[
        \frac{x^2}{4}-\frac{(x-3)^2}{5}=1 \Rightarrow x=-12\pm 10\sqrt{2}
        \]
        因此
        \[
        |x_1-x_2|=20\sqrt{2},\quad PQ=\sqrt{1^2+(-1)^2}\,|x_1-x_2|=\sqrt{2}\cdot 20\sqrt{2}=40
        \]
        由双曲线定义,
        \[
        PF'-PF=2a=4,\quad QF'-QF=2a=4
        \]
        两式相加得
        \[
        PF'+QF'=(PF+QF)+8=40+8=48
        \]
    \end{solution}

    \question 已知双曲线焦点为 \(F_1(0,2), F_2(0,-2)\),实轴长为 2,点 \(Q\) 在一条渐近线上,且 \(\angle F_1QF_2 = 90^\circ\),求 \(\triangle QF_1F_2\) 面积。

    \begin{solution}
        已知 \(c = 2,\; a = 1 \Rightarrow b^2 = c^2 - a^2 = 3\),  
        双曲线方程为\(y^2 - \dfrac{x^2}{3} = 1\),渐近线为\(y = \pm \dfrac{\sqrt{3}}{3}x\)。
        
        设 \(Q(x, \dfrac{\sqrt{3}}{3}x)\),由 \(\angle F_1QF_2 = 90^\circ\),
        \[
        m_{QF_1}\cdot m_{QF_2}=\frac{\frac{\sqrt{3}}{3}x-2}{x-0}\cdot \frac{\frac{\sqrt{3}}{3}x+2}{x-0}=-1 \Rightarrow x = \pm\sqrt{3}
        \]
        \(\triangle QF_1F_2\)面积为
        \[
        \frac{1}{2} \cdot 4 \cdot \sqrt{3} = 2\sqrt{3}
        \]
    \end{solution}
    
    \question 已知双曲线 $\Gamma: xy = k,\; k < 0$,点 $P(2,2)$,过 $P$ 作 $\Gamma$ 的两条切线,切点为 $A$、$B$,若三角形 $\triangle PAB$ 是正三角形,求 $k$。
    \begin{solution}
        由 $\Gamma: xy = k < 0$ 可知其对称轴为 $y = x$,若 $\triangle PAB$ 为正三角形,且 $P$ 在 $y = x$ 上,故切线 ${PA}$ 的斜率为 
        \[
        \tan 75^\circ = \tan(45^\circ + 30^\circ) = \frac{1 + \frac{1}{\sqrt{3}}}{1 - \frac{1}{\sqrt{3}}} = 2 + \sqrt{3}
        \]
        切线方程为
        \[
        y = (2 + \sqrt{3})(x - 2) + 2
        \]
        将其代入 $xy = k$ 整理得
        \[
        (2 + \sqrt{3})x^2 + (-2 - 2\sqrt{3})x - k = 0
        \]
        令判别式为 0,
        \[
        (2 + 2\sqrt{3})^2 + 4k(2 + \sqrt{3}) = 0
        \]
        解得
        \[
        k = -2
        \]
    \end{solution}

    \question 已知双曲线 $$\frac{x^{2}}{16}-\frac{y^{2}}{12}=1$$ 上有两点 $P, Q,F_1, F_2$ 为其焦点,且 $PQ$ 过 $F_2,\angle F_2PF_1 = 60^\circ$,求 $\triangle PQF_1$ 的周长。
    \ifprintanswers
    \begin{figure}[H]
        \centering        
        \includegraphics[width=0.5\textwidth]{images/image117.jpg}
    \end{figure}
    \fi
    \begin{solution}
        由双曲线方程得$c = \sqrt{16+12} = 2\sqrt{7}$,焦点坐标
        \[
        F_1(-2\sqrt{7},0), \quad F_2(2\sqrt{7},0).
        \]
        设$PF_2 = m, QF_2 = n$,则
        \[
        PF_1 = m + 2a = m+8, \quad QF_1 = n+8.
        \]
        在 $\triangle PF_1F_2$中,由余弦定理,
        \[
        (4\sqrt{7})^2=(m+8)^2 + m^2-2 m (m+8)\cos 60^\circ \Rightarrow m = 4
        \]
        在 $\triangle PF_1F_2$中,由余弦定理,
        \[
        (n+8)2=(m+8)^2+(m+n)^2-2(m+8)(m+n)\cos 60^\circ \Rightarrow n = \frac{12}{5}
        \]
        故 $\triangle PQF_1$ 周长
        \[
        (m+8) + m + n + (n+8)= \frac{144}{5}
        \]
    \end{solution}

    \question 双曲线$$\frac{x^2}{16} - \frac{y^2}{9} = 1$$的左、右焦点分别为$F_1, F_2,P$是双曲线上一点, 若$\triangle PF_1F_2$的内切圆圆心为$(4,2)$, 求$\triangle PF_1F_2$外接圆的半径。
    \begin{solution}
        双曲线焦点分别为$F_1(-5,0), F_2(5,0)$,且已知$I(4,2)$,设$\angle IF_1F_2 = \alpha$, $\angle IF_2F_1 = \beta$,
        则 $$\tan \alpha = \frac{2}{9},\; \tan \beta = 2$$
        故 $$\tan \angle F_1PF_2 =\tan(2\alpha+2\beta)=\frac{2\tan(\alpha+\beta)}{1-\tan^2(\alpha+\beta)} = -\frac{8}{15}$$ 
        其中$$\tan(\alpha + \beta) = \frac{\frac29+2}{1-\frac29\cdot2}=4$$
        故$$\sin \angle F_1PF_2 = \frac{8}{17}$$
        $\therefore \triangle PF_1F_2$外接圆的半径为$$\frac{10}{2\sin \angle F_1PF_2} = \frac{85}{8}$$.
    \end{solution} 

    \question 设双曲线 
    \[
    \Gamma: \frac{x^2}{3} - \frac{y^2}{2} = 1
    \]
    两焦点为 $F_1, F_2$,点 $P(3,-2)$ 在 $\Gamma$ 上。以 $P$ 点为切点的切线交 $x$ 轴于点 $Q$,求 $\tan \angle FPQ$。
    \ifprintanswers
    \begin{figure}[H]
        \centering
        \includegraphics[width=0.5\linewidth]{images/image50.png}
    \end{figure}
    \fi
    \begin{solution}
        由双曲线知$a = \sqrt{3}, \quad b = \sqrt{2}, c = \sqrt{a^2 + b^2} = \sqrt{5}$,焦点为
        \[
        F_1(\sqrt{5}, 0), \quad F_2(-\sqrt{5}, 0).
        \]
        在$P(3, -2)$的切线斜率为
        \[
        \frac{2}{3} x - y y' = 0 \Rightarrow y'|_{x=3,y=-2} = \frac{2x}{3y}= \frac{6}{-6} = -1.
        \]
        过$P$切线方程为
        \[
        y + 2 = -1(x - 3) \Rightarrow x + y = 1
        \]
        与 $x$ 轴交与
        \[
        Q(1, 0).
        \]
        在 $\triangle PAQ$ 及$\triangle PAF_1$,设$\theta_1 = \angle F_1 P A,\theta_2 = \angle F_1 P Q$
        \[
        \tan \theta_1 = \frac{3 - \sqrt{5}}{2}, \quad \tan(\theta_1 + \theta_2) = 1,
        \]
        利用和角公式,解得
        \[
        \frac{\tan \theta_2 + \frac{3 - \sqrt{5}}{2}}{1 - \tan \theta_2 \cdot \frac{3 - \sqrt{5}}{2}} = 1 \Rightarrow \tan \theta_2 = \frac{-1 + \sqrt{5}}{5 - \sqrt{5}} = \frac{\sqrt{5}}{5}
        \]
    \end{solution}
    
    \question 已知双曲线与椭圆的方程分别为
    \[
    \frac{x^{2}}{a^{2}}-\frac{y^{2}}{b^{2}}=1,
    \quad
    \frac{x^{2}}{a^{2}}+\frac{y^{2}}{b^{2}}=1,
    \]
    其中 $a>b>0$。在双曲线上取一坐标均为正的点作切线,该切线经过椭圆的正 $x$ 轴焦点。证明该切线的斜率恒为 $1$。
    \begin{solution}
        设双曲线上一点$P(a\sec\theta,b\tan\theta)$,其中 $\theta\in\left(0,\dfrac{\pi}{2}\right)$,对双曲线方程求导得
        \[
        \frac{2x}{a^{2}}-\frac{2y}{b^{2}}\frac{dy}{dx}=0 \Rightarrow \frac{dy}{dx}=\frac{b^{2}x}{a^{2}y} 
        \]
        则在点$P$处的切线斜率为$\dfrac{b^{2}a\sec\theta}{a^{2}b\tan\theta}=\dfrac{b}{a\sin\theta}$,因此切线方程为
        \[
        y-b\tan\theta=\frac{b}{a\sin\theta}(x-a\sec\theta).
        \]
        该切线经过椭圆的正 $x$ 轴焦点 $(ae,0)$,其中 $e$ 为椭圆的离心率,代入得
        \[
        0-b\tan\theta=\frac{b}{a\sin\theta}(ae-a\sec\theta) \Rightarrow e = \cos\theta
        \]
        而椭圆的离心率满足$b^{2}=a^{2}(1-e^{2})$,代入 $e=\cos\theta$,得
        \[
        \frac{b^{2}}{a^{2}}=1-\cos^{2}\theta=\sin^{2}\theta \Rightarrow \frac{b}{a}=\sin\theta
        \]
        因此得证切线斜率为
        \[
        \frac{b}{a\sin\theta}=\sin\theta\cdot\frac{1}{\sin\theta}=1
        \]
    \end{solution}

    \question 已知双曲线 
    \[
    H: \dfrac{x^2}{a^2} - \dfrac{y^2}{b^2} = 1\ ,a > 0,\ b > 0,
    \]
    $l$ 为双曲线 $H$ 的一条渐近线,$\ F_1, F_2$ 是双曲线 $H$ 的左、右焦点。若 $F_1$ 关于直线 $l$ 的对称点在圆 
    \[
    C:(x - c)^2 + y^2 = c^2
    \]
    上,其中 $c$ 为双曲线的半焦距,求双曲线 $H$ 的离心率。
    \begin{solution}
        双曲线左焦点为 $F_1 = (-c,\ 0)$,其中 $c^2 = a^2 + b^2$。  
        渐近线 $l$ 的一个方程为 $y = \dfrac{b}{a}x$,即 
        $$bx - ay = 0$$
        设 $F_1$ 关于直线 $l$ 的对称点为 $F_1' = (x',\ y'),F_1F_1'$ 的中点为 $\left(\dfrac{x'-c}{2},\ \dfrac{y'}{2} \right)$,
        \[
        b\cdot \dfrac{x'-c}{2} - a\cdot \dfrac{y'}{2} = 0 \Rightarrow b(x'-c) - a y' = 0 \tag{1}
        \]
        且
        \[
        m_{F_1F_1'}=m_{l}=\dfrac{y'}{x'+c} \cdot \dfrac{b}{a} = -1 \Rightarrow b y' = -a(x'+c) \tag{2}
        \]
        由$(1),(2)$得
        \[
        F_1' = \left( \dfrac{(b^2 - a^2)c}{a^2 + b^2},\ -\dfrac{2ab c}{a^2 + b^2} \right)
        \]
        将其代入圆$C$:
        \begin{align*}
        \left( \dfrac{(b^2 - a^2)c}{a^2 + b^2} - c \right)^2 + \left( \dfrac{2ab c}{a^2 + b^2} \right)^2 &= c^2 
        \end{align*}
        可得
        \[
        3a^2 = b^2 \Rightarrow c^2 = a^2 + b^2 = 4a^2 \Rightarrow c = 2a
        \]
        所以双曲线离心率为
        \[
        e = \frac{c}{a} = 2
        \]
    \end{solution}

    \question 已知双曲线 
    \[
    C: \frac{x^{2}}{a^{2}} - \frac{y^{2}}{b^{2}} = 1 \; (a>0,\; b>0)
    \]
    与椭圆
    \[
    \frac{x^{2}}{4} + \frac{y^{2}}{3} = 1,
    \]
    过椭圆上一点 $P\left(-1,\dfrac{3}{2}\right)$ 作椭圆的切线 $l$与 $x$ 轴交于 $M$ 点,且$l$ 与双曲线 $C$ 的两条渐近线分别交于 $N,Q$,若 $N$ 为 $MQ$ 的中点,求双曲线 $C$ 的离心率。
    \begin{solution}
        设过$P(-1,\dfrac{3}{2})$的切线$l$方程为
        \[
        y - \frac{3}{2} = k(x + 1)
        \Rightarrow y = kx + k + \frac{3}{2}.
        \]
        将此代入椭圆方程
        \[
        \frac{x^{2}}{4} + \frac{y^{2}}{3} = 1.
        \]
        整理得
        \[
        (4k^2+3)x^2+(8k^2+12k)x+4k^2+12k-3=0
        \]
        令判别式为零得
        \[
        \Delta = (8k^2+12k)^2-4(4k^2+3)(4k^2+12k-3)=0\Rightarrow k = \frac{1}{2}.
        \]
        所以切线 $l$ 的方程为$y = \dfrac{1}{2}x + 2$,令$y = 0$得 $M(-4, 0)$
        
        双曲线的渐近线方程为$y = \pm \dfrac{b}{a}x$,联立 $y = \dfrac{b}{a}x$ 与 $y = \dfrac{1}{2}x + 2$,得 $x_Q = \dfrac{4a}{2b - a}$,
        
        联立 $y = -\dfrac{b}{a}x$ 与 $y = \dfrac{1}{2}x + 2$,得
        $x_N = -\dfrac{4a}{2b + a}$,
        
        由于 $N$ 是 $MQ$ 的中点,
        \[
        -\frac{4a}{2b + a} = \frac{1}{2} \left( \frac{4a}{2b - a} - 4 \right) \Rightarrow\frac{b}{a} = \frac{3}{2}
        \]
        故双曲线的离心率为
        \[
        e = \sqrt{1 + \left( \frac{3}{2} \right)^2 } = \frac{\sqrt{13}}{2}
        \]
    \end{solution}

    \question 已知双曲线 
    \[
    C:\dfrac{x^{2}}{a^{2}} - \dfrac{y^{2}}{b^{2}} = 1\ (a>0,\ b>0)
    \]
    的左、右顶点分别是 $A_1,A_2$,  圆 $x^2 + y^2 = a^2$ 与 $C$ 的渐近线在第一象限的交点为 $M$,直线 $A_1M$ 交 $C$ 的右支于点 $P$。若 $\triangle MPA_2$ 是等腰三角形,且 $\angle PA_2M$ 的内角平分线与 $y$ 轴平行,求双曲线的离心率。
    \begin{solution}
        双曲线顶点 $A_1(-a, 0), A_2(a, 0)$,渐近线为 $y = \dfrac{b}{a}x$,圆为 $x^2 + y^2 = a^2$,联立得
        \[
        x^2\left(1 + \frac{b^2}{a^2}\right) = a^2 \Rightarrow x = \frac{a^2}{c}, \quad y = \frac{ab}{c}
        \]
        故点 $M\left( \dfrac{a^2}{c}, \dfrac{ab}{c} \right)$,其中 $c = \sqrt{a^2 + b^2}$ 为半焦距,又
        \[
        MA_1^2 = \left(\frac{a^2}{c} + a\right)^2 + \left( \frac{ab}{c} \right)^2 = 2a^2\left(1 + \frac{a}{c}\right)
        \]
        \[
        MA_2^2 = \left(\frac{a^2}{c} - a\right)^2 + \left( \frac{ab}{c} \right)^2 = 2a^2\left(1 - \frac{a}{c}\right)
        \]
        由题设$\angle A_{1}MA_{2}=\angle PMA_{2}=90^{\circ},\triangle MPA_{2}$是等腰直角三角形,$\;\angle MA_{2}P=45^{\circ}$
        
        $\angle PA_{2}M$的内角平分线与$y$轴平行,所以$\angle MA_{1}A_{2}=22.5^{\circ}$
        
        又$\tan 45^{\circ}=\dfrac{2\tan 22.5^{\circ}}{1-\tan^{2}22.5^{\circ}}=1$可得$\tan 22.5^{\circ}=\sqrt{2}-1$,所以
        \[
        \tan^{2}\angle MA_{1}A_{2}=\left(\frac{MA_{2}}{MA_{1}}\right)^{2}=\frac{1-\frac{a}{c}}{1+\frac{a}{c}}=(\sqrt{2}-1)^{2}
        \]
        解得
        \[
        \frac{e-1}{e+1}=3-2\sqrt{2}\Rightarrow e=\sqrt{2}
        \]
    \end{solution}

    \question 已知双曲线的两个焦点分别为 $F_1, F_2$,若过点 $F_1$ 的直线与双曲线交于 $A, B$ 两点,且 $\angle AF_1F_2 = 30^\circ,F_2A = F_2B$,求双曲线的离心率。
    \begin{solution}
        不妨 $A$ 在双曲线靠近 $F_1$ 的这一支上, 取 $AB$ 中点 $C$, 则 $F_2C \perp AB$。
        
        设 $AF_1=x$, 由双曲线定义知
        \[
        AF_2 = BF_2= x+2a, \;BF_1=x+4a
        \]
        从而 
        \[
        AB=4a,\;AC=2a
        \]
        由 $\angle AF_1F_2=30^\circ$ 知 $CF_1=\dfrac{\sqrt{3}}{2}F_1F_2$,可得
        \[
        x=\sqrt{3}c-2a\tag{1}
        \]
        又由 $CF_2=\dfrac{\sqrt3}{3}CF_1$ 及 $AC^2+CF_2^2=AF_2^2$,可得 
        \[
        (2a)^2 + \frac{(x+2a)^2}{3}=(x+2a)^2 \Rightarrow x=(\sqrt{6}-2)a \tag{2}
        \]
        由$(1),(2)$解得离心率 
        \[
        e=\frac{c}{a}=\sqrt{2}
        \]
    \end{solution}

    \question 双曲线的中心为原点 $O$,焦点在 $x$ 轴上,两条渐近线分别为 $l_1, l_2$,过右焦点 $F$ 垂直于 $l_1$ 的直线分别交 $l_1, l_2$ 于 $A, B$ 两点。已知 $|\overrightarrow{OA}|, |\overrightarrow{AB}|, |\overrightarrow{OB}|$ 成等差数列,且 $\overrightarrow{BF}$ 与 $\overrightarrow{FA}$ 同向。
    \begin{parts}
    \part 求双曲线的离心率;
    \begin{solution}
        已知$OA,AB,OB$成等差数列,设 $OA = m-d,AB = m,OB = m+d$,由毕氏定理,
        \[
        (m-d)^2 + m^2 = (m+d)^2 \Rightarrow d = \frac{1}{4} m
        \]
        设双曲线半轴为 $a, b$,右焦点 $F(c,0)$,则
        \[
        \tan \angle AOF = \frac{b}{a}, \quad \tan \angle AOB = \tan (2 \ \angle AOF) = \frac{AB}{OA} = \frac{4}{3}
        \]
        由倍角公式 $\tan 2\theta = \dfrac{2\tan\theta}{1-(\tan\theta)^2}$得
        \[
        \frac{\frac{2b}{a}}{1-\frac{b^2}{a^2}} = \frac{4}{3} \Rightarrow \frac{b}{a} = \frac{1}{2}
        \]
        故离心率为 
        \[
        e = \sqrt{1 + \left(\frac{b}{a}\right)^2} = \frac{\sqrt{5}}{2}
        \]
    \end{solution}
    \part 若 $AB$ 被双曲线所截得的线段的长为 $4$,求双曲线的方程。
    \begin{solution}
        过 $F$ 直线方程为 
        \[
        y = -\frac{a}{b}(x-c)
        \]
        与双曲线方程 
        \[
        \frac{x^2}{a^2} - \frac{y^2}{b^2} = 1
        \]
        联立,将 $a = 2b,c = \sqrt{5}b$ 代入,化简得
        \[
        \frac{15}{4b^2}x^2 - \frac{8\sqrt{5}}{b}x + 21 = 0
        \]
        由韦达定理,
        \begin{align*}
        4 = \sqrt{1+\left(\frac{a}{b}\right)^2} |x_1-x_2| 
        &= \sqrt{1+\left(\frac{a}{b}\right)^2} \sqrt{(x_1+x_2)^2 - 4x_1x_2} \\
        &= \sqrt{5} \sqrt{\left(\frac{32\sqrt{5}b}{15}\right)^2 - 4 \cdot \frac{28b^2}{5}}
        \end{align*}
        解得 
        \[
        b = 3
        \]
        故双曲线方程为
        \[
        \frac{x^2}{36} - \frac{y^2}{9} = 1
        \]
    \end{solution}
    \end{parts}

    \question 已知双曲线
    \[
    C:\frac{x^2}{a^2}-\frac{y^2}{b^2}=1 \quad a>0,b>0
    \]
    的两个焦点为 $F_1(-2,0),F_2(2,0)$,点 $P(3,\sqrt7)$ 在曲线 $C$ 上。
    \begin{parts}
    \part 求双曲线 $C$ 的方程;
    \begin{solution}
        据题意,由$a^2+b^2=4$,双曲线方程为
        \[
        \frac{x^2}{a^2}-\frac{y^2}{4-a^2}=1\quad 0<a^2<4
        \]
        将点 $(3,\sqrt7)$ 代入得
        \[
        \frac{9}{a^2}-\frac{7}{4-a^2}=1 \Rightarrow a^2=18\ (\text{舍去})\quad\text{或}\quad a^2=2
        \]
        故所求双曲线方程为
        \[
        \frac{x^2}{2}-\frac{y^2}{2}=1
        \]
    \end{solution}
    \begin{solution}
        依题意得双曲线的半焦距 $c=2$。又
        \[
        2a=PF_1-PF_2
        =\sqrt{(3+2)^2+(\sqrt7)^2}-\sqrt{(3-2)^2+(\sqrt7)^2}
        =2\sqrt2
        \]
        故
        \[
        a^2=2,\quad b^2=c^2-a^2=2
        \]
        因此双曲线 $C$ 的方程为
        \[
        \frac{x^2}{2}-\frac{y^2}{2}=1
        \]
    \end{solution}
    \part 记 $O$ 为坐标原点,过点 $Q(0,2)$ 的直线与双曲线 $C$ 相交于不同的两点 $E,F$,若 $\triangle OEF$ 的面积为 $2\sqrt2$,求直线的方程。
    \begin{solution}
        设直线的方程为$y=kx+2$,代入双曲线 $C$ 整理,
        \[
        (1-k^2)x^2-4kx-6=0
        \]
        故直线 $l$ 与双曲线 $C$ 相交于不同的两点 $E,F$,需满足
        \[
        \begin{cases}
        1-k^2\neq0,\\
        \Delta=(-4k)^2+4\cdot 6(1-k^2)>0,
        \end{cases}
        \]
        即$k\in(-\sqrt3,-1)\cup(1,\sqrt3)$,设 $E(x_1,y_1),F(x_2,y_2)$,由韦达定理,
        \[
        x_1+x_2=\frac{4k}{1-k^2},\quad x_1x_2=\frac{6}{1-k^2}.
        \]
        于是
        \begin{align*}
        EF&=\sqrt{(x_1-x_2)^2+(y_1-y_2)^2}\\
        &=\sqrt{1+k^2}\,|x_1-x_2|\\
        &=\sqrt{1+k^2}\sqrt{(x_1+x_2)^2-4x_1x_2}\\
        &=\sqrt{1+k^2}\cdot\frac{2\sqrt2\sqrt{3-k^2}}{|1-k^2|}
        \end{align*}
        原点 $O$ 到直线的距离为
        \[
        d=\frac{2}{\sqrt{1+k^2}},
        \]
        故
        \[
        [\triangle OEF]
        =\frac12 d\cdot EF
        =\frac{2\sqrt2\sqrt{3-k^2}}{|1-k^2|}=2\sqrt2 \Rightarrow k^4-k^2-2=0
        \]
        解得
        \[
        k=\pm\sqrt2 \in(-\sqrt3,-1)\cup(1,\sqrt3)
        \]
        故所求直线方程为
        \[
        y=\sqrt2x+2,\quad y=-\sqrt2x+2.
        \]
    \end{solution}
    \end{parts}

    \question 点$P\left(p+\dfrac{1}{p},\,p-\dfrac{1}{p}\right),\ p\neq 0$在直角双曲线
    \[
    x^2-y^2=4
    \]
    上。曲线在 $P$ 处的法线与 $y$ 轴交于点 $Q(0,-k)$,其中 $k>0$。已知三角形 $OPQ$ 的面积为 $\dfrac{15}{4}$,其中 $O$ 为原点,求点 $P$ 的坐标。
    \begin{solution}
        曲线可参数表示为
        \[
        x=p+\frac{1}{p},\quad y=p-\frac{1}{p}
        \]
        由链导法,
        \[
        \frac{dy}{dx}=\frac{dy}{dp} \div \frac{dx}{dp}
        =\frac{1+\frac{1}{p^2}}{1-\frac{1}{p^2}}
        =\frac{p^2+1}{p^2-1}
        \]
        因此法线斜率为$-\dfrac{p^2-1}{p^2+1}$,点 $P$ 处法线方程为
        \[
        y-\left(p-\frac{1}{p}\right)
        =-\frac{p^2-1}{p^2+1}\left(x-\left(p+\frac{1}{p}\right)\right)
        \]
        该法线经过点 $Q(0,-k)$,代入得
        \[
        -k-p+\frac{1}{p}=\frac{p^2-1}{p^2+1}\left(p+\frac{1}{p}\right) \Rightarrow k=-\frac{2(p^2-1)}{p} \tag{1}
        \]
        已知 $\triangle OPQ$ 的面积为 $\dfrac{15}{4}$,有
        \[
        \frac{1}{2}\,k\left(p+\frac{1}{p}\right)=\frac{15}{4} \tag{2}
        \]
        联立$(1),(2)$解得
        \[
        (4p^2-1)(p^2+4)=0 \Rightarrow p=\pm\frac{1}{2}
        \]
        因此点 $P$ 的可能坐标为
        \[
        \left(\frac{5}{2},-\frac{3}{2}\right),        \left(-\frac{5}{2},\frac{3}{2}\right)
        \]
    \end{solution}

    \question 已知中心在原点的双曲线 $C$ 的一个焦点为 $F_{1}(-3,0)$,一条渐近线的方程是$\sqrt{5}x-2y=0$,
    \begin{parts}
    \part 求双曲线 $C$ 的方程;
    \begin{solution}
        设双曲线 $C$ 的方程为
        \[
        \frac{x^{2}}{a^{2}}-\frac{y^{2}}{b^{2}}=1 
        \]
        其中$a>0,b>0$,由焦点坐标得$c=\sqrt{a^2+b^2}=3$,又由渐近线方程$y=\dfrac{\sqrt{5}}{2}x$得
        \[
        \frac{b}{a}=\frac{\sqrt{5}}{2}
        \]
        于是可解得
        \[
        a^{2}=4,\quad b^{2}=5
        \]
        故双曲线 $C$ 的方程为
        \[
        \frac{x^{2}}{4}-\frac{y^{2}}{5}=1
        \]
    \end{solution}
    \part 若以 $k\ne 0$ 为斜率的直线 $l$ 与双曲线 $C$ 相交于两个不同的点$M,N$,且线段 $MN$ 的垂直平分线与两坐标轴围成的三角形面积为$\dfrac{81}{2}$,求 $k$ 的取值范围。
    \begin{solution}
        设直线 $l$ 的方程为
        \[
        y=kx+m, \quad k\ne 0,
        \]
        与双曲线交于 $M(x_{1},y_{1}),N(x_{2},y_{2})$,将 $y=kx+m$ 代入双曲线方程得
        \[
        \frac{x^{2}}{4}-\frac{(kx+m)^{2}}{5}=1,
        \]
        化简为
        \[
        (5-4k^{2})x^{2}-8kmx-4m^{2}-20=0.
        \]
        该方程有两个不等实根,故
        \[
        5-4k^{2}\ne 0,
        \]
        且判别式
        \[
        \Delta>0 \;\Rightarrow\; m^{2}+5-4k^{2}>0. \tag{1}
        \]
        由韦达定理,线段 $MN$ 的中点
        \[
        (x_{0},y_{0})=\left(\frac{4km}{5-4k^{2}},\ \frac{5m}{5-4k^{2}}\right).
        \]
        因此$MN$ 的垂直平分线与 $x$ 轴、$y$ 轴的截距分别为
        \[
        \left(\frac{9km}{5-4k^{2}},0\right),\quad
        \left(0,\frac{9m}{5-4k^{2}}\right).
        \]
        由题设三角形面积得
        \[
        \frac{1}{2}\left|\frac{9km}{5-4k^{2}}\right|
        \left|\frac{9m}{5-4k^{2}}\right|
        =\frac{81}{2},
        \]
        化简得
        \[
        m^{2}=\frac{(5-4k^{2})^{2}}{|k|},\quad k\ne 0. \quad \tag{2}
        \]
        将 (2) 代入 (1),整理为
        \[
        (4k^{2}-5)(4k^{2}-|k|-5)>0,\quad k\ne 0.
        \]
        解得
        \[
        0<|k|<\frac{\sqrt{5}}{2}
        \quad \text{或} \quad
        |k|>\frac{5}{4}.
        \]
        因此$k$ 的取值范围为
        \[
        k\in\left(-\infty,-\frac{5}{4}\right)\cup\left(-\frac{\sqrt{5}}{2},0\right)\cup
        \left(0,\frac{\sqrt{5}}{2}\right)\cup\left(\frac{5}{4},+\infty\right)
        \]
    \end{solution}
    \end{parts}

    \question 已知双曲线
    \[
    E: \frac{x^{2}}{a^{2}} - \frac{y^{2}}{b^{2}} = 1
    \]
    的左、右顶点分别为 $A, B$,点 $M$ 在 $E$ 上使得 $\triangle ABM$ 是等腰三角形,且$\triangle ABM$外接圆面积为 $3 \pi a^{2}$,求双曲线的离心率。
    \begin{solution}
        不妨设$M$在第二象限,则在等腰$\triangle ABM$中, $AB=AM=2a$,
        
        设 $\angle ABM=\angle AMB=\theta$, 则 $\angle FAM=2\theta$,又$\triangle ABM$外接圆面积 $3\pi a^{2}$, 则半径为 $\sqrt{3}a$.
        
        由正弦定理,
        \[
        2R = \frac{AB}{\sin\theta} \Rightarrow 2\sqrt{3}a = \frac{2a}{\sin\theta} \Rightarrow \sin\theta = \frac{\sqrt{3}}{3}
        \]
        于是
        \[
        \cos\theta = \frac{\sqrt{6}}{3},\;\sin 2\theta = \frac{2\sqrt{2}}{3},\;\cos 2\theta = \frac{1}{3}
        \]
        设$M$点坐标为 $(x,y)$, 则 
        \[
        x = -a - AM\cos 2\theta = -a - 2a \cdot \frac{1}{3} = -\frac{5a}{3}
        \]
        \[
        y = AM\sin 2\theta = 2a \cdot \frac{2\sqrt{2}}{3} = \frac{4\sqrt{2}a}{3}
        \]
        即$M$点坐标为 $\left(-\dfrac{5a}{3},\dfrac{4\sqrt{2}a}{3}\right)$,由$M$点在双曲线上得 
        \[
        \frac{\left(-\frac{5a}{3}\right)^{2}}{a^{2}}-\frac{\left(\frac{4\sqrt{2}a}{3}\right)^{2}}{b^{2}}=1
        \Rightarrow b^2 = 2a^2
        \]
        故
        \[
        e = \sqrt{1+\dfrac{b^{2}}{a^{2}}}  =\sqrt{3}
        \]
    \end{solution}  
        
    \question 双曲线 
    \[
    \Gamma: x^{2}-y^{2}=1
    \] 
    的右顶点为 $A$,将圆心在 $y$ 轴上, 且与 $\Gamma$ 的两支各恰有一个公共点的圆称为“好圆”, 若两个好圆外切于点 $P$, 圆心距为 $d$, 求 $\dfrac{d}{PA}$ 的所有可能值。
    \begin{solution}
        考虑以 $(0, y_{0})$ 为圆心的好圆 
        \[
        \Omega_{0}: x^{2}+(y-y_{0})^{2}=r_{0}^{2}\left(r_{0}>0\right)
        \]
        由 $\Omega_{0}$ 与 $\Gamma$ 的方程消去 $x$, 得关于 $y$ 的二次方程
        \[
        2y^{2}-2y_{0}y+y_{0}^{2}+1-r_{0}^{2}=0
        \]
        令判别式为零,
        \[
        \Delta=4y_{0}^{2}-8\left(y_{0}^{2}+1-r_{0}^{2}\right)=0\Rightarrow y_{0}^{2}=2r_{0}^{2}-2
        \]
        对于外切于点 $P$ 的两个好圆 $\Omega_{1}, \Omega_{2}$, 显然 $P$ 在 $y$ 轴上;设 $P(0, h)$, $\Omega_{1}, \Omega_{2}$ 的半径分别为 $r_{1}, r_{2}$, 不妨设 $\Omega_{1}, \Omega_{2}$ 的圆心分别为 $(0, h+r_{1}), (0, h-r_{2})$, 则有
        \[
        (h+r_{1})^{2}=2r_{1}^{2}-2,(h-r_{2})^{2}=2r_{2}^{2}-2
        \]
        两式相减得 
        \[
        2h(r_{1}+r_{2})=r_{1}^{2}-r_{2}^{2}
        \]
        而 $r_{1}+r_{2}>0$, 故化简得 $h=\dfrac{r_{1}-r_{2}}{2}$,进而 
        \[
        \left(\frac{r_{1}-r_{2}}{2}+r_{1}\right)^{2}=2r_{1}^{2}-2 \Rightarrow r_{1}^{2}-6r_{1}r_{2}+r_{2}^{2}+8=0 \tag{1}
        \]
        由于 $d=r_{1}+r_{2},A(1,0)$,且 
        \[
        PA^{2}=h^{2}+1=\frac{(r_{1}-r_{2})^{2}}{4}+1
        \]
        而 (1)可等价地写为
        \[
        2(r_{1}-r_{2})^{2}+8=(r_{1}+r_{2})^{2}
        \]
        即 $8PA^{2}=d^{2}$, 所以 
        \[
        \frac{d}{PA}=2\sqrt{2}
        \]
    \end{solution}
    
    \question 设$A,B$为双曲线
    \[
    W: \frac{x^{2}}{a^{2}}-\frac{y^{2}}{b^{2}}=1
    \]
    与实轴的交点, $P(0,1)$为双曲线外一点,$PA,PB$分别交双曲线于另一点$C,D$,过$C,D$的切线相交于$Q$,若$\triangle QCD$是一个正三角形且面积为$\dfrac{16\sqrt{3}}{27}$,求双曲线$W$的方程式。
    \begin{solution}
        设
        \[
        A(-a,0),B(a,0),C(-a \sec \alpha,b \tan \alpha),D(a \sec \alpha,b \tan \alpha)
        \]
        且$\sec \alpha > 0$,从而$PB$的直线方程为
        \[
        y=-\frac{1}{a}x+1
        \] 
        $D$在$PB$上,于是有
        \[
        b\tan \alpha=-\sec \alpha+1 \tag{1}
        \]
        且$QC,QD$的直线方程为
        \[
        -\frac{x \sec \alpha}{a}-\frac{y \tan \alpha}{b}=1,\quad \frac{x \sec \alpha}{a}-\frac{y \tan \alpha}{b}=1
        \]
        两式联立得
        \[
        Q\left(0,-\dfrac{b}{\tan \alpha}\right)
        \]
        又由对称性及$\triangle QCD$是正三角形可得
        \[
        \frac{1}{2}\cdot2a \sec \alpha\cdot\sqrt{3}a \sec \alpha=\frac{16\sqrt{3}}{27},\quad -\frac{b}{\tan \alpha}-b \tan \alpha=\sqrt{3}a \sec \alpha
        \]
        即
        \[
        a^{2}\sec^{2}\alpha=\frac{16}{27},\quad -b \sec \alpha=\sqrt{3}a \tan \alpha
        \]
        联立(1)可得 
        \[
        \sec \alpha=\frac{\sqrt{3}a}{\sqrt{3}a-b^{2}},\quad \tan \alpha=\frac{-b}{\sqrt{3}a-b^{2}}
        \]
        由 $1+\tan^{2}\alpha=\sec^{2}\alpha$得$b^{2}=2\sqrt{3}a-1$,代入$a \sec \alpha=\dfrac{4}{3\sqrt{3}}$得
        \[
        9a^{2}+4\sqrt{3}a-4=0\Rightarrow a=\frac{2\sqrt{3}}{9}, b=\frac{\sqrt{3}}{3}
        \]
        所以双曲线$W$方程为
        \[
        \frac{27x^{2}}{4}-3y^{2}=1
        \]
    \end{solution}

    \question 已知双曲线 
    \[
    \frac{x^2}{a^2}-\frac{y^2}{b^2}=1\ (a>0,b>0)
    \]
    的左、右焦点分别为 $F_1, F_2$。  
    过点 $F_1$ 作直线分别交双曲线左支和一条渐近线于点 $A,B$($A,B$ 在同一象限),且满足 $F_1A = AB$。  
    连接 $AF_2$,若满足 $AF_2 \perp BF_1$,求该双曲线离心率的平方$e^2$。
    \begin{solution}
        不妨设 $A(x_0,y_0)$在第二象限,由 $AF_2\perp BF_1$ 得 
        \[
        \frac{y_0}{x_0-c}\cdot\frac{y_0}{x_0+c}=-1 \Rightarrow y_0^2+x_0^2-c^2=0 \tag{1}
        \]
        因为$A$ 在双曲线上,所以 
        \[
        \frac{x_0^2}{a^2}-\frac{y_0^2}{b^2}=1 \Rightarrow
        x_0^2=a^2+\frac{a^2y_0^2}{b^2}
        \]
        代入(1)解得 $y_0=\dfrac{b^2}{c}$;又$ F_1A=AB$,所以$ B(2x_0+c,2y_0)$;又 $B$ 在渐近线 $y=-\dfrac{b}{a}x$ 上,
        \[
        2y_0=-\frac{b}{a}(2x_0+c) \Rightarrow -2bx_0=2ay_0+bc
        \] 
        两边平方得 
        \[
        b^2(2x_0+c)^2=a^2(2y_0)^2 \Rightarrow 4b^2x_0^2+4b^2cx_0+b^2c^2=4a^2y_0^2 \tag{2}
        \]
        将 $x_0^2=a^2+\dfrac{a^2y_0^2}{b^2}$ 和 $y_0=\dfrac{b^2}{c}$ 代入(2)得 
        \[
        4b^2(a^2+\frac{a^2b^4}{b^2c^2})+4b^2c\sqrt{a^2+\frac{a^2b^4}{b^2c^2}}+b^2c^2=4a^2\frac{b^4}{c^2} 
        \]
        解得 
        \[
        3a^2-4ab-b^2=0\Rightarrow \frac{a}{b}=\frac{\sqrt{7}+2}{3} 
        \]
        于是
        \[
        e^2=1+\frac{b^2}{a^2}=1+\frac{9}{(\sqrt{7}+2)^2}\cdot\frac{(\sqrt7-2)^2}{(\sqrt7-2)^2}=12-4\sqrt{7}
        \] 
    \end{solution}

    \question 设双曲线 
    \[
    C: \frac{x^{2}}{a^{2}} - \frac{y^{2}}{b^{2}} = 1, \quad a > 0, b > 0
    \]
    的左、右焦点分别为 $F_1, F_2$。过点 $F_1$ 作斜率为 $\dfrac{\sqrt{3}}{3}$ 的直线 $l$,与双曲线左右两支分别交于点 $M, N$。且满足
    \[
    (\overrightarrow{F_2 M} + \overrightarrow{F_2 N}) \cdot \overrightarrow{M N} = 0.
    \]
    求双曲线的离心率。
        \begin{solution}
        设$D$为$MN$的中点,连接$F_{2}D$,易知$\overrightarrow{F_{2}M}+\overrightarrow{F_{2}N}=2\overrightarrow{F_{2}D}$,所以
        \[
        (\overrightarrow{F_{2}M}+\overrightarrow{F_{2}N})\cdot\overrightarrow{MN}=2\overrightarrow{F_{2}D}\cdot\overrightarrow{MN}=0\Rightarrow F_{2}D \perp MN
        \]
        所以$F_{2}M=F_{2}N$,现设$t=F_{2}M=F_{2}N$,
        由双曲线定义
        \[
        MF_{1}=t-2a,\;NF_{1}=t+2a\Rightarrow MN=NF_{1}-MF_{1}=4a
        \]
        又$D$是$MN$的中点,
        \[
        MD=ND=2a,\;F_{1}D=F_{1}M+MD=t
        \]
        在直角$\triangle F_{1}DF_{2}$及直角$\triangle MDF_{2}$中,
        \[
        F_{2}D=\sqrt{4c^{2}-t^{2}}=\sqrt{t^{2}-4a^{2}}\Rightarrow t^{2}=2a^{2}+2c^{2}
        \]
        所以 
        \[
        F_{1}D=\sqrt{2c^{2}-2a^{2}}, F_{2}D=t=\sqrt{2a^{2}+2c^{2}}
        \]
        直线的斜率为 $\frac{\sqrt{3}}{3}$,
        \[
        \tan\angle DF_{1}F_{2}=\frac{F_{2}D}{F_{1}D}=\frac{\sqrt{2c^{2}-2a^{2}}}{\sqrt{2a^{2}+2c^{2}}}=\frac{\sqrt{3}}{3}
        \]
        所以
        \[
        \frac{c^{2}-a^{2}}{a^{2}+c^{2}}=\frac{1}{3}\Rightarrow c=\sqrt{2}a
        \]
        离心率为 
        \[
        \frac{c}{a}=\sqrt{2}
        \]
    \end{solution}
    
    \question 已知双曲线 
    \[
    C:\frac{x^{2}}{a^{2}}-\frac{y^{2}}{b^{2}}=1\quad a>0, b>0
    \] 
    的左焦点为 $F_1$,离心率为 $e$, 直线 $y = kx\ (k \ne 0)$ 分别与 $C$ 的左右两支交于点 $M, N$,  
    若 $\triangle MF_1N$ 的面积为 $\sqrt{3}$,且 $\angle MF_1N = 60^\circ$,求 $e^2 + 3a^2$ 的最小值。
    \begin{solution}
        连接$NF_{2},MF_{2}$,由对称性可知四边形$MF_{1}NF_{2}$为平行四边形,故
        \[
        NF_{2}=MF_{1},NF_{1}=MF_{2},\angle F_{1}NF_{2}=120^{\circ},S_{\triangle F_{1}NF_{2}}=S_{\triangle MF_{1}N}=\sqrt{3}
        \]
        由面积公式得
        \[
        \frac{1}{2}NF_{1} \cdot NF_{2}\sin120^{\circ}=\sqrt{3}\Rightarrow NF_{1}\cdot NF_{2}=4
        \]
        由$F_{1}N-F_{2}N=2a$,在三角形$F_{1}NF_{2}$中,由余弦定理得
        \begin{align*}
        \cos120^{\circ}
        &=\frac{F_{1}N^{2}+F_{2}N^{2}-4c^{2}}{2F_{1}N\cdot F_{2}N} \\
        &=\frac{(F_{1}N-F_{2}N)^{2}+2F_{1}N\cdot F_{2}N-4c^{2}}{2F_{1}N\cdot F_{2}N} \\
        &=\frac{(2a)^{2}+2\cdot4-4c^{2}}{2\cdot4}\\
        &=\frac{a^{2}+2-c^{2}}{2}=-\frac{1}{2}
        \end{align*}
        解得$b^{2}=c^{2}-a^{2}=3$,由AM-GM不等式,
        \[
        e^{2}+3a^{2}=1+\frac{b^{2}}{a^{2}}+3a^{2}=1+\frac{3}{a^{2}}+3a^{2}\ge1+2\sqrt{\frac{3}{a^{2}}\cdot3a^{2}}=7
        \]
        当且仅当$\dfrac{3}{a^{2}}=3a^{2}$即$a^{2}=1$时等号成立。
    \end{solution}

    \question 点 \(A(-1,1)\),\(B,C\) 在双曲线 \(x^{2}-y^{2}=1\) 上,且 \(\triangle ABC\) 为正三角形。求其面积。  
    \begin{solution}
    $\frac{3\sqrt3}{2}$
    \textcolor{red}{(待解)}
    \end{solution}    

\end{questions}
\pagebreak

\begin{center}
  {\fontsize{30pt}{26pt}\selectfont
    \hypertarget{坐标变换}{坐标变换} \label{坐标变换}
  }
\end{center}
\separator
\vspace{1pt}
\begin{questions}
    \question 已知平移坐标轴到新原点 $O'(h, k)$ 后,二直线 $2x + 3y - 4 = 0$、$x - 2y + 1 = 0$ 在新坐标中的方程式分别变为 $2x' + 3y' - 3 = 0$ 及 $x' - 2y' + 5 = 0$,求 $(h, k)$。
    \begin{solution}
        设平移公式为 $x = x' + h, y = y' + k$,代入原直线方程整理得
        \[
        \begin{cases} 
        2x' + 3y' + (2h + 3k - 4) = 0 \\
        x' - 2y' + (h - 2k + 1) = 0 
        \end{cases}
        \]
        对照已知新坐标系下的方程,
        \[
        \begin{cases} 
        2h + 3k - 4 = -3 \\
        h - 2k + 1 = 5 
        \end{cases}
        \]
        解得
        \[
        (h, k) = (2, -1)
        \]
    \end{solution}

    \question 已知当坐标轴旋转 $\theta$ 角时,点 $M(-\sqrt{2}, \sqrt{2})$ 的新坐标是 $(0, 2)$。求 $(1, 1)$ 的原坐标。
    \begin{solution}
        点 $M(-\sqrt{2}, \sqrt{2})$ 的新坐标是 $(0, 2)$,故
        \[
        \begin{bmatrix} 0 \\ 2 \end{bmatrix} 
        = \begin{bmatrix} \cos\theta & \sin\theta \\ -\sin\theta & \cos\theta \end{bmatrix} \begin{bmatrix} -\sqrt{2} \\ \sqrt{2} \end{bmatrix}= \begin{bmatrix} -\sqrt{2} \cos\theta + \sqrt{2}\sin\theta \\ \sqrt{2}\sin\theta + \sqrt{2}\cos\theta \end{bmatrix}
        \]
        给出的两方程皆有
        \[
        \theta = 45^\circ
        \]
        故
        \[
        \begin{bmatrix} x \\ y \end{bmatrix} 
        = \begin{bmatrix} \cos 45^\circ & -\sin 45^\circ \\ \sin 45^\circ & \cos 45^\circ \end{bmatrix} \begin{bmatrix} 1 \\ 1 \end{bmatrix}
        = \begin{bmatrix} 0 \\ \sqrt{2} \end{bmatrix}
        \]
        $(1, 1)$ 的原坐标为 $(0, \sqrt{2})$。
    \end{solution}

    \question 以 $O$ 为原点的 $xy$ 平面上,取两点 $A(\sqrt{3},1),B(-1,\sqrt{3}),t \in \mathbb{R}$,点 $P$ 满足
    \[
    \overrightarrow{OP} = t^2 \overrightarrow{OA} + t \overrightarrow{OB},
    \]
    求 $P$ 点的轨迹与 $x$ 轴所围成的图形面积。
    \begin{solution}
        设 $P(x,y)$,则
        \[
        \begin{bmatrix} x \\ y \end{bmatrix} = 
        \begin{bmatrix}
        \sqrt{3} & -1 \\
        1 & \sqrt{3}
        \end{bmatrix}
        \begin{bmatrix}
        t^2 \\ t
        \end{bmatrix}=
        \begin{bmatrix}
        \cos 30^\circ & -\sin 30^\circ \\
        \sin 30^\circ & \cos 30^\circ
        \end{bmatrix}
        \begin{bmatrix}
        x' \\ y'
        \end{bmatrix}
        \]
        其中
        \[
        \begin{bmatrix} x' \\ y' \end{bmatrix} =
        \begin{bmatrix}
        \frac{t^2}{2} \\
        \frac{t}{2}
        \end{bmatrix}.
        \]
        在新坐标系下,轨迹满足
        \[
        \Gamma: y'^2 = 2 x' \tag{1}
        \]
        即一条抛物线,既然旋转保持面积不变,即求新坐标系下所围面积,将$x$轴顺时针转$30^\circ$得
        \[
        L: y' = -\frac{1}{\sqrt{3}} x' \tag{2}
        \]
        联立$(1),(2)$解得
        \[
        O(0,0), \quad A(6, -2 \sqrt{3})
        \]
        所围成图形面积为
        \[
        \int_{-2\sqrt{3}}^{0} \left(-\sqrt{3} y - \frac{y^2}{2}\right) dy = 2 \sqrt{3}.
        \]
    \end{solution}

    \question 坐标平面上有一椭圆 
    \[
    \Gamma_1: \frac{x^2}{b^2} + \frac{y^2}{a^2} = 1
    \] 
    以原点 $O(0,0)$ 为中心,将椭圆 $\Gamma_1$ 逆时针旋转 $\dfrac{\pi}{3}$ 后得到椭圆 
    \[
    \Gamma_2: 43x^2 + 14\sqrt{3}xy + 57y^2 = 576
    \] 
    求椭圆 $\Gamma_1$ 的面积。
    \begin{solution}
        有
        \[
        43x^2 + 14\sqrt{3}xy + 57y^2 = 
        \begin{bmatrix} x & y \end{bmatrix}
        \begin{bmatrix} 43 & 7\sqrt{3} \\ 7\sqrt{3} & 57 \end{bmatrix} 
        \begin{bmatrix} x \\ y \end{bmatrix}
        \]
        设矩阵 
        \[
        A = \begin{bmatrix} 43 & 7\sqrt{3} \\ 7\sqrt{3} & 57 \end{bmatrix}
        \] 
        特征值为
        \[
        \lambda_1 = 36, \quad \lambda_2 = 64
        \]
        于是旋转后的标准形式为
        \[
        36x'^2 + 64y'^2 = 576 \Rightarrow \frac{x'^2}{16} + \frac{y'^2}{9} = 1
        \]
        由此可得半轴长
        \[
        a = 4, \quad b = 3
        \]
        故椭圆面积为
        \[
        S = \pi a b = 12 \pi
        \]
    \end{solution}

    \question 设椭圆 
    \[
    \frac{x^{2}}{25}+\frac{y^{2}}{9}=1
    \]
    的右焦点为 $F(4,0)$,点 $P$ 为椭圆上一动点,若以 $PF$ 为一边作正方形 $FPQR$($FPQR$ 按逆时针方向排列),当 $P$ 点沿着椭圆绕行一周时,试求 $R$ 点的轨迹方程式。
    \begin{figure}[H]
        \centering        
        \includegraphics[width=0.5\textwidth]{images/image103.jpg}
    \end{figure}
    \begin{solution}
        以 $F$ 为旋转中心,将 $P$ 逆时针旋转 $90^\circ$ 即为 $R$。因此 $P(5\cos\theta, 3\sin \theta)$ 先平移 $(-4,0)$,得
        \[
        P' = (5\cos\theta - 4, \ 3\sin \theta)
        \]
        再逆时针旋转 $90^\circ$,有
        \[
        P'' = 
        \begin{bmatrix} 
        0 & -1 \\ 
        1 & 0 
        \end{bmatrix}
        \begin{bmatrix} 
        5\cos\theta - 4 \\ 
        3\sin \theta 
        \end{bmatrix}
        =
        \begin{bmatrix} 
        -3\sin \theta \\ 
        5\cos\theta - 4
        \end{bmatrix}
        \]
        再平移 $(4,0)$,得
        \[
        R = (-3\sin \theta + 4, \ 5\cos\theta - 4)
        \]
        即
        \[
        \begin{cases}
        x = -3\sin \theta + 4 \\
        y = 5\cos\theta - 4
        \end{cases}
        \]
        由此得
        \[
        \frac{(x-4)^2}{9} + \frac{(y+4)^2}{25} = 1
        \]
    \end{solution}

    \question 将椭圆 
    \[
    \Gamma_1: \frac{x^2}{9} + \frac{y^2}{4} = 1
    \]
    以原点 $O$ 为中心,逆时针旋转锐角 $\theta$ 后,得到椭圆 $\Gamma_2$。已知 $\Gamma_2$ 的长轴方程为 $y=2x$,求 $\Gamma_2$ 的方程。
    \begin{solution}
        椭圆 $$\Gamma_1: \frac{x^2}{9} + \frac{y^2}{4} = 1$$ 的长轴在 $x$ 轴上,长轴旋转至 $y=2x$,设旋转角满足 $\tan \theta = 2$,则
        \[
        \sin \theta = \frac{2}{\sqrt{5}}, \quad \cos \theta = \frac{1}{\sqrt{5}}.
        \]
        设 $(x, y)$ 是 $\Gamma_2$ 上一点,对应 $\Gamma_1$ 上某点 $(s, t)$,则有
        \[
        \begin{bmatrix} x \\ y \end{bmatrix} =
        \begin{bmatrix} \frac{1}{\sqrt{5}} & -\frac{2}{\sqrt{5}} \\ \frac{2}{\sqrt{5}} & \frac{1}{\sqrt{5}} \end{bmatrix}
        \begin{bmatrix} s \\ t \end{bmatrix}.
        \]
        所以反解得
        \[
        \begin{bmatrix} s \\ t \end{bmatrix} =
        \begin{bmatrix} \frac{1}{\sqrt{5}}(x+2y) \\ \frac{1}{\sqrt{5}}(-2x+y) \end{bmatrix}.
        \]
        因为 $(s,t)$ 满足 $\Gamma_1$:
        \[
        \frac{s^2}{9} + \frac{t^2}{4} = 1,
        \]
        代入得
        \[
        \frac{\frac{1}{5}(x+2y)^2}{9} + \frac{\frac{1}{5}(-2x+y)^2}{4} = 1,
        \]
        即
        \[
        8x^2 - 4xy + 5y^2 - 36 = 0.
        \]
    \end{solution}

    \question 在坐标平面上,考虑二阶方阵 
    \[
    A=\frac{1}{5}\begin{bmatrix}3&-4\\ 4&3\end{bmatrix}
    \]
    所定义的线性变换。对于平面上异于原点 $O$ 的点 $P_1$,设 $P_1$ 经 $A$ 变换成 $P_2,P_2$ 经 $A$ 变换成 $P_3$。假设 $P_1$ 是图形 $y=\dfrac{1}{10}x^{2}-10$ 上的动点,求 $\triangle P_1P_2P_3$ 面积的最小可能值。
    \begin{solution}
        我们有
        \[
        A = \frac{1}{5}\begin{bmatrix}3 & -4 \\ 4 & 3 \end{bmatrix} 
        = \begin{bmatrix} \frac{3}{5} & -\frac{4}{5} \\ \frac{4}{5} & \frac{3}{5} \end{bmatrix} 
        = \begin{bmatrix} \cos\theta & -\sin\theta \\ \sin\theta & \cos\theta \end{bmatrix}
        \Rightarrow \sin \theta = \frac{4}{5},\ \cos \theta = \frac{3}{5}
        \]
        故
        \[
        \sin \angle P_1OP_2 = \sin 2\theta = 2\sin\theta\cos\theta = \frac{24}{25}
        \]
        假设 $OP_1=a$,则
        \begin{align*}
        [\triangle P_1P_2P_3] &= [\triangle OP_1P_2] + [\triangle OP_2P_3] - [\triangle OP_1P_3]\\
        &= \frac{1}{2}a^2 \sin \theta + \frac{1}{2}a^2 \sin \theta - \frac{1}{2}a^2 \sin 2\theta 
        = \frac{8}{25}a^2
        \end{align*}
        又$P_1$  在 $y = \dfrac{x^2}{10} - 10$ 上,设$P_1\left(m, \dfrac{m^2}{10} - 10\right)$,则
        \[
        a^2 = \frac{1}{100}m^4 - m^2 + 100 = \frac{1}{100}(m^2 - 50)^2 + 75
        \]
        故
        \[
        [\triangle P_1P_2P_3]_{\min}=\frac{8}{25} \cdot 75=24
        \]
    \end{solution}

    \question 在直角坐标平面上,圆 $x^{2}+y^{2}=1$ 先被变换为曲线 $\Gamma_{1}$,再被
    \[
    A=\begin{bmatrix}\frac{\sqrt{3}}{2} & \sqrt{3} \\[4pt] \frac{1}{2} & -3\end{bmatrix}
    \]
    变换为曲线 $\Gamma_{1}$,再被
    \[
    B=\prod_{k=8}^{67}\begin{bmatrix}\cos k^\circ & \sin k^\circ \\[4pt] \sin k^\circ & -\cos k^\circ\end{bmatrix}
    \]
    变换为曲线 $\Gamma_{2}$,求 $\Gamma_{2}$ 的方程。
    \begin{solution}
        首先注意到
        \[
        \begin{bmatrix}\cos a^\circ & \sin a^\circ\\[4pt] \sin a^\circ & -\cos a^\circ \end{bmatrix}
        \begin{bmatrix}\cos (a+1)^\circ & \sin (a+1)^\circ\\[4pt] \sin (a+1)^\circ & -\cos (a+1)^\circ \end{bmatrix}
        =\begin{bmatrix}\cos 1^\circ & \sin 1^\circ \\[4pt] -\sin 1^\circ & \cos 1^\circ \end{bmatrix}.
        \]
        由此,
        \[
        B=\begin{bmatrix}\cos 1^\circ & \sin 1^\circ \\[4pt] -\sin 1^\circ & \cos 1^\circ \end{bmatrix}^{30}
        =\begin{bmatrix}\cos 30^\circ & \sin 30^\circ \\[4pt] -\sin 30^\circ & \cos 30^\circ \end{bmatrix}
        =\begin{bmatrix}\tfrac{\sqrt3}{2} & \tfrac{1}{2} \\[4pt] -\tfrac{1}{2} & \tfrac{\sqrt3}{2} \end{bmatrix}.
        \]
        所以
        \[
        BA=\begin{bmatrix}\tfrac{\sqrt3}{2} & \tfrac{1}{2} \\[4pt] -\tfrac{1}{2} & \tfrac{\sqrt3}{2} \end{bmatrix}
        \begin{bmatrix}\tfrac{\sqrt3}{2} & \sqrt3 \\[4pt] \tfrac{1}{2} & -3\end{bmatrix}
        =\begin{bmatrix}1 & 0 \\[4pt] 0 & -2\sqrt3 \end{bmatrix}.
        \]
        设变换后的坐标为 $(x',y')^{\mathsf T}=BA\,(x,y)^{\mathsf T}$,则
        \[
        x'=x,\qquad y'=-2\sqrt3\,y.
        \]
        原方程 $x^{2}+y^{2}=1$ 在新坐标下变为
        \[
        x'^{2}+\Big(\frac{y'}{-2\sqrt3}\Big)^{2}=1 \Rightarrow x'^{2}+\frac{y'^{2}}{12}=1.
        \]
        即
        \[
        \Gamma_{2}:\quad x^{2}+\frac{y^{2}}{12}=1.
        \]
    \end{solution}
\end{questions}
\pagebreak

\begin{center}
  {\fontsize{30pt}{26pt}\selectfont
    \hypertarget{轨迹方程式、参数方程式}{轨迹方程式、参数方程式} \label{轨迹方程式、参数方程式}
  }
\end{center}
\separator
\vspace{1pt}

\begin{questions}
    \question 设 $A(-1,0),B(2,0)$, 在 $\triangle ABC$ 中, 若 $\angle B = 2 \angle A$, 求点 $C$ 的轨迹方程式。
    \ifprintanswers
    \begin{figure}[H]
        \centering
        \includegraphics[width=0.5\linewidth]{images/image65.png}
    \end{figure}
    \fi
    \begin{solution}
        设 $C(x,y)$,作 $\angle B$ 的角平分线交 $AC$ 于点 $D$,由
        \[
        \angle B = 2 \angle A
        \]
        知$D$ 在 $AB$ 的垂直平分线上,故$D\left(\dfrac{1}{2}, a\right)$,又由角平分线定理,
        \[
        \frac{AD}{DC} = \frac{AB}{BC} = \frac{3}{\sqrt{(x-2)^2 + y^2}}
        \]
        由分比公式
        \[
        x_D=\frac{1}{2} = \frac{-\sqrt{(x-2)^2 + y^2} + 3x}{\sqrt{(x-2)^2 + y^2} + 3}
        \]
        得点 $C$ 的轨迹
        \[
        x^2 - \frac{y^2}{3} = 1, \quad x > 1
        \]
    \end{solution}

    \question 一抛物线 $y^2 = 4x$ 与一直线交于 $A,B$ 两点,已知抛物线与直线所围出的面积为 $\dfrac{9}{8}$, 求 $A,B$ 的中点轨迹方程。
    \begin{solution}
        不失一般性,设交点为$A\left(\dfrac{1}{4}y_1^2, y_1\right), B\left(\dfrac{1}{4}y_2^2, y_2\right),y_2\le y_1$,斜率
        \[
        m_L = \frac{y_1 - y_2}{\frac{1}{4}(y_1^2 - y_2^2)} = \frac{4}{y_1 + y_2},
        \]
        则直线方程为
        \[
        x = \frac{1}{4}(y_1 + y_2)y - \frac{1}{4}y_1 y_2.
        \]
        所围部分面积为
        \[
        \frac{9}{8} = \int_{y_2}^{y_1} \left( \frac{1}{4}(y_1 + y_2)y - \frac{1}{4}y_1 y_2 - \frac{1}{4}y^2 \right) dy
        = \frac{1}{24} (y_1 - y_2)^3
        \]
        由此得
        \[
        y_1 - y_2 = 3.
        \]
        设 $P$ 为 $AB$ 中点
        \[
        P\Big(\frac{1}{8}(y_1^2 + y_2^2), \frac{1}{2}(y_1 + y_2)\Big) \equiv (x, y)
        \]
        由于
        \[
        y^2 = \frac{1}{4}(y_1 + y_2)^2 = \frac{1}{4}(y_1^2 + y_2^2) + \frac{1}{2}y_1 y_2
        = \frac{1}{2}(y_1^2 + y_2^2) - \frac{9}{4} = 4x - \frac{9}{4},
        \]
        所以中点轨迹方程为
        \[
        y^2 = 4x - \frac{9}{4}.
        \]
    \end{solution}

    \question 设 $AB$ 为抛物线 $y^2=4ax$ 之一动弦, 且 $\angle AOB = 90^\circ$, 其中 $O$ 为原点,从抛物线外一点$T$作点$A,B$的切线,且$A,B$处的法线交于点$N$,证明 $TN$ 之中点轨迹方程式为 
    \[
    2y^2 = 25a(x-a).
    \]
    \begin{solution}
        设 $A(at_1^2, 2at_1),B(at_2^2, 2at_2)$,已知$OA \perp OB$,故
        \[
        m_{OA}\cdot m_{OB}=\frac{2at_1}{at_1^2} \cdot \frac{2at_2}{at_2^2} = -1 \Rightarrow t_1 t_2 = -4
        \]
        由链导法,在点$A(at_1^2, 2at_1)$的切线斜率为
        \[
        \frac{dy}{dx}=\frac{dy}{dt} \div \frac{dy}{dt}=\frac{2a}{2at_1}=\frac{1}{t_1}
        \]
        故在 $A$的切线方程为
        \[
        y-2at_1=\frac{1}{t}(x-at_1^2) \Rightarrow t_1y - x = at_1^2 \tag{1}
        \]
        同理, 点$B$ 的切线方程为
        \[
        t_2y - x = at_2^2 \tag{2}
        \]
        联立$(1),(2)$,解得切线交点 $T$
        \[
        T(a t_1 t_2,a(t_1 + t_2))=(-4a,a(t_1 + t_2))
        \]
        在$A$的法线方程为
        \[
        y - 2at_1 = -t_1(x - at_1^2) \Rightarrow t_1 x + y = a t_1^3 + 2at_1 \tag{3}
        \]
        同理, 点$B$ 的切线方程为
        \[
        t_2 x + y = a t_2^3 + 2at_2 \tag{4}
        \]
        联立$(3),(4)$,解得 $N$ 点坐标
        \[
        N(a(t_1+t_2)^2 + 6a, 4a(t_1+t_2))
        \]
        故$TN$ 中点坐标为
        \[
        M\left(\frac{1}{2}a(t_1+t_2)^2 + a,\frac{5a}{2}(t_1+t_2)\right)
        \]
        消去参数$t_1+t_2$,即得
        \[
        2y^2 = 25a(x-a)
        \]
    \end{solution}

    \question 抛物线$\Gamma: y^2 = 4x$, $F$为$\Gamma$的焦点, $A, B$为$\Gamma$上的两个不重合的动点, 使得线段$AB$的一个三等分点$P$位于线段$OF$上(含端点), 记$Q$为线段$AB$的另一个三等分点, 求$Q$的轨迹方程.
    \begin{solution}
        设$A(x_1, y_1),B(x_2, y_2)$. 不妨设$\overrightarrow{AP} = \overrightarrow{PQ} = \overrightarrow{QB}$, 则
        \[
        P\left(\frac{2x_1+x_2}{3}, \frac{2y_1+y_2}{3}\right)
        \]
        易知$F(1,0)$. 由于点$P$位于线段$OF$上, 故
        \[
        \frac{2x_1+x_2}{3}\in[0,1],\ \frac{2y_1+y_2}{3}=0
        \]
        设$y_1 = t,y_2 = -2t$, 则$x_1 = \dfrac{t^2}{4},x_2 = t^2$. 此时有
        \[
        \frac{2x_1+x_2}{3}=\frac{t^2}{2}\in[0,1]
        \]
        且由$A, B$不重合知$t\neq0$, 所以$t^2\in(0, 2]$.
        设$Q(x_Q, y_Q)$, 则
        \[
        x_Q = \frac{x_1+2x_2}{3} = \frac{3}{4}t^2, y_Q = \frac{y_1+2y_2}{3} = -t
        \]
        故有 $y_Q^2 = \dfrac{4}{3}x_Q$,注意到 $x_Q = \dfrac{3}{4}t^2 \in \left[0, \dfrac{3}{2}\right]$, 故点$Q$的轨迹方程为 
        \[
        y^2 = \frac{4}{3}x
        \]
    \end{solution}

    \question 已知抛物线 $\Gamma:y = x^2$ 上三个不同点 $A(1,1),B,C$ 满足 $AB \perp AC$,过 $B,C$ 分别作 $\Gamma$ 的切线交于点 $P$,求点 $P$ 的轨迹方程。
    \begin{solution}
        设 $B(x_1, x_1^2),C(x_2, x_2^2)$,直线 $BC$ 的方程为 $y = kx + m$,联立$\Gamma:y = x^2$得
        \[
        x^2 - kx - m = 0
        \]
        由韦达定理
        \[
        x_1 + x_2 = k,\quad x_1 x_2 = -m
        \]
        于是
        \[
        y_1 + y_2 = x_1^2 + x_2^2 = (x_1 + x_2)^2 - 2x_1 x_2 = k^2 + 2m
        \]
        \[
        y_1 y_2 = x_1^2 x_2^2 = (x_1 x_2)^2 = m^2
        \]
        由 $AB \perp AC$,即 $\overrightarrow{AB} \cdot \overrightarrow{AC} = 0$,有
        \[
        (x_1 - 1)(x_2 - 1) + (y_1 - 1)(y_2 - 1) = 0
        \]
        展开整理得
        \[
        x_1 x_2 - (x_1 + x_2) + 1 + y_1 y_2 - (y_1 + y_2) + 1 = 0
        \]
        代入上式各值,得到
        \[
        -m - k + 1 + m^2 - (k^2 + 2m) + 1 = 0
        \Rightarrow m^2 - 3m - k^2 - k + 2 = 0
        \]
        解得
        \[
        m = k + 2 \quad \text{或} \quad m = -k + 1
        \]
        若 $m = -k + 1$,则直线 $BC$ 方程为 $y = k(x - 1) + 1$,经过 $A(1,1)$,舍去;因此 $m = k + 2,BC$ 方程为
        \[
        y = k(x + 1) + 2
        \]
        即 $BC$ 恒过定点 $(-1, 2)$。分别作点 $B(x_1, x_1^2)$ 与 $C(x_2, x_2^2)$ 的切线方程:
        \[
        y - x_1^2 = 2x_1(x - x_1)\; ,\; y - x_2^2 = 2x_2(x - x_2) 
        \]
        联立两式得切线交点$P$,
        \[
        x = \frac{x_1 + x_2}{2} = \frac{k}{2}, \quad y = x_1 x_2 = -m = -k-2
        \]
        消去参数 $k$即得点P的轨迹方程
        \[
        2x + y + 2 = 0
        \]
    \end{solution}

    \question 已知点 $P$ 在一半径为 $r$ 的圆的内部,在 $P$ 作一直角,使得直角的两边分别与圆相交于 $A,B$ 两点。设点 $Q$ 使得 $PAQB$ 构成矩形,求点 $Q$ 的轨迹。
    \begin{solution}
        不失平移及旋转的一般性,设圆心为原点 $O$,点 $P(d,0)$为圆
        \[
        x^2+y^2=r^2,
        \]
        内一点其中 $d=OP,0<d<r$,设在点 $P$ 作的直角的两条边的方向分别与 $x$ 轴成角 $\theta,\theta+\dfrac{\pi}{2}$,对应的单位方向向量分别为
        \[
        (\cos\theta,\sin\theta),\quad(-\sin\theta,\cos\theta).
        \]
        故可设
        \[
        A\left(d+s\cos\theta,\;s\sin\theta\right),\quad B\left(d-t\sin\theta,\;t\cos\theta\right)
        \]
        其中 $s,t>0,$ 为 $PA,PB$ 的长度,由于 $A,B$ 在圆上,有
        \[
        (d+s\cos\theta)^2+(s\sin\theta)^2=r^2,\quad 
        (d-t\sin\theta)^2+(t\cos\theta)^2=r^2
        \]
        整理得
        \[
        s^2=r^2-d^2-2ds\cos\theta,\quad t^2=r^2-d^2+2dt\sin\theta \tag{1}
        \]
        由于$PAQB$为矩形,点 $Q$坐标为
        \[
        Q(x,y)=\left(d+s\cos\theta-t\sin\theta,\;s\sin\theta+t\cos\theta\right)
        \]
        观察
        \begin{align*}
        x^2+y^2
        &=\bigl(d+s\cos\theta-t\sin\theta\bigr)^2+\bigl(s\sin\theta+t\cos\theta\bigr)^2 \\
        &=d^2+2d\bigl(s\cos\theta-t\sin\theta\bigr)+s^2+t^2.
        \end{align*}
        将$(1)$代入得
        \begin{align*}
        x^2+y^2
        &=d^2+2d\bigl(s\cos\theta-t\sin\theta\bigr)
        +2(r^2-d^2)+2d\bigl(t\sin\theta-s\cos\theta\bigr) \\
        &=2r^2-d^2.
        \end{align*}
        右式与 $\theta,s,t$ 无关,仅与 $r$ 及 $d=OP$ 有关。因此
        \[
        OQ^2=2r^2-d^2
        \]
        为常数。由此可知,在适当的平移、旋转后的坐标系中,点 $Q$ 的轨迹是以原点为圆心、半径为 $\sqrt{2r^2-d^2}$ 的圆。将坐标系平移回去,得点 $Q$ 的轨迹为以圆心 $O(h,k)$ 为圆心、半径为 $\sqrt{2r^2-OP^2}$ 的圆。
    \end{solution}
    \begin{solution}
        不失一般性,设圆心为原点 $O$, 则向量 $\mathbf{p}=\overrightarrow{OP},\ |\mathbf{p}|^2<r^2$,且
        \[
        \mathbf{v}=\overrightarrow{PA},\quad \mathbf{w}=\overrightarrow{PB},\quad \mathbf{a}=\overrightarrow{OA},\quad \mathbf{b}=\overrightarrow{OB},
        \]
        其中 $\mathbf{v}\perp\mathbf{w},|\mathbf{a}|=|\mathbf{b}|=r$,且$\mathbf{a}=\mathbf{p}+\mathbf{v},\mathbf{b}=\mathbf{p}+\mathbf{w}$,设 $\mathbf{q}=\overrightarrow{OQ}=\mathbf{p}+\mathbf{v}+\mathbf{w}$,于是
        \begin{align*}
        |\mathbf{q}|^2
        &=(\mathbf{p}+\mathbf{v}+\mathbf{w})\cdot(\mathbf{p}+\mathbf{v}+\mathbf{w})\\
        &=|\mathbf{p}|^2+2\mathbf{p}\cdot\mathbf{v}+2\mathbf{p}\cdot\mathbf{w}+\mathbf{v}\cdot\mathbf{v}+\mathbf{w}\cdot\mathbf{w} \\
        &=|\mathbf{p}|^2+\mathbf{a}\cdot\mathbf{v}+\mathbf{b}\cdot\mathbf{w}+\mathbf{p}\cdot\mathbf{v}+\mathbf{p}\cdot\mathbf{w} \\
        &=|\mathbf{p}|^2+(\mathbf{a}+\mathbf{p})\cdot\mathbf{v}+(\mathbf{b}+\mathbf{p})\cdot\mathbf{w} \\
        &=|\mathbf{p}|^2+(\mathbf{a}+\mathbf{p})\cdot(\mathbf{a}-\mathbf{p})+(\mathbf{b}+\mathbf{p})\cdot(\mathbf{b}-\mathbf{p}) \\
        &=|\mathbf{p}|^2+|\mathbf{a}|^2-|\mathbf{p}|^2+|\mathbf{b}|^2-|\mathbf{p}|^2 \\
        &=2r^2-|\mathbf{p}|^2。
        \end{align*}
        因此 $Q$ 到圆心 $O$ 的距离为常数, 所以 $Q$ 的轨迹是以 $O$ 为圆心、半径为 $\sqrt{2r^2-|\mathbf{p}|^2}$ 的圆。
    \end{solution}

    \question 设 $A(1,\sqrt{3})$, $B(1,-\sqrt{3})$ 为平面上两定点,动点 $P$ 在线段 ${AB}$ 上。$O$ 为原点,且 $Q$ 在射线 $OP$ 上,满足 ${OP} \cdot {OQ} = 4$。当动点 $P$ 从 $A$ 沿线段 ${AB}$ 移动到 $B$,求点 $Q$ 轨迹的路径长。
    \ifprintanswers
    \begin{figure}[H]
    \centering
    \includegraphics[width=0.5\textwidth]{images/image27.png}
    \end{figure}
    \fi
    \begin{solution}
        设$P$ 点坐标为$P(1,t), -\sqrt{3} \le t \le \sqrt{3},$则
        \[
        {OP} = \sqrt{1 + t^2},
        \]
        $Q$ 在射线 $OP$ 上,则设
        \[
        Q(x,y), \quad y = t x, \quad x \ge 0.
        \]
        由条件${OP} \cdot {OQ} = 4,$
        \[
        \sqrt{1 + t^2} \cdot \sqrt{x^2 + y^2} = 4.
        \]
        代入 $y = t x$,得
        \[
        \sqrt{1 + t^2} \cdot \sqrt{x^2 + t^2 x^2} = (1 + t^2) |x| = 4 \Rightarrow x = \frac{4}{1 + t^2}.
        \]
        点 $Q$ 坐标为
        \[
        Q\left(\frac{4}{1+t^2}, \frac{4 t}{1 + t^2}\right).
        \]
        由 $x = \frac{4}{1+t^2}$,得$t^2 = \frac{4}{x} -1$,所以
        \[
        y^2 = t^2 x^2 = \left(\frac{4}{x} -1 \right) x^2 = 4 x - x^2.
        \]
        即
        \[
        (x - 2)^2 + y^2 = 2^2,
        \]
        为以 $(2,0)$ 为圆心,半径为 $2$ 的圆;由于 $P$ 在 ${AB}$,则 $1 \le {OP} \le 2$,即
        \[
        1 \le \sqrt{1 + t^2} \le 2 \implies 1 \le x = \frac{4}{1+t^2} \le 4,
        \]
        故点 $Q$ 轨迹为圆周在 $x \ge 1$ 的部分,即为该圆的三分之二周长,路径长为
        \[
        \frac{2}{3} \cdot 4 \pi = \frac{8 \pi}{3}.
        \]
    \end{solution}

    \question 已知曲线 $C$ 是到点 $P\!\left(-\dfrac12,\dfrac38\right)$ 和到直线 $y=-\dfrac58$ 距离相等的点的轨迹。直线 $l$ 过点 $Q(-1,0)$,点 $M$ 是 $C$ 上(不在 $l$ 上)的动点;$A,B$ 在 $l$ 上,且 $MA\perp l,\ MB\perp x$ 轴。
    \begin{parts}
    \part 求曲线 $C$ 的方程;
    \begin{solution}
        设 $N(x,y)$ 为 $C$ 上的点,据题意有
        \[
        \sqrt{\left(x+\frac12\right)^2+\left(y-\frac38\right)^2}=\left|y+\frac58\right|.
        \]
        两边平方并化简得
        \[
        y=\frac12(x^2+x)
        \]
    \end{solution}
    \part 求直线 $l$ 的方程,使 $\dfrac{QB^2}{QA}$ 为常数。
    \end{parts}
    \begin{solution}
        设$M\left(x,\frac{x^2+x}{2}\right),l: y=kx+k\ (k\neq0)$,则 $B(x,kx+k)$,从而
        \[
        QB=\sqrt{1+k^2}\,|x+1|
        \]
        在直角 $\triangle QMA$ 中,由毕氏定理,
        \[
        QA^2=QM^2-MA^2=(x+1)^2\!\left(1+\frac{x^2}{4}\right)-\frac{(x+1)^2\left(k-\frac{x}{2}\right)^2}{1+k^2}=\frac{(x+1)^2}{4(1+k^2)}(kx+2)^2,
        \]
        即
        \[
        QA=\frac{|x+1||kx+2|}{2\sqrt{1+k^2}}
        \]
        故欲使
        \[
        \frac{QB^2}{QA}
        =\frac{2(1+k^2)\sqrt{1+k^2}}{|k|}
        \left|\frac{x+1}{x+\frac{2}{k}}\right|
        \]
        为常数,需 $\dfrac{2}{k}=1$即 $k=2$,此时$\dfrac{QB^2}{QA}=5\sqrt5$,故直线 $l$ 的方程为
        \[
        l: 2x-y+2=0
        \]
    \end{solution}
    \begin{solution}
        设$M\left(x,\frac{x^2+x}{2}\right), l: y=kx+k$,则
        \[
        QB=\sqrt{1+k^2}\,|x+1|
        \]
        过 $Q(-1,0)$ 作垂直于 $l$ 的直线
        \[
        l_1: y=-\frac1k(x+1)
        \]
        设$H$ 为 $M$ 到 $l_1$ 的垂足,因 $QA=MH$,得
        \[
        QA=\frac{|x+1||kx+2|}{2\sqrt{1+k^2}}
        \]
        于是
        \[
        \frac{QB^2}{QA}
        =\frac{2(1+k^2)\sqrt{1+k^2}}{|k|}
        \left|\frac{x+1}{x+\frac{2}{k}}\right|
        \]
        同理得 $k=2$,从而
        \[
        l: 2x-y+2=0
        \]
    \end{solution}

    \question 已知平面内动点 $P$ 到 $F(1,0)$ 的距离与点 $P$ 到 $y$ 轴的距离差为 $1$,求:
    \ifprintanswers
    \begin{figure}[H]
    \centering
    \includegraphics[width=0.35\textwidth]{images/image41.png}
    \end{figure}
    \fi
    \begin{parts}
    \part 此动点 $P$ 的轨迹方程。
    \begin{solution}
        即 $P$ 到 $F(1,0)$ 的距离与 $P$ 到直线 $x = -1$ 的距离相等,依抛物线定义,轨迹为
        \[
        y^2 = 4x
        \]
    \end{solution}
    \part 过 $F$ 作两条互相垂直的直线 $L_1, L_2$。设 $L_1$ 与 $P$ 点轨迹相交于点 $A$ 和 $B,L_2$ 与 $P$ 点轨迹相交于点 $C,D$,求 $\overrightarrow{AD} \cdot \overrightarrow{CB}$ 的最小值。
\begin{solution}
极值出现在 $\overline{AB} = \overline{CD}$,可假设:

$A\left(\frac{a^2}{4}, a\right)$,$B\left(\frac{b^2}{4}, -b\right)$,$C\left(\frac{b^2}{4}, b\right)$,$D\left(\frac{a^2}{4}, -a\right)$

则:

\[
\overrightarrow{AD} = (0, -2a), \quad \overrightarrow{CB} = (0, -2b)
\Rightarrow \overrightarrow{AD} \cdot \overrightarrow{CB} = 4ab
\]

设 $P, Q$ 分别为 $A, B$ 在 $x$ 轴的垂足,则:

$\triangle APF \backsim \triangle BQF$(AAA)

\[
\frac{\overline{AP}}{\overline{BQ}} = \frac{\overline{AF}}{\overline{BF}} = \frac{d(L, A)}{d(L, B)}
\]

即:

\[
\frac{a}{b} = \frac{\frac{a^2}{4} + 1}{\frac{b^2}{4} + 1} = \frac{a^2 + 4}{b^2 + 4}
\]

两边交叉相乘得:

\[
ab^2 + 4a = a^2b + 4b
\Rightarrow ab(a - b) - 4(a - b) = 0
\Rightarrow (ab - 4)(a - b) = 0
\]

若 $a = b$,则 $\overline{CD}$ 为 $x$ 轴,不符合条件。

故取 $ab = 4$,则:

\[
\overrightarrow{AD} \cdot \overrightarrow{CB} = 4ab = 16
\]
\textcolor{red}{(待验证,为何极值出现在 $\overline{AB} = \overline{CD}$)}
\end{solution}
\end{parts}

    \question 在曲线
    \[
    \left(\frac{x}{a}\right)^{\frac{2}{3}} + \left(\frac{y}{b}\right)^{\frac{2}{3}} = 1,
    \]
    上的点 $P$ 处作曲线的切线,该切线与 $x$ 轴交于 $(h, 0)$,与 $y$ 轴交于 $(0, k)$。当 $P$ 沿给定曲线移动时,求点 $Q(h, k)$ 的轨迹。
    \begin{solution}
        考虑参数变换$P(x,y)=(a \cos^3 \theta,b \sin^3 \theta)$,则
        \[
        \frac{dy}{dx} 
        = \frac{dy}{d\theta} \div \frac{dx}{d\theta}
        = \frac{3b \sin^2 \theta \cos \theta}{3a \cos^2 \theta (-\sin \theta)} = -\frac{b}{a} \tan \theta
        \]
        在任意点 $\theta$ 处的切线方程为
        \[
        y - b \sin^3 \theta = -\frac{b}{a} \tan \theta (x - a \cos^3 \theta).
        \]
        整理得
        \[
        b x \sin \theta + a y \cos \theta = a b \sin \theta \cos \theta
        \]
        分别令$x=0,y=0$可得
        \[
        k = b \sin \theta,\quad h = a \cos \theta
        \]
        消去参数 $\theta$即得$Q(h, k)$的轨迹方程式
        \[
        \left(\frac{h}{a}\right)^2 + \left(\frac{k}{b}\right)^2 = 1.
        \]
        因此,点 $Q(h, k)$ 的轨迹是一个椭圆,其方程为
        \[
        \frac{x^2}{a^2} + \frac{y^2}{b^2} = 1.
        \]
    \end{solution}

    \question 等边三角形 $ABC$ 的边长为 2。正方形 $PQRS$ 满足 $P$ 在边 $AB$ 上,$Q$ 在边 $BC$ 上,$R,S$ 在边 $AC$ 上。若动点 $S$ 从边 $AC$ 移动到边 $AB$,使得 $P, Q, R, S$ 始终在三角形边上或内部构成一个正方形,证明点 $S$ 的轨迹是一条平行于 $BC$ 的直线。
    \begin{figure}[H]
        \centering        
        \includegraphics[width=0.4\textwidth]{images/image188.png}
    \end{figure}
    \ifprintanswers
    \begin{figure}[H]
        \centering        
        \includegraphics[width=0.4\textwidth]{images/image189.png}
    \end{figure}
    \fi
    \begin{solution}
        设正方形的边长为 $s$,且$\angle RQC=\theta$。$D,P,F$为 $R,S,T$ 在底边$BC$的垂足,过 $S$ 作平行于 $BC$ 的直线,交从 $R$ 作的垂线于 $E$。
        由 $\triangle RQD,\triangle SER$, $S$ 到 $BC$ 的距离为
        \[
        RD+ER = s\sin\theta+s\cos\theta,
        \]
        欲证其为常数。在 $\triangle RDQ,\triangle TFQ$ 中,
        \[
        QD = s\cos\theta,\quad FQ = s\sin\theta,\quad TF = s\cos\theta.
        \]
        在 $\triangle RDC,\triangle TFB$ 中,
        \[
        DC=RD\tan 30^\circ=\frac{\sqrt{3}}{3}s\sin\theta, \quad
        BF=TF\tan 30^\circ=\frac{\sqrt{3}}{3}s\cos\theta,  
        \]
        故
        \[
        2=DC+QD+FQ+BF
        =\frac{\sqrt{3}}{3}(s\cos\theta+s\sin\theta)+(s\cos\theta+s\sin\theta)
        \]
        即
        \[
        RD+ER = s\sin\theta+s\cos\theta=\frac{2}{\frac{\sqrt{3}}{3}+1}
        \]
        为一常数,于是得证点 $S$ 的轨迹是一条平行于 $BC$ 的直线。
    \end{solution}

    \question 求内接于抛物线 $y^2=4cx$ ($c>0$) 的正三角形的重心所形成的轨迹方程。
    \begin{solution}
        设内接抛物线 $\Gamma: y^2 = 4cx$ 的正三角形顶点为
        \[
        A\left(\frac{\alpha^2}{4c}, \alpha\right), 
        B\left(\frac{\beta^2}{4c}, \beta\right),
        C\left(\frac{\gamma^2}{4c}, \gamma\right),
        \]
        则重心为
        \[
        G\left(\frac{\alpha^2+\beta^2+\gamma^2}{12c},\, \frac{\alpha+\beta+\gamma}{3}\right)
        \]
        设各边斜率
        \[
        m_1 = \frac{4c(\beta-\alpha)}{(\beta^2-\alpha^2)} = \frac{4c}{\beta+\alpha},
        m_2 = \frac{4c(\gamma-\beta)}{(\gamma^2-\beta^2)} = \frac{4c}{\gamma+\beta},
        m_3 = \frac{4c(\alpha-\gamma)}{(\alpha^2-\gamma^2)} = \frac{4c}{\alpha+\gamma}
        \]
        由正三角形性质:
        \[
        \tan 60^\circ = \sqrt{3} = \frac{m_2-m_1}{1+m_1 m_2} = \frac{m_3-m_2}{1+m_2 m_3} = \frac{m_1-m_3}{1+m_3 m_1}
        \]
        化简得
        \[
        \alpha\beta + \beta\gamma + \gamma\alpha = -16c^2 - \frac{\alpha^2+\beta^2+\gamma^2}{3}
        \]
        现设 $G=(x,y)$,则
        \[
        9y^2 = (\alpha+\beta+\gamma)^2 = \alpha^2+\beta^2+\gamma^2 + 2(\alpha\beta+\beta\gamma+\gamma\alpha) = 12cx + 2(-16c^2 - 4cx) = 4cx - 32c^2
        \]
        因此正三角形重心轨迹方程为
        \[
        9y^2 = 4cx - 32c^2
        \]
    \end{solution}
    
    \question 如图,一长度为 4 单位的刚性杆 $AB$穿过位于定点 $M(1,0)$ 的铰链滑动,该铰链允许杆在 $x-y$ 平面内的任意方向转动。杆端$A$在 $y$ 轴上滑动使得 $OA \le 4$,令 $\theta$ 为杆在正 $x$ 轴的倾角。
    \begin{figure}[H]
        \centering        
        \includegraphics[width=0.5\textwidth]{images/image235.png}
    \end{figure}
    \begin{parts}
    \part 证明当 $A$ 在 $y$ 轴上滑动时,$B$ 的轨迹满足参数方程
    \[
    x=4\cos\theta, \quad y=4\sin\theta-\tan\theta, \quad -\theta_0 \le \theta \le \theta_0,
    \]
    并给出 $\theta_0$ 的值。
    \begin{solution}
        在$\triangle AOM$中,
        \[
        \cos \theta = \frac{1}{AM} \Rightarrow AM = \sec \theta
        \]
        于是$MB=4-\sec \theta$,由几何关系得到 $B$的坐标,
        \begin{align*}
        x &= OM + MB\cos\theta = 4\cos\theta, \\
        y &= MB\sin\theta = 4\sin\theta - \tan\theta
        \end{align*}
        且明显地,
        \[
        \theta_0 = \tan^{-1} 4
        \]
    \end{solution}
    \part 证明轨迹的直角坐标方程为
    \[
    y^2 = \frac{(16-x^2)(x-1)^2}{x^2}.
    \]
    \begin{solution}
        写成
        \[
        y = 4\sin\theta - \tan\theta = \sin\theta \left(4 - \frac{1}{\cos\theta}\right), 
        \]
        两边平方得
        \[
        y^2 = \sin^2\theta \frac{(4\cos\theta - 1)^2}{\cos^2\theta} = (1 - \cos^2\theta) \frac{(4\cos\theta - 1)^2}{\cos^2\theta}
        \]
        将 $\cos\theta = \dfrac{x}{4}$代入得
        \[
        y^2 = \left(1 - \frac{x^2}{16}\right) \frac{(x-1)^2}{\frac{x^2}{16}} 
        \]
        即
        \[
        y^2 = \frac{(16-x^2)(x-1)^2}{x^2}.
        \]
    \end{solution}
    \end{parts}

    \question 直线
    \[
    L:\frac{x}{p}+\frac{y}{q}=1
    \]
    其中 $p,q\neq 0$满足
    \[
    \frac{1}{p^2}+\frac{1}{q^2}=\frac{1}{2}.
    \]
    点 $P$ 为原点 $O$ 到直线 $L$ 的垂足。证明无论 $p,q$ 取何值,点 $P$ 恒在一圆 $C$ 上,并求该圆的半径。
    \begin{solution}
        直线方程$L$即
        \[
        y=q-\frac{q}{p}x
        \]
        于是$OP$直线方程为
        \[
        y=\frac{p}{q}x
        \]
        联立两式,解得点 $P$ 的坐标为
        \[
        P\left(\frac{pq^2}{p^2+q^2},\frac{p^2q}{p^2+q^2}\right)
        \]
        将已知条件
        \[
        \frac{1}{p^2}+\frac{1}{q^2}=\frac{1}{2}\Rightarrow p^2+q^2=\frac{1}{2}p^2q^2
        \]
        代入点 $P$ 的坐标得
        \[
        P\left(\frac{2}{p},\frac{2}{q}\right)
        \]
        因此点 $P$ 的轨迹满足
        \[
        x^2+y^2=4\left(\frac{1}{p^2}+\frac{1}{q^2}\right)=2
        \]
        即点 $P$ 恒在以原点为圆心、半径为 $\sqrt{2}$ 的圆上。
    \end{solution}

    \question 由椭圆 
    \[
    \frac{x^2}{a^2}+\frac{y^2}{b^2}=1
    \]
    的焦点 $F$,作椭圆上一动点$P(\cos\theta, b\sin\theta)$的切线的垂线,求该垂足 $N$ 的轨迹。
    \begin{solution}
        设 $P(a\cos\theta, b\sin\theta)$ 为椭圆上一动点,切线斜率为
        \[
        \frac{dy}{dx} = -\frac{b^2 x}{a^2 y} = -\frac{b\cos\theta}{a\sin\theta}
        \]
        切线方程为
        \[
        y - b\sin\theta = -\frac{b\cos\theta}{a\sin\theta} (x - a\cos\theta) \Rightarrow bx\cos\theta + ay\sin\theta = ab.
        \]
        过焦点 $F(ae,0)$,垂直于切线的直线 $FN$:
        \[
        y - 0 = \frac{a\sin\theta}{b\cos\theta}(x - ae) \Rightarrow -ay\cos\theta + bx\sin\theta = a^2 e \sin\theta.
        \]
        设 $N(x,y)$ 为交点,解方程组
        \[
        \begin{cases}
        bx\cos\theta + ay\sin\theta = ab, \\
        - ay\cos\theta + bx\sin\theta = a^2 e \sin\theta.
        \end{cases}
        \]
        两式平方相加得
        \[
        (bx\cos\theta + ay\sin\theta)^2 + (-ay\cos\theta + bx\sin\theta)^2 = (ab)^2 + (a^2 e \sin\theta)^2.
        \]
        \[
        x^2(b^2\cos^2\theta + b^2\sin^2\theta) + y^2(a^2\sin^2\theta + a^2\cos^2\theta) = a^2 b^2 + a^4 e^2 \sin^2\theta.
        \]
        由 $b^2 = a^2(1-e^2)$化简得垂足 $N$ 的轨迹为
        \[
        x^2 + y^2 = a^2
        \]
    \end{solution}

    \question 椭圆由参数方程
    \[
    x=3\sqrt{2}\cos\theta,\quad y=4\sin\theta,\quad 0<\theta<2\pi
    \]
    给出。
    \begin{parts}
    \part 求椭圆的焦点坐标及准线方程。
    \begin{solution}
        由已知得
        \[
        \cos\theta=\frac{x}{3\sqrt{2}},\quad \sin\theta=\frac{y}{4}
        \]
        平方相加得
        \[
        \frac{x^2}{18}+\frac{y^2}{16}=1
        \]
        因此
        \[
        a^2=18,b^2=16, \Rightarrow e=\frac{1}{3}
        \]
        焦点为
        \[
        (\pm ae,0)=(\pm\sqrt{2},0)
        \]
        准线方程为
        \[
        x=\pm\frac{a}{e}=\pm 9\sqrt{2}
        \]
    \end{solution}
    \part 证明椭圆在任意点的切线方程为
    \[
    \frac{y\sin\theta}{4}+\frac{x\cos\theta}{3\sqrt{2}}=1
    \]
    \begin{solution}
        由参数方程求导,
        \[
        \frac{dy}{dx}=\frac{4\cos\theta}{-3\sqrt{2}\sin\theta}
        \]
        在点 $(3\sqrt{2}\cos\theta,4\sin\theta)$ 处的切线为
        \[
        y-4\sin\theta=\frac{4\cos\theta}{-3\sqrt{2}\sin\theta}(x-3\sqrt{2}\cos\theta)
        \]
        整理得
        \[
        \frac{y\sin\theta}{4}+\frac{x\cos\theta}{3\sqrt{2}}=1
        \]
    \end{solution}
    \part 一条过原点的直线与上述切线相交于点 $P$,证明点 $P$ 的轨迹满足
    \[
    (x^2+y^2)^2=2(9x^2+8y^2).
    \]
    \begin{solution}
        设过原点 $O$ 且垂直于切线的直线方程为
        \[ 
        y = \frac{3\sqrt{2} \sin \theta}{4 \cos \theta} x 
        \]
        将其与切线方程联立得
        \[
        \frac{3\sqrt{2} \sin^2 \theta}{16 \cos \theta} x + \frac{\cos \theta}{3\sqrt{2}} x = 1 \Rightarrow
        x = \frac{24\sqrt{2} \cos \theta}{8 + \sin^2 \theta},y=\frac{36 \sin \theta}{8 + \sin^2 \theta}
        \]
        考虑$\dfrac{y}{x}$,可得
        \[ 
        \frac{y}{x} = \frac{3}{2\sqrt{2}} \tan \theta \Rightarrow \tan \theta = \frac{2\sqrt{2} y}{3x} 
        \]
        由恒等式 $\sec^2 \theta = 1 + \tan^2 \theta$,将 $x$ 的表达式写成
        \[ x = \frac{24\sqrt{2} \sec \theta}{8 \sec^2 \theta + \tan^2 \theta} = \frac{24\sqrt{2} \sec \theta}{9 \tan^2 \theta + 8} \]
        平方后,将$\tan^2 \theta = \dfrac{8y^2}{9x^2}$代入,
        \[
        x^2 = \frac{1152 (1 + \tan^2 \theta)}{(8 + 9 \tan^2 \theta)^2} 
        = \frac{18 \left( 1 + \frac{8y^2}{9x^2} \right)}{\left( 1 + \frac{y^2}{x^2} \right)^2}
        = \frac{2x^2 (9x^2 + 8y^2)}{(x^2 + y^2)^2}
        \]
        故$P$轨迹方程为
        \[ 
        (x^2 + y^2)^2 = 2(9x^2 + 8y^2).
        \]
    \end{solution}
    \end{parts}

    \question 已知双曲线 $H$ 与直线 $L$ 的方程分别为
    \[
    H: \frac{x^{2}}{a^{2}}-\frac{y^{2}}{b^{2}}=1, \quad L: y=mx+c,
    \]
    其中 $a,b,m,c$ 均为非零实数。
    \begin{parts}
    \part 证明直线 $L$ 与双曲线 $H$ 的交点的 $x$ 坐标满足方程
    \[
    (a^{2}m^{2}-b^{2})x^{2}+2a^{2}mcx+a^{2}(b^{2}+c^{2})=0
    \]
    \begin{solution}
        将 $y=mx+c$ 代入双曲线方程,
        \[
        \frac{x^{2}}{a^{2}}-\frac{(mx+c)^{2}}{b^{2}}=1
        \]
        展开整理得
        \[
        (a^{2}m^{2}-b^{2})x^{2}+2a^{2}mcx+a^{2}(b^{2}+c^{2})=0
        \]
    \end{solution}
    \part 已知直线 $L$ 为双曲线 $H$ 的切线,证明
    \[
    a^{2}m^{2}=b^{2}+c^{2}
    \]
    \begin{solution}
        已知 $L$ 为切线,则上述关于 $x$ 的二次方程有重根,其判别式为零:
        \[
        \Delta = 4a^{4}m^{2}c^{2}-4a^{2}(a^{2}m^{2}-b^{2})(b^{2}+c^{2})=0
        \]
        化简得
        \[
        b^{2}(b^{2}+c^{2}-a^{2}m^{2})=0
        \]
        由于 $b\neq0$,故得证
        \[
        a^{2}m^{2}=b^{2}+c^{2}
        \]
    \end{solution}
    \part 求经过点 $(1,4)$ 且与双曲线
    \[
    \frac{x^{2}}{25}-\frac{y^{2}}{16}=1
    \]
    相切的两条切线方程,并分别求出切点坐标。
    \begin{solution}
        由题意 $a^{2}=25,b^{2}=16$,切线$y=mx+c$经过点 $(1,4)$,则
        \[
        m=4-c
        \]
        由切线条件 $a^{2}m^{2}=b^{2}+c^{2}$,解得
        \[
        25(4-c)^{2}=16+c^{2} \Rightarrow c=3 \quad \text{或} \quad c=\frac{16}{3}
        \]
        对应
        \[
        m=1 \quad \text{或} \quad m=-\frac{4}{3}
        \]
        故两条切线方程为
        \[
        y=x+3, \quad y=-\frac{4}{3}x+\frac{16}{3}
        \]
        现求切点坐标。当 $y=x+3$ 时,代入双曲线解得
        \[
        x=-\frac{25}{3},\quad y=-\frac{16}{3}
        \]
        当 $y=-\dfrac{4}{3}x+\dfrac{16}{3}$ 时,代入双曲线解得
        \[
        x=\frac{25}{4},\quad y=-3
        \]
        因此两条切线及其切点分别为
        \[
        y=x+3,\; \left(-\frac{25}{3},-\frac{16}{3}\right)\quad \text{及} \quad y=-\frac{4}{3}x+\frac{16}{3},\; \left(\frac{25}{4},-3\right)
        \]
    \end{solution}
    \end{parts}

    \question 已知双曲线 
    \[
    \frac{x^2}{a^2}-\frac{y^2}{b^2}=1
    \] 
    上一点 $P$,过 $P$ 的法线分别交 $x$ 轴、$y$ 轴于 $A,B$,若 $OABQ$ 为矩形,求点 $Q$ 的轨迹方程。
    \begin{solution}
        设 $P(a\sec\theta, b\tan\theta)$ 为双曲线上的动点,有
        \[
        \frac{dy}{dx} = \frac{b^2 x}{a^2 y}.
        \]
        在 $P$ 点的切线斜率为$\dfrac{b\sec\theta}{a\tan\theta}$,法线斜率为$-\dfrac{a\tan\theta}{b\sec\theta}$,故$P$处的法线方程为
        \[
        y - b\tan\theta = -\frac{a\tan\theta}{b\sec\theta} (x - a\sec\theta) \Rightarrow a\tan\theta\, x + b\sec\theta\, y - \frac{ab}{\cos\theta} = 0
        \]
        与坐标轴交点分别为
        \[
        A\left(\frac{a^2+b^2}{a}\sec\theta, 0\right),\quad B\left(0, -\frac{a^2+b^2}{b}\tan\theta\right)
        \]
        由矩形性质,$Q$ 点坐标为
        \[
        Q\left(\frac{a^2+b^2}{a}\sec\theta, -\frac{a^2+b^2}{b}\tan\theta \right)
        \]
        利用恒等式 $\sec^2\theta - \tan^2\theta = 1$,得点 $Q$ 的轨迹方程为
        \[
        \left(\frac{ax}{a^2+b^2}\right)^2 - \left(\frac{by}{a^2+b^2}\right)^2 = 1 \Rightarrow a^2 x^2 - b^2 y^2 = (a^2+b^2)^2
        \]
    \end{solution}

    \question 动点 $P$ 在矩形双曲线
    \[
    xy=a^{2}
    \]
    上运动,其中 $a>0$ 为常数。过点 $P$ 作双曲线的法线,该法线再次与双曲线相交于点 $Q$。设 $M$ 为线段 $PQ$ 的中点。求点 $M$ 的轨迹方程。
    \begin{solution}
        设点 $P\left(ap, \frac{a}{p}\right)$,其中 $p \neq 0$。由链导法,在点 $P(ap, \frac{a}{p})$ 处切线的斜率为
        \[
        \frac{dy}{dx}=\frac{dy}{dp} \div \frac{dx}{dp} = \frac{-\frac{a}{p^2}}{a}=-\frac{1}{p^{2}}
        \]
        从而 $P$ 处法线方程为
        \[
        y - \frac{a}{p} = p^{2}(x - ap)
        \]
        与$xy=a^{2}$联立解得
        \[
        \frac{a^{2}}{x} - \frac{a}{p} = p^{2}(x - ap) \Rightarrow Q\left(-\frac{a}{p^{3}}, -ap^{3}\right)
        \]
        其中$x = ap$对应点$P$;点 $M(x, y)$ 为 $PQ$ 的中点, 其坐标为
        \[
        x = \frac{ap - \frac{a}{p^{3}}}{2} = \frac{a(p^{4}-1)}{2p^{3}}, \quad y = \frac{\frac{a}{p} - ap^{3}}{2} = \frac{a(1-p^{4})}{2p}
        \]
        两式相除得
        \[
        p^{2} = -\frac{y}{x}
        \]
        将之带入$y$ 的表达式平方得
        \[
        y^{2} = \frac{a^{2}\left(1 - \frac{y^{2}}{x^{2}}\right)^{2}}{4\left(-\frac{y}{x}\right)} = \frac{a^{2}(x^{2}-y^{2})^{2}}{-4x^{3}y}
        \]
        整理得点 $M$ 的轨迹方程为
        \[
        a^{2}(x^{2}-y^{2})^{2} + 4x^{3}y^{3} = 0
        \]
    \end{solution}

    \question 点 $P,Q$ 为曲线
    \[
    xy=1,\ x\in\mathbb{R} \setminus \{0\}
    \]
    上的两个相异点,使得线段 $PQ$ 为曲线在 $P$ 处的法线。曲线在 $P,Q$ 处的切线相交于点 $R$。证明点 $R$ 的轨迹方程为
    \[
    (y^2-x^2)^2+4xy=0。
    \]
    \begin{solution}
        对曲线$y=\frac{1}{x}$求导得
        \[
        \frac{dy}{dx}=-\frac{1}{x^2}
        \]
        设$P\left(p,\dfrac{1}{p}\right), Q\left(q,\dfrac{1}{q}\right), p\neq q$,弦 $PQ$ 的斜率为
        \[
        \frac{\frac{1}{p}-\frac{1}{q}}{p-q}=-\frac{1}{pq}
        \]
        曲线在 $P$ 处切线斜率为 $-\frac{1}{p^2}$,故法线斜率为 $p^2$。
        由于 $PQ$ 为 $P$ 处的法线,有
        \[
        \left(-\frac{1}{pq}\right)\left(-\frac{1}{p^2}\right)=-1 \Rightarrow q=-\frac{1}{p^2}
        \]
        曲线在 $P$ 处的切线方程为
        \[
        y-\frac{1}{p}=-\frac{1}{p^2}(x-p) \Rightarrow y=\frac{2}{p}-\frac{1}{p^2}x
        \]
        同理,曲线在 $Q$ 处的切线方程为
        \[
        y=\frac{2}{q}-\frac{1}{q^2}x
        \]
        联立两切线方程解得交点为
        \[
        R\left(\frac{2pq}{p+q},\frac{2}{p+q}\right)
        \]
        由 $q=-\dfrac{1}{p^2}$,
        \[
        x = \frac{2p\left(-\frac{1}{p^2}\right)}{p-\frac{1}{p^2}}= -\frac{2p}{p^3-1},\quad
        y = \frac{2}{p-\frac{1}{p^2}}= \frac{2p^2}{p^3-1}
        \]
        于是由
        \[
        p=-\frac{y}{x}
        \]
        可得点 $R$ 的轨迹方程
        \[
        (y^2-x^2)^2+4xy=0
        \]
    \end{solution}

    \question 已知动点
    \[
    P\left(5t,\frac{5}{t}\right),\ t\neq 0
    \]
    在双曲线
    \[
    xy=25
    \]
    上。
    \begin{parts}
    \part 证明该双曲线在点 $P$ 处的法线方程为
    \[
    y=t^2x+\frac{5}{t}-5t^3
    \]
    \begin{solution}
        对双曲线$y=\dfrac{25}{x}$求导得
        \[
        \frac{dy}{dx}=-\frac{25}{x^2}
        \]
        在$P$处的斜率为
        \[
        \frac{dy}{dx}=-\frac{25}{(5t)^2}=-\frac{1}{t^2}
        \]
        故法线斜率为 $t^2$。点 $P\left(5t,\dfrac{5}{t}\right)$ 处法线方程为
        \[
        y-\frac{5}{t}=t^2(x-5t) \Rightarrow  y=t^2x+\frac{5}{t}-5t^3
        \]
    \end{solution}
    \part 该法线再次与双曲线相交于点 $Q$,证明点 $Q$ 的坐标为
    \[
    \left(-\frac{5}{t^3},-5t^3\right)
    \]
    \begin{solution}
        联立
        \[
        xy=25,\quad y=t^2x+\frac{5}{t}-5t^3
        \]
        化简得
        \[
        t^3x^2+(5-5t^4)x-25t=0
        \]
        由于 $x=5t$ 为一根,因式分解得
        \[
        (x-5t)(t^3x+5)=0
        \]
        故另一交点满足
        \[
        x=-\frac{5}{t^3} \Rightarrow y=\frac{25}{x}=-5t^3
        \]
        因此得证
        \[
        Q\left(-\frac{5}{t^3},-5t^3\right)
        \]
    \end{solution}
    \part 证明线段 $PQ$ 的中点轨迹的笛卡尔方程为
    \[
    4xy+25\left(\frac{y}{x}-\frac{x}{y}\right)^2=0
    \]
    \begin{solution}
        设 $M$ 为 $PQ$ 的中点,则
        \[
        M\left(\frac{5}{2}\left(t-\frac{1}{t^3}\right),\frac{5}{2}\left(\frac{1}{t}-t^3\right)\right)
        \]
        设
        \[
        x=\frac{5}{2}\left(t-\frac{1}{t^3}\right),\quad
        y=\frac{5}{2}\left(\frac{1}{t}-t^3\right)
        \]
        则
        \[
        \frac{x}{y}=-\frac{1}{t^2} \Rightarrow t^2=-\frac{y}{x}
        \]
        将$y$表达式平方得
        \[
        4y^2t^2=25(1-t^4)^2
        \]
        代入$t^2=-\dfrac{y}{x}$得,
        \[
        -4xy^3=25\frac{(x^2-y^2)^2}{x^3}
        \]
        即
        \[
        4xy+25\left(\frac{y}{x}-\frac{x}{y}\right)^2=0
        \]
    \end{solution}
    \end{parts}

    \question 曲线由参数方程给出
    \[
    x = \sin^2 t, \quad y = \sin t \cos t + \cos t, \quad 0 \le t < 2\pi.
    \]
    证明其直角坐标方程为
    \[
    (x^2 + y^2 - 1)^2 = 4x(1-x)^2.
    \]
    \begin{solution}
        将 $y$ 写为
        \[
        y = \cos t (\sin t + 1) \implies y^2 = \cos^2 t (\sin t + 1)^2.
        \]
        利用恒等式 $\cos^2 t = 1 - \sin^2 t$,并令 $x = \sin^2 t$:
        \[
        y^2 = (1 - x)(x + 2\sin t + 1) \Rightarrow \frac{y^2}{1-x} - (1+x) = 2\sin t
        \]
        两边平方,
        \[
        \left( \dfrac{y^2 - (1-x^2)}{1-x} \right)^2 = 4 \sin^2 t = 4x
        \]
        于是所求直角坐标方程为
        \[
        (y^2 + x^2 - 1)^2 = 4x (1-x)^2
        \]
    \end{solution}

    \question 曲线 $C$ 由参数方程
    \[
    x = \tan\theta - \sec\theta, \quad y = \cot\theta - \csc\theta, \quad 0 < \theta < \frac{\pi}{2}.
    \]
    给出,证明
    \begin{parts}
    \part $C$ 的直角坐标方程为
    \[
    (x^2 - 1)(y^2 - 1) = 4xy,
    \]
    \begin{solution}
        从 $x = \tan\theta - \sec\theta$ 开始:
        \begin{align*}
        x^2 &= (\tan\theta - \sec\theta)^2 \\
        &= \tan^2\theta - 2\tan\theta\sec\theta + \sec^2\theta \\
        &= \tan^2\theta - 2\tan\theta\sec\theta + (1+\tan^2\theta) \\
        &= 2\tan^2\theta - 2\tan\theta\sec\theta + 1 \\
        &= 2\tan\theta(\tan\theta - \sec\theta) + 1 \\
        &= 2\tan\theta \cdot x + 1
        \end{align*}
        所以
        \[
        \tan\theta = \frac{x^2-1}{2x}
        \]
        类似地,由 $y = \cot\theta - \csc\theta$ 和恒等式 $1+\cot^2\theta = \csc^2\theta$ 得
        \[
        \cot\theta = \frac{y^2-1}{2y}
        \]
        于是
        \[
        \tan\theta \cot\theta = \frac{x^2-1}{2x} \cdot \frac{y^2-1}{2y} = 1 \Rightarrow (x^2-1)(y^2-1) = 4xy
        \]
    \end{solution}
    \part 
    \[
    \frac{dy}{dx} = \frac{1-y^2}{2x}.
    \]
    \begin{solution}
        参数求导得
        \[
        \frac{dx}{d\theta} = \sec^2\theta - \sec\theta\tan\theta = \sec\theta(\sec\theta - \tan\theta)
        \]
        \[
        \frac{dy}{d\theta} = -\csc^2\theta + \csc\theta \cot\theta = \csc\theta(\cot\theta - \csc\theta)
        \]
        所以
        \[
        \frac{dy}{dx} = \frac{dy}{d\theta} \div \frac{dx}{d\theta} = \frac{\csc\theta(\cot\theta - \csc\theta)}{\sec\theta(\sec\theta - \tan\theta)}
        \]
        注意到
        \[
        \frac{\csc\theta}{\sec\theta} = \frac{\cos\theta}{\sin\theta} = \cot\theta,
        \]
        且
        \[
        \frac{\cot\theta - \csc\theta}{\sec\theta - \tan\theta} = -\frac{y}{x}
        \]
        得到
        \[
        \frac{dy}{dx} = -\frac{y}{x} \cdot \frac{y^2-1}{2y} = \frac{1-y^2}{2x}
        \]
        故得证。
    \end{solution}
    \end{parts}

    \question 曲线的参数方程为
    \[
    x = \sin\theta, \quad y = \theta \cos\theta, \quad -\pi < \theta < \pi.
    \]
    已知曲线在 $\theta = -\dfrac{\pi}{4}$ 及 $\theta = \dfrac{\pi}{4}$ 处的切线平行,且间距为 $d$,证明:
    \[
    d = \sqrt{\frac{8\pi^2 - 32\pi + 32}{\pi^2 - 8\pi + 32}}.
    \]
    \begin{solution}
        先求曲线的导数,
        \[
        \frac{dx}{d\theta} = \cos\theta, \quad \frac{dy}{d\theta} = \cos\theta - \theta\sin\theta \implies \frac{dy}{dx} = \frac{dy/d\theta}{dx/d\theta} = 1 - \theta\tan\theta.
        \]
        当 $\theta=\pm \frac{\pi}{4}$ ,
        \[
        x = \pm \frac{1}{\sqrt{2}}, \quad y = \pm \frac{\pi}{4\sqrt{2}}, \quad \frac{dy}{dx} = 1 - \frac{\pi}{4}.
        \]
        故在$\theta=\pm \frac{\pi}{4}$处的切线方程为
        \[
        y - \frac{\pi}{4\sqrt{2}} = \left(1 - \frac{\pi}{4}\right)\left(x - \frac{1}{\sqrt{2}}\right), \quad y + \frac{\pi}{4\sqrt{2}} = \left(1 - \frac{\pi}{4}\right)\left(x + \frac{1}{\sqrt{2}}\right).
        \]
        两切线在 $x=0$ 时的纵坐标差为
        \[
        y_1 = \frac{\sqrt{2}(2-\pi)}{4}, y_2 = \frac{\sqrt{2}(\pi-2)}{4} \Rightarrow |y_2 - y_1| = \frac{\sqrt{2}(\pi-2)}{2}
        \]
        且切线与水平的夹角 $\theta$ 满足
        \[
        \cos\theta = \frac{1}{\sqrt{1+(\frac{dy}{dx})^2}} = \frac{4}{\sqrt{\pi^2 - 8\pi + 32}}.
        \]
        故切线间距为
        \[
        d = |y_2 - y_1| \cos\theta = \frac{\sqrt{2}(\pi-2)}{2} \cdot \frac{4}{\sqrt{\pi^2 - 8\pi + 32}} = \sqrt{\frac{8\pi^2 - 32\pi + 32}{\pi^2 - 8\pi + 32}}.
        \]
    \end{solution}

    \question 曲线 $C$ 的方程为
    \[
    x^2 + xy + y^2 = 1, \quad 0 \le x \le 3.
    \]
    考虑参数化
    \[
    x = A\cos\theta + B\sin\theta, \quad y = A\cos\theta - B\sin\theta,
    \]
    确定 $A$ 和 $B$,并求出第一象限被曲线和坐标轴围成的有限区域的面积。
    \begin{solution}
        尝试 $x = \cos\theta + \sin\theta$, $y = \cos\theta - \sin\theta$,则
        \[
        (x+y)^2 - xy = (2\cos\theta)^2 - (\cos^2\theta-\sin^2\theta) 
        = 3\cos^2\theta + \sin^2\theta \neq 1
        \]
        因此调整成
        \[
        x = \frac{1}{\sqrt{3}}\cos\theta + \sin\theta, \quad y = \frac{1}{\sqrt{3}}\cos\theta - \sin\theta.
        \]
        发现
        \[
        (x+y)^2 - xy = \left(\frac{2}{\sqrt{3}}\cos\theta\right)^2 - \left(\frac{1}{3}\cos^2\theta - \sin^2\theta\right)
        = \cos^2\theta + \sin^2\theta = 1
        \]
        成功!则所求面积为
        \begin{align*}
        \int_{-\frac{\pi}{6}}^{\frac{\pi}{6}} y(\theta) \frac{dx}{d\theta}\, d\theta
        &= \int_{-\frac{\pi}{6}}^{\frac{\pi}{6}} \left(\frac{1}{\sqrt{3}}\cos\theta - \sin\theta\right)\left(-\frac{1}{\sqrt{3}}\sin\theta + \cos\theta\right)\, d\theta \\
        &= \int_{-\frac{\pi}{6}}^{\frac{\pi}{6}} \frac{\sqrt{3}}{3}(\cos^2\theta + \sin^2\theta) \, d\theta \\
        &= \frac{\sqrt{3}}{3} \cdot \frac{\pi}{3} = \frac{\pi \sqrt{3}}{9}
        \end{align*}
    \end{solution}

    \question 已知曲线的参数方程为
    \[
    x = t^3, \quad y = t^2, \quad t \in \mathbb{R}.
    \]
    曲线在点 $P$ 处的切线再次与曲线相交于点 $Q$。如图,曲线与切线所围成的区域面积为 $2.7$ 平方单位,求点 $P$ 的坐标。
    \begin{figure}[H]
        \centering        
        \includegraphics[width=0.5\textwidth]{images/image237.png}
    \end{figure}
    \begin{solution}
        设参数 $t = p$对应 $P(p^3, p^2)$,切线斜率为
        \[
        \frac{dy}{dx} 
        = \frac{dy}{dt} \div \frac{dx}{dt} = \frac{2t}{3t^2} = \frac{2}{3t} \Rightarrow \frac{dy}{dx}\bigg|_{t=p} = \frac{2}{3p}
        \]
        切线方程
        \[
        y - p^2 = \frac{2}{3p} (x - p^3) \Rightarrow 3p y = 2x + p^3.
        \]
        与曲线方程联立得
        \[
        2 t^3 - 3p t^2 + p^3 = 0
        \]
        因式分解为
        \[
        (t - p)^2 (2 t + p) = 0 \Rightarrow Q \left(-\frac{1}{8} p^3, \frac{1}{4} p^2\right)
        \]
        阴影部分面积即梯形面积减去曲线下的面积,而曲线下的面积为
        \[
        \int_{-\frac{p}{2}}^{p} t^2 (3t^2) \, dt = \int_{-\frac{p}{2}}^{p} 3t^4 \, dt = \left[ \frac{3}{5}t^5 \right]_{-\frac{p}{2}}^{p} = \frac{99}{160}p^5
        \]
        梯形面积为
        \[
        \frac{1}{2} \left(\frac{1}{4}p^2 + p^2\right)\cdot \left(p^3 + \frac{1}{8}p^3\right) = \frac{45}{64}p^5
        \]
        于是
        \[
        \frac{45}{64}p^5 - \frac{99}{160}p^5 = \frac{27}{10} \Rightarrow p = 2
        \]
        所以点 $P$ 的坐标为 
        \[
        (2^3, 2^2) = (8, 4)
        \]
    \end{solution}

    \question 已知曲线的参数方程为
    \[
    x = \sin\left(t + \frac{\pi}{6}\right), \quad y = 1 + \cos 2t, \quad 0 \le t < 2\pi.
    \]
    且曲线关于 $y$ 轴对称,求曲线两环所围面积。
    \begin{figure}[H]
        \centering        
        \includegraphics[width=0.5\textwidth]{images/image238.png}
    \end{figure}
    \begin{solution}
        利用参数积分,两侧环的面积为
        \[
        I = 2 \int_{\frac{\pi}{6}}^{\frac{\pi}{2}} (1+\cos 2t) \cos\left(t+\frac{\pi}{6}\right) \, dt.
        \]
        积化和差得
        \[
        \cos 2t \cos\left(t+\frac{\pi}{6}\right) = \frac{1}{2} \left[ \cos\left(3t + \frac{\pi}{6}\right) + \cos\left(t - \frac{\pi}{6}\right) \right]
        \]
        于是
        \begin{align*}
        I &= 2 \int_{\frac{\pi}{6}}^{\frac{\pi}{2}} \left[ \cos\left(t+\frac{\pi}{6}\right) + \frac{1}{2} \cos\left(3t+\frac{\pi}{6}\right) + \frac{1}{2} \cos\left(t-\frac{\pi}{6}\right) \right] \, dt \\
        &= 2 \left[ \sin\left(t+\frac{\pi}{6}\right) + \frac{1}{6}\sin\left(3t+\frac{\pi}{6}\right) + \frac{1}{2}\sin\left(t-\frac{\pi}{6}\right) \right]_{\frac{\pi}{6}}^{\frac{\pi}{2}} \\
        &= \frac{4\sqrt{3}}{3}
        \end{align*}
    \end{solution}

    \question 已知曲线 $C$ 的参数方程为
    \[
    x = t \sin t, \quad y = \cos t, \quad 0 \le t < 2\pi.
    \]
    曲线与坐标轴交于 $P,Q,R,S$ 四点。
    \begin{figure}[H]
        \centering        
        \includegraphics[width=0.5\textwidth]{images/image239.png}
    \end{figure}
    \begin{parts}
    \part 求在点 $P,Q,R,S$ 处的 $t$ 的值。
    \begin{solution}
        设 $x = 0$,
        \[
        t_P = \pi \implies P(0,-1), \quad t_Q = 0 \implies Q(0,1)
        \]
        设 $y = 0$,
        \[
        t_R = \frac{3\pi}{2} \implies R\left(-\frac{3\pi}{2},0\right), \quad
        t_S = \frac{\pi}{2} \implies S\left(\frac{\pi}{2},0\right)
        \]
    \end{solution}
    \part 求曲线沿着$P,Q,R,S$所围成的区域面积。
    \begin{solution}
        由参数积分公式
        \[
        I = \int_{0}^{2\pi} y(t) \frac{dx}{dt} \, dt.
        \]
        被积函数为
        \[
        y(t)\frac{dx}{dt} = \cos t (\sin t + t \cos t) = \cos t \sin t + t \cos^2 t = \frac{1}{2}\sin 2t + \frac{1}{2} t + \frac{1}{2} t \cos 2t
        \]
        对 $\dfrac{1}{2} t \cos 2t$ 分部积分得
        \[
        \int t \cos 2t \, dt = \frac{1}{2} t \sin 2t - \frac{1}{2} \int \sin 2t \, dt = \frac{1}{2} t \sin 2t + \frac{1}{4} \cos 2t + C.
        \]
        因此
        \[
        \int y \frac{dx}{dt} \, dt = \frac{1}{4} t^2 - \frac{1}{8} \cos 2t + \frac{1}{4} t \sin 2t + C.
        \]
        故所围成的面积为
        \[
        I = \left[ \frac{1}{4} t^2 - \frac{1}{8} \cos 2t + \frac{1}{4} t \sin 2t \right]_0^{2\pi} = \pi^2
        \]
    \end{solution}
    \end{parts}

    \question 如图所示,曲线的参数方程为
    \[ 
    x = t^2 + 2t, \quad y = t^3 - 9t, \quad t \in \mathbb{R} 
    \]
    该曲线与坐标轴交于原点 $O$ 以及点 $A,B,C$。
    \begin{figure}[H]
        \centering        
        \includegraphics[width=0.5\textwidth]{images/image236.png}
    \end{figure}
    \begin{parts}
    \part 确定 $A$,$B$ 和 $C$ 的坐标。
    \begin{solution}
        令 $x=0$,则
        \[
        0 = t(t+2) \Rightarrow t = 0 \ (O), \quad t = -2 \ (C),
        \]
        因此$C(0,10)$,令 $y=0$,
        \[
        0 = t(t^2-9) = t(t-3)(t+3) \Rightarrow t = 0 \ (O), \quad t=3 \ (B), \quad t=-3 \ (A),
        \]
        因此 $A(3,0),B(15,0)$。
    \end{solution}
    \part 图中阴影部分的区域 $R$ 由曲线和坐标轴围成,求区域 $R$ 的面积。
    \begin{solution}
        由参数积分公式,面积为
        \begin{align*}
        \int_{t_1}^{t_2} y(t) \frac{dx}{dt} \, dt
        &= \int_{-3}^{-2} (t^3-9t)(t+2) \, dt \\
        &= \int_{-3}^{-2} (t^4 + 2t^3 - 9t^2 - 18t) \, dt\\
        &= \left[\frac{t^5}{5} + \frac{1}{2} t^4 - 3 t^3 - 9 t^2\right]_{-3}^{-2} \\
        &= \frac{171}{10}
        \end{align*}
    \end{solution}
    \part 由曲线和 $y$ 轴围成的满足 $x < 0$ 的有界区域,绕 $x$ 轴旋转一周形成一个旋转体。计算该旋转体的体积。
    \begin{solution}
        设曲线最左端处的参数值为$t_1$,旋转体体积为
        \begin{align*}
        &\pi \int_{t_1}^{-2} [y(t)]^2 \frac{dx}{dt} \, dt - \pi \int_{t_1}^{0} [y(t)]^2 \frac{dx}{dt} \, dt \\
        &= \pi \int_{0}^{-2} (t^3-9t)^2 (t^2+2t) \, dt \\
        &= 2\pi \int_{0}^{-2} (t^7 - 18 t^5 + 81 t^3 + t^6 - 18 t^4 + 81 t^2) \, dt\\
        &= 2\pi \left[\frac{t^8}{8} + \frac{t^7}{7} - 3 t^6 - \frac{18}{5} t^5 + \frac{81}{4} t^4 + 27 t^3\right]_{0}^{-2}\\
        &= \frac{3144 \pi}{35}
        \end{align*}
    \end{solution}
    \end{parts}
\end{questions}
\pagebreak

\begin{center}
  {\fontsize{30pt}{26pt}\selectfont
    \hypertarget{极坐标}{极坐标} \label{极坐标}
  }
\end{center}
\separator
\vspace{1pt}

\begin{questions}
    \question 曲线 $C$ 的极坐标方程为
    \[
    r=\tan\theta,\quad 0\le\theta<\frac{\pi}{2}.
    \]
    求曲线 $C$ 的直角坐标方程,并写成 $y=f(x)$ 的形式。
    \begin{solution}
        由已知$r=\tan\theta$,两边平方得
        \[
        r^2=\tan^2\theta \Rightarrow r^2\cos^2\theta=\sin^2\theta
        \]
        由 $r\cos\theta=x$,得
        \[
        x^2=\sin^2\theta
        \]
        至此,欲以$\tan^2\theta$表示$\sin^2\theta$,不妨考虑
        \[
        1+\tan^2\theta=\sec^2\theta=\frac{1}{\cos^2 \theta}=\frac{1}{1-\sin^2 \theta}
        \]
        据此有
        \[
        \tan^2\theta=\frac{1}{1-x^2}-1=\frac{x^2}{1-x^2}
        \]
        又有 $r=\tan\theta$及$r^2=x^2+y^2$,于是
        \[
        x^2+y^2=\frac{x^2}{1-x^2} \Rightarrow y^2=\frac{x^4}{1-x^2}
        \]
        因为 $0\le\theta<\dfrac{\pi}{2}$,有 $y\ge0$,故
        \[
        y=\frac{x^2}{\sqrt{1-x^2}},
        \]
        其中 $0\le x<1$。
    \end{solution}

    \question 心脏线的极坐标方程为 
    \[
    r=4(1+\cos \theta),0 \le \theta \le \frac{\pi}{2}
    \]
    曲线在点 $P$ 处的切线斜率为 $-1$。求该切线的极坐标方程,结果以 $r=f(\theta)$ 的形式表示。
    \begin{solution}
        由链导法,
        \[ 
        \frac{dy}{dx} = \frac{dy/d\theta}{dx/d\theta} = \frac{\frac{d}{d\theta}(r \sin \theta)}{\frac{d}{d\theta}(r \cos \theta)} = \frac{\frac{d}{d\theta}[4(1+\cos \theta)\sin \theta]}{\frac{d}{d\theta}[4(1+\cos \theta)\cos \theta]} 
        \]
        展开并求导得
        \[ 
        \frac{dy}{dx} = \frac{\cos \theta + \cos^2 \theta - \sin^2 \theta}{-\sin \theta - 2\sin \theta \cos \theta} = \frac{\cos \theta + \cos 2\theta}{-\sin \theta - \sin 2\theta} 
        \]
        令斜率为 $-1$,
        \[ 
        \frac{\cos \theta + \cos 2\theta}{-\sin \theta - \sin 2\theta} = -1 \Rightarrow \cos \theta + \cos 2\theta = \sin \theta + \sin 2\theta 
        \]
        和差化积得
        \[ 
        2\cos \frac{3\theta}{2} \cos \frac{\theta}{2} = 2\sin \frac{3\theta}{2} \cos \frac{\theta}{2} \Rightarrow
        \cos \frac{\theta}{2} \left( \cos \frac{3\theta}{2} - \sin \frac{3\theta}{2} \right) = 0 
        \]
        在范围 $0 \le \theta \le \dfrac{\pi}{2}$ 内,$\cos \dfrac{\theta}{2} \neq 0$,故有
        \[ 
        \tan \frac{3\theta}{2} = 1 \Rightarrow \theta = \frac{\pi}{6} 
        \]
        此时
        \[ 
        r = 4\left(1 + \cos \frac{\pi}{6}\right) = 4 + 2\sqrt{3} 
        \]
        点 $P$ 的直角坐标为
        \[ 
        P\left(r \cos \frac{\pi}{6},r \sin \frac{\pi}{6}\right) = (2\sqrt{3} + 3 ,2 + \sqrt{3})
        \]
        故切线的方程为 
        \[ 
        y - (2 + \sqrt{3}) = -(x - (3 + 2\sqrt{3})) \implies y + x = 5 + 3\sqrt{3} 
        \]
        转换为极坐标方程:
        \[ 
        r\sin \theta + r\cos \theta = 5 + 3\sqrt{3} \Rightarrow r = \frac{5 + 3\sqrt{3}}{\cos \theta + \sin \theta} 
        \]
    \end{solution}

    \question 曲线 $C_1,C_2$ 的极坐标方程分别为
    \[
    C_1:\ r=2\cos\theta-\sin\theta,\quad 0<\theta\le\frac{\pi}{3},
    \]
    \[
    C_2:\ r=\sqrt{2}+\sin\theta,\quad 0\le\theta<2\pi.
    \]
    点 $P$ 在 $C_1$ 上,且 $P$ 点处的切线与极轴平行。
    \begin{parts}
    \part 证明在点 $P$ 处有
    \[
    \tan 2\theta=2.
    \]
    \begin{solution}
        $P$ 点处的切线与极轴平行,意味
        \[
        \frac{dy}{dx}=0\Rightarrow\frac{dy}{d\theta}=0
        \]
        即
        \[
        \frac{d}{d\theta}(r\sin\theta)=\frac{d}{d\theta}\left((2\cos\theta-\sin\theta)\sin\theta\right)=\frac{d}{d\theta}(\sin 2\theta-\sin^2\theta)=0
        \]
        对 $\theta$ 求导得
        \[
        2\cos 2\theta-2\sin\theta\cos\theta=0.
        \]
        即
        \[
        2\cos 2\theta=\sin 2\theta \Rightarrow \tan 2\theta=2
        \]
    \end{solution}
    \part 证明点 $P$ 到原点 $O$ 的距离为
    \[
    \sqrt{\frac{5-\sqrt{5}}{2}}.
    \]
    \begin{solution}
        由 $\tan 2\theta=2$,利用倍角公式
        \[
        \tan 2\theta=\frac{2\tan\theta}{1-\tan^2\theta}
        \]
        解得
        \[
        \tan\theta=\frac{-1\pm\sqrt{5}}{2}
        \]
        由于 $0<\theta<\dfrac{\pi}{3}$,取
        \[
        \tan\theta=\frac{-1+\sqrt{5}}{2}
        \]
        由此可得
        \[
        \sin\theta=\frac{-1+\sqrt{5}}{\sqrt{10-2\sqrt{5}}},\quad
        \cos\theta=\frac{2}{\sqrt{10-2\sqrt{5}}}
        \]
        代入 $r=2\cos\theta-\sin\theta$,得到
        \[
        r_P=\frac{5-\sqrt{5}}{\sqrt{10-2\sqrt{5}}}=\sqrt{\frac{5-\sqrt{5}}{2}}
        \]
    \end{solution}
    \part 设点 $Q$ 为 $C_1$ 与 $C_2$ 的交点,求点 $Q$ 处对应的 $\theta$ 值。
    \begin{solution}
        联立$C_1,C_2$得
        \[
        \cos\theta-\sin\theta=\frac{\sqrt{2}}{2}.
        \]
        于是
        \[
        \sqrt{2}\cos\left(\theta+\frac{\pi}{4}\right)=\frac{\sqrt{2}}{2},
        \]
        即
        \[
        \cos\left(\theta+\frac{\pi}{4}\right)=\frac{1}{2}
        \]
        在 $0\le\theta\le\dfrac{\pi}{3}$ 内解得
        \[
        \theta+\frac{\pi}{4}=\frac{\pi}{3} \Rightarrow \theta=\frac{\pi}{12}
        \]
    \end{solution}
    \end{parts}

    \question 曲线 $C_1,C_2$ 的极坐标方程分别为 
    \[
    r_1 = 3 + 2\cos \theta, 0 \le \theta < 2\pi,\quad r_2 = 2
    \]
    两曲线交于点 $P,Q$,经过 $P$ 点和极点 $O$ 的直线与 $C_1$ 再次交于 $R$ 点,证明 $RQ$ 是 $C_2$ 在 $Q$ 点处的切线。
    \begin{figure}[H]
        \centering        
        \includegraphics[width=0.5\textwidth]{images/image240.png}
    \end{figure}
    \begin{solution}
        联立$C_1,C_2$得
        \[ 
        3 + 2\cos \theta = 2 \Rightarrow \theta = \frac{2\pi}{3},\frac{4\pi}{3} \Rightarrow P\left(2, \frac{2\pi}{3}\right),Q\left(2, \frac{4\pi}{3}\right)
        \]
        由于 $PR$ 通过极点 $O$,则 $R$ 点的角度为 $\dfrac{2\pi}{3} + \pi = \dfrac{5\pi}{3}$,代入 $C_1$ 方程:
        \[
        r_R = 3 + 2\cos(\frac{5\pi}{3}) = 4
        \]
        故 $R(4, \frac{5\pi}{3})$,在 $\triangle OQR$ 中,由余弦定理,
        $\angle QOR = \frac{5\pi}{3} - \frac{4\pi}{3} = \frac{\pi}{3}$。
        \[ 
        QR^2 = 2^2 + 4^2 - 2 \cdot 2 \cdot 4 \cos \left(\frac{5\pi}{3} - \frac{4\pi}{3}\right) \Rightarrow QR = \sqrt{12} 
        \]
        此时发现
        \[
        QR^2+OQ^2=OR^2=16
        \]
        由毕氏定理,$\angle OQR = \dfrac{\pi}{2}$,即$RQ$ 与圆 $C_2$ 在 $Q$ 点垂直,故$RQ$ 是 $C_2$ 在 $Q$ 点处的切线。
    \end{solution}

    \question 已知两曲线极坐标方程
    \[
    C_1: r = 12 \cos \theta, \quad -\frac{\pi}{2}< \theta \le \frac{\pi}{2}, \quad C_2: r = 4 + 4 \cos \theta, \quad -\pi < \theta \le \pi
    \]
    已知 $C_1$ 与 $C_2$ 的一个交点为 $A$,由 $C_1$ 与线段 $OA$ 围成的较小区域面积为 $6\pi - 9\sqrt{3}$。若有限区域 $R$ 表示在 $C_1$ 内但在 $C_2$ 外的点,证明 $R$ 的面积为 $16\pi$。
    \begin{solution}
        首先求交点:
        \[ 
        12 \cos \theta = 4 + 4 \cos \theta \Rightarrow \theta = \pm \frac{\pi}{3} \Rightarrow
        \]
        记$A\left(6,\frac{\pi}{3}\right)$,所求面积即
        \[ 
        2 \cdot \left[ \frac{1}{2}\pi(6)^2 - \left( \int_{0}^{\frac{\pi}{3}} \frac{1}{2}(4 + 4 \cos \theta)^2 \, d\theta + 6\pi - 9\sqrt{3} \right) \right] =16\pi
        \]
        其中
        \begin{align*}
        \int_{0}^{\frac{\pi}{3}} \frac{1}{2}(4 + 4 \cos \theta)^2 \, d\theta = 4\pi + 9\sqrt{3} 
        &= \int_{0}^{\frac{\pi}{3}} \left(8 + 16 \cos\theta+8\cos^2\theta\right) \, d\theta \\
        &= \int_{0}^{\frac{\pi}{3}} \left(12 + 16 \cos\theta+4\cos 2\theta\right) \, d\theta \\
        &= \left[12\theta +16\sin \theta +2 \sin 2 \theta\right]_{0}^{\frac{\pi}{3}} \\
        &= 4\pi + 9\sqrt{3}
        \end{align*}
    \end{solution}

    \question 曲线 $C_1,C_2$ 的极坐标方程分别为
    \[ 
    r = 1 + \sin \theta, \quad 0 < \theta < \frac{\pi}{2}, \quad r = 1 + \cos 2\theta, \quad 0 < \theta < \frac{\pi}{2}
    \]
    点 $P$ 是 $C_1,_2$ 的交点。一条平行于极轴的直线通过点 $P$ 并与曲线 $C_2$ 交于点 $Q$。证明
    \[ 
    PQ = \frac{1}{32} \left[ 24\sqrt{3} - (2 + 2\sqrt{13})^{\frac{3}{2}} \right] 
    \]
    \begin{solution}
        联立方程求交点 $P$ 的极坐标:
        \[ 
        1 + \sin \theta = 1 + \cos 2\theta \Rightarrow \theta = \frac{1}{2}
        \]
        故 $P\left(\dfrac{3}{2}, \dfrac{\pi}{6}\right)$,过 $P$且平行于极轴的直线极坐标方程为 
        \[
        r \sin \theta = \frac{3}{4}
        \]
        与 $C_2$ 联立可解得
        \[
        (1+\cos 2\theta)\sin \theta = \frac{3}{4} \Rightarrow (2\sin \theta-1)(4\sin^2 \theta+2\sin \theta-3)=0 \Rightarrow \sin \theta = \frac{\sqrt{13}-1}{4}
        \]
        此时
        \[
        r = \frac{1 + \sqrt{13}}{4}, \quad \cos \theta = \frac{1}{4}\sqrt{2 + 2\sqrt{13}}
        \]
        故$PQ$的长度为
        \begin{align*}
        |x_P - x_Q| 
        &= r_P \cos \theta_P - r_Q \cos \theta_Q \\
        &= \frac{3}{2} \cdot \frac{\sqrt{3}}{2} - \left( \frac{1 + \sqrt{13}}{4} \right) \left( \frac{\sqrt{2 + 2\sqrt{13}}}{4} \right) \\
        &= \frac{1}{32} \left[ 24\sqrt{3} - (2 + 2\sqrt{13})^{\frac{3}{2}} \right]
        \end{align*}
        证毕。
    \end{solution}

    \question 曲线的极坐标方程为 
    \[
    r = 1 + \tan \theta,\quad 0 \le \theta \le \frac{\pi}{2}
    \]
    点 $P$ 位于曲线上 $\theta = \dfrac{\pi}{3}$ 处。点 $Q$ 位于极轴上,使得直线 $L$ 经过 $P$ 和 $Q$ 且与极轴垂直。求由该曲线与直线 $L$ 所围成的有限区域的面积。
    \begin{solution}
        $P$ 点坐标为$\left(1 + \sqrt{3},\dfrac{\pi}{3}\right)$,
        因此 
        \[
        OQ = (1 + \sqrt{3}) \cos \frac{\pi}{3} = \frac{1}{2}(1 + \sqrt{3})
        \]
        直线 $L$ 的极坐标方程为 
        \[
        r \cos \theta = \frac{1}{2}(1 + \sqrt{3})
        \]
        与曲线联立,解得
        \[
        (1 + \tan\theta)\cos\theta = \frac{1}{2}(1 + \sqrt{3}) \Rightarrow \theta = \frac{\pi}{6}
        \]
        故得$R\left(1 + \dfrac{\sqrt{3}}{3},\dfrac{\pi}{6}\right)$,于是$\triangle OPR$ 的面积为
        \[
        \frac{1}{2}\cdot OP \cdot OR \sin \left(\frac{\pi}{3}-\frac{\pi}{6}\right)=\frac{1}{2}(1 + \sqrt{3})\left(1 + \frac{\sqrt{3}}{3}\right)\sin \frac{\pi}{6}=\frac{1}{2}+\frac{\sqrt{3}}{3}
        \]
        而极坐标扇形面积为
        \begin{align*}
        \frac{1}{2} \int_{\frac{\pi}{6}}^{\frac{\pi}{3}} (1 + \tan \theta)^2 \, d\theta 
        &= \int_{\frac{\pi}{6}}^{\frac{\pi}{3}} \left( \tan \theta + \frac{1}{2} \sec^2 \theta \right) d\theta \\
        &= \left[ \ln(\sec \theta) + \frac{1}{2} \tan \theta \right]_{\frac{\pi}{6}}^{\frac{\pi}{3}} \\
        &= \frac{1}{3} \sqrt{3} + \frac{1}{2} \ln 3
        \end{align*}
        所求面积为
        \[ 
        \frac{1}{3} \sqrt{3} + \frac{1}{2} \ln 3-\left(\frac{1}{2}+\frac{\sqrt{3}}{3}\right) = \frac{1}{2}(\ln3-1)
        \]
    \end{solution}

    \question 如图,曲线方程为 
    \[
    r = 2 + 2\sin\theta, \quad 0 \le \theta \le 2\pi
    \]
    直线方程为 
    \[
    2r\sin\theta = 3, \quad 0 < \theta < \pi
    \]
    \begin{figure}[H]
        \centering        
        \includegraphics[width=0.5\textwidth]{images/image243.png}
    \end{figure}
    \begin{parts}
    \part 求交点 $P,Q$ 的坐标。
    \begin{solution}
        联立得
        \[ 
        2(2 + 2\sin\theta)\sin\theta = 3 \Rightarrow (2\sin\theta - 1)(2\sin\theta + 3) = 0
        \]
        由于 $-1 \le \sin\theta \le 1$,方程 $2\sin\theta + 3 = 0$ 无解,因此
        \[ 
        \sin\theta = \frac{1}{2} \Rightarrow \theta = \frac{\pi}{6}, \frac{5\pi}{6}
        \]
        所以交点坐标为
        \[
        P\left(3, \frac{\pi}{6}\right),Q\left(3, \frac{5\pi}{6}\right)
        \]
    \end{solution}
    \part 证明$\triangle OPQ$ 的面积为 $\dfrac{9\sqrt{3}}{4}\sqrt{3}$。
    \begin{solution}
        已知 $OP = 3,OQ = 3$,夹角为 
        \[
        \frac{5\pi}{6} - \frac{\pi}{6} = \frac{2\pi}{3}
        \]
        故$\triangle OPQ$面积为
        \[ 
        \frac{1}{2} \cdot 3 \cdot 3 \cdot \sin\frac{2\pi}{3} = \frac{9\sqrt{3}}{4} 
        \]
    \end{solution}
    \part 求由曲线和直线围成的阴影区域的面积。
    \begin{solution}
        阴影区域面积等于曲线在对应角度内围成的面积减去$\triangle OPQ$ 的面积,
        \[
        \int_{\frac{\pi}{6}}^{\frac{5\pi}{6}} \frac{1}{2} (2 + 2\sin\theta)^2\, d\theta - [\triangle OPQ]
        \]
        其中
        \begin{align*}
        \int_{\frac{\pi}{6}}^{\frac{5\pi}{6}} \frac{1}{2} (2 + 2\sin\theta)^2\, d\theta
        &= \int_{\frac{\pi}{6}}^{\frac{5\pi}{6}} (3 + 4\sin\theta - \cos 2\theta)\, d\theta \\
        &= \Bigl[3\theta - 4\cos\theta - \frac{1}{2}\sin 2\theta\Bigr]_{\frac{\pi}{6}}^{\frac{5\pi}{6}} \\
        &= 2\pi + \frac{9\sqrt{3}}{2}
        \end{align*}
        故阴影部分的面积为
        \[ 
        (2\pi + \frac{9\sqrt{3}}{2}) - \frac{9\sqrt{3}}{4} = 2\pi + \frac{9}{4}\sqrt{3} 
        \]
    \end{solution}
    \end{parts}

    \question 如图,曲线的极坐标方程为
    \[
    r^2=\cos2\theta,\quad 
    \theta\in\left[0,\frac{\pi}{4}\right]\cup\left[\frac{3\pi}{4},\frac{5\pi}{4}\right]\cup\left[\frac{7\pi}{4},2\pi\right].
    \]
    该曲线被矩形 $ABCD$ 包围,其中 $AB,CD$ 为与极轴平行的切线,$AD,BC$ 为与极轴垂直的切线。证明曲线与矩形 $ABCD$ 之间所围成的总面积为$\sqrt{2}-1$。
    \begin{figure}[H]
        \centering        
        \includegraphics[width=0.5\textwidth]{images/image230.png}
    \end{figure}
    \begin{solution}
        首先确定矩形的边界,由
        \[
        r^2=\cos2\theta
        \]
        可知当 $\cos2\theta=1$ 时有$r=1$,因此矩形宽为 $2$,而切线与极轴平行当且仅当
        \[
        \frac{dy}{dx}=0 \Rightarrow \frac{dy}{d\theta}=0 
        \]
        由于 
        \[
        y = r\sin\theta = \sqrt{\cos \theta}\sin\theta
        \]
        为简化计算,不妨解
        \[ 
        \frac{d}{d\theta}(y^2)=\frac{d}{d\theta}(r^2 \sin^2 \theta) = \frac{d}{d\theta}(\cos 2\theta \sin^2 \theta) = 0 
        \]
        求导得
        \[ 
        -2\sin 2\theta \sin^2 \theta + \cos 2\theta (2\sin\theta \cos\theta) = 0 
        \]
        \[ 
        \sin 2\theta [-2\sin^2 \theta + \cos 2\theta] = 0 
        \]
        若 $\sin 2\theta = 0$,则 
        \[
        \theta = \frac{n\pi}{2}, n \in \mathbb{Z},
        \]
        不适合。若 $-2\sin^2 \theta + \cos 2\theta = 0$,则由
        \[ 
        -2\left(\frac{1}{2} - \frac{1}{2}\cos 2\theta\right) + \cos 2\theta = 0 \Rightarrow \cos 2\theta = \frac{1}{2}
        \]
        取$\theta = \dfrac{\pi}{6}$,此时 $r = \dfrac{\sqrt{2}}{2}$,矩形高为
        \[
        2\cdot \frac{\sqrt{2}}{2} \cdot \sin \frac{\pi}{6} = \frac{\sqrt{2}}{2} 
        \]
        故矩形的总面积为$\sqrt{2}$,而由曲线围成的面积为
        \[ 
        4 \int_{0}^{\frac{\pi}{4}} \frac{1}{2} r^2 d\theta 
        = 4 \int_{0}^{\frac{\pi}{4}} \frac{1}{2} \cos 2\theta d\theta 
        = 4 \left[\frac{1}{4}\sin 2\theta \right]_{0}^{\frac{\pi}{4}} 
        = 1 
        \]
        因此曲线与矩形之间围成的面积为$\sqrt{2} - 1$。
    \end{solution}

    \question 如图,两条封闭曲线的极坐标方程分别为
    \[
    C_1:\ r=a(1+\sin\theta),\quad 0\le\theta\le2\pi, \quad 
    C_2:\ r=3a(1-\sin\theta),\quad 0\le\theta\le2\pi,
    \]
    其中$a>0$,且在极点 $O$ 及点 $P,Q$ 处相交。
    \begin{figure}[H]
        \centering        
        \includegraphics[width=0.4\textwidth]{images/image233.png}
    \end{figure}
    \begin{parts}
    \part 求点 $P,Q$ 的极坐标。
    \begin{solution}
        交点 $P,Q$ 满足
        \[
        a(1+\sin\theta)=3a(1-\sin\theta) \Rightarrow \sin\theta=\frac{1}{2} \Rightarrow \theta=\frac{\pi}{6},\ \frac{5\pi}{6}.
        \]
        因此
        \[
        P\left(\frac{3a}{2},\frac{5\pi}{6}\right),\quad Q\left(\frac{3a}{2},\frac{\pi}{6}\right)
        \]
    \end{solution}
    \part 证明线段 $PQ$ 的长度为 $\dfrac{3\sqrt{3}}{2}a$。
    \begin{solution}
        由对称性,设 $M$ 为 $PQ$ 的中点,则
        \[
        PQ=2MQ=2\cdot\frac{3a}{2}\sin\frac{\pi}{3}
        =\frac{3\sqrt{3}}{2}a
        \]
    \end{solution}
    \part 已知 $PQ=\dfrac{3}{2}$,证明阴影部分的面积为$3\sqrt{3}-\dfrac{4}{3}\pi$。
    \begin{solution}
        由$PQ=\dfrac{3}{2}$知
        \[
        \frac{3\sqrt{3}}{2}a=\frac{3}{2} \Rightarrow a=\frac{\sqrt{3}}{3}
        \]
        阴影部分的面积为
        \begin{align*}
        &2 \left[ \int_{\frac{\pi}{6}}^{\frac{\pi}{2}} \frac{1}{2} [a(1+\sin\theta)]^2 d\theta - \int_{\frac{\pi}{6}}^{\frac{\pi}{2}} \frac{1}{2} [3a(1-\sin\theta)]^2 d\theta \right] \\
        &= a^2 \int_{\frac{\pi}{6}}^{\frac{\pi}{2}} (-8 + 20\sin\theta - 8\sin^2 \theta) d\theta \\
        &= a^2 \int_{\frac{\pi}{6}}^{\frac{\pi}{2}} (-12 + 20\sin\theta + 4\cos 2\theta) d\theta \\
        &= a^2 \Bigl[ -12\theta - 20\cos\theta + 2\sin 2\theta \Bigr]_{\frac{\pi}{6}}^{\frac{\pi}{2}} \\
        &= 3\sqrt{3} - \frac{4\pi}{3}
        \end{align*}
        \end{solution}
    \end{parts}

    \question 如图,两条封闭曲线 $C_1$ 与 $C_2$ 的极坐标方程分别为
    \[
    C_1:\ r=3+\cos\theta,\quad 0\le\theta<2\pi,\quad 
    C_2:\ r=5-3\cos\theta,\quad 0\le\theta<2\pi.
    \]
    两曲线相交于两点 $P,Q$。
    \begin{figure}[H]
        \centering        
        \includegraphics[width=0.5\textwidth]{images/image234.png}
    \end{figure}
    \begin{parts}
    \part 求点 $P,Q$ 的极坐标。
    \begin{solution}
        交点 $P,Q$ 满足
        \[
        3+\cos\theta=5-3\cos\theta \Rightarrow \cos\theta=\frac{1}{2}
        \]
        在 $0\le\theta<2\pi$ 内解得
        \[
        P\left(\frac{7}{2},\frac{\pi}{3}\right),\quad
        Q\left(\frac{7}{2},\frac{5\pi}{3}\right).
        \]
    \end{solution}
    \part 证明$C_1$ 与 $C_2$ 的相交区域面积为
    \[
    \frac{97\pi-102\sqrt{3}}{6}.
    \]
    \begin{solution}
        当 $0\le\theta\le\dfrac{\pi}{3}$ 时,区域由曲线 $C_2$ 给出;当 $\dfrac{\pi}{3}\le\theta\le\pi$ 时,区域由曲线 $C_1$ 给出。所求面积为
        \[
        2(A_1+A_2)
        \]
        其中
        \begin{align*}
        A_1 &=\int_{0}^{\frac{\pi}{3}}\frac{1}{2}(5-3\cos\theta)^2\,d\theta \\
        &= \int_{0}^{\frac{\pi}{3}}\left(\frac{25}{2}-15\cos\theta+\frac{9}{2}\cos^2\theta\right)d\theta \\
        &= \int_{0}^{\frac{\pi}{3}}\left(\frac{59}{4}-15\cos\theta+\frac{9}{4}\cos2\theta\right)d\theta \\
        &= \left[\frac{59}{4}\theta-15\sin\theta+\frac{9}{8}\sin2\theta\right]_{0}^{\frac{\pi}{3}} \\
        &=\frac{59\pi}{12}-\frac{111}{16}\sqrt{3},
        \end{align*}
        \begin{align*}
        A_2 &=\int_{\frac{\pi}{3}}^{\pi}\frac{1}{2}(3+\cos\theta)^2\,d\theta \\
        &=\frac{1}{2}\int_{\frac{\pi}{3}}^{\pi}\left(9+6\cos\theta+\cos^2\theta\right)d\theta\\
        &=\int_{\frac{\pi}{3}}^{\pi}\left(\frac{19}{4}+3\cos\theta+\frac{1}{4}\cos2\theta\right)d\theta\\
        &=\left[\frac{19}{4}\theta+3\sin\theta+\frac{1}{8}\sin2\theta\right]_{\frac{\pi}{3}}^{\pi}\\
        &=\frac{19\pi}{6}-\frac{25\sqrt{3}}{16}
        \end{align*}
        因此阴影区域面积为
        \[
        2(A_1+A_2)=\frac{97\pi-102\sqrt{3}}{6}
        \]
    \end{solution}
    \end{parts}

    \question 曲线 $C$ 的极坐标方程为 $r = \tan\frac{\theta}{2}$,其中 $0 \le \theta < \frac{\pi}{2}$。点 $P$ 位于曲线 $C$ 上,且该点处的切线垂直于极轴。射线 $\theta = \alpha$ 经过点 $P$。求曲线 $C$ 与该射线所围成区域的面积。
    \begin{figure}[H]
        \centering        
        \includegraphics[width=0.5\textwidth]{images/image242.png}
    \end{figure}
    \begin{solution}
        由于点 $P$ 处的切线垂直于极轴,意味 
        \[ 
        \frac{dx}{d\theta} = \frac{d}{d\theta} \left[ \tan\frac{\theta}{2}\cos \theta \right] = \frac{1}{2} \sec^2\frac{\theta}{2} \cos \theta - \tan\frac{\theta}{2} \sin \theta = 0 
        \]
        注意到 $\theta < \frac{\pi}{2}$,故 $\cos \theta \neq 0$,于是
        \[ 
        \sec^2\frac{\theta}{2} - 2 \tan\frac{\theta}{2} \tan \theta = 0 
        \]
        设$t=\tan\dfrac{\theta}{2}$,解得
        \[ 
        (1 + t^2) - 2t \left( \frac{2t}{1 - t^2} \right) = 0 \Rightarrow t^2 = -2 + \sqrt{5} \Rightarrow t = \sqrt{-2 + \sqrt{5}}
        \]
        因此点 $P$ 对应的极角 $\alpha$ 为
        \[ 
        \alpha = 2 \arctan \sqrt{-2 + \sqrt{5}} 
        \]
        所围图形的面积为
        \[
        \frac{1}{2} \int_{0}^{\alpha} \left[ \sec^2\frac{\theta}{2} - 1 \right] \, d\theta 
        = \frac{1}{2} \left[ 2 \tan\frac{\theta}{2} - \theta \right]_{0}^{\alpha} 
        = \tan\left(\frac{1}{2}\alpha\right) - \frac{1}{2}\alpha
        = \sqrt{-2 + \sqrt{5}} - \arctan \sqrt{-2 + \sqrt{5}}
        \]
    \end{solution}

    \question 曲线 $C_1$ 和 $C_2$ 的极坐标方程分别为 
    \[
    r_1 = \sec \theta(1 - \tan^2 \theta), \quad r_2 = \frac{1}{2}\sec^3 \theta,\quad 0 \le \theta < \frac{\pi}{4},
    \]
    \begin{figure}[H]
        \centering        
        \includegraphics[width=0.5\textwidth]{images/image241.png}
    \end{figure}
    \begin{parts}
    \part 证明 $\tan \angle OAP = -3\sqrt{3}$。
    \begin{solution}
        联立可知
        \[ 
        \frac{1}{2}\sec^3 \theta = \sec \theta(1 - \tan^2 \theta) \Rightarrow \tan \theta = \frac{1}{\sqrt{3}} 
        \]
        于是$A\left(\dfrac{4}{3\sqrt{3}},\dfrac{\pi}{6}\right)$,设 $\angle OAP = \psi$,在 $\triangle OAP$ 中,由正弦定理,
        \[ 
        \frac{\sin \psi}{1} = \frac{\sin(\frac{5\pi}{6} - \psi)}{\frac{4}{3\sqrt{3}}}
        \]
        化简得
        \[ 
        4\sqrt{3} \sin \psi = 9 \left(\frac{1}{2}\cos \psi+\frac{\sqrt{3}}{2}\sin\psi\right)
        \Rightarrow \tan \psi = -3\sqrt{3} 
        \]
        故得证。
    \end{solution}
    \part 求曲边三角形 $OAQ$ 的面积,其中 $O$ 为极点。
    \begin{solution}
        面积为 
        \begin{align*}
        &\frac{1}{2} \int_{0}^{\frac{\pi}{6}} \frac{1}{4}\sec^6\theta\, d\theta + \frac{1}{2} \int_{\frac{\pi}{6}}^{\frac{\pi}{4}} \sec^2\theta\ (1 - \tan\theta)^2\, d\theta \\
        &= \frac{1}{8} \int_{0}^{\frac{\pi}{6}} \sec^2\theta(1 + \tan\theta)^2\, d\theta + \frac{1}{2} \int_{\frac{\pi}{6}}^{\frac{\pi}{4}} \sec^2\theta\ (1 - \tan\theta)^2\, d\theta \\
        &= \frac{1}{8}\left[\frac{1}{5}\tan^5 \theta + \frac{2}{3}\tan^3 \theta + \tan \theta\right]_{0}^{\frac{\pi}{6}} + \frac{1}{2} \left[\frac{1}{5}\tan^5 \theta - \frac{2}{3}\tan^3 \theta + \tan \theta\right]_{\frac{\pi}{6}}^{\frac{\pi}{4}} \\
        &= \frac{36-11\sqrt{3}}{135}
        \end{align*}
    \end{solution}
    \end{parts}

    \question 一斜率为 $-\frac{3}{11}$ 的直线 $L$ 是曲线
    \[
    r = 25\cos 2\theta,\quad 0 \le \theta \le \frac{\pi}{2}
    \] 
    的切线。证明由该曲线、直线 $L$ 以及极轴所围成的区域的面积为
    \[ 
    \frac{25}{12} \left( 46 - 75 \tan^{-1} \frac{1}{3} \right) 
    \]
    \begin{solution}
        由$r = 25\cos 2\theta$及链导法得
        \[ 
        \frac{dy}{dx} = \frac{\frac{d}{d\theta}(r \sin \theta)}{\frac{d}{d\theta}(r \cos \theta)} = \frac{\cos 2\theta \cos \theta - 2\sin 2\theta \sin \theta}{-2\sin 2\theta \cos \theta - \cos 2\theta \sin \theta}=-\frac{3}{11}
        \]
        即
        \[
        \frac{2\tan\theta\tan2\theta-1}{2\tan2\theta+\tan\theta}=-\frac{3}{11}
        \]
        设$t=\tan\theta$,化简得
        \[ 
        (3t - 1)(t^2 - 18t - 11) = 0 \Rightarrow \tan \theta = \frac{1}{3},9 \pm \sqrt{92}
        \]
        为了丢弃增根,解
        \[
        \frac{dy}{dx}=0 \Rightarrow \cos 2\theta \cos \theta - 2\sin 2\theta \sin \theta = 0 \Rightarrow \theta = \sin^{-1}\frac{\sqrt{6}}{6} \approx 24.1^\circ
        \]
        由于
        \[
        \tan^{-1}(9+\sqrt{92})>24.1^\circ,\quad \tan^{-1}(9-\sqrt{92})<0
        \]
        解为
        \[
        \theta=\tan^{-1}\frac{1}{3}<24.1^\circ
        \]
        当 $\tan \theta = \dfrac{1}{3}$ 时,易知$P$直角坐标为
        \[
        \left(20\cdot\frac{3}{\sqrt{10}},20\cdot\frac{1}{\sqrt{10}}\right)=(6\sqrt{10}, 2\sqrt{10})
        \]
        故直线 $L$ 的方程为
        \[ 
        y - 2\sqrt{10} = -\frac{3}{11}(x - 6\sqrt{10}) 
        \]
        当 $y = 0$ ,得 $x = \dfrac{40\sqrt{10}}{3}$。设$Q$ 是直线与 $x$ 轴的交点, $\triangle OPQ$ 的面积为
        \[ 
        [\triangle OPQ] = \frac{1}{2} \cdot \frac{40\sqrt{10}}{3} \cdot 2\sqrt{10} = \frac{400}{3} 
        \]
        从 $\theta = 0$ 到 $\theta = \tan^{-1} \dfrac{1}{3}$扫过的面积为
        \begin{align*}
        I &= \frac{1}{2} \int_{0}^{\tan^{-1} \frac{1}{3}} (25 \cos 2\theta)^2\, d\theta \\
        &= \frac{625}{4} \int_{0}^{\tan^{-1} \frac{1}{3}} (1+\cos 4\theta)2\theta\, d\theta \\
        &= \frac{625}{4} \left[ \theta + \frac{1}{4}\sin 4\theta \right]_{0}^{\tan^{-1} \frac{1}{3}} \\
        &= \frac{625}{4} \left[ \theta + \frac{1}{2}\sin 2\theta \cos 2\theta \right]_{0}^{\tan^{-1} \frac{1}{3}} \\
        &= \frac{625}{4} \tan^{-1} \frac{1}{3} + \frac{75}{2} 
        \end{align*}
        故所求面积为
        \[ 
        \frac{400}{3} - \left( \frac{625}{4} \tan^{-1} \frac{1}{3} + \frac{75}{2} \right) = \frac{25}{12} \left( 46 - 75\tan^{-1} \frac{1}{3} \right) 
        \]
        证毕。
    \end{solution}
\end{questions}
\pagebreak

\begin{center}
  {\fontsize{30pt}{26pt}\selectfont
    \hypertarget{立体几何、空间向量}{立体几何、空间向量} \label{立体几何、空间向量}
  }
\end{center}
\separator
\vspace{1pt}

\begin{questions}
    \question 一个立方体的棱长为 $4$。一根长为 $5$ 的绳子的一端固定在立方体上表面的中心。求绳子另一端能接触到的立方体表面的面积。
    \ifprintanswers
    \begin{figure}[H]
        \centering
        \includegraphics[width=0.3\textwidth]{images/image176.png}
    \end{figure}
    \fi
    \begin{solution}
        设立方体顶面为正方形 $ABCD$,中心为 $O$,则$O$到顶点的距离为 $\sqrt{2^2 + 2^2} = \sqrt{8}$。绳长 $5 > \sqrt{8}$,所以绳端可到达整个顶面面积 $16$。

        绳子无法到达底面,因为沿表面的最短路径长度为 $6 > 5$。每个侧面能到达的面积相同,设为 $a$,总可达面积为 $A = 16 + 4a$。

        考虑某一侧面正方形 $ABEF$。以 $O$ 为圆心,半径 $5$,在侧面上形成弧 $\wideparen{PQ}$,中点为 $M$,则$PM=\sqrt{5^2-2^2}=\sqrt{21}$,故绳尾能接触到的面积为
        \begin{align*}
        &[\text{矩形}\ ABQP]+[\text{扇形}\ POQ]-[\triangle OPQ]\\
        &=4(\sqrt{21}-2)+ 25 \arcsin\frac{2}{5} - 2\sqrt{21}\\
        &=2\sqrt{21}-8+ 25 \arcsin\frac{2}{5}
        \end{align*}
        总面积为
        \[
        16 + 8\sqrt{21} - 32 + 100 \arcsin\frac{2}{5} = 8\sqrt{21} -16 + 100 \arcsin\frac{2}{5} \approx 61.81
        \]
    \end{solution}

    \question 边长为 $9$, $12$, $15$ 的直角三角形框架水平放置,一个半径为 $5$ 的球卡在其中并与三边均相切。求球体高出三角形平面的高度。
    \ifprintanswers
    \begin{figure}[H]
        \centering
        \includegraphics[width=0.4\linewidth]{images/image195.png}
    \end{figure}
    \fi
    \begin{solution}
        考虑球在三角形所在平面的截面。球体的截面为$\triangle ABC$的内切圆。设圆心为 $O$,半径为 $r$,连接 $O$ 到$BC,CA,AB$上的切点 $P,Q,R$。$\triangle ABC$面积为
        \[
        \frac{1}{2}\cdot 9\cdot 12
        = \frac{1}{2}r (9+12+15) \Rightarrow r=3
        \]
        设球心为 $O'$,则 $OO'$ 垂直于三角形平面,记 $OO'=h$。任取圆周上一点,与 $O'$ 构成直角三角形。由毕氏定理,
        \[
        h=\sqrt{5^2-3^2}=4
        \]
        因此球心在平面以上 $4$ 个单位,而球的半径为 $5$,所以球顶部高出平面的高度为
        \[
        h+5=9
        \]
    \end{solution}

    \question 空间中一个立方体的三个顶点坐标为 $A(3,4,1),B(5,2,9),C(1,6,5)$。求立方体的中心坐标。
    \begin{solution}
        观察
        \[
        AB=\sqrt{(5-3)^2+(2-4)^2+(9-1)^2}=\sqrt{72},
        \]
        \[
        AC=\sqrt{(1-3)^2+(6-4)^2+(5-1)^2}=\sqrt{24},
        \]
        \[
        BC=\sqrt{(1-5)^2+(6-2)^2+(5-9)^2}=\sqrt{48}
        \]
        此时发现
        \[
        AB^2=AC^2+BC^2
        \]
        故可知 $AB$ 是经过立方体中心的对角线,因此立方体中心即为 $AB$ 的中点
        \[
        O=\left(\frac{3+5}{2}, \frac{4+2}{2}, \frac{1+9}{2}\right)=(4,3,5)
        \]
    \end{solution}

    \question 求内接在单位立方体中的正方形的最大面积,其中正方形的每个顶点都在立方体的边上。
    \begin{solution}
        设立方体的顶点坐标为 $(0,0,0), (0,0,1), \dots, (1,1,1)$。由对称性,最大内接正方形的顶点取
        \[
        (x,0,0), \quad (1,0,1-x), \quad (1-x,1,1), \quad (0,1,x)
        \]
        其中 $0<x<1$。两条相邻边的平方长度相等,解得
        \[
        2x^2 + 1 = 2(1-x)^2 \implies x = \frac{1}{4}
        \]
        此时最大正方形的面积为
        \[
        ||(x,0,0)-(1,0,1-x)||^2 = \frac{9}{8}
        \]
    \end{solution}

    \question 已知一边长为 $4$、顶面为 $ABCD$、底面为 $EFGH$ 的立方体,且 $A$ 在 $E$ 正上方,依此类推。若点 $X,Y,Z$ 在 $AD,HG,BF$ 上使得$AX,HY,FZ=1$,求 $\triangle XYZ$ 的面积。
    \ifprintanswers
    \begin{figure}[H]
        \centering
        \includegraphics[width=0.4\linewidth]{images/image144.png}
    \end{figure}
    \fi
    \begin{solution}
        在$\triangle AXZ$中,由毕氏定理,
        \[
        AZ=\sqrt{3^2 + 4^2} = 5
        \]
        在$\triangle ABZ$中,由毕氏定理,
        \[
        XZ=\sqrt{1^2 + 5^2} = \sqrt{26}
        \]
        $\triangle XYZ$为等边三角形,因此面积为
        \[
        [\triangle XYZ] = \frac{\sqrt{3}}{4} \cdot 26 = \frac{13\sqrt{3}}{2}
        \]
    \end{solution}

    \question 从天花板点 $O$ 悬挂三根长 $100$ cm 的导线 $OXM,OYN,OZP$,依次穿过边长为 $60$ cm 的等边三角形 $XYZ$ 的顶点,末端挂有灯泡。三角形平面与天花板平行,三个灯泡距天花板均为 $90$ cm。求三角形到天花板的垂直距离。
    \begin{figure}[H]
        \centering
        \includegraphics[width=0.3\linewidth]{images/image212.png}
    \end{figure}
    \ifprintanswers
    \begin{figure}[H]
        \centering
        \includegraphics[width=0.35\linewidth]{images/image213.png}
    \end{figure}
    \fi
    \begin{solution}
        设$OX=OY=OZ=x$。由于导线全长为$100$, 则$XM=YN=ZP=100-x$。又灯泡距天花板为$90$, 所以木质三角形至天花板的距离为
        \[
        90-(100-x)=x-10
        \]
        设$C$为$\triangle XYZ$的中心, 由对称性$O$在$C$正上方, 且$OC=x-10$。在$\triangle OXC$中,由毕氏定理,
        \[
        x^2 = (x-10)^2+\left(\frac{30}{\cos 30^\circ}\right)^2 
        \]
        解得
        \[
        x=65
        \]
        因此木质三角形到天花板的高度为
        \[
        x-10=55\ \text{cm}.
        \]
    \end{solution}

    \question 立方体被两平面分成四块:一平面平行于面 $ABCD$ 且过棱 $BG$ 中点,另一平面过棱 $AB,AD,GH,HE$ 的中点。求最小块与最大块的体积比。
    \begin{figure}[H]
        \centering
        \includegraphics[width=0.35\linewidth]{images/image203.png}
    \end{figure}
    \ifprintanswers
    \begin{figure}[H]
        \centering
        \includegraphics[width=0.35\linewidth]{images/image204.png}
    \end{figure}
    \fi
    \begin{solution}
        不失一般性,设立方体边长为 2。平面 $PQRS$ 经过 $BG$ 的中点并平行于面 $ABCD$,其中$P,Q,R,S$为$AF,BG,CH,DE$的中点。另一个平面经过 $K, L, M, N$,即 $AB, GH,HE,AD$ 的中点。延长 $QK$ 和 $SN$ 相交于点 $T$,注意到 $\triangle TAN \backsim \triangle TPS \ \text{(AAA)}$,于是
        \[
        \frac{TA}{TP} = \frac{AN}{PS} \Rightarrow TA = 1
        \]
        于是
        \begin{align*}
        V_{\text{小}}=[AKN-PQS]&=[\text{四面体} \ T-PQS]-[\text{四面体} \ T-AKN]\\
        &=\frac{1}{3}\left(\frac{1}{2}\cdot 2 \cdot 2\right) - \frac{1}{3}\left(\frac{1}{2}\cdot 1 \cdot 1\right) = \frac{7}{6}
        \end{align*}
        且
        \[
        V_{\text{大}} = [ABCD-PQRS]- [AKN-PQS]= 2 \cdot 2 \cdot 1 - \frac{7}{6} = \frac{17}{6}
        \]
        故所求体积之比为
        \[
        \frac{V_{\text{小}}}{V_{\text{大}}} = \frac{7}{17}
        \]
    \end{solution}

    \question 边长为$3$的立方体置于底面直径为$8$、高为$24$的圆锥内,立方体的体对角线与圆锥轴线重合。求圆锥顶点到立方体最近顶点的距离。
    \begin{figure}[H]
        \centering
        \includegraphics[width=0.3\linewidth]{images/image205.png}
    \end{figure}
    \ifprintanswers
    \begin{figure}[H]
        \centering
        \includegraphics[width=0.35\linewidth]{images/image206.png}
        \includegraphics[width=0.35\linewidth]{images/image207.png}
        \includegraphics[width=0.18\linewidth]{images/image208.png}
    \end{figure}
    \fi
    \begin{solution}
        如图,设点$A$为立方体最底端的顶点,顶点$B,C,D$在圆锥内表面上且由对称性形成等边三角形,点$Q$为立方体最顶端的顶点。$BQ$ 是立方体的面对角线,$G$ 为$BCD$的重心,且 $QGA$ 在圆锥轴线上。

        所求距离为圆锥顶点 $T$ 到平面 $BCD$ 的距离,减去 $A$ 到平面 $BCD$ 的距离。

        等四面体$O-BCD$边长为
        \[
        BQ=BC=BD=CD=\sqrt{3^2+3^2}=3\sqrt{2},
        \]
        在 $\triangle BCD$中,
        \[
        BG=\frac{2}{3}\cdot 3\sqrt{2}\cos 30^\circ=\sqrt{6}
        \]
        设 $O$ 为圆锥底面圆心,$S$ 为底面上一点。由 $\triangle TGB \backsim \triangle TOS \ \text{(AAA)}$,得
        \[
        \frac{TG}{TO}=\frac{GB}{OS} \Rightarrow TG=24\cdot \frac{\sqrt{6}}{4}=6\sqrt{6}
        \]
        由毕氏定理,立方体空间对角线为
        \[
        AQ=\sqrt{3^2+3^2+3^2}=3\sqrt{3}
        \]
        在 $\triangle BGQ$ 中,由毕氏定理,
        \[
        QG=\sqrt{(3\sqrt{2})^2-(\sqrt{6})^2}=2\sqrt{3}
        \]
        因此
        \[
        AG=AQ-QG=\sqrt{3}
        \]
        故所求距离为
        \[
        TA=TG-AG=6\sqrt{6}-\sqrt{3}
        \]
    \end{solution}

    \question 已知 $ABCD$ 和 $PNCD$ 都是边长为 $2$ 的正方形,且这两个正方形互相垂直(它们共享边 $CD$)。在 $AB$ 一侧取点 $M$,使得平面 $\triangle PMN$ 平行于平面 $ABCD$,且 $\angle PMN=90^\circ,PM=MN$。求凸多面体 $PMN-ABCD$ 的体积。
    \begin{figure}[H]
        \centering
        \includegraphics[width=0.4\linewidth]{images/image180.png}
    \end{figure}
    \ifprintanswers
    \begin{figure}[H]
        \centering
        \includegraphics[width=0.4\linewidth]{images/image181.png}
        \includegraphics[width=0.4\linewidth]{images/image182.png}
    \end{figure}
    \fi
    \begin{solution}
        过 $M$ 作一条与 $PN$ 平行的线分别与平面 $PAD,NBC$ 相交于 $Q,R$,取 $T$ 为 $PN$ 的中点,由题意可知 $\triangle MTN$ 为直角等腰三角形,故 
        \[
        NR=MT=1
        \]
        考虑多面体 $PQRN-ABCD$以及两个全等棱锥 $PMQ-A,NMR-B$,于是
        \begin{align*}
        [PMN-ABCD]
        &=[PQRN-ABCD]-2\cdot [PMQ-A]\\
        &=\frac{1}{2}(1+2)\cdot 2\cdot 2 - 2 \cdot \frac{1}{3}\cdot\frac12\cdot 2
        =\frac{16}{3}
        \end{align*}
    \end{solution}
    \ifprintanswers
    \begin{figure}[H]
        \centering
        \includegraphics[width=0.4\linewidth]{images/image182.png}
        \includegraphics[width=0.4\linewidth]{images/image183.png}
    \end{figure}
    \fi
    \begin{solution}
        沿平面 $ABNP$ 分割凸多面体$PMN-ABCD$,考虑棱柱$PN-ABCD$及角锥$M-ABNP$,在$\triangle BCN$中,由毕氏定理,
        \[
        N=\sqrt{2^2+2^2}=2\sqrt2
        \]
        取 $S,T$ 为 $AB,PN$ 中点,由于 $\triangle MTN$ 为直角等腰三角形,故$US=BC-MT=1$, 在$\triangle MUS$中,由毕氏定理,
        \[
        MS=\sqrt{1^2+2^2}=\sqrt{5}
        \]
        在$\triangle MTS$中,由余弦定理,
        \[
        (\sqrt{5})^2 = 1^2+(2\sqrt{2})^2 -2\cdot 1 \cdot 2\sqrt{2} \cdot \cos \angle MTS
        \Rightarrow \angle MTS=45^\circ
        \]
        设顶点 $M$ 到底面的垂高为 $h$,则
        \[
        h=MT\sin45^\circ=\frac{\sqrt{2}}{2}
        \]
        故凸多面体$PMN-ABCD$体积为
        \begin{align*}
        [PMN-ABCD]
        &=[PN-ABCD]+[M-ABNP] \\
        &=\frac{1}{2}\cdot 2 \cdot 2 \cdot 2 + \frac{1}{3}\cdot 2\cdot 2\sqrt2\cdot\frac{\sqrt2}{2}
        =\frac{16}{3}
        \end{align*}
    \end{solution}

    \question 三个完全相同的圆锥,底面半径为50,高为120,它们的底面两两外切。在这三个圆锥围成的空隙中放置一个球,使得球的最高点与三个圆锥的顶点处于同一水平高度。求该球的半径。
    \begin{figure}[H]
        \centering
        \includegraphics[width=0.4\linewidth]{images/image209.png}
    \end{figure}
    \ifprintanswers
    \begin{figure}[H]
        \centering
        \includegraphics[width=0.4\linewidth]{images/image210.png}
        \includegraphics[width=0.3\linewidth]{images/image211.png}
    \end{figure}
    \fi
    \begin{solution}
        设三圆锥圆心为$A,B,C$,则底面圆心形成边长为$100$的正三角形$\triangle ABC$,且球心位于该正三角形的重心$G$正上方。设$D$为$BC$中点。在$\triangle BGD$中,
        \[
        BG = \frac{BD}{\cos 30^\circ} = \frac{100\sqrt{3}}{3}
        \]
        即重心到每个顶点的距离为$\dfrac{100\sqrt{3}}{3}$,也即为圆锥轴与通过球心的竖直线的水平距离。

        现考虑球体与圆锥的竖直截面。设$O$为球心, $r$为球半径,$P$为球体与圆锥斜面上的切点,$X$为圆锥顶点,$Y$为$XP$延长线与$AG$的交点,$H$为球体的最高点,则
        \[
        OH=OP=r,XH=AG=\frac{100\sqrt{3}}{3},XA=HG=120
        \]  
        在$\triangle AXY$中,由毕氏定理,
        \[
        XY = \sqrt{50^2+120^2} =130 \Rightarrow PY = XY - XP = 130 - \frac{100\sqrt{3}}{3}
        \]
        在$\triangle AXY$中,由毕氏定理,有$OY^2=OP^2+PY^2=OG^2+GY^2$,即
        \[
        r^2 + \left(130-\frac{100\sqrt{3}}{3}\right)^2 = (120-r)^2+\left(\frac{100\sqrt{3}}{3}-50\right)^2
        \]
        解得
        \[
        r=\frac{200\sqrt{3}}{9}
        \]
    \end{solution}
    \begin{solution}
        已知$XH=AG=\dfrac{100\sqrt{3}}{3},AX=120,AY=50$,由于$XY \perp OP$,
        \[
        m_{XY}=-\frac{12}{5} \Rightarrow m_{OP}=\frac{5}{12}
        \]
        设$OS=5t,SP=12t$,则
        \[
        OP=13t,\quad OH=13t,\quad XR=18t
        \] 
        又
        \[
        RS=RP+PS=\frac{15t}{2}+12t=\frac{39t}{2}=AG=\frac{100\sqrt{3}}{3}
        \]
        因此$t=\dfrac{200}{39\sqrt{3}}$, 球半径为
        \[
        r=OP=13t=\frac{200\sqrt{3}}{9}
        \]
    \end{solution}

    \question 一个正四棱锥的底面是边长为20的正方形。现有一个底面直径为10、高为10的圆柱横放(侧卧),要完全放置在锥体内部。圆柱的中心轴线平行于底面对角线,且位于对角线正上方。圆柱中心轴的中点在底面中心的正上方。求金字塔的最小高度。
    \begin{figure}[H]
        \centering
        \includegraphics[width=0.5\linewidth]{images/image220.png}
    \end{figure}
    \ifprintanswers
    \begin{figure}[H]
        \centering
        \includegraphics[width=0.3\linewidth]{images/image221.png}
        \includegraphics[width=0.4\linewidth]{images/image222.png}
        \includegraphics[width=0.25\linewidth]{images/image223.png}
    \end{figure}
    \fi
    \begin{solution}
        当正四棱锥四个侧面与圆柱两端面的圆周均相切,正四棱锥高度为最小。设正方形底面为 $ABCD$,顶点为 $T$,对角线交点为 $M$,则 $AM=BM=CM=DM=10\sqrt{2}$,顶点 $T$ 位于 $M$ 的正上方,正四棱锥高度为 $t=TM$。

        圆柱的中心轴线位于对角线 $AC$ 上,其中点在 $M$ 处。圆柱沿 $AC$ 方向在 $M$ 两侧各延伸 $5$ 个单位,两端点分别为 $E$ 和 $F$。从 $A$ 到最近的圆柱端点 $E$ 的距离为 $AE = 10\sqrt{2}-5$。

        在包含 $AC$ 和 $T$ 的垂直截面中,设 $L$ 在 $AT$ 上,$G$, $H$ 分别在 $AB$, $AD$ 上。由于 $\angle BAM=45^\circ$ 且圆柱端面垂直于对角线,$\triangle GEA$ 和 $\triangle HEA$ 均为等腰直角三角形,因此 $GE=HE=AE=10\sqrt{2}-5$。设圆柱横截面的圆心为 $O$,半径为 $5$。

        设 $LE=h$,则 $LO=h-5,LJ=x$,由$\triangle LJO \backsim \triangle LEG \ \text{(AAA)}$,
        \[
        \frac{LJ}{JO}=\frac{LE}{EG} \Rightarrow \frac{x}{h} = \frac{5}{10\sqrt{2}-5} \Rightarrow x = \frac{h}{2\sqrt{2}-1}
        \]
        且
        \[
        \frac{LG}{GE}=\frac{LO}{OJ} \Rightarrow \frac{x + (10\sqrt{2}-5)}{h-5} = \frac{10\sqrt{2}-5}{5}
        \]
        代入 $x = \dfrac{h}{2\sqrt{2}-1}$ 得
        \[
        h = \frac{10(2\sqrt{2}-1)^2}{8-4\sqrt{2}}
        \]
        又由 $\triangle AEL \backsim \triangle AMT \ \text{(AAA)}$,有
        \[
        \frac{t}{AM} = \frac{LE}{AE} \Rightarrow t = \frac{10\sqrt{2}\cdot h}{10\sqrt{2}-5}=15+5\sqrt{2}
        \]
    \end{solution}

    \question 在空间中,有三个半径均为 $10$ 的圆,它们两两外切,且都与同一平面相切。每个圆所在的平面与该平面成 $45^\circ$ 的倾角。这三个圆的三个切点位于一个与该平面平行的圆上。求此圆的半径。
    \begin{figure}[H]
        \centering
        \includegraphics[width=0.5\linewidth]{images/image215.png}
    \end{figure}
    \ifprintanswers
    \begin{figure}[H]
        \centering
        \includegraphics[width=0.8\linewidth]{images/image216.png}
        \includegraphics[width=0.3\linewidth]{images/image217.png}
        \includegraphics[width=0.35\linewidth]{images/image218.png}
        \includegraphics[width=0.3\linewidth]{images/image219.png}
    \end{figure}
    \fi
    \begin{solution}
        设三个圆与平面 $\pi$ 的切点分别为 $A,B,C$。每个圆所在平面与 $\pi$ 的交线两两相交,交点分别为 $D,E,F$,其中 $A,B,C$ 在 $DE,EF,FD$ 上。由对称性,$\triangle DEF$ 是等边三角形,记 $O$ 为 $\triangle DEF$ 的重心。

        设三个圆两两相切的切点为 $G,H,J$,且 $\triangle GHJ$ 是等边三角形,所求圆即为过 $G,H,J$ 三点的圆。延长$DG,EH,FJ$交于$T$,于是$T$在$O$正上方, $T,D,E,F$ 构成一四面体。

        侧面$\triangle DET$与$\pi$成 $45^\circ$ 角,且 $TO\perp \pi$,故 $\triangle TAO$ 是等腰直角三角形,因此
        \[
        TA=\sqrt{2}\;AO
        \]
        考虑$\triangle DEF$,设 $DA=AE=x$。在 $\triangle ADO$中,
        \[
        OA=\frac{x}{\sqrt{3}}, \quad OD=\frac{2x}{\sqrt{3}}.
        \]
        在 $\triangle TAD$中,由毕氏定理,
        \[
        TD=\sqrt{x^2+\left(\sqrt{2}\cdot\frac{x}{\sqrt{3}}\right)^2} = \sqrt{\frac{5}{3}}x
        \]
        考虑侧面$\triangle DET$,设 $R$ 为 $\triangle DET$的内切圆圆心, $S$ 为 $TA$ 与 $GH$ 的交点,则 $RG=10$。由于$\triangle TSG \backsim \triangle TGR \backsim \triangle TAD \ \text{(AAA)}$,
        \[
        \frac{TG}{GR} = \frac{TA}{AD} = \frac{\frac{\sqrt{2}}{\sqrt{3}}x}{x} \Rightarrow TG = \frac{\sqrt{2}}{\sqrt{3}}GR = 10\frac{\sqrt{2}}{\sqrt{3}}
        \]
        且
        \[
        \frac{SG}{TG} = \frac{AD}{TD} = \frac{x}{\frac{\sqrt{5}}{\sqrt{3}}x} \Rightarrow SG = \frac{\sqrt{3}}{\sqrt{5}}TG=2\sqrt{10}
        \]
        考虑等边三角形$\triangle GHJ$,设 $L$ 为 $\triangle GHJ$ 的中心,则 $LG$ 即为所求圆的半径:
        \[
        LG=\frac{2}{\sqrt{3}}\,SG=\frac{4\sqrt{30}}{3}
        \]
    \end{solution}

    \question 一个球的内接圆锥的最大体积与这个球的体积之比为?
    \begin{solution}
        设球的半径为 $R$,其内接圆锥的高为 $h$,底面半径为 $r$。由几何关系可得:
        \[
        r^2 = R^2 - |h - R|^2 = (2R - h)h.
        \]
        故圆锥的体积为:
        \[
        V_{\text{锥}} = \frac{1}{3} \pi r^2 h = \frac{1}{3} \pi h^2 (2R - h).
        \]
        设函数 \( V(h) = \dfrac{1}{3} \pi h^2 (2R - h) \),由AM-GM不等式得
        \[
        V(h) \le \frac{4\pi}{3} \left( \frac{\frac{h}{2} + \frac{h}{2} + 2R - h}{3} \right)^3 
        = \frac{32}{81} \pi R^3\]
        当 $h = \dfrac{4}{3}R$ 时取得最大值$\dfrac{32}{81} \pi R^3$,故体积之比为
        \[
        \frac{V_{\text{锥}}}{V_{\text{球}}} = \frac{32}{81} \pi R^3 \div \frac{4}{3} \pi R^3 = \frac{8}{27}.
        \]
    \end{solution}

    \question 一个圆柱形容器内装有水。现有一个实心圆锥,其高度等于圆柱的高度,底面半径是圆柱底面半径的一半。将此圆锥完全浸入水中,底面紧贴圆柱底面,顶点朝上。此时观察到水面高度恰好是圆柱高度的 $\dfrac{1}{2}$。若将圆锥从水中取出,求水面高度占圆柱高度的比值。
    \begin{figure}[H]
        \centering
        \includegraphics[width=0.35\linewidth]{images/image177.png}
    \end{figure}
    \begin{solution}
        设圆柱半径为 $r$,高度为 $h$,则圆锥体积为
        \[
        \frac{1}{3} \pi \left(\frac{1}{2} r\right)^2 h = \frac{1}{12} \pi r^2 h.
        \]
        当圆锥在圆柱内且水深为 $\dfrac{h}{2}$ 时,水体积为半圆柱体积减去下半圆锥体积
        \[
        \frac{1}{2} \pi r^2 h - \left(\frac{1}{12}\pi r^2 h - \left(\dfrac{1}{2}\right)^3 \cdot \frac{1}{12} \pi r^2 h \right)= \frac{41}{96} \pi r^2 h.
        \]
        圆锥移开后,水体积保持不变,因此水深是圆柱高度的 $\dfrac{41}{96}$。
    \end{solution}

    \question 一个三棱柱形容器竖直放置,底面为三角形。容器内放入三个半径为 $1$ 的球,每个球都与三角形底面相切,与两个相邻的长方形侧面相切,且三个球两两外切。在这三个球的上方放入第四个半径为 $1$ 的球,该球与下方三个球均外切,并与棱柱的顶面(上底面)相切。求该三棱柱的体积。
    \ifprintanswers
    \begin{figure}[H]
        \centering
        \includegraphics[width=0.4\linewidth]{images/image163.png}
        \includegraphics[width=0.4\linewidth]{images/image164.png}
    \end{figure}
    \fi
    \begin{solution}
        先计算底面积。取底面上方 1 个单位的截面,该截面通过每个球的球心以及球与侧面的切点。设$A,B,C$为截面三角形顶点,$X,Y,Z$为球心,$M,N,P,Q,R,S$为球与侧面的切点且在$AB,BC,CA$上。由对称性可知 $\triangle ABC$ 是等边三角形,边长为
        \[
        BC = BP + PQ + QC = 2 + 2\sqrt{3}
        \]  
        底面积为
        \[
        [\triangle ABC] = \frac{\sqrt{3}}{4}(2+2\sqrt{3})^2
        \]
        现求高。设$W$为顶球球心,则四球相切,$WXYZ$ 为正四面体,边长皆为 2。设 $V$ 为 $\triangle XYZ$ 的重心,$G$ 为 $YZ$ 中点,
        在$\triangle YVG$中,
        \[
        YV = \frac{2}{\sqrt{3}},\quad YG = \frac{2\sqrt{3}}{3}
        \]  
        在$\triangle WYV$中,
        \[
        WV = \sqrt{2^2 - \frac{4}{3}} = \sqrt{\frac{8}{3}}
        \]  
        底面到截面 $XYZ$ 平面距离为 1,顶球球心到顶面的垂直距离为 1,三棱柱高度为$ 2 + \sqrt{\dfrac{8}{3}}$,故三棱柱体积为
        \[
        \frac{\sqrt{3}}{4}(2+2\sqrt{3})^2 \cdot \left(2 + \sqrt{\frac{8}{3}}\right) \approx 46.97
        \]
    \end{solution}

    \question 如图,已知一边长为 $200$ 的立方体 $FGHJKLMN$,点 $P$ 在 $HG$ 上。已知从 $G$ 到 $\triangle PFM$ 上一点的最短距离为 $100$,求 $HP$ 的长度。
    \begin{figure}[H]
        \centering
        \includegraphics[width=0.4\linewidth]{images/image168.png}
    \end{figure}
    \begin{solution}
        设 $GP = x$,则 $HP = 200 - x$,考虑四面体 $FGMP$体积:
        \[
        \frac{1}{3}[\triangle FGM]\cdot x=\frac{1}{3}[\triangle PFM]\cdot 100 \tag{1}
        \]
        其中
        \[
        [\triangle FGM] = \frac{1}{2}\cdot FG \cdot GM = 20000
        \]
        现以$x$表示$\triangle PFM$ 的面积。在$\triangle FGM,\triangle FGP$中,由毕氏定理,
        \[
        FM = 200\sqrt{2}, \quad FP = \sqrt{x^2 + 200^2}
        \]
        由于$FP = MP$, $\triangle PFM$ 为等腰三角形。设 $T$ 为 $FM$ 中点,则
        \[
        FT = TM = 100\sqrt{2}
        \]  
        在$\triangle FTP$中,由毕氏定理,
        \[
        PT = \sqrt{FP^2 - FT^2} = \sqrt{x^2 + 40000 - (100\sqrt{2})^2} = \sqrt{x^2 + 20000}
        \]
        所以
        \[
        [\triangle PFM] = \frac{1}{2} \cdot FM \cdot PT = \frac{1}{2} \cdot 200\sqrt{2} \cdot \sqrt{x^2 + 20000}
        \]
        由$(1)$得,
        \[
        \frac{1}{3} \cdot 20~000 \cdot x = \frac{1}{3} \cdot \left(\frac{1}{2} \cdot 200\sqrt{2} \cdot \sqrt{x^{2}+20~000}\right) \cdot 100
        \]
        解得
        \[
        x = 100\sqrt{2} >0 \Rightarrow HP = 200 - 100\sqrt{2}
        \]
    \end{solution}

    \question 长方体 $PQRSTUVW$ 的顶点按图中方式标记。已知 $PR=1867,PV=2019,PT=x$,问:有多少个正整数 $x$ 使得满足这些条件的长方体存在?
    \begin{figure}[H]
        \centering
        \includegraphics[width=0.5\linewidth]{images/image167.png}
    \end{figure}
    \begin{solution}
        设 $PQ=a,PS=b,PU=c$,在$\triangle PQR,\triangle PQV,\triangle PST$中,由毕氏定理,
        \[
        a^2 + b^2 = 1867^2, \quad a^2 + c^2 = 2019^2, \quad b^2 + c^2 = x^2
        \]
        三式解得
        \[
        a^2 = \frac{1867^2 + 2019^2 - x^2}{2}, \quad
        b^2 = \frac{1867^2 - 2019^2 + x^2}{2}, \quad
        c^2 = \frac{-1867^2 + 2019^2 + x^2}{2}
        \]
        由 $a,b,c>0$,
        \[
        \begin{cases}
            x^2 < 1867^2 + 2019^2 \\
            x^2 > 2019^2 - 1867^2 \\
            x^2 > 1867^2 - 2019^2 
        \end{cases}
        \]
        其中最后一个不等式恒成立,即  
        \[
        768.55 \approx \sqrt{2019^2 - 1867^2} < x < \sqrt{2019^2 + 1867^2} \approx 2749.92
        \]
        因此满足条件的整数 $x$ 个数为
        \[
        2749 - 769 + 1 = 1981
        \]
    \end{solution}

    \question 如图所示,有一个直圆锥台,上底面直径为 $2$ cm,下底面直径为 $8$ cm,高为 $6\sqrt{2}$ cm。$AB$ 与 $CD$ 为直圆锥台的两侧边,$E$ 为 $AB$ 上一点,且 $AE=3$ cm。求从点 $B$ 出发,沿圆锥台侧面经过母线 $CD$,最后到达点 $E$ 的最短路径长度。
    \begin{figure}[H]
        \centering
        \includegraphics[width=0.4\linewidth]{images/image111.png}
    \end{figure}
    \ifprintanswers
    \begin{figure}[H]
        \centering
        \includegraphics[width=0.4\linewidth]{images/image112.jpg}
    \end{figure}
    \begin{figure}[H]
        \centering
        \includegraphics[width=0.7\linewidth]{images/image113.jpg}
    \end{figure}
    \fi
    \begin{solution}
        设$O$ 为 $BC$ 的中点,$P$ 为 $BA$ 及$DC$的交点,$Q$ 为 $AD$ 的中点,由$\triangle PQD \backsim \triangle POC \ \text{(AAA)}$,
        \[
        \frac{PQ}{PQ+6\sqrt 2} = \frac{1}{4} \Rightarrow PQ=2\sqrt 2
        \]
        在 $\triangle POC$ 中,由毕氏定理,
        \[
        PC = \sqrt{(6\sqrt 2 + 2\sqrt 2)^2 + 4^2} =12
        \]
        又
        \[
        \frac{PD}{PC} = \frac{1}{4} \Rightarrow PD=3
        \]
        沿 $PB$ 将圆锥剪开,则 $AB=CD=9-3=6$。由弧长公式
        \[
        \wideparen{AA'} = 3 \angle APA' \Rightarrow \angle APA' = \frac{2\pi}{3}
        \]
        在 $\triangle PBE'$ 中,由余弦定理,
        \[
        BE^2 = 12^2 + 6^2 - 2 \cdot 12 \cdot 6 \cdot \cos \frac{2\pi}{3} \Rightarrow BE = 6\sqrt{7}
        \]
        因此最短曲线长为 $6\sqrt{7}$。
    \end{solution}

    \question 已知点 \(O(0,0,0)\), \(A(1,0,0)\), \(B\left(\dfrac{1}{2}, \dfrac{\sqrt{3}}{2}, 0\right)\),  
    \begin{parts}
    \part 求在 \(xy\) 平面上方的点 \(C\),使得四面体 \(OABC\) 是正四面体。  
    \begin{solution}
        向量 \(\overrightarrow{OA}, \overrightarrow{OB}\) 长度均为 1,且夹角为 \(60^\circ\),可知 \(\triangle OAB\) 是边长为 1 的正三角形,重心为
        \[
        G = \left( \frac{0+1+\frac{1}{2}}{3}, \frac{0+0+\frac{\sqrt{3}}{2}}{3}, 0 \right) = \left( \frac{1}{2}, \frac{\sqrt{3}}{6}, 0 \right)
        \]
        为使四面体为正四面体,点 \(C\) 应在重心垂直向上、距离为正三角形高的方向上:
        \[
        OC = \sqrt{1^2 - \left( \frac{1}{\sqrt{3}} \right)^2 }=\frac{\sqrt{6}}{3}
        \]
        故
        \[
        C = \left( \frac{1}{2}, \frac{\sqrt{3}}{6}, \frac{\sqrt{6}}{3} \right)
        \]
    \end{solution}
    \part 计算其体积。
    \begin{solution}
        体积为
        \[
        V = \frac{1}{3} [\triangle ABC] \cdot h 
        = \frac{1}{3} \cdot \frac{\sqrt{3}}{4} \cdot 1^2 \cdot \frac{\sqrt{6}}{3} 
        = \frac{\sqrt{2}}{12}        
        \]
    \end{solution}
    \end{parts}
    
    \question 空间中,设 $\triangle ABC$ 的三边长 $AB=3,AC=5,BC=7$,另有一点 $P$ 满足 $PA = PB = PC = \dfrac{25\sqrt{3}}{3}$,则锥体 $P-ABC$ 的体积为多少?
    \begin{solution}
        由题意可知,$P$ 在 $\triangle ABC$ 的投影点为外心。设$a = BC = 7, b = AC = 5, c = AB = 3$,半周长$s = \dfrac{15}{2},\triangle ABC$面积为
        \[
        [\triangle ABC] = \sqrt{\frac{15}{2} \cdot \frac{1}{2} \cdot \frac{5}{2} \cdot \frac{9}{2}} = \frac{15\sqrt 3}{4}
        \]  
        所以外接圆半径
        \[
        R = \frac{abc}{4 [\triangle ABC]} =\frac{7\sqrt 3}{3}
        \]  
        由毕氏定理,锥体$P-ABC$高为
        \[
        h = \sqrt{\left(\frac{25\sqrt 3}{3}\right)^2 - \left(\frac{7\sqrt 3}{3}\right)^2} = 8\sqrt 3
        \]  
        体积为
        \[
        [P-ABC] = \frac{1}{3} \cdot [\triangle ABC] \cdot h =30
        \]  
    \end{solution}

    \question 已知一四面体 $P-ABC$ 中,$\ \angle APB = \angle BPC = \angle CPA = 60^{\circ}$,且 $\triangle APB,\triangle BPC,\triangle CPA$ 的面积分别为 $\dfrac{\sqrt{3}}{2},2,1$,求四面体 $P-ABC$ 的体积。
    \begin{solution}
        设$PA=a,PB=b,PC=c$,由已知面积和角度,
        \[
        \begin{cases}
            \dfrac{1}{2} ab \sin 60^\circ = \dfrac{\sqrt{3}}{2}\\[6pt]
            \dfrac{1}{2} bc \sin 60^\circ = 2 \\[6pt]
            \dfrac{1}{2} ac \sin 60^\circ = 1 \\[6pt]
        \end{cases}
        \Rightarrow ab = 2 , bc = \frac{8}{\sqrt{3}} , ac = \frac{4}{\sqrt{3}}
        \]
        解得
        \[
        a = 1, \quad b = 2, \quad c = \frac{4}{\sqrt{3}}
        \]
        在 $PB,PC$ 上取点 $B',C'$使得 $PB' = PC' = PA = 1$,此时 $PAB'C'$ 为正四面体,其高为
        \[
        h = \sqrt{\frac{2}{3}} \cdot 1 = \sqrt{\frac{2}{3}}
        \]
        由于点 $A$ 到平面 $PB'C'$ 的距离等于点 $A$ 到平面 $PBC$ 的距离,四面体 $P-ABC$ 的体积为
        \[
        [P-ABC] = \frac{1}{3} \cdot [\triangle BPC] \cdot h = \frac{2\sqrt{6}}{9}
        \]
    \end{solution}

    \question 已知正四面体 $O-ABC$ 的边长为 $6\sqrt{2},D,E,F,G$ 四点分别在 $OA, OB, OC, BC$ 上。若 
    \[
    OD=\frac{1}{6}OA,\quad OE=\frac{1}{3}OB,\quad OF=\frac{1}{2}OC
    \] 
    且 $G$ 为 $BC$ 中点,求四面体 $DEFG$ 的体积。
    \begin{solution}
        四面体边长为 $6\sqrt{2}$,构造坐标系如下:
        \[
        \begin{cases}
        O = (0,0,0) \\
        A = (6,0,6) \\
        B = (0,6,6) \\
        C = (6,6,0)
        \end{cases}
        \Rightarrow
        \begin{cases}
        {OD} = \frac{1}{6} {OA} \Rightarrow D = \frac{A + 5O}{6} = (1,0,1) \\
        {OE} = \frac{1}{3} {OB} \Rightarrow E = \frac{B + 2O}{3} = (0,2,2) \\
        {OF} = \frac{1}{2} {OC} \Rightarrow F = \frac{O + C}{2} = (3,3,0) \\
        G = \text{BC 中点} = \frac{B + C}{2} = (3,6,3)
        \end{cases}
        \]
        于是
        \[
        \overrightarrow{GD} = (-2,-6,-2),\overrightarrow{GE} = (-3,-4,-1),\overrightarrow{GF} = (0,-3,-3)
        \]
        体积为
        \[
        [DEFG] = \frac{1}{6} \left| \overrightarrow{GD} \cdot ( \overrightarrow{GE} \times \overrightarrow{GF} ) \right|
        = \frac{1}{6} \cdot 18 = 3
        \]
    \end{solution}

    \question 四面体 $D-ABC$ 的体积为$\dfrac{1}{6}$,且 $\angle ACB=60^{\circ}$,  
    \[
    AD + \sqrt{3}\: BC + \frac{AC}{2} = 3,
    \]
    求 $CD$ 的长度。
    \begin{solution}
        由AM-GM不等式,
        \[
        1=\dfrac{AD + \sqrt{3} \cdot BC + \dfrac{AC}{2}}{3} \ge \sqrt[3]{\frac{\sqrt{3}}{2} \cdot AD \cdot BC \cdot AC}
        \]
        即  
        \[
        AD \cdot BC \cdot AC \le \frac{2}{\sqrt{3}} \tag{1}
        \]
        四面体体积为  
        \[
        \frac{1}{6}=[D-ABC] \le \frac{1}{3} \cdot [\triangle ABC] \cdot AD = \frac{1}{6} AC \cdot BC \sin 60^\circ \cdot AD
        \]
        即  
        \[
        AD \cdot BC \cdot AC \ge \frac{2}{\sqrt{3}} \tag{2}
        \]
        由$(1),(2)$得  
        \[
        AD \cdot BC \cdot AC = \frac{2}{\sqrt{3}}
        \]
        即不等式$(1)$等号成立,当且仅当
        \[
        AD = \sqrt{3} \cdot BC = \frac{AC}{2} = 1
        \]
        且由(2)知$AD$ 与 $\triangle ABC$ 垂直,在 $\triangle ACD$中,由毕氏定理,  
        \[
        CD = \sqrt{1^2 + 2^2} = \sqrt{5}
        \]
    \end{solution}
    
    \question 已知一边长为2的正方形纸 $ABCD,M,N$ 分别为 $BC$ 与 $CD$ 的中点,今沿着 $AM,AN$ 与 $MN$ 折起,使 $B,C,D$ 三点重合。若平面 $ABM$ 与平面 $AMN$ 的夹角为 $\theta$,求 $\sin \theta$。
    \begin{figure}[H]
        \centering
        \includegraphics[width=0.5\linewidth]{images/image58.png}
    \end{figure}
    \begin{solution}
        构造坐标系,设$A = (0,0,0), M=(1,2,0), N=(2,1,0)$,折起后$B = (a,a,b)$,由已知得
        \[
        AB^2=2^2 = 2a^2 + b^2 = 4,\quad BM^2 = (a-1)^2 + (a-2)^2 + b^2 = 1
        \]
        解得
        \[
        a = \frac{4}{3}, \quad b = \frac{2}{3}
        \]
        由
        \[
        \overrightarrow{AM} = (1,2,0), \quad \overrightarrow{AB} = \left(\frac{4}{3}, \frac{4}{3}, \frac{2}{3}\right),
        \]
        平面 $AMN,ABM$ 的法向量为
        \[
        \vec{n}_1 = (0,0,1),\quad \vec{n}_2 = \overrightarrow{AB} \times \overrightarrow{AM} = \frac{2}{3}(-2,1,2),
        \]
        两平面夹角 $\theta$ 满足
        \[
        \cos \theta = \frac{\vec{n}_1 \cdot \vec{n}_2}{|\vec{n}_1||\vec{n}_2|} = \frac{2}{3} \Rightarrow \sin \theta = \sqrt{1 - \left(\frac{2}{3}\right)^2} = \frac{\sqrt{5}}{3}
        \]
    \end{solution}

    \question 有一边长为 2 的正四面体 $ABCD$,设 $A'$ 为 $A$ 关于平面 $BCD$ 的对称点,$\ B'$ 为 $B$ 关于平面 $ACD$ 的对称点,求四面体 $A'CB'D$ 的体积。
    \begin{solution}
        设 $A(0,0,0),B(\sqrt{2}, 0, \sqrt{2}),C(0, \sqrt{2}, \sqrt{2}),D(\sqrt{2}, \sqrt{2}, 0)$,则
        \[
        \overrightarrow{AC}=(0, \sqrt{2}, \sqrt{2}),\quad \overrightarrow{BC}=(−\sqrt{2},\sqrt{2},0),\quad \overrightarrow{CD}=(\sqrt{2},0,−\sqrt{2})
        \]
        \[
        \textbf{n}_{BCD}=\overrightarrow{BC}\times\overrightarrow{CD}=(−2,−2,−2),\quad \textbf{n}_{ACD}=\overrightarrow{AC}\times\overrightarrow{CD}=(−2,2,−2)
        \]
        故$\triangle BCD,\triangle ACD$方程式为
        \[
        E_1 :  x + y + z = 2\sqrt{2},\quad E_2 : -x + y - z = 0
        \]
        且对称点$A',B'$坐标为
        \[
        A' = \left( \dfrac{4\sqrt{2}}{3} , \dfrac{4\sqrt{2}}{3} , \dfrac{4\sqrt{2}}{3}  \right),\quad
        B' = \left( -\dfrac{\sqrt{2}}{3}, \dfrac{4\sqrt{2}}{3} , -\dfrac{\sqrt{2}}{3} \right)
        \]
        现求三角形 $\triangle A'CD$ 面积:
        \[
        \vec{A'C} = \left( -\dfrac{4\sqrt{2}}{3} , -\dfrac{\sqrt{2}}{3}, -\dfrac{\sqrt{2}}{3} \right) ,\quad
        \vec{A'D} = \left( -\dfrac{\sqrt{2}}{3} , -\dfrac{\sqrt{2}}{3}, -\dfrac{4\sqrt{2}}{3}  \right) 
        \]
        \[
        [\triangle A'CD]=\dfrac{1}{2} \left\lVert \vec{A'C} \times \vec{A'D} \right\rVert = \dfrac{1}{2} \sqrt{ \left( \dfrac{2}{3} \right)^2 + \left( -\dfrac{10}{3} \right)^2 + \left( \dfrac{2}{3} \right)^2 } = \sqrt{3}
        \]
        且$\triangle A'CD$ 方程式为
        \[
        E_3 : x - 5y + z + 4\sqrt{2} = 0
        \]
        于是
        \[
        d(B', E_3) = \dfrac{ \left| -\dfrac{\sqrt{2}}{3}  - 5 \cdot \dfrac{4\sqrt{2}}{3}  - \dfrac{\sqrt{2}}{3}  + 4\sqrt{2} \right| }{ \sqrt{1^2 + (-5)^2 + 1^2} }
        = \dfrac{10\sqrt{2}}{9\sqrt{3}}
        \]
        四面体 $A'CB'D$体积为
        \[
        [A'CB'D]=\dfrac{1}{3} \cdot \sqrt{3} \cdot \dfrac{10\sqrt{2}}{9\sqrt{3}} =  \dfrac{10\sqrt{2}}{27}
        \]
    \end{solution}

    \question 空间中有一四棱锥 $E-ABCD$,其中底面 $ABCD$ 为矩形,$\triangle AED$ 为正三角形,且平面 $EAD$ 与平面 $ABCD$ 垂直。设 $G, F$ 分别为 $AD, CD$ 的中点,且 $\angle EBG = 30^\circ$。若点 $A$ 到平面 $EFB$ 的距离为 $2$,求 $AD$。
    \ifprintanswers
    \begin{figure}[H]
        \centering            
        \includegraphics[width=0.6\textwidth]{images/image20.png}
    \end{figure}
    \fi
    \begin{solution}
        设 $AB = CD = a ,AD = BC = 2b,$ 构造空间坐标系
        \[
        A(0,0,0),\
        B(a,0,0),\
        C(a,2b,0),\
        D(0,2b,0),\
        E(0,b,\sqrt{3}b),\
        F\left(\dfrac{a}{2}, 2b, 0\right), \
        G(0,b,0)
        \]
        则
        \[
        \overrightarrow{BE} = (-a, b, \sqrt{3}b), 
        \overrightarrow{BG} = (-a, b, 0), 
        \overrightarrow{BF} = \left(-\dfrac{a}{2}, 2b, 0\right)
        \]
        由已知$\angle EBG = 30^\circ$,
        \[
        \cos \angle EBG = \dfrac{\overrightarrow{BE} \cdot \overrightarrow{BG}}{|\overrightarrow{BE}||\overrightarrow{BG}|} = \dfrac{a^2 + b^2}{\sqrt{a^2 + 4b^2} \cdot \sqrt{a^2 + b^2}} = \sqrt{\dfrac{a^2 + b^2}{a^2 + 4b^2}} = \dfrac{\sqrt{3}}{2}
        \]
        两边平方得
        \[
        \dfrac{a^2 + b^2}{a^2 + 4b^2} = \dfrac{3}{4} \Rightarrow a = 2\sqrt{2}b
        \]
        设平面 $EFB$ 的法向量为
        \[
        \vec{n} = \overrightarrow{BE} \times \overrightarrow{BF} =
        \begin{vmatrix}
        \mathbf{i} & \mathbf{j} & \mathbf{k} \\
        -a & b & \sqrt{3}b \\
        -\dfrac{a}{2} & 2b & 0
        \end{vmatrix}
        = (-2\sqrt{3}b^2,\ -\dfrac{\sqrt{3}}{2}ab,\ -\dfrac{3}{2}ab)
        \]
        平面 $EFB$ 的方程为
        \[
        -2\sqrt{3}b(x - a) - \dfrac{\sqrt{3}}{2}a y - \dfrac{3}{2}a z = 0
        \Rightarrow 4bx + ay + \sqrt{3}a z = 4ab
        \]
        点 $A(0,0,0)$ 到该平面的距离为
        \[
        d = \dfrac{|4ab|}{\sqrt{(4b)^2 + a^2 + 3a^2}} = \dfrac{4ab}{\sqrt{16b^2 + 4a^2}} = 2
        \]
        代入 $a = 2\sqrt{2}b$ 得
        \[
        \dfrac{4b \cdot 2\sqrt{2}b}{\sqrt{16b^2 + 4 \cdot 8b^2}} = 2
        \Rightarrow b = \dfrac{\sqrt{6}}{2}
        \]
        故
        \[
        {AD} = 2b = \sqrt{6}
        \]
    \end{solution}
    
    \question 已知空中有一边长为 $5\sqrt{2}$ 的正四面体,$A$ 为此四面体中距离地面最近的顶点,其他三个顶点距离地面的距离分别为 $5$、$6$、$7$,求 $A$ 到地面的距离。
        \ifprintanswers
        \begin{figure}[H]
            \centering
            \includegraphics[width=0.6\textwidth]{images/image22.png}
        \end{figure}
        \fi
    \begin{solution}
        设$B,C,D$分别在平面$$z=5,z=6,z=7$$ 上,$P$为 ${BD}$ 的中点,$D'$为
        $D \text{ 在平面 } z=6 \text{ 的投影点 },\theta$为
        $\triangle BCD \text{ 与平面 } z=6 \text{ 的夹角 } ,$
        则
        \[
        \begin{cases}
        P \text{ 也在平面 } z=6 \text{ 上,\ 且 } {PD} = \dfrac{5\sqrt{2}}{2} \\
        {DD'} = 1
        \end{cases}
        \Rightarrow \sin \theta = \frac{{DD'}}{{PD}} = \frac{2}{5\sqrt{2}} \Rightarrow \cos \theta = \frac{\sqrt{46}}{5\sqrt{2}}
        \]
        设 $A$ 在 $\triangle BCD$ 的投影点为 $G$,则 $G$ 为 $\triangle BCD$ 的重心,$G$ 的$z$ 坐标为
        \[
        \frac{5+6+7}{3} = 6 
        \]
        即 $G$ 也在平面  $z=6$ 上;由正四面体性质
        \[
        {AG} = \frac{\sqrt{6}}{3} \cdot 5\sqrt{2} = \frac{10\sqrt{3}}{3}, \quad {AG} \perp \triangle BCD
        \]
        因此$A$ 到平面 $z=6$ 的距离为
        \[
        AG \sin\left(\frac{\pi}{2} - \theta\right) = {AG} \cos \theta= \frac{10\sqrt{3}}{3} \cdot \frac{\sqrt{46}}{5\sqrt{2}} = \frac{2}{3} \sqrt{69}
        \]
        所以 $A$ 到地面的距离为
        \[
        6 - \frac{2}{3}\sqrt{69} = \frac{18 - 2\sqrt{69}}{3}
        \]
    \end{solution}
    
    \question 一正方形纸张 $ABCD$,设点 $E, F$ 分别在 ${BC}, {DC}$ 边上,且 $BE:EC = DF:FC = 2:1$。现以正方形 $ABCD$ 为底面,分别将 $B,D$ 以 $AE, AF$ 为谷折线向上折起,使得 $AB, AD$ 重合,并令重合后的点 $B=D=G$。此时,若侧面 $\triangle AEG$(或 $\triangle AFG$)与鸢形底面 $AECF$ 的夹角为 $\theta$,求 $\sin \theta$。
    \begin{solution}
        正方形边长为 $4$,构造空间坐标系,
        $$A(0,0,0), \
        B(3,0,0), \
        C(3,3,0), \
        D(0,3,0), \
        E(3,2,0), \
        F(2,3,0), \
        G(t,t,a).$$
        由长度条件
        \[
        \begin{cases}
        AG = 3 \Rightarrow t^2 + t^2 + a^2 = 9, \\
        GE = 2 \Rightarrow (t-3)^2 + (t-2)^2 + a^2 = 4,
        \end{cases}
        \]
        解得
        \[
        t = \frac{9}{5}, \quad a = \frac{3\sqrt{7}}{5}.
        \]
        向量
        \[
        \overrightarrow{AF} = (2,3,0), \quad \overrightarrow{AG} = \left(\frac{9}{5}, \frac{9}{5}, \frac{3\sqrt{7}}{5}\right).
        \]
        法向量
        \[
        \vec{n} = \overrightarrow{AF} \times \overrightarrow{AG} \parallels (3\sqrt{7}, -2\sqrt{7}, -3)
        \]
        侧面 $AEG$ 平面方程与底面 $AECF$ 方程分别为
        \[
        3\sqrt{7} x - 2\sqrt{7} y - 3 z = 0, \quad z = 0.
        \]
        且
        \[
        \cos \theta = \frac{\vec{n} \cdot (0,0,1)}{|\vec{n}|} = \frac{-3}{\sqrt{(3\sqrt{7})^2 + (-2\sqrt{7})^2 + (-3)^2}} = -\frac{3}{10}
        \]
        故
        \[
        \sin \theta = \sqrt{1 - \cos^2 \theta} = -\frac{\sqrt{91}}{10}
        \]
    \end{solution}
    
    \question 已知平行六面体 $ABCD-EFGH$,直线方程式为:
    \[
    {AB}: \frac{x+3}{2} = \frac{y}{1} = \frac{z+1}{2}, \quad
    {EH}: \frac{x-1}{2} = \frac{y-5}{-1} = \frac{z-1}{1}, \quad
    {CG}: \frac{x-7}{2} = \frac{y+1}{2} = \frac{z-13}{-1}.
    \]
    若点 $(9,7,8)$ 不在上述三条直线上,且为此平行六面体的一个顶点,求此平行六面体的体积。
            \ifprintanswers
            \begin{figure}[H]
                \centering            
                \includegraphics[width=0.25\textwidth]{images/image28.png}
            \end{figure}
            \fi
    \begin{solution}
        设 $D=(9,7,8)$,则 $AD$ 方程式为
        \[
        AD:\frac{x-9}{2} = \frac{y-7}{-1} = \frac{z-8}{1},
        \]
        联立$AB$方程式解得无实数解,故不合题意;设 $F=(9,7,8)$,则
        \[
        BF:\frac{x-9}{2} = \frac{y-7}{2} = \frac{z-8}{-1},
        \]
        解得 $B=(7,5,9)$;由于
        \[
        \vec{v}_{AB} \times \vec{v}_{AD} = (2,1,2)\times(2,-1,1)=(3,2,-4),
        \]
        平面 $ABCD$ 方程为
        \[
        3x + 2y - 4z = -5.
        \]
        设 $F$ 在 $EH,CG$ 上的垂足分别为
        \[
        P(2m + 1, -m + 5, m + 1),\quad Q(2n + 7, 2n - 1, -n + 13),
        \]
        则
        \[
        \overrightarrow{FP} = (2m - 8, -m - 2, m - 7), \quad \overrightarrow{FQ} = (2n - 2, 2n - 8, -n + 5).
        \]
        由垂直条件$\overrightarrow{FP} \cdot \vec{v}_{EH} = 0, \ \overrightarrow{FQ} \cdot \vec{v}_{CG} = 0,$解得
        \[
        m = \frac{7}{2}, \quad n = \frac{25}{9}.
        \]
        故
        \[
        |\overrightarrow{FP}| = \frac{\sqrt{174}}{2}, \quad |\overrightarrow{FQ}| = \frac{2 \sqrt{53}}{3}.
        \]
        设 $\overrightarrow{EF}$ 与 $\overrightarrow{EH}$ 的锐角为 $\alpha,\overrightarrow{FG}$ 与 $\overrightarrow{CG}$ 的锐角为 $\beta$,则
        \[
        \cos \alpha = \frac{|\overrightarrow{EF} \cdot \overrightarrow{EH}|}{|\overrightarrow{EF}||\overrightarrow{EH}|} = \frac{5}{3 \sqrt{6}}, \quad \cos \beta = \frac{|\overrightarrow{FG} \cdot \overrightarrow{CG}|}{|\overrightarrow{FG}||\overrightarrow{CG}|} = \frac{1}{3 \sqrt{6}},
        \]
        \[
        \sin \alpha = \frac{\sqrt{29}}{3 \sqrt{6}}, \quad \sin \beta = \frac{\sqrt{53}}{3 \sqrt{6}}.
        \]
        又由
        \[
        |\overrightarrow{FP}| = EF \sin \alpha = \frac{\sqrt{174}}{2} \implies EF = 9,
        \]
        \[
        |\overrightarrow{FQ}| = FG \sin \beta = \frac{2 \sqrt{53}}{3} \implies FG = 2 \sqrt{6}.
        \]
        设平行六面体高为点 $F$ 到平面 $ABCD$ 的距离
        \[
        d = \frac{|3 \cdot 9 + 2 \cdot 7 - 4 \cdot 8 + 5|}{\sqrt{3^2 + 2^2 + (-4)^2}} = \frac{14}{\sqrt{29}}.
        \]
        故体积为
        \[
        V = EF \times FG \times \sin \alpha \times d = 9 \times 2 \sqrt{6} \times \frac{\sqrt{29}}{3 \sqrt{6}} \times \frac{14}{\sqrt{29}} = 84.
        \]
    \end{solution}

    \question 将8个半径为2的球分两层放置于一个圆柱形容器中,使得每个球和与其相邻的四个球均相切,且与圆柱的一个底面和侧面都相切,则圆柱的高为
    \begin{solution}
        \textcolor{red}{(待解)}
    \end{solution}
    
    \question 已知三棱柱 $\Omega:ABC-A_{1}B_{1}C_{1}$ 的9条棱长均相等,记底面$ABC$所在平面为$\alpha$,若$\Omega$的另外四个面(即面 $A_{1}B_{1}C_{1}$, $ABBA_{1}$, $ACCA_{1}$, $BCCB_{1}$) 在$\alpha$上投影的面积从小到大重排后依次为 $2\sqrt{3},3\sqrt{3},4\sqrt{3},5\sqrt{3}$ ,求$\Omega$的体积.
        \ifprintanswers
        \begin{figure}[H]
            \centering            
            \includegraphics[width=0.35\textwidth]{images/image18.png}
        \end{figure}
        \fi
    \begin{solution}
        设点 $A_{1}, B_{1}, C_{1}$ 在平面$\alpha$上的投影分别为$D,E,F$,则面 
        \[
        A_{1}B_{1}C_{1}, ABB_{1}A_{1}, ACC_{1}A_{1}, BCC_{1}B_{1}
        \]
        在$\alpha$上的投影面积分别为 
        \[
        [\triangle DEF],[ABED],[ACFD],[BCFE]
        \]
        由已知及三棱柱的性质, $\triangle DEF$ 为正三角形,且$ABED, ACFD, BCFE$均为平行四边形.
        由对称性,仅需考虑点$D$位于$\angle BAC$内的情形(如图所示)。
        显然此时有 
        \[
        [ABED]+[ACFD]=[BCFE]
        \]
        由于 
        \[
        \{[\triangle DEF],[ABED],[ACFD],[BCFE]\}=\{2\sqrt{3},3\sqrt{3},4\sqrt{3},5\sqrt{3}\}\] 
        故 $[ABED],[ACFD]$ 必为 $2\sqrt{3},3\sqrt{3}$ 的排列, $[BCFE]=5\sqrt{3}$ ,进而 $S_{\triangle DEF}=4\sqrt{3}$ ,得 $\triangle DEF$ 的边长为4,即正三棱柱的各棱长均为4.
        
        不妨设 $[ABED]=2\sqrt{3},[ACFD]=3\sqrt{3}$ ,则 
        \[
        [\triangle ABD]=\sqrt{3},[\triangle ACD]=\frac{3\sqrt{3}}{2}
        \]
        取射线$AD$与线段$BC$的交点$X$,则 
        \[
        \frac{BX}{CX}=\frac{[\triangle ABD]}{[\triangle ACD]}=\frac{2}{3}
        \] 
        故 $BX=\dfrac{8}{5}$ ,因此 
        \[
        AX=\sqrt{AB^{2}+BX^{2}-2AB\cdot BX\cdot \cos 60^{\circ}}=\frac{4\sqrt{19}}{5}
        \]
        而 
        \[
        \frac{AD}{AX}=\frac{[\triangle ABD]+[\triangle ACD]}{[\triangle ABC]}=\frac{5}{8}
        \] 
        故 $AD=\dfrac{\sqrt{19}}{2}$,于是$\Omega$的高 \[
        h=\sqrt{AA_{1}^{2}-AD^{2}}=\frac{3\sqrt{5}}{2}\]
        又$[\triangle ABC]=4\sqrt{3}$ ,故$\Omega$的体积
        \[
        V=[\triangle ABC]\cdot h=6\sqrt{15}
        \]
    \end{solution}

    
    \question 正方体 $ABCD-A_1B_1C_1D_1$ 的底面 $A_1B_1C_1D_1$ 内有一个动点 $M$, 且 $BM \parallel$ 平面 $AD_1C$, 则 $\tan \angle D_1MD$ 的最大值是
    \begin{solution}
设正方体 $ABCD - A_1B_1C_1D_1$ 的边长为 $a$,建立坐标系,使 $D(0,0,0)$,$A(a,0,0)$,$C(0,a,0)$,$D_1(0,0,a)$。因此 $A_1(a,0,a)$,$B_1(a,a,a)$,$C_1(0,a,a)$。

设底面动点 $M$ 的坐标为 $(x,y,a)$,其中 $0 \le x,y \le a$。点 $B$ 的坐标为 $(a,a,0)$,因此
\[
\vec{BM} = (x - a, y - a, a).
\]

题设中 $BM \parallel$ 平面 $AD_1C$,则 $\vec{BM}$ 垂直于该平面的法向量。

求平面 $AD_1C$ 的法向量,可取 $\vec{AD_1} = (-a, 0, a)$ 和 $\vec{AC} = (-a, a, 0)$,则
\[
\vec{n} = \vec{AD_1} \times \vec{AC} = (-a^2, a^2, -a^2) \propto (1, -1, 1).
\]

由 $\vec{BM} \cdot \vec{n} = 0$ 得:
\[
(x - a)(1) + (y - a)(-1) + a(1) = 0 \Rightarrow x - y + a = 0 \Rightarrow y = x + a.
\]

\textcolor{red}{结合 $0 \le y \le a$,代入得 $0 \le x + a \le a \Rightarrow -a \le x \le 0$,又因 $x \ge 0$,故 $x = 0$,从而 $y = a$。}

故点 $M$ 为 $C_1(0,a,a)$,接下来计算 $\tan \angle D_1MD$,其中 $D_1(0,0,a)$,$D(0,0,0)$。

设 $\theta = \angle D_1MD$,则有:

\[
\vec{MD_1} = (0, -a, 0), \quad \vec{MD} = (0, -a, -a),
\]
\[
\cos \theta = \frac{\vec{MD_1} \cdot \vec{MD}}{|\vec{MD_1}| \cdot |\vec{MD}|} = \frac{a^2}{a \cdot a\sqrt{2}} = \frac{1}{\sqrt{2}} \Rightarrow \theta = 45^\circ,
\]
\[
\tan \angle D_1MD = \tan 45^\circ = \boxed{1}.
\]
\textcolor{red}{(by chatgpt,待验证)}

\end{solution}
    \question 如图,在面积为2的矩形$ABCD$中,点$E$为$AD$边的中点.将$\triangle ABE$和$\triangle DEC$分别沿$BE,CE$翻折,使得$A,D$重合于点$P$,则三棱锥$P-EBC$体积的最大值为
    \begin{solution}
        \textcolor{red}{(待解)}
    \end{solution}
    
    \question 设四棱锥 $P-ABCD$,面 $PAB \perp$ 底面 $ABCD$,且 $PA=PB=\sqrt{5},AB=BC=AD=2$,求该四棱锥体积的最大值。
        
    \begin{solution}
        由题意,四棱锥的高为 $2$,只需求底面四边形 $ABCD$ 面积的最大值。
        
        将四边形 $ABCD$ 分为 $\triangle ABD$ 和 $\triangle BCD$,设 $\angle BAD = \alpha$,由余弦定理得
        \[
        BD^2 = AB^2 + AD^2 - 2 \cdot AB \cdot AD \cdot \cos\alpha = 4 + 4 - 8 \cos\alpha = 8(1 - \cos\alpha)
        \]
        由三角恒等式 $1 - \cos\alpha = 2\sin^2 \dfrac{\alpha}{2}$,得
        \[
        BD = \sqrt{8(1 - \cos\alpha)} = \sqrt{16 \sin^2 \frac{\alpha}{2}} = 4 \sin \frac{\alpha}{2}
        \]
        三角形 $ABD$ 的面积为:
        \[
        [\triangle ABD] = \frac{1}{2} \cdot AB \cdot AD \cdot \sin\alpha = 2 \sin\alpha
        \]
        当 $BC \perp BD$ 时,$\;\triangle BCD$ 面积最大(贪婪算法),
        \[
        [\triangle BCD] = \frac{1}{2} \cdot BC \cdot BD = 4 \sin \frac{\alpha}{2}.
        \]
        因此四边形面积为
        \[
        S = [ABCD] = 2 \sin \alpha + 4 \sin \frac{\alpha}{2}.
        \]
        令 $x = \sin \dfrac{\alpha}{2} \in (0,1)$,则
        \[
        S(x) = 4 \left( x + x \sqrt{1 - x^{2}} \right).
        \]
        求导:
        \[
        S'(x) = 4 \left( 1 + \sqrt{1 - x^{2}} - \frac{x^{2}}{\sqrt{1 - x^{2}}} \right).
        \]
        解 $S'(x) = 0$,得极值点$x = \frac{\sqrt{3}}{2}$,
        代入得最大面积
        \[
        S_{\max} = 3 \sqrt{3}.
        \]
        故四棱锥体积最大为
        \[
        V_{\max} = \frac{1}{3} \times 3\sqrt{3} \times 2 = 2 \sqrt{3}.
        \]
    \end{solution}

    \question 一个盛满水的半球体容器, 其半径为 $6$, 若倾斜 $45^{\circ}$ 后,求容器溢出的水体积。
    \ifprintanswers
    \begin{figure}[H]
        \centering        
        \includegraphics[width=0.4\textwidth]{images/image101.jpg}
    \end{figure}
    \fi
    \begin{solution}
        剩下的水体积即为上图着色区域绕 $x$ 轴旋转的体积,即
        \[
        \pi \int_{3\sqrt 2}^{6} (36 - x^2)\; dx = \pi \left[36x - {1\over 3} x^3 \right]_{3\sqrt 2}^{6} = (144 - 90\sqrt 2)\pi
        \]
        因此溢出的水体积为
        \[
        \frac{2}{3}\pi \cdot 6^3 - (144 - 90\sqrt 2)\pi = 90\sqrt 2 \pi
        \]
    \end{solution}

    \question 已知一个底面半径为 $3$,高为 $3$ 的直圆柱,平面 $E$ 通过底面的直径 $AB$,且平面 $E$ 与底面的夹角为 $45^\circ$,此时平面 $E$ 将直圆柱切割成两部分,求较小部分的体积。
    \begin{solution}
        设圆柱底面中心为原点,直径 $AB$ 在 $x$ 轴上,平面 $E$ 与 $AB$成 $45^\circ$,故平面方程可设为 \(z = x\),底面为半径 $3$ 的半圆区域 
        \[
        D: x^{2}+y^{2} \le 9,\ x \ge 0
        \]
        体积为
        \[
        V = \iint_D z \, dA = \iint_D x \, dA
        \]
        用极坐标积分,令 \(x = r\cos\theta,y = r\sin\theta\),则
        \[
        V = \int_{-\frac{\pi}{2}}^{\frac{\pi}{2}} \int_0^3 r\cos\theta \cdot r\, dr\, d\theta = \int_{-\frac{\pi}{2}}^{\frac{\pi}{2}} \cos\theta \int_0^3 r^{2} dr\, d\theta
        = 9 \int_{-\frac{\pi}{2}}^{\frac{\pi}{2}} \cos\theta \, d\theta
        =9[\sin\theta]_{-\frac{\pi}{2}}^{\frac{\pi}{2}} = 18
        \]
    \end{solution}
    \begin{solution}
        设底面圆方程式为
        \[
        x^2 + y^2 = R^2,
        \]
        所求体积由许多直角三角形累积而成;该三角形与底面圆的交点坐标为 $(x,y)$,则三角形的底长为 $y$,高为 $y \tan \theta$,因此三角形面积为
        \[
        \frac{1}{2} y^2 \tan \theta = \frac{1}{2} (R^2 - x^2) \tan \theta.
        \]
        故所求体积为
        \[
        \left. \int_{-R}^R \frac{1}{2} (R^2 - x^2) \tan \theta \, dx \right|_{R=3,\theta=45^\circ}= \left.\frac{2}{3} R^3 \tan \theta \right|_{R=3,\theta=45^\circ}=18
        \]
    \end{solution}

    \question 奥里克有一个杯子,是一个底面为圆的直圆锥体,底面半径为 \(\frac{1}{2}\),斜高为 \(1\),杯中装满了奶茶。奶茶恰好填满杯子斜高的一半。当奥里克将杯子倾斜到刚好要溢出的程度时,如图所示,新的斜高 \(EA\) 与倾斜后的奶茶表面所形成的椭圆的长轴 \(ET\) 构成夹角 \(\angle TEA\)。求 \(\cos{\angle TEA}\)。     %https://artofproblemsolving.com/community/c3073284h3152961p28644263
        \begin{figure}[H]
            \centering
            \includegraphics[width=0.2\textwidth]{images/image6.png}
        \end{figure}
    \begin{solution}
        \textcolor{red}{(待解)}
    \end{solution}

    \question 在正三棱锥 $P-ABC$ 中, $AB=1$, $AP=2$, 过 $AB$ 的平面 $\alpha$ 将其体积平分, 则棱 $PC$ 与平面 $\alpha$ 所成角的余弦值为
    \begin{solution}
        设 $\triangle ABC$ 的外接圆半径为 $R$,且$P$ 到底面 $ABC$ 的距离为 $d$ ,则由正弦定理知, 
        \[
        2R=\frac{1}{\sin 120^\circ}=\frac{2\sqrt{3}}{3},\;d=\sqrt{2^2-\left(\dfrac{\sqrt{3}}{3}\right)^2}=\dfrac{\sqrt{33}}{3}
        \]
        故
        \[
        [P-ABC]=\frac{1}{3}\cdot\frac{\sqrt{3}}{4}\cdot 1^2\cdot\frac{\sqrt{33}}{3}=\frac{\sqrt{11}}{12}
        \]
        设 $PC$ 的中点为 $D$, 由
        \[
        A=(0,0,0),B=(1,0,0),C=\left(\frac12,\frac{\sqrt3}{2},0\right),P\left(\frac12,\frac{\sqrt3}{6},\frac{\sqrt{33}}{6}\right),
        \]
        则 
        \[
        AD^2=\left(\frac12\right)^2+\left(\frac{\sqrt3}{3}\right)^2+\left(\frac{\sqrt{33}}{6}\right)^2 \Rightarrow AD=BD=\frac{\sqrt{6}}{2}\]
        故 
        \[
        [\triangle ABD]=\frac{1}{2}\cdot 1\cdot\sqrt{\left(\frac{\sqrt{6}}{2}\right)^2-\left(\frac{1}{2}\right)^2}=\frac{\sqrt{5}}{4}
        \]
        设 $P$ 到平面 $\alpha$ 的距离为 $h$, 则由等体积法知 
        \[
        \frac{1}{3}\times\frac{\sqrt{5}}{4}\times h=\frac{\sqrt{11}}{24}\Rightarrow h=\frac{\sqrt{55}}{10}
        \]       
        所以棱 $PC$ 与平面 $\alpha$ 所成角的余弦值为 
        \[
        1-\left(\frac{\frac{\sqrt{55}}{10}}{1}\right)^2=\frac{3\sqrt{5}}{10}
        \]
    \end{solution}
    
    \question 一个底面半径为 $5$,高为 $12$ 的直圆锥中放入了三个互相接触的全等球体,每个球的半径为 $r$。每个球都与另外两个球相切,并且也与圆锥的底面及侧面相切。求 $r$ 的值。
     \ifprintanswers
    \begin{figure}[H]
        \centering
        \includegraphics[width=0.3\textwidth]{images/image11.png}
        \quad
        \includegraphics[width=0.15\textwidth]{images/image12.png}
    \end{figure}
    \fi
    \begin{solution}
        设圆锥的顶点为 $A$,底面圆心为 $B$。取其中一个球的球心为 $E$,三球心形成正三角形,中心为 $C$,满足 $AC = AE + EC$。我们来分别表示 $AE$ 与 $EC$。
        
        从顶部鸟瞰圆锥,三个球心构成一个边长为 $2r$ 的正三角形。点 $C$ 是其重心,正三角形边长为 $r\sqrt{3}$,故
        \[
        EC = \frac{2r\sqrt{3}}{3}
        \]
        将 ${OE}$ 延长至与底面交于点 $F$使得$\triangle AEF \backsim \triangle ACB$,由角平分线定理,
        \[
        \frac{OE}{OF} = \frac{AE}{AF} = \frac{5}{13} \Rightarrow OF = \frac{13r}{5}
        \]
        因此
        \[
        EF = OE + OF = r + \frac{13r}{5} = \frac{18r}{5}
        \]
        再由相似三角形
        \[
        AE = \frac{5}{12} EF  = \frac{3r}{2}
        \]
        我们已知 $AC = AE + EC = 5$,代入得
        \[
        \frac{3r}{2} + \frac{2r\sqrt{3}}{3} = 5
        \Rightarrow  r = \frac{30}{9 + 4\sqrt{3}} = \frac{90 - 40\sqrt{3}}{11}
        \]
    \end{solution}

    \question 求空间中一点 $P(-5,0,-8)$ 到直线 
    \[
    L:\frac{x-3}{1}=\frac{y-2}{-2}=\frac{z+1}{2}
    \] 
    的距离。
    \begin{solution}
        设 $Q(3+t,2-2t,-1+2t)$ 满足 $\overrightarrow{PQ} \perp L$,则
        \[
        \overrightarrow{PQ}=(8+t,2-2t,7+2t),\quad \overrightarrow{PQ} \cdot (1,-2,2)=0
        \]
        解得
        \[
        t=-2 \Rightarrow Q(1,6,-5)
        \]
        于是
        \[
        d(P,L)=\overline{PQ}=\sqrt{(-5-1)^{2}+(0-6)^{2}+(-8+5)^{2}}=\sqrt{81}=9
        \]
    \end{solution}
    \begin{solution}
        设直线 $L$ 上动点 $Q(3+t,2-2t,-1+2t)$,则
        \[
        PQ=\sqrt{(8+t)^{2}+(2-2t)^{2}+(7+2t)^{2}}=\sqrt{9(t+2)^{2}+81},
        \]  
        当 $t=-2$ 时有最小值 $9$。
    \end{solution}
    \begin{solution}
        直线 $L$ 上取一点 $A(4,0,1)$ ,有$\overrightarrow{AP}=(-9,0,-9)$,直线的方向向量 $\vec{v}=(1,-2,2)$,则
        \[
        \overrightarrow{AP}\times\vec{v}=(-18,9,18)
        \] 
        平行四边形面积为 
        \[
        \sqrt{(-18)^{2}+9^{2}+18^{2}}=27,
        \] 
        又 $|\vec{v}|=3$,故    
        \[
        d(P,L)=\frac{27}{|\vec{v}|}=9
        \]
    \end{solution}
    \begin{solution}
        直线 $L$ 上取一点 $A(4,0,1)$,$\overrightarrow{AP}=(-9,0,-9)$ 在 $\vec{v}=(1,-2,2)$ 上的正射影 
        \[
        \overrightarrow{AH}=(-3,6,-6),
        \]  
        $P$ 在直线 $L$ 上的投影点 $H$坐标为
        \[
        (4,0,1)+(-3,6,-6)=(1,6,-5)
        \]  
        故
        \[
        PH=d(P,L)=\sqrt{(-5-1)^{2}+(0-6)^{2}+(-8+5)^{2}}=9
        \]
    \end{solution}

    \question 设 $A(0,1,2),B(-1,2,1),C(1,0,1)$ 为空间中的三点,求 $\triangle ABC$ 的垂心坐标。
    \begin{solution}
        首先有
        \[
        \vec n = \overrightarrow{AB} \times \overrightarrow{AC} =(-1,1,-1) \times (1,-1,-1)= (-2,-2,0)
        \]
        得平面
        \[
        E:\ x+y=1
        \]
        假设垂心 $H(a,b,c)$,由 $\overrightarrow{AH} \perp \overrightarrow{BC},\overrightarrow{BH} \perp \overrightarrow{AC},\overrightarrow{CH} \perp \overrightarrow{AB}$,
        \[
        (a,b-1,c-2) \cdot (2,-2,0) = (a+1,b-2,c-1) \cdot (1,-1,-1) = (a-1,b,c-1) \cdot (-1,1,-1) = 0
        \]
        解得 
        \[
        H(a,a+1,3)
        \]
        又 $H$ 在平面 $x+y=1$ 上,故
        \[
        a + (a+1) = 1 \Rightarrow a = 0
        \]
        因此 $H = (0,1,3)$。
    \end{solution}

    \question 坐标空间中有三个彼此互相垂直的向量 $\textbf{u}, \textbf{v}, \textbf{w}$。已知 $\textbf{u} - \textbf{v} = (2, -1, 0)$,且 $\textbf{v} - \textbf{w} = (-1, 2, 3)$。求由 $\textbf{u}, \textbf{v}, \textbf{w}$ 所张出的平行六面体之体积。
    \begin{solution}
        由已知$\textbf{u}, \textbf{v}, \textbf{w}$互相垂直且$\textbf{u} - \textbf{v} = (2, -1, 0),\textbf{v} - \textbf{w} = (-1, 2, 3)$,
        \[
        (\textbf u - \textbf v) \cdot (\textbf v - \textbf w)
        = -|\textbf v|^2= -4 \Rightarrow |\textbf v| = 2
        \]
        且
        \[
        (\textbf u - \textbf v) \cdot (\textbf u - \textbf v)
        = |\textbf u|^2 + 4 = 5
        \Rightarrow |\textbf u| = 1
        \]
        及
        \[
        (\textbf v - \textbf w) \cdot (\textbf v - \textbf w)
        = 4 + |\textbf w|^2
        = 14
        \Rightarrow |\textbf w| = \sqrt{10}
        \]
        故体积为
        \[
        |\textbf u|\cdot|\textbf v|\cdot|\textbf w| = 2\sqrt{10}
        \]
    \end{solution}
    
    \question 在空间中,有三个不共平面的非零向量 $\textbf{a}, \textbf{b}, \textbf{c}$,满足
    \[
    (\textbf{a} \times \textbf{b}) \cdot ((\textbf{b} \times \textbf{c}) \times (\textbf{c} \times \textbf{a}))=7,
    \]
    求以三向量 $(3\textbf{a}+\textbf{b}+\textbf{c}), (\textbf{a}-\textbf{b}+2\textbf{c}), (\textbf{b}+\textbf{c})$ 所张成的平行六面体体积。
    \begin{solution}
        由向量性质,
        \[
        \textbf{a}\times (\textbf{b}\times \textbf{c})=(\textbf{a}\cdot \textbf{c})\textbf{b}- (\textbf{a}\cdot \textbf{b})\textbf{c}, 
        \quad (\textbf{a}\times \textbf{b})\cdot \textbf{c}=\textbf{a} \cdot (\textbf{b}\times \textbf{c}),
        \quad (\textbf{a}\times \textbf{b})\cdot \textbf{a} = (\textbf{a}\times \textbf{b})\cdot \textbf{b} =0
        \]
        因此
        \begin{align*}
        (\textbf{a}\times \textbf{b})\cdot ((\textbf{b}\times \textbf{c})\times (\textbf{c}\times \textbf{a}))
        &= (\textbf{a}\times \textbf{b})\cdot \big(((\textbf{b}\times \textbf{c})\cdot \textbf{a})\textbf{c} - ((\textbf{b}\times \textbf{c})\cdot \textbf{c})\textbf{a}\big) \\
        &= (\textbf{a}\times \textbf{b})\cdot ((\textbf{b}\times \textbf{c})\cdot \textbf{a})\textbf{c} \\
        &= ((\textbf{b}\times \textbf{c})\cdot \textbf{a})\,((\textbf{a}\times \textbf{b})\cdot \textbf{c}) \\
        &= (\textbf{a}\cdot (\textbf{b}\times \textbf{c}))^2 = 7
        \end{align*}
        即
        \[
        \textbf{a} \cdot (\textbf{b}\times \textbf{c}) = \pm \sqrt{7}
        \]
        三向量张成的平行六面体体积为
        \begin{align*}
        V &= \big|(3\textbf{a}+\textbf{b}+\textbf{c})\cdot  \big((\textbf{a}-\textbf{b}+2\textbf{c})\times (\textbf{b}+\textbf{c})\big)\big| \\
        &= \text{abs}\left(\begin{vmatrix} 3 & 1 & 1 \\ 1 & -1 & 2 \\ 0 & 1 & 1 \end{vmatrix}\right) \cdot |\textbf{a} \cdot (\textbf{b}\times \textbf{c})| = 9\sqrt{7}
        \end{align*}
    \end{solution}

    \question 两平面
    \[
    E_1:\;x + k y + z = 3, \quad E_2:\;x + y + k z = 5
    \]
    夹角为 $60^\circ$,求 $k$。  
    \begin{solution}
        设 $E_1, E_2$ 的法向量分别为  
        $\vec{n_1} = (1, k, 1)$,$\vec{n_2} = (1, 1, k)$。
        两平面夹角满足:
        \[
        \cos\theta = \frac{|\vec{n_1} \cdot \vec{n_2}|}{|\vec{n_1}||\vec{n_2}|}
        \quad \text{其中 } \theta = 60^\circ,\; \cos\theta = \frac{1}{2}
        \]
        计算内积与模长:
        \[
        \vec{n_1} \cdot \vec{n_2} = 1 + k + k = 1 + 2k
        \]
        \[
        |\vec{n_1}| = \sqrt{1 + k^2 + 1} = \sqrt{k^2 + 2},\quad
        |\vec{n_2}| = \sqrt{1 + 1 + k^2} = \sqrt{k^2 + 2}
        \]
        代入公式:
        \[
        \frac{|1 + 2k|}{k^2 + 2} = \frac{1}{2}
        \Rightarrow |1 + 2k| = \frac{1}{2}(k^2 + 2)
        \]
        分正负两种情况:
        \begin{itemize}
          \item 若 $1 + 2k \ge 0$,则 $1 + 2k = \frac{1}{2}(k^2 + 2)\Rightarrow k=0,4$;
          \item 若 $1 + 2k < 0$,则 $-(1 + 2k) = \frac{1}{2}(k^2 + 2) \Rightarrow k=-2$。
        \end{itemize}
        解得$k = -2,\; 0,\; 4$
    \end{solution}

    \question 两平面
    \[
    E_1:\;2x+y-z-3=0,\quad E_2:\;x+2y+z=0
    \]
    过点 $(2,1,-1)$ 且同时垂直于 $E_1,E_2$ 的平面方程为  
    \begin{solution}
        设所求平面法向量为 $\vec{n} = \vec{n_1} \times \vec{n_2}$,其中 $\vec{n_1} = (2,1,-1),\vec{n_2} = (1,2,1)$,则
        \[
        \vec{n} = 
        \begin{vmatrix}
        \mathbf{i} & \mathbf{j} & \mathbf{k} \\
        2 & 1 & -1 \\
        1 & 2 & 1
        \end{vmatrix}
        = \mathbf{i}(1 \cdot 1 - (-1)\cdot 2) 
        - \mathbf{j}(2 \cdot 1 - (-1)\cdot 1) 
        + \mathbf{k}(2 \cdot 2 - 1 \cdot 1) 
        = (3, -3, 3)
        \]
        所以可取法向量为 $(1, -1, 1)$,故平面方程为 $x - y + z = d$,将点 $(2,1,-1)$ 代入得
        \[
        2 - 1 - 1 = 0 \Rightarrow d = 0
        \]
        故所求平面方程为 $x - y + z = 0$。
    \end{solution}

    \question 动点 $P(x,y,z)$ 在平面
    \[
    2x+y-2z-5=0
    \]
    上移动,求
    \[
    \sqrt{(x-3)^2+(y+1)^2+(z+2)^2}
    \]
    的最小值。  
    \begin{solution}
        设定点 $A(3, -1, -2)$,原式即动点 $P(x,y,z)$ 到 $A$ 的距离,最小距离即点 $A$ 到平面的距离。
        
        平面法向量为 $\vec{n} = (2, 1, -2)$,于是
        \[
        d = \frac{|2\cdot3 + (-1) + (-2)\cdot(-2) - 5|}{\sqrt{2^2 + 1^2 + (-2)^2}} = \frac{4}{3}
        \]
    \end{solution}

    \question 已知 $\alpha,\beta,\gamma,\delta\in\mathbb{R}$,且
    \[
    \alpha^2+\beta^2+\gamma^2-2\alpha-4\beta-6\gamma+13=0,
    \]
    求
    \[
    (\alpha-\delta-8)^2+(\beta-2\delta-9)^2+(\gamma-3\delta-10)^2
    \]
    的最小值。
    \begin{solution}
        写成
        \[
        (\alpha-1)^2+(\beta-2)^2+(\gamma-3)^2=1,
        \]
        即点 $P(\alpha,\beta,\gamma)$ 在以 $C(1,2,3)$ 为球心、半径 $r=1$ 的球面上。记点 $Q$ 为参数 $t$ 下的直线
        \[
        L:\ \frac{x-8}{1}=\frac{y-9}{2}=\frac{z-10}{3}=t,
        \]
        则 $Q=(t+8,\;2t+9,\;3t+10)$,且
        \[
        (\alpha-(\delta+8))^2 +(\beta-(2\delta+9))^2 +(\gamma-(3\delta+10))^2 = PQ^2,
        \]
        其中
        \[
        PQ = d(C,L) - r = \sqrt{(t+7)^2+ (2t+7)^2+ (3t+7)^2} -1 =\sqrt{14(t+3)^2+21}-1
        \]
        在$t=-3$时取最小值$\sqrt{21}-1$,此时$PQ^2$为
        \[
        (\sqrt{21}-1)^2=22-2\sqrt{21}
        \]
    \end{solution}

    \question 在空间中,给定两歪斜线  $$L_1:\frac{x-7}{2}=\frac{y-2}{1}=\frac{z-10}{-2},L_2:\frac{x-3}{1}=\frac{y-9}{-2}=\frac{z-2}{1},$$若在直线 $L_1$ 上取一点 $P$,在直线 $L_2$ 上取一点 $Q$使得线段 ${PQ}$ 最短,试求 ${PQ}$ 的距离。
    \begin{solution}
        $L_1,L_2$ 的方向向量分别为
        \[
        \textbf{u} = (2,1,-2) ,\textbf{v} = (1,-2,1)
        \]
        则
        \[
        \textbf{n} = \textbf{u} \times \textbf{v} = (-3,-4,-5)
        \]
        包含$L_1$ 且法向量为$\textbf{n}$的平面方程式为
        \[
        E:-3(x-7) - 4(y-2) - 5(z-10) = 0 \Rightarrow 3x + 4y + 5z = 79
        \]
        取 $P(3,9,2) \in L_2$,则点到平面 $E$ 的距离为
        \[
        d(P, E) = \frac{|3(3) + 4(9) + 5(2) - 79|}{\sqrt{3^2 + 4^2 + 5^2}} = \frac{12\sqrt{2}}{5}
        \]
    \end{solution}

    \question 已知平面 $E$ 包含直线 $\begin{cases} x=1\\ y+z=3\\\end{cases}$,且两条直线
    \[
    L_1:\ \frac{x-9}{2}=\frac{y+10}{1}=\frac{z-11}{-1},\quad
    L_2:\ \frac{x-9}{2}=\frac{y+10}{2}=\frac{z-11}{-1}.
    \]
    求平面 $E$ 的方程式。
    \begin{solution}
        可得两直线$L_1,L_2$的方向向量为
        \[
        \mathbf{u}=(2,1,-1),\quad \mathbf{v}=(2,2,-1).
        \]
        由已知直线 $\begin{cases}x=1\\ y+z=3 \\\end{cases}$ 可取方向向量
        \[
        \mathbf{w}=(0,1,-1)
        \]
        计算得
        \[
        (\vec u\times\vec v)\times\vec w
        =(2, -1, -1 )
        \]
        又$E$通过$(1,1,2)$,由点法式平面方程得
        \[
        E:2(x-1)+1(y-1)+1(z-2)=0 \Rightarrow 2x-y-z=-1
        \]
    \end{solution}

    \question 空间中两歪斜线 $$L_1: \frac{x-3}{1} = \frac{y}{2} = \frac{z+2}{-2},\quad L_2: \frac{x}{3} = \frac{y-2}{1} = \frac{z+1}{-2},$$若正 $\triangle PQR$ 中,$P$ 在 $L_1$ 上,且 $Q$、$R$ 都在 $L_2$ 上,求 $\triangle PQR$ 的最小面积。
    \begin{solution}
        $L_1,L_2$ 的方向向量分别为
        \[
        \textbf{u} = (1,2,-2) ,\textbf{v} = (3,1,-2) 
        \]
        则\[
        \textbf{n} = \textbf{u} \times \textbf{v} = (-2, -4, -5)
        \]
        $\textbf{n}$ 为 $L_1$ 与 $L_2$ 所张平面的法向量,设包含 $L_2$ 的平面 $E$,带入 $L_2$ 上点 $(0,2,-1)$ 得:
        \[
        E: -2x -4(y-2) -5(z+1) = 0 \Rightarrow 2x + 4y + 5z = 3
        \]
        取 $P = (3,0,-2) \in L_1$,则 $P$ 到平面 $E$ 的距离为正三角形的高:
        \[
        d(P,E) = \frac{|2(3) + 4(0) + 5(-2) - 3|}{\sqrt{2^2 + 4^2 + 5^2}} = \frac{7}{3\sqrt{5}}
        \]
        设边长为 $s$,正三角形面积公式为 $A = \frac{\sqrt{3}}{4}s^2$,其中高 $h = \frac{\sqrt{3}}{2}s$,反解出 $s$ 得
        \[
        h = \frac{7}{3\sqrt{5}} = \frac{\sqrt{3}}{2}s \Rightarrow s = \frac{14}{3\sqrt{15}}
        \]
        最小面积为\[
        A = \frac{\sqrt{3}}{4} \cdot \left( \frac{14}{3\sqrt{15}} \right)^2 = \frac{49\sqrt{3}}{135}
        \]
    \end{solution}

    \question 空间中,已知点 $A(3,-10,11)$,直线
    \[
    L_1: 
    \begin{cases}
    x = 2 + t \\
    y = -1 + 2t, \quad t \in \mathbb{R} \\
    z = 3 - 2t
    \end{cases}
    \]
    与
    \[
    L_2: 
    \begin{cases}
    x = -1 + 2s \\
    y = 2 - 2s, \quad s \in \mathbb{R} \\
    z = -s
    \end{cases}
    \]
    今由 $A$ 点出发,经 $L_1$ 上一点再到达 $L_2$ 上一点,求此路径的最小值。
    \ifprintanswers
    \begin{figure}[H]
        \centering
        \includegraphics[width=0.3\linewidth]{images/image49.png}
    \end{figure}
    \fi
    \begin{solution}
        首先发现
        \begin{itemize}
            \item $L_1$ 与 $L_2$ 为歪斜线,且方向向量垂直。  
            \item $L_1$ 与 $L_2$ 的公垂线在 $L_1$ 上的垂足为 $P(3,1,1)$,在 $L_2$ 上的垂足为 $Q(1,0,-1)$,且$PQ = 3$。
            \item 点 $A$ 到 $L_1$ 的距离 $d(A,L_1) = 5$,且 $A$ 在 $L_1$ 上的投影点 $R$ 距离 $P$ 为 $PR = 14$。 
        \end{itemize}
        设 $L_1' \parallels L_1$,且 $\overleftrightarrow{A'R} \parallels \overleftrightarrow{PQ},A'R = AR = 5$。题目所求等同于「点 $A'$ 经 $L_1$ 上一点再到达 $L_2$ 上一点的最短路径」,亦即  
        \[
        A'Q = \sqrt{(3+5)^2 + 14^2} = 2\sqrt{65}
        \]
    \end{solution}
    
    \question 空间中有一直圆锥,已知其高 $AH$ 所在的直线方程为 
    \[
    \frac{x-4}{-1} = \frac{y+4}{2} = \frac{z+14}{6},
    \] 
    以及直圆锥侧面上两点 $P(3,0,5),Q(11,-9,-18)$,求直圆锥顶点 $A$ 坐标。
    \begin{figure}[H]
        \centering        
        \includegraphics[width=0.25\textwidth]{images/image84.png}
    \end{figure}
    \begin{solution}
        设 $A$ 在直线上,令 
        \[
        A(-t+4, 2t-4, 6t-14), \quad t\in \mathbb{R}
        \]
        则
        \[
        \overrightarrow{AP} = (t-1, -2t+4, -6t+19), \quad \overrightarrow{AQ} = (t+7, -2t-5, -6t-4),
        \] 
        直线$AH$方向向量 
        \[
        \vec{h} = (-1,2,6)
        \]
        由于 $\angle PAH = \angle QAH$,有
        \[
        \frac{\overrightarrow{AP}\cdot \vec{h}}{|\overrightarrow{AP}||\vec{h}|} = \frac{\overrightarrow{AQ}\cdot \vec{h}}{|\overrightarrow{AQ}||\vec{h}|}.
        \]
        即
        \[
        \frac{(-41t+123)^2}{(t-1)^2 + (2t-4)^2 + (6t-19)^2} = \frac{(-41t-41)^2}{(t+7)^2 + (2t+5)^2 + (6t+4)^2}
        \]
        解得
        \[
        t=6 \ \text{或} \ t=\frac{9}{5}
        \]
        其中当$t=\dfrac{9}{5}$,
        \[
        \vec p\cdot \vec h> 0 ,\quad \vec q\cdot \vec h< 0
        \]
        不合题意,故取 $t=6$,则
        \[
        A = (-6+4, 12-4, 36-14) = (-2,8,22)
        \]
    \end{solution}

    \question 有四个平行平面:
    \[
    E_1: 3x + 4y + 5z = 0, \; E_2: 3x + 4y + 5z = 1, \; E_3: 3x + 4y + 5z = 2, \; E_4: 3x + 4y + 5z = 3,
    \]
    若一个正四面体的四顶点$A, B, C, D$分别在$E_1, E_2, E_3, E_4$,求此正四面体与$E_2$相交的截面积。
    \ifprintanswers
    \begin{figure}[H]
        \centering
        \includegraphics[width=0.5\linewidth]{images/image64.png}
    \end{figure}
    \fi
    \begin{solution}
        取
        \[
        A = (1, 0, 1), \quad B = (0, 1, 1), \quad C = (1, 1, 0), \quad O = (0, 0, 0),
        \]
        则四面体$OABC$为正四面体,边长为$\sqrt{2}$,再取
        \[
        D = AO \ \text{中点} = \left(\frac{1}{2}, 0, \frac{1}{2}\right), \quad
        E = \frac{2A + B}{3} = \left(\frac{2}{3}, \frac{1}{3}, 1\right),
        \]
        \[
        F = \frac{2B + A}{3} = \left(\frac{1}{3}, \frac{2}{3}, 1\right), \quad
        G = BC \ \text{中点} = \left(\frac{1}{2}, 1, \frac{1}{2}\right).
        \]
        则$\triangle CDE \parallels \triangle OFG$,且$E_2 = \triangle CDE, E_3 = \triangle OFG$,且
        \[
        \overrightarrow{CD} = \left(-\frac{1}{2}, -1, \frac{1}{2}\right), \quad \overrightarrow{CE} = \left(-\frac{1}{3}, -\frac{2}{3}, 1\right).
        \]
        $\triangle CDE$的法向量为
        \[
        \vec{n} = \overrightarrow{CD} \times \overrightarrow{CE} = \left(-\frac{2}{3}, \frac{1}{3}, 0\right).
        \]
        故面积为
        \[
        [\triangle CDE] = \frac{1}{2}|\vec{n}| = \frac{1}{2} \sqrt{\left(-\frac{2}{3}\right)^2 + \left(\frac{1}{3}\right)^2 + 0^2} = \frac{\sqrt{5}}{6}.
        \]
        又
        \[
        \overrightarrow{AB} = (-1, 1, 0), \quad |\overrightarrow{AB}| = \sqrt{2}, \quad |\vec{n}| = \frac{\sqrt{5}}{3}.
        \]
        于是
        \[
        \cos \theta = \frac{\overrightarrow{AB} \cdot \vec{n}}{|\overrightarrow{AB}||\vec{n}|} = \frac{3}{\sqrt{10}}.
        \]
        而平面间距离
        \[
        d(E_1, E_4) = \frac{3}{5 \sqrt{2}}.
        \]
        原四面体的棱长为
        \[
        \frac{d(E_1, E_4)}{\cos \theta} = \frac{1}{\sqrt{5}}
        \]
        截面面积为
        \[
        [\triangle CDE] \cdot \frac{1}{2} \left(\frac{1}{\sqrt{5}}\right)^2 = \frac{\sqrt{5}}{60}
        \]
    \end{solution}

    \question 空间中三点$P,Q,R$分别在直线
    \[
    L_1: \frac{x-3}{1} = \frac{y-6}{2} = \frac{z+1}{-2}, \;
    L_2: \frac{x-2}{-2} = \frac{y-7}{2} = \frac{z-4}{1}, \;
    L_3: \frac{x-1}{2} = \frac{y-5}{1} = \frac{z-6}{2}
    \]
    上,求$PQ+PR$的最小值。
    \begin{solution}
        直线方向向量为
        \[
        \vec{u}_1 = (1,2,-2), \quad \vec{u}_2 = (-2,2,1), \quad \vec{u}_3 = (2,1,2).
        \]
        发现
        \[
        \vec{u}_1 \times \vec{u}_2 = (6,3,6) \parallels \vec{u}_3,
        \]
        说明三条直线两两垂直且互为歪斜线。已知点
        \[
        P(3,6,-1) \in L_1, \quad Q(2,7,4) \in L_2, \quad R(1,5,6) \in L_3,
        \]
        则向量
        \[
        \overrightarrow{PQ} = (-1,1,5), \quad \overrightarrow{QR} = (-1,-2,2), \quad \overrightarrow{RP} = (2,1,-7).
        \]
        满足
        \[
        d(L_1,L_2) = \frac{|\overrightarrow{PQ} \cdot \vec{u}_3|}{|\vec{u}_3|} = 3, \quad
        d(L_2,L_3) = \frac{|\overrightarrow{QR} \cdot \vec{u}_1|}{|\vec{u}_1|} = 3, \quad
        d(L_3,L_1) = \frac{|\overrightarrow{RP} \cdot \vec{u}_2|}{|\vec{u}_2|} = 3.
        \]
        因此问题等价于在棱长为3的立方体中考虑三条互相垂直的线段:
        \[
        L_1 = \{(a,0,3) \mid a \in \mathbb{R}\}, \quad
        L_2 = \{(3,b,0) \mid b \in \mathbb{R}\}, \quad
        L_3 = \{(0,3,c) \mid c \in \mathbb{R}\}.
        \]
        设点$A =(3,b,0),B =(0,3,c),P =(a,0,3)$,则
        \[
        PA+PB = \sqrt{(a-3)^2 + b^2 + 9} + \sqrt{a^2 + 9 + (c-3)^2}.
        \]
        在
        \[
        a = \frac{3}{2}, \quad b=0, \quad c=3
        \]
        有最小值
        \[
        \sqrt{\frac{45}{4}} + \sqrt{\frac{45}{4}} = 3\sqrt{5}
        \]
    \end{solution}

    \question 空间中一定点 $A(2,6,-3)$,一平面 $E: x+2y+2z+1=0$,已知平面 $E$ 上有一圆 $C$,圆心为 $Q(-3,2,-1)$,半径为 $2$。若动点 $P$ 在圆 $C$ 上,试求 $AP$ 的最大值与最小值。
    \begin{solution}
        过点 $A(2,6,-3)$ 且方向向量为 $(1,2,2)$ 的直线为
        \[
        L: \frac{x-2}{1} = \frac{y-6}{2} = \frac{z+3}{2}
        \]
        令 $t$ 为参数,则该直线上的点可表示为
        \[
        A'(x,y,z) = (t+2,\,2t+6,\,2t-3)
        \]
        代入平面 $E$ 的方程求交点:
        \[
        (t+2) + 2(2t+6) + 2(2t-3) + 1 = 0\Rightarrow t = -1
        \]
        得投影点 $A' = (1,4,-5)$,而
        \[
        A'Q = \sqrt{(-3-1)^2 + (2-4)^2 + (-1+5)^2}  = 6 > 2 = r
        \Rightarrow A' \text{ 在圆 } C \text{ 外}
        \]
        设直线 $A'Q$ 与圆交于两点 $B$(较近),$C$(较远),则
        \[
        AA' = \sqrt{(2-1)^2 + (6-4)^2 + (-3+5)^2} = \sqrt{1 + 4 + 4} = 3 
        \]
        \[
        A'B = A'Q - r = 6 - 2 = 4 \Rightarrow AB^2 = 3^2 + 4^2 = 25 \Rightarrow AB = 5 
        \]
        \[
        A'C = A'Q + r = 6 + 2 = 8 \Rightarrow AC^2 = 3^2 + 8^2 = 73 \Rightarrow AC = \sqrt{73}
        \]
    \end{solution}

    \question 在正立方体 $ABCD-EFGH$ 中,有两质点分别从顶点 $A,C$ 同时以等速直线运动到顶点 $B,D$,均在 $1$ 秒后到达。求两质点之间的最小距离。  
    \begin{figure}[H]
        \centering
        \includegraphics[width=0.2\textwidth]{images/image10.png}
    \end{figure}
    \begin{solution}
        建立空间坐标系
        \[
        A(1,0,0),B(1,1,0),B(1,1,0),C(0,1,0),D(0,1,1),
        \]
        则两质点在$AB$及$CD$的坐标分别为$(1,t,0),(0,1,t)$,两质点间距离
        \[
        \sqrt{(1-0)^2+(t-1)^2+(0-t)^2}=\sqrt{2t^2-2t+2}=\sqrt{2\left(t-\frac12\right)^2+\frac32}
        \]
        在$t=\dfrac{1}{2}$有最小值$\dfrac{\sqrt6}{2}$。
    \end{solution}

    \question 长方体 $ABCD-EFGH$ 中,已知直线 $AC$ 的方程为 
    \[
    \frac{x-3}{-2}=\frac{y+1}{2}=\frac{z+7}{1},
    \]
    直线 $HF$ 的方程为 
    \[
    \frac{x}{1}=\frac{y}{4}=\frac{z}{-3},
    \]
    且 $A(3,-1,-7)$,求矩形 $ABCD$ 的面积。
    \ifprintanswers
    \begin{figure}[H]
        \centering        
        \includegraphics[width=0.5\textwidth]{images/image81.png}
    \end{figure}
    \fi
    \begin{solution}
        记
        \[
        L_1=\overleftrightarrow{AC}:\ \frac{x-3}{-2}=\frac{y+1}{2}=\frac{z+7}{1},\quad
        L_2=\overleftrightarrow{HF}:\ \frac{x}{1}=\frac{y}{4}=\frac{z}{-3}.
        \]
        方向向量分别为
        \[
        \vec u=(-2,2,1),\quad \vec v=(1,4,-3),
        \]
        $L_1,L_2$所成平面的法向量为
        \[
        \vec n=\vec u\times \vec v=(-10,-5,-10)
        \]
        取参数点
        \[
        P\in L_1:\ P(-2t+3,\,2t-1,\,t-7),\quad
        Q\in L_2:\ Q(s,\,4s,\,-3s),
        \]
        令 $\overrightarrow{PQ}\parallels \vec n$,解得
        \[
        \frac{s+2t-3}{2}=\frac{4s-2t+1}{1}=\frac{-3s-t+7}{2}\ \Rightarrow\ s=1,\ t=2,
        \]
        从而
        \[
        P(-1,3,-5),\quad Q(1,4,-3),\quad PA=PB=PC=6,
        \]
        设 $L_1$ 与 $L_2$ 的夹角为 $\theta$,则
        \[
        \cos\theta=\frac{\vec u\cdot \vec v}{\lVert\vec u\rVert\,\lVert\vec v\rVert}=\frac{1}{\sqrt{26}}.
        \]
        由余弦定理,
        \[
        AB^2=6^2+6^2-2\cdot6\cdot6\cdot\frac{1}{\sqrt{26}},\quad
        AD^2=6^2+6^2-2\cdot6\cdot6\cdot\left(-\frac{1}{\sqrt{26}}\right)
        \]
        于是矩形 $ABCD$面积为
        \[
        [ABCD]=AB\cdot AD
        =\sqrt{\frac{72(\sqrt{26}-1)}{\sqrt{26}}\cdot \frac{72(\sqrt{26}+1)}{\sqrt{26}}}
        =\frac{180\sqrt{26}}{13}
        \]
    \end{solution}

    \question 已知四面体 $ABCD$顶点坐标分别为 $A(0,0,0),B(1,0,0),C(0,1,0),D(1,1,1)$,求此四面体内切球球心的 $x$ 坐标。
    \begin{solution}
        四面体各面所在平面为:
        \[
        \begin{cases}
        E_1 = \triangle ABC: z=0 \\
        E_2 = \triangle ACD: x - z = 0 \\
        E_3 = \triangle ABD: -y + z = 0 \\
        E_4 = \triangle BCD: x + y - z = 1
        \end{cases}
        \]
        设球心为 $P(a,b,c)$,球心到四面的距离相等,即
        \[
        \begin{cases}
        d(P,E_1) = c \\[4pt]
        d(P,E_2) = \dfrac{|a - c|}{\sqrt{2}} \\[6pt]
        d(P,E_3) = \dfrac{| - b + c|}{\sqrt{2}} \\[6pt]
        d(P,E_4) = \dfrac{|a + b - c - 1|}{\sqrt{3}}\\[6pt]
        \end{cases}
        \]
        由于 $\triangle OAB$ 为等腰直角三角形,故 $a = b$。假设 $a < c$,则
        \[
        d(P,E_2) = \frac{c - a}{\sqrt{2}} = c \Rightarrow a = (1 - \sqrt{2}) c < 0,
        \]
        不合题意,故 $a \ge c$,得
        \[
        a = (\sqrt{2} + 1) c
        \]
        考虑
        \[
        d(P,E_4) = \frac{|(2 \sqrt{2} + 1) c - 1|}{\sqrt{3}} = c
        \]
        假设 $c \ge \dfrac{1}{2 \sqrt{2} + 1}$,则
        \[
        c = \frac{1}{2 \sqrt{2} - \sqrt{3} + 1} \Rightarrow a = \frac{\sqrt{2} + 1}{2 \sqrt{2} - \sqrt{3} + 1} > 1,
        \]
        球心在四面体外,不合题意。因此
        \[
        c \le \frac{1}{2 \sqrt{2} + 1} \Rightarrow c = \frac{1}{2 \sqrt{2} + \sqrt{3} + 1},
        \]
        故内切球球心的 $x$ 坐标为
        \[
        a=\frac{\sqrt{2} + 1}{2 \sqrt{2} + \sqrt{3} + 1}
        \]
    \end{solution}

    \question 某房间内设有一盏聚光灯,其照射的光线为直圆锥状。为描述其空间几何关系,建立如图所示的空间坐标系。已知光源 $S(2\sqrt{6},3\sqrt{6},12)$,分别沿对称轴 $L$ 的方向向量 $\left(\dfrac{\sqrt{2}}{2},-\dfrac{\sqrt{2}}{2},-\sqrt{3}\right)$ 以及其中一条母线 $M$ 的方向向量 $(0,0,-1)$ 向地面 $z=0$ 照射。试问:此聚光灯照射在墙面(即平面 $y=0$)上的光线边缘,为哪一种圆锥曲线图形的一部分?该圆锥曲线在墙面上的顶点坐标为何?
    \ifprintanswers
    \begin{figure}[H]
        \centering        
        \includegraphics[width=0.4\textwidth]{images/image229.png}
    \end{figure}
    \fi
    \begin{solution}
        直线 $L$ 的方向向量 $\vec \ell = \left(\dfrac{\sqrt{2}}{2},-\dfrac{\sqrt{2}}{2},-\sqrt{3}\right)$,母线 $M$ 的方向向量 $\vec m = (0,0,-1)$。设两直线夹角为 $\theta$,则  
        \[
        \cos \theta = \frac{\vec \ell \cdot \vec m}{|\vec \ell||\vec m|} = \frac{\sqrt 3}{2} \Rightarrow \theta = 30^\circ \text{ 或 } 150^\circ
        \]
        假设点 $P(x,y,z)$ 在光线边缘上,则  
        \[
        \overrightarrow{SP} = (x-2\sqrt 6,\,y-3\sqrt 6,\,z-12)
        \]  
        与 $\vec \ell$ 的夹角为 $30^\circ$,由
        $\overrightarrow{SP} \cdot \vec \ell = |\overrightarrow{SP}|\,|\vec \ell| \cos 30^\circ$得到  
        \[
        \frac{\sqrt 2}{2}(x-2\sqrt 6) - \frac{\sqrt 2}{2}(y-3\sqrt 6) - \sqrt 3 (z-12) = \sqrt 3 \sqrt{(x-2\sqrt 6)^2 + (y-3\sqrt 6)^2 + (z-12)^2}
        \]  
        在墙面 $y=0$ 代入上式,化简得  
        \[
        5x^2 + 2\sqrt 6 x z - 50\sqrt 6 x + 12 z + 318 = 0
        \]  
        判别式为 
        \[
        (2\sqrt 6)^2 - 0 = 24 > 0
        \]  
        因此图形为双曲线。对称轴 $L$ 的参数方程为  
        \[
        \frac{x-2\sqrt 6}{\frac{\sqrt{2}}{2}} = \frac{y-3\sqrt 6}{-\frac{\sqrt{2}}{2}} = \frac{z-12}{-\sqrt 3}
        \]  
        在 $y=0$ 的投影为  
        \[
        \frac{x-2\sqrt 6}{\frac{\sqrt{2}}{2}} = \frac{z-12}{-\sqrt 3} \Rightarrow x = 2\sqrt 6 - \frac{z-12}{\sqrt 6}
        \]  
        代入双曲线方程可得  
        \[
        x = -\sqrt 6 \pm 18 \sqrt{\frac{2}{7}}, \quad z = 30 \pm 36 \sqrt{\frac{3}{7}}
        \]  
        由图形可知,选择  
        \[
        z = 30 - 36 \sqrt{\frac{3}{7}}
        \]  
        因此顶点坐标为  
        \[
        \left(-\sqrt 6 + 18 \sqrt{\frac{2}{7}},\,0,\,30 - 36 \sqrt{\frac{3}{7}}\right)
        \]
    \end{solution}

\end{questions}
