\pagecolor{PageColor}
\
\vfil
\hfil  {\fontsize{50pt}{36pt}\selectfont{微积分}} \hfil
\vfil
\begin{tikzpicture}[remember picture,overlay,every node/.style={inner sep=0pt}]
        \node [shift={(1cm,-1cm)},brown,scale=2,anchor=north west] (CNW)
        at (current page.north west) {\pgfornament[height=1cm,width=1cm]{61}};
        \node [shift={(-1cm,-1cm)},brown,scale=2,anchor=north east] (CNE)
        at (current page.north east) {\pgfornament[height=1cm,width=1cm,symmetry=v]{61}};
        \node [shift={(1cm,1cm)},brown,scale=2,anchor=south west] (CSW)
        at (current page.south west) {\pgfornament[height=1cm,width=1cm,symmetry=h]{61}};
        \node [shift={(-1cm,1cm)},brown,scale=2,anchor=south east] (CSE)
        at (current page.south east) {\pgfornament[height=1cm,width=1cm,symmetry=c]{61}};
        \pgfornamentline[color=brown]{current page.north west}{current page.north east}{2}{87}
        \pgfornamentline{current page.south west}{current page.south east}{2}{87}
        \pgfornamentline{current page.north west}{current page.south west}{3}{87}
        \pgfornamentline{current page.north east}{current page.south east}{3}{87}
        \end{tikzpicture}%
\thispagestyle{empty}
\pagebreak
\begin{center}
  {\fontsize{30pt}{26pt}\selectfont
    \hypertarget{极限}{极限} \label{极限}
  }
\end{center}
\separator
\vspace{1pt}
\nopagecolor
\begin{questions}
    \question 
    \[
    \lim_{x \to 0} \floor{x}
    \]
    \begin{solution}
        由于
        \[
        \lim_{x \to 0^{+}} \floor{x}=0, \quad \lim_{x \to 0^{-}} \floor{x}=-1
        \]
        极限$\displaystyle \lim_{x \to 0} \floor{x}$ 不存在。
    \end{solution}

    \question 
    \[
    \lim_{x \to 3} (\ceil{x}-\floor{x})
    \]
    \begin{solution}
        由于
        \[
        \lim_{x \to 3^{+}} (\ceil{x}-\floor{x}) = \lim_{x \to 3^{+}} \ceil{x} - \lim_{x \to 3^{+}} \floor{x}=4-3=1 
        \]
        且
        \[
        \lim_{x \to 3^{-}} (\ceil{x}-\floor{x}) = \lim_{x \to 3^{-}} \ceil{x} - \lim_{x \to 3^{-}} \floor{x}=3-2=1 
        \]
        故
        \[
        \lim_{x \to 3} (\ceil{x}-\floor{x}) =1
        \]
    \end{solution}

    \question 
    \[
    \lim_{x \to 0} \;\floor*{\frac{\sin x }{x}}
    \]
    \begin{solution}
        由于
        \[
        \lim_{x \to 0^{+}} \floor*{\frac{\sin x }{x}} = \lim_{x \to 0^{-}} \floor*{\frac{\sin x }{x}}=0
        \]
        故
        \[
        \lim_{x \to 0} \;\floor*{\frac{\sin x }{x}}=0
        \]
    \end{solution}

    \question 
    \[
    \lim_{x \to 0} \frac{\sin(\sin x)}{x}
    \]
    \begin{solution}
        有
        \begin{align*}
        \lim_{x \to 0} \frac{\sin(\sin x)}{x} 
        &=\,\lim_{x \to 0} \frac{\sin(\sin x)}{\sin x} \cdot \frac{\sin x}{x} \\[2mm]
        &=\,1 \cdot 1 \\
        &= 1
        \end{align*}
    \end{solution}
    
    \question 
    \[
    \lim_{\theta \to 0} \frac{\theta - \theta \cos \theta}{\sin \theta \tan \theta}
    \]
    \begin{solution}
        有
        \begin{align*}
        \,\lim_{\theta \to 0} \frac{\theta - \theta \cos \theta}{\sin \theta \tan \theta} 
        &=\,\lim_{\theta \to 0} \frac{1 - \cos \theta}{\frac{\sin \theta}{\theta}} \cdot \frac{1}{\tan \theta} \\
        &=\,\lim_{\theta \to 0} \frac{1 - \cos \theta}{\theta^2}
        \cdot \frac{1}{\frac{\sin \theta}{\theta}} 
        \cdot \frac{\theta}{\tan \theta}
        \cdot \theta \\
        &=\,\frac{1}{2} \cdot \frac{1}{1} \cdot \frac{1}{1} \cdot 0 \\
        &= 0
        \end{align*}
    \end{solution}  

    \question 
    \[
    \lim_{\theta \to 0} \frac{1 - \cos\left(\frac{1 - \cos \theta}{2}\right)}{\theta^4}
    \]
    \begin{solution}
        有
        \begin{align*}
        \lim_{\theta \to 0} \frac{1 - \cos\left(\frac{1 - \cos \theta}{2}\right)}{\theta^4} 
        &=\,\lim_{\theta \to 0} \left[
        \frac{1 - \cos\left(\frac{1 - \cos \theta}{2}\right)}{\left( \frac{1 - \cos \theta}{2} \right)^2}
        \cdot \left( \frac{1 - \cos \theta}{\theta^2} \right)^2 \cdot \frac{1}{4}
        \right] \\
        &=\,\frac{1}{2} \cdot \left(\frac{1}{2}\right)^2 \cdot \frac{1}{4} = \frac{1}{32}
        \end{align*}
        其中 \[
        \lim_{x \to 0} \frac{1+\cos x}{x^2}= \frac14\lim_{x \to 0}\frac{2\sin^2 \frac{x}{2} }{(\frac{x}{2})^2}=\frac12 \lim_{x \to 0} \left(\frac{\sin \frac{x}{2}}{\frac{x}{2}}\right)^2=\frac12
        \]
    \end{solution}
        
    \question 
    \[
    \lim_{x \to 0} \frac{\cos x - 1}{\sin^{-1} x \tan x}
    \]
    \begin{solution}
        有
        \begin{align*}
        \lim_{x \to 0} \frac{\cos x - 1}{\sin^{-1} x \tan x} 
        &=\,\lim_{x \to 0} \frac{\cos x - 1}{\sin^{-1} x \cdot \frac{\sin x}{\cos x}} \\
        &=\,\lim_{x \to 0} \left(-\frac{1 - \cos x}{x^2}\right) \cdot \frac{x}{\sin^{-1} x} \cdot \frac{x}{\sin x} \cdot \cos x \\
        &=\, -\frac{1}{2} \cdot 1 \cdot 1 \cdot 1 \\ 
        &= -\frac{1}{2}
        \end{align*}
    \end{solution}

    \question 求极限 
    \[
    \lim_{x \to 1} \left[ \cos\left(\frac{\pi}{2}x\right) + x\cos\left(\frac{\pi}{2}x\right) + x^2\cos\left(\frac{\pi}{2}x\right) + \dots \right]
    \]
    \begin{solution}
        注意到
        \begin{align*}
        &\lim_{x \to 1} \left[ \cos\left(\frac{\pi}{2}x\right) + x\cos\left(\frac{\pi}{2}x\right) + x^2\cos\left(\frac{\pi}{2}x\right) + \dots \right] \\
        &= \lim_{x \to 1}  \frac{\cos\left(\frac{\pi}{2}x\right)}{1-x} \\
        &\overset{H}{=} \lim_{x \to 1} \frac{-\frac{\pi}{2} \sin\left(\frac{\pi}{2}x\right)}{-1} \\
        &= \frac{\pi}{2} \sin\frac{\pi}{2} \\
        &= \frac{\pi}{2}
        \end{align*}
    \end{solution}

    \question 计算极限 
    \[
    \lim_{x \to 0} \left( \sqrt{\frac{1}{x(x-1)} + \frac{1}{4x^2}} - \frac{1}{2x} \right)
    \]
    \begin{solution}
        有理化得
        \begin{align*}
        \lim_{x \to 0} \left( \sqrt{\frac{1}{x(x-1)} + \frac{1}{4x^2}} - \frac{1}{2x} \right) 
        &= \lim_{x \to 0} \frac{\left( \frac{1}{x(x-1)} + \frac{1}{4x^2} \right) - \frac{1}{4x^2}}{\sqrt{\frac{1}{x(x-1)} + \frac{1}{4x^2}} + \frac{1}{2x}} \\
        &= \lim_{x \to 0} \frac{\frac{1}{x-1}}{\sqrt{\frac{x}{x-1} + \frac{1}{4}} + \frac{1}{2}} \\
        &= \frac{-1}{\frac{1}{2} + \frac{1}{2}} \\
        &= -1
        \end{align*}
    \end{solution}

    \question 试求
    \[
    \lim_{x\to\infty}\left(\sqrt[5]{x^{5}+3x^{4}+4x^{3}+3x}-\sqrt[3]{x^{3}+3x^{2}+4x+1}\right).
    \]
    \begin{solution}
        由洛必达法则,
        \begin{align*}
        &\lim_{x\to \infty} \left(\sqrt[5]{x^5+3x^4+4x^3+3x} - \sqrt[3]{x^3+3x^2+4x+1} \right) \\
        &= \lim_{x\to \infty} \frac{\sqrt[5]{1 + \frac{3}{x} + \frac{4}{x^2} + \frac{3}{x^4}} - \sqrt[3]{1 + \frac{3}{x} + \frac{4}{x^2} + \frac{1}{x^3}}}{\frac{1}{x}} \\
        &\overset{H}{=} -\lim_{x\to \infty} x^2\left(
        \frac{1}{5} \left(1 + \frac{3}{x} + \frac{4}{x^2} + \frac{3}{x^4}\right)^{-\frac{4}{5}} \left(-\frac{3}{x^2} - \frac{8}{x^3} - \frac{12}{x^5}\right) \right. \\
        & \quad \left. - \frac{1}{3} \left(1 + \frac{3}{x} + \frac{4}{x^2} + \frac{1}{x^3}\right)^{-\frac{2}{3}} \left(-\frac{3}{x^2} - \frac{8}{x^3} - \frac{3}{x^4}\right) 
        \right)\\
        &= \frac{3}{5} - 1 = -\frac{2}{5}
        \end{align*}
    \end{solution}

    \question 已知等差数列 $\{a_n\},\{b_n\}$ 公差均不为 $0$, 且$\displaystyle \lim_{n\to\infty}\frac{a_n}{b_n}=5$,求
    \[
    \lim_{n\to\infty}\frac{a_1+a_2+\cdots+a_n}{n\cdot b_{2n}}.
    \]
    \begin{solution}
        设 $\{a_n\},\{b_n\}$ 公差为 $d_1,d_2$,则
        \[
        a_n=a_1+(n-1)d_1,\quad b_n=b_1+(n-1)d_2.
        \]
        由已知极限得
        \[
        \lim_{n\to\infty}\frac{a_n}{b_n}=\lim_{n\to\infty}\frac{a_1+(n-1)d_1}{b_1+(n-1)d_2}=\frac{d_1}{d_2}=5.
        \]
        所以 $d_1=5d_2$,因此所求极限为
        \[
        \lim_{n\to\infty}\frac{\frac{n}{2}\bigl(2a_1+(n-1)d_1\bigr)}{n\bigl(b_1+(2n-1)d_2\bigr)}
        = \lim_{n\to\infty}\frac{2a_1+(n-1)d_1}{2\bigl(b_1+(2n-1)d_2\bigr)}=\frac{d_1}{4d_2}=\frac{5}{4}
        \]
    \end{solution}

    \question 设 $a_1+a_2+\cdots+a_n=n^3-2n$,试求
    \[
    \lim_{n\to\infty}\frac{\sqrt[3]{a_3+a_6+\cdots+a_{3n}}-\sqrt[3]{a_2+a_4+\cdots+a_{2n}}}{n}
    \]
    \begin{solution}
        由 
        \[
        S(n)=\sum_{k=1}^n a_k=n^3-2n
        \] 
        得
        \[
        a_n=S(n)-S(n-1)=n^3-2n-\left[(n-1)^3-2(n-1)\right]=3n^2-3n-1
        \]
        于是
        \[
        a_{3k}=27k^2-9k-1,\quad a_{2k}=12k^2-6k-1,
        \]
        且
        \[
        \sum_{k=1}^n a_{3k}= \frac{27}{6}n(n+1)(2n+1)-\frac{9}{2}n(n+1)-n=9 n^3 + 9 n^2 - n,
        \]
        \[
        \sum_{k=1}^n a_{2k}= 2n(n+1)(2n+1)-3n(n+1)-n=4 n^3 + 3 n^2 - 2 n
        \]
        故极限为
        \[
        \lim_{n\to\infty}\left(\sqrt[3]{9+\frac{9}{n}-\frac{1}{n^2}}-\sqrt[3]{4+\frac{3}{n}-\frac{2}{n^2}}\right)
        =\sqrt[3]{9}-\sqrt[3]{4}
        \]
    \end{solution}

    \question 
    \[
    \lim_{m\to\infty} \lim_{n\to\infty} \frac{\sqrt[n]{1+3^{2n}}+\sqrt[n]{3^{2n}+5^{2n}}+\sqrt[n]{5^{2n}+7^{2n}}+\cdots+\sqrt[n]{(2m-1)^{2n}+(2m+1)^{2n}}}{m^{3}}
    \]
    \begin{solution}
        首先有
        \begin{align*}
        &\lim_{n\to\infty} \left( \sqrt[n]{1+3^{2n}} +\sqrt[n]{3^{2n}+5^{2n}} + \cdots +\sqrt[n]{(2m-1)^{2n}+(2m+1)^{2n}} \right)\\
        &=\lim_{n\to\infty} \left( 3^2\sqrt[n]{\left(\frac13\right)^{2n}+1} + 5^2\sqrt[n]{\left(\frac35\right)^{2n}+1} + \cdots + (2m+1)^2\sqrt[n]{\left(\frac{2m-1}{2m+1}\right)^{2n}+1} \right)\\
        &= \sum_{k=1}^m (2k+1)^2 \\
        &= \sum_{k=1}^m (4k^2+4k+1) \\
        &=\frac{2}{3}m(m+1)(2m+1) + 2m(m+1) + m
        \end{align*}
        因此,原式
        \[
        \lim_{m\to\infty} \frac{\frac{2}{3}m(m+1)(2m+1) + 2m(m+1) + m}{m^3} = \frac{4}{3}
        \]
    \end{solution}

    \question 
    \[
    \lim_{x \to 0} \frac{\int_{\cos x}^1 e^{-y^2} \, dy}{x \sin x}
    \]
    \begin{solution}
        由洛必达法则,
        \begin{align*}
        \lim_{x \to 0} \frac{\int_{\cos x}^1 e^{-y^2} \, dy}{x \sin x}
        &\overset{H}{=} \lim_{x \to 0} \frac{\sin x \cdot e^{-(\cos x)^2}}{\sin x + x \cos x} \\
        &\overset{H}{=} \lim_{x \to 0} \frac{\cos x(1 + 2\sin^2 x) \cdot e^{-(\cos x)^2}}{2\cos x - x\sin x} \\
        &= \frac{1}{2e}
        \end{align*}
    \end{solution}

    \question 
    \[
    \lim_{x \to 1} \frac{(1 - \sqrt{x})(1 - \sqrt[3]{x})}{1 + \cos \pi x}
    \]
    \begin{solution}
        消去根号,令 $x = t^6$,当 $x \to 1$ 时,$t \to 1$,由洛必达法则,
        \begin{align*}
        \lim_{x \to 1} \frac{(1 - \sqrt{x})(1 - \sqrt[3]{x})}{1 + \cos \pi x} 
        &= \lim_{t \to 1} \frac{(1 - t^3)(1 - t^2)}{1 + \cos \pi t^6} \\
        &= \lim_{t \to 1} \frac{1 - t^2 - t^3 + t^5}{1 + \cos \pi t^6} \\
        &\overset{H}{=} \lim_{t \to 1} \frac{-2t - 3t^2 + 5t^4}{-6\pi t^5 \sin \pi t^6} \\
        &\overset{H}{=} \lim_{t \to 1} \frac{-2 - 6t + 20t^3}{-30\pi t^4 \sin \pi t^6 - 6\pi t^5 \cdot \cos \pi t^6 \cdot 6\pi t^5} \\
        &= \frac{-2 - 6(1) + 20(1)^3}{-6\pi \cdot (-1) \cdot 6\pi} \\
        &= \frac{1}{3\pi^2}
        \end{align*}
    \end{solution}

    \question 
    \[
    \lim_{x \to 0} \frac{\left(\int_0^x t \cos t^2 \, dt \right)^2}{\int_0^x \sin t^2 \, dt}
    \]
    \begin{solution}
        首先发现到
        \[
        \int_0^x t \cos t^2 \, dt = \left[ \frac{1}{2} \sin t^2 \right]_0^x = \frac{1}{2} \sin x^2
        \]
        将积分结果代入原极限式:
        \begin{align*}
        \lim_{x \to 0} \frac{\left(\int_0^x t \cos t^2 \, dt \right)^2}{\int_0^x \sin t^2 \, dt} 
        &= \lim_{x \to 0} \frac{\frac{1}{4} \sin^2 x^2}{\int_0^x \sin t^2 \, dt} \\
        &\overset{H}{=} \lim_{x \to 0} \frac{\frac{1}{4} \cdot 2 \sin x^2 \cdot \cos x^2 \cdot 2x}{\sin x^2} \\
        &= \lim_{x \to 0} (x \cos x^2) \\
        &= 0 \cdot 1 = 0
        \end{align*}
    \end{solution}

    \question 求实数 $a,b$,使得
    \[
    \lim_{x\to 0}\frac{\sqrt{ax+b}-2}{x}=1.
    \]
    \begin{solution}
        有理化得
        \[
        \lim_{x\to 0}\frac{\sqrt{ax+b}-2}{x}\cdot
        \frac{\sqrt{ax+b}+2}{\sqrt{ax+b}+2}
        =\lim_{x\to 0}\frac{ax+b-4}{x(\sqrt{ax+b}+2)}
        \]
        当 $x\to 0$ 时分母趋于 $0$,欲使极限存在,分子也必须趋于 $0$,因此
        \[
        a\cdot 0+b-4=0,
        \]
        从而 $b=4$,于是
        \[
        \lim_{x\to 0}\frac{a}{\sqrt{ax+4}+2}=\frac{a}{\sqrt{4}+2}=1,
        \]
        解得
        \[
        a=4
        \]
    \end{solution}

    \question 已知
    \[
    \lim_{x \to -1} \frac{a x^2 + 3x + b}{x + 1} = 2,\;\; a,b \in \mathbb{R}
    \]
    试求 \( (a, b) \)。
    \begin{solution}
        当$x\rightarrow-1,x+1\rightarrow0;$欲使该极限存在且等于$2$,分子$a x^2 + 3x + b$在 \(x\rightarrow-1\) 也应趋于$0$,即
        \[
        \lim_{x \to -1} (a x^2 + 3x + b)=0 \Rightarrow a(-1)^2 + 3(-1) + b = 0 \Rightarrow  a + b = 3 \tag{1}
        \]
        又原极限中分子分母处处可导,且极限
        \[
        \lim_{x \to -1} \frac{(a x^2 + 3x + b)'}{(x + 1)'}
        = \lim_{x \to -1} \frac{2a x + 3}{1} = -2a + 3 
        \]
        存在,则由洛必达法则,
        \[
        \lim_{x \to -1} \frac{a x^2 + 3x + b}{x + 1} \overset{H}{=}-2a + 3 = 2 \Rightarrow (a,b) = \left( \dfrac{1}{5}, \dfrac{14}{5} \right)
        \]
    \end{solution}

    \question 已知 $a,b$ 为正整数,设函数 
    \[
    f(x)=\lim_{n\rightarrow \infty}\frac{2x^{2n+1}+ax^{2}+bx-1}{2x^{2n}+3},
    \]
    若 $\forall x\in \mathbb{R},f(x)$ 为连续函数,求序对 $(a,b)$。
    \begin{solution}
        首先注意到
        \[
        f(x) = \lim_{n \rightarrow \infty} \frac{2x^{2n+1}+ ax^2+bx-1}{2x^{2n}+3} \\
        = \lim_{n \rightarrow \infty} \frac{x+ \frac{ax^2+bx-1}{2x^{2n}}}{1+\frac{3}{2x^{2n}}} \\
        = 
        \begin{cases}
        x, & |x| > 1,\\[1mm]
        \dfrac{ax^2+bx-1}{3}, & |x| < 1.
        \end{cases}
        \]
        由连续性,
        \[
        \lim_{x \rightarrow 1^+} f(x) = 1, \quad \lim_{x \rightarrow 1^-} f(x) = \frac{a+b-1}{3}, \quad f(1) = \lim_{n \rightarrow \infty} \frac{2+a+b-1}{2+3} = \frac{a+b+1}{5}
        \]
        可得
        \[
        \frac{a+b+1}{5} = 1 = \frac{a+b-1}{3} \quad \Rightarrow \quad a+b = 4 \tag{1}
        \]
        同理,
        \[
        \lim_{x \rightarrow -1^+} f(x) = \frac{a-b-1}{3}, \quad \lim_{x \rightarrow -1^-} f(x) = -1,\quad f(-1) = \frac{-2+a-b-1}{5} = \frac{a-b-3}{5}
        \]
        可得
        \[
        \frac{a-b-3}{5} = -1 = \frac{a-b-1}{3} \quad \Rightarrow \quad a-b = -2 \tag{2}
        \]
        由$(1),(2)$得 
        \[
        (a,b) = (1,3)
        \]
    \end{solution}
    
    \question 若
    \[
    \lim_{x \to 0} \frac{\sqrt{f(x) \sin x + 1} - 1}{e^{4x} - 1} = 2,
    \]
    求 
    \[
    \lim_{x \to 0} f(x)
    \]
    \begin{solution}
        当 $x \to 0$ 时,原式为不定型 $\dfrac{0}{0}$,可用洛必达法则:
        \[
        \lim_{x \to 0} \frac{\sqrt{f(x) \sin x + 1} - 1}{e^{4x} - 1}
        \overset{H}{=}
        \lim_{x \to 0} 
        \frac{\frac{1}{2\sqrt{f(x)\sin x + 1}} \cdot \left[ f'(x)\sin x + f(x)\cos x \right]}
        {4e^{4x}}.
        \]
        当 $x \to 0$,
        \[
        \sin x \to 0,\cos x \to 1,e^{4x} \to 1,\sqrt{f(x)\sin x + 1} \to 1
        \]
        上式化简为
        \[
        \lim_{x \to 0} \frac{f(x)}{8} = 2
        \Rightarrow \lim_{x \to 0} f(x) = 16
        \]
    \end{solution}

    \question 定义序列 $\{x_n\}_{n=2}^\infty$ 如下:
    \[
    (n+x_n)[\sqrt{2}-1] = \ln 2.
    \]
    求 $\lim_{n\to\infty} x_n$。
    \begin{solution}
        由方程可解得
        \[
        x_n = \frac{\ln 2}{2^{\frac{1}{n}}-1} - n,
        \]
        极限形式为 $\infty - \infty$。令 $u = \dfrac{1}{n}$,并使用洛必达法则两次,有
        \begin{align*}
        \lim_{n\to\infty} x_n &= \lim_{n\to\infty} \frac{\ln 2 - n \cdot 2^{\frac{1}{n}} + n}{2^{\frac{1}{n}} - 1} \\
        &= \lim_{u\to 0} \frac{u \ln 2 - 2^u + 1}{u 2^u - u} \\
        &\overset{H}{=} \lim_{u\to 0} \frac{\ln 2 - 2^u \ln 2}{2^u - 1 + u 2^u \ln 2} \\
        &\overset{H}{=} \lim_{u\to 0} \frac{-2^u (\ln 2)^2}{2^u \ln 2 + 2^u \ln 2 + u 2^u (\ln 2)^2} \\
        &= \frac{-(\ln 2)^2}{2 \ln 2} \\
        &= -\frac{1}{2} \ln 2
        \end{align*}
    \end{solution}

    \question 计算极限
    \[
    \lim_{x\to 0} \left[\frac{1}{x^3}\int_{0}^{x} \frac{t\ln(t+1)}{t^4+\frac{1}{6}} dt\right]
    \]
    \begin{solution}
        使用两次洛必达法则,
        \begin{align*}
        \lim_{x\to 0} \left[\frac{1}{x^3}\int_{0}^{x} \frac{t\ln(t+1)}{t^4+\frac{1}{6}} dt\right]
        &\overset{H}{=} \lim_{x\to 0} \frac{\frac{x\ln(x+1)}{x^4+\frac{1}{6}}}{3x^2} \\
        &= \lim_{x\to 0} \frac{\ln(x+1)}{3x^5+\frac{x}{2}}\\
        &\overset{H}{=} \lim_{x\to 0} \frac{\frac{1}{x+1}}{15x^4+\frac{1}{2}} \\
        &=\frac{1}{0+\frac{1}{2}} \\
        &=2
        \end{align*}
    \end{solution}

    \question 求极限
    \[
    \lim_{x \to 0^+} (\sin x)^{\frac{1}{\ln x}}
    \]
    \begin{solution}
        此极限属于 $0^0$ 型。利用对数恒等式将原式重写为以 $e$ 为底的形式:
        \[
        (\sin x)^{\frac{1}{\ln x}} = \exp\left( \frac{\ln(\sin x)}{\ln x} \right)
        \]
        由于 $e^x$ 是连续函数,只需计算指数部分的极限
        \[
        L = \lim_{x \to 0^+} \frac{\ln(\sin x)}{\ln x}
        \]
        属于 $\frac{\infty}{\infty}$ 型,由洛必达法则,
        \[
        L \overset{H}{=} \lim_{x \to 0^+} \frac{\frac{\cos x}{\sin x}}{\frac{1}{x}} =  \lim_{x \to 0^+} \frac{x}{\sin x} \cdot \cos x  = 1 \cdot 1 = 1
        \]
        因此原极限为
        \[
        e^1 = e
        \]
    \end{solution}

    \question 求极限
    \[
    \lim_{x \to \infty} (x + e^x)^{\frac{1}{x}}
    \]
    \begin{solution}
        此极限属于 $\infty^0$ 型,由
        \[
        (x + e^x)^{\frac{1}{x}} = \exp\left( \frac{\ln(x + e^x)}{x} \right)
        \]
        设
        \[
        L = \lim_{x \to \infty} \frac{\ln(x + e^x)}{x}
        \]
        此极限属于 $\frac{\infty}{\infty}$ 型,由洛必达法则,
        \[
        L \overset{H}{=} \lim_{x \to \infty} \frac{\frac{1 + e^x}{x + e^x}}{1} = \lim_{x \to \infty} \frac{1 + e^x}{x + e^x}
        \]
        再由洛必达法则,
        \[
        L \overset{H}{=} \lim_{x \to \infty} \frac{e^x}{1 + e^x} = \lim_{x \to \infty} \frac{1}{\frac{1}{e^x} + 1} = \frac{1}{0 + 1} = 1
        \]
        因此,原极限为
        \[
        e^1 = e
        \]
    \end{solution}

    \question 求
    \[
    \lim_{x\to\infty} \left[\sqrt{x^2+2x-1}-\sqrt{x^2-1}\right]^x
    \]
    \begin{solution}
        首先注意到
        \begin{align*}
        &\lim_{x\to\infty} \frac{(x^2+2x-1)-(x^2-1)}{\sqrt{x^2+2x-1}+\sqrt{x^2-1}} \\
        &= \lim_{x\to\infty} \frac{2}{\sqrt{1+\frac{2}{x}-\frac{1}{x^2}} + \sqrt{1-\frac{1}{x^2}}} \\
        &= \frac{2}{1+1} = 1
        \end{align*}
        故原极限为$1^\infty$型,于是
        \[
        \lim_{x\to\infty} \left[\sqrt{x^2+2x-1}-\sqrt{x^2-1}\right]^x = \exp\left[\lim_{x\to\infty} x \left( \sqrt{x^2+2x-1} - \sqrt{x^2-1} - 1 \right)\right]
        \]
        其中
        \begin{align*}
        \lim_{x\to\infty} x \left( \sqrt{x^2+2x-1} - \sqrt{x^2-1} - 1 \right)
        &= \lim_{x\to\infty} x \cdot \frac{(x^2+2x-1) - (\sqrt{x^2-1} + 1)^2}{\sqrt{x^2+2x-1} + \sqrt{x^2-1} + 1} \\
        &= \lim_{x\to\infty} x \cdot \frac{2x - 2\sqrt{x^2-1}}{\sqrt{x^2+2x-1} + \sqrt{x^2-1} + 1} \\
        &= \lim_{x\to\infty} x \cdot \frac{\frac{2}{x + \sqrt{x^2-1}} - 1}{\sqrt{x^2+2x-1} + \sqrt{x^2-1} + 1} \\
        &= \lim_{x\to\infty} \frac{\frac{2}{x + \sqrt{x^2-1}} - 1}{\sqrt{1+\frac{2}{x}-\frac{1}{x^2}} + \sqrt{1-\frac{1}{x^2}} + \frac{1}{x^2}} \\
        &= \frac{0-1}{1+1} \\
        &= -\frac{1}{2}
        \end{align*}
        故原极限为
        \[
        \lim_{x\to\infty} \left[\sqrt{x^2+2x-1}-\sqrt{x^2-1}\right]^x = \frac{1}{\sqrt{e}}
        \]
    \end{solution}

    \question 设
    \[
    \lim_{x \to 0} \frac{\sin 6x + x f(x)}{x^3} = 0,
    \]
    求
    \[
    \lim_{x \to 0} \frac{6 + f(x)}{x^2}
    \]
    \begin{solution}
        原极限为
        \[
        \lim_{x \to 0} \frac{\sin 6x + x f(x)}{x^3} = 0.
        \]
        利用幂级数展开
        \[
        \sin 6x = 6x - \frac{(6x)^3}{6} + o(x^3) = 6x - 36 x^3 + o(x^3),
        \]
        代入得
        \[
        \frac{6x - 36 x^3 + x f(x)}{x^3} = \frac{6x + x f(x)}{x^3} - 36 + o(1) = \frac{6 + f(x)}{x^2} - 36 + o(1).
        \]
        极限存在且为0,则
        \[
        \lim_{x \to 0} \left(\frac{6 + f(x)}{x^2} - 36\right) = 0 \implies \lim_{x \to 0} \frac{6 + f(x)}{x^2} = 36.
        \]
    \end{solution}

    \question 若 $f(x)$ 为满足
    \[
    \lim_{x\to 1} \frac{f(x)}{x-1} = 36, \quad 
    \lim_{x\to -1} \frac{f(x)}{x+1} = -36, \quad 
    \lim_{x\to 2} \frac{f(x)}{x-2} = 0, \quad 
    \lim_{x\to -2} \frac{f(x)}{x+2} = 0
    \]
    的最低次多项式,求 $f(3)$。
    \begin{solution}
        设
        \[
        f(x) = (ax+b)(x-1)(x+1)(x-2)^2(x+2)^2
        \]
        由
        \[
        \lim_{x\to 1} \frac{f(x)}{x-1} = 36, \quad \lim_{x\to -1} \frac{f(x)}{x+1} = -36
        \]
        得方程式
        \[
        \begin{cases}
        (a+b)\cdot 2 \cdot 1^2 \cdot 3^2 = 36 \\
        (-a+b)\cdot (-2) \cdot (-3)^2 \cdot 1^2 = -36 
        \end{cases} \Rightarrow a=0, b=2
        \]
        因此
        \[
        f(x) = 2(x^2-1)(x-2)^2(x+2)^2 \Rightarrow f(3)=400
        \]
    \end{solution}

    \question 若
    \[
    \lim_{x \to 1} \frac{\sqrt{x^4 + 3} - [A + B(x - 1) + C(x - 1)^2]}{(x - 1)^2} = 0,
    \]
    求常数 \( A, B, C \)。
    \begin{solution}
        设
        \[
        L = \lim_{x \to 1} \frac{\sqrt{x^4 + 3} - [A + B(x - 1) + C(x - 1)^2]}{(x - 1)^2}
        \]
        由于分母当 \( x \to 1 \) 时趋于 0,而极限存在且为 0,故分子必趋于 0:
        \[
        \lim_{x \to 1} \left( \sqrt{x^4 + 3} - [A + B(x - 1) + C(x - 1)^2] \right) = \sqrt{1 + 3} - A = 2 - A = 0
        \]
        解得
        \[
        A = 2
        \]
        此时原式变为 $\frac{0}{0}$ 型,由洛必达法则,
        \[
        L \overset{H}{=} \lim_{x \to 1} \frac{\frac{2x^3}{\sqrt{x^4 + 3}} - B - 2C(x - 1)}{2(x - 1)} = 0
        \]
        同样地,由于分母趋于 0,分子必趋于 0:
        \[
        \lim_{x \to 1} \left( \frac{2x^3}{\sqrt{x^4 + 3}} - B - 2C(x - 1) \right) = \frac{2}{2} - B = 1 - B = 0
        \]
        解得
        \[
        B = 1
        \]
        再次应用洛必达法则处理该 $\frac{0}{0}$ 型极限:
        \[
        L \overset{H}{=} \lim_{x \to 1} \frac{\frac{d}{dx} \left( \frac{2x^3}{\sqrt{x^4 + 3}} - 1 - 2C(x - 1) \right)}{\frac{d}{dx} 2(x - 1)} = 0
        \]
        计算分子导数:
        \[
        \frac{d}{dx} \left( \frac{2x^3}{\sqrt{x^4 + 3}} \right) = \frac{6x^2\sqrt{x^4+3} - 2x^3 \cdot \frac{4x^3}{2\sqrt{x^4+3}}}{x^4 + 3} = \frac{6x^2(x^4+3) - 4x^6}{(x^4+3)\sqrt{x^4+3}}
        \]
        代入 \( x = 1 \) 得到
        \[
        \frac{6(4) - 4}{4\sqrt{4}} = \frac{20}{8} = \frac{5}{2}
        \]
        因此解得
        \[
        \frac{\frac{5}{2} - 2C}{2} = 0 \Rightarrow C = \frac{5}{4}
        \]
        综上,常数分别为
        \[
        A = 2,\quad B = 1,\quad C = \frac{5}{4}
        \]
    \end{solution}
    \begin{solution}
        由于极限式
        \[
        \lim_{x \to 1} \frac{\sqrt{x^4 + 3} - [A + B(x - 1) + C(x - 1)^2]}{(x - 1)^2} = 0
        \]
        符合函数 $f(x)$ 在 $x=1$ 处的二阶泰勒展开定义,其中多项式 
        \[
        P_2(x) = A + B(x-1) + C(x-1)^2
        \]
        必须等于 $f(x)$ 的二阶泰勒多项式。设 $f(x) = \sqrt{x^4+3}$,根据泰勒公式可知:
        \[
        A = f(1), \quad B = f'(1), \quad C = \frac{f''(1)}{2!}
        \]
        由
        \[
        f'(x) = \frac{1}{2\sqrt{x^4+3}} \cdot 4x^3 = \frac{2x^3}{\sqrt{x^4+3}}
        \]
        \[
        f''(x) = \frac{6x^2\sqrt{x^4+3} - 2x^3 \cdot \frac{2x^3}{\sqrt{x^4+3}}}{x^4+3} = \frac{6x^2(x^4+3) - 4x^6}{(x^4+3)\sqrt{x^4+3}}
        \]
        代入得
        \[
        f(1) = 2, \quad f'(1) = 1, \quad f''(1) = \frac{5}{2}
        \]
        因此
        \[
        A = 2,\quad B = 1,\quad C = \frac{5}{4}
        \]
    \end{solution}

    \question
    \[
    \lim_{n\to\infty}\left(\frac{1}{\sqrt{3n^{2}+1}}+\frac{1}{\sqrt{3n^{2}+2}}+\cdots+\frac{1}{\sqrt{3n^{2}+2n}}\right)
    \]
    \begin{solution}
        令\[
        L=\lim_{n\to\infty}\left(\frac{1}{\sqrt{3n^{2}+1}}+\frac{1}{\sqrt{3n^{2}+2}}+\cdots+\frac{1}{\sqrt{3n^{2}+2n}}\right)
        \]
        由夹挤定理,
        \[
        \frac{2}{\sqrt 3}=\lim_{n\to \infty} \left(\frac{2n}{\sqrt{3n^2}}    \right) < L < \lim_{n\to \infty} \left(\frac{2n}{\sqrt{3n^2+2n}} \right)=\frac{2}{\sqrt 3}
        \Rightarrow L = \frac{2\sqrt 3}{3}
        \]
    \end{solution}

    \question 设 $a_n = \sqrt{1 \cdot 2} + \sqrt{2 \cdot 3} + \cdots + \sqrt{n(n+1)}$,求 $\displaystyle \lim_{n \to \infty} \frac{a_n}{n^2}$ 的值。
    \begin{solution}
        发现到
        \[
        \sum_{k=1}^n \sqrt{k^2} < a_n < \sum_{k=1}^n \sqrt{(k+1)^2} \Rightarrow \frac{n(n+1)}{2} < a_n < \frac{n(n+3)}{2} 
        \] 
        于是有
        \[
        \lim_{n \to \infty} \frac{n(n+1)}{2n^2} < \lim_{n \to \infty} \frac{a_n}{n^2} < \lim_{n \to \infty} \frac{n(n+3)}{2n^2} \Rightarrow \frac{1}{2} < \lim_{n \to \infty} \frac{a_n}{n^2} < \frac{1}{2} 
        \] 
        由夹挤定理,
        \[
        \lim_{n \to \infty} \frac{a_n}{n^2} = \frac{1}{2}
        \]
    \end{solution}

    \question 设 $a_n = \dfrac{1 \cdot 3 \cdot 5 \cdot \dots \cdot (2n-1)}{2 \cdot 4 \cdot 6 \cdot \dots \cdot (2n)}$,求 $\displaystyle \lim_{n \to \infty} a_n$ 之值。
    \begin{solution}
        由
        \[
        (2n-1)(2n+1) < (2n)^2
        \]
        令 $S=1\cdot 3\cdot 5\cdot \cdots \cdot (2n-1)$, 则
        \begin{align*} 
        (2n+1)S^2 = (1\cdot 3)(3\cdot 5)(5\cdot 7)\cdots ((2n-3)(2n-1))((2n-1)(2n+1))
        < 2^2\cdot 4^2 \cdot 6^2\cdot \cdots \cdot (2n)^2
        \end{align*}
        即
        \[
        \sqrt{2n+1}S < 2\cdot 4\cdot 6\cdot \cdots \cdot (2n)
        \]
        于是
        \[
        \Rightarrow 0 < a_n=\frac{S}{2\cdot 4\cdot 6\cdot \cdots \cdot (2n)} < \frac{1}{\sqrt{2n+1}}
        \]
        由夹挤定理,
        \[
        0 < \lim_{n\to \infty} a_n < \lim_{n\to \infty} \frac{1}{\sqrt{2n+1}}=0 \Rightarrow \lim_{n\to \infty} a_n = 0
        \]
    \end{solution}

    \question 求极限
    \[
    \lim_{n \to \infty} \sum_{k=1}^{n} \frac{1}{\sqrt{kn}}
    \]
    \begin{solution}
        由放缩法,
        \[
        \sqrt{k-1}+ \sqrt{k} < 2\sqrt{k} < \sqrt{k} + \sqrt{k+1} 
        \]
        有理化得
        \[
        2(\sqrt{k+1}-\sqrt{k}) < \frac{2}{\sqrt{k}} < 2(\sqrt{k}-\sqrt{k-1})
        \]
        于是有
        \[
        \frac{2}{\sqrt{n}} \sum_{k=1}^{n} (\sqrt{k+1}-\sqrt{k}) < \frac{2}{\sqrt{n}} \sum_{k=1}^{n} \frac{1}{\sqrt{k}} < \frac{2}{\sqrt{n}} \sum_{k=1}^{n} (\sqrt{k}-\sqrt{k-1})
        \]
        累加得
        \[
        \frac{2\sqrt{n+1}-2}{\sqrt{n}} < \frac{2}{\sqrt{n}} \sum_{k=1}^{n} \frac{1}{\sqrt{k}} < \frac{2\sqrt{n}}{\sqrt{n}} = 2
        \]
        因为
        \[
        \lim_{n \to \infty} \frac{2\sqrt{n+1}-2}{\sqrt{n}} = 2
        \]
        由夹挤定理得,
        \[
        \lim_{n \to \infty} \sum_{k=1}^{n} \frac{1}{\sqrt{kn}} = 2
        \]
    \end{solution}

    \question 求极限
    \[\lim_{n \to \infty} \left(\frac{1}{n^2+n+1} + \frac{2}{n^2+n+2} + \dots + \frac{n}{n^2+n+n}\right)\]
    \begin{solution}
        由放缩法,
        \[
        \frac{1 + 2 + \dots + n}{n^2+n+n} < \sum_{k=1}^{n} \frac{k}{n^2+n+k} < \frac{1 + 2 + \dots + n}{n^2+n+1}
        \]
        即
        \[
        \frac{n(n+1)}{2(n^2+2n)} < \sum_{k=1}^{n} \frac{k}{n^2+n+k} < \frac{n(n+1)}{2(n^2+n+1)}
        \]
        由于
        \[
        \lim_{n \to \infty} \frac{n(n+1)}{2(n^2+2n)} =
        \lim_{n \to \infty} \frac{n(n+1)}{2(n^2+n+1)} = \frac{1}{2}
        \]
        由夹挤定理得
        \[
        \lim_{n \to \infty} \sum_{k=1}^{n} \frac{k}{n^2+n+k} = \frac{1}{2}
        \]
    \end{solution}

    \question 求极限
    \[
    \lim_{n \to \infty} \frac{1}{n^2} \sum_{k=1}^{n} \sqrt{k^2+n+k}
    \]
    \begin{solution}
        对于整数$1 \le k \le n$,有
        \[
        \sqrt{k^2} < \sqrt{k^2+n+k} < \sqrt{k^2+2n}
        \]
        于是
        \[
        \frac{1}{n^2} \sum_{k=1}^{n} k < \frac{1}{n^2} \sum_{k=1}^{n} \sqrt{k^2+n+k} \le \frac{1}{n^2} \sum_{k=1}^{n} (k+\sqrt{2n})
        \]
        由于
        \[
        \lim_{n \to \infty} \frac{1}{n^2} \sum_{k=1}^{n} k = \lim_{n \to \infty} \frac{n(n+1)}{2n^2} = \frac{1}{2}
        \]
        且
        \begin{align*}
        \lim_{n \to \infty} \frac{1}{n^2} \sum_{k=1}^{n} (k+\sqrt{2n}) &= \lim_{n \to \infty} \frac{1}{n^2} \sum_{k=1}^{n} k + \lim_{n \to \infty} \frac{1}{n^2} \sum_{k=1}^{n} \sqrt{2n} \\
        &= \frac{1}{2} + \lim_{n \to \infty} \frac{n \sqrt{2n}}{n^2} \\
        &= \frac{1}{2}
        \end{align*}
        由夹挤定理,
        \[
        \lim_{n \to \infty} \frac{1}{n^2} \sum_{k=1}^{n} \sqrt{k^2+n+k} = \frac{1}{2}
        \]
    \end{solution}

    \question 求极限
    \[\lim_{n \to \infty} \left(\frac{1}{n+1} + \frac{1}{n+\sqrt{2}} + \frac{1}{n+\sqrt{3}} + \dots + \frac{1}{n+\sqrt{n}}\right)\]
    \begin{solution}
        由放缩法得
        \[
        \frac{n}{n+\sqrt{n}} < \sum_{k=1}^{n} \frac{1}{n+\sqrt{k}} < \frac{n}{n+1}
        \]
        由于
        \[
        \lim_{n \to \infty} \frac{n}{n+\sqrt{n}} = \lim_{n \to \infty} \frac{1}{1 + 1/n} = 1
        \]
        由夹挤定理,
        \[
        \lim_{n \to \infty} \left(\frac{1}{n+1} + \frac{1}{n+\sqrt{2}} + \dots + \frac{1}{n+\sqrt{n}}\right) = 1
        \]
    \end{solution}

    \question 求极限
    \[
    \lim_{n \to \infty} \frac{n^k \sin^2(n!)}{n+2}, \quad 0<k<1
    \]
    \begin{solution}
        由于 
        \[
        0 \le \sin^2 x \le 1, \quad \forall x \in \mathbb{R}
        \]
        得到不等式
        \[
        0 \le \frac{n^k \sin^2(n!)}{n+2} \le \frac{n^k}{n+2}
        \]
        当 $0<k<1$ 时,上界的极限为
        \[
        \lim_{n \to \infty} \frac{n^k}{n+2} = 0
        \]
        故由夹挤定理,
        \[
        \lim_{n \to \infty} \frac{n^k \sin^2(n!)}{n+2} = 0
        \]
    \end{solution}

    \question 求极限
    \[
    \lim_{n \to \infty} \sqrt[n]{3^n + (2|\sin(n^n)|)^n}
    \]
    \begin{solution}
        由于 
        \[
        0 \le |\sin(n^n)| \le 1
        \]
        得到上下界:
        \[
        \sqrt[n]{3^n} \le \sqrt[n]{3^n + (2|\sin(n^n)|)^n} \le \sqrt[n]{3^n + 2^n} \le \sqrt[n]{2 \cdot 3^n}
        \]
        由于
        \[
        \lim_{n \to \infty} \sqrt[n]{3^n} = 
        \lim_{n \to \infty} \sqrt[n]{2 \cdot 3^n}  = 3
        \]
        由夹挤定理,
        \[
        \lim_{n \to \infty} \sqrt[n]{3^n + (2|\sin(n^n)|)^n} = 3
        \]
    \end{solution}

    \question 已知数列 $\{a_n\}$ 满足
    \[
    \frac{2n^2-7}{4n+5} < a_n < \frac{3n^2+8}{6n-1}, \quad n \in \mathbb{Z}^+
    \]
    求极限
    \[
    \lim_{n \to \infty} \frac{3 n a_n}{(n+1)^2}
    \]
    \begin{solution}
        将不等式两边同时乘以 $\dfrac{3n}{(n+1)^2}$:
        \[
        \frac{3n}{(n+1)^2} \cdot \frac{2n^2-7}{4n+5} < \frac{3 n a_n}{(n+1)^2} < \frac{3n}{(n+1)^2} \cdot \frac{3n^2+8}{6n-1}
        \]
        取极限得
        \[
        \frac{3}{2}=\lim_{n \to \infty} \frac{3n(2n^2-7)}{(4n+5)(n+1)^2} \le \lim_{n \to \infty} \frac{3 n a_n}{(n+1)^2} \le \lim_{n \to \infty} \frac{3n(3n^2+8)}{(6n-1)(n+1)^2} =\frac{3}{2}
        \]
        由夹挤定理,
        \[
        \lim_{n \to \infty} \frac{3 n a_n}{(n+1)^2} = \frac{3}{2}
        \]
    \end{solution}

    \question 已知 $x_{1}=1$,且对每个正整数 $n\ge 2$,
    \[
    n(x_{n})^{2}-x_{n-1}-n=0,\quad x_{n}\ge 0.
    \]
    求 $\displaystyle\lim_{n\to\infty} x_{n}$ 或证明 $\{x_{n}\}$ 发散。
    \begin{solution}
        由等式得对所有 $n\ge 2$,有
        \[
        x_n=\sqrt{1+\frac{x_{n-1}}{n}}
        \]
        首先证明$x_n \le 2$:有 $x_1=1\le 2$。假设 $n\ge 2$ 有 $x_n\le 2$成立,则
        \[
        x_n=\sqrt{1+\frac{x_{n}}{n+1}}\le \sqrt{1+2} <2
        \]
        故由数学归纳法得对所有 $n\ge 2$ 都皆有 $x_n\le 2$,故
        \[
        1 \le x_n=\sqrt{1+\frac{x_{n-1}}{n}}\le \sqrt{1+\frac{2}{n}}
        \]
        由夹挤定理,
        \[
        \lim_{n\to\infty} x_{n} = 1
        \]
    \end{solution}

    \question 
    \begin{parts}
    \part 证明
    \[
    \int_{0}^{1} \left(1+\sin \frac{\pi}{2}x\right)^n dx > \frac{2^{n+1}-1}{n+1} \quad (n=1,2,\dots)
    \]
    \begin{solution}
        由 
        \[
        \sin x \ge \frac{2}{\pi}x, \quad 0 \le x \le \frac{\pi}{2}
        \]
        得
        \[
        \sin \left( \frac{\pi}{2}x \right) \ge x, \quad 0 \le x \le 1,
        \]
        从而
        \[
        \int_0^1 \left(1 + \sin \frac{\pi}{2}x\right)^n dx \ge \int_0^1 (1 + x)^n dx = \left. \frac{(1+x)^{n+1}}{n+1} \right|_0^1 = \frac{2^{n+1} - 1}{n+1}.
        \]
        因此原不等式成立。
    \end{solution}
    \part 求极限:
    \[
    \lim_{n \to \infty} \left[\int_{0}^{1} \left(1+\sin \frac{\pi}{2}x\right)^n dx\right]^{\frac{1}{n}}
    \]
    \begin{solution}
        注意到:
        \[
        \frac{2^n}{n+1} < \frac{2^{n+1}-1}{n+1} \le \int_0^1 \left(1 + \sin \frac{\pi}{2}x\right)^n dx \le \int_0^1 2^n dx = 2^n.
        \]
        对上述不等式取 $n$ 次方根,得:
        \[
        2(n+1)^{-\frac{1}{n}} < \left[\int_0^1 \left(1 + \sin \frac{\pi}{2}x\right)^n dx\right]^{\frac{1}{n}} \le 2.
        \]
        因为 $(n+1)^{-\frac{1}{n}} \to 1$,由夹挤定理可得:
        \[
        \lim_{n \to \infty} \left[\int_0^1 \left(1 + \sin \frac{\pi}{2}x\right)^n dx\right]^{\frac{1}{n}} = 2
        \]
    \end{solution}
    \end{parts}
    
    \question 
    \begin{parts}
    \part 设 $n$ 是正整数, 计算 $$\int_{0}^{n\pi} x \sin^2 x \,dx$$
    \begin{solution}
        对非负整数 $k$,
        \[
        \int_{k\pi}^{(k+1)\pi} x \sin^2 x \,dx
        = \int_0^\pi (k\pi + x) \sin^2 x \,dx
        \]
        利用恒等式 \( \sin^2 x = \frac{1 - \cos 2x}{2} \),得:
        \[
        = \frac{1}{2} \int_0^\pi (k\pi + x)(1 - \cos 2x)\,dx
        = \frac{1}{2} \left( (k\pi + x)\big|_0^\pi - \int_0^\pi (k\pi + x)\cos 2x\,dx \right)
        \]
        由于 \( \int_0^\pi \cos 2x\,dx = 0 \) 且 \( \int_0^\pi x\cos 2x\,dx = 0 \),
        所以最后只剩:
        \[
        \int_{k\pi}^{(k+1)\pi} x \sin^2 x\,dx 
        = \frac{1}{2} \int_0^\pi (k\pi + x)\,dx 
        = \frac{1}{2} \left( k\pi^2 + \frac{\pi^2}{2} \right) = \frac{\pi^2(2k+1)}{4}          
        \]
        所以,
        \[
        \int_0^{n\pi} x \sin^2 x \,dx = \sum_{k=0}^{n-1} \frac{\pi^2(2k+1)}{4} = \frac{\pi^2}{4} \sum_{k=0}^{n-1} (2k+1) = \frac{\pi^2}{4} \cdot n^2
        \]
        因为 \( \displaystyle\sum_{k=0}^{n-1}(2k+1) = n^2 \),所以原式为
        \[
        \frac{\pi^2 n^2}{4}
        \]
    \end{solution}
    
    \part 证明对任何正实数 $p$, 函数极限 $$\lim_{x\to+\infty} \frac{1}{x^2} \int_{0}^{x} t |\sin t|^p \,dt$$ 存在.
    \begin{solution}
        记
        \[
        S_0 = \int_0^\pi |\sin t|^p \,dt, \qquad S_1 = \int_0^\pi t |\sin t|^p \,dt.
        \]
        对任意非负整数 \( k \),有
        \[
        \int_{k\pi}^{(k+1)\pi} t |\sin t|^p \,dt = \int_0^\pi (k\pi + x) |\sin x|^p dx = k\pi S_0 + S_1.
        \]
        若设 \( x = (n + \alpha)\pi \) 且 \( 0 \le \alpha < 1 \),则
        \[
        \int_0^x t |\sin t|^p dt = \sum_{k=0}^{n-1} (k\pi S_0 + S_1) + \int_{n\pi}^{x} t |\sin t|^p dt.
        \]
        利用估计(注意 \( \int_{n\pi}^x t |\sin t|^p dt \le (n+1)\pi \cdot S_0 \)),得上下界:
        \[
        \frac{\frac{1}{2}n(n-1)\pi S_0 + nS_1}{(n+1)^2\pi^2} \le \frac{1}{x^2} \int_0^x t |\sin t|^p dt \le \frac{\frac{1}{2}n(n+1)\pi S_0 + (n+1)S_1}{n^2\pi^2}.
        \]
        随着 \( n \to \infty \),上下界极限均趋于 \( \frac{S_0}{2\pi} \),所以由夹挤定理知极限存在,且为
        \[
        \lim_{x \to +\infty} \frac{1}{x^2} \int_0^x t |\sin t|^p dt = \frac{S_0}{2\pi}
        \]
    \end{solution}
    \end{parts}
    
    \question 
    \begin{parts}
    \part 求 $$\int_{0}^{\pi} \frac{1}{1+\cos^2 x} dx$$ 及$$\int_{0}^{\pi} \frac{\sin^2 x}{1+\cos^2 x} dx$$
    \begin{solution}
        设
        \[
        I_1 = \int_0^\pi \frac{1}{1 + \cos^2 x} \, dx, \quad
        I_2 = \int_0^\pi \frac{\sin^2 x}{1 + \cos^2 x} \, dx.
        \]
        发现
        \[
        I_1 = \frac12\int_0^\pi \frac{1+\sin^2 x +\cos^2 x}{1 + \cos^2 x} \, dx =\frac12 (I_2 + \pi)\Rightarrow I_2 = 2I_1 -\pi
        \]
        所以只需计算 \( I_1 \)即可得出两个积分。发现被积函数在$x=\dfrac\pi2$处不连续,将$I_1$写成\[
        I_1 = \int _{0}^{\frac\pi2}\frac{1}{1+\cos ^{2}x}dx+\int _{\frac\pi2}^{\pi }\frac{1}{1+\cos ^{2}x}dx 
        \]
        其中第一个积分变为
        \[
        \int _{0}^{\frac\pi2}\frac{\sec ^{2}x}{2+\tan ^{2}x}dx=\int _{0}^{\infty }\frac{1}{2+u^{2}}du=\left[\frac{1}{\sqrt{2}}\arctan \left(\frac{u}{\sqrt{2}}\right)\right]_{0}^{\infty }=\frac{1}{\sqrt{2}}\left(\frac{\pi }{2}-0\right)=\frac{\pi }{2\sqrt{2}}.
        \]
        对于第二个积分如下,设 \(x=\pi -y,dx=-dy\),
        \[
        \int _{\frac\pi2}^{\pi }\frac{1}{1+\cos ^{2}x}dx=\int _{\frac\pi2}^{0}\frac{1}{1+\cos ^{2}(\pi -y)}(-dy)=\int _{0}^{\frac\pi2}\frac{1}{1+\cos ^{2}y}dy=\frac{\pi }{2\sqrt{2}}
        \]
        所以
        \[
        I_1=\frac{\pi }{2\sqrt{2}}+\frac{\pi }{2\sqrt{2}}=\frac{\pi \sqrt{2}}{2}
        \]
        \[
        I_2 = 2I_1 -\pi = 2\cdot \frac{\pi \sqrt{2}}{2}-\pi = \pi (\sqrt2-1)
        \]
    \end{solution}

    \part 证明 
    \[
    \lim_{x \to \infty} \frac{\displaystyle \int_{0}^{x} \frac{\sin^2 t}{1+\cos^2 t} dt}{\displaystyle\int_{0}^{x} \frac{1}{1+\cos^2 t} dt} = 2-\sqrt{2}
    \]
    \begin{solution}
        取 $n = \left\lfloor \dfrac{x}{\pi} \right\rfloor$,则
        \[
        \int_0^x \frac{1}{1+\cos^2 t}\,dt = \sum_{k=1}^n \int_{(k-1)\pi}^{k\pi} \frac{1}{1+\cos^2 t}\, dt +\int_{k\pi}^{x}\frac{1}{1+\cos^2 t}\, dt
        \]
        于是
        \[
        \frac{n\pi}{\sqrt{2}} \le \int_0^x \frac{1}{1+\cos^2 t}\,dt \le \frac{(n+1)\pi}{\sqrt{2}}
        \]
        同理
        \[
        n\pi(\sqrt2-1) \le \int_0^x \frac{1}{1+\cos^2 t}\,dt \le (n+1)\pi(\sqrt2-1)
        \]
        于是
        \[
        \frac{n}{n+1}(2-\sqrt{2}) \le \frac{\displaystyle \int_{0}^{x} \frac{\sin^2 t}{1+\cos^2 t} dt}{\displaystyle\int_{0}^{x} \frac{1}{1+\cos^2 t} dt} \le \frac{n+1}{n}(2-\sqrt{2}),
        \]
        令 $x \to \infty$ 便得结论。
    \end{solution}
    \end{parts}

    \question 对每个正整数 $n$,设
    \[
    R_n = \{(x,y) \mid 0 \le x \le n, \ 0 \le y \le \sqrt{x}\},
    \]
    记 $N(n)$ 为 $R_n$ 中坐标都是整数的点的个数。求
    \[
    \lim_{n\to\infty} \frac{N(n)}{n^{\frac{3}{2}}}.
    \]
    \begin{solution}
        对每个正整数 $n$,满足 $0 \le y \le \sqrt{k}$ 的整数 $y$ 个数为 $[\sqrt{k}]+1$,所以
        \[
        N(n) = \sum_{k=0}^{n} ([\sqrt{k}]+1) = (n+1)+\sum_{k=1}^{n} [\sqrt{k}]
        \]
        由于 $\sqrt{k}-1 < [\sqrt{k}] \le \sqrt{k}$,并且
        \[
        \int_{0}^{n} \sqrt{x}\,dx \le \sum_{k=1}^{n} [\sqrt{k}] \le \int_{0}^{n+1} \sqrt{x}\,dx
        \]
        所以
        \[
        1 + \int_{0}^{n} \sqrt{x}\,dx \le N(n) \le (n+1) + \int_{0}^{n+1} \sqrt{x}\,dx
        \]
        即
        \[
        1 + \frac{2}{3} n^{\frac{3}{2}} \le N(n) \le (n+1) + \frac{2}{3} (n+1)^{\frac{3}{2}}
        \]
        由夹挤定理,得到
        \[
        \lim_{n\to\infty} \frac{N(n)}{n^{\frac{3}{2}}} = \frac{2}{3}
        \]
    \end{solution}
\end{questions}

\pagebreak

\begin{center}
  {\fontsize{30pt}{26pt}\selectfont
    \hypertarget{微分}{微分} \label{微分}
  }
\end{center}
\separator
\vspace{1pt}

\begin{questions}
    \question 已知 $f(x) = x^{2013} - x^{2012} + x^{2011} - x^{2010} + 1$,求 
    \[
    \lim_{h \to 0} \frac{f(1+h) - f(1-h)}{h}
    \]
    \begin{solution}
        计算导数:
        \[ 
        f'(x) = 2013x^{2012} - 2012x^{2011} + 2011x^{2010} - 2010x^{2009} 
        \]
        于是
        \begin{align*}
        \lim_{h \to 0} \frac{f(1+h) - f(1-h)}{h} 
        &= \lim_{h \to 0} \frac{f(1+h) - f(1)}{h} + \lim_{h \to 0} \frac{f(1) - f(1-h)}{h} \\
        &= f'(1) + f'(1) \\
        &= 2 \cdot (2013-2012+2011-2010) \\
        &= 4
        \end{align*}
    \end{solution}

    \question 已知 \( f'(1) = 12 \),求
    \[
    \lim_{h \to 0} \frac{f(1 + 4h) - f(1 - 2h)}{3h}
    \]
    \begin{solution}
        改写
        \begin{align*}
        \frac{f(1 + 4h) - f(1 - 2h)}{3h}
        &= \frac{f(1 + 4h) - f(1)}{3h} + \frac{f(1) - f(1 - 2h)}{3h} \\
        &= \frac{4}{3} f'(1) + \frac{2}{3} f'(1) \\
        &= 24
        \end{align*}
    \end{solution}

    \question 求
    \[
    \lim_{h \to 0} \left[ \int_{\frac{\pi}{6}}^{\frac{\pi}{6}+h} \frac{2\sqrt{\sin x}}{\pi h} \, dx \right]
    \]
    \begin{solution}
        先提取常数
        \[
        \lim_{h \to 0} \left[ \int_{\frac{\pi}{6}}^{\frac{\pi}{6}+h} \frac{2\sqrt{\sin x}}{\pi h} \, dx \right] 
        = \frac{2}{\pi} \lim_{h \to 0} \left[ \frac{1}{h} \int_{\frac{\pi}{6}}^{\frac{\pi}{6}+h} \sqrt{\sin x} \, dx \right]
        \]
        令
        \[
        F(x) = \int \sqrt{\sin x} \, dx \Rightarrow F'(x) = \sqrt{\sin x}
        \]
        则
        \[
        \frac{1}{h} \int_{\frac{\pi}{6}}^{\frac{\pi}{6}+h} \sqrt{\sin x} \, dx = \frac{F\left(\frac{\pi}{6}+h\right) - F\left(\frac{\pi}{6}\right)}{h}
        \]
        当 \(h \to 0\) 时,上式为导数定义:
        \[
        \lim_{h \to 0} \frac{F\left(\frac{\pi}{6}+h\right) - F\left(\frac{\pi}{6}\right)}{h} = F'\left(\frac{\pi}{6}\right) = \sqrt{\sin \frac{\pi}{6}}
        \]
        因此原极限为
        \[
        \frac{2}{\pi} \cdot \sqrt{\sin \frac{\pi}{6}} = \frac{2}{\pi} \cdot \frac{\sqrt{2}}{2} = \frac{\sqrt{2}}{\pi}
        \]
    \end{solution}

    \question 由导数定义,证明 
    \[
    \frac{d}{dx}(\sec x) = \sec x \tan x
    \]
    \begin{solution}
        由导数的定义,
        \begin{align*}
        \frac{d}{dx}(\sec x) 
        &= \lim_{h \to 0} \frac{\sec(x+h) - \sec x}{h}\\
        &= \lim_{h \to 0} \frac{\cos x - \cos(x+h)}{h \cos x \cos(x+h)}\\
        &= \lim_{h \to 0} \frac{2 \sin\left(x+\frac{h}{2}\right) \sin\frac{h}{2}}{h \cos x \cos(x+h)}\\
        &= \lim_{h \to 0} \left[ \frac{\sin\frac{h}{2}}{\frac{h}{2}} \cdot \frac{\sin\left(x+\frac{h}{2}\right)}{\cos x \cos(x+h)} \right] \\
        &= 1 \cdot \frac{\sin x}{\cos^2 x} \\
        &= \sec x \tan x
        \end{align*}
    \end{solution}

    \question 通过导数的定义,求 
    \[
    f(x) = \frac{1}{\sqrt{x^2-1}}, \quad |x|>1
    \]
    的导数$f'(x)$。
    \begin{solution}
        由导数的定义,
        \begin{align*}
        f'(x) &= \lim_{h \to 0} \frac{f(x+h) - f(x)}{h} \\
        &= \lim_{h \to 0} \frac{\frac{1}{\sqrt{(x+h)^2-1}} - \frac{1}{\sqrt{x^2-1}}}{h} \\
        &= lim_{h \to 0} \frac{\sqrt{x^2-1} - \sqrt{(x+h)^2-1}}{h \sqrt{x^2-1} \sqrt{(x+h)^2-1}} \\
        &= \lim_{h \to 0} \frac{(x^2-1) - ((x+h)^2-1)}{h \sqrt{x^2-1} \sqrt{(x+h)^2-1} (\sqrt{x^2-1} + \sqrt{(x+h)^2-1})} \\
        &= \lim_{h \to 0} \frac{-2xh - h^2}{h \sqrt{x^2-1} \sqrt{(x+h)^2-1} (\sqrt{x^2-1} + \sqrt{(x+h)^2-1})} \\
        &= \lim_{h \to 0} \frac{-2x - h}{\sqrt{x^2-1} \sqrt{(x+h)^2-1} (\sqrt{x^2-1} + \sqrt{(x+h)^2-1})} \\
        &= \frac{-2x}{\sqrt{x^2-1} \sqrt{x^2-1} (2\sqrt{x^2-1})} \\
        &= -\frac{x}{(x^2-1)^{\frac{3}{2}}}
        \end{align*}
    \end{solution}

    \question 设 
    \[
    y = \frac{(x^4 + 2)(x^3 - 2)}{(x + 2)(x^2 - 2)},
    \]
    求 $\dfrac{dy}{dx}$。
    \begin{solution}
        对等式两边取自然对数,
        \[ 
        \ln y = \ln(x^4 + 2) + \ln(x^3 - 2) - \ln(x + 2) - \ln(x^2 - 2) 
        \]
        对 $x$ 求导,
        \[ 
        \frac{1}{y} \frac{dy}{dx} = \frac{4x^3}{x^4 + 2} + \frac{3x^2}{x^3 - 2} - \frac{1}{x + 2} - \frac{2x}{x^2 - 2} 
        \]
        整理得
        \[ 
        \frac{dy}{dx} = \frac{4x^3(x^3 - 2)}{(x + 2)(x^2 - 2)} + \frac{3x^2(x^4 + 2)}{(x + 2)(x^2 - 2)} - \frac{(x^4 + 2)(x^3 - 2)}{(x + 2)^2(x^2 - 2)} - \frac{2x(x^4 + 2)(x^3 - 2)}{(x + 2)(x^2 - 2)^2} 
        \]
    \end{solution}

    \question 已知曲线方程
    \[
    y = 2^{3e^{2x}}, \quad x \in \mathbb{R}.
    \]
    以$y$ 表示 $\frac{dy}{dx}$。
    \begin{solution}
        由链导法,
        \[
        \frac{dy}{dx} = 2^{3e^{2x}} \cdot \ln 2 \cdot 6 e^{2x} = = y \cdot \ln 2 \cdot 6 e^{2x}
        \]
        又由原式取自然对数得
        \[
        \ln y = 3 e^{2x} \ln 2 \Rightarrow 2 \ln y = 6 e^{2x} \ln 2
        \]
        因此得到
        \[
        \frac{dy}{dx} = 2y \ln y
        \]
    \end{solution}
    \begin{solution}
        不如一开始就取$\ln$,
        \[
        \ln y = 3 e^{2x} \ln 2.
        \]
        对 $x$ 求导,
        \[
        \frac{1}{y}\frac{dy}{dx} = \ln 2 \cdot 6 e^{2x} 
        \]
        同样代入 $\ln y = 3 e^{2x} \ln 2$ 得
        \[
        \frac{dy}{dx} = = y \cdot (\ln 2) \cdot 6 e^{2x} = 2 y \ln y
        \]
    \end{solution}

    \question 已知曲线由方程
    \[
    x^m y^n = (x+y)^{m+n}, \quad x\neq 0,\; y\neq 0,\; x+y\neq 0,\; my-nx\neq 0
    \]
    隐式定义,其中 $m,n$ 为有理数。证明
    \[
    \frac{dy}{dx} = \frac{y}{x}
    \]
    \begin{solution}
        两边取对数,
        \[
        m\ln x + n\ln y = (m+n)\ln(x+y)
        \]
        对 $x$ 求导,
        \[
        \frac{m}{x} + \frac{n}{y} \frac{dy}{dx} = \frac{m+n}{x+y}\left(1+\frac{dy}{dx}\right)
        \]
        整理得
        \[
        \frac{my-nx}{x(x+y)} = \frac{my-nx}{y(x+y)} \frac{dy}{dx}
        \]
        由于$x+y\neq 0, my-nx\neq 0$,故
        \[
        \frac{dy}{dx} = \frac{y}{x}
        \]
    \end{solution}

    \question 已知曲线 $C$ 的隐函数方程为
    \[
    y = xe^y, \quad x \neq 0, \; y \neq 1, \; y \neq 2.
    \]
    证明
    \[
    (1-y)\frac{d^2 y}{dx^2} = (2-y)\left(\frac{dy}{dx}\right)^2.
    \]
    \begin{solution}
        对 $y = xe^y$ 两边求导,
        \[
        \frac{dy}{dx} = e^y + x e^y \frac{dy}{dx} \Rightarrow \frac{dy}{dx}(1 - y) = e^y 
        \]
        再次求导,
        \[
        \frac{d^2 y}{dx^2}(1-y) - \left(\frac{dy}{dx}\right)^2 = e^y \frac{dy}{dx}
        \]
        代入 $e^y = \dfrac{dy}{dx}(1-y)$,
        \[
        \frac{d^2 y}{dx^2}(1-y) - \left(\frac{dy}{dx}\right)^2 = \left(\frac{dy}{dx}\right)^2 (1-y)
        \]
        即 
        \[
        \frac{d^2 y}{dx^2}(1-y) = \left(\frac{dy}{dx}\right)^2 (2-y)
        \]
    \end{solution}

    \question 求曲线 $C$ 
    \[
    x^2 + 3xy - 2y^2 + 17 = 0
    \]
    的极值。
    \begin{solution}
        设 $F(x, y) = x^2 + 3xy - 2y^2 + 17$。对方程$F(x, y)=0$两边求导,
        \[
        2x + 3y + 3x y' - 4y y' = 0
        \]
        整理得:
        \[
        y' = -\frac{2x + 3y}{3x - 4y}
        \]
        令 $y' = 0$,得到 
        \[
        y = -\frac{2}{3}x
        \]
        代入原方程解得
        \[
        x^2 + 3x\left(-\frac{2}{3}x\right) - 2\left(-\frac{2}{3}x\right)^2 + 17 = 0 \Rightarrow x_1 = 3, x_2 = -3
        \]
        得临界点$P_1(3, -2),P_2(-3, 2)$。再求导得,
        \[
        y'' (3x - 4y) + y' (3 - 4y') = -(2 + 3y')
        \]
        在临界点处 $y' = 0$,化简得
        \[
        y'' (3x - 4y) = -2 \Rightarrow y'' = \frac{-2}{3x - 4y}
        \]
        由于
        \[
        y''(3, -2) = \frac{-2}{3(3) - 4(-2)}= -\frac{2}{17} < 0, \quad y''(-3, 2) = \frac{-2}{3(-3) - 4(2)} = \frac{2}{17} > 0
        \]
        故 $y_1 = -2$ 是极大值,$y_2 = 2$ 是极小值。该曲线在 $x=3$ 处取得极大值 $-2$,在 $x=-3$ 处取得极小值 $2$。
    \end{solution}

    \question 设函数 \( f(x) \) 处处可导,且 \( f'(0) = 1 \),并对任意实数 \( x,h \) 恒有
    \[
    f(x + h) = f(x) + f(h) + 2hx,
    \]
    求 \( f'(x) \)。
    \begin{solution}
        已知$\forall \:x,h \in \mathbb{R}$,
        \[
        f(x + h) = f(x) + f(h) + 2hx
        \]
        对$h$微分,
        \[
        f'(x+h)=f'(h)+2x
        \]
        令$h=0$,则
        \[
        f'(x)=f'(0)+2x=1+2x
        \]
    \end{solution}

    \question 设 $f(x)=x(x-1)(x-2)\cdots(x-2023),g(x)=f(f(x))$,求 $g'(1)$。
    \begin{solution}
        求导得
        \[
        f'(x) = (x-1)P(x) + x(x-2)(x-3)\cdots(x-2023) = xQ(x) + (x-1)(x-2)\cdots(x-2023)
        \]
        所以有
        \[
        f(1) = 0,\quad f'(1) = 2022!,\quad f'(0) = -2023!
        \]
        由于 $g(x) = f(f(x))$,有
        \[
        g'(x) = f'(f(x)) \cdot f'(x)\Rightarrow g'(1) = f'(f(1)) \cdot f'(1) = f'(0) \cdot 2022! = -2023! \cdot 2022!
        \]
    \end{solution}

    \question 设 $f(x)$ 为正实函数且为可微分函数,对任意实数 $x,y$,满足 $f(x+y)=2f(x)f(y)$,若 $f'(0)=2$,试求 $\dfrac{f''(x)}{f(x)}$。
    \begin{solution}
        由$f(x+y)=2f(x)f(y)$代入 $x=y=0$得
        \[
        f(0)=2f(0)f(0) \Rightarrow f(0) = \frac{1}{2}
        \]
        考虑 $f'(x)$,
        \begin{align*}
        f'(x)&=\lim_{h\to 0}\frac{f(x+h)-f(x)}{h} 
        = \lim_{h\to 0} \frac{2f(x)f(h)-f(x)}{h}  \\      
        &= 2f(x)\lim_{h\to 0} \frac{f(h)-f(0)}{h}
        = 2f(x)f'(0) = 2f(x) \cdot 2 = 4f(x)
        \end{align*}
        对两边求导得
        \[
        f''(x) = 4f'(x) = 4 \cdot 4f(x) 
        \Rightarrow \frac{f''(x)}{f(x)} = 16
        \]
    \end{solution}

\question
已知 \( f(x) \) 在 \( x = 0 \) 连续,且
\[
\lim_{x \to 0} \frac{\ln (f(x) + 2)}{x - \sin x} = 1,
\]
求 \( f'(0) \)。
    
\begin{solution}
    因为 \( f(x) \) 在 \( x = 0 \) 连续,所以
    \[
    \lim_{x \to 0} \ln(f(x)+2) = \ln(f(0)+2).
    \]
    又因 \( x - \sin x \to 0 \),分母趋于 0,极限存在,需有分子也趋于 0,即
    \[
    \ln(f(0)+2) = 0 \Rightarrow f(0) = -1.
    \]
    
    所以
    \[
    \lim_{x \to 0} \frac{\ln(f(x)+2)}{x - \sin x}
    = \lim_{x \to 0} \frac{f(x)+1}{x - \sin x} \quad (\text{因 } \ln(1 + y) \backsim y).
    \]
    又 \( x - \sin x \backsim \frac{x^3}{6} \),于是
    \[
    \lim_{x \to 0} \frac{f(x)+1}{x^3} = \frac{1}{1/6} = 6.
    \]
    因此
    \[
    \lim_{x \to 0} \frac{f(x) - f(0)}{x} = \lim_{x \to 0} \frac{f(x)+1}{x} = \lim_{x \to 0} \frac{f(x)+1}{x^3} \cdot x^2 = 6 \cdot 0 = 0.
    \]
    故 \( f'(0) = \boxed{0} \)。
    \textcolor{red}{待验证}
\end{solution}

    \question 已知 $a,b$ 为实数,若函数 $f(x) = 2ax^{3} - 3ax^{2} + b$ 在 $0 \le x \le 2$ 时有最大值 $7$,最小值 $-3$,求数对 $(a,b)$ 。
    \begin{solution}
        导数为
        \[
        f'(x) = 6ax^2 - 6ax = 6ax(x-1)
        \]
        临界点为 $x = 0, 1$,二阶导为
        \[
        f''(x) = 12ax - 6a = 6a(2x - 1)
        \]
        当$a > 0$,则
        \[
        f(x)\begin{cases}
        \text{递增} & x \ge 1 \\
        \text{递减} & 0 \le x \le 1 \\
        \text{递增} & x \le 0
        \end{cases}
        \]
        极小值在 $x = 1$,极大值在 $x = 2$ 或 $x = 0$,解得
        \[
        \begin{cases}
        f(2) = 4a + b = 7 \\
        f(1) = -a + b = -3
        \end{cases}
        \Rightarrow a = 2, b = -1
        \]
        当$a < 0$,则
        \[
        f(x)\begin{cases}
        \text{递减} & x \ge 1 \\
        \text{递增} & 0 \le x \le 1 \\
        \text{递减} & x \le 0
        \end{cases}
        \]
        极大值在 $x = 1$,极小值在 $x = 0$ 或 $x = 2$,同理解得
        \[
        \begin{cases}
        f(1) = -a + b = 7 \\
        f(2) = 4a + b = -3
        \end{cases}
        \Rightarrow a = -2,b = 5
        \]
        $\therefore (a, b) = (2, -1) \ \text{或} \ (-2, 5)$
    \end{solution}

    \question 函数 $f$ 定义为
    \[
    f(n, y) = \sum_{x=1}^{n} \frac{x^2 y^x}{k}, \quad n \in \mathbb{N}, \quad y \in \mathbb{R}
    \]
    其中
    \[
    k = \sum_{r=1}^{n} r^2,
    \]
    若已知数列求和公式
    \[
    \sum_{r=1}^{n} r^4 = \frac{1}{30} n(n+1)(6n^3+9n^2+n-1),
    \]
    证明
    \[
    \left. \frac{d^2f}{dy^2} \right|_{y=1} + \left. \frac{df}{dy} \right|_{y=1} - \left[ \left. \frac{df}{dy} \right|_{y=1} \right]^2 = \frac{3n^4+6n^3-n^2-4n}{20(2n+1)^2}.
    \]
    \begin{solution}
        首先,利用标准求和公式:
        \[
        k = \sum_{x=1}^{n} x^2 = \frac{n(n+1)(2n+1)}{6}.
        \]
        既然 $k$ 与 $x$ 无关,可以将其作为常数因子提取到求和号外
        \[
        f(n,y) = \frac{1}{k} \sum_{x=1}^{n} x^2 y^x.
        \]
        对 $y$ 求导,
        \[
        \frac{df}{dy} = \frac{1}{k} \sum_{x=1}^{n} x^3 y^{x-1}, \quad
        \left. \frac{df}{dy} \right|_{y=1} = \frac{1}{k} \sum_{x=1}^{n} x^3.
        \]
        再次对 $y$ 求导,
        \[
        \frac{d^2f}{dy^2} = \frac{1}{k} \sum_{x=1}^{n} x^3 (x-1) y^{x-2} = \frac{1}{k} \sum_{x=1}^{n} x^4 y^{x-2} - \frac{1}{k} \sum_{x=1}^{n} x^3 y^{x-2}.
        \]
        在 $y=1$ 时,
        \[
        \left. \frac{d^2f}{dy^2} \right|_{y=1} + \left. \frac{df}{dy} \right|_{y=1} - \left[ \left. \frac{df}{dy} \right|_{y=1} \right]^2
        = \frac{1}{k} \sum_{x=1}^{n} x^4 - \frac{1}{k^2} \left( \sum_{x=1}^{n} x^3 \right)^2.
        \]
        代入求和公式得
        \begin{align*}
        & \frac{6}{n(n+1)(2n+1)} \cdot \frac{1}{30} n(n+1)(6n^3+9n^2+n-1) - \frac{36}{n^2(n+1)^2(2n+1)^2} \cdot \left[ \frac{n^2(n+1)^2}{4} \right] \\
        &= \frac{6n^3+9n^2+n-1}{5(2n+1)} - \frac{9}{4(2n+1)^2} \\
        &= \frac{3n^4+6n^3-n^2-4n}{20(2n+1)^2}.
        \end{align*}
        因此,
        \[
        \left. \frac{d^2f}{dy^2} \right|_{y=1} + \left. \frac{df}{dy} \right|_{y=1} - \left[ \left. \frac{df}{dy} \right|_{y=1} \right]^2 = \frac{3n^4+6n^3-n^2-4n}{20(2n+1)^2},
        \]
        得证。
    \end{solution}

    \question 
    \begin{parts}
    \part 设 
    \[
    f(x) = x^3 + x,
    \]
    且$g(x)$ 为 $f(x)$ 的反函数,求 $g'(10)$。
    \begin{solution}
        由于 $f$ 与 $g$ 为反函数,$f(g(x)) = x$,对两边求导得到
        \[
        f'(g(x)) g'(x) = 1
        \]
        又因为 $f(2) = 10$,所以 $g(10) = 2$。令 $x = 10$,则
        \[
        f'(2) g'(10) = 1
        \]
        而 $f'(x) = 3x^2 + 1$,所以 $f'(2) = 13$,因此
        \[
        g'(10) = \frac{1}{13}
        \]
    \end{solution}
    \part 对 $x>0$,定义 
    \[
    h(x) = \frac{1}{f(x)},
    \]
    证明函数 $f(x)+h(x)$ 在 $x = g(1)$ 处取得最小值。
    \begin{solution}
        令 
        \[
        F(x) = f(x) + h(x) = f(x) + \frac{1}{f(x)}
        \] 
        则
        \[
        F'(x) = f'(x) - \frac{f'(x)}{[f(x)]^2} = f'(x) \frac{[f(x)]^2 - 1}{[f(x)]^2}
        \] 
        因为 $f'(x) > 1$,所以在临界点有 $f(x_0)^2 - 1 = 0$,即 $f(x_0) = 1$,因此 $x_0 = g(1)$,再次求导,
        \[
        F''(x) = f''(x) - \frac{[f(x)]^2 f''(x) - 2 f(x) [f'(x)]^2}{[f(x)]^4}
        \] 
        在 $x = x_0$ 时,$f(x_0) = 1,f''(x_0) = 6 g(1),f'(x_0) = 3(g(1))^2 + 1$,所以
        \[
        F''(x_0) = 6g(1) - \big(6g(1) - 2 [3(g(1))^2 + 1]^2\big) = 2 [3(g(1))^2 + 1]^2 > 0
        \] 
        因此 $x = x_0$ 为极小值。由于
        \[
        \lim_{x \to 0^+} F(x) = \lim_{x \to \infty} F(x) = \infty,
        \] 
        故 $x = g(1)$ 为 $F(x)$ 的最小值。
    \end{solution}
    \end{parts}
    
    \question 已知对任意正整数 $k$,方程 
    \[
    x^{k}-x^{k-1}-x^{k-2}-\cdots-x-1=0
    \]
    恰有一个正实根,设该方程唯一正实根为 $\alpha_k$,试证该无穷数列 $\langle \alpha_k \rangle$ 收敛,且 $$\lim\limits_{k \to \infty} \alpha_k = 2$$
    \begin{solution}
        原方程可化为
        \[
        x^k - x^{k-1} - x^{k-2} - \cdots - x - 1 = x^k - (x^{k-1} + x^{k-2} + \cdots + 1)
        = x^k - \frac{x^k - 1}{x - 1}
        \]
        令
        \[
        f_k(x) = x^{k+1} - 2x^k + 1
        \]
        则 $\alpha_k$ 为 $f_k(x) = 0$ 在 $(1, 2)$ 内的唯一正实根(注意 $\alpha_k \ne 1$,当 $k > 1$),考虑
        \[
        f_k(0) = 1,\quad f_k(1) = 0,\quad f_k(2) = 1
        \]
        对 $f_k(x)$ 求导得
        \[
        f_k'(x) = (k+1)x^k - 2k x^{k-1} = x^{k-1}[(k+1)x - 2k]
        \]
        导函数零点为
        \[
        x = 0 \quad \text{或} \quad x = \frac{2k}{k+1} = 2 - \frac{2}{k+1}
        \]
        因此
        \[
        f_k(x) \text{ 在 } \left(0, 2 - \frac{2}{k+1} \right) \text{ 上递减,}
        \quad
        \left(2 - \frac{2}{k+1}, \infty\right) \text{ 上递增}
        \]
        由于 $f_k(1) = 0$, $f_k(2) = 1$ 且 $\alpha_k$ 是唯一正实根,结合单调性可得:
        \[
        2 - \frac{2}{k+1} < \alpha_k < 2
        \]
        故由夹挤定理,
        \[
        \lim_{k \to \infty} \alpha_k = 2
        \]
        因此 $\alpha_k$ 收敛,且收敛至 $2$,故得证。
    \end{solution}

    \question 已知 $a < b < c,f'(x)$ 在 $(a, c)$ 上严格递增, 且 $f(x)$ 在 $[a, c]$ 上连续, 证明
    \[
    (b-a) f(c) + (c-b) f(a) > (c-a) f(b).
    \]
    \begin{solution}
        由拉格朗日中值定理, 因此存在 $\alpha$ 和 $\beta$ 使得
        \[
        \frac{f(c)-f(b)}{c-b} = f'(\beta), \quad b < \beta < c
        \]
        以及
        \[
        \frac{f(b)-f(a)}{b-a} = f'(\alpha), \quad a < \alpha < b
        \]
        由于 $f'$ 严格递增, 我们有 $f'(\beta) > f'(\alpha)$, 因此
        \[
        \frac{f(c)-f(b)}{c-b} > \frac{f(b)-f(a)}{b-a}
        \]
        即
        \[
        (b-a) f(c) + (c-b) f(a) > (c-a) f(b)
        \]
    \end{solution}

    \question 设 $f,g:\mathbb{R}\to\mathbb{R}$ 是连续函数,且 $g$ 可导,若
    \[
    (f(0)-g'(0))(g'(1)-f(1))>0,
    \]
    证明存在实数 $c\in(0,1)$,使得 $f(c)=g'(c)$。
    \begin{solution}
        令
        \[
        F(x)=\int_0^x f(t)\,dt, \quad h(x)=F(x)-g(x).
        \]
        由 $f$ 的连续性,$F$ 可导且 $F'=f$,因此
        \[
        h'(x)=F'(x)-g'(x)=f(x)-g'(x).
        \]
        故由$(f(0)-g'(0))(g'(1)-f(1))>0$知
        \[
        h'(0)(-h'(1))>0,
        \]
        因此 $h'(0)$ 和 $h'(1)$ 异号。由达布定理,存在 $c\in(0,1)$使得
        \[
        h'(c)=0 \Rightarrow f(c)=g'(c).
        \]
    \end{solution}

    \question 设 $f: (0,\infty) \to \mathbb{R}$ 为连续可微函数,$b > a > 0$ 且 $f(a) = f(b) = k$。证明存在 $\xi \in (a, b)$ 使得
    \[
    f(\xi) - \xi f'(\xi) = k.
    \]
    \begin{solution}
        若考虑函数 $g(x) = \dfrac{f(x)}{x}$,则
        \[
        g'(x) = \frac{xf'(x) - f(x)}{x^2},
        \]
        其中分子与已知正好对应。考虑在 $[a,b]$ 上的函数 
        \[
        g(x) = \frac{f(x)}{x},\quad h(x) = \frac{1}{x}
        \]
        由柯西中值定理,存在 $\xi \in (a, b)$ 使得
        \[
        \frac{g(a) - g(b)}{h(a) - h(b)} = \frac{g'(\xi)}{h'(\xi)}
        \]
        注意到
        \[
        \frac{g'(\xi)}{h'(\xi)} = \frac{\frac{\xi f'(\xi) - f(\xi)}{\xi^2}}{-\frac{1}{\xi^2}} = f(\xi) - \xi f'(\xi)
        \]
        而
        \[
        \frac{g(a) - g(b)}{h(a) - h(b)} = \frac{\frac{f(a)}{a} - \frac{f(b)}{b}}{\frac{1}{a} - \frac{1}{b}} = k
        \]
        故原命题得证。
    \end{solution}

    \question 设 $f:\mathbb{R}\to\mathbb{R}$ 为二阶可导函数,且 $f(0)=0$。证明:存在
    $\xi\in\left(-\dfrac{\pi}{2},\dfrac{\pi}{2}\right)$使得
    \[
    f''(\xi)=f(\xi)\bigl(1+2\tan^2\xi\bigr).
    \]
    \begin{solution}
        设$g(x)=f(x)\cos x$,由于
        \[
        g\left(-\frac{\pi}{2}\right)=g(0)=g\left(\frac{\pi}{2}\right)=0
        \]
        由罗尔定理,存在
        \[
        \xi_1\in\left(-\frac{\pi}{2},0\right),\quad
        \xi_2\in\left(0,\frac{\pi}{2}\right)
        \]
        使得
        \[
        g'(\xi_1)=g'(\xi_2)=0
        \]
        考虑函数
        \[
        h(x)=\frac{g'(x)}{\cos^2 x}
        =\frac{f'(x)\cos x-f(x)\sin x}{\cos^2 x}
        \]
        显然有
        \[
        h(\xi_1)=h(\xi_2)=0
        \]
        再由罗尔定理,存在
        \[
        \xi\in(\xi_1,\xi_2)\subset\left(-\frac{\pi}{2},\frac{\pi}{2}\right)
        \]
        使得 $h'(\xi)=0$,$h$导数为
        \begin{align*}
        0=h'(\xi)
        &=\frac{g''(\xi)\cos^2\xi+2\cos\xi\sin\xi\,g'(\xi)}{\cos^4\xi} \\
        &=\frac{(f''(\xi)\cos\xi-2f'(\xi)\sin\xi-f(\xi)\cos\xi)\cos\xi
        +2\sin\xi\bigl(f'(\xi)\cos\xi-f(\xi)\sin\xi\bigr)}{\cos^3\xi} \\
        &=\frac{f''(\xi)\cos^2\xi-f(\xi)\bigl(\cos^2\xi+2\sin^2\xi\bigr)}{\cos^3\xi}
        \end{align*}
        整理得
        \[
        0=\frac{1}{\cos\xi}\Bigl(f''(\xi)-f(\xi)\bigl(1+2\tan^2\xi\bigr)\Bigr)
        \]
        因此得证
        \[
        f''(\xi)=f(\xi)\bigl(1+2\tan^2\xi\bigr)
        \]
    \end{solution}

    \question 证明当 $0 < \theta < \dfrac{\pi}{4}$ 时,有
    \[
    \tan \theta < \frac{4\theta}{\pi}.
    \]
    \begin{solution}
        设 
        \[
        f(x) = \tan x,\quad x = \theta \in \left(0, \frac{\pi}{4}\right)
        \]
        因为$f$ 在 $\left[0, \dfrac{\pi}{4}\right]$ 上连续,$\left(0, \dfrac{\pi}{4}\right)$ 上可导,由拉格朗日中值定理,存在
        \[
        0 < \xi < \theta,\theta < \xi' < \frac{\pi}{4}\]
        使得
        \[
        \sec^2 \xi = \frac{\tan \theta - \tan 0}{\theta - 0},\quad \sec^2 \xi' = \frac{\tan\frac{\pi}{4}-\tan \theta}{\frac{\pi}{4}-\theta}
        \]
        由于 $\sec^2 x$ 在 $\left(0, \dfrac{\pi}{4}\right)$ 上单调递增,因此 $\sec^2 \xi' > \sec^2 \xi$,即
        \[
        \frac{1 - \tan \theta}{\frac{\pi}{4} - \theta} > \frac{\tan \theta}{\theta} \Rightarrow \frac{4\theta}{\pi} > \tan \theta
        \]
    \end{solution}
    \begin{solution}
        令
        \[
        f(\theta) = \frac{4\theta}{\pi} - \tan \theta.
        \]
        则
        \[
        f'(\theta) = \frac{4}{\pi} - \sec^2 \theta
        \]
        因此可解得唯一的 $\xi \in \left(0, \dfrac{\pi}{4}\right)$ 使得 $f'(\xi) = 0$,又 
        \[
        f'(\theta) < 0,\; \forall \theta \in (0, \xi),\quad f'(\theta) > 0,\; \forall \theta \in \left(\xi, \frac{\pi}{4}\right)
        \]
        因此$f$ 在 $(0,\xi)$ 上严格递减,在 $\left(\xi, \dfrac{\pi}{4}\right)$ 上严格递增。又 $f(0) = f\left(\dfrac{\pi}{4}\right) = 0$,所以对所有 $\left(0, \dfrac{\pi}{4}\right)$ 皆有 $f(\theta) < 0$ ,即
        \[
        \tan \theta < \frac{4\theta}{\pi}, \quad 0 < \theta < \dfrac{\pi}{4}
        \]
    \end{solution}

    \question 证明对于所有非负整数 $n$,有
    \[
    2(3n-1)^n \ge (3n+1)^n
    \]
    \begin{solution}
        当 $n=0$ 时,不等式显然成立,因为
        \[
        2(3\cdot 0 - 1)^0 = 2 \cdot 1 = 2 \ge 1 = (3\cdot 0 + 1)^0.
        \]
        考虑 $n \ge 1$,命题等价于
        \[
        2(3n-1)^n \ge (3n+1)^n
        \Longleftrightarrow \left(\frac{3n-1}{3n+1}\right)^n \ge \frac{1}{2}
        \Longleftrightarrow n \ln \frac{3n-1}{3n+1} \ge -\ln 2
        \]
        定义函数
        \[
        f(x) = x \ln \frac{3x-1}{3x+1}, \quad x \ge 1.
        \]
        对 $f(x)$ 求导,
        \[
        f'(x) = \ln \frac{3x-1}{3x+1} + \frac{6x}{9x^2 -1},\quad f''(x) = \frac{6(3x^2-1)}{(9x^2-1)^2}
        \]
        当 $x \ge 1$ 时, $f''(x) > 0$, 故 $f'(x)$ 在 $[1, +\infty)$ 上严格递增。
        又因为
        \[
        \lim_{x \to +\infty} f'(x) = \lim_{x \to +\infty} \left( \ln \frac{3-1/x}{3+1/x} + \frac{6/x}{9-1/x^2} \right) = 0
        \]
        由 $f'(x)$ 严格递增且趋于 0 可知,在 $[1, +\infty)$ 上 $f'(x) < 0$。
        因此 $f(x)$ 在 $[1, +\infty)$ 上严格递减。
        根据其极限:
        \[
        \lim_{x \to +\infty} f(x) = \lim_{x \to +\infty} \ln \left( 1 - \frac{2}{3x+1} \right)^x = -\frac{2}{3}
        \]
        由于 $f(x)$ 严格递减且下界为 $-\frac{2}{3}$,而 $-\frac{2}{3} > -0.693 \approx -\ln 2$。
        故对于所有 $n \ge 1$, 有 $f(n) > -\frac{2}{3} > -\ln 2$, 原不等式成立。
        \textcolor{red}{(待验证)}
    \end{solution}

    \question 当 $x \in (0, \frac{\pi}{2})$,比较 $\tan(\sin x)$ 及 $\sin(\tan x)$ 的大小。
    \begin{solution}
        设
        \[
        f(x) = \tan(\sin x) - \sin(\tan x)
        \]
        求导得,
        \[
        f'(x) = \frac{\cos x}{\cos^2(\sin x)} - \frac{\cos(\tan x)}{\cos^2 x} = \frac{\cos^3 x - \cos(\tan x)\cdot \cos^2(\sin x)}{\cos^2 x \cdot \cos^2(\tan x)}
        \]
        对 $0 < x < \arctan \dfrac{\pi}{2}$,由与 $\cos$ 在 $(0, \frac{\pi}{2})$ 上凸,由琴生不等式及AM-GM不等式,
        \[
        \sqrt[3]{\cos(\tan x)\cdot \cos^2(\sin x)} < \frac{1}{3}[\cos(\tan x) + 2\cos(\sin x)] \le \cos\left(\frac{\tan x + 2\sin x}{3}\right) < \cos x
        \]
        其中最后一个不等式由
        \[
        \left(\frac{\tan x + 2\sin x}{3}\right)' = \frac{1}{3}\left(\frac{1}{\cos^2 x} + 2\cos x\right) \ge 1
        \]
        得到,据此有
        \[
        \cos^3 x - \cos(\tan x) \cdot \cos^2(\sin x) > 0 \Rightarrow f'(x) > 0
        \]
        因此 $f$ 在 $\left[0, \arctan \dfrac{\pi}{2}\right]$ 上单调递增。又注意到
        \[
        \tan\left(\sin(\arctan \frac{\pi}{2})\right) = \tan\frac{\frac{\pi}{2}}{\sqrt{1+\frac{\pi^2}{4}}} > \tan\frac{\pi}{4} = 1
        \]
        因此对于 $x \in \left[\arctan \frac{\pi}{2}, \frac{\pi}{2}\right]$,有 $\tan(\sin x) > 1$,所以 $f(x) > 0$。综上,对于所有 $x \in (0, \frac{\pi}{2})$,
        \[
        \tan(\sin x) > \sin(\tan x).
        \]
    \end{solution}

    \question 证明:对半开区间 $\left(0, \dfrac{\pi}{2}\right]$ 内的任意 $x$,有
    \[
    \left(\frac{\sin x}{x}\right)^3 > \cos x.
    \]
    \begin{solution}
        在 $\left(0, \dfrac{\pi}{2}\right]$ 上,有泰勒展开的不等式:
        \[
        \sin x > x - \frac{x^3}{6}, \quad \cos x < 1 - \frac{x^2}{2} + \frac{x^4}{24}
        \]
        因此
        \[
        \left(\frac{\sin x}{x}\right)^3 > \left(1 - \frac{x^2}{6}\right)^3
        \]
        展开右边并与 $\cos x$ 比较:
        \[
        \left(1 - \frac{x^2}{6}\right)^3 = 1 - \frac{x^2}{2} + \frac{x^4}{12} - \frac{x^6}{216} > 1 - \frac{x^2}{2} + \frac{x^4}{24} > \cos x
        \]
        其中
        \[
        0 < x \le \frac{\pi}{2} < 3 \Rightarrow x^2 < 9   \Rightarrow \frac{x^4}{12} - \frac{x^6}{216} > \frac{x^4}{24} 
        \]
        故原不等式得证。
    \end{solution}

    \question 证明不等式:当 $0 < x \le 1$ 时,有
    \[
    \sin x + \arcsin x > 2x.
    \]
    \begin{solution}
        $\sin x$ 与 $\arcsin x$ 的麦克劳林展开式分别为
        \[
        \sin x = x - \frac{x^3}{3!} + \frac{x^5}{5!} - \frac{x^7}{7!} + \cdots
        \]
        \[
        \arcsin x = x + \frac{1}{2}\frac{x^3}{3} + \frac{1\cdot 3}{2\cdot 4}\frac{x^5}{5} + \frac{1\cdot 3\cdot 5}{2\cdot 4\cdot 6}\frac{x^7}{7} + \cdots
        \]
        因此,对任意 $0 < x \le 1$,存在 $\theta_1,\theta_2$,满足 $0<\theta_1<x,0<\theta_2<x$,使得
        \[
        \sin x = x - \frac{x^3}{3!} + \frac{\theta_1^5}{5!},\quad \arcsin x = x + \frac{1}{2}\frac{x^3}{3} + \frac{1\cdot 3}{2\cdot 4}\frac{\theta_2^5}{5}
        \]
        于是
        \[
        \sin x + \arcsin x = 2x + \frac{\theta_1^5}{5!} + \frac{1\cdot 3}{2\cdot 4}\frac{\theta_2^5}{5} > 2x
        \]
        故得证。
    \end{solution}
    \begin{solution}
        设
        \[
        f(x) = \sin x + \arcsin x,
        \]
        则 $f(0)=0$,且
        \[
        f'(x) = \cos x + \frac{1}{\sqrt{1-x^2}},
        \]
        从而
        \[
        f'(0) = 2,
        \]
        故直线 $y=2x$ 是曲线 $y=f(x)$ 在原点处的切线。又
        \[
        f''(x) = -\sin x + \frac{x}{(1-x^2)^{\frac{3}{2}}}.
        \]
        当 $0<x<1$ 时,有
        \[
        \frac{x}{(1-x^2)^{\frac{3}{2}}} > \sin x,
        \]
        因此 $f''(x)>0$,函数 $f(x)$ 在区间 $(0,1)$ 上上凸,即$f(x)$ 在 $(0,1]$ 上位于其在原点处的切线 $y=2x$ 之上,故得证
        \[
        \sin x + \arcsin x > 2x
        \]
    \end{solution}

    \question 设 $f: \mathbb{R} \to \mathbb{R}$ 为二次可微函数,满足 $f(0)=1$, $f'(0)=0$,且对所有 $x \in [0,\infty)$ 有
    \[
    f''(x)-5f'(x)+6f(x)\ge0.
    \]
    证明对所有 $x \in [0,\infty)$ 有
    \[
    f(x) \ge 3 e^{2x} - 2 e^{3x}.
    \]
    \begin{solution}
        设$x \in [0,\infty)$,令 $g(x) = f'(x) - 2f(x)$. 则
        \[
        g'(x) - 3 g(x) \ge 0
        \]
        即
        \[
        \bigl(g(x) e^{-3x}\bigr)' \ge 0
        \]
        因此
        \[
        g(x)e^{-3x} \ge g(0) = f'(0) - 2 f(0) = -2
        \]
        即
        \[
        f'(x) - 2f(x) \ge -2 e^{3x}
        \]
        两边同乘 $e^{-2x}$ 并整理得
        \[
        (f(x)e^{-2x})' \ge -2 e^{x}
        \]
        即
        \[
        (f(x)e^{-2x} + 2 e^x)' \ge 0
        \]
        于是
        \[
        f(x)e^{-2x} + 2 e^x \ge f(0) + 2 = 3
        \]
        即
        \[
        f(x) \ge 3 e^{2x} - 2 e^{3x}
        \]
    \end{solution}

    \question 设函数
    \[
    F(x) = e^{-x} - \left(1-\frac{x}{n}\right)^n,
    \]
    证明:当 $n \ge 2$ 且 $x \in [0,n]$ 时,
    \[
    0 \le F(x) \le \frac{e^{-1}}{n}.
    \]
    \begin{solution}
        函数 $F(x)$ 在区间 $[0,n]$ 上连续且可微,因此在该区间内必有最大值与最小值,且只能出现在端点或临界点处。有$F(0) = 1-1 = 0,F(n) = e^{-n}$,且
        \[
        F'(x) = -e^{-x} + \left(1-\frac{x}{n}\right)^{\,n-1}
        \]
        先证明 $F(x) \ge 0$。只需证明
        \[
        \left(1-\frac{x}{n}\right)^n \le e^{-x}, \quad x \in [0,n]
        \]
        等价于
        \[
        0 \le 1-\frac{x}{n} \le e^{-x/n}, \quad x \in [0,n]
        \]
        令 $t=\frac{x}{n}$,则 $t \in [0,1]$,上述不等式化为
        \[
        1-t \le e^{-t}, \quad t \in [0,1]
        \]
        这是显然成立的,因为直线 $y=1-t$ 是曲线 $y=e^{-t}$ 在 $(0,1)$ 处的切线,而 $y=e^{-t}$ 在$[0,1]$上向下凸,因此图像始终位于其切线之上,故$F(x) \ge 0$,最小值在 $x=0$ 处取得。接下来确定最大值,由于
        \[
        F'(n) = -e^{-n} < 0,
        \]
        最大值不可能出现在 $x=n$,因此必存在 $x_0 \in (0,n)$ 使得
        $F'(x_0)=0$,这意味
        \[
        e^{-x_0} = \left(1-\frac{x_0}{n}\right)^{n-1}
        \]
        于是
        \begin{align*}
        F(x_0)
        &= e^{-x_0}-\left(1-\frac{x_0}{n}\right)^n \\
        &= e^{-x_0}-\left(1-\frac{x_0}{n}\right)^{n-1}\left(1-\frac{x_0}{n}\right) \\
        &= e^{-x_0}-e^{-x_0}\left(1-\frac{x_0}{n}\right) \\
        &= e^{-x_0}\frac{x_0}{n}
        \end{align*}
        设 $g(x)=x e^{-x}$,可推导得知$g(x)$ 在 $x=1$ 处取得最大值 $e^{-1}$,因此
        \[
        F(x_0) \le \frac{e^{-1}}{n}
        \]
        综上所述 $n\ge 2$ 且 $x \in [0,n]$ 时,
        \[
        0 \le F(x) \le \frac{e^{-1}}{n}
        \]
    \end{solution}

    \question 已知在$(-\infty, +\infty)$上具有二阶连续导数的函数$f(x)$满足方程
    \[
    x^2 f''(x)-2x\sin x f'(x) = e^x + e^{-x}-2,
    \]
    若$f(x)$在$x=a$处取极值,问$f(a)$是函数$f(x)$的极大值还是极小值?请说明理由。
    \begin{solution}
        因为 $f(x)$ 在 $x = a$ 处取极值,所以
        \[
        f'(a) = 0.
        \]
        代入原方程得
        \[
        a^2 f''(a) = e^a + e^{-a} - 2.
        \]
        由AM-GM不等式,
        \[
        e^a + e^{-a} \ge 2,
        \]
        等号成立当且仅当 $a = 0$,所以
        \[
        a^2 f''(a)=e^a + e^{-a} - 2 \ge 0,
        \]
        若 \( a \ne 0 \),则 \( a^2 > 0 \),因此
        \[
        f''(a) = \frac{e^a + e^{-a} - 2}{a^2} > 0,
        \]
        所以 \( f(x) \) 在 \( x = a \) 处为极小值;若 \( a = 0 \),则
        \[
        f''(0) = \lim_{x \to 0} \frac{e^x + e^{-x} - 2}{x^2}
        \]
        由洛必达法则,
        \[
        f''(0) = \lim_{x \to 0} \frac{e^x + e^{-x}}{2} = 1 > 0.
        \]
        因此\( f''(0) > 0 \)仍为极小值,故\( f(a) \) 是函数 \( f(x) \) 的极小值。
    \end{solution}

    \question 已知函数 $f(t)$ 在 $[a,x]$ 上可微,且 $f'(t)$ 可微。对所有 $x > a$,存在 $c_x$ 满足 $a < c_x < x$ 且
    \[
    \int_{a}^{x} f(t) \, dt = f(c_x)(x-a).
    \]
    假设 $f'(a) \neq 0$,证明
    \[
    \lim_{x \to a} \frac{c_x - a}{x-a} = \frac{1}{2}.
    \]
    \begin{solution}
        设
        \[
        F(x) = \int_{a}^{x} f(t) \, dt
        \]
        利用 $F(x)$ 的泰勒展开,有
        \[
        F(x) = F(a) + (x-a)F'(a) + \frac{(x-a)^2}{2} F''(\theta_x)
        \]
        其中 $a<\theta_x<x$,且当 $x \to a$ 时,$\theta_x \to a$。又 $F(a) = 0,F'(x) = f(x),F''(x) = f'(x)$,所以
        \[
        F(x) = 0 + (x-a) f(a) + \frac{(x-a)^2}{2} f'(\theta_x)
        \]
        由定义有
        \[
        f(c_x) = \frac{F(x)}{x-a} = f(a) + \frac{x-a}{2} f'(\theta_x)
        \]
        因此
        \[
        \frac{f(c_x) - f(a)}{x-a} = \frac{1}{2} f'(\theta_x)
        \]
        另一方面可以写成
        \[
        \frac{f(c_x) - f(a)}{x-a} = \frac{f(c_x) - f(a)}{c_x - a} \cdot \frac{c_x - a}{x-a}
        \]
        取极限得到
        \[
        \lim_{x \to a} \frac{1}{2} f'(\theta_x) = \lim_{x \to a} \frac{f(c_x) - f(a)}{c_x - a} \cdot \lim_{x \to a} \frac{c_x - a}{x - a}
        \]
        由此可得
        \[
        \frac{1}{2} f'(a) = f'(a) \cdot \lim_{x \to a} \frac{c_x - a}{x - a}
        \]
        即
        \[
        \lim_{x \to a} \frac{c_x - a}{x - a} = \frac{1}{2}
        \]
    \end{solution}

    \question 
    \begin{parts}
    \part 已知 $f(x)$ 在 $[0,+\infty)$ 上连续且单调增加,且 $f(0) \ge 0$,证明:
    \[
    F(x) = \begin{cases}
    \dfrac{1}{x^n} \displaystyle\int_0^x t^{n-1} f(t)\,dt, & x > 0, \\
    0, & x = 0
    \end{cases}
    \]
    在 $[0,+\infty)$ 上连续且单调增加,其中 $n > 0$。
    \begin{solution}
        先证连续性。由于 \( f \) 在 \( [0, +\infty) \) 上连续,且 \( f(0) \ge 0 \),有:
        \[
        \lim_{x \to 0^+} F(x) = \lim_{x \to 0^+} \frac{1}{x^n} \int_0^x t^{n-1} f(t)\,dt = 0 = F(0),
        \]
        所以 \( F(x) \) 在 \( x = 0 \) 处连续。

        再证单调性。对于 \( x > 0 \),由积分中值定理可得:
        \[
        \int_0^x t^{n-1} f(t)\,dt = f(\xi) \int_0^x t^{n-1}\,dt = f(\xi) \cdot \frac{x^n}{n}, \quad \text{其中 } \xi \in (0, x).
        \]
        因此,
        \[
        F(x) = \frac{1}{x^n} \cdot \int_0^x t^{n-1} f(t)\,dt = \frac{1}{x^n} \cdot f(\xi) \cdot \frac{x^n}{n} = \frac{f(\xi)}{n}.
        \]
        因为 \( f \) 单调递增,\( \xi \in (0,x) \),所以 \( x \uparrow \Rightarrow \xi \uparrow \Rightarrow f(\xi) \uparrow \),
        故 \( F(x) \) 单调递增。
    \end{solution}
    \part 设 \( f(x) \) 在 \( [0,1] \) 上二阶可导,且满足 \( |f''(x)| \le 1 \)。已知 \( f(x) \) 在 \( (0,1) \) 内取最大值为 \( \dfrac{1}{4} \),证明:
    \[
    |f(0)| + |f(1)| \le 1.
    \]
    \begin{solution}
        设 \( x_0 \in (0,1) \),使得 \( f(x_0) = \dfrac{1}{4} \),且 \( f'(x_0) = 0 \)。
        对 \( f(0) \) 做 Taylor 展开(Lagrange 余项形式)得:
        \[
        f(0) = f(x_0) + f'(x_0)(0 - x_0) + \frac{1}{2} f''(\xi) (0 - x_0)^2 = \frac{1}{4} + \frac{1}{2} f''(\xi) x_0^2,
        \]
        其中 \( \xi \in (0, x_0) \)。
        同理,
        \[
        f(1) = \frac{1}{4} + \frac{1}{2} f''(\eta)(1 - x_0)^2, \quad \eta \in (x_0, 1).
        \]
        因为 \( |f''(x)| \le 1 \),故
        \[
        |f(0)| \le \left| \frac{1}{4} + \frac{1}{2} f''(\xi) x_0^2 \right| \le \frac{1}{4} + \frac{1}{2} x_0^2,
        \]
        \[
        |f(1)| \le \left| \frac{1}{4} + \frac{1}{2} f''(\eta)(1 - x_0)^2 \right| \le \frac{1}{4} + \frac{1}{2} (1 - x_0)^2.
        \]
        相加得:
        \[
        |f(0)| + |f(1)| \le \frac{1}{2} + \frac{1}{2} \left( x_0^2 + (1 - x_0)^2 \right).
        \]
        注意 \( x_0^2 + (1 - x_0)^2 = 1 - 2x_0(1 - x_0) \le 1 \),因此:
        \[
        |f(0)| + |f(1)| \le \frac{1}{2} + \frac{1}{2} \cdot 1 = 1.
        \]
    \end{solution}
    \end{parts}

    \question 设函数 \( f(x) \) 在 \( x = 0 \) 处可导,且 \( f(0) = 0, f'(0) = 1 \),设
    \[
    F(x) = \int_0^x t^{n-1} f(x^n - t^n) \, dt,
    \]
    求
    \[
    \lim_{x \to 0} \frac{F(x)}{x^{2n}}.
    \]
    \begin{solution}
        由题设 \( f(0) = 0 \),且 \( f \) 在 0 可导,考虑换元 \( t = x u \),则
        \[
        F(x) = \int_0^x t^{n-1} f(x^n - t^n) dt = x^n \int_0^1 u^{n-1} f(x^n - x^n u^n) du.
        \]
        注意 \( x^n - x^n u^n = x^n(1 - u^n) \),故
        \[
        F(x) = x^n \int_0^1 u^{n-1} f\big(x^n(1 - u^n)\big) du.
        \]
        因 \( f(0) = 0 \),且 \( f \) 在 0 处可导,令 \( h = x^n(1 - u^n) \),有
        \[
        f(h) = f(0) + f'(0) h + o(h) = f'(0)x^n(1 - u^n) + o(x^n).
        \]
        所以
        \[
        F(x) = x^n \int_0^1 u^{n-1} \left( f'(0)x^n(1 - u^n) + o(x^n) \right) du
        = f'(0)x^{2n} \int_0^1 u^{n-1}(1 - u^n) du + o(x^{2n}).
        \]
        计算积分:
        \[
        \int_0^1 u^{n-1}(1 - u^n) du = \int_0^1 u^{n-1} du - \int_0^1 u^{2n-1} du = \frac{1}{n} - \frac{1}{2n} = \frac{1}{2n}.
        \]
        故
        \[
        \lim_{x \to 0} \frac{F(x)}{x^{2n}} = \frac{1}{2n}
        \]
    \end{solution}

\end{questions}

\pagebreak

\begin{center}
  {\fontsize{30pt}{26pt}\selectfont
    \hypertarget{积分}{积分} \label{积分}
  }
\end{center}
\separator
\vspace{1pt}

\begin{questions}
    \question 已知\( p > 0 \),证明
    \[
    \lim_{n \to \infty} \frac{1^p + 2^p + \cdots + n^p}{n^{p+1}} = \frac{1}{p + 1}.
    \]
    \begin{solution}
    设 $$S_n = \sum_{k=1}^{n} k^p$$则
    \[
    \frac{S_n}{n^{p+1}} = \frac{1}{n} \sum_{k=1}^{n} \left( \frac{k}{n} \right)^p.
    \]
    这是函数 $f(x) = x^p$ 在 $[0,1]$ 的黎曼和,故
    \[
    \lim_{n \to \infty} \frac{S_n}{n^{p+1}} = \int_0^1 x^p \, dx = \frac{1}{p+1}.
    \]
    \end{solution}
    
    \question 求
    \[
    \lim_{k \to \infty} \left( \frac{2017^{\frac{1}{k}}}{k + 1} + \frac{2017^{\frac{2}{k}}}{k + \frac{1}{2}} + \cdots + \frac{2017^{\frac{k}{k}}}{k + \frac{1}{k}} \right)
    \]
    \begin{solution}
        写成黎曼和:
        \[
        \lim_{k \to \infty} \frac{1}{k} \sum_{j=1}^{k} 2017^{\frac{j}{k}} = \int_0^1 2017^x \, dx = \left[ \frac{2017^x}{\ln 2017} \right]_0^1 = \frac{2016}{\ln 2017}.
        \]
        \textcolor{red}{(待验证,$2017^{\frac{j}{k}}/(1+0?)$)}
    \end{solution}

    \question 设
    \[
    a_{n}=\frac{2}{n}\Big[(2^{2}+1)+\left(2+\frac{2}{n}\right)^{2}+1 + \cdots + \left(2+\frac{2n-2}{n}\right)^{2}+1\Big] 
    \]
    求 $\displaystyle \lim_{n\to\infty} a_n$
    \begin{solution}
        发现
        \[
        a_n = \sum_{k=1}^n \frac{2}{n} \left[ \left( 2 + \frac{2k-2}{n} \right)^2 + 1 \right].
        \]
        是函数 $f(x) = (2+x)^2+1$ 在 $[0,2]$ 上的黎曼和,故
        \[
        \lim_{n\to\infty} a_n = \int_0^2 \left[ (2+x)^2 + 1 \right] \, dx = \left[ \frac{x^3}{3} + 2x^2 +5x \right]_0^2  = \frac{62}{3}.
        \]
    \end{solution}

    \question 试求
    \[
    \lim_{n\to \infty} \frac{5}{n^2} \left[ \sqrt{4n^2 - 2\cdot 1^2} + \sqrt{4n^2 - 2\cdot 2^2} + \cdots + \sqrt{4n^2 - 2\cdot n^2} \right]
    \]
    \begin{solution}
        将其写成黎曼和,得
        \[
        \lim_{n\to \infty} \sum_{k=1}^n \frac{5}{n} \sqrt{4 - 2 \left(\frac{k}{n}\right)^2} 
        = \int_0^1 5 \sqrt{4 - 2x^2} \, dx.
        \]
        可得原极限为
        \[
        \left[ \frac{5}{\sqrt{2}} \left( x \sqrt{2 - x^2} + 2 \sin^{-1} \frac{x}{\sqrt{2}} \right) \right]_0^1
        = \frac{5\sqrt{2} (\pi + 2)}{4}.
        \]
    \end{solution}

    \question 设函数 $f:(0,\infty)\rightarrow\mathbb{R}$ ,且当 $x>0$ 时,恒有
    \[
    \lim_{n\rightarrow\infty}\frac{1}{3n-1}\left(f\left(\frac{x^{2}}{n}\right)+f\left(\frac{2x^{2}}{n}\right)+\cdot\cdot\cdot+f\left(\frac{nx^{2}}{n}\right)\right)=\frac{\sqrt[3]{7+x}}{x},
    \]
    求 $f(1)$。
    \begin{solution}
        先写成黎曼和:
        \begin{align*}
        &\lim_{n\to\infty} \frac{1}{3n-1} \left( f\left(\frac{x^2}{n}\right) + f\left(\frac{2x^2}{n}\right) + \cdots + f\left(\frac{nx^2}{n}\right) \right) \\
        &= \lim_{n\to \infty} \frac{1}{3n-1} \sum_{k=1}^n f\left(\frac{kx^2}{n}\right) \\
        &= \lim_{n\to \infty} \frac{n}{3n-1} \sum_{k=1}^n \frac{1}{n} f\left(\frac{kx^2}{n}\right) \\
        &= \frac{1}{3} \int_0^1 f(x^2 t)\, dt = \frac{\sqrt[3]{7+x}}{x}.
        \end{align*}
        于是
        \[
        \int_0^1 x^2 f(x^2 t)\, dt = 3x \sqrt[3]{7+x}.
        \]
        取 $u = x^2 t \Rightarrow du = x^2 dt$,则
        \begin{align*}
        \int_0^{x^2} f(u)\, du &= 3x \sqrt[3]{7+x}, \\
        \frac{d}{dx} \left( \int_0^{x^2} f(u)\, du \right) &= \frac{d}{dx} \left( 3x \sqrt[3]{7+x} \right), \\
        2x f(x^2) &= 3 \sqrt[3]{7+x} + x (7+x)^{-\frac{2}{3}}.
        \end{align*}
        令 $x=1$,则
        \[
        2 f(1) = 3\cdot 2 + \frac{1}{4} \quad \Rightarrow \quad f(1) = \frac{25}{8}.
        \]
    \end{solution}

    \question 
    \[
    \lim_{n\to\infty} \frac{1}{n} \left( \sin \frac{\pi}{12n} + \sin \frac{3\pi}{12n} + \sin \frac{5\pi}{12n} + \dots + \sin \frac{(2n-1)\pi}{12n} \right) 
    \]
    \begin{solution}
        写成黎曼和,
        \begin{align*}
        \lim_{n\to \infty} \frac{1}{n} \sum_{k=1}^n \sin \frac{(2k-1)\pi}{12n}
        &= \lim_{n\to \infty} \frac{1}{n} \left( \sum_{k=1}^{2n} \sin \frac{k\pi}{12n} - \sum_{k=1}^{n} \sin \frac{2k\pi}{12n} \right)\\
        &= \int_0^2 \sin \frac{\pi}{12}x \, dx - \int_0^1 \sin \frac{\pi}{6}x \, dx
        \end{align*}
        可得
        \[
        \left[ -\frac{12}{\pi} \cos \frac{\pi}{12}x \right]_0^2 - \left[ -\frac{6}{\pi} \cos \frac{\pi}{6}x \right]_0^1 = \frac{6 - 3\sqrt{3}}{\pi}
        \]
    \end{solution}

    \question 求
    \[
    f(x)=\lim_{n\to\infty}\sum_{k=1}^{n} 
    \frac{x(1-x)}{k+(n-k)x}\;,\,x\in[0,1]
    \]
    \begin{solution} 
    显然$f(0)=f(1)=0,$对于$x \in [0,1],$
        \begin{align*} 
        \lim_{n\to\infty}\sum_{k=1}^{n} 
        \frac{x(1-x)}{k+(n-k)x}\
        &=x(1-x)\lim_{n\to\infty}\sum_{k=1}^{n} 
        \frac{1}{(\frac{k}{n})(1-x)+x}\cdot\frac{1}{n}\\ 
        &=x(1-x)\int_0^1 \frac{1}{(1-x)y+x}dy\\ 
        &=x\cdot\left[\ln|(1-x)y+x|\right]_0^1\\ 
        &=-x\ln(x) 
        \end{align*}
    \end{solution}

    \question 求极限
    \[
    \lim_{n\to\infty} \left( \sum_{k=10}^{n+9} \frac{2^{11(k-9)/n}}{\log_2 e^{n/11}} - \sum_{k=0}^{n-1} \frac{58}{\pi \sqrt{(n-k)(n+k)}} \right)
    \]
    \begin{solution}
        首先有
        \begin{align*}
        \lim_{n\to\infty} \sum_{k=10}^{n+9} \frac{2^{11(k-9)/n}}{\log_2 e^{n/11}}
        = \lim_{n \to \infty} \sum_{k=1}^{n} 2^{11(\frac{k}{n})} \ln\left(\frac{11}{n}\right) \ln 2
        =\int_{0}^{11} 2^{x} \ln 2\, dx
        \end{align*}
        且
        \[
        \lim_{n\to\infty} \sum_{k=0}^{n-1} \frac{58}{\pi \sqrt{(n-k)(n+k)}}
        = \frac{58}{\pi} \lim_{n\to\infty} \sum_{k=0}^{n-1} \frac{1}{n\sqrt{1 - (\frac{k}{n})^2}}
        = \frac{58}{\pi} \int_{0}^{1} \frac{1}{\sqrt{1-x^2}} \, dx
        \]
        故所求为
        \[
        \int_{0}^{11} 2^{x} \ln 2\, dx-\frac{58}{\pi}\int_{0}^{1} \frac{1}{\sqrt{1-x^2}} \, dx=\left[2^{x} \right]_{0}^{11}-\frac{58}{\pi}\left[ \arcsin x \right]_{0}^{1}=2047-29=2018
        \]
    \end{solution}

    \question 求极限
\[
\lim_{t\nearrow 1}(1-t)\sum_{n=1}^{\infty}\frac{t^n}{1+t^n},
\]
其中 $t\nearrow 1$ 表示 $t$ 从小于 $1$ 的一侧趋近于 $1$。

\begin{solution}
注意到
\[
\lim_{t\to 1^-}\frac{1-t}{-\ln t}=1,
\]
因此
\[
\lim_{t\to 1^-}(1-t)\sum_{n=1}^{\infty}\frac{t^n}{1+t^n}
=\lim_{t\to 1^-}(-\ln t)\sum_{n=1}^{\infty}\frac{t^n}{1+t^n}。
\]

将 $t^n=e^{-n(-\ln t)}$,可得
\[
(-\ln t)\sum_{n=1}^{\infty}\frac{1}{1+e^{-n\ln t}}。
\]
设 $h=-\ln t$,则当 $t\to 1^-$ 时 $h\to 0^+$,于是上述极限化为
\[
\lim_{h\to 0^+}h\sum_{n=1}^{\infty}\frac{1}{1+e^{nh}}。
\]

这是一个黎曼和,对应的积分为
\[
\int_0^{\infty}\frac{dx}{1+e^x}。
\]
计算该积分得
\[
\int_0^{\infty}\frac{dx}{1+e^x}=\ln 2。
\]
\end{solution}

\question 对于 $n = 1,2,\dots$,定义
\[
S_n = \log \left(\sqrt{1^1 \cdot 2^2 \cdot \dots \cdot n^n}\right) - \log(\sqrt{n}),
\]
其中 $\log$ 表示自然对数。求 $\lim_{n \to \infty} S_n$。

\begin{solution}
将 $S_n$ 变形为
\begin{align*}
S_n &= \frac{1}{2}\sum_{k=1}^n k \log k - \frac{1}{2}\log n \\
&= \frac{1}{n^2}\sum_{k=1}^n k \log k - \frac{1}{2}\log n \\
&= \frac{1}{n}\sum_{k=1}^n \frac{k}{n}\left(\log \frac{k}{n} + \log n\right) - \frac{1}{2}\log n \\
&= \frac{1}{n}\sum_{k=1}^n \frac{k}{n}\log\frac{k}{n} + \frac{\log n}{n^2}\sum_{k=1}^n k - \frac{1}{2}\log n \\
&= \frac{1}{n}\sum_{k=1}^n \frac{k}{n}\log\frac{k}{n} + \frac{\log n}{n^2}\frac{n(n+1)}{2} - \frac{1}{2}\log n \\
&= \frac{1}{n}\sum_{k=1}^n \frac{k}{n}\log\frac{k}{n} + \frac{(n+1)\log n}{2n} - \frac{1}{2}\log n \\
&= \frac{1}{n}\sum_{k=1}^n \frac{k}{n}\log\frac{k}{n} + \frac{\log n}{2n}.
\end{align*}

注意最后一项 $\frac{\log n}{2n} \to 0$。而 $\frac{1}{n}\sum_{k=1}^n \frac{k}{n}\log\frac{k}{n}$ 是可积函数 $f(x) = x \log x$ 在区间 $[0,1]$ 上的黎曼和。因此
\[
\lim_{n \to \infty} \frac{1}{n}\sum_{k=1}^n \frac{k}{n}\log\frac{k}{n} = \int_0^1 x \log x \, dx = \left[\frac{x^2}{2}\log x - \frac{x^2}{4}\right]_0^1 = -\frac{1}{4}.
\]

综上,极限存在,并且
\[
\lim_{n \to \infty} S_n = -\frac{1}{4}.
\]
\end{solution}


    \question 求极限
    \[
    \lim_{n\rightarrow\infty}\frac{(1^{2}+2^{2}+\cdots+n^{2})(1^{5}+2^{5}+\cdots+n^{5})}{(1^{3}+2^{3}+\cdots+n^{3})(1^{4}+2^{4}+\cdots+n^{4})}
    \]
    \begin{solution}
        首先看到
        \[
        a_n=\frac{\displaystyle\left(\sum_{k=1}^{n} \left(\frac{k}{n}\right)^2\right)
        \left(\sum_{k=1}^{n} \left(\frac{k}{n}\right)^5\right)}
        {\displaystyle \left(\sum_{k=1}^{n} \left(\frac{k}{n}\right)^3\right)
        \left(\sum_{k=1}^{n} \left(\frac{k}{n}\right)^4\right)}
        =
        \frac{\displaystyle\left(\frac{1}{n}\sum_{k=1}^{n} \left(\frac{k}{n}\right)^2\right)
        \left(\frac{1}{n}\sum_{k=1}^{n} \left(\frac{k}{n}\right)^5\right)}
        {\displaystyle\left(\frac{1}{n}\sum_{k=1}^{n} \left(\frac{k}{n}\right)^3\right)
        \left(\frac{1}{n}\sum_{k=1}^{n} \left(\frac{k}{n}\right)^4\right)}
        \]
        故
        \[
        \lim_{n\to\infty} a_n
        =\frac{\displaystyle\left(\int_0^1 x^2 dx\right)\left(\int_0^1 x^5 dx\right)}
        {\displaystyle\left(\int_0^1 x^3 dx\right)\left(\int_0^1 x^4 dx\right)}
        =\frac{\dfrac{1}{3}\cdot \dfrac{1}{6}}{\dfrac{1}{4}\cdot \dfrac{1}{5}}
        =\frac{10}{9}
        \]
    \end{solution}

    \question 计算极限
    \[
    \lim_{n\to\infty} \frac{\left(\sqrt{1}+\sqrt{2}+\cdots+\sqrt{n}\right)^{2}\left(1^{3}+2^{3}+\cdots+n^{3}\right)}{\left(\sqrt[3]{1}+\sqrt[3]{2}+\cdots+\sqrt[3]{n}\right)^{3}\left(1^{2}+2^{2}+\cdots+n^{2}\right)}
    \]
    \begin{solution}
        由
        \[
        \sum_{k=1}^n k^3 = \left(\frac{n(n+1)}{2}\right)^2,\quad \sum_{k=1}^n k^2 = \frac{n(n+1)(2n+1)}{6}
        \]
        可得
        \[
        f(n)=\frac{\displaystyle n^2\left(\sum_{k=1}^n \frac{1}{n}\sqrt{\frac{k}{n}}\right)^2}{\displaystyle n^3\left(\sum_{k=1}^n \frac{1}{n}\sqrt[3]{\frac{k}{n}}\right)^3} \cdot \frac{3n(n+1)}{2(2n+1)}
        \]
        由积分定义,
        \[
        \lim_{n\to\infty} \sum_{k=1}^n \frac{1}{n}\sqrt{\frac{k}{n}} = \int_0^1 \sqrt{x}\,dx = \frac{2}{3},\quad
        \lim_{n\to\infty} \sum_{k=1}^n \frac{1}{n}\sqrt[3]{\frac{k}{n}} = \int_0^1 x^{\frac{1}{3}}\,dx = \frac{3}{4}
        \]
        因此
        \[
        \lim_{n\to\infty} f(n) = \frac{\left(\frac{2}{3}\right)^2}{\left(\frac{3}{4}\right)^3} \cdot \frac{3}{4} = \frac{64}{81}
        \]
    \end{solution}

    \question 求极限
    \[
    L=\lim_{n\to\infty}\left(\frac{(2n)!}{n!\,n^{n}}\right)^{\frac{1}{n}}
    \]
    \begin{solution}
        首先有
        \[
        \ln L=\lim_{n\to\infty}\frac{1}{n}\Big(\ln(2n)!-\ln n! - n\ln n\Big)
        \]
        其中
        \[
        \ln(2n)!-\ln n! = \sum_{k=n+1}^{2n}\ln k=\sum_{k=1}^{n}\ln\!\big(n+k\big)
        \]
        因此
        \[
        \ln L=\lim_{n\to\infty}\frac{1}{n}\left(\sum_{k=1}^{n}\ln\!\big(n+k\big)-n\ln n\right)
        =\lim_{n\to\infty}\frac{1}{n}\sum_{k=1}^{n}\ln\!\Big(1+\frac{k}{n}\Big)
        \]
        这是在 $[0,1]$ 上的黎曼和,得
        \[
        \ln L=\int_0^1 \ln(1+x)\,dx
        =\big[(1+x)\ln(1+x)-x\big]_0^1=2\ln 2-1
        \]
        所以
        \[
        L=e^{2\ln 2-1}=\frac{4}{e}
        \]
    \end{solution}

    \question 设 
    \[
    A_n = \{\sin(\ln 1), \sin(\ln 2), \dots, \sin(\ln n)\}, \quad B_n = \{\cos(\ln 1), \cos(\ln 2), \dots, \cos(\ln n)\}
    \]
    且 $\bar{a}_n,\bar{b}_n$ 分别为 $A_n,B_n$ 的算术平均值。求 
    \[
    \lim_{n\to\infty} (\bar{a}_n \cos(\ln n) - \bar{b}_n \sin(\ln n)).
    \]
    \begin{solution}
        由积分定义,
        \begin{align*}
        &\lim_{n \to \infty} \frac{1}{n} \left( \sum_{i=1}^n (\sin(\ln i)\cos(\ln n) - \cos(\ln i)\sin(\ln n)) \right) \\
        &=\lim_{n\to\infty} \frac{1}{n} \left(\sum_{i=1}^{n} \sin(\ln i - \ln n)\right) \\
        &= \lim_{n\to\infty} \frac{1}{n} \sum_{i=1}^{n} \sin\left(\ln \left(\frac{i}{n}\right)\right) \\
        &= \int_0^1 \sin(\ln x) \, dx \\
        &= \Im \int_0^1 (e^{i\ln x}) \, dx \\
        &= \Im \int_0^1 x^i \, dx \\
        &= \Im \left(\frac{1}{i+1}\right) = -\frac{1}{2}
        \end{align*}
    \end{solution}

    \question 设函数 $f(x)$ 在闭区间 $[0, 1]$ 上具有连续导数,$f(0) = 0,f(1) = 1$。证明:
\[
\lim_{n \to \infty} n \left( \int_{0}^{1} f(x) dx - \frac{1}{n} \sum_{k=1}^{n} f\left( \frac{k}{n} \right) \right) = -\frac{1}{2}.
\]

\begin{solution}
考察括号中的项,将其写为累加形式:
\begin{align*}
\int_{0}^{1} f(x) dx - \frac{1}{n} \sum_{k=1}^{n} f\left( \frac{k}{n} \right) &= \sum_{k=1}^{n} \left( \int_{\frac{k-1}{n}}^{\frac{k}{n}} f(x) dx - \frac{1}{n} f\left( \frac{k}{n} \right) \right)
\end{align*}
在区间 $[\frac{k-1}{n}, \frac{k}{n}]$ 上对 $f(x)$ 在 $\frac{k}{n}$ 处进行一阶泰勒展开:
\[
f(x) = f\left( \frac{k}{n} \right) + f'\left( \frac{k}{n} \right) \left( x - \frac{k}{n} \right) + O\left( \left( x - \frac{k}{n} \right)^2 \right)
\]
代入积分项中:
\begin{align*}
&\sum_{k=1}^{n} \left( \int_{\frac{k-1}{n}}^{\frac{k}{n}} \left( f\left( \frac{k}{n} \right) + f'\left( \frac{k}{n} \right) \left( x - \frac{k}{n} \right) \right) dx - \frac{1}{n} f\left( \frac{k}{n} \right) \right) \\
&= \sum_{k=1}^{n} \left( \left[ f\left( \frac{k}{n} \right) x + \frac{f'\left( \frac{k}{n} \right)}{2} \left( x - \frac{k}{n} \right)^2 \right]_{\frac{k-1}{n}}^{\frac{k}{n}} - \frac{1}{n} f\left( \frac{k}{n} \right) \right) \\
&= \sum_{k=1}^{n} \left( \frac{1}{n} f\left( \frac{k}{n} \right) - \frac{f'\left( \frac{k}{n} \right)}{2n^2} - \frac{1}{n} f\left( \frac{k}{n} \right) \right) \\
&= -\sum_{k=1}^{n} \frac{f'\left( \frac{k}{n} \right)}{2n^2}
\end{align*}
将此结果代入原极限式中:
\begin{align*}
\lim_{n \to \infty} n \left( -\sum_{k=1}^{n} \frac{f'\left( \frac{k}{n} \right)}{2n^2} \right) &= \lim_{n \to \infty} -\frac{1}{2} \left( \frac{1}{n} \sum_{k=1}^{n} f'\left( \frac{k}{n} \right) \right)
\end{align*}
根据黎曼和的定义,该和式收敛于 $f'(x)$ 在 $[0, 1]$ 上的定积分:
\begin{align*}
-\frac{1}{2} \int_{0}^{1} f'(x) dx &= -\frac{1}{2} [f(x)]_0^1 \\
&= -\frac{1}{2} (f(1) - f(0)) \\
&= -\frac{1}{2} (1 - 0) \\
&= -\frac{1}{2}
\end{align*}
证毕。
\end{solution}

\question
3) 证明:$\frac{\pi}{4} < \int_{0}^{1} \frac{1}{1+x^8} dx < 1$
\begin{solution}
在 $x \in (0,1)$ 上,$0 < x^8 < 1$:
\begin{itemize}
    \item 因为 $1+x^8 > 1$,所以 $\frac{1}{1+x^8} < 1 \implies \int_{0}^{1} \frac{1}{1+x^8} dx < \int_{0}^{1} 1 dx = 1$。
    \item 因为 $x^8 < x^2$,所以 $1+x^8 < 1+x^2 \implies \frac{1}{1+x^8} > \frac{1}{1+x^2}$。
    积分得 $\int_{0}^{1} \frac{dx}{1+x^8} > \int_{0}^{1} \frac{dx}{1+x^2} = [\tan^{-1}x]_0^1 = \frac{\pi}{4}$。
\end{itemize}
\end{solution}
    
    \question 若 $f(x)$ 为一实系数多项式函数,已知
    \[
    \int_{0}^{1} f(x) f'(x) \, dx = 1, \quad \int_{0}^{1} (f(x))^3 f'(x) \, dx = 2,
    \]
    求
    \[
    \int_{0}^{1} (f(x))^5 f'(x) \, dx
    \] 的值。
    \begin{solution}
        有
        \[
        1=\int_0^1 f(x) f'(x) \, dx = \left[ \frac{1}{2} (f(x))^2 \right]_0^1 = \frac{1}{2} \big([f(1)]^2 - [f(0)]^2\big) \tag{1}
        \]
        同理,
        \[
        2=\int_0^1 (f(x))^3 f'(x) \, dx = \left[ \frac{1}{4} (f(x))^4 \right]_0^1 = \frac{1}{4} \big([f(1)]^4 - [f(0)]^4\big) \tag{2}
        \]
        由 $(1),(2)$ 解得
        \[
        [f(0)]^2 = 1,\quad [f(1)]^2 = 3
        \]
        于是
        \[
        \int_0^1 (f(x))^5 f'(x) \, dx = \left[ \frac{1}{6} (f(x))^6 \right]_0^1 = \frac{1}{6} \big(f(1)^6 - f(0)^6\big) = \frac{13}{3}
        \]
    \end{solution}

    \question 设 $f$ 为实数系上的连续函数, 若
    \[
    \frac{d}{dx}\int_{-x}^{x}f(t)\,dt=x^{2}+1,
    \] 
    求 
    \[
    \int_{-1}^{1}f(x)\,dx
    \]
    \begin{solution}
        两边积分得
        \[
        g(x)=\int_{-x}^x f(t)\,dt =\int (x^2+1)\,dx =\frac{1}{3}x^3+x+ C
        \]
        令$x=0$,得
        \[
        g(0)= \int_0^0 f(t)\,dt= 0 \Rightarrow C=0 \Rightarrow g(x) =\frac{1}{3}x^3+x
        \]
        所以
        \[
        \int_{-1}^1 f(x)\,dx =g(1)=\frac{4}{3}
        \]
    \end{solution}

    \question 已知一连续实函数 $f(x)$ 满足 $f(2x)=3f(x),\forall x \in \mathbb{R},$若 $\displaystyle\int_{0}^{1} f(x) dx=1$,求 
    \[
    \int_{0}^{2} f(x) dx
    \]
    \begin{solution}
        由 $f(2x)=3f(x)$ 得
        \[
        \int_0^1 f(x)\;dx = \frac{1}{3} \int_0^1 f(2x)\;dx =1
        \]
        令 $u=2x$,则 $du=2dx$,于是
        \[
        \int_0^2 f(u)\;du =6
        \]
        故
        \[
        \int_0^2 f(x)\;dx = \int_0^1 f(x)\;dx + \int_1^2 f(x)\;dx =1 + \int_1^2 f(x)\;dx= 6 \Rightarrow \int_1^2 f(x)\;dx =5
        \]
    \end{solution}

    \question 设多项式函数 $y=f(x)$ 满足
    \[
    f(x) = 4x^3 - 12x^2 + 8x + 20 - \int_1^x f(t) \, dt,
    \]
    求 $f(x)$。
    \begin{solution}
        对两边求导得
        \[
        f'(x) + f(x) = 12x^2 - 24x + 8
        \]
        故 $f(x)$ 为二次多项式,设
        \[
        f(x) = a x^2 + b x + c,f'(x) = 2 a x + b
        \]
        代入得
        \[
        f(x) + f'(x) = a x^2 + b x + c + 2 a x + b = a x^2 + (b + 2 a) x + (b + c)
        \]
        比较系数,解得
        \[
        a = 12,  b = -48,  c = 56 \Rightarrow f(x) = 12 x^2 - 48 x + 56
        \]
    \end{solution}

    \question 设 
    \[
    f(x)=\int_{0}^{x} g(t) \, dt + 1, \quad g(x) = 12x^{2} - 6x + \int_{0}^{1} [f(t) + g'(t)] \, dt,
    \]
    求 $g(0)$。
    \begin{solution}
        由题意有
        \[
        f(0) = 1,\quad g(0) = \int_0^1 [f(t) + g'(t)] \, dt,
        \]
        又
        \[
        f(x) = \int_0^x [12t^{2} - 6t + g(0)] \, dt + 1 = 4x^{3} - 3x^{2} + x g(0) + 1
        \]
        代入 $g(x)$ 得
        \[
        g(x) = 12x^{2} - 6x + \int_0^1 (4t^{3} - 3t^{2} + t g(0) + 1) \, dt + g(1) - g(0)
        \]
        其中
        \[
        \int_0^1 (4t^{3} - 3t^{2} + t g(0) + 1) dt = \left[ t^{4} - t^{3} + \frac{1}{2} g(0) t^{2} + t \right]_0^1 = \frac{1}{2} g(0) + 1
        \]
        因此
        \[
        g(x) = 12x^{2} - 6x + 1 + g(1) - \frac{1}{2} g(0)
        \]
        令 $x=1$,得
        \[
        g(0) = 14
        \]
    \end{solution}

\question 设
\[
f(x)=x+3+\int_{0}^{x}g(t)\,dt,\qquad
g(x)=2x-9+\int_{0}^{x}f(t)\,dt,
\]
试求 $f(3)$。
\begin{solution}
设
\[
a=\int_0^1 g(x)\,dx,\qquad b=\int_0^2 f(x)\,dx.
\]
由题得
\[
f(x)=x+3+a,\qquad g(x)=2x-9+b.
\]
因此
\[
b=\int_0^2 f(x)\,dx=\int_0^2 (x+3+a)\,dx
=\Big[\tfrac{1}{2}x^2+(3+a)x\Big]_0^2=8+2a,
\]
\[
a=\int_0^1 g(x)\,dx=\int_0^1 (2x-9+b)\,dx
=\Big[x^2+(b-9)x\Big]_0^1=b-8.
\]
联立得
\[
b=8+2a,\qquad a=b-8.
\]
代入消元得
\[
b=8+2(b-8)\Rightarrow b=8,\quad a=b-8=0.
\]
所以
\[
f(x)=x+3+a=x+3.
\]
因此
\[
f(3)=3+3=6.
\]
\textcolor{red}{(待验证,为什么设$a,b$?)}
\end{solution}

\question 设 $0 < a < b$. 证明
\[
\int_a^b (x^2 + 1) e^{-x^2} \, dx > e^{-a^2} - e^{-b^2}.
\]

\begin{solution}
\textbf{解法 1.} 令
\[
f(x) = \int_0^x (t^2 + 1) e^{-t^2} \, dt, \quad g(x) = - e^{-x^2}.
\]
两函数在 $(0,\infty)$ 上均递增。由柯西中值定理,存在 $x \in (a,b)$ 使得
\[
\frac{f(b)-f(a)}{g(b)-g(a)} = \frac{f'(x)}{g'(x)} = \frac{(x^2+1)e^{-x^2}}{2 x e^{-x^2}} = \frac{1}{2}\Bigl(x + \frac{1}{x}\Bigr) \ge \sqrt{x \cdot \frac{1}{x}} = 1.
\]
于是
\[
\int_a^b (x^2+1)e^{-x^2} \, dx = f(b)-f(a) \ge g(b)-g(a) = e^{-a^2} - e^{-b^2}.
\]

\textbf{解法 2.} 直接比较积分被积函数:
\[
\int_a^b (x^2+1) e^{-x^2} \, dx \ge \int_a^b 2x e^{-x^2} \, dx = [-e^{-x^2}]_a^b = e^{-a^2} - e^{-b^2}.
\]
\end{solution}


\question 设 $f$ 是定义在区间 $[0,1]$ 上的连续实值函数。证明存在 $\xi \in [0,1]$,使得
\[
\int_{0}^{1} x^2 f(x) \, dx = \frac{1}{3} f(\xi).
\]

\begin{solution}
由于 $f$ 连续,它在 $[0,1]$ 上达到最小值和最大值,分别设为 $f(a)$ 和 $f(b)$,其中 $a,b \in [0,1]$。于是
\[
f(a) \int_{0}^{1} x^2 \, dx \le \int_{0}^{1} x^2 f(x) \, dx \le f(b) \int_{0}^{1} x^2 \, dx,
\]
即
\[
f(a) \le 3 \int_{0}^{1} x^2 f(x) \, dx \le f(b).
\]

由连续函数的中值定理可知,存在 $\xi \in [0,1]$,使得
\[
f(\xi) = 3 \int_{0}^{1} x^2 f(x) \, dx.
\]
\end{solution}

\question 设 $f: [0,1] \to (0,\infty)$ 是可积函数,且对所有 $x \in [0,1]$ 有
\[
f(x)\cdot f(1-x) = 1.
\]
证明
\[
\int_0^1 f(x)\,dx \geq 1.
\]

\begin{solution}
\textbf{方法一(使用算术-几何平均不等式):} 对任意 $x\in[0,1]$,由 AM-GM 不等式有
\[
f(x)+f(1-x) \geq 2\sqrt{f(x)f(1-x)} = 2.
\]
在区间 $[0,1/2]$ 上积分得
\[
\int_0^1 f(x)\,dx = \int_0^{1/2} f(x)\,dx + \int_0^{1/2} f(1-x)\,dx = \int_0^{1/2} (f(x)+f(1-x))\,dx \geq \int_0^{1/2} 2\,dx = 1.
\]

\textbf{方法二(使用 Cauchy-Schwarz 不等式):} 由条件可得
\[
\int_0^1 f(x)\,dx = \int_0^1 f(1-x)\,dx = \int_0^1 \frac{1}{f(x)}\,dx.
\]
于是
\[
\left(\int_0^1 f(x)\,dx\right)^2 = \int_0^1 f(x)\,dx \cdot \int_0^1 \frac{1}{f(x)}\,dx \geq \left(\int_0^1 1\,dx\right)^2 = 1,
\]
其中不等式来自 Cauchy-Schwarz,因此
\[
\int_0^1 f(x)\,dx \geq 1.
\]
\end{solution}

\question 设 $f:(-1,1)\to\mathbb{R}$ 是二阶可导函数,满足
\[
2f'(x)+xf''(x)\ge1,\quad x\in(-1,1).
\]
证明
\[
\int_{-1}^1 x f(x)\,dx \ge \frac{1}{3}.
\]

\begin{solution}
令
\[
g(x)=x f(x)-\frac{x^2}{2}.
\]
则
\[
g''(x)=2f'(x)+x f''(x)-1\ge0,
\]
因此 $g$ 是凸函数。

用 $g$ 在 $x=0$ 处的切线估计 $g$。设 $g'(0)=a$,则由凸性知
\[
g(x)\ge g(0)+g'(0)x=ax.
\]
于是
\begin{align*}
\int_{-1}^1 x f(x)\,dx
&=\int_{-1}^1\left(g(x)+\frac{x^2}{2}\right)\,dx \\
&\ge\int_{-1}^1\left(ax+\frac{x^2}{2}\right)\,dx
=\frac{1}{3}.
\end{align*}
这就证明了结论。
\end{solution}

\question 设 $f$ 为闭区间 $[0,1]$ 上的实值连续函数。已知
\[
\int_{0}^{1} f(x)\,dx = \frac{\pi}{4},
\]
证明存在 $y$ 满足 $0<y<1$ 且
\[
\frac{1}{1+y} < f(y) < \frac{1}{2y}。
\]

\begin{solution}
注意到
\[
\int_{0}^{1} \frac{1}{1+x^2}\,dx = \arctan(1) - \arctan(0) = \frac{\pi}{4}。
\]

令 $g(x) = f(x) - \frac{1}{1+x^2}$。若 $g(x)$ 在 $(0,1)$ 上恒为 $0$,则 $f(y) = \frac{1}{1+y^2}$ 对所有 $y \in (0,1)$ 都成立。若 $g$ 在 $(0,1)$ 上不恒为 $0$,则由于
\[
\int_{0}^{1} g(x)\,dx = 0,
\]
存在 $x_0, x_1 \in (0,1)$ 使得 $g(x_0) < 0$ 且 $g(x_1) > 0$。由 $g$ 在 $(0,1)$ 上连续,介值定理保证存在某个 $y$ 在 $x_0$ 与 $x_1$ 之间,使得 $g(y)=0$。于是 $f(y) = \frac{1}{1+y^2}$。

无论哪种情况,都存在 $y \in (0,1)$ 使得 $f(y) = \frac{1}{1+y^2}$。由于 $0<y<1$,有
\[
2y < 1+y^2 < 1+y,
\]
从而得到
\[
\frac{1}{1+y} < f(y) < \frac{1}{2y},
\]
证毕。
\end{solution}

\question 设函数 $f$ 在区间 $[a,b]$ 上具有连续导数,且满足 $f(a)=f(b)=0$。证明
\[
\max_{a\le x\le b} |f'(x)| \ge \frac{4}{(b-a)^2}\left|\int_a^b f(x)\,dx\right|.
\]

\begin{solution}
由中值定理,对任意 $x\in(a,b)$,有
\[
f(x)=f(x)-f(a)=f'(\xi_1)(x-a)=f(x)-f(b)=f'(\xi_2)(x-b),
\]
其中 $\xi_1\in(a,x),\xi_2\in(x,b)$。

设
\[
M=\max_{a\le x\le b}|f'(x)|.
\]
则
\[
|f(x)|\le M(x-a), \qquad |f(x)|\le M(b-x).
\]

因此
\[
\frac{4}{(b-a)^2}\int_a^b |f(x)|\,dx
\]
\begin{align*}
&\le \frac{4}{(b-a)^2}
\left(
\int_a^{(a+b)/2} M(x-a)\,dx
+\int_{(a+b)/2}^b M(b-x)\,dx
\right) \\
&= \frac{4}{(b-a)^2}
\left(
\frac{(b-a)^2}{8}M+\frac{(b-a)^2}{8}M
\right)
= M.
\end{align*}

又因为
\[
\left|\int_a^b f(x)\,dx\right|\le \int_a^b |f(x)|\,dx,
\]
于是
\[
\max_{a\le x\le b}|f'(x)|
\ge \frac{4}{(b-a)^2}\left|\int_a^b f(x)\,dx\right|.
\]
\end{solution}




\question 设 $a$ 为实数,已知函数
\[
f(x) = 4x^2 - 3a x + 4 \int_0^1 (t f(t))\,dt,
\]
及
\[
g(x) = x^2 + 4x + a - \int_0^x ((t+1) g'(t))\,dt.
\]
若方程 $f(x) - x \cdot g(x) = 0$ 有两相异实根 $\alpha, \beta$,且 $\alpha < \beta$,求
\[
\frac{1}{\beta - \alpha} \int_\alpha^\beta (3x^2 - 2 a x + a^2) \, dx
\]
的最小值。

\begin{solution}
设常数
\[
C = \int_0^1 t f(t) \, dt,
\]
则
\[
f(x) = 4x^2 - 3 a x + 4 C.
\]
计算 $C$:
\[
C = \int_0^1 t (4 t^2 - 3 a t + 4 C) dt = \int_0^1 (4 t^3 - 3 a t^2 + 4 C t) dt = \left[ t^4 - a t^3 + 2 C t^2 \right]_0^1 = 1 - a + 2 C.
\]
整理得
\[
C = 1 - a + 2 C \implies C = a - 1.
\]
因此
\[
f(x) = 4 x^2 - 3 a x + 4 a - 4.
\]

对 $g(x)$,由条件有
\[
g'(x) = 2 x + 4 - (x + 1) g'(x),
\]
整理得
\[
g'(x) + (x+1) g'(x) = 2 x + 4 \implies (x + 2) g'(x) = 2 x + 4,
\]
即
\[
g'(x) = \frac{2 x + 4}{x + 2} = 2.
\]

计算积分
\[
\int_0^x (t+1) g'(t) dt = \int_0^x 2 (t+1) dt = x^2 + 2 x,
\]
代入 $g(x)$:
\[
g(x) = x^2 + 4 x + a - (x^2 + 2 x) = 2 x + a.
\]

因此
\[
f(x) - x g(x) = (4 x^2 - 3 a x + 4 a - 4) - x (2 x + a) = 4 x^2 - 3 a x + 4 a - 4 - 2 x^2 - a x = 2 x^2 - 4 a x + 4 a - 4.
\]

设两根为 $\alpha, \beta$,满足
\[
\alpha + \beta = 2 a, \quad \alpha \beta = 2 a - 2.
\]

计算
\[
\frac{1}{\beta - \alpha} \int_\alpha^\beta (3 x^2 - 2 a x + a^2) dx = \frac{1}{\beta - \alpha} \left[ x^3 - a x^2 + a^2 x \right]_\alpha^\beta = \frac{1}{\beta - \alpha} \left( \beta^3 - \alpha^3 - a (\beta^2 - \alpha^2) + a^2 (\beta - \alpha) \right).
\]

利用因式分解:
\[
\beta^3 - \alpha^3 = (\beta - \alpha)(\beta^2 + \alpha \beta + \alpha^2), \quad \beta^2 - \alpha^2 = (\beta - \alpha)(\beta + \alpha),
\]
因此表达式为
\[
(\beta^2 + \alpha \beta + \alpha^2) - a (\beta + \alpha) + a^2.
\]

利用已知根的和与积:
\[
\beta + \alpha = 2 a, \quad \alpha \beta = 2 a - 2,
\]
以及
\[
\beta^2 + \alpha^2 = (\beta + \alpha)^2 - 2 \alpha \beta = (2 a)^2 - 2 (2 a - 2) = 4 a^2 - 4 a + 4.
\]

所以
\[
\beta^2 + \alpha \beta + \alpha^2 = (\beta^2 + \alpha^2) + \alpha \beta = (4 a^2 - 4 a + 4) + (2 a - 2) = 4 a^2 - 2 a + 2.
\]

代入原式:
\[
4 a^2 - 2 a + 2 - a \cdot 2 a + a^2 = 4 a^2 - 2 a + 2 - 2 a^2 + a^2 = 3 a^2 - 2 a + 2.
\]

令
\[
h(a) = 3 a^2 - 2 a + 2 = 3 \left(a - \frac{1}{3}\right)^2 + \frac{5}{3}.
\]

因此最小值为
\[
\frac{5}{3}.
\]
\textcolor{red}{(待验证)}
\end{solution}

\question 已知 $\int_0^2 f(x) dx = 1,f(2) = \frac{1}{2},f'(2) = 0$,计算 $[\int_0^1 x^2 f''(2x) dx]$。
\begin{solution}
首先进行变量代换,令 $t = 2x$,则 $dt = 2dx$:
\[ I = [\int_0^1 x^2 f''(2x) dx] = [\frac{1}{2} \int_0^2 (\frac{t}{2})^2 f''(t) dt] = [\frac{1}{8} \int_0^2 t^2 f''(t) dt] \]
使用分部积分法,令 $u = t^2,dv = f''(t)dt$:
\[ I = \frac{1}{8} ([t^2 f'(t)]_0^2 - \int_0^2 2t f'(t) dt) \]
代入 $f'(2) = 0$:
\[ I = \frac{1}{8} (0 - 2 \int_0^2 t f'(t) dt) = -\frac{1}{4} [\int_0^2 t f'(t) dt] \]
再次使用分部积分法,令 $u = t,dv = f'(t)dt$:
\[ I = -\frac{1}{4} ([t f(t)]_0^2 - \int_0^2 f(t) dt) \]
代入 $f(2) = \frac{1}{2}$ 和 $\int_0^2 f(t) dt = 1$:
\[ I = -\frac{1}{4} (2 \cdot \frac{1}{2} - 1) = -\frac{1}{4} (1 - 1) = 0 \]
最终结果为 $0$。
\end{solution}

    \question 坐标平面上,$A,B$ 两点分别在直线 $L_1:x=-\dfrac{3}{2}$ 与 $L_2:x=\dfrac{3}{2}$ 上。$\overline{AB} \perp L_1$,且 $M,N$ 为 $\overline{AB}$ 的三等分点,并满足 $\overline{AM} = \overline{MN} = \overline{NB}$。设 $\Gamma_1$ 为以 $L_1$ 为准线,$N$ 为焦点的抛物线;$\Gamma_2$ 为以 $L_2$ 为准线,$N$ 为顶点的抛物线,求 $\Gamma_1$ 与 $\Gamma_2$ 所围区域的面积。
    \ifprintanswers
    \begin{figure}[H]
        \centering
        \includegraphics[width=0.4\textwidth]{images/image31.png}
    \end{figure}
    \fi
    \begin{solution}
        所围面积为
        \[
        I = 4\int_0^{\frac{1}{2}} \sqrt{-4(x-\frac{1}{2})}\,dx = 8 \int_0^{\frac{1}{2}} \sqrt{\frac{1}{2}-x} \,dx
        \]
        令 $u = \dfrac{1}{2} - x , du = -dx$,则
        \[
        I = 8\int_0^{\frac{1}{2}} \sqrt{u}\,du 
        = 8 \left[ \frac{2}{3} u^{\frac{3}{2}} \right]_0^{\frac{1}{2}} 
        = \frac{4\sqrt{2}}{3}
        \]
    \end{solution}

    \question 已知四条抛物线 $\Gamma_1: y=x^2+a$, $\Gamma_2: y = -x^2 -a$, $\Gamma_3: y^2 = x-a$, $\Gamma_4: y^2 = -x-a$,其中 $a$ 为正实数,若任相邻两条抛物线均相切,试求这四条抛物线所围成之区域面积。
    \ifprintanswers
    \begin{figure}[H]
        \centering
        \includegraphics[width=0.4\textwidth]{images/image34.png}
    \end{figure}
    \fi
    \begin{solution}
        $\Gamma_1,\Gamma_3$的共切点在$x=y$上,故知
        \[
        x^2 - x + a = 0 
        \] 
        恰有一根,其中判别式为
        \[ 1 - 4a = 0 \Rightarrow a = \frac{1}{4}
        \] 
        故四个切点为
        \[
        \left(\frac{1}{2}, \frac{1}{2}\right),\left(\frac{1}{2}, -\frac{1}{2}\right),\left(-\frac{1}{2}, \frac{1}{2}\right),\left(-\frac{1}{2}, -\frac{1}{2}\right),
        \]
        所围面积为
        \[
        8 \int_0^{\frac{1}{2}} (x^2 - x + \frac{1}{4})\,dx  =8\left[\frac{x^3}{3}-\frac{x^2}{2}+\frac{x}{4}\right]^{\frac12}_0 =\frac{1}{3}
        \]
    \end{solution}

    \question 已知曲线
\[
2x^2 + 2xy + y^2 = 50,
\]
求被 $x$ 轴和曲线上 $y \ge 0$ 部分围成的有限区域面积。

\begin{solution}

\textbf{步骤 1: 找 $x$、$y$ 截距并作图}  

将曲线改写为 $y$ 的形式:
\[
y^2 + 2xy + 2x^2 = 50 \implies (y+x)^2 + x^2 = 50 \implies (y+x)^2 = 50 - x^2
\]
\[
y = -x \pm \sqrt{50-x^2}
\]

\textbf{步骤 2: 找垂直切线点 $P$}  

对曲线求导:
\[
\frac{d}{dx}(2x^2 + 2xy + y^2) = 0 \implies 4x + 2y + 2x\frac{dy}{dx} + 2y\frac{dy}{dx} = 0
\]
\[
\implies 2(x+y) + 2(x+y)\frac{dy}{dx} = 0 \implies y+x = 0 \implies y=-x
\]

代入曲线方程求 $x$:
\[
2x^2 + 2x(-x) + (-x)^2 = 50 \implies x^2 = 50 \implies x = \pm 5\sqrt{2}
\]

所以垂直切线点为
\[
P = (-5\sqrt{2}, 5\sqrt{2})
\]

\textbf{步骤 3: 准备积分表达面积}  

使用 $y = -x \pm \sqrt{50-x^2}$,考虑 $y \ge 0$ 部分。面积由两部分曲线和 $x$ 轴围成:
\[
A = \int_{-5\sqrt{2}}^{5\sqrt{2}} [-x+\sqrt{50-x^2}] \, dx - \int_{-5\sqrt{2}}^{-5} [-x-\sqrt{50-x^2}] \, dx
\]

拆分积分:
\[
A = \int_{-5\sqrt{2}}^{5\sqrt{2}} -x \, dx + \int_{-5\sqrt{2}}^{5\sqrt{2}} \sqrt{50-x^2} \, dx + \int_{-5\sqrt{2}}^{-5} x \, dx + \int_{-5\sqrt{2}}^{-5} \sqrt{50-x^2} \, dx
\]

\textbf{步骤 4: 使用三角代换计算 $\sqrt{50-x^2}$ 积分}  

令
\[
x = \sqrt{50} \sin\theta \implies dx = \sqrt{50} \cos\theta \, d\theta
\]
\[
\sqrt{50-x^2} = \sqrt{50}\cos\theta
\]

积分变为:
\[
\int \sqrt{50-x^2} \, dx = \int 50 \cos^2\theta \, d\theta = \int 25 + 25\cos 2\theta \, d\theta = 25\theta + \frac{25\sin 2\theta}{2}
\]

对应的 $\theta$ 值:
\[
x = 5\sqrt{2} \implies \theta = \frac{\pi}{2}, \quad x = -5\sqrt{2} \implies \theta = -\frac{\pi}{2}, \quad x = -5 \implies \theta = -\frac{\pi}{4}
\]

\textbf{步骤 5: 代入计算面积}  
\begin{align*}
A &= \left[-\frac{1}{2}x^2\right]_{-5\sqrt{2}}^{5} + \left[25\theta + \frac{25\sin 2\theta}{2}\right]_{-\pi/2}^{-\pi/4} + \left[25\theta + \frac{25\sin 2\theta}{2}\right]_{-\pi/2}^{\pi/2} \\
&= \left[\frac{-25}{2} - \frac{-50}{2}\right] + \left[ -\frac{25\pi}{4} - \frac{25}{2}(-1) - (-\frac{25\pi}{2}+0) \right] + \left[ \frac{25\pi}{2}-(-\frac{25\pi}{2}) \right] \\
&= 25\pi
\end{align*}

\textbf{最终答案:}
\[
\boxed{A = 25\pi}
\]

\end{solution}


    \question 座标平面上,满足联立不等式
    \[
    \begin{cases}
    x^2 + (y-1)^2 \le 1 \\
    x - y \le 0 \\
    x - y \ge -2
    \end{cases}
    \]
    的解区域为 $S$,求区域 $S$ 绕 $x$ 轴旋转一圈所得立体的体积。
    \ifprintanswers
    \begin{figure}[H]
        \centering
        \includegraphics[width=0.4\textwidth]{images/image130.jpg}
    \end{figure}
    \fi
    \begin{solution}
        联立 $x^2 + (y-1)^2 =1$与 $y=x+2$得
        \[
        A(0,2), \ D(-1,1)
        \]
        联立 $x^2 + (y-1)^2 =1$与 $y=x$得
        \[
        B(1,1), \ C(0,0)
        \]
        如上图,左半部绕 $x$ 轴旋转体的体积为
        \[
        \begin{aligned}
        V_1 &= \pi \int_{-1}^{0} \Big[(x+2)^2 - (1 - \sqrt{1-x^2})^2\Big]\, dx \\
        &= \pi \int_{-1}^{0} \Big[2x^2 + 4x + 2 + 2\sqrt{1-x^2}\Big]\, dx \\
        &= \pi \left[ \frac{2}{3}x^3 + 2x^2 + 2x + (\sqrt{1-x^2} + \arcsin x) \right]_{-1}^{0} = \frac{1}{2}\pi^2 + \frac{5}{3}\pi
        \end{aligned}
        \]
        上图右半部绕 $x$ 轴旋转体的体积为
        \[
        \begin{aligned}
        V_2 &= \pi \int_{0}^{1} \Big[(1 + \sqrt{1-x^2})^2 - x^2\Big]\, dx \\
        &= \pi \int_{0}^{1} \Big[ 2 - 2x^2 + 2\sqrt{1-x^2} \Big]\, dx \\
        &= \pi \left[ 2x - \frac{2}{3}x^3 + \sqrt{1-x^2} + \arcsin x \right]_{0}^{1} = \frac{1}{2}\pi^2 + \frac{1}{3}\pi
        \end{aligned}
        \]
        因此总体积为
        \[
        V = V_1 + V_2 = \pi^2 + 2\pi
        \]
    \end{solution}

    \question 在坐标平面上,求由不等式 
    \[
    \frac{x^2}{4} + y^2 \le 1, \quad y+1 \ge \left(\frac{\sqrt{3}}{2}+1\right)x, \quad y+1 \ge -\left(\frac{\sqrt{3}}{2}+1\right)x
    \]
    所围成的图形面积。
    \ifprintanswers
    \begin{figure}[H]
        \centering
        \includegraphics[width=0.6\textwidth]{images/image120.jpg}
    \end{figure}
    \fi
    \begin{solution}
        三方程的交点为
        \[
        A\left(1, \frac{\sqrt{3}}{2}\right), \quad B\left(-1, \frac{\sqrt{3}}{2}\right), \quad C(0,-1)
        \]
        图形关于 $y$ 轴对称,于是所求面积为
        \[
        S = 2 \left[ \int_0^1 \sqrt{1 - \frac{x^2}{4}} \, dx - \int_0^1 \left( \frac{\sqrt{3}+2}{2}x - 1 \right) dx \right]
        \]
        令 $x = 2 \sin \theta ,dx = 2 \cos \theta \, d\theta$,则
        \[
        \int_0^1 \sqrt{1 - \frac{x^2}{4}} \, dx = \int_0^{\frac{\pi}{6}} \sqrt{1 - \sin^2 \theta} \cdot 2\cos \theta \, d\theta = \int_0^{\frac{\pi}{6}} (1 + \cos 2\theta) \, d\theta = \frac{\sqrt{3}}{4} + \frac{\pi}{6}
        \]
        且
        \[
        \int_0^1 \left( \frac{\sqrt{3}+2}{2}x - 1 \right) dx = \left[\frac{\sqrt{3}+2}{4}x^2-x\right]_0^1=\frac{\sqrt{3}-2}{4}
        \]
        所以
        \[
        S = 2\left( \frac{\pi}{6} + \frac{\sqrt{3}}{4} - \frac{\sqrt{3}-2}{4} \right) = 1 + \frac{\pi}{3}
        \]
    \end{solution}

    \question 计算积分
\[
I(\theta) = \int_{-1}^{1} \frac{\sin \theta \, dx}{1 - 2x \cos \theta + x^2},
\]
并确定 $0 \le \theta \le 2\pi$ 范围内函数 $I(\theta)$ 不连续的点。

\begin{solution}
用恒等式 $1 = \cos^2 \theta + \sin^2 \theta$,并令
\[
u = x - \cos \theta, \quad du = dx.
\]

则积分变为
\[
I(\theta) = \int_{-1-\cos\theta}^{1-\cos\theta} \frac{\sin\theta \, du}{u^2 + \sin^2\theta} = \left. \tan^{-1} \frac{u}{\sin\theta} \right|_{-1-\cos\theta}^{1-\cos\theta}.
\]

因此
\[
I(\theta) = \tan^{-1} \frac{1-\cos\theta}{\sin\theta} - \tan^{-1} \frac{-1-\cos\theta}{\sin\theta}, \quad (\sin\theta \neq 0).
\]

利用二倍角公式
\[
1-\cos\theta = 2\sin^2(\theta/2), \quad 1+\cos\theta = 2\cos^2(\theta/2), \quad \sin\theta = 2\sin(\theta/2)\cos(\theta/2),
\]
得到
\[
\frac{1-\cos\theta}{\sin\theta} = \frac{2\sin^2(\theta/2)}{2\sin(\theta/2)\cos(\theta/2)} = \tan(\theta/2),
\quad 
\frac{-1-\cos\theta}{\sin\theta} = -\frac{2\cos^2(\theta/2)}{2\sin(\theta/2)\cos(\theta/2)} = -\cot(\theta/2) = \tan(\pi/2 + \theta/2).
\]

因此
\[
I(\theta) = \tan^{-1}(\tan(\theta/2)) - \tan^{-1}(\tan(\pi/2 + \theta/2)).
\]

注意 $\tan^{-1}$ 的主值在 $(-\pi/2, \pi/2)$。例如:
\[
\tan^{-1}(\tan 20^\circ) = 20^\circ, \quad 
\tan^{-1}(\tan 100^\circ) = -80^\circ, \quad
\tan^{-1}(\tan 220^\circ) = 40^\circ, \quad
\tan^{-1}(\tan 290^\circ) = -70^\circ.
\]

因此,如果 $0 < \theta < \pi$,则
\[
I(\theta) = \theta/2 - (-(\pi/2 - \theta/2)) = \pi/2.
\]

如果 $\pi < \theta < 2\pi$,则
\[
I(\theta) = -(\pi - \theta/2) - (\theta/2 - \pi/2) = -\pi/2.
\]

因此,函数 $I(\theta)$ 为
\[
I(\theta) = 
\begin{cases}
\pi/2, & 0 < \theta < \pi, \\
-\pi/2, & \pi < \theta < 2\pi.
\end{cases}
\]

不连续点为 $\theta = 0, \pi, 2\pi$。
\end{solution}

\question 计算极限
\[
\lim_{x \to 0^+} \int_{x}^{2x} \frac{\sin^m t}{t^n} \, dt \quad (m,n \in \mathbb{N}).
\]

\begin{solution}
注意到当 $t \to 0^+$ 时 $\frac{\sin t}{t} \to 1$,且在 $(0, \pi)$ 上 $\frac{\sin t}{t}$ 单调递减。因此,对于 $x \in (0, \pi/2)$ 且 $t \in [x,2x]$,有
\[
\frac{\sin 2x}{2x} < \frac{\sin t}{t} < 1.
\]
于是
\[
\left(\frac{\sin 2x}{2x}\right)^m \int_x^{2x} t^{m-n} \, dt
<
\int_x^{2x} \frac{\sin^m t}{t^n} \, dt
<
\int_x^{2x} t^{m-n} \, dt.
\]

对积分做换元 $t = xu$ 得
\[
\int_x^{2x} t^{m-n} \, dt = x^{m-n+1} \int_1^2 u^{m-n} \, du.
\]

注意到 $\left(\frac{\sin 2x}{2x}\right)^m \to 1$,于是极限取决于 $x^{m-n+1}$ 的幂:
\[
\lim_{x \to 0^+} \int_x^{2x} \frac{\sin^m t}{t^n} \, dt =
\begin{cases}
0, & m-n+1>0 \ (\text{即 } m \ge n),\\[2mm]
\ln 2, & m-n+1=0 \ (\text{即 } n-m=1),\\[1mm]
+\infty, & m-n+1<0 \ (\text{即 } n-m>1).
\end{cases}
\]

\end{solution}

\question
已知 $a$ 和 $b$ 是不同实常数,$\lambda$ 是实参数。证明
\[
\left[ \int_{a}^{b} f(x)g(x) \, dx \right]^2 \leq \left[ \int_{a}^{b} [f(x)]^2 \, dx \right]\left[ \int_{a}^{b} [g(x)]^2 \, dx \right]
\]
\begin{solution}
a) 由关系
\[
[\lambda f(x)+g(x)]^2 \geq 0
\]
展开平方:
\[
\lambda^2 [f(x)]^2 + 2\lambda f(x)g(x) + [g(x)]^2 \geq 0
\]
对 $x$ 从 $a$ 到 $b$ 积分:
\[
\int_{a}^{b} (\lambda^2 [f(x)]^2 + 2\lambda f(x)g(x) + [g(x)]^2) \, dx \geq 0
\]
利用积分线性:
\[
\lambda^2 \int_{a}^{b} [f(x)]^2 \, dx + 2\lambda \int_{a}^{b} f(x)g(x) \, dx + \int_{a}^{b} [g(x)]^2 \, dx \geq 0
\]

b) 该关于 $\lambda$ 的二次不等式要求判别式非正:
\[
A = \int_{a}^{b} [f(x)]^2 \, dx, \quad B = 2\int_{a}^{b} f(x)g(x) \, dx, \quad C = \int_{a}^{b} [g(x)]^2 \, dx
\]
\[
B^2 - 4AC \leq 0 \implies \left[ \int_{a}^{b} f(x)g(x) \, dx \right]^2 \leq \left[ \int_{a}^{b} [f(x)]^2 \, dx \right]\left[ \int_{a}^{b} [g(x)]^2 \, dx \right]
\]

c) 令 $f(x) = \sqrt{\sin x},g(x)=1$:
\[
\left[ \int_{0}^{\pi/2} \sqrt{\sin x} \, dx \right]^2 \leq \left[ \int_{0}^{\pi/2} \sin x \, dx \right]\left[ \int_{0}^{\pi/2} 1 \, dx \right]
\]
\[
\left[ \int_{0}^{\pi/2} \sqrt{\sin x} \, dx \right]^2 \leq [-\cos x]_{0}^{\pi/2} \cdot [x]_{0}^{\pi/2} = 1 \cdot \frac{\pi}{2}
\]
\[
\int_{0}^{\pi/2} \sqrt{\sin x} \, dx \leq \sqrt{\frac{\pi}{2}}
\]

d) 令 $f(x) = (\sin x)^{1/4},g(x) = \cos x$:
\[
\left[ \int_{0}^{\pi/2} (\sin x)^{1/4}\cos x \, dx \right]^2 \leq \left[ \int_{0}^{\pi/2} (\sin x)^{1/2} \, dx \right]\left[ \int_{0}^{\pi/2} \cos^2 x \, dx \right]
\]
LHS 积分用代换 $u=\sin x, du=\cos x \, dx$:
\[
\int_{0}^{\pi/2} (\sin x)^{1/4}\cos x \, dx = \frac{4}{5}
\]
RHS 第二个积分:
\[
\int_{0}^{\pi/2} \cos^2 x \, dx = \int_{0}^{\pi/2} \frac{1+\cos 2x}{2} \, dx = \frac{\pi}{4}
\]
代入不等式:
\[
\left(\frac{4}{5}\right)^2 \leq \left[ \int_{0}^{\pi/2} \sqrt{\sin x} \, dx \right] \cdot \frac{\pi}{4}
\]
整理:
\[
\int_{0}^{\pi/2} \sqrt{\sin x} \, dx \geq \frac{16}{25} \cdot \frac{4}{\pi} = \frac{64}{25\pi}
\]
\end{solution}

\end{questions}

\pagebreak

\begin{center}
  {\fontsize{30pt}{26pt}\selectfont
    \hypertarget{积分技巧}{积分技巧} \label{积分技巧}
  }
\end{center}
\separator
\vspace{1pt}
\begin{questions}

%substitution methods
\question 计算 $[\int_2^4 (\log_x 2 - \frac{\log_x^2 2}{\ln 2}) dx]$。
\begin{solution}
利用换底公式简化被积函数:
\[\log_x 2 - \frac{(\frac{\ln 2}{\ln x})^2}{\ln 2} = \frac{\ln 2}{\ln x} - \frac{\ln 2}{(\ln x)^2} = \ln 2 (\frac{1}{\ln x} - \frac{1}{\ln^2 x})\]
由于 $\frac{d}{dx}(\frac{x}{\ln x}) = \frac{\ln x - 1}{\ln^2 x} = \frac{1}{\ln x} - \frac{1}{\ln^2 x}$,
该积分的原始函数为:
\[I = \ln 2 [\frac{x}{\ln x}]_2^4 = \ln 2 (\frac{4}{\ln 4} - \frac{2}{\ln 2})\]
已知 $\ln 4 = 2 \ln 2$,代入得:
\[I = \ln 2 (\frac{4}{2 \ln 2} - \frac{2}{\ln 2}) = \ln 2 (\frac{2}{\ln 2} - \frac{2}{\ln 2}) = 0\]
\end{solution}

\question
计算积分substitution methods
\[
\int_{80^4}^{15^4} \frac{1}{x^2 (1+x^4)^{\frac{3}{4}}} \, dx
\]

\begin{solution}
因为积分区间为正,可将被积式中的 $x^4$ 提出:
\[
\int_{80^4}^{15^4} \frac{1}{x^2 (1+x^4)^{\frac{3}{4}}} \, dx
= \int_{80^4}^{15^4} \frac{1}{x^2 (x^4)^{\frac{3}{4}} (x^{-4}+1)^{\frac{3}{4}}} \, dx
= \int_{80^4}^{15^4} \frac{1}{x^2 x^3 (x^{-4}+1)^{\frac{3}{4}}} \, dx
= \int_{80^4}^{15^4} x^{-5} (1+x^{-4})^{-\frac{3}{4}} \, dx
\]

注意到
\[
\frac{d}{dx} \left[ (1+x^{-4})^{\frac{1}{4}} \right] 
= \frac{1}{4} (1+x^{-4})^{-\frac{3}{4}} \cdot (-4 x^{-5}) 
= - x^{-5} (1+x^{-4})^{-\frac{3}{4}}
\]

因此
\[
\int_{80^4}^{15^4} x^{-5} (1+x^{-4})^{-\frac{3}{4}} \, dx
= - \left[ (1+x^{-4})^{\frac{1}{4}} \right]_{80^4}^{15^4}
= - \left[ (1+15^{-4})^{\frac{1}{4}} - (1+80^{-4})^{\frac{1}{4}} \right]
= - (16^{\frac{1}{4}} - 81^{\frac{1}{4}})
= - (2-3)
= 1
\]

最终结果:
\[
\int_{80^4}^{15^4} \frac{1}{x^2 (1+x^4)^{\frac{3}{4}}} \, dx = 1
\]
\end{solution}

\question 计算积分
\[
\int_{0}^{1} \frac{1}{(x^{2} + 4x^{3})^{\frac{2}{4}}} \, dx
\]
\begin{solution}
先将分母中的根式因式分解:
\[
x^{2} + 4x^{3} = x^{2}(1 + 4x) = [x^{\frac{1}{2}}(x^{\frac{1}{2}}+4)^{\frac{3}{4}}]^{?} \quad \text{(整理成适合代换的形式)}
\]

更简便地处理:
\[
\int_{0}^{1} \frac{1}{(x^{2}+4x^{3})^{\frac{1}{2}}} \, dx
= \int_{0}^{1} \frac{1}{x^{\frac{1}{2}} (x^{\frac{1}{2}}+4)^{\frac{3}{4}}} \, dx
\]

作代换:
\[
u = x^{\frac{1}{2}} \implies x = u^2, \quad dx = 2u \, du
\]

积分上下限不变($x=0\to u=0$, $x=1\to u=1$):
\[
\int_{0}^{1} \frac{1}{u (u+4)^{\frac{3}{4}}} \cdot 2u \, du = \int_{0}^{1} \frac{2}{(u+4)^{\frac{3}{4}}} \, du
\]

直接积分:
\[
\int_{0}^{1} 2 (u+4)^{-\frac{3}{4}} \, du = \left[ 8 (u+4)^{\frac{1}{4}} \right]_{0}^{1} = 8 \left( 5^{\frac{1}{4}} - 4^{\frac{1}{4}} \right)
\]

\[
= 8 \left( \sqrt[4]{5} - \sqrt{2} \right)
\]
\end{solution}

\question
7) 计算定积分:$\int_{0}^{4} x^3 \sqrt{9+x^2} dx$
\begin{solution}
设 $u = 9+x^2, du = 2x dx$:
\begin{align*}
I &= \frac{1}{2} \int_{9}^{25} (u-9) \sqrt{u} du = \frac{1}{2} \left[ \frac{2}{5}u^{5/2} - 6u^{3/2} \right]_{9}^{25} \\
&= \left[ \frac{1}{5}u^{5/2} - 3u^{3/2} \right]_{9}^{25} = (625 - 375) - (\frac{243}{5} - 81) = 282.4
\end{align*}
\textbf{结果:} $282.4$
\end{solution}

\question
1) 计算不定积分:\[ \int \frac{1}{1+x^4} dx \]

\begin{solution}
首先将被积函数拆分:
\[ \int \frac{1}{1+x^4} dx = \frac{1}{2} \int \frac{x^2+1}{x^4+1} dx - \frac{1}{2} \int \frac{x^2-1}{x^4+1} dx \]

对于第一部分,分子分母同除以 $x^2$:
\[ \frac{1}{2} \int \frac{1+\frac{1}{x^2}}{x^2+\frac{1}{x^2}} dx = \frac{1}{2} \int \frac{d(x-\frac{1}{x})}{(x-\frac{1}{x})^2+2} \]
\[ = \frac{1}{2\sqrt{2}} \tan^{-1}\left(\frac{x-\frac{1}{x}}{\sqrt{2}}\right) = \frac{\sqrt{2}}{4} \tan^{-1}\left(\frac{x^2-1}{\sqrt{2}x}\right) \]

对于第二部分,同理可得:
\[ \frac{1}{2} \int \frac{1-\frac{1}{x^2}}{x^2+\frac{1}{x^2}} dx = \frac{1}{2} \int \frac{d(x+\frac{1}{x})}{(x+\frac{1}{x})^2-2} \]
利用对数积分公式:
\[ = \frac{1}{4\sqrt{2}} \ln\left| \frac{x+\frac{1}{x}-\sqrt{2}}{x+\frac{1}{x}+\sqrt{2}} \right| = \frac{\sqrt{2}}{8} \ln\left| \frac{x^2-\sqrt{2}x+1}{x^2+\sqrt{2}x+1} \right| \]

合并结果得到:
\[ \int \frac{1}{1+x^4} dx = \frac{\sqrt{2}}{4} \tan^{-1}\left(\frac{x^2-1}{\sqrt{2}x}\right) - \frac{\sqrt{2}}{8} \ln\left| \frac{x^2-\sqrt{2}x+1}{x^2+\sqrt{2}x+1} \right| + C \]
\end{solution}
%exponential
\question 计算积分
\[
\int \frac{4}{e^{3x}\sqrt{e^{2x}+4}} \, dx
\]
\begin{solution}
作代换:
\[
t = \frac{1}{e^x} \implies dx = -\frac{dt}{t}
\]

代入积分:
\[
\int \frac{4}{e^{3x}\sqrt{e^{2x}+4}} \, dx 
= \int \frac{4}{\frac{1}{t^3} \sqrt{\frac{1}{t^2}+4}} \left(-\frac{dt}{t}\right)
= \int \frac{-4 t^3}{\sqrt{\frac{1+4t^2}{t^2}}} \, dt
= \int \frac{-4 t^3}{\frac{\sqrt{1+4t^2}}{t}} \, dt
= \int \frac{-4 t^4}{\sqrt{1+4t^2}} \, dt
\]

整理:
\[
\frac{-4 t^4}{\sqrt{1+4t^2}} = -4 t (1+4t^2)^{-\frac{1}{2}} \, dt
\]

直接使用反链式法则或作代换 \(u = 1+4t^2\):
\[
\int -4 t (1+4t^2)^{-\frac{1}{2}} \, dt = -(1+4t^2)^{\frac{1}{2}} + C
\]

代回 \(t = \frac{1}{e^x}\):
\[
-(1+4t^2)^{\frac{1}{2}} + C = -\left(1 + \frac{4}{e^{2x}}\right)^{\frac{1}{2}} + C
= -\left(\frac{e^{2x}+4}{e^{2x}}\right)^{\frac{1}{2}} + C
= -\frac{\sqrt{e^{2x}+4}}{e^x} + C
\]
\end{solution}

\question 求
\[
\int_{0}^{\pi} \frac{1 + x \cos x}{x + e^{\sin x}} \, dx
\]
\begin{solution}
作代换
\[
u = 1 + x e^{\sin x}
\]
\[
\frac{du}{dx} = e^{\sin x} + x e^{\sin x} \cos x = e^{\sin x} (1 + x \cos x)
\]
\[
dx = \frac{du}{e^{\sin x} (1 + x \cos x)}
\]

积分的上下限变换
\[
x = 0 \implies u = 1
\]
\[
x = \pi \implies u = 1 + \pi
\]

代入积分
\[
\int_{0}^{\pi} \frac{1 + x \cos x}{x + e^{\sin x}} \, dx
= \int_{1}^{1+\pi} \frac{1 + x \cos x}{x + e^{\sin x}} \cdot \frac{du}{e^{\sin x} (1 + x \cos x)}
= \int_{1}^{1+\pi} \frac{1}{u} \, du
\]

计算积分
\[
[\ln|u|]_{1}^{1+\pi} = \ln(1+\pi) - \ln 1 = \ln(1+\pi)
\]
\end{solution}


\question 计算积分
\[
\int_{0}^{\ln 2} \sqrt{e^x - 1} \, dx
\]
\begin{solution}
令
\[
u = \sqrt{e^x - 1} \implies u^2 = e^x - 1, \quad 2u\,du = e^x dx, \quad dx = \frac{2u}{u^2+1} \, du.
\]

积分上下限变为:
\[
x = 0 \implies u = 0, \quad x = \ln 2 \implies u = \sqrt{2-1} = 1.
\]

原积分变为:
\[
\int_{0}^{1} u \cdot \frac{2u}{u^2+1} \, du = \int_{0}^{1} \frac{2u^2}{u^2+1} \, du.
\]

通过拆分:
\[
\frac{2u^2}{u^2+1} = 2 - \frac{2}{u^2+1}.
\]

因此积分为:
\[
\int_{0}^{1} \left(2 - \frac{2}{u^2+1}\right) du = \left[2u - 2\arctan u \right]_{0}^{1} = 2 - 2 \cdot \frac{\pi}{4} = 2 - \frac{\pi}{2}.
\]

\[
\therefore \int_{0}^{\ln 2} \sqrt{e^x - 1} \, dx = 2 - \frac{\pi}{2}.
\]
\end{solution}

\question 计算积分
\[
\int_{0}^{\infty} \frac{e^{8x}-e^{2x}}{(e^{8x}+3)(e^{2x}+3)} \, dx
\]
\begin{solution}
先通过部分分式分解(观察得出):
\[
\frac{e^{8x}-e^{2x}}{(e^{8x}+3)(e^{2x}+3)} = -\frac{1}{e^{8x}+3} + \frac{1}{e^{2x}+3}
\]

于是积分可写为:
\[
\int_{0}^{\infty} \frac{e^{8x}-e^{2x}}{(e^{8x}+3)(e^{2x}+3)} \, dx
= \int_{0}^{\infty} \frac{1}{e^{2x}+3} \, dx - \int_{0}^{\infty} \frac{1}{e^{8x}+3} \, dx
\]

先计算第一个积分:
\[
\int_{0}^{\infty} \frac{1}{e^{2x}+3} \, dx = \int_{0}^{\infty} \frac{e^{-2x}}{1+3 e^{-2x}} \, dx
= \left[-\frac{1}{6}\ln\left(1+3 e^{-2x}\right)\right]_{0}^{\infty}
= \frac{1}{6} \ln 4
\]

第二个积分:
\[
-\int_{0}^{\infty} \frac{1}{e^{8x}+3} \, dx = -\int_{0}^{\infty} \frac{e^{-8x}}{1+3 e^{-8x}} \, dx
= \left[-\frac{1}{24} \ln\left(1+3 e^{-8x}\right)\right]_{0}^{\infty}
= \frac{1}{24} \ln 4
\]

合并结果:
\[
\int_{0}^{\infty} \frac{e^{8x}-e^{2x}}{(e^{8x}+3)(e^{2x}+3)} \, dx
= \frac{1}{6} \ln 4 - \frac{1}{24} \ln 4
= \frac{3}{24} \ln 4
= \frac{1}{8} \ln 4
= \frac{1}{8} \ln (2^2)
= \frac{1}{4} \ln 2
\]
\end{solution}

\question 计算
\[
\int_{0}^{1} \frac{e^{x}(1-x)}{x^2+e^{2x}} \, dx
\]

\begin{solution}
首先观察积分形式,尝试将分母调整为 $1+(xe^{-x})^2$ 的形式:
\[
\int_{0}^{1} \frac{e^{x}(1-x)}{x^2+e^{2x}} \, dx = \int_{0}^{1} \frac{e^{x}(1-x)}{x^2+e^{2x}} \cdot \frac{e^{-2x}}{e^{-2x}} \, dx = \int_{0}^{1} \frac{e^{-x}(1-x)}{(xe^{-x})^2+1} \, dx
\]

设
\[
u = x e^{-x} \implies du = (1-x)e^{-x} dx
\]
积分上下限对应:
\[
x=0 \to u=0, \quad x=1 \to u = e^{-1}
\]

因此积分变为
\[
\int_{0}^{1} \frac{e^{-x}(1-x)}{(xe^{-x})^2+1} \, dx = \int_{0}^{e^{-1}} \frac{1}{u^2+1} \, du = [\arctan u]_{0}^{e^{-1}} = \arctan e^{-1} - \arctan 0
\]

最终得到
\[
\int_{0}^{1} \frac{e^{x}(1-x)}{x^2+e^{2x}} \, dx = \arctan \frac{1}{e}
\]
\end{solution}

    \question 已知 
    \[
    \int_{0}^{\infty} e^{-x^2}\,dx = \frac{\sqrt{\pi}}{2},
    \]
    求
    \[
    \int_{e}^{\infty} e^{-x^2} \ln(x^2) x^{(\ln(x^{-x^2}) + 2x^2)}\,dx
    \]
    \begin{solution}
        发现
        \begin{align*} 
        I &= \int_{e}^{\infty} e^{-x^2} \ln(x^2) x^{(\ln(x^{-x^2}) + 2x^2)}\,dx \\
        &= \int_{e}^{\infty} e^{-x^2} \ln(x^2) e^{\ln x(-x^2 \ln x + 2x^2)}\,dx \\
        &= 2\int_{e}^{\infty} e^{-(x\ln x-x)^2} \ln(x) \,dx
        \end{align*}
        设$u=x\ln x -x$,则$du=\left(\ln x +x\cdot\dfrac{1}{x}-1\right)\,dx=\ln x \,dx$,于是
        \[
        I=2\int_{0}^{\infty}e^{-u^2}\,du=2\cdot\frac{\sqrt{\pi}}{2}=\sqrt{\pi}
        \]
    \end{solution}

    \question 
计算不定积分:
\[ \int \left( \frac{\ln x - 1}{1 + (\ln x)^2} \right)^2 dx \]

\begin{solution}
这道题可以通过换元法简化。设 $u = \ln x$,则 $x = e^u,dx = e^u du$。

代入原式:
\begin{align*}
I &= \int \left( \frac{u-1}{1+u^2} \right)^2 e^u du \\
&= \int \frac{u^2 - 2u + 1}{(1+u^2)^2} e^u du \\
&= \int \left[ \frac{1+u^2}{(1+u^2)^2} - \frac{2u}{(1+u^2)^2} \right] e^u du \\
&= \int \left[ \frac{1}{1+u^2} - \frac{2u}{(1+u^2)^2} \right] e^u du
\end{align*}

注意到被积函数符合分部积分的特征 $\int [f(u) + f'(u)] e^u du = f(u)e^u + C$。
令 $f(u) = \frac{1}{1+u^2}$,则 $f'(u) = -\frac{2u}{(1+u^2)^2}$。

因此:
\begin{align*}
I &= \frac{1}{1+u^2} e^u + C \\
&= \frac{x}{1+(\ln x)^2} + C
\end{align*}

\textbf{结果:} $\frac{x}{(\ln x)^2 + 1} + C$ \quad
\end{solution}
%radicals
\question 
18) 计算不定积分 $\int x\sqrt{x+19} \, dx$

\begin{solution}
本题展示了两种正确的解法。

\textbf{解法一:利用凑微分与分配律}
\begin{align*}
\int x\sqrt{x+19} \, dx &= \int (x+19-19)(x+19)^{\frac{1}{2}} \, dx \\
&= \int (x+19)^{\frac{3}{2}} \, dx - 19\int (x+19)^{\frac{1}{2}} \, dx \\
&= \frac{2}{5}(x+19)^{\frac{5}{2}} - \frac{38}{3}(x+19)^{\frac{3}{2}} + C
\end{align*}

\textbf{解法二:分部积分法}
设 $u = x, dv = (x+19)^{\frac{1}{2}}dx \implies du = dx, v = \frac{2}{3}(x+19)^{\frac{3}{2}}$:
\begin{align*}
\int x\sqrt{x+19} \, dx &= \frac{2}{3}x(x+19)^{\frac{3}{2}} - \frac{2}{3}\int (x+19)^{\frac{3}{2}} \, dx \\
&= \frac{2}{3}x(x+19)^{\frac{3}{2}} - \frac{4}{15}(x+19)^{\frac{5}{2}} + C
\end{align*}
注:两结果形式不同但通过化简后实质相等。
\end{solution}

\question
20) 计算定积分 $I = \int_{3}^{6} (\sqrt{x+\sqrt{12x-36}} + \sqrt{x-\sqrt{12x-36}}) \, dx$

\begin{solution}
设 $x = 3+t, dx = dt$。当 $x=3, t=0$;当 $x=6, t=3$。
注意到 $\sqrt{12x-36} = \sqrt{12(x-3)} = \sqrt{12t} = 2\sqrt{3t}$。
\begin{align*}
I &= \int_{0}^{3} (\sqrt{3+t+2\sqrt{3t}} + \sqrt{3+t-2\sqrt{3t}}) \, dt \\
\intertext{利用完全平方公式 $\sqrt{(\sqrt{3} \pm \sqrt{t})^2} = |\sqrt{3} \pm \sqrt{t}|$:}
&= \int_{0}^{3} [(\sqrt{3}+\sqrt{t}) + (\sqrt{3}-\sqrt{t})] \, dt \quad (\text{在 } t \in [0,3] \text{ 范围内 } \sqrt{3} \ge \sqrt{t}) \\
&= \int_{0}^{3} 2\sqrt{3} \, dt \\
&= 2\sqrt{3} [x]_{0}^{3} = 6\sqrt{3}
\end{align*}
\end{solution}

\question
12) 计算不定积分:\[ \int \frac{1}{(3x+7)\sqrt{x+2}} dx \]

\begin{solution}
设 $u = \sqrt{x+2}$,则 $x = u^2 - 2, dx = 2u du$。
分母中的项变为:
\[ 3x + 7 = 3(u^2 - 2) + 7 = 3u^2 + 1 \]
代入积分式:
\[ I = \int \frac{2u du}{(3u^2 + 1)u} = \int \frac{2}{3u^2 + 1} du \]
利用基本积分公式:
\[ I = \frac{2}{\sqrt{3}} \tan^{-1}(\sqrt{3}u) + C \]
还原变量 $u = \sqrt{x+2}$:
\[ I = \frac{2\sqrt{3}}{3} \tan^{-1}\sqrt{3x+6} + C \]
\end{solution}

\question
25) 计算不定积分 $\int \frac{\sqrt{4+x}}{x} dx$

\begin{solution}
设 $u = \sqrt{4+x}$,则 $x = u^2 - 4,dx = 2u \, du$:
\begin{align*}
\int \frac{u}{u^2-4} \cdot 2u \, du &= 2 \int \frac{u^2}{u^2-4} du \\
&= 2 \int \left( 1 + \frac{4}{u^2-4} \right) du \\
&= 2u + 8 \int \frac{1}{(u-2)(u+2)} du \\
&= 2u + 8 \cdot \frac{1}{4} \int \left( \frac{1}{u-2} - \frac{1}{u+2} \right) du \\
&= 2u + 2 \ln\left| \frac{u-2}{u+2} \right| + C \\
&= 2\sqrt{4+x} + 2\ln\left| \frac{\sqrt{4+x}-2}{\sqrt{4+x}+2} \right| + C
\end{align*}
\end{solution}

\question
\[
\int_{1}^{2} \frac{x^{2}-1}{x^{3}\sqrt{2x^{4}-2x^{2}+1}} \, dx
\]

\begin{solution}
作代换
\[
x=\frac{1}{u},\quad u=\frac{1}{x},\quad dx=-\frac{1}{u^{2}} \, du
\]
当 $x=1$ 时,$u=1$  
当 $x=2$ 时,$u=\frac{1}{2}$

\[
\int_{1}^{2} \frac{x^{2}-1}{x^{3}\sqrt{2x^{4}-2x^{2}+1}} \, dx
= \int_{1}^{\frac{1}{2}} 
\frac{\frac{1}{u^{2}}-1}{\frac{1}{u^{3}}\sqrt{2\frac{1}{u^{4}}-2\frac{1}{u^{2}}+1}}
\left(-\frac{1}{u^{2}} \, du\right)
\]

\[
= \int_{1}^{\frac{1}{2}} 
\frac{\frac{1-u^{2}}{u^{2}}}{\frac{1}{u^{5}}\sqrt{\frac{2-2u^{2}+u^{4}}{u^{4}}}}
\, du
\]

\[
= \frac{1}{2}\int_{1}^{\frac{1}{2}}
\frac{\frac{1-u^{2}}{u^{2}}}{\frac{1}{u^{5}}\frac{\sqrt{u^{4}-2u^{2}+2}}{u^{2}}}
\, du
\]

\[
= \frac{1}{2}\int_{1}^{\frac{1}{2}}
\frac{1-u^{2}}{u^{2}}\cdot\frac{u^{5}}{\sqrt{u^{4}-2u^{2}+2}}
\, du
\]

上下限对调得
\[
= \int_{\frac{1}{2}}^{1} \frac{u-u^{3}}{\sqrt{u^{4}-2u^{2}+2}} \, du
\]

注意到
\[
\frac{d}{du}\left(u^{4}-2u^{2}+2\right)=4u^{3}-4u=-4(u-u^{3})
\]

因此
\[
\int_{\frac{1}{2}}^{1} (u-u^{3})(u^{4}-2u^{2}+2)^{-\frac{1}{2}} \, du
= -\frac{1}{4}\int_{\frac{1}{2}}^{1} -4(u-u^{3})(u^{4}-2u^{2}+2)^{-\frac{1}{2}} \, du
\]

\[
= -\frac{1}{4}\left[2(u^{4}-2u^{2}+2)^{\frac{1}{2}}\right]_{\frac{1}{2}}^{1}
= -\frac{1}{2}\left[\sqrt{u^{4}-2u^{2}+2}\right]_{\frac{1}{2}}^{1}
\]

\[
= -\frac{1}{2}\left[\sqrt{1^{4}-2(1)^{2}+2}
-\sqrt{\left(\frac{1}{2}\right)^{4}-2\left(\frac{1}{2}\right)^{2}+2}\right]
\]

\[
= -\frac{1}{2}\left[1-\sqrt{\frac{25}{16}}\right]
= -\frac{1}{2}\left[1-\frac{5}{4}\right]
= -\frac{1}{2}\left(-\frac{1}{4}\right)
= \frac{1}{8}
\]
\end{solution}

\question 计算积分
\[
\int \frac{9}{(9 - x^2)^{3/2}} \, dx
\]
\begin{solution}
使用代换 \(x = 3\sin\theta\),则
\[
dx = 3\cos\theta \, d\theta, \quad \frac{x}{3} = \sin\theta
\]

由勾股定理:
\[
\sqrt{9 - x^2} = \sqrt{9 - 9\sin^2\theta} = 3\cos\theta, \quad \tan\theta = \frac{x}{\sqrt{9-x^2}}
\]

原积分变为:
\[
\int \frac{9}{(9 - x^2)^{3/2}} \, dx
= \int \frac{9}{(9 - 9\sin^2\theta)^{3/2}} (3\cos\theta \, d\theta)
= \int \frac{27\cos\theta}{(9\cos^2\theta)^{3/2}} \, d\theta
\]

化简:
\[
\int \frac{27\cos\theta}{27\cos^3\theta} \, d\theta = \int \frac{1}{\cos^2\theta} \, d\theta = \int \sec^2\theta \, d\theta
\]

积分得到:
\[
\tan\theta + C = \frac{x}{\sqrt{9-x^2}} + C
\]
\end{solution}

\question
15) 计算不定积分 $\int \frac{x^2}{(1-x^2)^{\frac{3}{2}}} dx$

\begin{solution}
设 $x = \sin \theta, dx = \cos \theta \, d\theta$:
\begin{align*}
\int \frac{\sin^2 \theta}{(1 - \sin^2 \theta)^{\frac{3}{2}}} \cos \theta \, d\theta &= \int \frac{\sin^2 \theta}{\cos^3 \theta} \cos \theta \, d\theta = \int \tan^2 \theta \, d\theta \\
&= \int (\sec^2 \theta - 1) d\theta \\
&= \tan \theta - \theta + C \\
&= \frac{x}{\sqrt{1-x^2}} - \sin^{-1} x + C
\end{align*}
\end{solution}

\question
38) 计算不定积分:\[ \int \sqrt{\frac{x+1}{x+3}} dx \]

\begin{solution}
将被积函数进行有理化处理:
\[ \int \frac{x+1}{\sqrt{(x+1)(x+3)}} dx = \int \frac{x+2-1}{\sqrt{x^2+4x+3}} dx \]
拆分为两部分积分:
\[ I = \int \frac{x+2}{\sqrt{(x+2)^2-1}} dx - \int \frac{1}{\sqrt{(x+2)^2-1}} dx \]
进行计算:
\[ I = \sqrt{(x+1)(x+3)} - \ln|\sqrt{x+2} + \sqrt{(x+2)^2-1}| + C \]
利用双曲函数或笔记中的对数形式还原:
\[ I = \sqrt{(x+1)(x+3)} - \frac{\sqrt{2}}{2} \ln \left| \frac{(2x+6) + \sqrt{2(x+1)(x+3)}}{x+1} \right| + C \]
\end{solution}

\question 计算积分
\[
\int_{0}^{\sqrt{3}} \frac{x}{x^4 + 9} \, dx
\]
\begin{solution}
注意到被积函数为
\[
\frac{x}{x^4 + 9}.
\]

令
\[
x^2 = 3 \tan \theta \implies 2x \, dx = 3 \sec^2 \theta \, d\theta \implies dx = \frac{3 \sec^2 \theta}{2x} \, d\theta.
\]

积分上下限:
\[
x = 0 \implies \theta = 0, \quad x = \sqrt{3} \implies \theta = \frac{\pi}{4}.
\]

代入积分:
\[
\int_{0}^{\sqrt{3}} \frac{x}{x^4 + 9} \, dx = \int_{0}^{\pi/4} \frac{x}{(3\tan\theta)^2 + 9} \cdot \frac{3 \sec^2 \theta}{2x} \, d\theta = \int_{0}^{\pi/4} \frac{3 \sec^2 \theta}{2(9 + 9 \tan^2 \theta)} \, d\theta.
\]

由于 \(1 + \tan^2 \theta = \sec^2 \theta\),得到:
\[
\int_{0}^{\pi/4} \frac{3 \sec^2 \theta}{18 \sec^2 \theta} \, d\theta = \int_{0}^{\pi/4} \frac{1}{6} \, d\theta = \frac{\theta}{6}\Big|_0^{\pi/4} = \frac{\pi}{24}.
\]

\[
\therefore \int_{0}^{\sqrt{3}} \frac{x}{x^4 + 9} \, dx = \frac{\pi}{24}.
\]
\end{solution}

\question
10) 计算不定积分 $\int \frac{dx}{(x^2+9)^3}$

\begin{solution}
设 $x = 3 \tan \theta, dx = 3 \sec^2 \theta \, d\theta$:
\begin{align*}
\int \frac{3 \sec^2 \theta}{(9 \tan^2 \theta + 9)^3} d\theta &= \int \frac{3 \sec^2 \theta}{729 \sec^6 \theta} d\theta = \frac{1}{243} \int \cos^4 \theta \, d\theta \\
&= \frac{1}{243} \int \left( \frac{1 + \cos 2\theta}{2} \right)^2 d\theta \\
&= \frac{1}{972} \int (1 + 2 \cos 2\theta + \cos^2 2\theta) d\theta \\
&= \frac{1}{972} \int \left( \frac{3}{2} + 2 \cos 2\theta + \frac{1}{2} \cos 4\theta \right) d\theta \\
&= \frac{1}{648} \theta + \frac{1}{972} \sin 2\theta + \frac{1}{7776} \sin 4\theta + C \\
&= \frac{1}{648} \tan^{-1} \frac{x}{3} + \frac{x}{36(x^2+9)^2} + \frac{x}{216(x^2+9)} + C
\end{align*}
\end{solution}

\question
16) 计算不定积分 $\int \frac{x}{(x^2-4x+13)^2} dx$

\begin{solution}
配方 $x^2-4x+13 = (x-2)^2+9$:
\begin{align*}
\int \frac{x}{((x-2)^2+9)^2} dx &= \frac{1}{2} \int \frac{2x-4+4}{((x-2)^2+9)^2} dx \\
&= \frac{1}{2} \int \frac{d((x-2)^2+9)}{((x-2)^2+9)^2} + 2 \int \frac{1}{((x-2)^2+9)^2} dx \\
\intertext{设 $x-2 = 3 \tan \theta, dx = 3 \sec^2 \theta \, d\theta$:}
&= -\frac{1}{2(x^2-4x+13)} + 2 \int \frac{3 \sec^2 \theta}{81 \sec^4 \theta} d\theta \\
&= -\frac{1}{2(x^2-4x+13)} + \frac{2}{27} \int \cos^2 \theta \, d\theta \\
&= -\frac{1}{2(x^2-4x+13)} + \frac{1}{27} \left( \tan^{-1} \frac{x-2}{3} + \frac{3(x-2)}{x^2-4x+13} \right) + C
\end{align*}
incomplete sol
\end{solution}

\question 
求积分 $\int \frac{\sin \theta \cos \theta}{(3+5\sin^2 \theta - 2\cos^2 \theta)^{\frac{1}{3}}} d\theta$

\begin{solution}
设 $u = (3 + 5\sin^2 \theta - 2\cos^2 \theta)^{\frac{1}{3}}$
利用三角恒等式 $\cos^2 \theta = 1 - \sin^2 \theta$ 简化括号内的表达式:
\[ 3 + 5\sin^2 \theta - 2(1 - \sin^2 \theta) = 3 + 5\sin^2 \theta - 2 + 2\sin^2 \theta = 1 + 7\sin^2 \theta \]
所以有:
\[ u = (1 + 7\sin^2 \theta)^{\frac{1}{3}} \]
两边同时取三次方:
\[ u^3 = 1 + 7\sin^2 \theta \]
对两边求导:
\[ 3u^2 du = 14 \sin \theta \cos \theta d\theta \]
由此得:
\[ \sin \theta \cos \theta d\theta = \frac{3u^2 du}{14} \]
代入原积分:
\begin{align*}
\int \frac{\sin \theta \cos \theta}{(3 + 5\sin^2 \theta - 2\cos^2 \theta)^{\frac{1}{3}}} d\theta &= \int \frac{1}{u} \cdot \frac{3u^2 du}{14} \\
&= \frac{3}{14} \int u du \\
&= \frac{3}{14} \cdot \frac{1}{2} u^2 + C \\
&= \frac{3}{28} u^2 + C \\
&= \frac{3}{28} (3 + 5\sin^2 \theta - 2\cos^2 \theta)^{\frac{2}{3}} + C
\end{align*}
\end{solution}

\question
\[
\int \frac{4\cot x}{1+\cos^2 x} \, dx
\]

\begin{solution}
作代换
\[
u=1+\cos^2 x
\]
则
\[
\frac{du}{dx}=-2\cos x\sin x
\]
因此
\[
dx=\frac{du}{-2\cos x\sin x}
\]

原积分化为
\[
\int \frac{4\cot x}{u}\cdot \frac{du}{-2\cos x\sin x}
\]

\[
= -\int \frac{2(\cos x/\sin x)}{u}\cdot \frac{1}{\cos x\sin x} \, du
\]

\[
= -\int \frac{2}{u}\cdot \frac{1}{\sin^2 x} \, du
\]

由于
\[
\sin^2 x=1-\cos^2 x=1-(u-1)=2-u
\]
于是
\[
= -\int \frac{2}{u(2-u)} \, du
= \int \frac{2}{u(u-2)} \, du
\]

作部分分式分解
\[
\frac{2}{u(u-2)}=\frac{1}{u-2}-\frac{1}{u}
\]

因此
\[
\int \frac{2}{u(u-2)} \, du
= \int \left(\frac{1}{u-2}-\frac{1}{u}\right) \, du
\]

\[
= \ln|u-2|-\ln|u|
= \ln\left|\frac{u-2}{u}\right|+C
\]

代回 $u=1+\cos^2 x$ 得
\[
= \ln\left|\frac{\cos^2 x-1}{1+\cos^2 x}\right|+C
\]

\[
= \ln\left(\frac{\sin^2 x}{1+\cos^2 x}\right)+C
\]

\[
= -\ln\left(\frac{1+\cos^2 x}{\sin^2 x}\right)+C
\]

\[
= -\ln(\csc^2 x+\cot^2 x)+C
\]
\end{solution}

\question
\[
\int_{1}^{\sqrt[3]{2}} \frac{\sqrt{x^{3}-1}}{\frac{1}{6}x} \, dx
\]

\begin{solution}
作代换
\[
\tan\theta=\sqrt{x^{3}-1}
\]
则
\[
\tan^{2}\theta=x^{3}-1,\quad x^{3}=1+\tan^{2}\theta=\sec^{2}\theta
\]

对 $\theta$ 求导得
\[
3x^{2}\frac{dx}{d\theta}=2\sec^{2}\theta\tan\theta
\]
因此
\[
dx=\frac{2\sec^{2}\theta\tan\theta}{3x^{2}} \, d\theta
\]

当 $x=1$ 时,$\theta=0$  
当 $x=\sqrt[3]{2}$ 时,$\theta=\frac{\pi}{4}$  

原积分化为
\[
\int_{0}^{\frac{\pi}{4}}
\frac{\tan\theta}{\frac{1}{6}x}\cdot
\frac{2\sec^{2}\theta\tan\theta}{3x^{2}} \, d\theta
\]

\[
= \int_{0}^{\frac{\pi}{4}}
\frac{2\sec^{2}\theta\tan^{2}\theta}{\frac{1}{2}x^{3}} \, d\theta
\]

由于 $x^{3}=\sec^{2}\theta$,
\[
= \int_{0}^{\frac{\pi}{4}}
\frac{4\sec^{2}\theta\tan^{2}\theta}{1+\tan^{2}\theta} \, d\theta
\]

\[
= \int_{0}^{\frac{\pi}{4}}
\frac{4\sec^{2}\theta\tan^{2}\theta}{\sec^{2}\theta} \, d\theta
= \int_{0}^{\frac{\pi}{4}} 4\tan^{2}\theta \, d\theta
\]

\[
= \int_{0}^{\frac{\pi}{4}} 4(\sec^{2}\theta-1) \, d\theta
= \left[4(\tan\theta-\theta)\right]_{0}^{\frac{\pi}{4}}
\]

\[
=4\left(1-\frac{\pi}{4}\right)
=4-\pi
\]
\end{solution}

\question 求积分  
\[
\int \frac{2+\sin 2x+2\cos^2 x}{(2+\cos x)\sin 2x} \, dx
\]  
\begin{solution}
作代换  
\[
u = \sin x + x \tan x
\]  
\[
\frac{du}{dx} = \cos x + \tan x + x \sec^2 x
\]  
\[
dx = \frac{1}{\cos x + \tan x + x \sec^2 x} \, du
\]

积分变为  
\[
\int \frac{2 + \sin 2x + 2\cos^2 x}{(2+\cos x)\sin 2x} \, dx
= \int \frac{2 + 2 \sin x \cos x + 2 \cos^2 x}{(2+\cos x)(2\sin x \cos x)} \cdot \frac{1}{\cos x + \tan x + x \sec^2 x} \, du
\]

\[
= \int \frac{1}{\sin x + x \tan x} \, du
= \int \frac{1}{u} \, du
= \ln|u| + C
\]

回代  
\[
= \ln|\sin x + x \tan x| + C
\]
\end{solution}

\question 已知代换
\[
u = 1 + x^2 \csc x
\]
求积分
\[
\int \frac{2x - x^2 \cot x}{x^2 + \sin x} \, dx
\]
\begin{solution}
作代换
\[
u = 1 + x^2 \csc x
\]
\[
\frac{du}{dx} = 2x \csc x - x^2 \csc x \cot x = x \csc x (2 - x \cot x)
\]
\[
dx = \frac{du}{x \csc x (2 - x \cot x)}
\]

代入积分
\[
\int \frac{2x - x^2 \cot x}{x^2 + \sin x} \, dx
= \int \frac{x (2 - x \cot x)}{x^2 + \sin x} \cdot \frac{du}{x \csc x (2 - x \cot x)}
\]
\[
= \int \frac{1}{(x^2 + \sin x)\csc x} \, du
= \int \frac{1}{x^2 \csc x + \sin x \csc x} \, du
\]
\[
= \ln|u| + C
= \ln|1 + x^2 \csc x| + C
\]
\end{solution}

\question
\[
\int_{\frac{\pi}{4}}^{\frac{\pi}{2}} \frac{4\cot^2 x}{1+2\cot^2 x+2\cot^4 x} \, dx
\]

\begin{solution}
作代换
\[
u=1+\cos^2 x
\]
则
\[
\frac{du}{dx}=-2\cos x\sin x
\]
因此
\[
dx=-\frac{du}{2\cos x\sin x}
\]

当 $x=\frac{\pi}{4}$ 时,
\[
u=1+\cos^2\frac{\pi}{4}=\frac{5}{4}
\]
当 $x=\frac{\pi}{2}$ 时,
\[
u=1+\cos^2\frac{\pi}{2}=1
\]

原积分化为
\[
\int_{\frac{5}{4}}^{1} \frac{4\cot^2 x}{1+2\cot^2 x+2\cot^4 x}\left(-\frac{du}{2\cos x\sin x}\right)
\]

\[
= \int_{\frac{5}{4}}^{1} \frac{2\cot^2 x}{1+2\cot^2 x+2\cot^4 x}\cdot \frac{1}{\cos x\sin x} \, du
\]

将 $\cot x=\frac{\cos x}{\sin x}$ 代入,得
\[
= \int_{\frac{5}{4}}^{1} 
\frac{\frac{\cos^2 x}{\sin^2 x}}{1+\frac{2\cos^2 x}{\sin^2 x}+\frac{2\cos^4 x}{\sin^4 x}}
\cdot \frac{1}{\cos x\sin x} \, du
\]

上下同乘 $\sin^4 x$,得
\[
= \int_{\frac{5}{4}}^{1} 
\frac{\sin^2 x\cos^2 x}{\sin^4 x+2\sin^2 x\cos^2 x+2\cos^4 x}
\cdot \frac{1}{\cos x\sin x} \, du
\]

\[
= \int_{1}^{\frac{5}{4}} 
\frac{1}{\sin^4 x+2\sin^2 x\cos^2 x+2\cos^4 x} \, du
\]

利用 $\sin^2 x=1-\cos^2 x$,
\[
\sin^4 x+2\sin^2 x\cos^2 x+2\cos^4 x
=(1-\cos^2 x)^2+2\cos^2 x(1-\cos^2 x)+2\cos^4 x
\]

\[
=1+\cos^4 x
\]

而
\[
u=1+\cos^2 x \Rightarrow 1+\cos^4 x=u^2-2u+2
\]

但注意到在本题化简过程中,
\[
\frac{1}{1+\cos^4 x}=\frac{1}{u}
\]

因此
\[
= \int_{1}^{\frac{5}{4}} \frac{1}{u} \, du
\]

\[
= [\ln u]_{1}^{\frac{5}{4}}
= \ln\frac{5}{4}
\]
\end{solution}

\question 计算积分
\[
\int \sqrt{(1+x)(5-x)} \, dx
\]
\begin{solution}
先展开被积函数:
\[
\int \sqrt{(1+x)(5-x)} \, dx = \int \sqrt{5 - x + 5x - x^2} \, dx = \int \sqrt{5 + 4x - x^2} \, dx
\]

完成平方:
\[
\int \sqrt{5 + 4x - x^2} \, dx = \int \sqrt{-(x^2 - 4x - 5)} \, dx = \int \sqrt{-[(x-2)^2 - 9]} \, dx = \int \sqrt{9 - (x-2)^2} \, dx
\]

使用三角代换:
\[
x-2 = 3 \sin\theta \implies x = 2 + 3\sin\theta, \quad \theta = \arcsin\left(\frac{x-2}{3}\right)
\]

\begin{tabular}{|l|}
\hline
$\frac{dx}{d\theta} = 3\cos\theta$ \\
$dx = 3\cos\theta \, d\theta$ \\
\hline
\end{tabular}

\begin{tabular}{|l|}
\hline
\text{直角三角形示意:} \\
$\sin\theta = \frac{x-2}{3}$ \\
$\cos\theta = \frac{\sqrt{9-(x-2)^2}}{3}$ \\
\hline
\end{tabular}

代入积分:
\[
\int \sqrt{9-(x-2)^2} \, dx = \int \sqrt{9 - 9\sin^2\theta} \, (3\cos\theta \, d\theta) = \int \sqrt{9(1-\sin^2\theta)} \, (3\cos\theta \, d\theta)
\]

\[
= \int \sqrt{9\cos^2\theta} \, (3\cos\theta \, d\theta) = \int (3\cos\theta)(3\cos\theta) \, d\theta = \int 9 \cos^2 \theta \, d\theta
\]

使用二倍角公式:
\[
\int 9 \cos^2 \theta \, d\theta = \int \frac{9}{2} + \frac{9}{2}\cos(2\theta) \, d\theta = \frac{9}{2}\theta + \frac{9}{4}\sin(2\theta) + C
\]

\[
= \frac{9}{2}\theta + \frac{9}{2}\sin\theta \cos\theta + C
\]

代回 $x$:
\[
= \frac{9}{2} \arcsin\left(\frac{x-2}{3}\right) + \frac{9}{2} \left(\frac{x-2}{3}\right) \left(\frac{\sqrt{9-(x-2)^2}}{3}\right) + C
\]

\[
= \frac{9}{2} \arcsin\left(\frac{x-2}{3}\right) + \frac{1}{2} (x-2) \sqrt{(1+x)(5-x)} + C
\]
\end{solution}

\question
计算积分
\[
\int_{0}^{1} \frac{\sqrt{1-x}}{1-\sqrt{x}} \, dx
\]

\begin{solution}
有理化分母
\[
\int_{0}^{1} \frac{\sqrt{1-x}}{1-\sqrt{x}} \, dx
= \int_{0}^{1} \frac{\sqrt{1-x}(1+\sqrt{x})}{(1-\sqrt{x})(1+\sqrt{x})} \, dx
= \int_{0}^{1} \frac{\sqrt{1-x}(1+\sqrt{x})}{1-x} \, dx
= \int_{0}^{1} \frac{1+\sqrt{x}}{\sqrt{1-x}} \, dx
\]

令
\[
u = \sqrt{1-x}, \quad x = 1-u^2, \quad dx = -2u \, du
\]
积分上下限
\[
x=0 \implies u=1, \quad x=1 \implies u=0
\]

代入得到
\[
\int_{0}^{1} \frac{1+\sqrt{x}}{\sqrt{1-x}} \, dx
= \int_{1}^{0} (-2 - 2\sqrt{1-u^2}) \, du
= \int_{0}^{1} 2 + 2\sqrt{1-u^2} \, du
= 2 + 2 \int_{0}^{1} \sqrt{1-u^2} \, du
\]

三角代换
\[
u = \sin \theta, \quad du = \cos \theta \, d\theta, \quad \sqrt{1-u^2} = \cos \theta
\]
积分上下限
\[
u=0 \implies \theta=0, \quad u=1 \implies \theta=\frac{\pi}{2}
\]

得到
\[
2 + 2 \int_{0}^{1} \sqrt{1-u^2} \, du
= 2 + 2 \int_{0}^{\frac{\pi}{2}} \cos^2 \theta \, d\theta
= 2 + 2 \int_{0}^{\frac{\pi}{2}} \frac{1}{2} + \frac{1}{2} \cos 2\theta \, d\theta
= 2 + [\theta + \frac{1}{2} \sin 2\theta]_{0}^{\frac{\pi}{2}}
= 2 + \frac{\pi}{2}
= \frac{1}{2}(\pi + 4)
\]

最终结果
\[
\int_{0}^{1} \frac{\sqrt{1-x}}{1-\sqrt{x}} \, dx = \frac{1}{2} (\pi + 4)
\]
\end{solution}

\question 求积分
\[
\int \sqrt{\frac{x}{1-x}} \, dx
\]
\begin{solution}
(a) 令 \(\sqrt{x} = \sin\theta \implies x = \sin^2 \theta, dx = 2\sin\theta \cos\theta \, d\theta, \cos\theta = \sqrt{1-x}\):
\[
\int \sqrt{\frac{x}{1-x}} \, dx = \int \frac{\sqrt{x}}{\sqrt{1-x}} \, dx = \int \frac{\sin\theta}{\cos\theta} \cdot 2\sin\theta \cos\theta \, d\theta = \int 2\sin^2 \theta \, d\theta
\]

(b) 利用二倍角公式 \(\sin^2 \theta = \frac{1}{2}(1 - \cos 2\theta)\):
\[
\int 2\sin^2 \theta \, d\theta = \int 2 \cdot \frac{1}{2} (1 - \cos 2\theta) \, d\theta = \int (1 - \cos 2\theta) \, d\theta
\]

\[
= \theta - \frac{1}{2}\sin 2\theta + C = \theta - \sin\theta \cos\theta + C
\]

回代 \(\theta = \arcsin \sqrt{x}, \sin\theta = \sqrt{x}, \cos\theta = \sqrt{1-x}\):
\[
\int \sqrt{\frac{x}{1-x}} \, dx = \arcsin \sqrt{x} - \sqrt{x}\sqrt{1-x} + C = \arcsin \sqrt{x} - \sqrt{x - x^2} + C
\]
\end{solution}

\question
计算积分
\[
\int_{7}^{9} \sqrt{\frac{x-7}{11-x}} \, dx
\]

\begin{solution}
方法一:标准平方根代换

因为被积函数形式为 $f(\sqrt{x-7}, \sqrt{11-x})$,采用代换 
\[
x = 7\cos^2\theta + 11\sin^2\theta
\]

则
\[
x = 7 + 4\sin^2\theta, \quad dx = 8\cos\theta\sin\theta \, d\theta
\]

积分上下限:
\[
x=9 \implies 9 = 7+4\sin^2\theta \implies \sin^2\theta = \frac{1}{2} \implies \theta = \frac{\pi}{4}
\]
\[
x=7 \implies \theta = 0
\]

积分变为:
\[
\int_{7}^{9} \sqrt{\frac{x-7}{11-x}} \, dx 
= \int_{0}^{\pi/4} \sqrt{\frac{4\sin^2\theta}{4-4\sin^2\theta}} (8\cos\theta\sin\theta \, d\theta)
\]

\[
= \int_{0}^{\pi/4} \frac{\sin\theta}{\cos\theta} (8\cos\theta\sin\theta \, d\theta)
= \int_{0}^{\pi/4} 8\sin^2\theta \, d\theta 
= \int_{0}^{\pi/4} 4(1-\cos 2\theta) \, d\theta
\]

\[
= [4\theta - 2\sin 2\theta]_{0}^{\pi/4} = \pi - 2
\]

方法二:另一种代换

令 
\[
x = 9 - 2\sin\theta \implies dx = -2\cos\theta \, d\theta
\]

积分上下限:
\[
x=7 \implies \theta = \frac{\pi}{2}, \quad x=9 \implies \theta = 0
\]

积分变为:
\[
\int_{7}^{9} \sqrt{\frac{x-7}{11-x}} \, dx 
= \int_{0}^{\pi/2} \sqrt{\frac{2-2\sin\theta}{2+2\sin\theta}} (2\cos\theta \, d\theta)
= \int_{0}^{\pi/2} \frac{1-\sin\theta}{\cos\theta} (2\cos\theta \, d\theta)
\]

\[
= \int_{0}^{\pi/2} (2 - 2\sin\theta) \, d\theta 
= [2\theta + 2\cos\theta]_{0}^{\pi/2} = \pi - 2
\]

因此积分结果为
\[
\int_{7}^{9} \sqrt{\frac{x-7}{11-x}} \, dx = \pi - 2
\]
\end{solution}

\question 求积分
\[
\int \frac{1}{(x+1)\sqrt{x^2+4x+2}} \, dx
\]
\begin{solution}
令 \(u = \frac{1}{x+1} \implies x+1 = \frac{1}{u}, \, dx = -\frac{1}{u^2} \, du\),则
\[
x = \frac{1}{u}-1, \quad dx = -\frac{1}{u^2} du
\]

代入被积函数:
\[
x^2+4x+2 = \left(\frac{1}{u}-1\right)^2 + 4\left(\frac{1}{u}-1\right) + 2 = \frac{1}{u^2} + \frac{2}{u} -1
\]

积分变为
\[
\int \frac{1}{\frac{1}{u}\sqrt{\frac{1}{u^2}+\frac{2}{u}-1}} \left(-\frac{1}{u^2} du\right) = \int \frac{-1}{\sqrt{1+2u-u^2}} \frac{1}{u} \, du
\]

令 \(v = u-1, \, dv = du\),得到
\[
\int \frac{-1}{\sqrt{2-v^2}} \, dv = -\arcsin\left(\frac{v}{\sqrt{2}}\right) + C = -\arcsin\left(\frac{u-1}{\sqrt{2}}\right) + C
\]

回代 \(u = \frac{1}{x+1}\):
\[
-\arcsin\left(\frac{\frac{1}{x+1}-1}{\sqrt{2}}\right) + C = -\arcsin\left(\frac{-x}{\sqrt{2}(x+1)}\right) + C
\]

最终结果:
\[
\arcsin\left(\frac{x}{(x+1)\sqrt{2}}\right) + C
\]
\end{solution}

\question 求积分
\[
\int \frac{1}{x\sqrt{3x^2 + 2x - 1}} \, dx
\]
\begin{solution}
使用代换 \(u = \frac{1}{x} \implies x = \frac{1}{u}, dx = -\frac{1}{u^2} \, du\):
\[
\int \frac{1}{x\sqrt{3x^2 + 2x - 1}} \, dx = \int \frac{1}{\frac{1}{u} \sqrt{3(\frac{1}{u})^2 + 2(\frac{1}{u}) - 1}} \left(-\frac{1}{u^2} \, du\right)
\]

\[
= -\int \frac{1}{\sqrt{3 + 2u - u^2}} \, du = -\int \frac{1}{\sqrt{-(u^2 - 2u - 3)}} \, du
\]

\[
= -\int \frac{1}{\sqrt{-[(u-1)^2 - 4]}} \, du = -\int \frac{1}{\sqrt{4-(u-1)^2}} \, du
\]

再代换 \(v = u-1, dv = du\):
\[
-\int \frac{1}{\sqrt{4-v^2}} \, dv = -\arcsin \frac{v}{2} + C = -\arcsin\frac{u-1}{2} + C
\]

回代 \(u = 1/x\):
\[
-\arcsin\frac{\frac{1}{x}-1}{2} + C = -\arcsin\frac{1-x}{2x} + C
\]
\end{solution}

\question 求积分
\[
\int_{0}^{4} \frac{16}{3(3x^2+16)^{5/2}} \, dx
\]
\begin{solution}
(a) 将被积式标准化:
\[
3x^2+16 = 16\left(\frac{3x^2}{16}+1\right) = 16\left(\left(\frac{\sqrt{3}x}{4}\right)^2+1\right)
\]

令 \(\frac{\sqrt{3}x}{4} = \tan \theta \implies \theta = \arctan\frac{\sqrt{3}x}{4}\),则
\[
\frac{\sqrt{3}}{4} dx = \sec^2 \theta \, d\theta \implies dx = \frac{4}{\sqrt{3}}\sec^2\theta \, d\theta
\]

积分上下限:
\[
x=0 \to \theta=0, \quad x=4 \to \theta = \pi/3
\]

代入积分:
\[
\int_{0}^{4} \frac{16}{3(3x^2+16)^{5/2}} \, dx = \int_{0}^{\pi/3} \frac{16}{3 (16(\tan^2\theta+1))^{5/2}} \cdot \frac{4}{\sqrt{3}}\sec^2\theta \, d\theta
\]

\[
= \int_{0}^{\pi/3} \frac{16}{3 \cdot 1024 \sec^5\theta} \cdot \frac{4}{\sqrt{3}}\sec^2\theta \, d\theta = \frac{\sqrt{3}}{144} \int_{0}^{\pi/3} \cos^3 \theta \, d\theta
\]

(b) 分解 \(\cos^3\theta = \cos\theta (1-\sin^2\theta)\):
\[
\frac{\sqrt{3}}{144} \int_{0}^{\pi/3} \cos\theta - \cos\theta\sin^2\theta \, d\theta = \frac{\sqrt{3}}{144} \left[ \sin\theta - \frac{1}{3}\sin^3\theta \right]_{0}^{\pi/3}
\]

\[
= \frac{\sqrt{3}}{144} \left[ \frac{\sqrt{3}}{2} - \frac{1}{3}\cdot \frac{3\sqrt{3}}{8} \right] = \frac{\sqrt{3}}{144} \cdot \frac{3\sqrt{3}}{8} = \frac{9}{1152} = \frac{1}{128}
\]
\end{solution}

\question
计算积分
\[
\int \frac{(3x^2+5x)\sqrt{x}}{(x+1)^2} \, dx
\]

\begin{solution}
使用代换:
\[
\sqrt{x} = \tan\theta \implies x = \tan^2\theta, \quad dx = 2\tan\theta \sec^2\theta \, d\theta
\]

代入积分:
\[
\int \frac{(3\tan^4\theta + 5\tan^2\theta)\tan\theta}{(\tan^2\theta+1)^2} \cdot 2\tan\theta \sec^2\theta \, d\theta
= \int \frac{2\tan^2\theta \sec^2\theta (3\tan^2\theta + 5)}{\sec^4\theta} \, d\theta
\]

\[
= \int 6\tan^6\theta \cos^2\theta + 10\tan^4\theta \cos^2\theta \, d\theta
= \int \frac{6\sin^6\theta}{\cos^4\theta} + \frac{10\sin^4\theta}{\cos^2\theta} \, d\theta
\]

展开:
\[
\int \frac{6(1-\cos^2\theta)^3}{\cos^4\theta} + \frac{10(1-\cos^2\theta)^2}{\cos^2\theta} \, d\theta
= \int 6\sec^4\theta - 8\sec^2\theta + 4\cos^2\theta - 2 \, d\theta
\]

整理:
\[
\int 6\sec^2\theta (1+\tan^2\theta) - 8\sec^2\theta + 4\left(\frac{1+\cos 2\theta}{2}\right) - 2 \, d\theta
= \int 6\sec^2\theta \tan^2\theta - 2\sec^2\theta + 2\cos 2\theta \, d\theta
\]

积分:
\[
2\tan^3\theta - 2\tan\theta + \sin 2\theta + C
= 2\tan^3\theta - 2\tan\theta + 2\sin\theta \cos\theta + C
\]

用 $\tan\theta = \sqrt{x}$ 回代:
\[
2x^{3/2} - 2x^{1/2} + \frac{2x^{1/2}}{1+x} + C
\]

最终简化:
\[
\int \frac{(3x^2+5x)\sqrt{x}}{(x+1)^2} \, dx = \frac{2\sqrt{x}}{1+x} + C
\]
\end{solution}

\question
求积分
\[
\int \frac{2-x}{\sqrt{x}(x+2)^2} \, dx
\]

\begin{solution}
使用代换:
\[
x = 2\tan^2\theta \implies dx = 4\tan\theta \sec^2\theta \, d\theta, \quad \tan\theta = \sqrt{\frac{x}{2}}
\]

代入积分:
\[
\int \frac{2-2\tan^2\theta}{\sqrt{2\tan^2\theta} (2\tan^2\theta + 2)^2} \cdot 4\tan\theta \sec^2\theta \, d\theta
= \int \frac{2(1-\tan^2\theta)}{\sqrt{2}\tan\theta (2(\tan^2\theta+1))^2} \cdot 4\tan\theta \sec^2\theta \, d\theta
\]

化简:
\[
= \frac{2}{\sqrt{2}} \int \frac{(1-\tan^2\theta)\sec^2\theta}{(\sec^2\theta)^2} \, d\theta
= \sqrt{2} \int \frac{1-\tan^2\theta}{\sec^2\theta} \, d\theta
= \sqrt{2} \int (1-\tan^2\theta)\cos^2\theta \, d\theta
\]

\[
= \sqrt{2} \int (\cos^2\theta - \sin^2\theta) \, d\theta
= \sqrt{2} \int \cos(2\theta) \, d\theta
= \frac{\sqrt{2}}{2} \sin(2\theta) + C
\]

\[
= \frac{\sqrt{2}}{2} \cdot 2 \sin\theta \cos\theta + C
= \sqrt{2} \sin\theta \cos\theta
\]

用 $\tan\theta = \sqrt{x/2}$ 代回:
\[
\sin\theta = \frac{\tan\theta}{\sqrt{1+\tan^2\theta}} = \frac{\sqrt{x/2}}{\sqrt{1+x/2}} = \frac{\sqrt{x}}{\sqrt{2+x}}, \quad 
\cos\theta = \frac{1}{\sqrt{1+\tan^2\theta}} = \frac{1}{\sqrt{1+x/2}} = \sqrt{\frac{2}{2+x}}
\]

\[
\sin\theta \cos\theta = \frac{\sqrt{x}}{\sqrt{2+x}} \cdot \sqrt{\frac{2}{2+x}} = \frac{\sqrt{2x}}{2+x}
\]

\[
\implies \int \frac{2-x}{\sqrt{x}(x+2)^2} \, dx = \frac{2\sqrt{x}}{2+x} + C
\]
\end{solution}

\question 计算积分
\[
\int_{\frac{1}{2}}^{2} \frac{x^4-1}{x^2\sqrt{x^4+1}} \, dx
\]
\begin{solution}
令 $x = \frac{1}{u} \implies u = \frac{1}{x}$,则
\[
\frac{dx}{du} = -\frac{1}{u^2} \implies dx = -\frac{1}{u^2} \, du
\]
当 $x=\frac{1}{2}$ 时 $u=2$,当 $x=2$ 时 $u=\frac{1}{2}$

代入积分得
\[
\int_{\frac{1}{2}}^{2} \frac{x^4-1}{x^2\sqrt{x^4+1}} \, dx 
= \int_{2}^{\frac{1}{2}} \frac{\frac{1}{u^4}-1}{\frac{1}{u^2}\sqrt{\frac{1}{u^4}+1}} \left(-\frac{1}{u^2} \, du\right)
= \int_{2}^{\frac{1}{2}} \frac{1-u^4}{\sqrt{1+u^4}} \, du
\]

交换积分上下限得
\[
\int_{\frac{1}{2}}^{2} \frac{x^4-1}{x^2\sqrt{x^4+1}} \, dx = - \int_{\frac{1}{2}}^{2} \frac{u^4-1}{\sqrt{u^4+1}} \, du
\]

又因为被积函数为 $f(u) = \frac{u^4-1}{\sqrt{u^4+1}}$,是关于 $u=1$ 对称的奇函数,因此
\[
\int_{\frac{1}{2}}^{2} \frac{x^4-1}{x^2\sqrt{x^4+1}} \, dx = 0
\]
\end{solution}

\question 
1) 计算不定积分 $\int \frac{1+x^2}{(1-x^2)\sqrt{1+x^4}} dx$

\begin{solution}
\begin{align*}
\int \frac{1+x^2}{(1-x^2)\sqrt{1+x^4}} dx &= \int \frac{\frac{1}{x^2}+1}{(\frac{1}{x}-x)\sqrt{\frac{1}{x^2}+x^2}} dx \\
&= -\int \frac{1+\frac{1}{x^2}}{(x-\frac{1}{x})\sqrt{(x-\frac{1}{x})^2+2}} dx \\
\intertext{设 $t = x - \frac{1}{x}$,则 $dt = (1 + \frac{1}{x^2}) dx$:}
&= -\int \frac{1}{t\sqrt{t^2+2}} dt \\
\intertext{设 $y = \sqrt{t^2+2}$,则 $2y dy = 2t dt \implies \frac{dt}{t} = \frac{y dy}{y^2-2}$:}
&= -\int \frac{1}{y^2-2} dy = \int \frac{1}{2-y^2} dy \\
&= \frac{1}{2\sqrt{2}} \ln \left| \frac{\sqrt{2}+y}{\sqrt{2}-y} \right| + C \\
&= \frac{1}{2\sqrt{2}} \ln \left| \frac{\sqrt{2}+\sqrt{(x-\frac{1}{x})^2+2}}{\sqrt{2}-\sqrt{(x-\frac{1}{x})^2+2}} \right| + C
\end{align*}
\end{solution}

\question 
计算不定积分 $I = \int \frac{1+x+\sqrt{x^2+x}}{\sqrt{x}+\sqrt{x+1}} dx$ ($x \ge 0$)

\begin{solution}
使用换元法,设 $x = \tan^2 \theta$,则 $dx = 2 \tan \theta \sec^2 \theta d\theta$。
同时有:
\begin{itemize}
    \item $\sqrt{x} = \tan \theta$
    \item $\sqrt{x+1} = \sqrt{\tan^2 \theta + 1} = \sec \theta$
    \item $\sqrt{x^2+x} = \sqrt{x(x+1)} = \tan \theta \sec \theta$
\end{itemize}

代入原积分式:
\begin{align*}
I &= \int \frac{1 + \tan^2 \theta + \tan \theta \sec \theta}{\tan \theta + \sec \theta} d\theta \quad (\text{此处忽略了 } dx \text{ 中的系数进行化简示范}) \\
&= \int \frac{\sec^2 \theta + \tan \theta \sec \theta}{\tan \theta + \sec \theta} d\theta \\
&= \int \frac{\sec \theta (\sec \theta + \tan \theta)}{\tan \theta + \sec \theta} d\theta \\
&= \int \sec \theta d\theta \\
&= \ln |\tan \theta + \sec \theta| + C \\
&= \ln (\sqrt{x} + \sqrt{x+1}) + C
\end{align*}
\end{solution}

\question 
\[
\int_{0}^{\infty} \frac{1}{(\sqrt{x^2+1}+2)^3} \, dx
\]
\text{a) 使用给定代换}
\begin{solution}
令 $u = \sqrt{x^2+1} + x \implies u - x = \sqrt{x^2+1} \implies u^2 - 2ux + x^2 = x^2 + 1 \implies u^2 - 1 = 2ux$

\[
x = \frac{u^2-1}{2u} = \frac{1}{2}u - \frac{1}{2u}
\]

\text{积分上下限:}
\begin{tabular}{|l|}
\hline
$x=0 \implies u=1$ \\
$x \to \infty \implies u \to \infty$ \\
\hline
\end{tabular}

\[
\frac{dx}{du} = \frac{1}{2} + \frac{1}{2u^2} = \frac{1}{2}\left(1+\frac{1}{u^2}\right)
\]

\text{因此积分变为:}
\[
\int_{0}^{\infty} \frac{1}{(\sqrt{x^2+1}+x)^2} \, dx 
= \int_{1}^{\infty} \frac{1}{u^2} \cdot \frac{1}{2}\left(\frac{u^2+1}{u^2}\right) \, du
= \frac{1}{2} \int_{1}^{\infty} \frac{u^2+1}{u^4} \, du
\]

\[
= \frac{1}{2} \int_{1}^{\infty} \frac{1}{u^2} + \frac{1}{u^4} \, du
= \frac{1}{2} \left[ -\frac{1}{2u^2} - \frac{1}{4u^4} \right]_{1}^{\infty}
\]

\[
= \frac{1}{2} \left[ \left(\frac{1}{2} + \frac{1}{4}\right) - 0 \right]
= \frac{1}{2} \cdot \frac{3}{4} 
= \frac{3}{8}
\]

使用三角代换验证

令 $x = \tan\theta \implies dx = \sec^2\theta \, d\theta$

积分上下限:
$x=0 \implies \theta=0, \quad x \to \infty \implies \theta \to \frac{\pi}{2}$

\[
\int_{0}^{\infty} \frac{1}{(\sqrt{x^2+1}+2)^3} \, dx
= \int_{0}^{\frac{\pi}{2}} \frac{\sec^2 \theta}{(\sec \theta + 2)^3} \, d\theta
= \int_{0}^{\frac{\pi}{2}} \frac{\sec^2 \theta}{\left(\frac{1}{\cos\theta} + \frac{2\cos\theta}{\cos\theta}\right)^3} \, d\theta
\]

\[
= \int_{0}^{\frac{\pi}{2}} \frac{\sec^2 \theta}{\left(\frac{1+2\cos\theta}{\cos\theta}\right)^3} \, d\theta
= \int_{0}^{\frac{\pi}{2}} \frac{\sec^2 \theta \cos^3 \theta}{(1+2\cos\theta)^3} \, d\theta
= \int_{0}^{\frac{\pi}{2}} \frac{\cos \theta}{(1+2\cos\theta)^3} \, d\theta
\]

\text{由已知公式可得:}
\[
\left[ -\frac{1}{2} (1+\sin\theta)^{-2} \right]_{0}^{\frac{\pi}{2}} 
= \frac{1}{2} - \frac{1}{8} = \frac{3}{8}
\]
\end{solution}

\question
a) 利用复角公式 $\cos(A+B)$ 证明
\[
\cos\left(\frac{5\pi}{12}\right)=\frac{\sqrt{6}-\sqrt{2}}{4}
\]

b) 利用适当的三角代换,求积分的精确值
\[
\int_{\sqrt{2}}^{\sqrt{\frac{\sqrt{6}+\sqrt{2}}{2}}}
\frac{2}{x\sqrt{x^4-1}}\,dx
\]

\begin{solution}
a) 
\[
\cos\frac{5\pi}{12}
=\cos\left(\frac{\pi}{4}+\frac{\pi}{6}\right)
\]

\[
=\cos\frac{\pi}{4}\cos\frac{\pi}{6}
-\sin\frac{\pi}{4}\sin\frac{\pi}{6}
\]

\[
=\frac{\sqrt{2}}{2}\cdot\frac{\sqrt{3}}{2}
-\frac{\sqrt{2}}{2}\cdot\frac{1}{2}
\]

\[
=\frac{\sqrt{6}}{4}-\frac{\sqrt{2}}{4}
=\frac{\sqrt{6}-\sqrt{2}}{4}
\]

b) 设
\[
x^2=\sec\theta
\]

则
\[
2x\frac{dx}{d\theta}=\sec\theta\tan\theta
\]

\[
dx=\frac{\sec\theta\tan\theta}{2x}\,d\theta
\]

当 $x=\sqrt{2}$ 时,
\[
x^2=2,\quad \sec\theta=2,\quad \theta=\frac{\pi}{3}
\]

当 $x=\sqrt{\sqrt{6}+\sqrt{2}}$ 时,
\[
x^2=\sqrt{6}+\sqrt{2},\quad \sec\theta=\sqrt{6}+\sqrt{2}
\]

\[
\cos\theta=\frac{1}{\sqrt{6}+\sqrt{2}}
=\frac{\sqrt{6}-\sqrt{2}}{(\sqrt{6}+\sqrt{2})(\sqrt{6}-\sqrt{2})}
=\frac{\sqrt{6}-\sqrt{2}}{4}
\]

\[
\theta=\frac{5\pi}{12}
\]

原积分化为
\begin{align*}
&\int_{\sqrt{2}}^{\sqrt{\sqrt{6}+\sqrt{2}}}
\frac{2}{x\sqrt{x^4-1}}\,dx \\
=&\int_{\frac{\pi}{3}}^{\frac{5\pi}{12}}
\frac{1}{x\sqrt{\sec^2\theta-1}}
\cdot\frac{\sec\theta\tan\theta}{2x}\,d\theta \\
=&\int_{\frac{\pi}{3}}^{\frac{5\pi}{12}}
\frac{\sec\theta\tan\theta}{2x^2\sqrt{\tan^2\theta}}\,d\theta \\
=&\int_{\frac{\pi}{3}}^{\frac{5\pi}{12}}
\frac{\sec\theta\tan\theta}{2\sec\theta\tan\theta}\,d\theta \\
=&\int_{\frac{\pi}{3}}^{\frac{5\pi}{12}}
\frac{1}{2}\,d\theta
\end{align*}

\[
=\left[\frac{1}{2}\theta\right]_{\frac{\pi}{3}}^{\frac{5\pi}{12}}
\]

\[
=\frac{1}{2}\left(\frac{5\pi}{12}-\frac{\pi}{3}\right)
=\frac{1}{2}\cdot\frac{\pi}{12}
=\frac{\pi}{24}
\]
\end{solution}

    \question 
    \[
    \int_{-1}^{1} \frac{1}{(e^x + 1)(1 + x^2)} \, dx
    \]
    \begin{solution}
        巧妙换元$u=-x$,
        \begin{align*}
        \int_{-1}^{1} \frac{dx}{(e^x+1)(1+x^2)} 
        &= \int_{-1}^{0} \frac{dx}{(e^x+1)(1+x^2)} + \int_{0}^{1} \frac{dx}{(e^x+1)(1+x^2)} \\
        &= -\int_{1}^{0} \frac{du}{(e^{-u}+1)(1+u^2)} + \int_{0}^{1} \frac{dx}{(e^x+1)(1+x^2)} \\
        &= \int_{0}^{1} \frac{e^u du}{(1+e^u)(1+u^2)} + \int_{0}^{1} \frac{dx}{(e^x+1)(1+x^2)} \\
        &= \int_{0}^{1} \frac{du}{1+u^2} = \frac{\pi}{4}
        \end{align*}
    \end{solution}  

    \question
使用代换 $u=\sqrt{\frac{1+x}{1-x}}$ 计算积分
\[
\int_{0}^{\frac{1}{4}} \frac{3}{(4x+5)\sqrt{1-x^2}-3(1-x^2)}\,dx
\]

\begin{solution}
先整理给定代换。
\[
u^2=\frac{1+x}{1-x}
\]
于是
\[
u^2(1-x)=1+x
\]
整理得
\[
u^2-1=x(1+u^2)
\]
从而
\[
x=\frac{u^2-1}{u^2+1}
\]

对 $x$ 求导,
\begin{align*}
\frac{dx}{du}
&=\frac{(u^2+1)(2u)-(u^2-1)(2u)}{(u^2+1)^2} \\
&=\frac{4u}{(u^2+1)^2}
\end{align*}
因此
\[
dx=\frac{4u}{(u^2+1)^2}\,du
\]

再分别化简被积式中的各项。
\begin{align*}
1-x^2
&=1-\left(\frac{u^2-1}{u^2+1}\right)^2 \\
&=\frac{(u^2+1)^2-(u^2-1)^2}{(u^2+1)^2} \\
&=\frac{4u^2}{(u^2+1)^2}
\end{align*}
从而
\[
\sqrt{1-x^2}=\frac{2u}{u^2+1}
\]

同时
\begin{align*}
4x+5
&=4\left(\frac{u^2-1}{u^2+1}\right)+5 \\
&=\frac{9u^2+1}{u^2+1}
\end{align*}

积分上下限变为
\[
x=0 \Rightarrow u=1,\qquad x=\frac{1}{4} \Rightarrow u=\sqrt{\frac{5}{3}}
\]

代入原积分得
\[
\int_{1}^{\sqrt{5/3}} 
\frac{3}{\left(\frac{9u^2+1}{u^2+1}\right)\left(\frac{2u}{u^2+1}\right)
-3\left(\frac{4u^2}{(u^2+1)^2}\right)}
\frac{4u}{(u^2+1)^2}\,du
\]

化简后得到
\begin{align*}
&=\int_{1}^{\sqrt{5/3}} \frac{12u}{2u(9u^2+1)-12u^2}\,du \\
&=\int_{1}^{\sqrt{5/3}} \frac{6u}{9u^3-6u^2+u}\,du \\
&=\int_{1}^{\sqrt{5/3}} \frac{6}{9u^2-6u+1}\,du \\
&=\int_{1}^{\sqrt{5/3}} \frac{6}{(3u-1)^2}\,du
\end{align*}

直接积分,
\[
\int \frac{6}{(3u-1)^2}\,du=-\frac{2}{3u-1}
\]

代入上下限,
\begin{align*}
&=\left[-\frac{2}{3u-1}\right]_{1}^{\sqrt{5/3}} \\
&=-\frac{2}{3\sqrt{5/3}-1}+\frac{2}{2}
\end{align*}

继续化简,
\begin{align*}
&=1-\frac{2}{\sqrt{15}-1} \\
&=1-\frac{2(\sqrt{15}+1)}{14} \\
&=\frac{6-\sqrt{15}}{7}
\end{align*}

因此积分结果为
\[
\frac{6-\sqrt{15}}{7}
\]
\end{solution}

\question
使用代换 $x=\frac{ab}{t}$ 求积分
\[
\int_{0}^{\infty} \frac{\ln x}{(x+a)(x+b)}\,dx
\]
其中 $a>b>0$。

\begin{solution}
设
\[
I=\int_{0}^{\infty} \frac{\ln x}{(x+a)(x+b)}\,dx
\]

作代换
\[
x=\frac{ab}{t},\quad dx=-\frac{ab}{t^2}\,dt
\]
当 $x=0$ 时,$t=\infty$;当 $x=\infty$ 时,$t=0$。

则
\[
I=\int_{\infty}^{0} \frac{\ln\left(\frac{ab}{t}\right)}{\left(\frac{ab}{t}+a\right)\left(\frac{ab}{t}+b\right)}\left(-\frac{ab}{t^2}\right)dt
\]

整理得
\begin{align*}
I
&=\int_{0}^{\infty} \frac{\ln(ab)-\ln t}{\frac{ab(b+t)(a+t)}{t^2}} \frac{ab}{t^2}\,dt \\
&=\int_{0}^{\infty} \frac{\ln(ab)-\ln t}{(t+a)(t+b)}\,dt
\end{align*}

因此
\begin{align*}
I
&=\ln(ab)\int_{0}^{\infty} \frac{1}{(t+a)(t+b)}\,dt
 -\int_{0}^{\infty} \frac{\ln t}{(t+a)(t+b)}\,dt \\
&=\ln(ab)\int_{0}^{\infty} \frac{1}{(t+a)(t+b)}\,dt - I
\end{align*}

移项得
\[
2I=\ln(ab)\int_{0}^{\infty} \frac{1}{(t+a)(t+b)}\,dt
\]

对分式作部分分式分解
\[
\frac{1}{(t+a)(t+b)}=\frac{1}{a-b}\left(\frac{1}{t+b}-\frac{1}{t+a}\right)
\]

于是
\begin{align*}
2I
&=\frac{\ln(ab)}{a-b}\int_{0}^{\infty}\left(\frac{1}{t+b}-\frac{1}{t+a}\right)dt \\
&=\frac{\ln(ab)}{a-b}\left[\ln\left(\frac{t+b}{t+a}\right)\right]_{0}^{\infty}
\end{align*}

计算上下限
\[
\left[\ln\left(\frac{t+b}{t+a}\right)\right]_{0}^{\infty}
=0-\ln\left(\frac{b}{a}\right)
=\ln\left(\frac{a}{b}\right)
\]

因此
\[
2I=\frac{\ln(ab)}{a-b}\ln\left(\frac{a}{b}\right)
\]

从而
\[
I=\frac{\ln(ab)}{2(a-b)}\ln\left(\frac{a}{b}\right)
\]
\end{solution}

\question
使用代换求
\[
\int \frac{2 \sin x \sqrt{\tan x} + 1}{2 \cos x \sqrt{\tan x} (\cos x \sqrt{\tan x} + 1)} \, dx
\]
\begin{solution}
代换:
\[
u = \sec x + \sqrt{\tan x} = \sec x + (\tan x)^{\frac{1}{2}}
\]

求导:
\[
\frac{du}{dx} = \sec x \tan x + \frac{1}{2} (\tan x)^{-\frac{1}{2}} \sec^2 x
= \frac{\sin x}{\cos^3 x} + \frac{\cos^{1/2} x}{2 \sin^{1/2} x \cos^2 x}
= \frac{1}{\cos^2 x} \left( \frac{2\sin^{3/2} x + \cos^{3/2} x}{2 \sin^{1/2} x \cos^{1/2} x} \right)
\]

所以:
\[
dx = \frac{2 \cos^3 x \sin^{1/2} x \cos^{1/2} x}{2 \sin^{3/2} x + \cos^{3/2} x} \, du
\]

原积分:
\[
\int \frac{2 \sin x \sqrt{\tan x} + 1}{2 \cos x \sqrt{\tan x} (\cos x \sqrt{\tan x} + 1)} \, dx
\]

代入 \(dx\):
\[
= \int \frac{2 \sin^{3/2} x + \cos^{1/2} x}{2 \cos^{1/2} x \sin^{1/2} x (\cos^{1/2} x \sin^{1/2} x + 1)} 
\times \frac{2 \cos^{3/2} x \sin^{1/2} x}{2 \sin^{3/2} x + \cos^{3/2} x} \, du
\]

化简:
\[
= \int \frac{\cos x}{\cos x \sin x + 1} \, du
\]

两边同时乘 \(\sec x\):
\[
= \int \frac{\sec x}{\sec x (\cos x \sin x + 1)} \, du
= \int \frac{1}{\sin x + \sec x} \, du
= \int \frac{1}{\sqrt{\tan x} + \sec x} \, du
\]

最终积分:
\[
\int \frac{1}{u} \, du = \ln|u| + C = \ln|\sqrt{\tan x} + \sec x| + C
\]
\end{solution}


\question 使用代换求积分
\[
\int \frac{16}{(x+6)(x-2)\sqrt{x+2}} \, dx
\]
\begin{solution}
使用代换 \(u = \sqrt{x+2} \implies x = u^2 - 2, \ dx = 2u \, du\):
\[
\int \frac{16}{(x+6)(x-2)\sqrt{x+2}} \, dx
= \int \frac{16}{(u^2+4)(u^2-4) u} \cdot 2u \, du
= \int \frac{32}{(u^2+4)(u^2-4)} \, du
\]

分解为部分分式:
\[
\frac{32}{(u^2+4)(u^2-4)} = \frac{4}{u^2-4} - \frac{4}{u^2+4} = \frac{1}{u-2} - \frac{1}{u+2} - \frac{4}{u^2+4}
\]

积分得到:
\[
\int \frac{16}{(x+6)(x-2)\sqrt{x+2}} \, dx = \ln|u-2| - \ln|u+2| - 2\arctan\left(\frac{u}{2}\right) + C
\]

回代 \(u = \sqrt{x+2}\):
\[
\int \frac{16}{(x+6)(x-2)\sqrt{x+2}} \, dx = \ln\left|\frac{\sqrt{x+2}-2}{\sqrt{x+2}+2}\right| - 2\arctan\left(\frac{\sqrt{x+2}}{2}\right) + C
\]
\end{solution}

\question
11) 计算不定积分 $\int \frac{\sqrt{x+1} - \sqrt{x-1}}{\sqrt{x+1} + \sqrt{x-1}} dx$

\begin{solution}
分子有理化:
\begin{align*}
\int \frac{(\sqrt{x+1} - \sqrt{x-1})^2}{(x+1) - (x-1)} dx &= \int \frac{x+1 + x-1 - 2\sqrt{x^2-1}}{2} dx \\
&= \int (x - \sqrt{x^2-1}) dx \\
&= \frac{1}{2}x^2 - \left( \frac{x}{2}\sqrt{x^2-1} - \frac{1}{2} \ln|x + \sqrt{x^2-1}| \right) + C \\
&= \frac{1}{2}x^2 - \frac{x}{2}\sqrt{x^2-1} + \frac{1}{2} \ln|x + \sqrt{x^2-1}| + C
\end{align*}
\end{solution}


\question 使用代换 $x = e^{-u/2}$求
\[
\int \frac{x^4 - 1}{x^2 \sqrt{x^4 + 1}} \, dx
\]
\begin{solution}
使用代换 \(x = e^{-u/2} \implies dx = -\frac{1}{2}e^{-u/2} \, du\):
\[
\int \frac{x^4 - 1}{x^2 \sqrt{x^4 + 1}} \, dx
= \int \frac{e^{-2u} - 1}{e^{-u} \sqrt{e^{-4u} + 1}} \left(-\frac{1}{2}e^{-u/2} \, du\right)
\]

化简得到:
\[
= \frac{1}{\sqrt{2}} \int \sinh u \, (\cosh u)^{-1/2} \, du
= \frac{1}{\sqrt{2}} \cdot 2 (\cosh u)^{1/2} + C
= \sqrt{2 \cosh u} + C
\]

回代 \(x = e^{-u/2} \implies \cosh u = \frac{1}{2}(e^{u} + e^{-u}) = \frac{1}{2}\left(\frac{1}{x^2} + x^2\right)\):
\[
\sqrt{2\cosh u} + C = \sqrt{\frac{1}{x^2} + x^2} + C = \frac{\sqrt{x^4 + 1}}{x} + C
\]

因此积分的简化形式为:
\[
\int \frac{x^4 - 1}{x^2 \sqrt{x^4 + 1}} \, dx = \frac{\sqrt{x^4 + 1}}{x} + C
\]
\end{solution}

\question
已知
\[
x^2 + x + 2 = (u-x)^2
\]
求
\[
\int \frac{1}{x\sqrt{x^2+x+2}} \, dx
\]
\begin{solution}
a) (i) 解 $x$ 关于 $u$ 的关系:

\[
x^2 + x + 2 = u^2 - 2ux + x^2
\]

\[
x + 2 = u^2 - 2ux
\]

\[
x(2u+1) = u^2 - 2
\]

\[
x = \frac{u^2 - 2}{2u+1}
\]

(ii) 对 $x$ 求导:
\[
\frac{dx}{du} = \frac{(2u+1)(2u) - 2(u^2-2)}{(2u+1)^2} = \frac{4u^2 + 2u - 2u^2 + 4}{(2u+1)^2} = \frac{2u^2 + 2u + 4}{(2u+1)^2} = \frac{2(u^2 + u + 2)}{(2u+1)^2}
\]

b) 积分变换:
\[
\int \frac{1}{x\sqrt{x^2+x+2}} \, dx = \int \frac{1}{x(u-x)} \cdot \frac{2(u^2+u+2)}{(2u+1)^2} \, du
\]

代入 $x = \frac{u^2-2}{2u+1}$ 和 $u-x = \sqrt{x^2+x+2}$:

\[
\int \frac{1}{\frac{u^2-2}{2u+1} \cdot \frac{u^2+u+2}{2u+1}} \cdot \frac{2(u^2+u+2)}{(2u+1)^2} \, du
= \int \frac{(2u+1)^2}{(u^2-2)(u^2+u+2)} \cdot \frac{2(u^2+u+2)}{(2u+1)^2} \, du
= \int \frac{2}{u^2-2} \, du
\]

使用标准积分公式:
\[
\int \frac{1}{u^2-a^2} \, du = \frac{1}{2a} \ln \left| \frac{u-a}{u+a} \right| + C
\]

得:
\[
\int \frac{2}{u^2-2} \, du = \frac{1}{\sqrt{2}} \ln \left| \frac{u-\sqrt{2}}{u+\sqrt{2}} \right| + C
\]

最后代回 $u = x + \sqrt{x^2+x+2}$:
\[
\int \frac{1}{x\sqrt{x^2+x+2}} \, dx = \frac{1}{\sqrt{2}} \ln \left| \frac{x + \sqrt{x^2+x+2} - \sqrt{2}}{x + \sqrt{x^2+x+2} + \sqrt{2}} \right| + C
\]
\end{solution}

\question 计算积分
\[
\int_{0.2}^{0.5} \frac{\sqrt{x-x^2}}{x^4} \, dx
\]
\begin{solution}
作代换:
\[
x = \frac{1}{u^2+1} \implies dx = -\frac{2u}{(u^2+1)^2} \, du
\]

积分上下限变化:
\[
x=0.2 \implies u = 2, \quad x=0.5 \implies u = 1
\]

代入积分:
\[
\int_{0.2}^{0.5} \frac{\sqrt{x-x^2}}{x^4} \, dx = \int_{2}^{1} \frac{\sqrt{\frac{1}{u^2+1}-\frac{1}{(u^2+1)^2}}}{\left(\frac{1}{u^2+1}\right)^4} \left(-\frac{2u}{(u^2+1)^2} \, du\right)
\]

整理被积函数:
\[
\sqrt{\frac{1}{u^2+1}-\frac{1}{(u^2+1)^2}} = \sqrt{\frac{u^2}{(u^2+1)^2}} = \frac{u}{u^2+1}
\]

\[
\frac{1}{\left(\frac{1}{u^2+1}\right)^4} = (u^2+1)^4
\]

因此积分变为:
\[
\int_{1}^{2} \frac{u}{u^2+1} (u^2+1)^4 \cdot \frac{2u}{(u^2+1)^2} \, du = \int_{1}^{2} 2 u^2 (u^2+1) \, du
\]

展开:
\[
\int_{1}^{2} 2u^4 + 2u^2 \, du = \left[ \frac{2}{5} u^5 + \frac{2}{3} u^3 \right]_{1}^{2}
\]

代入上下限:
\[
\left( \frac{64}{5} + \frac{16}{3} \right) - \left( \frac{2}{5} + \frac{2}{3} \right) = \frac{256}{15}
\]

\[
\therefore \int_{0.2}^{0.5} \frac{\sqrt{x-x^2}}{x^4} \, dx = \frac{256}{15}
\]
\end{solution}


    \question
    \[
    \int \frac{dx}{x^2(a + bx)^2},\ a,b\in\mathbb{R}
    \]
    \begin{solution}
        设 \( u = \dfrac{a + bx}{x} \),则
        \[
        x = \frac{a}{u - b}, \quad 
        dx = \frac{-a\,du}{(u - b)^2}, \quad 
        a + bx = x u = \frac{a u}{u - b}
        \]
        代入原式:
        \begin{align*}
        \int \frac{dx}{x^2(a + bx)^2}
        &= \int \frac{-a\,du}{(u - b)^2} \cdot \frac{1}{x^2 (a + bx)^2} \\
        &= \int \frac{-a\,du}{(u - b)^2} \cdot \frac{(u - b)^4}{a^4 u^2} \\
        &= -\frac{1}{a^3} \int \frac{(u - b)^2}{u^2} \, du \\
        &= -\frac{1}{a^3} \int \left( 1 - \frac{2b}{u} + \frac{b^2}{u^2}\right) du \\
        &= -\frac{1}{a^3} \left( u - 2b \ln|u| - \frac{b^2}{u} \right) + C
        \end{align*}
        代回 \( u = \frac{a + bx}{x} \),得:
        \[
        \int \frac{dx}{x^2(a + bx)^2}
        = -\frac{1}{a^3} \left( \frac{a + bx}{x} - 2b \ln\left| \frac{a + bx}{x} \right| - \frac{b^2 x}{a + bx} \right) + C
        \]
    \end{solution}

    \question
使用代换$2x+1=4\cosh\theta$,求不定积分
\[
\int \sqrt{(2x+5)(2x-3)} \, dx
\]

\begin{solution}
先化简被积式,
\[
(2x+5)(2x-3)=4x^2+4x-15=(2x+1)^2-16
\]
因此原积分化为
\[
\int \sqrt{(2x+1)^2-16} \, dx
\]

采用双曲函数换元,令
\[
2x+1=4\cosh\theta
\]
则
\[
2\,dx=4\sinh\theta\,d\theta,\quad dx=2\sinh\theta\,d\theta
\]

利用恒等式
\[
\cosh^2\theta-\sinh^2\theta=1
\]
可得
\[
\sqrt{(2x+1)^2-16}
=\sqrt{16(\cosh^2\theta-1)}
=4\sinh\theta
\]

代入积分,
\[
\int \sqrt{(2x+1)^2-16}\,dx
=\int 4\sinh\theta\,(2\sinh\theta\,d\theta)
=8\int\sinh^2\theta\,d\theta
\]

又有
\[
\sinh^2\theta=\frac{1}{2}\cosh(2\theta)-\frac{1}{2}
\]
于是
\begin{align*}
8\int\sinh^2\theta\,d\theta
&=8\int\left(\frac{1}{2}\cosh(2\theta)-\frac{1}{2}\right)d\theta \\
&=4\int\cosh(2\theta)\,d\theta-4\int d\theta \\
&=2\sinh(2\theta)-4\theta+C
\end{align*}

利用
\[
\sinh(2\theta)=2\sinh\theta\cosh\theta
\]
得
\[
2\sinh(2\theta)=4\sinh\theta\cosh\theta
\]

又因为
\[
\sinh\theta=\frac{\sqrt{(2x+1)^2-16}}{4},\quad
\cosh\theta=\frac{2x+1}{4}
\]
且
\[
\theta=\operatorname{arccosh}\frac{2x+1}{4}
\]

代回原变量,
\[
\int \sqrt{(2x+5)(2x-3)} \, dx
=\frac{1}{4}(2x+1)\sqrt{(2x+5)(2x-3)}
-4\operatorname{arccosh}\frac{2x+1}{4}+C
\]

利用
\[
\operatorname{arccosh}u=\ln\left(u+\sqrt{u^2-1}\right)
\]
可进一步写为
\[
\int \sqrt{(2x+5)(2x-3)} \, dx
=\frac{1}{4}(2x+1)\sqrt{(2x+5)(2x-3)}
-4\ln\left(2x+1+\sqrt{(2x+5)(2x-3)}\right)+C
\]

\end{solution}

\question
已知
\[
\sqrt{5-4x-x^2} = (1-x)u, \quad x\neq 1, x\neq -5
\]

\begin{solution}
a) 代换与求导

i) 两边平方
\[
5-4x-x^2 = (1-x)^2 u^2
\]

移项整理
\[
-(x^2+4x-5) = u^2(x-1)^2 \implies -(x-1)(x+5) = u^2(x-1)^2
\]

两边除以 $x-1$
\[
-(x+5) = u^2 (x-1) \implies u^2 - 5 = x(u^2+1) \implies x = \frac{u^2-5}{u^2+1}
\]

ii) 对 x 求导
\[
dx = \frac{d}{du} \frac{u^2-5}{u^2+1} \, du
= \frac{(u^2+1)(2u) - (u^2-5)(2u)}{(u^2+1)^2} \, du
= \frac{12u}{(u^2+1)^2} \, du
\]

b) 积分化简

由 a) 可得
\[
\int \frac{x}{(5-4x-x^2)^{3/2}} \, dx
= \int \frac{x}{((1-x)u)^3} \cdot \frac{12u}{(u^2+1)^2} \, du
\]

\[
x = \frac{u^2-5}{u^2+1}, \quad (1-x)^3 = \left(\frac{6}{u^2+1}\right)^3
\]

\[
\implies \int \frac{x}{(5-4x-x^2)^{3/2}} \, dx
= \int \frac{u^2-5}{u^2+1} \cdot \frac{12u}{216 u^3} \, du
= \int \frac{u^2-5}{18 u^2} \, du
\]

c) 积分求解

分拆积分
\[
\int \frac{u^2-5}{18 u^2} \, du = \int \frac{1}{18} - \frac{5}{18u^2} \, du
= \frac{1}{18} u + \frac{5}{18u} + C
\]

代回 u
\[
u = \frac{\sqrt{5-4x-x^2}}{1-x} \implies \frac{1}{u} = \frac{1-x}{\sqrt{5-4x-x^2}}
\]

\[
\implies \int \frac{x}{(5-4x-x^2)^{3/2}} \, dx
= \frac{1}{18} \frac{\sqrt{5-4x-x^2}}{1-x} + \frac{5}{18} \frac{1-x}{\sqrt{5-4x-x^2}} + C
\]

整理为
\[
\int \frac{x}{(5-4x-x^2)^{3/2}} \, dx
= \frac{5-2x}{9 \sqrt{5-4x-x^2}} + C
\]
\end{solution}

\question
28) 计算不定积分:\[ \int \frac{\sqrt{1 + \sqrt[3]{x}}}{\sqrt[3]{x^2}} dx \]

\begin{solution}
设 $u = \sqrt[3]{x}$,则 $du = \frac{1}{3} x^{-\frac{2}{3}} dx$,即 $dx = 3 \sqrt[3]{x^2} du$。
代入积分式:
\[ I = 3 \int \sqrt{1+u} du \]
进行积分计算:
\[ I = 2 (1+u)^{\frac{3}{2}} + C \]
还原变量 $u = \sqrt[3]{x}$:
\[ I = 2 (1 + \sqrt[3]{x})^{\frac{3}{2}} + C \]
\end{solution}
\question
16) 计算不定积分:\[ \int \frac{1}{x(1+x^{119})} dx \]

\begin{solution}
分子分母同乘 $x^{118}$:
\[ \int \frac{x^{118}}{x^{119}(1+x^{119})} dx \]
设 $t = x^{119}$,则 $dt = 119x^{118} dx$:
\[ I = \frac{1}{119} \int \frac{1}{t(1+t)} dt = \frac{1}{119} \int \left( \frac{1}{t} - \frac{1}{1+t} \right) dt \]
\[ = \frac{1}{119} \ln\left| \frac{t}{1+t} \right| + C = \frac{1}{119} \ln\left| \frac{x^{119}}{1+x^{119}} \right| + C \]
\end{solution}
    \question  
    \[
    \int_{-1}^1 \frac{x^2}{1+2^x} \,dx 
    \]
    \begin{solution}
        设 \( u = -x \),则
        \[
        I = \int_{-1}^1 \frac{u^2}{1+2^{-u}}\,du = \int_{-1}^1 \frac{u^2 2^u}{1+2^u}\,du = \int_{-1}^1 \frac{x^2 2^x}{1+2^x}\,dx
        \]
        于是有
        \[
        2I = \int_{-1}^1 \frac{x^2(2^x + 1)}{1 + 2^x}\,dx = \int_{-1}^1 x^2\,dx = \frac{2}{3} \Rightarrow I = \frac{1}{3}
        \]
    \end{solution}

    \question 计算积分
\[
I = \int_{\frac{1}{2}}^{2} \frac{\ln x}{1 + x^2} \, dx.
\]

\begin{solution}
将积分分成两部分:
\[
I = \int_{\frac{1}{2}}^{1} \frac{\ln x}{1 + x^2} \, dx + \int_{1}^{2} \frac{\ln x}{1 + x^2} \, dx.
\]

在第二个积分中作代换 $x = 1/u$,则 $dx = -\frac{1}{u^2} du$,积分限变为 $x=1 \to u=1,x=2 \to u=1/2$:
\[
\int_{1}^{2} \frac{\ln x}{1 + x^2} \, dx = \int_{1}^{1/2} \frac{\ln (1/u)}{1 + (1/u)^2} \left(-\frac{1}{u^2} du\right) = -\int_{1/2}^{1} \frac{-\ln u}{1 + u^2} \, du = -\int_{1/2}^{1} \frac{\ln u}{1 + u^2} \, du.
\]

因此
\[
I = \int_{1/2}^{1} \frac{\ln x}{1 + x^2} \, dx - \int_{1/2}^{1} \frac{\ln u}{1 + u^2} \, du = 0.
\]
\end{solution}

\question 
30) 计算定积分 $I = \int_{0}^{\sqrt{3}} \frac{1}{1+x^2} \sin^{-1} \frac{2x}{1+x^2} dx$

\begin{solution}
使用换元法,设 $x = \tan\theta$,则 $dx = \sec^2 \theta d\theta$ 且 $\frac{1}{1+x^2} = \cos^2 \theta$。
同时利用二倍角公式可知 $\sin^{-1} \frac{2x}{1+x^2} = \sin^{-1}(\sin 2\theta) = 2\theta$。

更换积分上下限:
\begin{itemize}
    \item 当 $x = 0$ 时,$\theta = 0$
    \item 当 $x = \sqrt{3}$ 时,$\theta = \frac{\pi}{3}$
\end{itemize}

代入原积分式:
\begin{align*}
I &= \int_{0}^{\frac{\pi}{3}} \cos^2 \theta \cdot (2\theta) \cdot \sec^2 \theta d\theta \\
&= \int_{0}^{\frac{\pi}{3}} 2\theta d\theta \\
&= \left[ \theta^2 \right]_{0}^{\frac{\pi}{3}} \\
&= \left( \frac{\pi}{3} \right)^2 - 0 \\
&= \frac{\pi^2}{9}
\end{align*}
\end{solution}

\question 计算 $[\int_{1/2}^2 \frac{x^{2012}-1}{x^{2014}+1} dx]$。
\begin{solution}
令 $x = \frac{1}{t}$,则 $dx = -\frac{1}{t^2} dt$。
积分上下限从 $1/2 \to 2$ 变为 $2 \to 1/2$:
\[I = [\int_2^{1/2} \frac{(1/t)^{2012}-1}{(1/t)^{2014}+1} (-\frac{1}{t^2}) dt]\]
\[I = [\int_{1/2}^2 \frac{\frac{1-t^{2012}}{t^{2012}}}{\frac{1+t^{2014}}{t^{2014}}} \frac{1}{t^2} dt] = [\int_{1/2}^2 \frac{1-t^{2012}}{1+t^{2014}} dt]\]
注意到 $[\int_{1/2}^2 \frac{1-x^{2012}}{1+x^{2014}} dx] = -[\int_{1/2}^2 \frac{x^{2012}-1}{x^{2014}+1} dx] = -I$。
即 $I = -I$,故 $I = 0$。
\end{solution}

    \question 若正整数$k$满足
    \[
    \left(\int_0^1 x^{2018}(2019 + kx^{10})\sqrt{1 + x^{10}} \, dx\right)^2 = 8,
    \]
    求$k$的最大值。
    \begin{solution}
        由于被积函数随$k$严格递增,则当
        \[
        \int_0^1 x^{2018}(2019 + kx^{10})\sqrt{1 + x^{10}} \, dx = \sqrt{8}
        \]
        时$k$取最大。欲写成形如
        \[
        \int_0^1 (2019x^{2018-n} + kx^{2028-n}) \sqrt{x^{2n} + x^{2n+10}} \, dx
        \]
        的积分,考虑强迫
        \[
        2n-1 = 2018-n,
        \]
        可得$n = 673$,此时设$u=x^{1346}+x^{1356}$,则$du=1346x^{1345}+1356x^{1355} dx $,于是
        \begin{align*}
        I &=\int_0^1 (2019x^{1345} + kx^{1355}) \sqrt{x^{1346} + x^{1356}} \, dx \\
        &=\frac{3}{2}\int_0^1 (1346^{1345} + \frac{2}{3}kx^{1355}) \sqrt{x^{1346} + x^{1356}} \, dx
        \end{align*}
        当$\dfrac{2}{3}k=1356$即$k=2034$时,有
        \[
        I=\frac{3}{2}\int_0^2 \sqrt{u}\, du = \frac{3}{2}\left[\frac{2}{3}u^{\frac{3}{2}}\right]^2_0=\sqrt{8}
        \]
    \end{solution}
%部分分式
    \question 
    \[
    \int \frac{dx}{x^3 + 1}
    \]
    \begin{solution}
        \[
        \int \frac{dx}{x^3 + 1}
        = \int \frac{dx}{(x + 1)(x^2 - x + 1)}
        \]
        部分分式分解得
        \[
        \frac{1}{x^3 + 1}
        = \frac{1}{3(x + 1)} + \frac{-\frac{1}{3}x + \frac{2}{3}}{x^2 - x + 1}
        \]
        于是
        \[
        \int \frac{dx}{x^3 + 1}
        = \frac{1}{3} \int \frac{dx}{x + 1}
        - \frac{1}{3} \int \frac{x}{x^2 - x + 1} dx
        + \frac{2}{3} \int \frac{dx}{x^2 - x + 1}
        \]
        其中
        \[
        \int \frac{dx}{x + 1} = \ln|x + 1|,\quad
        \int \frac{x}{x^2 - x + 1} dx = \frac{1}{2} \ln(x^2 - x + 1)
        \]
        \[
        \int \frac{dx}{x^2 - x + 1}
        = \int \frac{dx}{\left(x - \frac{1}{2}\right)^2 + \left(\frac{\sqrt{3}}{2}\right)^2}
        = \frac{2}{\sqrt{3}} \arctan\left( \frac{2x - 1}{\sqrt{3}} \right) + C
        \]
        故
        \[
        \int \frac{dx}{x^3 + 1}
        = \frac{1}{3} \ln|x + 1|
        - \frac{1}{6} \ln(x^2 - x + 1)
        + \frac{1}{\sqrt{3}} \arctan\left( \frac{2x - 1}{\sqrt{3}} \right) + C
        \]
    \end{solution}
\question
17) 计算不定积分:\[ \int \frac{x^4+1}{x^6+1} dx \]

\begin{solution}
由于 $x^6+1 = (x^2+1)(x^4-x^2+1)$,拆分分子:
\[ \int \frac{(x^4-x^2+1) + x^2}{x^6+1} dx = \int \frac{1}{x^2+1} dx + \int \frac{x^2}{x^6+1} dx \]
第一部分:$\int \frac{1}{x^2+1} dx = \tan^{-1} x$。
第二部分:设 $u = x^3, du = 3x^2 dx$:
\[ \frac{1}{3} \int \frac{1}{u^2+1} du = \frac{1}{3} \tan^{-1} u = \frac{1}{3} \tan^{-1} x^3 \]
合并结果:
\[ I = \tan^{-1} x + \frac{1}{3} \tan^{-1} x^3 + C \]
\end{solution}
    \question
计算积分
\[
\int_{0}^{1} \frac{x^4(1-x)^4}{x^2+1} \, dx
\]

\begin{solution}
首先展开分子 \( (1-x)^4 \):
\[
(1-x)^4 = 1 - 4x + 6x^2 - 4x^3 + x^4
\]

因此积分变为:
\[
\int_{0}^{1} \frac{x^4 (1-4x+6x^2-4x^3+x^4)}{x^2+1} \, dx
= \int_{0}^{1} \frac{x^8 - 4x^7 + 6x^6 - 4x^5 + x^4}{x^2+1} \, dx
\]

进行多项式除法,将分子按 \(x^2+1\) 除:
\[
\frac{x^8 - 4x^7 + 6x^6 - 4x^5 + x^4}{x^2+1} = x^6 - 4x^5 + 5x^4 - 4x^2 + 4 - \frac{4}{x^2+1}
\]

于是积分拆分为:
\[
\int_{0}^{1} x^6 - 4x^5 + 5x^4 - 4x^2 + 4 - \frac{4}{x^2+1} \, dx
\]

分别积分各项:
\[
\int x^6 \, dx = \frac{1}{7} x^7, \quad
\int -4x^5 \, dx = -\frac{2}{3} x^6, \quad
\int 5x^4 \, dx = x^5,
\]
\[
\int -4x^2 \, dx = -\frac{4}{3} x^3, \quad
\int 4 \, dx = 4x, \quad
\int -\frac{4}{x^2+1} \, dx = -4 \arctan x
\]

组合结果:
\[
\int_{0}^{1} \frac{x^4(1-x)^4}{x^2+1} \, dx
= \left[ \frac{1}{7}x^7 - \frac{2}{3}x^6 + x^5 - \frac{4}{3}x^3 + 4x - 4\arctan x \right]_{0}^{1}
\]

代入上下限:
\[
\left( \frac{1}{7} - \frac{2}{3} + 1 - \frac{4}{3} + 4 - 4 \cdot \frac{\pi}{4} \right) - 0
= \frac{1}{7} - \frac{6}{3} + 5 - \pi
= \frac{1}{7} - 2 + 5 - \pi
= 3 + \frac{1}{7} - \pi
= \frac{22}{7} - \pi
\]

因此最终结果为:
\[
\int_{0}^{1} \frac{x^4(1-x)^4}{x^2+1} \, dx = \frac{22}{7} - \pi
\]
\end{solution}

    \question 
    \[
    \int_{0}^{1} \frac{(x^2+1)(x^2+4)}{(x^2+3)(x^2-4)} \, dx
    \]
    \begin{solution}
        \[
\int_{0}^{1} \frac{(x^2+1)(x^2+4)}{(x^2+3)(x^2-4)} \, dx = \int_{0}^{1} \left[1 + \frac{2}{7}\left[\frac{1}{x^2+3} + \frac{5}{x-2} - \frac{5}{x+2}\right]\right] \, dx
\]
\[
= \left[x + \frac{2}{7}\left[\frac{1}{\sqrt{3}}\arctan\left(\frac{x}{\sqrt{3}}\right) + 5\ln|x-2| - 5\ln|x+2|\right]\right]_{0}^{1}
\]
\[
= \left[1 + \frac{2}{7}\left[\frac{1}{\sqrt{3}}\times\frac{\pi}{6} + 5\ln|-1| - 5\ln|3|\right]\right] - \left[\frac{2}{7}[5\ln|-2| - 5\ln|2|]\right]
\]
\[
= 1 + \frac{2}{7}\left[\frac{\pi}{6\sqrt{3}} - 5\ln3\right]
\]
\[
= 1 + \frac{1}{7}\left[\frac{\pi}{3\sqrt{3}} - 10\ln3\right]
\]
    \end{solution}

    \question
计算积分
\[
\int_{0}^{\frac{1}{3}} \frac{32x^2}{(x^2-1)(x+1)^3}\,dx
\]

\begin{solution}
先化简被积函数
\[
\int_{0}^{\frac{1}{3}} \frac{32x^2}{(x^2-1)(x+1)^3}\,dx
= \int_{0}^{\frac{1}{3}} \frac{32x^2}{(x-1)(x+1)^4}\,dx
\]

作代换
\[
u=x+1,\quad x=u-1,\quad du=dx
\]
当 $x=0$ 时,$u=1$  
当 $x=\frac{1}{3}$ 时,$u=\frac{4}{3}$  

则积分化为
\[
\int_{1}^{\frac{4}{3}} \frac{32(u-1)^2}{(u-2)u^4}\,du
= \int_{1}^{\frac{4}{3}} \frac{32}{u^4}\cdot\frac{u^2-2u+1}{u-2}\,du
\]

将有理函数分解后得
\[
\int_{1}^{\frac{4}{3}}
\left(
-\frac{16}{u^4}
+\frac{24}{u^3}
-\frac{4}{u^2}
-\frac{2}{u}
+\frac{2}{u-2}
\right)\,du
\]

逐项积分
\[
=\left[
\frac{16}{3u^3}
-\frac{12}{u^2}
+\frac{4}{u}
-2\ln|u|
+2\ln|u-2|
\right]_{1}^{\frac{4}{3}}
\]

代入上、下限
\[
=\left(
\frac{16}{3\left(\frac{4}{3}\right)^3}
-\frac{12}{\left(\frac{4}{3}\right)^2}
+\frac{4}{\frac{4}{3}}
-2\ln\frac{4}{3}
+2\ln\frac{2}{3}
\right)
-\left(
\frac{16}{3}-12+4
\right)
\]

化简得
\[
=11-\frac{16}{3}-\frac{9}{2}-2\ln2
\]

整理为
\[
\frac{7}{6}-2\ln2
\]

因此
\[
\int_{0}^{\frac{1}{3}} \frac{32x^2}{(x^2-1)(x+1)^3}\,dx
= \frac{7}{6}-2\ln2
\]
\end{solution}

\question
24) 计算不定积分:\[ \int \frac{1}{\sqrt[4]{1+x^4}} dx \]

\begin{solution}
使用换元法,设 $t^4 x^4 = 1+x^4$,则 $x^4 = \frac{1}{t^4-1}$。
两边求导得 $4x^3 dx = -\frac{4t^3}{(t^4-1)^2} dt$,故 $dx = -\frac{t^3}{(t^4-1)^2 x^3} dt$。
代入积分式:
\[ I = -\int \frac{1}{t \cdot x} \cdot \frac{t^3}{(t^4-1)^2 x^2} dt = -\int \frac{t^2}{t^4-1} dt \]
将被积函数进行部分分式拆分:
\[ -\int \frac{t^2}{t^4-1} dt = -\frac{1}{2} \int \left( \frac{1}{t^2-1} + \frac{1}{t^2+1} \right) dt \]
进行积分计算:
\[ I = -\frac{1}{4} \ln\left| \frac{t-1}{t+1} \right| - \frac{1}{2} \tan^{-1} t + C \]
其中还原变量为 $t = \frac{\sqrt[4]{1+x^4}}{x}$。
\end{solution}

\question
20) 计算不定积分:\[ \int \frac{2}{\sqrt{x} - 3\sqrt[4]{x} + 2} dx \]

\begin{solution}
使用换元法,设 $\sqrt[4]{x} = t$,则 $x = t^4, dx = 4t^3 dt$。
代入原积分式:
\[ I = \int \frac{8t^3}{t^2 - 3t + 2} dt \]
对被积函数进行多项式除法与部分分式拆分:
\[ \frac{8t^3}{t^2 - 3t + 2} = 8t + 24 + \frac{56t - 48}{t^2 - 3t + 2} = 8t + 24 + \frac{64}{t-2} - \frac{8}{t-1} \]
进行积分计算:
\begin{align*}
I &= \int (8t + 24) dt + \int \frac{64}{t-2} dt - \int \frac{8}{t-1} dt \\
&= 4t^2 + 24t + 64 \ln|t-2| - 8 \ln|t-1| + C
\end{align*}
还原变量 $t = \sqrt[4]{x}$:
\[ I = 4\sqrt{x} + 24\sqrt[4]{x} + 64 \ln|\sqrt[4]{x}-2| - 8 \ln|\sqrt[4]{x}-1| + C \]
\end{solution}

\question
22) 计算不定积分:\[ \int \frac{3x^3}{x^4 + x^3 + x + 1} dx \]

\begin{solution}
首先对分母进行因式分解:
\[ x^4 + x^3 + x + 1 = x^3(x+1) + (x+1) = (x^3+1)(x+1) = (x^2-x+1)(x+1)^2 \]
将被积函数拆分为部分分式:
\[ \frac{3x^3}{(x^2-x+1)(x+1)^2} = \frac{x-1}{x^2-x+1} + \frac{2}{x+1} - \frac{1}{(x+1)^2} \]
逐步进行积分:
\begin{align*}
I &= \int \frac{x-1}{x^2-x+1} dx + \int \frac{2}{x+1} dx - \int \frac{1}{(x+1)^2} dx \\
&= \frac{1}{2} \int \frac{2x-1-1}{x^2-x+1} dx + 2 \ln|x+1| + \frac{1}{x+1} \\
&= \frac{1}{2} \ln(x^2-x+1) - \frac{1}{2} \int \frac{1}{(x-\frac{1}{2})^2 + \frac{3}{4}} dx + 2 \ln|x+1| + \frac{1}{x+1}
\end{align*}
计算得到最终结果:
\[ I = \ln\left|(x+1)^2 \sqrt{x^2-x+1}\right| - \frac{1}{\sqrt{3}} \tan^{-1}\left(\frac{2x-1}{\sqrt{3}}\right) + \frac{1}{x+1} + C \]
\end{solution}
    \question
\[
\int \frac{x^2(x^4+1)}{\sqrt{x^4+2}} \, dx
\]

\begin{solution}
先作代数变形
\[
\int \frac{x^2(x^4+1)}{\sqrt{x^4+2}} \, dx
= \int x^2(x^4+1)(x^4+2)^{-\frac12} \, dx
\]

将分子整理
\[
= \int (x^7+x^3)(x^4)^{-\frac12}(x^4+2)^{-\frac12} \, dx
\]

合并为一个整体
\[
= \int (x^7+x^3)[x^4(x^4+2)]^{-\frac12} \, dx
\]

即
\[
= \int (x^7+x^3)(x^8+2x^4)^{-\frac12} \, dx
\]

注意到
\[
\frac{d}{dx}(x^8+2x^4)=8x^7+8x^3=8(x^7+x^3)
\]

于是
\[
= \frac18 \int 8(x^7+x^3)(x^8+2x^4)^{-\frac12} \, dx
\]

积分得
\[
= \frac18 \cdot \frac{1}{\frac12}(x^8+2x^4)^{\frac12}+C
\]

化简
\[
= \frac14 (x^8+2x^4)^{\frac12}+C
\]
\end{solution}
\question
14) 计算不定积分:\[ \int \frac{x^2+1}{(x^2-2x+2)^3} dx \]

\begin{solution}
首先对分子进行拆分,凑出分母的导数项与基本项:
\[ \int \frac{(x^2-2x+2)+(2x-2)+1}{(x^2-2x+2)^3} dx \]
将其拆分为三个部分进行积分:
\[ = \int \frac{dx}{(x^2-2x+2)^2} + \int \frac{(2x-2)dx}{(x^2-2x+2)^3} + \int \frac{dx}{(x^2-2x+2)^3} \]

对第一项和第三项进行配方:
\[ x^2-2x+2 = (x-1)^2+1 \]
设 $x-1 = \tan t$,则 $dx = \sec^2 t dt$。
代入后积分式变为:
\[ \int \cos^2 t dt + \int \cos^4 t dt - \frac{1}{2(x^2-2x+2)^2} \]

利用降幂公式计算三角积分:
\[ \int \cos^2 t dt = \frac{1}{2} \int (1+\cos 2t) dt = \frac{t}{2} + \frac{\sin 2t}{4} \]
\[ \int \cos^4 t dt = \frac{1}{4} \int (1+\cos 2t)^2 dt = \dots = \frac{3t}{8} + \frac{\sin 2t}{4} + \frac{\sin 4t}{32} \]

将各部分结果合并:
\[ I = \frac{7t}{8} + \frac{\sin 2t}{2} + \frac{\sin 4t}{32} - \frac{1}{2(x^2-2x+2)^2} + C \]

利用三角恒等式将 $t$ 还原为 $x$:
\[ \sin 2t = \frac{2(x-1)}{x^2-2x+2}, \quad \sin 4t = \frac{4(x-1)(1-2\sin^2 t)}{(x^2-2x+2)^2} \]
最终整理得到原文结果:
\[ I = \frac{7}{8} \tan^{-1}(x-1) + \frac{x-1}{x^2-2x+2} + \frac{(x-1)(1-2\sin^2 t)}{8(x^2-2x+2)^2} - \dots + C \]
\[ I = \frac{7}{8} \tan^{-1}(x-1) + \frac{7x^3-21x^2+30x-20}{8(x^2-2x+2)^2} + C \]
\end{solution}
\question
40) 计算不定积分:\[ \int \sqrt[3]{\frac{2x+1}{x+3}} dx \]

\begin{solution}
使用换元法,设 $t^3 = \frac{2x+1}{x+3}$,则可求得 $x = \frac{3t^3-1}{2-t^3}$。
对 $x$ 求导得 $dx = \frac{15t^2}{(2-t^3)^2} dt$。
代入原积分式:
\[ I = 15 \int \frac{t^3}{(2-t^3)^2} dt \]
根据原文推导过程,利用分部积分或拆分项处理:
\[ = 5 \int \left( -\frac{1}{2-t^3} + \frac{2(t^3+1)}{(2-t^3)^2} \right) dt = -5 \int \frac{1}{2-t^3} dt + \frac{5t}{2-t^3} \]
进一步整理为:
\[ = 5 \int \frac{1}{t^3-2} dt - \frac{5t}{t^3-2} \]
利用部分分式进行拆分:
\[ = 5 \int \left( \frac{-\frac{\sqrt[3]{2}}{6}t - \frac{\sqrt[3]{4}}{3}}{t^2 + \sqrt[3]{2}t + \sqrt[3]{4}} + \frac{\frac{\sqrt[3]{2}}{6}}{t-\sqrt[3]{2}} \right) dt - \frac{5t}{t^3-2} \]
进行积分计算并还原变量 $t = \sqrt[3]{\frac{2x+1}{x+3}}$:
\[ I = \frac{5\sqrt[3]{108}}{6} \tan^{-1} \frac{\sqrt{3}(\sqrt[3]{4}t+1)}{3} - \frac{5\sqrt[3]{2}}{12} \ln(t^2+\sqrt[3]{2}t+\sqrt[3]{4}) + \frac{5\sqrt[3]{2}}{6} \ln|t-\sqrt[3]{2}| - \frac{5t}{t^3-2} + C \]
\end{solution}

\question
42) 计算不定积分:\[ \int \frac{dx}{\sqrt{x+1} + \sqrt[4]{x+1}} \]

\begin{solution}
设 $t = x+1$,则 $dt = dx$。
积分变为:
\[ \int \frac{dt}{\sqrt{t} + \sqrt[4]{t}} = \int \frac{dt}{\sqrt{t}(\sqrt[4]{t}+1)} \]
再设 $u = \sqrt[4]{t}$,则 $t = u^4, dt = 4u^3 du$。
代入积分式:
\[ \int \frac{4u^3}{u^2+u} du = \int \frac{4u^2}{u+1} du \]
对被积函数进行多项式除法:
\[ \int \left( 4u - 4 + \frac{4}{u+1} \right) du = 2u^2 - 4u + 4 \ln|u+1| + C \]
还原变量 $u = \sqrt[4]{x+1}$,得到最终结果:
\[ I = 2\sqrt{x+1} - 4\sqrt[4]{x+1} + 4 \ln(\sqrt[4]{x+1}+1) + C \]
\end{solution}

\question
43) 计算不定积分:\[ \int \frac{1+x^2}{x\sqrt{x^4-x^2+1}} dx \]

\begin{solution}
分子分母同时除以 $x^2$:
\[ I = \int \frac{1 + \frac{1}{x^2}}{\sqrt{x^2 - 1 + \frac{1}{x^2}}} dx \]
设 $t = x - \frac{1}{x}$,则 $dt = (1 + \frac{1}{x^2}) dx$,且 $x^2 - 1 + \frac{1}{x^2} = t^2 + 1$。
代入积分式:
\[ I = \int \frac{dt}{\sqrt{t^2+1}} = \ln|t + \sqrt{t^2+1}| + C \]
还原变量:
\[ I = \ln\left| x - \frac{1}{x} + \sqrt{x^2 - 1 + \frac{1}{x^2}} \right| + C \]
\end{solution}

\question
44) 计算不定积分:\[ \int \frac{1}{(2x^2+3)\sqrt{5x^2+4}} dx \]

\begin{solution}
将被积函数改写,提取 $x^3$:
\[ I = \int \frac{dx}{x^3(2 + \frac{3}{x^2})\sqrt{5 + \frac{4}{x^2}}} \]
设 $t^2 = 5 + \frac{4}{x^2}$,则 $2t dt = -\frac{8}{x^3} dx$,即 $\frac{1}{x^3} dx = -\frac{1}{4} t dt$。
同时 $\frac{1}{x^2} = \frac{t^2-5}{4}$,代入分母项 $2 + \frac{3}{x^2} = \frac{3t^2-7}{4}$。
代入积分式:
\[ I = -\int \frac{\frac{1}{4} t dt}{\frac{3t^2-7}{4} \cdot t} = -\int \frac{dt}{3t^2-7} = -\frac{1}{3} \int \frac{dt}{t^2 - \frac{7}{3}} \]
进行积分计算:
\[ I = -\frac{\sqrt{21}}{42} \ln\left| \frac{\sqrt{3}t - \sqrt{7}}{\sqrt{3}t + \sqrt{7}} \right| + C \]
还原变量 $t = \sqrt{5 + \frac{4}{x^2}}$。
\end{solution}

\question
45) 计算不定积分:\[ \int \frac{x+x^{\frac{2}{3}}+x^{\frac{1}{6}}}{x(1+x^{\frac{1}{3}})} dx \]

\begin{solution}
使用换元法,设 $x = u^6$,则 $dx = 6u^5 du$。
代入后简化:
\[ I = 6 \int \frac{u^6+u^4+u}{u^6(1+u^2)} \cdot u^5 du = 6 \int \frac{u^6+u^4+u}{1+u^2} du \]
利用多项式除法拆分:
\[ I = 6 \int \left( u^4 + \frac{u}{1+u^2} \right) du = \frac{6}{5} u^5 + 3 \ln(1+u^2) + C \]
还原变量 $u = x^{\frac{1}{6}}$:
\[ I = \frac{6}{5} x^{\frac{5}{6}} + 3 \ln(1+x^{\frac{1}{3}}) + C \]
\end{solution}

\question
47) 计算不定积分:\[ \int \frac{\sqrt[3]{1+\sqrt[4]{x}}}{\sqrt{x}} dx \]

\begin{solution}
设 $1 + x^{\frac{1}{4}} = t^3$,则 $x = (t^3-1)^4,dx = 12t^2(t^3-1)^3 dt$。
代入积分式:
\[ I = \int \frac{t}{(t^3-1)^2} \cdot 12t^2(t^3-1)^3 dt = 12 \int (t^6 - t^3) dt \]
进行积分计算:
\[ I = 12 \left( \frac{t^7}{7} - \frac{t^4}{4} \right) + C = \frac{12}{7} t^7 - 3t^4 + C \]
还原变量 $t = (1+x^{\frac{1}{4}})^{\frac{1}{3}}$。
\end{solution}

\question
48) 计算不定积分:\[ \int \frac{x-1}{(x+1)\sqrt{x^3+x^2+x}} dx \]

\begin{solution}
利用恒等变形:
\[ I = \int \frac{x^2-1}{(x+1)^2 \sqrt{x^3+x^2+x}} dx = \int \frac{1 - \frac{1}{x^2}}{(x + \frac{1}{x} + 2)\sqrt{x + 1 + \frac{1}{x}}} dx \]
设 $u = \sqrt{x + 1 + \frac{1}{x}}$,则 $2u du = (1 - \frac{1}{x^2}) dx$。
代入积分式:
\[ I = \int \frac{2u du}{(u^2+1) \cdot u} = \int \frac{2}{u^2+1} du = 2 \tan^{-1} u + C \]
还原变量:
\[ I = 2 \tan^{-1} \sqrt{x + 1 + \frac{1}{x}} + C \]
\end{solution}

\question
49) 计算不定积分:\[ \int \frac{x}{(x-1)\sqrt{x^2-2x}} dx \]

\begin{solution}
设 $t = x-1$,则 $x = t+1,x^2-2x = t^2-1$。
代入积分式:
\[ I = \int \frac{t+1}{t\sqrt{t^2-1}} dt = \int \frac{1}{\sqrt{t^2-1}} dt + \int \frac{1}{t\sqrt{t^2-1}} dt \]
进行积分计算:
\[ I = \ln|t + \sqrt{t^2-1}| + \sec^{-1} t + C \]
还原变量 $t = x-1$:
\[ I = \ln|x - 1 + \sqrt{x^2-2x}| + \sec^{-1}(x-1) + C \]
\end{solution}

\question
51) 计算不定积分:\[ \int \frac{1}{\sqrt{1-x} + \sqrt{1+x^2}} dx \]

\begin{solution}
对分母进行有理化:
\[ I = \int \frac{\sqrt{1+x^2} - \sqrt{1-x}}{(1+x^2) - (1-x)} dx = \int \frac{\sqrt{1+x^2} - \sqrt{1-x}}{x^2+x} dx \]
拆分为两个积分:
\[ I = \int \frac{\sqrt{1+x^2}}{x(x+1)} dx - \int \frac{\sqrt{1-x}}{x(x+1)} dx \]
对于第二部分,设 $u^2 = 1-x$。
根据笔记底部的最终整理结果:
\[ I = -\sqrt{2} \left( \ln\left|\sqrt{2x^2+2} + x - 1\right| - \ln|x+1| \right) + C \]
\end{solution}
\question
计算 $[\int \frac{3\sin 2x + 4\cos x e^{\sin x} + 8\cos x}{3\sin x + e^{\sin x} + 1} dx]$。

\begin{solution}
首先,利用二倍角公式 $\sin 2x = 2\sin x \cos x$ 展开分子:
\[ = [\int \frac{6\sin x \cos x + 4\cos x e^{\sin x} + 8\cos x}{3\sin x + e^{\sin x} + 1} dx] \]
令 $u = \sin x$,则 $du = \cos x dx$。代入原式得:
\[ = [\int \frac{6u + 4e^u + 8}{3u + e^u + 1} du] \]
提取分子中的常数 2:
\[ = 2 [\int \frac{3u + 2e^u + 4}{3u + e^u + 1} du] \]
将分子拆项,使其包含分母的形式:
\[ = 2 [\int \frac{(3u + e^u + 1) + (3 + e^u)}{3u + e^u + 1} du] \]
\[ = 2 ( [\int \frac{3u + e^u + 1}{3u + e^u + 1} du] + [\int \frac{e^u + 3}{3u + e^u + 1} du] ) \]
\[ = 2 [\int 1 du] + 2 [\int \frac{d(3u + e^u + 1)}{3u + e^u + 1}] \]
进行积分:
\[ = 2u + 2\ln|3u + e^u + 1| + C \]
最后回代 $u = \sin x$:
\[ = 2\sin x + 2\ln|3\sin x + e^{\sin x} + 1| + C \]
\end{solution}
\question
52) 计算不定积分:\[ \int \frac{\sqrt{x+2}}{\sqrt{x+1}} dx \]

\begin{solution}
方法一:设 $t = \sqrt{\frac{x+2}{x+1}}$,则 $x = \frac{2-t^2}{t^2-1}$。
方法二:设 $1+x = \sinh^2 u$,则 $dx = 2 \sinh u \cosh u du$。
代入积分式:
\[ I = \int \frac{\cosh u}{\sinh u} \cdot 2 \sinh u \cosh u du = \int 2 \cosh^2 u du = \int (1 + \cosh 2u) du \]
积分得 $u + \frac{1}{2} \sinh 2u + C = u + \sinh u \cosh u + C$。
还原变量:
\[ I = \sinh^{-1} \sqrt{1+x} + \sqrt{1+x}\sqrt{2+x} + C \]
\end{solution}

\question
54) 计算不定积分:\[ \int \frac{\sqrt{x}}{\sqrt[3]{x}-1} dx \]

\begin{solution}
设 $x = a^6$,则 $dx = 6a^5 da$。
代入积分式:
\[ I = \int \frac{a^3}{a^2-1} \cdot 6a^5 da = 6 \int \frac{a^8}{a^2-1} da \]
利用多项式除法:
\[ \frac{a^8}{a^2-1} = a^6 + a^4 + a^2 + 1 + \frac{1}{a^2-1} \]
逐项积分:
\[ I = 6 \left( \frac{a^7}{7} + \frac{a^5}{5} + \frac{a^3}{3} + a + \frac{1}{2} \ln \left| \frac{a-1}{a+1} \right| \right) + C \]
还原变量 $a = x^{\frac{1}{6}}$:
\[ I = \frac{6}{7}x^{\frac{7}{6}} + \frac{6}{5}x^{\frac{5}{6}} + 2x^{\frac{1}{2}} + 6x^{\frac{1}{6}} + 3 \ln \left| \frac{x^{1/6}-1}{x^{1/6}+1} \right| + C \]
\end{solution}
\question
7) 计算不定积分:\[ \int \frac{(x+1)^3(x-1)}{(x^2+1)^2 \sqrt{x^4+x^2+1}} dx \]

\begin{solution}
首先简化被积函数,分子展开并提取公因子:
\[ \int \frac{(x^2+2x+1)(x^2-1)}{(x^2+1)^2 \sqrt{x^4+x^2+1}} dx \]
分子分母同除以 $x^4$,利用 $x + \frac{1}{x}$ 进行换元:
\[ = \int \frac{(x+\frac{1}{x}+2)(1-\frac{1}{x^2})}{(x+\frac{1}{x})^2 \sqrt{x^2+1+\frac{1}{x^2}}} dx \]
\[ = \int \frac{(x+\frac{1}{x}+2)(1-\frac{1}{x^2})}{(x+\frac{1}{x})^2 \sqrt{(x+\frac{1}{x})^2-1}} dx \]

设 $x + \frac{1}{x} = t$,则 $(1-\frac{1}{x^2}) dx = dt$:
\[ I = \int \frac{t+2}{t^2 \sqrt{t^2-1}} dt = \int \frac{1}{t\sqrt{t^2-1}} dt + 2 \int \frac{1}{t^2\sqrt{t^2-1}} dt \]

对于第一部分,设 $t = \sec\theta$,则积分结果为 $\tan^{-1}\sqrt{t^2-1}$ 或 $\sec^{-1}t$。
对于第二部分,设 $t^2-1 = u^2$,则 $t dt = u du$:
\[ 2 \int \frac{1}{t^2\sqrt{t^2-1}} dt = \dots = \frac{2\sqrt{t^2-1}}{t} + C \]

合并结果并还原变量 $t = x + \frac{1}{x}$:
\[ I = \tan^{-1}\left(\sqrt{(x+\frac{1}{x})^2-1}\right) + \frac{2\sqrt{(x+\frac{1}{x})^2-1}}{x+\frac{1}{x}} + C \]
\end{solution}
    \question 
    \[
    \int_0^{\frac{\pi}{2}} \frac{e^x(\sin x + \cos x - 2)}{(\cos x - 2)^2} \, dx
    \]
    \begin{solution}
        有\[\int_0^{\frac{\pi}{2}} \frac{e^x(\sin x + \cos x - 2)}{(\cos x - 2)^2} \, dx 
        =\int_0^{\frac{\pi}{2}}e^{x}\left(\frac{\sin x}{(\cos x-2)^{2}}+\frac{1}{\cos x-2}\right) \, dx 
        \]
        发现被积函数形如 \(e^{x}(f(x)+f^{\prime }(x))\),其中\[
        f(x)= \frac{1}{\cos x-2}, f^{\prime }(x)=\frac{\sin x}{(\cos x-2)^{2}}
        \]
        所以\[
        \int _{0}^{\frac{\pi }{2}}e^{x}\left(\frac{1}{\cos x-2}+\frac{\sin x}{(\cos x-2)^{2}}\right)dx=\left[\frac{e^{x}}{\cos x-2}\right]_{0}^{\frac{\pi }{2}}=1 - \frac{e^{\frac{\pi}{2}}}{2}
        \]
    \end{solution}
    \question
8) 计算不定积分:\[ \int e^x \left( \frac{1}{\sqrt{x^2+1}} + \frac{1-2x^2}{(x^2+1)^{\frac{5}{2}}} \right) dx \]

\begin{solution}
观察被积函数形式,尝试利用公式 $\int e^x [f(x) + f'(x)] dx = e^x f(x) + C$。
令 $f(x) = \frac{1}{\sqrt{x^2+1}} + \frac{x}{(x^2+1)^{\frac{3}{2}}}$。
计算导数 $f'(x)$:
\[ f'(x) = -\frac{x}{(x^2+1)^{\frac{3}{2}}} + \frac{(x^2+1)^{\frac{3}{2}} - x \cdot \frac{3}{2}(x^2+1)^{\frac{1}{2}} \cdot 2x}{(x^2+1)^3} \]
\[ = -\frac{x}{(x^2+1)^{\frac{3}{2}}} + \frac{1-2x^2}{(x^2+1)^{\frac{5}{2}}} \]
将其代入原式:
\[ I = \int e^x \left( \frac{1}{\sqrt{x^2+1}} - \frac{x}{(x^2+1)^{\frac{3}{2}}} + \frac{x}{(x^2+1)^{\frac{3}{2}}} + \frac{1-2x^2}{(x^2+1)^{\frac{5}{2}}} \right) dx \]
根据恒等式性质得到结果:
\[ I = e^x \left( \frac{1}{\sqrt{x^2+1}} + \frac{x}{(x^2+1)^{\frac{3}{2}}} \right) + C \]
\end{solution}

\question
13) 计算不定积分:\[ \int \sqrt{x + \sqrt{x^2+2}} dx \]

\begin{solution}
使用双曲函数换元,设 $x = \sqrt{2} \sinh t$,则 $dx = \sqrt{2} \cosh t dt$。
同时利用恒等式 $\sqrt{x^2+2} = \sqrt{2} \cosh t$:
\[ I = \int \sqrt{\sqrt{2}\sinh t + \sqrt{2}\cosh t} \cdot \sqrt{2}\cosh t dt \]
\[ = \sqrt[4]{8} \int \sqrt{e^t} \cosh t dt = \sqrt[4]{8} \int e^{\frac{t}{2}} \frac{e^t + e^{-t}}{2} dt \]
\[ = \frac{\sqrt[4]{8}}{2} \int (e^{\frac{3t}{2}} + e^{-\frac{t}{2}}) dt = \frac{\sqrt[4]{8}}{2} \left[ \frac{2}{3}e^{\frac{3t}{2}} - 2e^{-\frac{t}{2}} \right] + C \]
还原变量 $t = \sinh^{-1} \frac{x}{\sqrt{2}}$ 得到最终结果:
\[ I = \frac{1}{\sqrt[4]{2}} \left( \frac{2}{3} e^{\frac{3}{2} \sinh^{-1} \frac{x}{\sqrt{2}}} - 2 e^{-\frac{1}{2} \sinh^{-1} \frac{x}{\sqrt{2}}} \right) + C \]
\end{solution}

%三角函数,万能公式
\question 
38) b) 证明 $\frac{d}{d\theta}(\tan^3 \theta) = 3\tan^4 \theta + 3\sec^2 \theta - 3$。
由此计算 $\int_{0}^{\frac{\pi}{4}} \tan^4 \theta d\theta$。万能公式

\begin{solution}
证明部分:
\begin{align*}
\frac{d}{d\theta} \tan^3 \theta &= 3\tan^2 \theta (\sec^2 \theta) \\
&= 3\tan^2 \theta (\tan^2 \theta + 1) \\
&= 3\tan^4 \theta + 3\tan^2 \theta \\
&= 3\tan^4 \theta + 3(\sec^2 \theta - 1) \\
&= 3\tan^4 \theta + 3\sec^2 \theta - 3
\end{align*}
计算积分部分:
\begin{align*}
\int_{0}^{\frac{\pi}{4}} \tan^4 \theta d\theta &= \frac{1}{3} \int_{0}^{\frac{\pi}{4}} (3\tan^4 \theta) d\theta \\
&= \frac{1}{3} \int_{0}^{\frac{\pi}{4}} \left( \frac{d}{d\theta} \tan^3 \theta - 3\sec^2 \theta + 3 \right) d\theta \\
&= \frac{1}{3} \left[ \tan^3 \theta - 3\tan \theta + 3\theta \right]_{0}^{\frac{\pi}{4}} \\
&= \frac{1}{3} \left[ 1 - 3 + \frac{3\pi}{4} \right] \\
&= \frac{\pi}{4} - \frac{2}{3}
\end{align*}
\end{solution}

\question
39) 证明 $\frac{d}{dx} \tan^3 x = 3\sec^4 x - 3\sec^2 x$。
由此计算 $\int_{0}^{\frac{\pi}{4}} \sec^4 x dx$。

\begin{solution}
证明部分:
\begin{align*}
\frac{d}{dx} \tan^3 x &= 3\tan^2 x \sec^2 x \\
&= 3(\sec^2 x - 1)(\sec^2 x) \\
&= 3\sec^4 x - 3\sec^2 x
\end{align*}
计算积分部分:
\begin{align*}
\int_{0}^{\frac{\pi}{4}} \sec^4 x dx &= \frac{1}{3} \int_{0}^{\frac{\pi}{4}} (3\sec^4 x) dx \\
&= \frac{1}{3} \int_{0}^{\frac{\pi}{4}} \left( \frac{d}{dx} \tan^3 x + 3\sec^2 x \right) dx \\
&= \frac{1}{3} \left[ \tan^3 x + 3\tan x \right]_{0}^{\frac{\pi}{4}} \\
&= \frac{1}{3} (1 + 3) \\
&= \frac{4}{3}
\end{align*}
\end{solution}

\question 
9) 计算不定积分 $\int \sin^3 2x \cos^2 3x \, dx$

\begin{solution}
利用积化和差及倍角公式:
\begin{align*}
\int \sin^3 2x \cos^2 3x \, dx &= \int \left( \frac{3 \sin 2x - \sin 6x}{4} \right) \left( \frac{1 + \cos 6x}{2} \right) dx \\
&= \frac{1}{8} \int (3 \sin 2x + 3 \sin 2x \cos 6x - \sin 6x - \sin 6x \cos 6x) dx \\
&= \frac{1}{8} \int \left( 3 \sin 2x + \frac{3}{2}(\sin 8x - \sin 4x) - \sin 6x - \frac{1}{2} \sin 12x \right) dx \\
&= \frac{1}{8} \left( -\frac{3}{2} \cos 2x - \frac{3}{16} \cos 8x + \frac{3}{8} \cos 4x + \frac{1}{6} \cos 6x + \frac{1}{24} \cos 12x \right) + C \\
&= -\frac{3}{16} \cos 2x + \frac{3}{64} \cos 4x + \frac{1}{48} \cos 6x - \frac{3}{128} \cos 8x + \frac{1}{192} \cos 12x + C
\end{align*}
\end{solution}

\question 
计算定积分 $I = \int_{0}^{\pi/4} \frac{3 - 4\cos 2x + \cos 4x}{3 + 4\cos 2x + \cos 4x} dx$

\begin{solution}
\textbf{第一步:利用倍角公式化简}
注意到 $1 + \cos 4x = 2\cos^2 2x$,分子分母可变形为:
\begin{itemize}
    \item 分子:$2 - 4\cos 2x + 2\cos^2 2x = 2(1 - \cos 2x)^2 = 2(2\sin^2 x)^2 = 8\sin^4 x$
    \item 分母:$2 + 4\cos 2x + 2\cos^2 2x = 2(1 + \cos 2x)^2 = 2(2\cos^2 x)^2 = 8\cos^4 x$
\end{itemize}

\textbf{第二步:化简积分式}
\[ I = \int_{0}^{\pi/4} \frac{8\sin^4 x}{8\cos^4 x} dx = \int_{0}^{\pi/4} \tan^4 x \, dx \]

\textbf{第三步:利用 $\tan^4 x = \tan^2 x(\sec^2 x - 1)$ 积分}
\begin{align*}
I &= \int_{0}^{\pi/4} (\sec^4 x - 2\sec^2 x + 1) dx \quad (\text{或利用之前证明过的结论}) \\
&= \int_{0}^{\pi/4} (\tan^2 x + 1)\sec^2 x \, dx - 2\int_{0}^{\pi/4} \sec^2 x \, dx + \int_{0}^{\pi/4} 1 \, dx \\
&= \int_{0}^{1} (u^2 + 1) du - 2[\tan x]_{0}^{\pi/4} + [x]_{0}^{\pi/4} \\
&= \left[ \frac{u^3}{3} + u \right]_{0}^{1} - 2(1) + \frac{\pi}{4} \\
&= (\frac{1}{3} + 1) - 2 + \frac{\pi}{4} \\
&= \frac{\pi}{4} - \frac{2}{3}
\end{align*}

\textbf{结果核对:} 与截图结果 $\frac{\pi}{4} - \frac{2}{3}$ 一致。
\end{solution}

\question 计算积分
\[
\int_{\frac{\pi}{12}}^{\frac{\pi}{4}}\frac{\sin 3x}{(\cos 7x+\cos x)^{2}+(\sin 7x+\sin x)^{2}} \, dx
\]
\begin{solution}
先展开分母:
\[
(\cos 7x+\cos x)^2 + (\sin 7x+\sin x)^2 = \cos^2 7x + 2\cos 7x \cos x + \cos^2 x + \sin^2 7x + 2\sin 7x \sin x + \sin^2 x
\]

利用 $\cos^2 \theta + \sin^2 \theta = 1$:
\[
= 1 + 1 + 2(\cos 7x \cos x + \sin 7x \sin x) = 2 + 2\cos(7x - x) = 2 + 2\cos 6x
\]

使用二倍角公式化简:
\[
\cos 6x = 2\cos^2 3x - 1 \implies 2 + 2\cos 6x = 4\cos^2 3x
\]

积分变为:
\[
\int_{\frac{\pi}{12}}^{\frac{\pi}{4}} \frac{\sin 3x}{4\cos^2 3x} \, dx = \frac{1}{4} \int_{\frac{\pi}{12}}^{\frac{\pi}{4}} \frac{\sin 3x}{\cos 3x} \cdot \frac{1}{\cos 3x} \, dx = \frac{1}{4} \int_{\frac{\pi}{12}}^{\frac{\pi}{4}} \tan 3x \sec 3x \, dx
\]

作代换 $u = 3x \implies du = 3 dx$:
\[
\frac{1}{4} \int \tan 3x \sec 3x \, dx = \frac{1}{12} \int \sec u \tan u \, du = \frac{1}{12} \sec u + C = \frac{1}{12} \sec 3x
\]

代入上下限:
\[
\left[\frac{1}{12} \sec 3x \right]_{\frac{\pi}{12}}^{\frac{\pi}{4}} = \frac{1}{12} \left( \sec \frac{3\pi}{4} - \sec \frac{\pi}{4} \right) = \frac{1}{12} \left( -\sqrt{2} - 2 \right) \text{(注意 $\sec 3\pi/4 = -\sqrt{2}$)}
\]

整理结果:
\[
\int_{\frac{\pi}{12}}^{\frac{\pi}{4}}\frac{\sin 3x}{(\cos 7x+\cos x)^{2}+(\sin 7x+\sin x)^{2}} \, dx = \frac{2 - \sqrt{2}}{12}
\]
\end{solution}

\question 
计算定积分:$I = 8 \int_{0}^{\pi/4} \left( \frac{\sin^5 \theta}{\sin^2 \theta} - \frac{\cos^5 \theta}{\cos^2 \theta} \right) d\theta$

\begin{solution}
首先简化被积函数:
\begin{align*}
\frac{\sin^5 \theta}{\sin^2 \theta} - \frac{\cos^5 \theta}{\cos^2 \theta} &= \sin^3 \theta - \cos^3 \theta \\
&= (\sin \theta - \cos \theta)(\sin^2 \theta + \sin \theta \cos \theta + \cos^2 \theta) \\
&= (\sin \theta - \cos \theta)(1 + \sin \theta \cos \theta)
\end{align*}

\textbf{另一种路径(根据笔记中的化简步骤):}
利用倍角公式和化简过程:
\begin{align*}
I &= \int_{0}^{\pi/4} 8 (3 + 1 + 2\cos 4\theta) \cos 2\theta \, d\theta \\
&= \int_{0}^{\pi/4} (16\cos 2\theta + 16\cos 4\theta \cos 2\theta) \, d\theta \\
&= \int_{0}^{\pi/4} (16\cos 2\theta + 8\cos 6\theta + 8\cos 2\theta) \, d\theta \\
&= \int_{0}^{\pi/4} (24\cos 2\theta + 8\cos 6\theta) \, d\theta
\end{align*}

\textbf{积分计算:}
\begin{align*}
I &= \left[ 12\sin 2\theta + \frac{4}{3}\sin 6\theta \right]_{0}^{\pi/4} \\
&= \left( 12\sin\frac{\pi}{2} + \frac{4}{3}\sin\frac{3\pi}{2} \right) - 0 \\
&= 12(1) + \frac{4}{3}(-1) = 12 - \frac{4}{3} = \frac{32}{3}
\end{align*}

\textbf{最终结果:} $\frac{32}{3}$ \quad wrong ans
\end{solution}

\question
46) 计算不定积分:\[ \int \cos^5 x \sin 5x dx \]

\begin{solution}
利用积化和差公式逐步降幂:
\begin{align*}
I &= \frac{1}{2} \int (2 \sin 5x \cos x) \cos^4 x dx = \frac{1}{2} \int (\sin 6x + \sin 4x) \cos^4 x dx \\
&= \frac{1}{4} \int (2 \sin 6x \cos x + 2 \sin 4x \cos x) \cos^3 x dx \\
&= \frac{1}{4} \int (\sin 7x + \sin 5x + \sin 5x + \sin 3x) \cos^3 x dx
\end{align*}
继续展开并对每一项正弦函数进行积分,最终结果为:
\[ I = -\frac{1}{320} \cos 10x - \frac{1}{128} \cos 8x - \dots + C \]
\end{solution}

\question
50) 计算不定积分:\[ \int \frac{\sin^4 x}{\sqrt{1-\sin 2x}} dx \]

\begin{solution}
利用恒等式 $1-\sin 2x = (\cos x - \sin x)^2$。
被积函数化为 $\frac{\sin^4 x}{|\cos x - \sin x|}$,其中 $\cos x - \sin x = \sqrt{2} \cos(x + \frac{\pi}{4})$。
设 $t = x + \frac{\pi}{4}$,则 $x = t - \frac{\pi}{4}$。
代入积分式:
\[ I = \frac{1}{\sqrt{2}} \int \frac{\sin^4(t-\frac{\pi}{4})}{\cos t} dt = \frac{1}{4\sqrt{2}} \int \frac{(\sin t - \cos t)^4}{\cos t} dt \]
通过展开分子并逐项积分:
\[ I = \frac{1}{4\sqrt{2}} \left( \ln|\sec t + \tan t| + \frac{4}{3} \sin^3 t \cos t - 4\sin t \right) + C \]
其中 $t = x + \frac{\pi}{4}$。
\end{solution}
\question
\[
\int_{\frac{\pi}{6}}^{\frac{\pi}{3}} \frac{\cot^3 x}{\csc x} \,dx
\]
\begin{solution}
令 $u=\sin x$,则
\[
du=\cos x \, dx
\]
当 $x=\frac{\pi}{6}$ 时,$u=\frac{1}{2}$  
当 $x=\frac{\pi}{3}$ 时,$u=\frac{\sqrt{3}}{2}$
\[
\int_{\frac{\pi}{6}}^{\frac{\pi}{3}} \frac{\cos^3 x}{\sin^2 x} \, dx
= \int_{\frac{1}{2}}^{\frac{\sqrt{3}}{2}} \frac{\cos^2 x}{u^2} \, du
\]
\[
= \int_{\frac{1}{2}}^{\frac{\sqrt{3}}{2}} \frac{1-\sin^2 x}{u^2} \, du
= \int_{\frac{1}{2}}^{\frac{\sqrt{3}}{2}} \frac{1-u^2}{u^2} \, du
\]
\[
= \int_{\frac{1}{2}}^{\frac{\sqrt{3}}{2}} \left(\frac{1}{u^2}-1\right) \, du
= \left[-\frac{1}{u}-u\right]_{\frac{1}{2}}^{\frac{\sqrt{3}}{2}}
\]
\[
= \left[\frac{1}{u}+u\right]_{\frac{\sqrt{3}}{2}}^{\frac{1}{2}}
= \left[\frac{1+u^2}{u}\right]_{\frac{\sqrt{3}}{2}}^{\frac{1}{2}}
\]
\[
= \frac{1+\frac{1}{4}}{\frac{1}{2}}-\frac{1+\frac{3}{4}}{\frac{\sqrt{3}}{2}}
= \frac{5}{2}-\frac{7}{2\sqrt{3}}
\]
\[
= \frac{5}{2}-\frac{7\sqrt{3}}{6}
= \frac{1}{6}(15-7\sqrt{3})
\]
\end{solution}
\question
21) 计算不定积分:\[ \int \frac{\sqrt{\cos 2x}}{\sin x} dx \]

\begin{solution}
利用三角恒等式 $\cos 2x = 2\cos^2 x - 1$:
\[ \int \frac{\sqrt{2\cos^2 x - 1}}{\sin^2 x} \sin x dx \]
设 $t = \cos x$,则 $dt = -\sin x dx$。
代入后积分变为:
\[ I = \int \frac{\sqrt{2t^2 - 1}}{t^2 - 1} dt = \int \left( \frac{2(t^2-1) + 1}{(t^2-1)\sqrt{2t^2-1}} \right) dt \]
拆分为两个部分:
\[ I = \int \frac{2}{\sqrt{2t^2-1}} dt + \int \frac{1}{(t^2-1)\sqrt{2t^2-1}} dt \]
计算并整理得:
\[ I = \sqrt{2} \ln(\sqrt{2}t + \sqrt{2t^2-1}) + \frac{1}{2} \ln\left| \frac{\sqrt{2t^2-1} - t}{\sqrt{2t^2-1} + t} \right| + C \]
还原变量 $t = \cos x$:
\[ I = \sqrt{2} \ln(\sqrt{2}\cos x + \sqrt{\cos 2x}) + \frac{1}{2} \ln\left| \frac{\sqrt{\cos 2x} - \cos x}{\sqrt{\cos 2x} + \cos x} \right| + C \]
\end{solution}

\question
56) 计算不定积分:\[ \int x^2 e^{\sin^{-1} x} dx \]

\begin{solution}
采用换元法,设 $x = \sin t$,则 $dx = \cos t dt,t = \sin^{-1} x$。
代入积分式:
\[ I = \int \sin^2 t e^t \cos t dt \]
利用三角恒等式 $\sin^2 t \cos t = \frac{1}{4} (\cos t - \cos 3t)$:
\[ I = \frac{1}{4} \left( \int e^t \cos t dt - \int e^t \cos 3t dt \right) \]
利用公式 $\int e^{at} \cos bt dt = \frac{e^{at}}{a^2+b^2} (a \cos bt + b \sin bt)$:
\[ I = \frac{1}{4} \left[ \frac{e^t}{2} (\cos t + \sin t) - \frac{e^t}{10} (\cos 3t + 3 \sin 3t) \right] + C \]
还原变量 $\sin t = x, \cos t = \sqrt{1-x^2}$:
利用 $\cos 3t = \sqrt{1-x^2}(1-4x^2)$ 和 $\sin 3t = 3x-4x^3$ 进行替换,得到最终结果。
\end{solution}
\question
5) 计算不定积分:\[ \int \frac{\sec^2 x}{(\sec x + \tan x)^{\frac{9}{2}}} dx \]

\begin{solution}
设 $t = \sec x + \tan x$。
对两边求导:
\[ dt = (\sec x \tan x + \sec^2 x) dx = \sec x (\tan x + \sec x) dx = t \sec x dx \]
由此得 $dx = \frac{dt}{t \sec x}$。

利用恒等式 $\sec^2 x - \tan^2 x = 1$:
\[ (\sec x + \tan x)(\sec x - \tan x) = 1 \implies \sec x - \tan x = \frac{1}{t} \]
将 $\sec x + \tan x = t$ 与 $\sec x - \tan x = \frac{1}{t}$ 相加得:
\[ 2 \sec x = t + \frac{1}{t} \implies \sec x = \frac{1}{2}(t + \frac{1}{t}) \]

代入原积分式:
\begin{align*}
I &= \int \frac{\sec x \cdot \sec x}{t^{\frac{9}{2}}} \cdot \frac{dt}{t \sec x} \\
&= \int \frac{\sec x}{t^{\frac{11}{2}}} dt \\
&= \int \frac{1}{2}(t + t^{-1}) t^{-\frac{11}{2}} dt \\
&= \frac{1}{2} \int (t^{-\frac{9}{2}} + t^{-\frac{13}{2}}) dt
\end{align*}

进行积分计算:
\[ I = \frac{1}{2} \left[ \frac{t^{-\frac{7}{2}}}{-\frac{7}{2}} + \frac{t^{-\frac{11}{2}}}{-\frac{11}{2}} \right] + C \]
\[ = -\frac{1}{7} t^{-\frac{7}{2}} - \frac{1}{11} t^{-\frac{11}{2}} + C \]

还原变量 $t = \sec x + \tan x$:
\[ I = -\frac{1}{7(\sec x + \tan x)^{\frac{7}{2}}} - \frac{1}{11(\sec x + \tan x)^{\frac{11}{2}}} + C \]
\end{solution}

\question 已知代换求积分
\[
\int_{0}^{\frac{\pi}{4}} \frac{4\tan x}{1+\sin^2 x} \, dx
\]
\begin{solution}
作代换
\[
u = 1 + \sin^2 x
\]
\[
\frac{du}{dx} = 2\sin x \cos x
\]
\[
dx = \frac{du}{2\sin x \cos x}
\]

积分的上下限变换
\[
x = 0 \implies u = 1
\]
\[
x = \frac{\pi}{4} \implies u = \frac{3}{2}
\]

代入积分
\[
\int_{0}^{\frac{\pi}{4}} \frac{4\tan x}{1+\sin^2 x} \, dx
= \int_{1}^{\frac{3}{2}} \frac{4\tan x}{u} \cdot \frac{du}{2\sin x \cos x}
= \int_{1}^{\frac{3}{2}} \frac{2}{u(2-u)} \, du
\]

部分分式分解
\[
\frac{2}{u(2-u)} = \frac{1}{u} + \frac{1}{2-u}
\]

计算积分
\[
\int_{1}^{\frac{3}{2}} \left( \frac{1}{u} + \frac{1}{2-u} \right) du
= [\ln|u| - \ln|2-u|]_{1}^{\frac{3}{2}}
= \ln \frac{3/2}{2-3/2} - \ln \frac{1}{2-1}
= \ln 3
\]
\end{solution}

\question 代换$u = 1 - \tan^2 x$,求定积分
\[
\int_{0}^{\pi/6} \tan x \sec 2x \, dx
\]
\begin{solution}
作代换
\[
u = 1 - \tan^2 x
\]
\[
\frac{du}{dx} = -2\tan x \sec^2 x
\]
\[
dx = -\frac{du}{2\tan x \sec^2 x}
\]

积分上下限变换
\[
x = 0 \implies u = 1
\]
\[
x = \frac{\pi}{6} \implies u = \frac{2}{3}
\]

代入积分
\[
\int_{0}^{\pi/6} \tan x \sec 2x \, dx
= \int_{1}^{2/3} \tan x \sec 2x \cdot \left(-\frac{du}{2\tan x \sec^2 x}\right)
= \frac{1}{2} \int_{2/3}^{1} \frac{1}{u} \, du
\]

计算积分
\[
\frac{1}{2} \int_{2/3}^{1} \frac{1}{u} \, du
= \left[ \frac{1}{2} \ln|u| \right]_{2/3}^{1}
= \frac{1}{2} \ln 1 - \frac{1}{2} \ln \frac{2}{3}
= \frac{1}{2} \ln \frac{3}{2}
\]
\end{solution}



\question
\[
\int_{\frac{\pi}{6}}^{\frac{\pi}{3}} \frac{1}{\sin x \sin 2x} dx
= \frac{1}{2}\int_{\frac{\pi}{6}}^{\frac{\pi}{3}} \frac{\cos^2 x + \sin^2 x}{2\sin^2 x \cos x} dx
\]

\begin{solution}
\[
= \frac{1}{2}\int_{\frac{\pi}{6}}^{\frac{\pi}{3}} \left( \frac{\cos x}{\sin x} + \sec x \right) dx
= \frac{1}{2}\int_{\frac{\pi}{6}}^{\frac{\pi}{3}} \left( \cot x \csc x + \sec x \right) dx
\]

\[
= \frac{1}{2}\left[ -\csc x + \ln|\sec x + \tan x| \right]_{\frac{\pi}{6}}^{\frac{\pi}{3}}
\]

\[
= \frac{1}{2}\left[\left(\ln(2+\sqrt{3}) - \frac{2}{\sqrt{3}}\right) - \left(-2 + \ln\left(\frac{2}{\sqrt{3}} + \frac{1}{\sqrt{3}}\right)\right)\right]
\]

\[
= \frac{1}{2}\left[\ln(2+\sqrt{3}) - \frac{2}{\sqrt{3}} + 2 - \ln\sqrt{3}\right]
\]

\[
= \frac{1}{2}\ln\left(\frac{2+\sqrt{3}}{\sqrt{3}}\right) + 1 - \frac{\sqrt{3}}{3}
\]
\end{solution}

\question
计算积分
\[
\int \frac{\sin^{8}x - \cos^{8}x}{1 - \frac{1}{2} \sin^{2} 2x} \, dx
\]

\begin{solution}
从分子平方差和分母的正弦二倍角公式出发:
\[
\int \frac{(\sin^{4}x - \cos^{4}x)(\sin^{4}x + \cos^{4}x)}{1 - \frac{1}{2} (2\sin x \cos x)^2} \, dx
= \int \frac{(\sin^{4}x - \cos^{4}x)(\sin^{4}x + \cos^{4}x)}{1 - 2\sin^{2}x \cos^{2}x} \, dx
\]

将分母写成平方差形式:
\[
\int \frac{(\sin^{4}x - \cos^{4}x)(\sin^{4}x + \cos^{4}x)}{1^2 - 2\sin^{2}x \cos^{2}x} \, dx
= \int \frac{(\sin^{4}x - \cos^{4}x)(\sin^{4}x + \cos^{4}x)}{(\sin^{2}x + \cos^{2}x)^2 - 2\sin^{2}x \cos^{2}x} \, dx
\]

展开分子平方差和分母括号:
\[
\int \frac{(\sin^{2}x - \cos^{2}x)(\sin^{2}x + \cos^{2}x)(\sin^{4}x + \cos^{4}x)}{\sin^{4}x + \cos^{4}x + 2\sin^{2}x \cos^{2}x - 2\sin^{2}x \cos^{2}x} \, dx
\]

化简:
\[
\int \frac{-\cos 2x (\sin^{4}x + \cos^{4}x)}{\cos^{4}x + \sin^{4}x} \, dx
= \int -\cos 2x \, dx
\]

积分结果:
\[
-\frac{1}{2} \sin 2x + C
\]
\end{solution}

\question 计算积分
\[
\int \frac{3 \sin^2 x \cos^2 x}{(\cos^3 x - \sin^3 x)^2} \, dx
\]
\begin{solution}
重写被积函数,提取 \(\tan x\) 和 \(\sec x\):
\[
\int \frac{3 \sin^2 x \cos^2 x}{(\cos^3 x - \sin^3 x)^2} \, dx 
= \int 3 \frac{(\sin x \cos x)^2}{(\cos^3 x - \sin^3 x)^2} \, dx
= \int 3 \tan^2 x \sec^2 x (1 - \tan^3 x)^{-2} \, dx
\]

作代换:
\[
u = \tan x \implies du = \sec^2 x \, dx
\]

代入:
\[
\int 3 \tan^2 x \sec^2 x (1 - \tan^3 x)^{-2} \, dx
= \int 3 u^2 (1 - u^3)^{-2} \, du
\]

识别反链式法则:
\[
d(1-u^3) = -3 u^2 du \implies u^2 du = -\frac{1}{3} d(1-u^3)
\]

因此:
\[
\int 3 u^2 (1 - u^3)^{-2} \, du = \int 3 \cdot \left(-\frac{1}{3} \right) (1-u^3)^{-2} d(1-u^3) = -\int (1-u^3)^{-2} d(1-u^3)
\]

积分:
\[
-\int (1-u^3)^{-2} d(1-u^3) = (1-u^3)^{-1} + C
\]

回代 \(u = \tan x\):
\[
\int \frac{3 \sin^2 x \cos^2 x}{(\cos^3 x - \sin^3 x)^2} \, dx = \frac{1}{1 - \tan^3 x} + C
\]

\end{solution}

\question 计算积分
\[
\int_{\frac{\pi}{8}}^{\frac{\pi}{4}} \frac{\sin^6 x + \cos^6 x}{\sin^2 x \cos^2 x} \, dx
\]
\begin{solution}
先拆分分数:
\[
\int_{\frac{\pi}{8}}^{\frac{\pi}{4}} \frac{\sin^6 x}{\sin^2 x \cos^2 x} + \frac{\cos^6 x}{\sin^2 x \cos^2 x} \, dx
= \int_{\frac{\pi}{8}}^{\frac{\pi}{4}} \frac{\sin^4 x}{\cos^2 x} + \frac{\cos^4 x}{\sin^2 x} \, dx
\]

将 \(\sin^2 x = 1 - \cos^2 x\),\(\cos^2 x = 1 - \sin^2 x\) 展开:
\[
\int_{\frac{\pi}{8}}^{\frac{\pi}{4}} \frac{(1-\cos^2 x)^2}{\cos^2 x} + \frac{(1-\sin^2 x)^2}{\sin^2 x} \, dx
= \int_{\frac{\pi}{8}}^{\frac{\pi}{4}} \sec^2 x - 2 + \cos^2 x + \csc^2 x - 2 + \sin^2 x \, dx
\]

合并同类项:
\[
\int_{\frac{\pi}{8}}^{\frac{\pi}{4}} \sec^2 x + \csc^2 x + (\cos^2 x + \sin^2 x) - 4 \, dx
= \int_{\frac{\pi}{8}}^{\frac{\pi}{4}} \sec^2 x + \csc^2 x - 3 \, dx
\]

直接积分:
\[
[\tan x - \cot x - 3x]_{\frac{\pi}{8}}^{\frac{\pi}{4}}
\]

化简为倍角形式:
\[
[\tan x - \cot x - 3x] = [3x + 2\cot 2x]_{\frac{\pi}{8}}^{\frac{\pi}{4}} = \frac{1}{8}(16 - 3\pi)
\]

\end{solution}

    \question 求积分
    \[
    \int_{0}^{\pi/2} \frac{\sin 2x}{\sqrt{4 - \sin^4 x}} \, dx
    \]
\begin{solution}
令 \(u = \sin^2 x \implies du = 2\sin x \cos x \, dx = \sin 2x \, dx\):
\[
\int_{0}^{\pi/2} \frac{\sin 2x}{\sqrt{4 - \sin^4 x}} \, dx = \int_{0}^{1} \frac{1}{\sqrt{4 - u^2}} \, du
\]

\[
= \left[\arcsin \frac{u}{2}\right]_{0}^{1} = \arcsin \frac{1}{2} - \arcsin 0 = \frac{\pi}{6}
\]
\end{solution}
\question 计算积分
\[
\int_{0}^{\frac{\pi}{2}} \frac{1+\cos x + \sin x - \tan x}{1+\tan x} \, dx
\]
\begin{solution}
先将 $\tan x$ 用正弦余弦表示:
\[
\int_{0}^{\frac{\pi}{2}} \frac{1+\cos x + \sin x - \frac{\sin x}{\cos x}}{1 + \frac{\sin x}{\cos x}} \, dx
= \int_{0}^{\frac{\pi}{2}} \frac{\cos x + \cos^2 x + \sin x \cos x - \sin x}{\cos x + \sin x} \, dx
\]
这里是通过分子分母同乘 $\cos x$ 得到的。

重组分子:
\[
\int_{0}^{\frac{\pi}{2}} \frac{\cos x - \sin x + \cos x \sin x + \cos^2 x}{\cos x + \sin x} \, dx
= \int_{0}^{\frac{\pi}{2}} \frac{\cos x - \sin x}{\cos x + \sin x} + \frac{\cos x \sin x + \cos^2 x}{\cos x + \sin x} \, dx
\]

提取公因子:
\[
= \int_{0}^{\frac{\pi}{2}} \frac{\cos x - \sin x}{\cos x + \sin x} + \frac{\cos x (\sin x + \cos x)}{\cos x + \sin x} \, dx
\]

使用 $\int \frac{f'(x)}{f(x)} \, dx = \ln|f(x)| + C$ 的形式:
\[
\int_{0}^{\frac{\pi}{2}} \frac{\cos x - \sin x}{\cos x + \sin x} + \cos x \, dx
= \left[ \ln|\cos x + \sin x| + \sin x \right]_{0}^{\frac{\pi}{2}}
\]

代入上下限:
\[
\left[ \ln(0+1) + 1 \right] - \left[ \ln(1+0) + 0 \right] = 1
\]

所以积分结果为
\[
\int_{0}^{\frac{\pi}{2}} \frac{1+\cos x + \sin x - \tan x}{1+\tan x} \, dx = 1
\]
\end{solution}


    \question $$\int \frac{\sin x \cos x}{\sin x + \cos x}\, dx $$
    \begin{solution}
        由$(\sin x+\cos x)^2=1+2\sin x \cos x, $
        \begin{align*}
            \int \frac{\sin x \cos x}{\sin x + \cos x}\, dx 
            &= \frac{1}{2}\int \frac{(\sin x+\cos x)^{2}-1}{\sin x+\cos x}\,dx \\
            & =\frac{1}{2}\int \left(\sin x+\cos x-\frac{1}{\sin x+\cos x}\right)dx
        \end{align*}
        现考虑积分
        \[
        I=\int \frac{1}{\sin x+\cos x}\,dx
        \]
        由于$ \sin x+\cos x=\sqrt{2}\sin \left(x+\dfrac{\pi }{4}\right),$变为
        \begin{align*}
            I&= \int \frac{1}{\sqrt{2}\sin (x+\frac{\pi }{4})}\,dx \\
            &= \frac{1}{\sqrt{2}}\int \csc \left(x+\frac{\pi }{4}\right)\,dx \\
            &= -\frac{1}{\sqrt{2}} \ln \left| \csc \left(x + \frac{\pi}{4}\right) + \cot \left(x + \frac{\pi}{4}\right)\right| + C_1
        \end{align*}
        代入原式可得
        $$\int \frac{\sin x \cos x}{\sin x + \cos x} \,dx = \frac{1}{2}(\sin x - \cos x) + \frac{1}{2\sqrt{2}} \ln \left| \csc \left(x + \frac{\pi}{4}\right) + \cot \left(x + \frac{\pi}{4}\right) \right| + C$$
    \end{solution}
    \begin{solution}
        由$\sin x+\cos x=\sqrt2 \sin\left(x+\dfrac{\pi}{4}\right),$
        \[
        \int \frac{\sin x \cos x}{\sin x + \cos x} \, dx 
        = \frac{1}{\sqrt{2}} \int \frac{\sin x \cos x}{\sin\left(x + \frac{\pi}{4}\right)} \, dx         
        \]
        令 $u = x + \dfrac{\pi}{4},\; dx = du,$ 则
        \[
        \sin x = \frac{\sin u - \cos u}{\sqrt{2}}, \;
        \cos x = \frac{\sin u + \cos u}{\sqrt{2}}
        \]
        \begin{align*}
        \therefore\quad 
        \int \frac{\sin x \cos x}{\sin x + \cos x} \, dx 
        &= \frac{1}{\sqrt{2}} \int \frac{\frac{1}{2}(\sin u - \cos u)(\sin u + \cos u)}{\sin u} \, du \\
        &= \frac{1}{2\sqrt{2}} \int \frac{2\sin^2 u - 1}{\sin u} \, du \\ 
        &= \frac{1}{\sqrt{2}} \int \sin u \, du- \frac{1}{2\sqrt{2}} \int \csc u \, du \\ 
        &=-\frac{\cos(x+\frac{\pi}{4})}{\sqrt{2}}
        +\frac{1}{2\sqrt{2}} \ln \left| \csc \left(x + \frac{\pi}{4}\right) + \cot \left(x + \frac{\pi}{4}\right) \right|+C 
        \end{align*}
    \end{solution}

    \question 求积分
\begin{solution}
考虑积分
\[
\int_{0}^{\pi/4} \frac{10}{2 - \tan x} \, dx
\]

令 \(u = \tan x \implies dx = \frac{du}{1+u^2}\),积分变为
\[
\int_{0}^{1} \frac{10}{(2-u)(1+u^2)} \, du
\]

使用部分分式分解:
\[
\frac{10}{(2-u)(1+u^2)} = \frac{A+Bu}{1+u^2} + \frac{C}{2-u}
\]

确定系数:
\begin{itemize}
    \item \(u=2 \Rightarrow 10 = 5C \Rightarrow C=2\)
    \item \(u=0 \Rightarrow 10 = 2A + C = 2A+2 \Rightarrow A=4\)
    \item \(u=1 \Rightarrow 10 = (A+B) + 2C = 4+B+4 \Rightarrow B=2\)
\end{itemize}

因此积分可写为
\[
\int_{0}^{1} \frac{4+2u}{u^2+1} + \frac{2}{2-u} \, du = \int_{0}^{1} \frac{4}{u^2+1} + \frac{2u}{u^2+1} + \frac{2}{2-u} \, du
\]

逐项积分:
\[
\int \frac{4}{u^2+1} \, du = 4\arctan u, \quad
\int \frac{2u}{u^2+1} \, du = \ln(u^2+1), \quad
\int \frac{2}{2-u} \, du = -2\ln|2-u|
\]

代入上下限:
\[
\left[4\arctan u + \ln(u^2+1) - 2\ln|2-u|\right]_{0}^{1} = 4 \arctan 1 + \ln 2 - 2\ln 1 - (0 + \ln 1 - 2\ln 2) = \pi + 3\ln 2
\]

最终结果:
\[
\pi + 3\ln 2
\]
\end{solution}

\question
\[
I=\int_{\arcsin \frac{3}{5}}^{\arccos \frac{3}{5}}
\frac{1}{(\sin x+2\cos x)(\sin x+3\cos x)}\,dx
\]

\begin{solution}
先化简被积函数
\[
=\int_{\arcsin \frac{3}{5}}^{\arccos \frac{3}{5}}
\frac{1}{\sin^2 x+5\sin x\cos x+6\cos^2 x}\,dx
\]

上下同除以 $\cos^2 x$
\[
=\int_{\arcsin \frac{3}{5}}^{\arccos \frac{3}{5}}
\frac{\frac{1}{\cos^2 x}}
{\frac{\sin^2 x}{\cos^2 x}
+5\frac{\sin x\cos x}{\cos^2 x}
+6\frac{\cos^2 x}{\cos^2 x}}\,dx
\]

化为正切函数
\[
=\int_{\arcsin \frac{3}{5}}^{\arccos \frac{3}{5}}
\frac{\sec^2 x}{\tan^2 x+5\tan x+6}\,dx
\]

令 $u=\tan x$,则 $du=\sec^2 x\,dx$

当 $x=\arcsin \frac{3}{5}$ 时,
\[
u=\frac{3}{4}
\]

当 $x=\arccos \frac{3}{5}$ 时,
\[
u=\frac{4}{3}
\]

积分化为
\[
I=\int_{\frac{3}{4}}^{\frac{4}{3}}
\frac{1}{u^2+5u+6}\,du
\]

因式分解
\[
=\int_{\frac{3}{4}}^{\frac{4}{3}}
\frac{1}{(u+2)(u+3)}\,du
\]

设
\[
\frac{1}{(u+2)(u+3)}
=\frac{A}{u+2}+\frac{B}{u+3}
\]

比较系数得
\[
1=A(u+3)+B(u+2)
\]

解得
\[
A=1,\qquad B=-1
\]

代回积分
\[
I=\int_{\frac{3}{4}}^{\frac{4}{3}}
\left(\frac{1}{u+2}-\frac{1}{u+3}\right)\,du
\]

计算得
\[
=\left[\ln(u+2)-\ln(u+3)\right]_{\frac{3}{4}}^{\frac{4}{3}}
\]

\[
=\left(\ln\frac{10}{3}-\ln\frac{13}{3}\right)
-\left(\ln\frac{11}{4}-\ln\frac{15}{4}\right)
\]

\[
=\ln\frac{10}{13}-\ln\frac{11}{15}
=\ln\frac{150}{143}
\]
\end{solution}


    \question
求
\[
\int_{0}^{\frac{\pi}{2}} \frac{1}{1+k^2 \tan^2 x} \, dx, \quad |k| \neq 1
\]

\begin{solution}
首先用代换:
\[
u = \tan x, \quad du = \sec^2 x \, dx, \quad dx = \frac{du}{\sec^2 x}, \quad x=0 \to u=0, \ x=\frac{\pi}{2} \to u=\infty
\]

积分变为:
\[
\int_{0}^{\infty} \frac{du}{1+k^2 u^2} \cdot \frac{1}{1+u^2} = \int_{0}^{\infty} \frac{1}{(u^2+1)(u^2+k^2)} \, du
\]

使用部分分式分解:
\[
\frac{k}{(u^2+1)(u^2+k^2)} = \frac{B}{u^2+1} + \frac{D}{u^2+k^2}
\]

由系数比较得到:
\[
B+D = 0 \implies D=-B, \quad Bk^2 + D = k \implies B = \frac{k}{k^2-1}, \ D = \frac{-k}{k^2-1}
\]

因此积分变为:
\[
\int_{0}^{\infty} \frac{k}{(u^2+1)(u^2+k^2)} \, du = \int_{0}^{\infty} B \left( \frac{1}{u^2+1} - \frac{1}{u^2+k^2} \right) \, du
\]

计算得到:
\[
\int_{0}^{\infty} \frac{1}{u^2+1} \, du = \arctan u \Big|_0^{\infty} = \frac{\pi}{2}, \quad
\int_{0}^{\infty} \frac{1}{u^2+k^2} \, du = \frac{1}{k} \arctan \frac{u}{k} \Big|_0^{\infty} = \frac{\pi}{2k}
\]

代入 B:
\[
\int_{0}^{\frac{\pi}{2}} \frac{1}{1+k^2 \tan^2 x} \, dx = B \left( \frac{\pi}{2} - \frac{\pi}{2k} \right) = \frac{k}{k^2-1} \cdot \frac{\pi}{2} \cdot \frac{k-1}{k} = \frac{\pi}{2(k+1)}
\]
\end{solution}

\question
计算积分
\[
\int_{0}^{\pi/3} \frac{6}{\sin x + \sin 2x} \, dx
\]

\begin{solution}
化简被积函数:
\[
\sin x + \sin 2x = \sin x + 2 \sin x \cos x = \sin x (1+2\cos x)
\]

代换:
\[
u = \cos x \implies du = -\sin x \, dx \implies dx = -\frac{du}{\sin x}
\]

积分上下限:
\[
x = 0 \implies u = 1, \quad x = \frac{\pi}{3} \implies u = \frac{1}{2}
\]

积分变为:
\[
\int_{0}^{\pi/3} \frac{6}{\sin x (1+2\cos x)} \, dx = \int_{1}^{1/2} \frac{6}{\sin x (1+2u)} \left(-\frac{du}{\sin x}\right)
= \int_{1/2}^{1} \frac{6}{\sin^2 x (1+2u)} \, du
\]

利用 $\sin^2 x = 1-u^2$:
\[
\int_{1/2}^{1} \frac{6}{(1-u^2)(1+2u)} \, du = \int_{1/2}^{1} \frac{6}{(1-u)(1+u)(1+2u)} \, du
\]

使用部分分式:
\[
\frac{6}{(1-u)(1+u)(1+2u)} = \frac{A}{1-u} + \frac{B}{1+u} + \frac{C}{1+2u}
\]

求系数:
\[
u=1 \implies 6 = 6A \implies A = 1, \quad
u=-1 \implies 6 = -2B \implies B=-3, \quad
u=-\frac{1}{2} \implies 6 = \frac{3}{4}C \implies C = 8
\]

所以:
\[
\frac{6}{(1-u)(1+u)(1+2u)} = \frac{1}{1-u} - \frac{3}{1+u} + \frac{8}{1+2u}
\]

积分:
\[
\int_{1/2}^{1} \left( \frac{1}{1-u} - \frac{3}{1+u} + \frac{8}{1+2u} \right) du
= \left[ \ln|1-u| - 3 \ln|1+u| + 4 \ln|1+2u| \right]_{1/2}^{1}
\]

代入上下限:
\[
= \left( \ln(1-1) - 3\ln(1+1) + 4 \ln(1+2\cdot 1) \right) - \left( \ln(1-1/2) - 3\ln(1+1/2) + 4 \ln(1+2\cdot 1/2) \right)
\]

化简:
\[
= (4\ln 3 - 3 \ln 2 - \ln 0) - (4\ln 2 - 3\ln \frac{3}{2} - \ln \frac{1}{2}) 
= 8 \ln 2 - 3 \ln 3
\]
\end{solution}

\question 计算积分
\[
\int \frac{\cos^3 x}{(1+\sin x)\sin x} \, dx
\]
\begin{solution}
使用代换
\[
u = \sin x + \csc x \implies \frac{du}{dx} = \cos x - \cot x \csc x \implies dx = \frac{1}{\cos x - \cot x \csc x} \, du
\]

代入积分:
\[
\int \frac{\cos^3 x}{(1+\sin x)\sin x} \, dx 
= \int \frac{\cos^3 x}{(1+\sin x)\sin x} \cdot \frac{1}{\cos x - \cot x \csc x} \, du
\]

将 $\cot x$ 和 $\csc x$ 用正弦余弦表示:
\[
= \int \frac{\cos^3 x}{(1+\sin x)\sin x} \cdot \frac{1}{\cos x - \frac{\cos x}{\sin x}\cdot \frac{1}{\sin x}} \, du
= \int \frac{\cos^3 x}{(1+\sin x)\sin x} \cdot \frac{1}{\cos x (1 - \frac{1}{\sin^2 x})} \, du
\]

化简:
\[
= \int \frac{\cos^3 x}{(1+\sin x)\sin x} \cdot \frac{\sin^2 x}{\cos x (\sin^2 x-1)} \, du
= \int \frac{\cos^2 x}{1+\sin x} \cdot \frac{\sin x}{\sin^2 x-1} \, du
= \int \frac{\cos^2 x}{1+\sin x} \cdot \frac{\sin x}{-\cos^2 x} \, du
\]

进一步化简:
\[
= \int -\frac{\sin x}{1+\sin x} \, du
= \int -\frac{1}{\csc x + \sin x} \, du
= \int -\frac{1}{u} \, du
= -\ln|u| + C
= \ln\left|\frac{1}{u}\right| + C
\]

代回原变量:
\[
= \ln\left|\frac{1}{\sin x + \csc x}\right| + C
= \ln\left|\frac{1}{\sin x + \frac{1}{\sin x}}\right| + C
= \ln\left|\frac{1}{\frac{\sin^2 x+1}{\sin x}}\right| + C
= \ln\left|\frac{\sin x}{\sin^2 x+1}\right| + C
\]
\end{solution}

\question 求积分
\[
\int_{0}^{\pi/4} \frac{1}{\cos^2 x + 25\sin^2 x} \, dx
\]
\begin{solution}
使用代换 \(u = \tan x\),则
\[
du = \sec^2 x \, dx \implies dx = \frac{du}{\sec^2 x} = \frac{du}{1+u^2}
\]

积分上下限:
\[
x=0 \implies u=0, \quad x=\frac{\pi}{4} \implies u=1
\]

原积分变为:
\[
\int_{0}^{\pi/4} \frac{1}{\cos^2 x + 25\sin^2 x} \, dx
= \int_{0}^{1} \frac{1}{1 + 25u^2} \, du
= \frac{1}{25} \int_{0}^{1} \frac{du}{(\frac{1}{5})^2 + u^2}
\]

标准反正切积分:
\[
\frac{1}{25} \int_{0}^{1} \frac{du}{(\frac{1}{5})^2 + u^2} = \frac{1}{25} \cdot 5 \left[ \arctan(5u) \right]_{0}^{1} = \frac{1}{5} \arctan 5
\]
\end{solution}

\question
使用给定代换计算积分:
\[
\int 2 \frac{1-x^2}{1+x^2} \, dx
\]

\begin{solution}
代换:
\[
x = \tan\frac{\theta}{2} \implies dx = \frac{1}{2} \sec^2 \frac{\theta}{2} \, d\theta
\]

观察右边:
\[
\frac{1-x^2}{1+x^2} = \frac{1-\tan^2\frac{\theta}{2}}{1+\tan^2\frac{\theta}{2}} 
= \frac{1-\tan^2\frac{\theta}{2}}{\sec^2\frac{\theta}{2}} 
= \cos^2 \frac{\theta}{2} - \sin^2 \frac{\theta}{2} 
= \cos\theta
\]

因此积分变为:
\[
\int 2 \frac{1-x^2}{1+x^2} \, dx = \int 2 \cos\theta \cdot \frac{1}{2} \sec^2 \frac{\theta}{2} \, d\theta
= \int \cos\theta \sec^2 \frac{\theta}{2} \, d\theta
\]

注意到:
\[
\arctan x = \frac{\theta}{2} \implies \theta = 2 \arctan x
\]

通过观察和代回:
\[
\int 2 \frac{1-x^2}{1+x^2} \, dx = (1+x^2) \arctan x - x + C
\]
\end{solution}

\question 计算
\[
\int_{0}^{\pi/3} \frac{6\sqrt{3}\cos x}{4 + \sin(2x)\tan(x/2)} \, dx
\]

\begin{solution}
首先使用半角替换:
\[
t = \tan\frac{x}{2} \implies dx = \frac{2}{1+t^2} \, dt, \quad x=0 \to t=0, \quad x=\frac{\pi}{3} \to t=\frac{\sqrt{3}}{3}
\]
并用标准半角公式:
\[
\sin x = \frac{2t}{1+t^2}, \quad \cos x = \frac{1-t^2}{1+t^2}
\]

代入积分:
\[
\int_{0}^{\sqrt{3}/3} \frac{6\sqrt{3}\frac{1-t^2}{1+t^2}}{4+2\frac{2t}{1+t^2}\frac{1-t^2}{1+t^2} t} \cdot \frac{2}{1+t^2} \, dt
= \int_{0}^{\sqrt{3}/3} \frac{12\sqrt{3}(1-t^2)}{4[(1+t^2)^2+t^2(1-t^2)]} \, dt
\]

分母化简:
\[
4[(1+t^2)^2+t^2(1-t^2)] = 4(1 + 3t^2)
\]

因此积分化为:
\[
\int_{0}^{\sqrt{3}/3} \frac{12\sqrt{3}(1-t^2)}{4(3t^2+1)} \, dt = \int_{0}^{\sqrt{3}/3} \frac{3\sqrt{3}(1-t^2)}{3t^2+1} \, dt
\]

拆分分子:
\[
\frac{3\sqrt{3}(1-t^2)}{3t^2+1} = \sqrt{3}\frac{3t^2+1 - 4t^2}{3t^2+1} = \sqrt{3}\left(1 - \frac{4t^2}{3t^2+1}\right)
\]

积分:
\[
\sqrt{3}\int_{0}^{\sqrt{3}/3} 1 - \frac{4t^2}{3t^2+1} \, dt = \sqrt{3}\int_{0}^{\sqrt{3}/3} 1 - \frac{4/3}{t^2+1/3} \, dt
= \sqrt{3} \left[ t - \frac{4\sqrt{3}}{3} \arctan(t\sqrt{3}) \right]_0^{\sqrt{3}/3}
\]

代入上下限:
\[
\sqrt{3}\left[ \frac{\sqrt{3}}{3} - \frac{4\sqrt{3}}{3}\arctan(1) - (0-0) \right] = \sqrt{3} \left[ \frac{\sqrt{3}}{3} - \frac{4\sqrt{3}}{3}\cdot \frac{\pi}{4} \right] = 1 - \pi +1
\]

最终结果:
\[
\int_{0}^{\pi/3} \frac{6\sqrt{3}\cos x}{4 + \sin(2x)\tan(x/2)} \, dx = \pi - 1
\]
\end{solution}

\question 求积分
\begin{solution}
使用 Weierstrass 代换 \(t = \tan\frac{x}{2}\),则
\[
dx = \frac{2 \, dt}{1+t^2}, \quad \cos x = \frac{1-t^2}{1+t^2}, \quad \sin x = \frac{2t}{1+t^2}
\]

积分上下限:
\[
x=0 \implies t=0, \quad x=\pi/2 \implies t=1
\]

原积分变为:
\[
\int_{0}^{1} \sqrt{\frac{1-\frac{1-t^2}{1+t^2}}{1+\frac{2t}{1+t^2}}} \cdot \frac{2 \, dt}{1+t^2} 
= \int_{0}^{1} \frac{\sqrt{2} \, t}{(t+1)^2} \cdot \frac{2 \, dt}{1+t^2} 
= \sqrt{2} \int_{0}^{1} \frac{2t}{(t+1)^2(1+t^2)} \, dt
\]

利用部分分式分解:
\[
\frac{2t}{(t+1)^2(1+t^2)} = -\frac{1}{t+1} + \frac{t+1}{t^2+1} = -\frac{1}{t+1} + \frac{t}{t^2+1} + \frac{1}{t^2+1}
\]

积分得到:
\[
\sqrt{2} \int_{0}^{1} \left(-\frac{1}{t+1} + \frac{t}{t^2+1} + \frac{1}{t^2+1}\right) dt
= \sqrt{2} \left[ -\ln(t+1) + \frac{1}{2}\ln(t^2+1) + \arctan t \right]_{0}^{1}
\]

计算边界值:
\[
= \sqrt{2} \left[ -\ln 2 + \frac{1}{2} \ln 2 + \frac{\pi}{4} \right] = \sqrt{2} \left[ \frac{\pi}{4} - \frac{1}{2}\ln 2 \right] = \frac{\sqrt{2}}{4} (\pi - 2\ln 2)
\]
\end{solution}

\question
\[
\int_{0}^{\frac{\pi}{4}} \frac{\sin x+\cos x}{9+16\sin 2x} \, dx
\]

\begin{solution}
令
\[
u=\sin x-\cos x
\]

则
\[
\frac{du}{dx}=\cos x+\sin x
\]
从而
\[
dx=\frac{du}{\cos x+\sin x}
\]

当 $x=0$ 时,
\[
u=-1
\]
当 $x=\frac{\pi}{4}$ 时,
\[
u=0
\]

又有
\[
u^2=(\sin x-\cos x)^2
\]
\[
= \sin^2 x+\cos^2 x-2\sin x\cos x
\]
\[
=1-\sin 2x
\]

因此
\[
\sin 2x=1-u^2
\]
\[
9+16\sin 2x=9+16(1-u^2)=25-16u^2
\]

原积分化为
\[
\int_{-1}^{0} \frac{\sin x+\cos x}{25-16u^2}\frac{du}{\cos x+\sin x}
= \int_{-1}^{0} \frac{1}{25-16u^2} \, du
\]

因式分解得
\[
25-16u^2=(5-4u)(5+4u)
\]

用部分分式法
\[
\int_{-1}^{0} \frac{1}{(5-4u)(5+4u)} \, du
= \int_{-1}^{0} \left(\frac{1}{10}\frac{1}{5+4u}+\frac{1}{10}\frac{1}{5-4u}\right) du
\]

于是
\[
= \frac{1}{10} \int_{-1}^{0} \left(\frac{1}{5+4u}+\frac{1}{5-4u}\right) du
\]

积分得
\[
= \frac{1}{10} \left[\frac{\ln|5+4u|}{4}-\frac{\ln|5-4u|}{4}\right]_{-1}^{0}
\]
\[
= \frac{1}{40} \left[\ln|5+4u|-\ln|5-4u|\right]_{-1}^{0}
\]

代入上下限
\[
= \frac{1}{40}\left[(\ln 5-\ln 5)-(\ln 1-\ln 9)\right]
\]
\[
= \frac{1}{40}\ln 9
= \frac{1}{20}\ln 3
\]
\end{solution}


\question 求积分
\begin{solution}
(a) 代换 \(u = \tan 2x\),则
\[
\frac{du}{dx} = 2\sec^2 2x \implies dx = \frac{du}{2\sec^2 2x}, \quad x=0 \to u=0, \ x=\frac{\pi}{8} \to u=1
\]

原积分变为:
\[
\int_{0}^{\pi/8} \frac{\sqrt{3}}{2+\sin 4x} \, dx = \int_{0}^{1} \frac{\sqrt{3}}{2+2\sin 2x\cos 2x} \cdot \frac{du}{2\sec^2 2x} = \int_{0}^{1} \frac{\sqrt{3}}{4\sec^2 2x + 4\tan 2x} \, du
\]

利用 \(\sec^2 2x = 1 + \tan^2 2x\) 以及 \(u = \tan 2x\):
\[
\int_{0}^{1} \frac{\sqrt{3}}{4u^2 + 4u + 4} \, du = \frac{\sqrt{3}}{4} \int_{0}^{1} \frac{du}{(u+1/2)^2 + 3/4}
\]

(b) 进一步代换 \(V = 2u+1\),则 \(dV = 2 du\),\(u=0 \to V=1\),\(u=1 \to V=3\):
\[
\frac{\sqrt{3}}{4} \int_{0}^{1} \frac{du}{(u+1/2)^2 + 3/4} = \frac{\sqrt{3}}{2} \int_{1}^{3} \frac{dV}{V^2 + (\sqrt{3})^2}
\]

标准反正切积分公式:
\[
\frac{\sqrt{3}}{2} \int_{1}^{3} \frac{dV}{V^2+3} = \frac{\sqrt{3}}{2} \cdot \frac{1}{\sqrt{3}} \left[ \arctan \frac{V}{\sqrt{3}} \right]_{1}^{3} = \frac{1}{2} \left[ \arctan \sqrt{3} - \arctan \frac{\sqrt{3}}{3} \right]
\]

\[
= \frac{1}{2} \left( \frac{\pi}{3} - \frac{\pi}{6} \right) = \frac{\pi}{12}
\]
\end{solution}

\question
36) 计算不定积分:\[ \int \frac{1}{1 + \sin^2 x} dx \]

\begin{solution}
分子分母同时除以 $\cos^2 x$:
\[ I = \int \frac{\sec^2 x}{\sec^2 x + \tan^2 x} dx = \int \frac{\sec^2 x}{1 + 2\tan^2 x} dx \]
设 $u = \sqrt{2} \tan x$,则 $du = \sqrt{2} \sec^2 x dx$。
代入积分式:
\[ I = \frac{1}{\sqrt{2}} \int \frac{1}{u^2+1} du = \frac{1}{\sqrt{2}} \tan^{-1} u + C \]
还原变量:
\[ I = \frac{1}{\sqrt{2}} \tan^{-1}(\sqrt{2} \tan x) + C \]
\end{solution}

\question 求积分
\[
\int_{\pi/6}^{2\pi/9} \frac{1}{2 + \cos 3x} \, dx
\]
\begin{solution}
使用代换 \(t = \tan\frac{3x}{2} \implies dx = \frac{2}{3(1+t^2)} \, dt\):
\[
x = \pi/6 \to t = 1, \quad x = 2\pi/9 \to t = \sqrt{3}
\]

利用三角恒等式 \(\cos 3x = \cos^2 \frac{3x}{2} - \sin^2 \frac{3x}{2} = \frac{1-t^2}{1+t^2}\),积分变为
\[
\int_{1}^{\sqrt{3}} \frac{1}{2 + \frac{1-t^2}{1+t^2}} \cdot \frac{2}{3(1+t^2)} \, dt
= \int_{1}^{\sqrt{3}} \frac{1+t^2}{3+t^2} \cdot \frac{2}{3(1+t^2)} \, dt
= \frac{2}{3} \int_{1}^{\sqrt{3}} \frac{1}{t^2+3} \, dt
\]

这是标准反正切积分:
\[
\frac{2}{3} \int_{1}^{\sqrt{3}} \frac{1}{t^2+(\sqrt{3})^2} \, dt
= \frac{2}{3\sqrt{3}} \left[ \arctan\left(\frac{t}{\sqrt{3}}\right) \right]_{1}^{\sqrt{3}}
= \frac{2}{3\sqrt{3}} \left[ \arctan(1) - \arctan\left(\frac{1}{\sqrt{3}}\right) \right]
\]

计算反正切值:
\[
\arctan(1) = \frac{\pi}{4}, \quad \arctan\left(\frac{1}{\sqrt{3}}\right) = \frac{\pi}{6}
\]

因此积分结果为
\[
\frac{2}{3\sqrt{3}} \cdot \frac{\pi}{12} = \frac{\pi\sqrt{3}}{54}.
\]
\end{solution}

\question
\[
\int \frac{\sec^2 x}{\sqrt{\sec x+\tan x}} \, dx
\]

\begin{solution}
注意到
\[
\frac{d}{dx}(\sec x+\tan x)=\sec^2 x+\sec x\tan x
\]

先将被积式拆分
\[
\int \frac{\sec^2 x}{\sqrt{\sec x+\tan x}} \, dx
= \int \frac{\sec^2 x+\sec x\tan x-\sec x\tan x}{\sqrt{\sec x+\tan x}} \, dx
\]

写成两部分
\[
= \frac{1}{2} \int \frac{\sec^2 x+\sec x\tan x}{\sqrt{\sec x+\tan x}} \, dx
+ \frac{1}{2} \int \frac{\sec^2 x-\sec x\tan x}{\sqrt{\sec x+\tan x}} \, dx
\]

处理第二个积分,利用恒等式
\[
1+\tan^2 x=\sec^2 x
\]

有
\[
\sec^2 x-\sec x\tan x
= \sec x(\sec x-\tan x)
= \sec x(\sec x-\tan x)\frac{\sec x+\tan x}{\sec x+\tan x}
\]

于是
\[
= \frac{1}{2} \int \frac{\sec^2 x+\sec x\tan x}{\sqrt{\sec x+\tan x}} \, dx
+ \frac{1}{2} \int \frac{\sec x(\sec^2 x-\tan^2 x)}{(\sec x+\tan x)^{3/2}} \, dx
\]

而
\[
\sec^2 x-\tan^2 x=1
\]

故
\[
= \frac{1}{2} \int \frac{\sec^2 x+\sec x\tan x}{\sqrt{\sec x+\tan x}} \, dx
+ \frac{1}{2} \int \frac{\sec x}{(\sec x+\tan x)^{3/2}} \, dx
\]

将第二项写成同一结构
\[
= \frac{1}{2} \int (\sec^2 x+\sec x\tan x)(\sec x+\tan x)^{-1/2} \, dx
+ \frac{1}{2} \int (\sec^2 x+\sec x\tan x)(\sec x+\tan x)^{-5/2} \, dx
\]

对两项分别作代换
\[
u=\sec x+\tan x
\]

得到
\[
= \frac{1}{2} \cdot \frac{u^{1/2}}{1/2}
+ \frac{1}{2} \cdot \frac{u^{-3/2}}{-3/2}
+ C
\]

化简并代回
\[
= (\sec x+\tan x)^{1/2}
- \frac{1}{3}(\sec x+\tan x)^{-3/2}
+ C
\]
\end{solution}

\question
计算积分
\[
I=\int_{0}^{\frac{\pi}{4}}4\sin x\sqrt{\cos 2x}\,dx
\]

\begin{solution}
先用三角恒等式
\[
\cos 2x=2\cos^2 x-1
\]
则
\[
I=\int_{0}^{\frac{\pi}{4}}4\sin x\sqrt{2\cos^2 x-1}\,dx
\]

作代换
\[
u=\cos x
\]
则
\[
\frac{du}{dx}=-\sin x,\quad dx=\frac{du}{-\sin x}
\]

当 $x=0$ 时,$u=1$  
当 $x=\frac{\pi}{4}$ 时,$u=\frac{\sqrt{2}}{2}$  

于是
\[
I=\int_{1}^{\frac{\sqrt{2}}{2}}4\sqrt{2u^2-1}(-du)
=\int_{\frac{\sqrt{2}}{2}}^{1}4\sqrt{2u^2-1}\,du
\]

再作三角代换
\[
\sec\theta=\sqrt{2}u
\]
则
\[
du=\frac{\sec\theta\tan\theta}{\sqrt{2}}\,d\theta
\]
且
\[
\sqrt{2u^2-1}=\sqrt{\sec^2\theta-1}=\tan\theta
\]

当 $u=\frac{\sqrt{2}}{2}$ 时,$\theta=0$  
当 $u=1$ 时,$\theta=\frac{\pi}{4}$  

因此
\[
I=\int_{0}^{\frac{\pi}{4}}\frac{4}{\sqrt{2}}\tan\theta(\sec\theta\tan\theta)\,d\theta
=\int_{0}^{\frac{\pi}{4}}\frac{4}{\sqrt{2}}\tan^2\theta\sec\theta\,d\theta
\]

对该积分分部积分,
\[
I=\left[\frac{4}{\sqrt{2}}\tan\theta\sec\theta\right]_{0}^{\frac{\pi}{4}}
-\frac{4}{\sqrt{2}}\int_{0}^{\frac{\pi}{4}}\sec^3\theta\,d\theta
\]

于是
\[
I=4-\frac{4}{\sqrt{2}}\int_{0}^{\frac{\pi}{4}}\sec\theta(1+\tan^2\theta)\,d\theta
\]
\[
=4-\frac{4}{\sqrt{2}}\int_{0}^{\frac{\pi}{4}}\sec\theta\,d\theta
-\frac{4}{\sqrt{2}}\int_{0}^{\frac{\pi}{4}}\sec\theta\tan^2\theta\,d\theta
\]

注意到后一个积分正好等于 $I$,于是
\[
I=4-\frac{4}{\sqrt{2}}\left[\ln(\sec\theta+\tan\theta)\right]_{0}^{\frac{\pi}{4}}-I
\]

从而
\[
2I=4-\frac{4}{\sqrt{2}}\ln(\sqrt{2}+1)
\]

解得
\[
I=2-\sqrt{2}\ln(\sqrt{2}+1)
\]
\end{solution}

\question
设积分
\[
I=\int_{0}^{\pi}\frac{\sin^2 x}{1+\cos^2 x}\,dx
\]

(a) 证明
\[
I+\pi=\int_{0}^{\frac{\pi}{2}}\frac{4}{1+\cos^2 x}\,dx
\]

(b) 从而求 $I$ 的精确化简值  

(c) 用另一种方法验证(b)的结果,先将被积函数写成 $\cot^2 x$ 的函数  

\begin{solution}
(a) 由
\[
I=\int_{0}^{\pi}\frac{\sin^2 x}{1+\cos^2 x}\,dx
\]
有
\[
I=\int_{0}^{\pi}\frac{\sin^2 x+\cos^2 x}{1+\cos^2 x}\,dx
\]
即
\[
I=\int_{0}^{\pi}\frac{1}{1+\cos^2 x}\,dx
\]

又
\[
\int_{0}^{\pi}1\,dx=\pi
\]
于是
\[
I+\pi=\int_{0}^{\pi}\frac{\sin^2 x}{1+\cos^2 x}\,dx+\int_{0}^{\pi}1\,dx
\]
\[
=2\int_{0}^{\pi}\frac{1}{1+\cos^2 x}\,dx
\]

注意到被积函数关于 $\frac{\pi}{2}$ 对称,
\[
I+\pi=4\int_{0}^{\frac{\pi}{2}}\frac{1}{1+\cos^2 x}\,dx
\]
即
\[
I+\pi=\int_{0}^{\frac{\pi}{2}}\frac{4}{1+\cos^2 x}\,dx
\]

(b) 将分子分母同乘 $\sec^2 x$,
\[
I+\pi=\int_{0}^{\frac{\pi}{2}}\frac{4\sec^2 x}{\sec^2 x+1}\,dx
\]
\[
=\int_{0}^{\frac{\pi}{2}}\frac{4\sec^2 x}{2+\tan^2 x}\,dx
\]

写成反正切形式,
\[
I+\pi=\int_{0}^{\frac{\pi}{2}}\frac{4\sec^2 x}{(\sqrt{2})^2+(\tan x)^2}\,dx
\]
\[
=4\left[\frac{1}{\sqrt{2}}\arctan\left(\frac{\tan x}{\sqrt{2}}\right)\right]_{0}^{\frac{\pi}{2}}
\]

因此
\[
I+\pi=2\sqrt{2}\left[\arctan\left(\frac{\tan(\frac{\pi}{2})}{\sqrt{2}}\right)-\arctan\left(\frac{\tan 0}{\sqrt{2}}\right)\right]
\]
\[
=2\sqrt{2}\cdot\frac{\pi}{2}=\pi\sqrt{2}
\]

故
\[
I=\pi\sqrt{2}-\pi=\pi(\sqrt{2}-1)
\]

(c) 将被积函数化为 $\cot^2 x$ 的形式,
\[
\int_{0}^{\pi}\frac{\sin^2 x}{1+\cos^2 x}\,dx
=\int_{0}^{\pi}\frac{1}{\csc^2 x+\cot^2 x}\,dx
=\int_{0}^{\pi}\frac{1}{1+2\cot^2 x}\,dx
\]

令
\[
u=\cot x
\]
则
\[
du=-\csc^2 x\,dx
\]
\[
dx=-\frac{du}{1+u^2}
\]

当 $x=0$ 时,$u=+\infty$;当 $x=\pi$ 时,$u=-\infty$

积分化为
\[
\int_{+\infty}^{-\infty}\frac{1}{1+2u^2}\left(-\frac{du}{1+u^2}\right)
=\int_{-\infty}^{\infty}\frac{1}{(1+2u^2)(1+u^2)}\,du
\]

作部分分式分解,
\[
=\int_{-\infty}^{\infty}\left(\frac{2}{1+2u^2}-\frac{1}{1+u^2}\right)du
\]

被积函数为偶函数,
\[
=2\int_{0}^{\infty}\left(\frac{1}{u^2+\left(\frac{1}{\sqrt{2}}\right)^2}-\frac{1}{1+u^2}\right)du
\]

积分得
\[
=2\left[\sqrt{2}\arctan(\sqrt{2}u)-\arctan u\right]_{0}^{\infty}
\]
\[
=2\left(\sqrt{2}\cdot\frac{\pi}{2}-\frac{\pi}{2}\right)
=\pi\sqrt{2}-\pi
\]

即
\[
I=\pi(\sqrt{2}-1)
\]
与(b)结果一致
\end{solution}

\question 证明三倍角公式并计算积分
\[
\int_{0}^{2-\sqrt{3}} \frac{6x(3-x^2)}{1-2x^2-3x^4} \, dx
\]
\begin{solution}
(a) 从 \(\tan 3\theta = \tan(2\theta + \theta)\) 开始:
\[
\tan 3\theta = \frac{\tan 2\theta + \tan \theta}{1 - \tan 2\theta \tan \theta} 
= \frac{\frac{2 \tan \theta}{1 - \tan^2 \theta} + \tan \theta}{1 - \frac{2 \tan \theta}{1 - \tan^2 \theta} \tan \theta}
\]

将分子分母同时乘以 \(1 - \tan^2 \theta\):
\[
\tan 3\theta = \frac{2 \tan \theta + \tan \theta (1 - \tan^2 \theta)}{1 - \tan^2 \theta - 2\tan^2 \theta} 
= \frac{3 \tan \theta - \tan^3 \theta}{1 - 3 \tan^2 \theta}
\]

(b) 考虑积分:
\[
\int_{0}^{2-\sqrt{3}} \frac{6x(3-x^2)}{1-2x^2-3x^4} \, dx
= \int_{0}^{2-\sqrt{3}} \frac{6x(3-x^2)}{-(3x^4+2x^2-1)} \, dx
= \int_{0}^{2-\sqrt{3}} \frac{6x(3-x^2)}{(1-3x^2)(x^2+1)} \, dx
\]

令 \(x = \tan \theta \implies dx = \sec^2 \theta \, d\theta, \quad x=0 \to \theta=0, \quad x=2-\sqrt{3} \to \theta=\frac{\pi}{12}\):
\[
\int_{0}^{2-\sqrt{3}} \frac{6x(3-x^2)}{(1-3x^2)(x^2+1)} \, dx
= \int_{0}^{\pi/12} \frac{6 \tan 3\theta}{\tan^2 \theta + 1} \sec^2 \theta \, d\theta
= \int_{0}^{\pi/12} 6 \tan 3\theta \, d\theta
\]

积分结果:
\[
\int 6 \tan 3\theta \, d\theta = 2 \ln |\sec 3\theta|
\]

代入上下限:
\[
[2 \ln |\sec 3\theta|]_{0}^{\pi/12} = 2 \ln \sec(\pi/4) - 2 \ln \sec 0 = 2 \ln \sqrt{2} - 0 = \ln 2
\]

因此:
\[
\int_{0}^{2-\sqrt{3}} \frac{6x(3-x^2)}{1-2x^2-3x^4} \, dx = \ln 2
\]
\end{solution}

\question 求积分
\[
\int \sec x \, dx
\]
\begin{solution}
(a) 验证恒等式:
\[
\sec x = \frac{1+\tan^2\left(\frac{x}{2}\right)}{1-\tan^2\left(\frac{x}{2}\right)}
\]

由半角公式:
\[
\sin x = \frac{2\tan\frac{x}{2}}{1+\tan^2\frac{x}{2}}, \quad 
\cos x = \frac{1-\tan^2\frac{x}{2}}{1+\tan^2\frac{x}{2}}
\]

则
\[
\sec x = \frac{1}{\cos x} = \frac{1+\tan^2\frac{x}{2}}{1-\tan^2\frac{x}{2}}
\]

(b) 部分分式分解:
\[
\frac{2}{1-t^2} = \frac{2}{(1-t)(1+t)} = \frac{1}{1-t} + \frac{1}{1+t}
\]

(c) 使用代换 \(t = \tan\frac{x}{2} \implies dx = \frac{2}{1+t^2} dt\):
\[
\int \sec x \, dx = \int \frac{1+t^2}{1-t^2} \cdot \frac{2}{1+t^2} \, dt
= \int \frac{2}{1-t^2} \, dt
= \int \frac{1}{1+t} + \frac{1}{1-t} \, dt
= \ln|1+t| - \ln|1-t| + C
= \ln\left|\frac{1+t}{1-t}\right| + C
\]

注意到 \(\frac{1+t}{1-t} = \frac{\tan\frac{\pi}{4} + \tan\frac{x}{2}}{1 - \tan\frac{\pi}{4}\tan\frac{x}{2}} = \tan\left(\frac{\pi}{4} + \frac{x}{2}\right)\),
于是最终结果为:
\[
\int \sec x \, dx = \ln\left|\tan\left(\frac{x}{2} + \frac{\pi}{4}\right)\right| + C
\]
\end{solution}

    \question
求
\[
\int_{\frac{\pi}{6}}^{\frac{\pi}{3}} \frac{\cos x + \sin x}{\sqrt{\sin 2x}} \, dx
\]

\begin{solution}
首先将被积式化简:
\[
\int_{\frac{\pi}{6}}^{\frac{\pi}{3}} \frac{\cos x + \sin x}{\sqrt{\sin 2x}} \, dx = \int_{\frac{\pi}{6}}^{\frac{\pi}{3}} \frac{\cos x + \sin x}{\sqrt{2 \sin x \cos x}} \, dx
\]

使用代换:
\[
u = \sin x - \cos x, \quad du = (\cos x + \sin x) \, dx, \quad dx = \frac{du}{\cos x + \sin x}
\]

积分上下限:
\[
x = \frac{\pi}{6} \to u = \frac{1}{2} - \frac{\sqrt{3}}{2} = -\alpha, \quad
x = \frac{\pi}{3} \to u = \frac{\sqrt{3}}{2} - \frac{1}{2} = \alpha
\]

关系式:
\[
u^2 = (\sin x - \cos x)^2 = \sin^2 x + \cos^2 x - 2 \sin x \cos x = 1 - 2 \sin x \cos x
\]
\[
\Rightarrow 2 \sin x \cos x = 1 - u^2
\]

代入积分:

\[
\int_{\frac{\pi}{6}}^{\frac{\pi}{3}} \frac{\cos x + \sin x}{\sqrt{2 \sin x \cos x}} \, dx = \int_{-\alpha}^{\alpha} \frac{\cos x + \sin x}{\sqrt{1-u^2}} \cdot \frac{du}{\cos x + \sin x} = \int_{-\alpha}^{\alpha} \frac{1}{\sqrt{1-u^2}} \, du
\]

由于被积函数是偶函数:
\[
\int_{-\alpha}^{\alpha} \frac{1}{\sqrt{1-u^2}} \, du = 2 \int_{0}^{\alpha} \frac{1}{\sqrt{1-u^2}} \, du = 2 [\arcsin u]_{0}^{\alpha} = 2 \arcsin \alpha
\]

代入 \(\alpha = \frac{\sqrt{3}}{2}-\frac{1}{2}\):

\[
\int_{\frac{\pi}{6}}^{\frac{\pi}{3}} \frac{\cos x + \sin x}{\sqrt{\sin 2x}} \, dx = 2 \arcsin \left( \frac{\sqrt{3}-1}{2} \right)
\]
\end{solution}

\question
27) 计算不定积分:\[ \int \frac{\sin^3 \frac{x}{2}}{\cos \frac{x}{2} \sqrt{\cos x + \cos^2 x + \cos^3 x}} dx \]

\begin{solution}
利用三角恒等式简化分子,并设 $t = \cos x, dt = -\sin x dx$:
\[ I = \frac{1}{2} \int \frac{\sin x (1 - \cos x)}{(1 + \cos x) \sqrt{\cos x + \cos^2 x + \cos^3 x}} dx \]
代入变量 $t$:
\[ = \frac{1}{2} \int \frac{t^2 - 1}{(t+1)^2 \sqrt{t + t^2 + t^3}} dt = \frac{1}{2} \int \frac{1 - \frac{1}{t^2}}{(t+2+\frac{1}{t}) \sqrt{t+1+\frac{1}{t}}} dt \]
设 $u^2 = t + 1 + \frac{1}{t}$,则 $2u du = (1 - \frac{1}{t^2}) dt$:
\[ I = \int \frac{du}{u^2+1} = \tan^{-1} u + C \]
还原变量:
\[ I = \tan^{-1} \sqrt{\cos x + \sec x + 1} + C \]
\end{solution}

\question 计算积分
\[
\int_{\frac{1}{2}}^{\frac{\sqrt{3}}{2}} \frac{1 - \sqrt{\arcsin x}}{\sqrt{1-x^2}\sqrt{\arcsin x}} \, dx
\]
\begin{solution}
使用代换
\[
u = \sqrt{\arcsin x} \implies u^2 = \arcsin x, \quad 2u\,du = \frac{1}{\sqrt{1-x^2}}\,dx, \quad dx = 2u \sqrt{1-x^2} \, du
\]

积分上下限变为:
\[
x = \frac{1}{2} \implies u = \sqrt{\arcsin \frac{1}{2}} = \sqrt{\frac{\pi}{6}}, \quad
x = \frac{\sqrt{3}}{2} \implies u = \sqrt{\arcsin \frac{\sqrt{3}}{2}} = \sqrt{\frac{\pi}{3}}
\]

原积分变为:
\[
\int_{\sqrt{\pi/6}}^{\sqrt{\pi/3}} \frac{1 - u^2}{u} \cdot 2u \, du = \int_{\sqrt{\pi/6}}^{\sqrt{\pi/3}} 2(1-u) \, du = \int_{\sqrt{\pi/6}}^{\sqrt{\pi/3}} (2 - 2u) \, du
\]

积分得到:
\[
\left[ 2u - u^2 \right]_{\sqrt{\pi/6}}^{\sqrt{\pi/3}} = \left(2\sqrt{\frac{\pi}{3}} - \frac{\pi}{3}\right) - \left(2\sqrt{\frac{\pi}{6}} - \frac{\pi}{6}\right)
\]

化简:
\[
= 2\sqrt{\frac{\pi}{3}} - 2\sqrt{\frac{\pi}{6}} - \frac{\pi}{3} + \frac{\pi}{6} = 2\sqrt{\frac{\pi}{3}} - 2\sqrt{\frac{\pi}{6}} - \frac{\pi}{6}
\]

因此积分结果为:
\[
\int_{\frac{1}{2}}^{\frac{\sqrt{3}}{2}} \frac{1 - \sqrt{\arcsin x}}{\sqrt{1-x^2}\sqrt{\arcsin x}} \, dx
= 2\sqrt{\frac{\pi}{3}} - 2\sqrt{\frac{\pi}{6}} - \frac{\pi}{6}.
\]
\end{solution}

\question
利用适当的积分方法证明
\[
\int_{0}^{\frac{1}{2}}
\frac{\arcsin\sqrt{x}-\arccos\sqrt{x}}
{\arcsin\sqrt{x}+\arccos\sqrt{x}}
\,dx
=\frac{1}{\pi}-\frac{1}{2}
\]

\begin{solution}
先作代换
\[
y=\sqrt{x}
\]
则
\[
x=y^2,\quad dx=2y\,dy
\]
积分上下限变为
\[
x=0\Rightarrow y=0,\qquad
x=\frac12\Rightarrow y=\frac{\sqrt2}{2}
\]

原积分化为
\begin{align*}
\int_{0}^{\frac{\sqrt2}{2}}
\frac{\arcsin y-\arccos y}{\arcsin y+\arccos y}
\cdot 2y\,dy
\end{align*}

利用恒等式
\[
\arccos y=\frac{\pi}{2}-\arcsin y
\]
得
\[
\frac{\arcsin y-\arccos y}{\arcsin y+\arccos y}
=\frac{2\arcsin y-\frac{\pi}{2}}{\frac{\pi}{2}}
=\frac{4}{\pi}\arcsin y-1
\]

因此
\begin{align*}
&=\int_{0}^{\frac{\sqrt2}{2}}
\left(\frac{4}{\pi}\arcsin y-1\right)2y\,dy \\
&=\frac{8}{\pi}\int_{0}^{\frac{\sqrt2}{2}}y\arcsin y\,dy
-\int_{0}^{\frac{\sqrt2}{2}}2y\,dy \\
&=\frac{8}{\pi}\int_{0}^{\frac{\sqrt2}{2}}y\arcsin y\,dy
-\left[y^2\right]_{0}^{\frac{\sqrt2}{2}} \\
&=\frac{8}{\pi}\int_{0}^{\frac{\sqrt2}{2}}y\arcsin y\,dy
-\frac12
\end{align*}

再作代换
\[
y=\sin\theta
\]
则
\[
dy=\cos\theta\,d\theta
\]
积分上下限变为
\[
y=0\Rightarrow\theta=0,\qquad
y=\frac{\sqrt2}{2}\Rightarrow\theta=\frac{\pi}{4}
\]

于是
\begin{align*}
\int_{0}^{\frac{\sqrt2}{2}}y\arcsin y\,dy
&=\int_{0}^{\frac{\pi}{4}}\theta\sin\theta\cos\theta\,d\theta \\
&=\frac12\int_{0}^{\frac{\pi}{4}}\theta\sin(2\theta)\,d\theta
\end{align*}

代回原式
\[
=\frac{4}{\pi}\int_{0}^{\frac{\pi}{4}}\theta\sin(2\theta)\,d\theta-\frac12
\]

对积分作分部积分
\[
u=\theta,\quad dv=\sin(2\theta)\,d\theta
\]
则
\[
du=d\theta,\quad v=-\frac12\cos(2\theta)
\]

\begin{align*}
\int_{0}^{\frac{\pi}{4}}\theta\sin(2\theta)\,d\theta
&=\left[-\frac12\theta\cos(2\theta)\right]_{0}^{\frac{\pi}{4}}
+\frac12\int_{0}^{\frac{\pi}{4}}\cos(2\theta)\,d\theta \\
&=\frac12\left[\frac12\sin(2\theta)\right]_{0}^{\frac{\pi}{4}} \\
&=\frac14
\end{align*}

因此
\[
\frac{4}{\pi}\cdot\frac14-\frac12
=\frac{1}{\pi}-\frac12
\]

证毕。
\end{solution}

\question
计算 $[\int \frac{3\sin 2x + 4\cos x e^{\sin x} + 8\cos x}{3\sin x + e^{\sin x} + 1} dx]$。

\begin{solution}
首先,利用二倍角公式 $\sin 2x = 2\sin x \cos x$ 展开分子:
\[ = [\int \frac{6\sin x \cos x + 4\cos x e^{\sin x} + 8\cos x}{3\sin x + e^{\sin x} + 1} dx] \]
令 $u = \sin x$,则 $du = \cos x dx$。代入原式得:
\[ = [\int \frac{6u + 4e^u + 8}{3u + e^u + 1} du] \]
提取分子中的常数 2:
\[ = 2 [\int \frac{3u + 2e^u + 4}{3u + e^u + 1} du] \]
将分子拆项,使其包含分母的形式:
\[ = 2 [\int \frac{(3u + e^u + 1) + (3 + e^u)}{3u + e^u + 1} du] \]
\[ = 2 ( [\int \frac{3u + e^u + 1}{3u + e^u + 1} du] + [\int \frac{e^u + 3}{3u + e^u + 1} du] ) \]
\[ = 2 [\int 1 du] + 2 [\int \frac{d(3u + e^u + 1)}{3u + e^u + 1}] \]
进行积分:
\[ = 2u + 2\ln|3u + e^u + 1| + C \]
最后回代 $u = \sin x$:
\[ = 2\sin x + 2\ln|3\sin x + e^{\sin x} + 1| + C \]
\end{solution}

\question 已知 $\int \csc x dx = \ln |\csc x - \cot x| + C$
\begin{solution}
(a) 证明 $\sin x + \sin 3x + \sin 5x = \frac{\sin^2 3x}{\sin x}$:
\begin{align*}
\sin x (\sin x + \sin 3x + \sin 5x) &= \sin x (2\sin \frac{x+5x}{2} \cos \frac{x-5x}{2} + \sin 3x) \\
&= \sin x (2\sin 3x \cos(-2x) + \sin 3x) \\
&= \sin x \sin 3x (2\cos 2x + 1) \\
&= \sin 3x (\sin(x+2x) + \sin(x-2x) + \sin x) \\
&= \sin 3x (\sin 3x - \sin x + \sin x) \\
&= \sin^2 3x
\end{align*}
因此得证:$\sin x + \sin 3x + \sin 5x = \frac{\sin^2 3x}{\sin x}$

(b) 计算 $\int \frac{\sin x + \sin 3x}{\cos x \sqrt{4 - \cos^4 x}} dx$:
设 $u = \cos^2 x$,则 $du = -2 \sin x \cos x dx$。
\begin{align*}
\int \frac{2 \sin \frac{x+3x}{2} \cos \frac{x-3x}{2}}{\cos x \sqrt{4 - \cos^4 x}} dx &= \int \frac{2 \sin 2x \cos(-x)}{\cos x \sqrt{4 - \cos^4 x}} dx \\
&= \int \frac{4 \sin x \cos x \cos x}{\cos x \sqrt{4 - \cos^4 x}} dx \\
&= \int \frac{4 \sin x \cos x}{\sqrt{4 - \cos^4 x}} dx \\
&= \int \frac{-2}{\sqrt{4 - u^2}} du \\
&= -2 \arcsin (\frac{u}{2}) + C \\
&= -2 \arcsin (\frac{\cos^2 x}{2}) + C
\end{align*}

(c) 计算 $\int \frac{16(\sin 2x + \sin 6x + \sin 10x)}{\cos 2x \sqrt{4 - \cos^4 2x}} dx$:
利用 (a) 栏的结果,设 $2x$ 为变量,则分子部分可化简为 $16 \frac{\sin^2 6x}{\sin 2x}$。
设 $u = \cos^2 2x$,则 $du = -4 \sin 2x \cos 2x dx$。
经过代换与积分(参考第二张图内容):
\begin{align*}
\int (\frac{32u-4}{u\sqrt{4-u^2}} - \frac{64u}{\sqrt{4-u^2}}) du &= 32 \arcsin (\frac{u}{2}) + 2 \ln |\frac{2 + \sqrt{4-u^2}}{u}| + 64 \sqrt{4-u^2} + C
\end{align*}
代回 $u = \cos^2 2x$ 得最终结果:
\[ I = 32 \arcsin (\frac{\cos^2 2x}{2}) + 2 \ln |\frac{2 + \sqrt{4 - \cos^4 2x}}{\cos^2 2x}| + 64 \sqrt{4 - \cos^4 2x} + C \]
\end{solution}

\question
(a) 计算 $[\int \frac{1}{4-3z^2} dz]$。
\begin{solution}
令 $z = \frac{2}{\sqrt{3}} \sin\theta$,则 $\frac{dz}{d\theta} = \frac{2}{\sqrt{3}} \cos\theta \implies dz = \frac{2}{\sqrt{3}} \cos\theta d\theta$。
\[ [\int \frac{1}{4-4\sin^2\theta} (\frac{2}{\sqrt{3}}\cos\theta) d\theta] = [\int \frac{1}{4\cos^2\theta} (\frac{2}{\sqrt{3}}\cos\theta) d\theta] \]
\[ = \frac{1}{2\sqrt{3}} [\int \sec\theta d\theta] = \frac{1}{2\sqrt{3}} \ln |\sec\theta + \tan\theta| + C \]
根据辅助三角形,$\sin\theta = \frac{\sqrt{3}z}{2},\sec\theta = \frac{2}{\sqrt{4-3z^2}},\tan\theta = \frac{\sqrt{3}z}{\sqrt{4-3z^2}}$。
\[ = \frac{1}{2\sqrt{3}} \ln |\frac{2+\sqrt{3}z}{\sqrt{4-3z^2}}| + C = \frac{1}{2\sqrt{3}} \ln |\sqrt{\frac{(2+\sqrt{3}z)^2}{(2-\sqrt{3}z)(2+\sqrt{3}z)}}| + C \]
\[ = \frac{1}{4\sqrt{3}} \ln |\frac{2+\sqrt{3}z}{2-\sqrt{3}z}| + C = \frac{\sqrt{3}}{12} \ln |\frac{2+\sqrt{3}z}{2-\sqrt{3}z}| + C \]
\end{solution}

\question
(b) 计算 $[\int \frac{\cos u}{\tan^2 u + 4} du]$。
\begin{solution}
\[ [\int \frac{\cos u}{\frac{\sin^2 u}{\cos^2 u} + 4} du] = [\int \frac{\cos^3 u}{\sin^2 u + 4\cos^2 u} du] \]
\[ = [\int \frac{\cos^3 u}{\sin^2 u + 4(1-\sin^2 u)} du] = [\int \frac{\cos^3 u}{4-3\sin^2 u} du] \]
令 $t = \sin u$,则 $dt = \cos u du,\cos^2 u = 1 - t^2$。
\[ = [\int \frac{1-t^2}{4-3t^2} dt] = [\int \frac{\frac{1}{3}(4-3t^2) - \frac{1}{3}}{4-3t^2} dt] = \frac{1}{3} ( [\int 1 dt] - [\int \frac{1}{4-3t^2} dt] ) \]
利用(a)的结果:
\[ = \frac{\sin u}{3} - \frac{1}{3} (\frac{\sqrt{3}}{12} \ln |\frac{2+\sqrt{3}\sin u}{2-\sqrt{3}\sin u}|) + C = \frac{\sin u}{3} - \frac{\sqrt{3}}{36} \ln |\frac{2+\sqrt{3}\sin u}{2-\sqrt{3}\sin u}| + C \]
\end{solution}

\question
(c) 计算 $[\int_{-\frac{\pi}{4}}^{\frac{\pi}{4}} \frac{90(\sec x - \cos x)}{(\sec x + 3\cos x)^2} dx]$。
\begin{solution}
\[ = 90 [\int_{-\frac{\pi}{4}}^{\frac{\pi}{4}} \frac{\frac{1}{\cos x} - \cos x}{(\frac{1}{\cos x} + 3\cos x)^2} dx] = 90 [\int_{-\frac{\pi}{4}}^{\frac{\pi}{4}} \frac{\sin^2 x \cos x}{(1+3\cos^2 x)^2} dx] \]
令 $t = \sin x$,则 $dt = \cos x dx$。当 $x = \pm\frac{\pi}{4}$ 时,$t = \pm\frac{\sqrt{2}}{2}$。
\[ = 90 [\int_{-\frac{\sqrt{2}}{2}}^{\frac{\sqrt{2}}{2}} \frac{t^2}{(1+3(1-t^2))^2} dt] = 180 [\int_{0}^{\frac{\sqrt{2}}{2}} \frac{t^2}{(4-3t^2)^2} dt] \]
对不定积分使用换元 $t = \frac{2}{\sqrt{3}}\sin\theta$:
\[ [\int \frac{t^2}{(4-3t^2)^2} dt] = [\int \frac{\frac{4}{3}\sin^2\theta}{16\cos^4\theta} \frac{2}{\sqrt{3}}\cos\theta d\theta] = \frac{1}{6\sqrt{3}} [\int \tan^2\theta \sec\theta d\theta] \]
\[ = \frac{1}{6\sqrt{3}} ([\int \sec^3\theta d\theta] - [\int \sec\theta d\theta]) = \frac{1}{12\sqrt{3}} (\sec\theta \tan\theta - \ln |\sec\theta + \tan\theta|) + C \]
回代 $t$ 并带入上下限:
\[ [\int_{0}^{\frac{\sqrt{2}}{2}} \frac{t^2}{(4-3t^2)^2} dt] = \frac{1}{6\sqrt{3}} [ \frac{\sqrt{3}t}{4-3t^2} - \frac{1}{2} \ln (\frac{2+\sqrt{3}t}{\sqrt{4-3t^2}}) ] \bigg|_{0}^{\frac{\sqrt{2}}{2}} \]
\[ = \frac{1}{6\sqrt{3}} [ \frac{\frac{\sqrt{6}}{2}}{4-\frac{3}{2}} - \frac{1}{2} \ln (\frac{2+\frac{\sqrt{6}}{2}}{\sqrt{4-\frac{3}{2}}}) ] = \frac{1}{15\sqrt{2}} - \frac{1}{12\sqrt{3}} \ln (\frac{2\sqrt{2}+\sqrt{3}}{\sqrt{5}}) \]
乘以系数 180:
\[ 180 [\int_{0}^{\frac{\sqrt{2}}{2}} \frac{t^2}{(4-3t^2)^2} dt] = 6\sqrt{2} - 5\sqrt{3} \ln (\frac{2\sqrt{2}+\sqrt{3}}{\sqrt{5}}) \]
\end{solution}

\question
计算以下积分:
(a) $[\int \frac{2\sin x}{\cos^2 x + 4} dx]$
(b) $[\int \frac{2\sec^2 x + 3\sec x \tan x}{4 + \tan^2 x} dx]$

\begin{solution}
(a) 令 $u = \cos x$,则 $du = -\sin x dx$。
\[ [\int \frac{2\sin x}{\cos^2 x + 4} dx] = -[\int \frac{2}{u^2 + 4} du] \]
进一步令 $u = 2\tan v$,则 $du = 2\sec^2 v dv$。
\[ = -[\int \frac{2}{(2\tan v)^2 + 4} \cdot 2\sec^2 v dv] = -[\int \frac{4\sec^2 v}{4(1 + \tan^2 v)} dv] \]
\[ = -[\int 1 dv] = -v + C \]
由于 $v = \tan^{-1}(\frac{u}{2})$ 且 $u = \cos x$:
\[ = -\tan^{-1}(\frac{\cos x}{2}) + C \]

(b) 将积分拆分为两部分:
\[ [\int \frac{2\sec^2 x + 3\sec x \tan x}{4 + \tan^2 x} dx] = [\int \frac{2\sec^2 x}{4 + \tan^2 x} dx] + [\int \frac{3\sec x \tan x}{4 + \tan^2 x} dx] \]
对于第一部分,令 $v = \tan x$,则 $dv = \sec^2 x dx$。
对于第二部分,令 $u = \sec x$,由于 $1 + \tan^2 x = \sec^2 x$,故 $\tan^2 x = u^2 - 1$。
\[ = [\int \frac{2}{4 + v^2} dv] + [\int \frac{3}{4 + (u^2 - 1)} du] \]
\[ = [\int \frac{2}{v^2 + 4} dv] + [\int \frac{3}{u^2 + 3} du] \]
使用积分公式 $[\int \frac{1}{x^2 + a^2} dx] = \frac{1}{a} \tan^{-1}(\frac{x}{a}) + C$:
第一部分:令 $v = 2\tan \alpha$,得 $\frac{2}{2} \tan^{-1}(\frac{v}{2}) = \tan^{-1}(\frac{\tan x}{2})$。
(注:原稿中此处使用了 $v = \sqrt{5}\tan \alpha$ 的换元,似乎针对的是分母为 $5 + \tan^2 x$ 的情况,若按题目 $4 + \tan^2 x$ 计算如下):
\[ = \tan^{-1}(\frac{\tan x}{2}) + \frac{3}{\sqrt{3}} \tan^{-1}(\frac{\sec x}{\sqrt{3}}) + C \]
\[ = \tan^{-1}(\frac{\tan x}{2}) + \sqrt{3} \tan^{-1}(\frac{\sec x}{\sqrt{3}}) + C \]
\end{solution}
%分部积分by parts
\question
2) 计算不定积分:\[ \int \frac{x^2}{(1+x^2)^2} dx \]

\begin{solution}
使用分部积分法,令 $u=x, dv = \frac{x}{(1+x^2)^2} dx$:
则 $du = dx, v = -\frac{1}{2(1+x^2)}$。

根据分部积分公式:
\begin{align*}
\int \frac{x^2}{(1+x^2)^2} dx &= -\frac{x}{2(1+x^2)} + \int \frac{1}{2(1+x^2)} dx \\
&= -\frac{x}{2(1+x^2)} + \frac{1}{2} \tan^{-1} x + C
\end{align*}
\end{solution}

\question
3) 计算不定积分:\[ \int x e^x \cos x dx \]

\begin{solution}
首先利用分部积分法处理 $e^x \cos x$。已知:
\[ \int e^x \cos x dx = \frac{1}{2} e^x (\sin x + \cos x) \]
\[ \int e^x \sin x dx = \frac{1}{2} e^x (\sin x - \cos x) \]

对原式使用分部积分,令 $u=x, dv = e^x \cos x dx$:
\begin{align*}
\int x e^x \cos x dx &= \frac{1}{2} x e^x (\sin x + \cos x) - \int \frac{1}{2} e^x (\sin x + \cos x) dx \\
&= \frac{1}{2} x e^x (\sin x + \cos x) - \frac{1}{2} \left[ \int e^x \sin x dx + \int e^x \cos x dx \right] \\
&= \frac{1}{2} x e^x (\sin x + \cos x) - \frac{1}{2} \left[ \frac{1}{2} e^x (\sin x - \cos x) + \frac{1}{2} e^x (\sin x + \cos x) \right] \\
&= \frac{1}{2} x e^x (\sin x + \cos x) - \frac{1}{2} e^x \sin x + C \\
&= \frac{1}{2} e^x (x \sin x + x \cos x - \sin x) + C
\end{align*}
\end{solution}

\question 计算 $[\int_0^1 \frac{\sin^{-1}x}{x} dx]$。
\begin{solution}
令 $x = \sin \theta$,则 $dx = \cos \theta d\theta$。
积分式转化为:
\[I = [\int_0^{\pi/2} \frac{\theta \cos \theta}{\sin \theta} d\theta] = [\int_0^{\pi/2} \theta \cot \theta d\theta]\]
使用分部积分法,令 $u = \theta,dv = \cot \theta d\theta$:
\[I = [\theta \ln(\sin \theta)]_0^{\pi/2} - [\int_0^{\pi/2} \ln(\sin \theta) d\theta]\]
第一项代入上下限后为 $0$。利用标准积分公式 $[\int_0^{\pi/2} \ln(\sin \theta) d\theta] = -\frac{\pi}{2} \ln 2$:
\[I = -(-\frac{\pi}{2} \ln 2) = \frac{\pi}{2} \ln 2\]
\end{solution}
    \question $$\int_0^{\infty} e^{-x} \sin x \, dx$$by parts
    \begin{solution}
        设
        \[
        I = \int_0^{\infty} e^{-x} \sin x \, dx.
        \]
        分部积分两次得
        \[
        I = -e^{-x} \sin x + \int e^{-x} \cos x \, dx
           = -e^{-x} (\sin x + \cos x) - I.
        \]
        移项得
        \[
        2I = -e^{-x} (\sin x + \cos x)
        \Rightarrow I = -\frac{1}{2} e^{-x} (\sin x + \cos x).
        \]
        取定积分,注意 $\lim_{x \to \infty} e^{-x} (\sin x + \cos x) = 0$,代入得:
        \[
        I = \left[ -\frac{1}{2} e^{-x} (\sin x + \cos x) \right]_0^{\infty}
        = 0 - \left(-\frac{1}{2} \cdot 1\right) = \frac{1}{2}.
        \]
    \end{solution}

    \question 
    \[
    \int \frac{xe^x}{(x+1)^2} \,dx
    \]
    \begin{solution}
        换元 $u=x+1,$再分部积分得 
        \begin{align*}
        \int \frac{xe^x}{(x+1)^2} \,dx 
        &= \int \frac{(u - 1)e^{u - 1}}{u^2} \,du \\
        &= \frac{1}{e} \int \frac{(u - 1)e^u}{u^2} \,du \\
        &= \frac{1}{e} \left( \int \frac{e^u}{u} \,du - \int \frac{e^u}{u^2} \,du \right) \\
        &= \frac{1}{e} \left( \frac{e^u}{u} + \int \frac{e^u}{u^2} \,du - \int \frac{e^u}{u^2} \,du \right) \\
        &= \frac{e^{u - 1}}{u} + C\\
        &= \frac{e^x}{x + 1} + C
        \end{align*}
    \end{solution}

\question
23) 计算不定积分:\[ \int \sin^{-1} \sqrt{\frac{x}{1+x}} dx \]

\begin{solution}
使用分部积分法,令 $u = \sin^{-1} \sqrt{\frac{x}{1+x}}, dv = dx$:
\begin{align*}
I &= x \sin^{-1} \sqrt{\frac{x}{1+x}} - \int x \cdot \frac{1}{\sqrt{1 - \frac{x}{1+x}}} \cdot \frac{1}{2\sqrt{\frac{x}{1+x}}} \cdot \frac{1}{(1+x)^2} dx \\
&= x \sin^{-1} \sqrt{\frac{x}{1+x}} - \frac{1}{2} \int x \cdot \sqrt{1+x} \cdot \frac{\sqrt{1+x}}{\sqrt{x}} \cdot \frac{1}{(1+x)^2} dx \\
&= x \sin^{-1} \sqrt{\frac{x}{1+x}} - \frac{1}{2} \int \frac{\sqrt{x}}{1+x} dx
\end{align*}

对剩余积分项进行凑项处理:
\[ \frac{1}{2} \int \frac{x+1-1}{\sqrt{x}(1+x)} dx = \int \frac{1}{2\sqrt{x}} dx - \int \frac{1}{2\sqrt{x}(1+x)} dx \]

进行积分计算:
\begin{itemize}
    \item 第一项:$\int \frac{1}{2\sqrt{x}} dx = \sqrt{x}$。
    \item 第二项:设 $\sqrt{x} = t$,则 $\int \frac{1}{1+t^2} dt = \tan^{-1} t = \tan^{-1}\sqrt{x}$。
\end{itemize}

合并最终结果:
\[ I = x \sin^{-1} \sqrt{\frac{x}{1+x}} - \sqrt{x} + \tan^{-1}\sqrt{x} + C \]
\end{solution}

\question
5) 计算不定积分 $\int x \tan^{-1}x^2 dx$

\begin{solution}
设 $u = x^2, du = 2x dx$:
\begin{align*}
\int x \tan^{-1}x^2 dx &= \frac{1}{2} \int \tan^{-1}u du \\
\intertext{使用分部积分法:}
&= \frac{1}{2} \left( u \tan^{-1}u - \int \frac{u}{1+u^2} du \right) \\
&= \frac{1}{2} u \tan^{-1}u - \frac{1}{4} \ln(1+u^2) + C \\
&= \frac{1}{2} x^2 \tan^{-1}x^2 - \frac{1}{4} \ln(1+x^4) + C
\end{align*}
\end{solution}
\question
15) 计算不定积分:\[ \int \frac{x \sin x}{\cos^5 x} dx \]

\begin{solution}
使用分部积分法,令 $u = x, dv = \frac{\sin x}{\cos^5 x} dx$。
对于 $v = \int \tan x \sec^4 x dx$,设 $\tan x = t$:
\[ v = \int t(1+t^2) dt = \frac{1}{2} \tan^2 x + \frac{1}{4} \tan^4 x = \frac{1}{4} \sec^4 x - \frac{1}{4} \]
代入分部积分公式:
\[ I = \frac{x}{4} \sec^4 x - \int \frac{1}{4} \sec^4 x dx \]
利用 $\int \sec^4 x dx = \tan x + \frac{1}{3} \tan^3 x$:
\[ I = \frac{x \sec^4 x}{4} - \frac{1}{4} \tan x - \frac{1}{12} \tan^3 x + C \]
\end{solution}
\question
9) 计算不定积分:\[ I = \int \cos(3 \ln x) dx \]

\begin{solution}
使用换元法,设 $x = e^y$,则 $dx = e^y dy$。
代入原积分:
\[ I = \int e^y \cos 3y dy \]

对上述积分使用两次分部积分法:
\textbf{第一次分部积分}:
令 $u = \cos 3y, dv = e^y dy$,则 $du = -3 \sin 3y dy, v = e^y$。
\[ I = e^y \cos 3y + 3 \int e^y \sin 3y dy \]

\textbf{第二次分部积分}:
对 $\int e^y \sin 3y dy$ 再次令 $u = \sin 3y, dv = e^y dy$,则 $du = 3 \cos 3y dy, v = e^y$。
\[ \int e^y \sin 3y dy = e^y \sin 3y - 3 \int e^y \cos 3y dy \]

将结果代回 $I$ 的表达式:
\[ I = e^y \cos 3y + 3 \left( e^y \sin 3y - 3I \right) \]
\[ I = e^y \cos 3y + 3 e^y \sin 3y - 9I \]

解关于 $I$ 的方程:
\[ 10I = e^y (\cos 3y + 3 \sin 3y) \]
\[ I = \frac{e^y}{10} (\cos 3y + 3 \sin 3y) + C \]

还原变量 $y = \ln x$:
\[ I = \frac{x}{10} [\cos(3 \ln x) + 3 \sin(3 \ln x)] + C \]

\textbf{结果:}
\[ \int \cos(3 \ln x) dx = \frac{x}{10} [\cos(3 \ln x) + 3 \sin(3 \ln x)] + C \]
\end{solution}

    \question
求下列积分的精确值
\[
\int_{\frac{3}{2}}^{\frac{5}{2}} (4x^2-16x+15)^4\,dx
\]

\begin{solution}
先进行因式分解
\[
4x^2-16x+15=(2x-3)(2x-5)
\]
因此
\begin{align*}
\int_{\frac{3}{2}}^{\frac{5}{2}} (4x^2-16x+15)^4\,dx
&=\int_{\frac{3}{2}}^{\frac{5}{2}}[(2x-3)(2x-5)]^4\,dx \\
&=\int_{\frac{3}{2}}^{\frac{5}{2}}(2x-3)^4(2x-5)^4\,dx
\end{align*}

第一次分部积分
\begin{align*}
&=\left[\frac{1}{10}(2x-3)^5(2x-5)^4\right]_{\frac{3}{2}}^{\frac{5}{2}}
-\int_{\frac{3}{2}}^{\frac{5}{2}}\frac{4}{5}(2x-3)^5(2x-5)^3\,dx
\end{align*}

第二次分部积分
\begin{align*}
&=\left[\frac{4}{55}(2x-3)^5(2x-5)^4\right]_{\frac{3}{2}}^{\frac{5}{2}}
-\left[\frac{2}{5}(2x-3)^6(2x-5)^3\right]_{\frac{3}{2}}^{\frac{5}{2}} \\
&\quad+\frac{6}{5}\int_{\frac{3}{2}}^{\frac{5}{2}}(2x-3)^6(2x-5)^2\,dx
\end{align*}

第三次分部积分
\begin{align*}
&=\frac{2}{5}\left\{
\left[\frac{1}{14}(2x-3)^7(2x-5)^2\right]_{\frac{3}{2}}^{\frac{5}{2}}
-\frac{2}{7}\int_{\frac{3}{2}}^{\frac{5}{2}}(2x-3)^7(2x-5)\,dx
\right\} \\
&=-\frac{4}{35}\int_{\frac{3}{2}}^{\frac{5}{2}}(2x-3)^7(2x-5)\,dx
\end{align*}

最后一次分部积分
\begin{align*}
&=-\frac{4}{35}\left\{
\left[\frac{1}{16}(2x-3)^8(2x-5)\right]_{\frac{3}{2}}^{\frac{5}{2}}
-\frac{1}{8}\int_{\frac{3}{2}}^{\frac{5}{2}}(2x-3)^8\,dx
\right\} \\
&=\frac{1}{70}\int_{\frac{3}{2}}^{\frac{5}{2}}(2x-3)^8\,dx
\end{align*}

直接积分
\begin{align*}
&=\frac{1}{70}\left[\frac{(2x-3)^9}{18}\right]_{\frac{3}{2}}^{\frac{5}{2}} \\
&=\frac{1}{1260}\left[(2x-3)^9\right]_{\frac{3}{2}}^{\frac{5}{2}} \\
&=\frac{1}{1260}\left[(5-3)^9-(3-3)^9\right] \\
&=\frac{512}{1260} \\
&=\frac{128}{315}
\end{align*}
\end{solution}

\question
计算积分
\[
\int_{\pi/4}^{\pi/2} (\cos 2x + \sin 2x)(\ln \cos x + \ln \sin x) \, dx
\]

\begin{solution}
首先利用对数性质:
\[
\ln \cos x + \ln \sin x = \ln (\cos x \sin x) = \ln \frac{\sin 2x}{2} = \ln (\sin 2x) - \ln 2
\]

因此积分可以写为:
\[
\int_{\pi/4}^{\pi/2} (\cos 2x + \sin 2x)(\ln \sin 2x - \ln 2) \, dx
= \int_{\pi/4}^{\pi/2} (\cos 2x + \sin 2x)\ln \sin 2x \, dx - \ln 2 \int_{\pi/4}^{\pi/2} (\cos 2x + \sin 2x) \, dx
\]

第二个积分简单计算:
\[
\int_{\pi/4}^{\pi/2} (\cos 2x + \sin 2x) \, dx = \left[ \frac{1}{2}\sin 2x - \frac{1}{2}\cos 2x \right]_{\pi/4}^{\pi/2} = \frac{1}{2} - \left(\frac{1}{2} - 0\right) = 0
\]

所以问题简化为:
\[
\int_{\pi/4}^{\pi/2} (\cos 2x + \sin 2x) \ln \sin 2x \, dx
\]

使用分部积分,设
\[
u = \ln \sin 2x, \quad dv = (\cos 2x + \sin 2x) dx
\]

得到
\[
du = \frac{2\cos 2x}{\sin 2x} dx, \quad v = \frac{1}{2}(\sin 2x - \cos 2x)
\]

分部积分公式:
\[
\int u\, dv = uv - \int v\, du
\]

代入得:
\[
\int_{\pi/4}^{\pi/2} (\cos 2x + \sin 2x) \ln \sin 2x \, dx
= \left[ \frac{1}{2} (\sin 2x - \cos 2x) \ln \sin 2x \right]_{\pi/4}^{\pi/2} - \frac{1}{2} \int_{\pi/4}^{\pi/2} (\sin 2x - \cos 2x) \frac{2\cos 2x}{\sin 2x} \, dx
\]

化简被积函数:
\[
(\sin 2x - \cos 2x) \frac{2\cos 2x}{\sin 2x} = 2 \cos 2x - 2 \frac{\cos^2 2x}{\sin 2x} = 2 \cos 2x - 2 \frac{1-\sin^2 2x}{\sin 2x} = 2 (\cos 2x + \sin 2x - \csc 2x)
\]

于是积分化为:
\[
\int_{\pi/4}^{\pi/2} (\cos 2x + \sin 2x) \ln \sin 2x \, dx = \left[ \frac{1}{2}(\sin 2x - \cos 2x) \ln \sin 2x - \frac{1}{2}\sin 2x + \frac{1}{2}\cos 2x + \frac{1}{2} \ln \tan x \right]_{\pi/4}^{\pi/2}
\]

在端点代入:
\[
x = \pi/2: \quad \sin 2x = 0, \quad \cos 2x = -1 \implies \text{对数项消失}
\]
\[
x = \pi/4: \quad \sin 2x = 1, \quad \cos 2x = 0 \implies \text{得到 } \frac{1}{2} \ln 2
\]

因此最终结果为:
\[
\int_{\pi/4}^{\pi/2} (\cos 2x + \sin 2x)(\ln \cos x + \ln \sin x) \, dx = \frac{1}{2} \ln 2
\]
\end{solution}

    \question Evaluate $\int \frac{e^x(1+\sin x)}{1+\cos x}\,dx$.

\begin{solution}
将分母化为半角形式:
\begin{align*}
\int \frac{e^x(1+\sin x)}{1+\cos x}\,dx 
&= \int \frac{e^x \bigl(1 + 2\sin\frac{x}{2}\cos\frac{x}{2}\bigr)}{2\cos^2\frac{x}{2}} \, dx \\
&= \int e^x \left( \frac{1}{2\cos^2\frac{x}{2}} + \frac{2\sin\frac{x}{2}\cos\frac{x}{2}}{2\cos^2\frac{x}{2}} \right) \, dx \\
&= \int e^x \left( \frac{1}{2}\sec^2\frac{x}{2} + \tan\frac{x}{2} \right) \, dx.
\end{align*}

注意 $\frac{d}{dx}\tan\frac{x}{2} = \frac{1}{2}\sec^2\frac{x}{2}$,于是利用公式 $\int e^x (f(x)+f'(x))\,dx = e^x f(x) + C$ 得:
\[
\int \frac{e^x(1+\sin x)}{1+\cos x}\,dx = e^x \tan\frac{x}{2} + C.
\]
\end{solution}

\question
计算积分
\[
\int \left(1 + x - \frac{1}{x}\right) e^{x+\frac{1}{x}} \, dx
\]

\begin{solution}
观察被积函数,猜测含指数的项需要积分,尝试寻找一个全微分。

注意
\[
\frac{d}{dx}\left[e^{x+\frac{1}{x}}\right] = \left(1 - \frac{1}{x^2}\right) e^{x+\frac{1}{x}}
\]

将积分重写为:
\[
\int \left(1 + x - \frac{1}{x}\right) e^{x+\frac{1}{x}} \, dx
= \int \left[ e^{x+\frac{1}{x}} + x \left(1 - \frac{1}{x^2}\right) e^{x+\frac{1}{x}} \right] \, dx
\]

拆分积分:
\[
= \int e^{x+\frac{1}{x}} \, dx + \int x \left(1 - \frac{1}{x^2}\right) e^{x+\frac{1}{x}} \, dx
\]

对第二项使用分部积分得到:
\[
\int x \left(1 - \frac{1}{x^2}\right) e^{x+\frac{1}{x}} \, dx
= x e^{x+\frac{1}{x}} - \int e^{x+\frac{1}{x}} \, dx
\]

因此原积分为:
\[
\int \left(1 + x - \frac{1}{x}\right) e^{x+\frac{1}{x}} \, dx
= \int e^{x+\frac{1}{x}} \, dx + \left[x e^{x+\frac{1}{x}} - \int e^{x+\frac{1}{x}} \, dx\right]
= x e^{x+\frac{1}{x}} + C
\]
\end{solution}

\question
证明
\[
\int_{0}^{1} 12x^2 \arctan x \, dx = \pi - 2 + \ln 4
\]

\begin{solution}
\text{步骤 1:分部积分} \\
令 
\[
u = \arctan x, \quad dv = 12x^2 \, dx \implies du = \frac{1}{1+x^2} \, dx, \quad v = 4x^3
\]
利用分部积分公式:
\[
\int u\,dv = uv - \int v\,du
\]
得到:
\[
\int_{0}^{1} 12x^2 \arctan x \, dx = \big[ 4x^3 \arctan x \big]_{0}^{1} - \int_{0}^{1} 4x^3 \cdot \frac{1}{1+x^2} \, dx
\]

\text{步骤 2:计算第一项} 
\[
\big[ 4x^3 \arctan x \big]_{0}^{1} = 4 \cdot \arctan 1 - 0 = 4 \cdot \frac{\pi}{4} = \pi
\]

\text{步骤 3:化简剩余积分} \\
\[
\int_{0}^{1} \frac{4x^3}{1+x^2} \, dx
\] 
令
\[
w = 1+x^2 \implies dw = 2x\,dx, \quad x^2 = w-1
\] 
积分上下限:
\[
x=0 \to w=1, \quad x=1 \to w=2
\]
于是:
\begin{align*}
\int_{0}^{1} \frac{4x^3}{1+x^2} \, dx &= \int_{1}^{2} \frac{4(w-1) \cdot x}{w} \frac{dw}{2x} = 2 \int_{1}^{2} \frac{w-1}{w} \, dw \\
&= 2 \int_{1}^{2} \left( 1 - \frac{1}{w} \right) \, dw = 2 \big[ w - \ln|w| \big]_{1}^{2}
\end{align*}

\text{步骤 4:代入上下限} 
\begin{align*}
2 \big[ w - \ln|w| \big]_{1}^{2} &= 2 \big( (2 - \ln 2) - (1 - \ln 1) \big) \\
&= 2 (1 - \ln 2) = 2 - 2\ln 2 = 2 - \ln 4
\end{align*}

\text{步骤 5:组合结果} 
\[
\int_{0}^{1} 12x^2 \arctan x \, dx = \pi - (2 - \ln 4) = \pi - 2 + \ln 4
\]
\end{solution}

\question 计算积分
\[
\int_{-\frac{1}{6}\ln 3}^{\frac{1}{6}\ln 3} 6 e^{-3x} \arctan(e^{3x}) \, dx
\]
\begin{solution}
作代换
\[
\theta = \arctan(e^{3x}) \implies \tan\theta = e^{3x}, \quad \sec^2\theta \, d\theta = 3 e^{3x} \, dx \implies dx = \frac{\sec^2\theta}{3 e^{3x}} \, d\theta
\]

积分上下限:
\[
x = -\frac{1}{6}\ln 3 \implies \theta = \arctan\frac{1}{\sqrt{3}} = \frac{\pi}{6}, \quad
x = \frac{1}{6}\ln 3 \implies \theta = \arctan(\sqrt{3}) = \frac{\pi}{3}
\]

代入积分:
\[
\int_{-\frac{1}{6}\ln 3}^{\frac{1}{6}\ln 3} 6 e^{-3x} \arctan(e^{3x}) \, dx
= \int_{\frac{\pi}{6}}^{\frac{\pi}{3}} 2 \theta \csc^2 \theta \, d\theta
\]

使用分部积分:
\[
\int 2 \theta \csc^2 \theta \, d\theta = -2 \theta \cot \theta + \int 2 \cot \theta \, d\theta = -2 \theta \cot \theta + 2 \ln|\sin\theta|
\]

代入上下限:
\[
\left[-2\theta\cot\theta + 2\ln|\sin\theta|\right]_{\frac{\pi}{6}}^{\frac{\pi}{3}}
= \left(2\ln\sin\frac{\pi}{3} - \frac{2\pi}{3}\cot\frac{\pi}{3}\right) - \left(2\ln\sin\frac{\pi}{6} - \frac{\pi}{3}\cot\frac{\pi}{6}\right)
\]

计算各项:
\[
2\ln\frac{\sqrt{3}}{2} - \frac{2\pi}{3}\cdot \frac{1}{\sqrt{3}} - 2\ln\frac{1}{2} + \frac{\pi}{3}\cdot \sqrt{3}
= \ln\frac{3}{4} - \frac{2\pi\sqrt{3}}{9} + \ln 4 + \frac{\pi\sqrt{3}}{3}
\]

化简:
\[
\ln\frac{3}{4}\cdot 4 + \left(-\frac{2}{9} + \frac{1}{3}\right)\pi\sqrt{3} = \ln 3 + \frac{\pi \sqrt{3}}{9}
\]

因此积分结果为:
\[
\int_{-\frac{1}{6}\ln 3}^{\frac{1}{6}\ln 3} 6 e^{-3x} \arctan(e^{3x}) \, dx = \ln 3 + \frac{\pi \sqrt{3}}{9}
\]
\end{solution}

    \question Evaluate $\int e^{\arcsin x} \, dx$.

\begin{solution}
令 $y = \arcsin x$,则 $-\frac{\pi}{2} \le y \le \frac{\pi}{2}$。于是 $\sin y = x$,且 $dx = \cos y \, dy$,又 $\cos y = \sqrt{1-x^2}$。积分变为
\[
\int e^{\arcsin x} \, dx = \int e^y \cos y \, dy.
\]

对 $\int e^y \cos y \, dy$ 使用分部积分两次:
\begin{align*}
I &= \int e^y \cos y \, dy \\
&= e^y \sin y - \int e^y \sin y \, dy \\
&= e^y \sin y - \left[-e^y \cos y + \int e^y \cos y \, dy \right] \\
&= e^y \sin y + e^y \cos y - I
\end{align*}

解得
\[
2I = e^y (\sin y + \cos y) \implies I = \frac{1}{2} e^y (\sin y + \cos y) + C.
\]

代回 $y = \arcsin x,\sin y = x,\cos y = \sqrt{1-x^2}$:
\[
\int e^{\arcsin x} \, dx = \frac{1}{2} e^{\arcsin x} \bigl(x + \sqrt{1-x^2}\bigr) + C.
\]
\end{solution}

\question Evaluate $\displaystyle \int_0^1 (x^2 - x + 1)(e^{2x-1} + 1) dx$.

\begin{solution}
使用代换 $y = 1-x$,则 $dy = -dx$。当 $x=0, y=1$;当 $x=1, y=0$。

\begin{align*}
I &= \int_0^1 (x^2 - x + 1)(e^{2x-1} + 1) dx \\
&= -\int_1^0 [(-y+1)^2 - (-y+1) + 1][e^{2(-y+1)-1} + 1]\,dy \\
&= \int_0^1 (y^2 - y + 1)(e^{-2y+1} + 1)\,dy
\end{align*}

将积分变量改回 $x$:
\[
I = \int_0^1 (x^2 - x + 1)(e^{-2x+1} + 1)\,dx
\]

原积分 $I$ 的两种形式为:
\[
I = \int_0^1 (x^2 - x + 1)(e^{2x-1} + 1) dx, \quad
I = \int_0^1 (x^2 - x + 1)(e^{-2x+1} + 1) dx
\]

两式相加:
\begin{align*}
2I &= \int_0^1 (x^2 - x + 1)\left[(e^{2x-1}+1) + (e^{-2x+1}+1)\right] dx \\
&= \int_0^1 (x^2 - x + 1)\left[e^{2x-1} + e^{-(2x-1)} + 2\right] dx \\
&= \int_0^1 (x^2 - x + 1)\left[2\cosh(2x-1) + 2\right] dx
\end{align*}

注意到积分对称性,可以使用标准方法化简,结果为:
\[
\int_0^1 (x^2 - x + 1)(e^{2x-1} + 1) dx = \frac{2\pi\sqrt{3}}{9}.
\]
\textcolor{red}{(待验证)}

\end{solution}

    \question Evaluate $\int \cos 3x (\log(\sin 3x))^2\,dx$.

\begin{solution}
令 $y = \sin 3x$,则 $dy = 3\cos 3x\,dx$,所以 $\cos 3x\,dx = \frac{1}{3}\,dy$。积分变为:
\[
I = \frac{1}{3} \int (\ln y)^2\,dy.
\]

分部积分,取 $u=(\ln y)^2,dv=dy$,则 $du = \frac{2\ln y}{y}\,dy,v=y$:
\begin{align*}
I &= \frac{1}{3} \left[ y(\ln y)^2 - \int y \frac{2\ln y}{y}\,dy \right] \\
&= \frac{1}{3} \left[ y(\ln y)^2 - 2 \int \ln y\,dy \right].
\end{align*}

再次分部积分 $\int \ln y\,dy$:
\[
\int \ln y\,dy = y\ln y - \int 1\,dy = y\ln y - y + C.
\]

代回主积分:
\[
I = \frac{1}{3}\left[ y(\ln y)^2 - 2(y\ln y - y) \right] = \frac{y(\ln y)^2}{3} - \frac{2y\ln y}{3} + \frac{2y}{3} + C.
\]

代回 $y = \sin 3x$:
\[
\int \cos 3x (\log(\sin 3x))^2\,dx = \frac{\sin 3x (\ln(\sin 3x))^2}{3} - \frac{2\sin 3x \ln(\sin 3x)}{3} + \frac{2\sin 3x}{3} + C.
\]
\end{solution}

    \question 已知
    \[
    \int_0^\pi \left[f(x) + f''(x)\right]\sin x \, dx = 3,\quad f(\pi) = 2.
    \]
    求 \( f(0) \)。
    \begin{solution}
    分部积分得
    \begin{align*}
    \int_0^\pi f''(x)\sin x\, dx 
    &= \left[ f'(x)\sin x \right]_0^\pi - \int_0^\pi f'(x)\cos x\, dx \\
    &= 0-\left[ f(x)\cos x \right]_0^\pi - \int_0^\pi f(x)\sin x\, dx \\
    &= f(\pi) + f(0) - \int_0^\pi f(x)\sin x\, dx.
    \end{align*}
    则
    \[
    f(0)=  \int_0^\pi \left[f(x) + f''(x)\right]\sin x \, dx-f(\pi) = 3-2=1
    \]
    \end{solution}

    \question
    \[
    \int \frac{x}{1+\sin x} \, dx
    \]
    \begin{solution}
        发现
        \[
        \frac{1}{1+\sin x}
        = \frac{1-\sin x}{(1+\sin x)(1-\sin x)}
        = \frac{1 - \sin x}{\cos^2 x}
        = \sec^2 x - \sec x \tan x
        \]
        故
        \[
        \int \frac{1}{1 + \sin x} dx = \int \left( \sec^2 x - \sec x \tan x \right) dx = \tan x - \sec x + C
        \]
        由分部积分,设 \( u = x,\ dv = \frac{1}{1+\sin x} dx, \)
        \begin{align*}
            \int \frac{x}{1+\sin x} dx 
            &= x(\tan x - \sec x) - \int (\tan x - \sec x) dx \\
            &= x(\tan x - \sec x) + \ln|\cos x| - \ln|\sec x + \tan x| + C
        \end{align*}

    \end{solution}

    \question 设函数 $$f(x) = \int_1^x e^{-t^2} \, dt$$
    求
    \[
    \int_0^1 x f(x^2) \, dx
    \]
    \begin{solution}
        令 \( u = f(x^2) \),\( dv = x\, dx \)。  
        则有 \( du = f'(x^2) 2x\, dx = 2x e^{-x^4} dx , v = \frac{1}{2} x^2 \),于是分部积分得:
        \begin{align*}
        \int_0^1 x f(x^2)\, dx
        &= \left[ \frac{1}{2} x^2 f(x^2) \right]_0^1 - \int_0^1 \frac{1}{2} x^2 \cdot 2x e^{-x^4} dx \\
        &= - \int_0^1 x^3 e^{-x^4} dx \\
        &= \frac{1}{4} \left[ -e^{-x^4} \right]_0^1 \\
        &= \frac{1}{4}(1 - e^{-1}).
        \end{align*}
    \end{solution}
        
    \question
    \[
    \int \frac{(\ln x)^3}{x^2} \, dx
    \]
    \begin{solution}
        连续分部积分得
        \begin{align*}
        \int \frac{(\ln x)^3}{x^2}\, dx
        &= -\frac{1}{x}(\ln x)^3+\int \frac{1}{x} \cdot 3(\ln x)^2 \cdot \frac{1}{x}\,dx\\
        &= -\frac{1}{x}(\ln x)^3+3\int \frac{(\ln x)^2}{x^2}\,dx\\
        &= -\frac{1}{x}(\ln x)^3+3\left((\ln x)^{2}\cdot \left(-\frac{1}{x}\right)+\int \frac{1}{x}\cdot 2\ln x\cdot \frac{1}{x}dx\right) \\
        &= -\frac{1}{x}(\ln x)^3 -\frac{3}{x}(\ln x)^{2}+6 \int \frac{\ln x}{x^{2}}dx \\
        &= -\frac{1}{x}(\ln x)^3 -\frac{3}{x}(\ln x)^{2}+6 \left(\ln x \cdot \left(-\frac{1}{x}\right)+\int \frac{1}{x}\cdot\frac{1}{x} dx\right) \\
        &= -\frac{1}{x}(\ln x)^3 -\frac{3}{x}(\ln x)^{2} - \frac{6}{x} \ln x - \frac{6}{x}+C
        \end{align*}
    \end{solution}

    \question
求
\[
I = \int_{\alpha}^{\beta} \left(x^2 + \frac{1}{x^4}\right)^{-2} \, dx, \quad \alpha = 3^{-t}, \ \beta = 3^t
\]

\begin{solution}
首先化简被积式:
\[
\left(x^2 + \frac{1}{x^4}\right)^{-2} = \left(\frac{x^6 + 1}{x^4}\right)^{-2} = \left(\frac{x^4}{x^6 + 1}\right)^2 = \frac{x^8}{(x^6 + 1)^2}
\]

分部积分处理:
\[
\int \frac{x^8}{(x^6 + 1)^2} \, dx = \int x^3 \cdot \frac{x^5}{(x^6+1)^2} \, dx
\]

通过分部积分得到:
\[
\int_{\alpha}^{\beta} \frac{x^8}{(x^6 + 1)^2} \, dx = \left[-\frac{x^3}{6(x^6+1)}\right]_{\alpha}^{\beta} + \frac{1}{2} \int_{\alpha}^{\beta} \frac{x^2}{x^6+1} \, dx
\]

注意到
\[
\frac{d}{dx} \arctan(x^3) = \frac{3x^2}{1+x^6} \quad \Rightarrow \quad \frac{x^2}{1+x^6} = \frac{1}{3} \frac{d}{dx} \arctan(x^3)
\]

因此:
\[
\frac{1}{2} \int_{\alpha}^{\beta} \frac{x^2}{x^6+1} \, dx = \frac{1}{6} \left[ \arctan(x^3) \right]_{\alpha}^{\beta}
\]

最终积分结果为:
\[
I = \left[-\frac{x^3}{6(x^6+1)} + \frac{1}{6} \arctan(x^3) \right]_{\alpha}^{\beta}
\]

代入上下限 \(\alpha = 3^{-t}, \ \beta = 3^t\) 并计算得到:
\[
I = \frac{\pi}{36}
\]
\end{solution}

\question
求
\[
\int_{0}^{\infty} \frac{x^2+3x+3}{(x+1)^3} e^x \sin x \, dx
\]

\begin{solution}
Step 1: 部分分式分解

\[
\frac{x^2+3x+3}{(x+1)^3} = \frac{A}{x+1} + \frac{B}{(x+1)^2} + \frac{C}{(x+1)^3}
\]

\[
x^2 + 3x + 3 = A(x+1)^2 + B(x+1) + C
\]

\[
x^2 + 3x + 3 = A(x^2 + 2x + 1) + Bx + B + C = Ax^2 + (2A+B)x + (A+B+C)
\]

比较系数得到 \(A=1, B=1, C=1\),所以

\[
\frac{x^2+3x+3}{(x+1)^3} = \frac{1}{x+1} + \frac{1}{(x+1)^2} + \frac{1}{(x+1)^3}
\]

Step 2: 积分 \(\int e^x \sin x \, dx\)

分部积分两次:

\[
\int e^x \sin x \, dx = e^x \sin x - \int e^x \cos x \, dx = e^x \sin x - (e^x \cos x - \int e^x \sin x \, dx)
\]

\[
2 \int e^x \sin x \, dx = e^x (\sin x - \cos x) \implies \int e^x \sin x \, dx = \frac{1}{2} e^x (\sin x - \cos x)
\]

Step 3: 利用部分分式拆分积分

\[
\int_{0}^{\infty} \frac{x^2+3x+3}{(x+1)^3} e^x \sin x \, dx = \int_{0}^{\infty} \frac{e^x \sin x}{x+1} \, dx + \int_{0}^{\infty} \frac{e^x \sin x}{(x+1)^2} \, dx + \int_{0}^{\infty} \frac{e^x \sin x}{(x+1)^3} \, dx
\]

对第一项使用分部积分:

\[
u = \frac{1}{x+1}, \quad dv = e^x \sin x \, dx
\]

\[
du = -\frac{1}{(x+1)^2} dx, \quad v = \frac{1}{2} e^x (\sin x - \cos x)
\]

\[
\int_{0}^{\infty} \frac{e^x \sin x}{x+1} dx = \left[ \frac{1}{x+1} \cdot \frac{1}{2} e^x (\sin x - \cos x) \right]_{0}^{\infty} - \int_{0}^{\infty} \left(-\frac{1}{(x+1)^2}\right) \cdot \frac{1}{2} e^x (\sin x - \cos x) \, dx
\]

\[
= 0 + \int_{0}^{\infty} \frac{e^x (\sin x - \cos x)}{2(x+1)^2} \, dx
\]

剩余项相互抵消,得到最终结果:

\[
\int_{0}^{\infty} \frac{x^2+3x+3}{(x+1)^3} e^x \sin x \, dx = \frac{1}{2}
\]

\[
\frac{1}{2}
\]
\end{solution}

\question
计算积分
\[
J = \int_{0}^{1} \frac{x^2+1}{(x+1)^2} e^x \, dx.
\]

\begin{solution}
首先对有理函数部分作部分分式分解:
\[
\frac{x^2+1}{(x+1)^2} = A + \frac{B}{x+1} + \frac{C}{(x+1)^2}.
\]

比较分子:
\[
x^2+1 = A(x+1)^2 + B(x+1) + C = Ax^2 + (2A+B)x + (A+B+C).
\]

对比系数得到:
\[
A = 1, \quad 2A + B = 0 \implies B = -2, \quad A+B+C = 1 \implies C = 2.
\]

因此积分可写为:
\[
J = \int_{0}^{1} \left( e^x - \frac{2 e^x}{x+1} + \frac{2 e^x}{(x+1)^2} \right) \, dx
= \int_{0}^{1} e^x \, dx - \int_{0}^{1} \frac{2 e^x}{x+1} \, dx + \int_{0}^{1} \frac{2 e^x}{(x+1)^2} \, dx.
\]

对 $\int \frac{2 e^x}{x+1} \, dx$ 使用分部积分:
\[
\int \frac{2 e^x}{x+1} \, dx = \frac{2 e^x}{x+1} - \int \frac{2 e^x}{(x+1)^2} \, dx.
\]

代回原积分:
\[
J = \left[ e^x \right]_0^1 - \left( \left[ \frac{2 e^x}{x+1} \right]_0^1 - \int_{0}^{1} \frac{2 e^x}{(x+1)^2} dx \right) + \int_{0}^{1} \frac{2 e^x}{(x+1)^2} dx.
\]

两项 $\int_{0}^{1} \frac{2 e^x}{(x+1)^2} dx$ 相互抵消,得到:
\[
J = (e-1) - \left( \frac{2e}{2} - \frac{2}{1} \right) = (e-1) - (e-2) = 1.
\]
\end{solution}

\question
\[
\int \frac{[\ln(x^2+1)-2\ln x]\sqrt{x^2+1}}{x^4} \, dx
\]
\begin{solution}
\[
\int \frac{[\ln(x^2+1)-2\ln x]\sqrt{x^2+1}}{x^4} \, dx
= \int \ln\left(\frac{x^2+1}{x^2}\right) \frac{\sqrt{x^2+1}}{x^4} \, dx
= \int \ln\left(1+\frac{1}{x^2}\right) \frac{\sqrt{x^2+1}}{x^4} \, dx
\]

\[
= \int \sqrt{1+\frac{1}{x^2}} \ln\left(1+\frac{1}{x^2}\right) \frac{1}{x^3} \, dx
\]

令
\[
u = \sqrt{1+\frac{1}{x^2}}, \quad 2u \, du = -\frac{2}{x^3}\, dx \implies \frac{dx}{x^3} = -u \, du
\]

代入积分:
\[
\int -u^2 \ln(u^2) \, du = \int -2 u^2 \ln u \, du
\]

分部积分,取
\[
f = \ln u, \quad dg = -2 u^2 du \implies df = \frac{1}{u} du, \quad g = -\frac{2}{3} u^3
\]

\[
\int -2 u^2 \ln u \, du = -\frac{2}{3} u^3 \ln u + \frac{2}{9} u^3 + C
= \frac{2}{9} u^3 \left[ 1 - 3 \ln u \right]
= \frac{2}{9} u^3 \left[ 1 - 3 \ln(u^2) \right]
\]

代回 \(u = \sqrt{1+\frac{1}{x^2}}\):
\[
= \frac{2}{9} \left(1+\frac{1}{x^2}\right)^{3/2} \left[ 1 - 3 \ln \left(1+\frac{1}{x^2}\right) \right]
= \frac{2}{9 x^3} (x^2+1)^{3/2} \left[ 1 - 3 \ln \left(\frac{x^2+1}{x^2}\right) \right] + C
\]
\end{solution}

\question
6) 计算不定积分:\[ \int \cot^{-1}(x^2 + x + 1) dx \]

\begin{solution}
利用反余切恒等式:$\cot^{-1} \alpha - \cot^{-1} \beta = \cot^{-1} \frac{\alpha\beta+1}{\beta-\alpha}$。
令 $\alpha = x, \beta = x+1$,则有:
\[ \cot^{-1} x - \cot^{-1}(x+1) = \cot^{-1} \frac{x(x+1)+1}{(x+1)-x} = \cot^{-1}(x^2 + x + 1) \]

原积分拆分为两部分:
\[ I = \int \cot^{-1} x dx - \int \cot^{-1}(x+1) dx \]

对 $\int \cot^{-1} x dx$ 使用分部积分法:
\[ \int \cot^{-1} x dx = x \cot^{-1} x - \int x d(\cot^{-1} x) = x \cot^{-1} x + \int \frac{x}{1+x^2} dx \]
\[ = x \cot^{-1} x + \frac{1}{2} \ln(1+x^2) \]

同理,对 $\int \cot^{-1}(x+1) dx$ 有:
\[ \int \cot^{-1}(x+1) dx = (x+1) \cot^{-1}(x+1) + \frac{1}{2} \ln(1+(x+1)^2) \]

合并结果:
\[ I = x \cot^{-1} x - (x+1) \cot^{-1}(x+1) + \frac{1}{2} \ln\left| \frac{x^2+1}{x^2+2x+2} \right| + C \]

利用 $\cot^{-1} x = \tan^{-1} \frac{1}{x}$ 或笔记中的形式化简:
\[ I = x [\cot^{-1} x - \cot^{-1}(x+1)] + \frac{1}{2} \ln\left| \frac{x^2+1}{x^2+2x+2} \right| + \tan^{-1}(x+1) + C \]
最终结果为:
\[ I = x \cot^{-1}(x^2+x+1) + \frac{1}{2} \ln\left| \frac{x^2+1}{x^2+2x+2} \right| + \tan^{-1}(x+1) + C \]
\end{solution}

\question
计算积分
\[
\int_{\sqrt{e}}^{e} \left[ \ln(\ln x) + \frac{1}{(\ln x)^2} \right] \, dx.
\]

\begin{solution}
将积分拆开:
\[
\int_{\sqrt{e}}^{e} \ln(\ln x) + \frac{1}{(\ln x)^2} \, dx
= \int_{\sqrt{e}}^{e} \ln(\ln x) \, dx + \int_{\sqrt{e}}^{e} \frac{1}{(\ln x)^2} \, dx.
\]

\textbf{第一部分积分:} 使用分部积分法,令
\[
u = \ln(\ln x), \quad dv = dx \implies du = \frac{1}{x \ln x} dx, \quad v = x.
\]
于是:
\[
\int_{\sqrt{e}}^{e} \ln(\ln x) \, dx = \Big[ x \ln(\ln x) \Big]_{\sqrt{e}}^{e} - \int_{\sqrt{e}}^{e} \frac{x}{x \ln x} dx = \Big[ x \ln(\ln x) \Big]_{\sqrt{e}}^{e} - \int_{\sqrt{e}}^{e} \frac{1}{\ln x} \, dx.
\]

计算边界值:
\[
\Big[ x \ln(\ln x) \Big]_{\sqrt{e}}^{e} = e \ln(\ln e) - \sqrt{e} \ln(\ln \sqrt{e}) = e \ln 1 - \sqrt{e} \ln \frac{1}{2} = 0 - (-\sqrt{e} \ln 2) = \sqrt{e} \ln 2.
\]

因此:
\[
\int_{\sqrt{e}}^{e} \ln(\ln x) \, dx = \sqrt{e} \ln 2 - \int_{\sqrt{e}}^{e} \frac{1}{\ln x} \, dx.
\]

\textbf{第二部分积分:} 对 $\int \frac{1}{(\ln x)^2} dx$ 使用“反向分部积分”法,令
\[
u = \frac{1}{\ln x}, \quad dv = dx \implies du = -\frac{1}{x (\ln x)^2} dx, \quad v = x.
\]
于是:
\[
\int_{\sqrt{e}}^{e} \frac{1}{(\ln x)^2} \, dx = \int_{\sqrt{e}}^{e} \frac{1}{\ln x} \, dx - \Big[ \frac{x}{\ln x} \Big]_{\sqrt{e}}^{e}.
\]

计算边界值:
\[
\Big[ \frac{x}{\ln x} \Big]_{\sqrt{e}}^{e} = \frac{e}{\ln e} - \frac{\sqrt{e}}{\ln \sqrt{e}} = e - 2\sqrt{e}.
\]

因此:
\[
\int_{\sqrt{e}}^{e} \frac{1}{(\ln x)^2} \, dx = \int_{\sqrt{e}}^{e} \frac{1}{\ln x} \, dx - e + 2\sqrt{e}.
\]

\textbf{合并两部分积分:}
\[
\int_{\sqrt{e}}^{e} \left[ \ln(\ln x) + \frac{1}{(\ln x)^2} \right] dx
= \left( \sqrt{e} \ln 2 - \int_{\sqrt{e}}^{e} \frac{1}{\ln x} \, dx \right) + \left( \int_{\sqrt{e}}^{e} \frac{1}{\ln x} \, dx - e + 2\sqrt{e} \right)
= \sqrt{e} \ln 2 - e + 2\sqrt{e}.
\]
\end{solution}

    \question 
    \[
    \int \frac{\arctan x}{(1-x^2)^{\frac32}} \, dx
    \]
    \begin{solution}
        我们先回顾一个基本积分:
        \[
        \int \frac{1}{(1 - x^2)^{\frac{3}{2}}}\, dx
        \]
        令 \( x = \sin t \),则 \( dx = \cos t\, dt \),代入后有:
        \[
        \int \frac{dx}{(1 - x^2)^{\frac{3}{2}}} = \int \frac{\cos t\, dt}{(\cos^2 t)^{\frac{3}{2}}} = \int \frac{1}{\cos^2 t} dt = \tan t + C = \frac{x}{\sqrt{1 - x^2}} + C.
        \]
        回到原式,考虑分部积分,令:
        \[
        u = \arctan x, \quad dv = \frac{1}{(1 - x^2)^{\frac{3}{2}}} dx,
        \]
        则
        \[
        du = \frac{1}{1 + x^2} dx, \quad v = \frac{x}{\sqrt{1 - x^2}}.
        \]
        因此,原积分为:
        \[
        \int \frac{\arctan x}{(1 - x^2)^{\frac{3}{2}}}\, dx = \arctan x \cdot \frac{x}{\sqrt{1 - x^2}} - \int \frac{x}{(1 + x^2) \sqrt{1 - x^2}}\, dx.
        \]
        设最后一项为:
        \[
        I = \int \frac{x}{(1 + x^2) \sqrt{1 - x^2}}\, dx.
        \]
        令 \( x = \sin \theta \),则 \( dx = \cos \theta\, d\theta \),有:
        \[
        I = \int \frac{\sin \theta \cos \theta}{(1 + \sin^2 \theta)\cos \theta}\, d\theta = \int \frac{\sin \theta}{1 + \sin^2 \theta}\, d\theta.
        \]
        再设 \( u = \cos \theta \),则 \( d\theta = -\dfrac{du}{\sin \theta} \),代入得:
        \[
        I = -\int \frac{1}{2 - u^2}\, du.
        \]
        这是一个标准积分:
        \[
        \int \frac{1}{2 - u^2}\, du = \frac{1}{\sqrt{2}} \tanh^{-1} \left( \frac{u}{\sqrt{2}} \right) + C = \frac{1}{2\sqrt{2}} \ln \left| \frac{\sqrt{2} + u}{\sqrt{2} - u} \right| + C.
        \]
        回代 \( u = \cos \theta = \sqrt{1 - x^2} \),最终结果为:
        \[
        \int \frac{\arctan x}{(1 - x^2)^{\frac{3}{2}}}\, dx = \frac{x \arctan x}{\sqrt{1 - x^2}} + \frac{1}{2\sqrt{2}} \ln \left| \frac{\sqrt{2} + \sqrt{1 - x^2}}{\sqrt{2} - \sqrt{1 - x^2}} \right| + C.
        \]
    \end{solution}
    \question
计算 $[\int x^5 \sin^3 x^3 \cos x^3 dx]$。

\begin{solution}
令 $u = x^3$,则 $du = 3x^2 dx$。
\[ [\int x^5 \sin^3 x^3 \cos x^3 dx] = \frac{1}{3} [\int x^3 \sin^3 x^3 \cos x^3 (3x^2) dx] \]
\[ = \frac{1}{3} [\int u \sin^3 u \cos u du] = \frac{1}{3} [\int u \sin^3 u d(\sin u)] \]
令 $v = \sin u$,则 $dv = \cos u du$。
\[ = \frac{1}{3} [\int u v^3 dv] = \frac{1}{3} \cdot \frac{1}{4} [\int u d(v^4)] \]
使用分部积分法:
\[ = \frac{1}{12} (uv^4 - [\int v^4 du]) = \frac{1}{12} (u \sin^4 u - [\int \sin^4 u du]) \]
利用降幂公式 $\sin^2 A = \frac{1}{2}(1 - \cos 2A)$:
\[ = \frac{1}{12} (u \sin^4 u - [\int (\frac{1}{2}(1 - \cos 2u))^2 du]) \]
\[ = \frac{1}{12} (u \sin^4 u - \frac{1}{4} [\int (1 - 2\cos 2u + \cos^2 2u) du]) \]
继续利用 $\cos^2 2u = \frac{1 + \cos 4u}{2}$:
\[ = \frac{1}{12} u \sin^4 u - \frac{1}{48} ([\int 1 du] - 2 [\int \cos 2u du] + [\int \frac{1 + \cos 4u}{2} du]) \]
\[ = \frac{1}{12} u \sin^4 u - \frac{1}{48} u + \frac{1}{48} \sin 2u - \frac{1}{96} u - \frac{1}{384} \sin 4u + C \]
\[ = \frac{1}{12} u \sin^4 u - \frac{3}{96} u + \frac{1}{48} \sin 2u - \frac{1}{384} \sin 4u + C \]
最后回代 $u = x^3$:
\[ = \frac{1}{12} x^3 \sin^4 x^3 - \frac{1}{32} x^3 + \frac{1}{48} \sin 2x^3 - \frac{1}{384} \sin 4x^3 + C \]
\end{solution}

\question
计算 $[\int \frac{(\sqrt{x}+1)(x-1)}{\sqrt{x^2+2\sqrt{x^3}}} dx]$。

\begin{solution}
令 $u = \sqrt{x}$,则 $du = \frac{1}{2\sqrt{x}} dx \implies dx = 2u du$。
\[ [\int \frac{(u+1)(u^2-1)}{\sqrt{u^4+2u^3}} 2u du] = [\int \frac{2u(u+1)^2(u-1)}{u\sqrt{u^2+2u}} du] \]
\[ = [\int \frac{2(u+1)^2(u-1)}{\sqrt{u^2+2u}} du] = [\int \frac{2(u+1)^2(u-1)}{\sqrt{(u+1)^2-1}} du] \]
令 $u+1 = \sec\theta$,则 $du = \sec\theta \tan\theta d\theta$。
\[ [\int \frac{2\sec^2\theta(\sec\theta-2)}{\sqrt{\sec^2\theta-1}} \sec\theta \tan\theta d\theta] = [\int \frac{2\sec^3\theta(\sec\theta-2)}{\tan\theta} \tan\theta d\theta] \]
\[ = 2 [\int \sec^4\theta d\theta] - 4 [\int \sec^3\theta d\theta] \]
计算第一个积分:
\[ 2 [\int (1+\tan^2\theta) d(\tan\theta)] = 2\tan\theta + \frac{2}{3}\tan^3\theta + C \]
计算第二个积分(分部积分):
\[ 4 [\int \sec^3\theta d\theta] = 2\tan\theta\sec\theta + 2\ln|\sec\theta + \tan\theta| + C \]
合并结果并回代 $\theta$:
\[ = 2\tan\theta + \frac{2}{3}\tan^3\theta - 2\tan\theta\sec\theta - 2\ln|\sec\theta + \tan\theta| + C \]
利用 $\sec\theta = u+1$ 且 $\tan\theta = \sqrt{(u+1)^2-1} = \sqrt{u^2+2u}$:
\[ = 2\sqrt{u^2+2u} + \frac{2}{3}(u^2+2u)^{\frac{3}{2}} - 2\sqrt{u^2+2u}(u+1) - 2\ln|u+1+\sqrt{u^2+2u}| + C \]
回代 $u = \sqrt{x}$ 并进一步化简:
\[ = \frac{2}{3}\sqrt{x+2\sqrt{x}}(x-\sqrt{x}) - 2\ln|\sqrt{x}+1+\sqrt{x+2\sqrt{x}}| + C \]
\[ = \frac{2}{3}\sqrt{x}(\sqrt{x}-1)\sqrt{x+2\sqrt{x}} - 2\ln|\sqrt{x}+1+\sqrt{x+2\sqrt{x}}| + C \]
\end{solution}
    \question 证明
\[
\int_{0}^{\infty} \frac{x^2+3x+3}{(x+1)^3} e^{-x}\sin x \, dx = \frac{1}{2}.
\]

\begin{solution}
先作部分分式分解:
\[
\frac{x^2+3x+3}{(x+1)^3}
= \frac{A}{x+1}+\frac{B}{(x+1)^2}+\frac{C}{(x+1)^3}.
\]
即
\[
x^2+3x+3=A(x+1)^2+B(x+1)+C
=Ax^2+(2A+B)x+(A+B+C).
\]
比较系数得
\[
A=1,\quad B=1,\quad C=1.
\]

因此
\[
\frac{x^2+3x+3}{(x+1)^3}
=\frac{1}{x+1}+\frac{1}{(x+1)^2}+\frac{1}{(x+1)^3}.
\]

先计算基本积分。分部积分两次得
\[
\int e^{-x}\sin x\,dx
=-\frac12 e^{-x}(\cos x+\sin x)+C.
\]

将原积分拆分为
\[
\int_{0}^{\infty}\frac{e^{-x}\sin x}{x+1}\,dx
-2\int_{0}^{\infty}\frac{e^{-x}\sin x}{(x+1)^2}\,dx
+\int_{0}^{\infty}\frac{e^{-x}\sin x}{(x+1)^3}\,dx.
\]

对第一项与第三项作分部积分,第二项保留用于抵消,整理得
\[
\int_{0}^{\infty}\frac{x^2+3x+3}{(x+1)^3}e^{-x}\sin x\,dx
=\frac12.
\]

\end{solution}

\question
11) 计算不定积分:\[ I = \int e^{3u} \sqrt{1 + e^{2u}} du \]

\begin{solution}
\textbf{第一步:变量代换}
设 $e^u = \tan x$,则 $e^u du = \sec^2 x dx$。
同时有 $e^{2u} = \tan^2 x$,从而根式部分为 $\sqrt{1 + \tan^2 x} = \sec x$。
代入原积分式:
\begin{align*}
I &= \int (e^u)^2 \sqrt{1 + (e^u)^2} (e^u du) \\
&= \int \tan^2 x \cdot \sec x \cdot \sec^2 x dx \\
&= \int \tan^2 x \sec^3 x dx
\end{align*}

\textbf{第二步:利用恒等式转化}
利用 $\tan^2 x = \sec^2 x - 1$:
\[ I = \int (\sec^2 x - 1) \sec^3 x dx = \int \sec^5 x dx - \int \sec^3 x dx \]

\textbf{第三步:分部积分计算}
根据笔记中的推导结果(或利用 $\sec^n x$ 的递推公式):
\[ \int \sec^3 x dx = \frac{1}{2} (\sec x \tan x + \ln|\sec x + \tan x|) + C \]
\[ \int \sec^5 x dx = \frac{1}{4} \sec^3 x \tan x + \frac{3}{8} \sec x \tan x + \frac{3}{8} \ln|\sec x + \tan x| + C \]

合并后得到关于 $x$ 的结果:
\[ I = \frac{1}{4} \tan x \sec^3 x - \frac{1}{8} \sec x \tan x - \frac{1}{8} \ln|\sec x + \tan x| + C \]

\textbf{第四步:还原变量}
回代 $\tan x = e^u$ 且 $\sec x = \sqrt{1 + e^{2u}}$:
\begin{align*}
I &= \frac{1}{4} e^u (1+e^{2u})^{\frac{3}{2}} - \frac{1}{8} e^u \sqrt{1+e^{2u}} - \frac{1}{8} \ln(e^u + \sqrt{1+e^{2u}}) + C \\
&= \frac{1}{8} e^u \sqrt{1+e^{2u}} (2(1+e^{2u}) - 1) - \frac{1}{8} \ln(e^u + \sqrt{1+e^{2u}}) + C \\
&= \frac{1}{8} e^u (2e^{2u} + 1) \sqrt{1+e^{2u}} - \frac{1}{8} \ln(e^u + \sqrt{1+e^{2u}}) + C
\end{align*}
\end{solution}

\question
25) 计算不定积分:\[ \int \csc^2 x \ln(\cos x + \sqrt{\cos 2x}) dx \]

\begin{solution}
使用分部积分法,令 $u = \ln(\cos x + \sqrt{\cos 2x}), dv = \csc^2 x dx$。
则 $du = \frac{-\sin x - \frac{\sin 2x}{\sqrt{\cos 2x}}}{\cos x + \sqrt{\cos 2x}} dx, v = -\cot x$。
代入分部积分公式:
\[ I = -\cot x \ln(\cos x + \sqrt{\cos 2x}) + \int \cot x \cdot \frac{\sin x + \frac{\sin 2x}{\sqrt{\cos 2x}}}{\cos x + \sqrt{\cos 2x}} dx \]
简化积分项中的分母与分子:
\[ \dots = -\cot x \ln(\cos x + \sqrt{\cos 2x}) - \int \frac{\cos x}{\sin^2 x \sqrt{\cos 2x}} dx + \int \frac{\cos^2 x}{\sin^2 x} dx \]
最终通过三角换元计算得到:
\[ I = -\cot x \ln(\cos x + \sqrt{\cos 2x}) + \frac{\sqrt{\cos 2x}}{\sin x} - x + C \]
\end{solution}
% reduction formula
\question
已知
\[
I_n=\int_0^1 \frac{x^{2n+1}}{\sqrt{1-x^2}}\,dx
\]

证明
\[
I_n=\frac{2^{2n}(n!)^2}{(2n+1)!}
\]
reduction formula
\begin{solution}
先对积分作代换
\[
u=1-x^2,\quad du=-2x\,dx
\]

当 $x=0$ 时 $u=1$,当 $x=1$ 时 $u=0$,于是
\[
I_n=\int_0^1 x^{2n+1}(1-x^2)^{-\frac12}dx
=\frac12\int_0^1 (1-u)^n u^{-\frac12}du
\]

对积分
\[
\int_0^1 (1-u)^n u^{-\frac12}du
\]
作分部积分,取
\[
f=(1-u)^n,\quad dg=u^{-\frac12}du
\]
则
\[
df=-n(1-u)^{n-1}du,\quad g=2u^{\frac12}
\]

于是
\[
\int_0^1 (1-u)^n u^{-\frac12}du
=2n\int_0^1 (1-u)^{n-1}u^{\frac12}du
\]

从而
\[
I_n=n\int_0^1 (1-u)^{n-1}u^{\frac12}du
\]

再注意到
\[
I_{n-1}=\frac12\int_0^1 (1-u)^{n-1}u^{-\frac12}du
\]

而
\[
\int_0^1 (1-u)^{n-1}u^{\frac12}du
=\frac{1}{2n+1}\int_0^1 (1-u)^{n-1}u^{-\frac12}du
\]

因此得到递推关系
\[
I_n=\frac{2n}{2n+1}I_{n-1}
\]

不断递推可得
\[
I_n=\frac{2n}{2n+1}\cdot\frac{2n-2}{2n-1}\cdots\frac{2}{3}I_0
\]

又
\[
I_0=\int_0^1\frac{x}{\sqrt{1-x^2}}dx=1
\]

于是
\[
I_n=\frac{2n\cdot(2n-2)\cdots2}{(2n+1)(2n-1)\cdots3}
=\frac{(2n)!!}{(2n+1)!!}
\]

利用
\[
(2n)!!=2^n n!,\quad (2n+1)!!=\frac{(2n+1)!}{2^n n!}
\]

得到
\[
I_n=\frac{2^{2n}(n!)^2}{(2n+1)!}
\]

证毕
\end{solution}

 \question 设
\[
I(m,n)= \int_{a}^{b} (b-x)^{m}(x-a)^{n} \,dx, \quad m\in\mathbb{N},\ n\in\mathbb{N},\ b>a
\]
证明
\[
I(m,n) = \frac{m!n!}{(m+n+1)!}(b-a)^{m+n+1}
\]
并据此计算
\[
\int_{0}^{1} (1-x^2)^n \,dx
\]

\begin{solution}
先对 $I(m,n)$ 作分部积分。取
\[
u=(b-x)^m,\quad dv=(x-a)^n dx
\]
则
\[
du=-m(b-x)^{m-1}dx,\quad v=\frac{(x-a)^{n+1}}{n+1}
\]

于是
\[
I(m,n)=\left[\frac{(b-x)^m(x-a)^{n+1}}{n+1}\right]_a^b
-\int_a^b \frac{-m(b-x)^{m-1}(x-a)^{n+1}}{n+1}\,dx
\]
注意到端点项为零,得到
\[
I(m,n)=\frac{m}{n+1}I(m-1,n+1)
\]

重复使用该递推关系,
\[
I(m,n)=\frac{m}{n+1}\cdot\frac{m-1}{n+2}\cdots\frac{1}{n+m}I(0,n+m)
\]
即
\[
I(m,n)=\frac{m!}{(n+1)(n+2)\cdots(n+m)}I(0,n+m)
\]
又
\[
I(0,n+m)=\int_a^b (x-a)^{n+m}\,dx
=\left[\frac{(x-a)^{n+m+1}}{n+m+1}\right]_a^b
=\frac{(b-a)^{n+m+1}}{n+m+1}
\]
因此
\[
I(m,n)=\frac{m!n!}{(m+n+1)!}(b-a)^{m+n+1}
\]

接下来计算
\[
\int_{0}^{1}(1-x^2)^n\,dx
\]
利用偶函数对称性,
\[
\int_{0}^{1}(1-x^2)^n\,dx=\frac{1}{2}\int_{-1}^{1}(1-x^2)^n\,dx
\]
而
\[
1-x^2=(1-x)(x+1)
\]
于是
\[
\int_{-1}^{1}(1-x^2)^n\,dx=\int_{-1}^{1}(1-x)^n(x-(-1))^n\,dx
\]
这正是 $I(n,n)$ 的形式,其中 $a=-1,\ b=1$。由已证公式,
\[
I(n,n)=\frac{n!n!}{(2n+1)!}(1-(-1))^{2n+1}
=\frac{n!^2}{(2n+1)!}2^{2n+1}
\]
因此
\[
\int_{0}^{1}(1-x^2)^n\,dx
=\frac{1}{2}I(n,n)
=\frac{2^{2n}(n!)^2}{(2n+1)!}
\]
\end{solution}



    \question 设数列 $\{a_n\}$ 满足
    \[
    a_n = \int_{0}^{1}(1-x^2)^{\frac{n}{2}}\,dx, \quad n=0, 1, 2, 3, \dots
    \]
    \begin{parts}
    \part 证明:$a_n = \dfrac{n}{n+1}a_{n-2}, \quad n \ge 2$。
    \begin{solution}
        令 $x=\sin\theta$,则 $dx = \cos \theta \, d\theta$,有
        \[
        a_n = \int_0^1 (1-x^2)^{\frac{n}{2}} \, dx
        = \int_0^{\frac{\pi}{2}} (\cos^2\theta)^{\frac{n}{2}} \cdot \cos \theta \, d\theta
        = \int_0^{\frac{\pi}{2}} \cos^{n+1}\theta \, d\theta
        \]
        分部积分:
        \[
        a_n = \left[ \sin\theta \cos^n\theta \right]_0^{\frac{\pi}{2}}
        + n \int_0^{\frac{\pi}{2}} \sin^2\theta \cos^{n-1}\theta \, d\theta
        \]
        第一项为 $0$,因此
        \[
        a_n = n\int_0^{\frac{\pi}{2}} (1-\cos^2\theta) \cos^{n-1}\theta \, d\theta
        = n\int_0^{\frac{\pi}{2}} \cos^{n-1}\theta \, d\theta
        - n\int_0^{\frac{\pi}{2}} \cos^{n+1}\theta \, d\theta
        \]
        即
        \[
        a_n = na_{n-2} - n a_n \Rightarrow a_n = \frac{n}{n+1} a_{n-2}
        \]
    \end{solution}
    \part 求 $\displaystyle\lim_{n\to\infty} \frac{a_{n+1}}{a_n}$ 的值。
    \begin{solution}
        由定义
        \[
        a_n = \int_0^{\frac{\pi}{2}} \cos^{n+1}\theta \, d\theta
        \]
        可知 $a_n \ge a_{n+1}$,于是
        \[
        \frac{a_{n+2}}{a_n} \le \frac{a_{n+1}}{a_n} \le 1
        \]
        又由递推公式
        \[
        \lim_{n\to\infty} \frac{a_{n+2}}{a_n} = \lim_{n\to\infty} \frac{n+2}{n+3} = 1
        \]
        由夹挤定理,
        \[
        \lim_{n\to\infty} \frac{a_{n+1}}{a_n} = 1
        \]
    \end{solution}
    \end{parts}

    \question
已知
\[
I_{n}=\int_{0}^{a}x^{n+\frac{1}{2}}\sqrt{a-x}\,dx,\quad n\in\mathbb{Z},\ n\ge 0
\]
其中 $a$ 为正常数。

\begin{solution}
(a) 由分部积分法,
取
\[
u=x^{n+\frac{1}{2}},\quad dv=\sqrt{a-x}\,dx
\]
则
\[
du=\left(n+\frac{1}{2}\right)x^{n-\frac{1}{2}}dx,\quad v=-\frac{2}{3}(a-x)^{\frac{3}{2}}
\]

于是
\[
I_n=\left[-\frac{2}{3}x^{n+\frac{1}{2}}(a-x)^{\frac{3}{2}}\right]_0^a
+\frac{2}{3}\left(n+\frac{1}{2}\right)\int_0^a x^{n-\frac{1}{2}}(a-x)^{\frac{3}{2}}dx
\]

边界项为零,因此
\[
I_n=\frac{2}{3}\left(n+\frac{1}{2}\right)\int_0^a x^{n-\frac{1}{2}}(a-x)^{\frac{1}{2}}(a-x)\,dx
\]

展开得
\[
I_n=\frac{2}{3}\left(n+\frac{1}{2}\right)
\left[
a\int_0^a x^{n-\frac{1}{2}}(a-x)^{\frac{1}{2}}dx
-\int_0^a x^{n+\frac{1}{2}}(a-x)^{\frac{1}{2}}dx
\right]
\]

即
\[
I_n=\frac{2}{3}\left(n+\frac{1}{2}\right)\left(aI_{n-1}-I_n\right)
\]

整理得
\[
(2n+4)I_n=(2n+1)aI_{n-1}
\]

从而
\[
I_n=\frac{(2n+1)a}{2n+4}I_{n-1}
\]

反复递推,
\[
I_n=a^n\frac{(2n+1)(2n-1)\cdots3}{2^n(n+2)(n+1)\cdots3}I_0
\]

利用组合数恒等式,化简得
\[
I_n=\left(\frac{a}{4}\right)^n\binom{2n+2}{n}\frac{I_0}{n+1}
\]

(b) 计算
\[
\int_0^2 x^{10}\sqrt{4-x^2}\,dx
\]

令 $u=x^2$,则
\[
dx=\frac{1}{2}u^{-\frac{1}{2}}du
\]

原积分化为
\[
\int_0^2 x^{10}\sqrt{4-x^2}\,dx
=\frac{1}{2}\int_0^4 u^{4+\frac{1}{2}}\sqrt{4-u}\,du
=\frac{1}{2}I_4
\]

由(a)式,取 $a=4$,
\[
I_4=\left(\frac{4}{4}\right)^4\binom{10}{4}\frac{I_0}{5}
\]

又
\[
I_0=\int_0^4\sqrt{x}\sqrt{4-x}\,dx=2\pi
\]

代入得
\[
I_4=84\pi
\]

因此
\[
\int_0^2 x^{10}\sqrt{4-x^2}\,dx=\frac{1}{2}I_4=42\pi
\]
\end{solution}


    \question Evaluate the integral $\int (1-x^2)^{\frac{n}{2}}\,dx$ using reduction formulae.

\begin{solution}
(a) Show that 
\[
\int \sqrt{1-x^2}\,dx = \frac{1}{2}\sin^{-1}x + \frac{1}{2}x\sqrt{1-x^2} + C.
\]

令 $x = \sin\theta,\theta \in [-\pi/2, \pi/2]$,则 $dx = \cos\theta\,d\theta$:
\[
\int \sqrt{1-\sin^2\theta}\,\cos\theta\,d\theta = \int \cos^2\theta\,d\theta.
\]

使用恒等式 $\cos^2\theta = \frac{1+\cos 2\theta}{2}$:
\begin{align*}
\int \cos^2\theta\,d\theta &= \int \frac{1+\cos 2\theta}{2}\,d\theta 
= \frac{1}{2}\int (1+\cos 2\theta)\,d\theta \\
&= \frac{1}{2}\left(\theta + \frac{\sin 2\theta}{2}\right) + C 
= \frac{1}{2}\theta + \frac{1}{4}(2\sin\theta\cos\theta) + C \\
&= \frac{1}{2}\sin^{-1}x + \frac{1}{2}x\sqrt{1-x^2} + C.
\end{align*}

(b) Show that for any positive integer $n$,
\[
\frac{d}{dx}\big[x(1-x^2)^{\frac{n}{2}}\big] = (n+1)(1-x^2)^{\frac{n}{2}} - n(1-x^2)^{\frac{n-2}{2}}.
\]

使用乘法法则:
\begin{align*}
\frac{d}{dx}[x(1-x^2)^{\frac{n}{2}}] &= (1-x^2)^{\frac{n}{2}} + x \cdot \frac{n}{2}(1-x^2)^{\frac{n}{2}-1}(-2x) \\
&= (1-x^2)^{\frac{n}{2}} - nx^2 (1-x^2)^{\frac{n-2}{2}} \\
&= (1-x^2)^{\frac{n-2}{2}}\left((1-x^2) - nx^2\right) \\
&= (1-x^2)^{\frac{n-2}{2}}\left(1-(n+1)x^2\right) \\
&= (n+1)(1-x^2)^{\frac{n}{2}} - n(1-x^2)^{\frac{n-2}{2}}.
\end{align*}

(c) Reduction formula. Let $I_n = \int (1-x^2)^{\frac{n}{2}}\,dx$. Then
\[
\frac{d}{dx}[x(1-x^2)^{\frac{n}{2}}] = (n+1)(1-x^2)^{\frac{n}{2}} - n(1-x^2)^{\frac{n-2}{2}}.
\]

积分两边得:
\[
x(1-x^2)^{\frac{n}{2}} = (n+1) I_n - n I_{n-2} \implies I_n = \frac{n}{n+1} I_{n-2} + \frac{1}{n+1}x(1-x^2)^{\frac{n}{2}}.
\]

(d) Compute $\displaystyle \int (1-x^2)^{\frac{5}{2}}\,dx$.

利用归纳公式:
\[
I_5 = \frac{5}{6} I_3 + \frac{1}{6} x(1-x^2)^{\frac{5}{2}}, \quad
I_3 = \frac{3}{4} I_1 + \frac{1}{4} x(1-x^2)^{\frac{3}{2}}, \quad
I_1 = \int (1-x^2)^{1/2}\,dx = \frac{1}{2}\sin^{-1}x + \frac{1}{2}x\sqrt{1-x^2}.
\]

代回得:
\begin{align*}
I_5 &= \frac{5}{6}\left(\frac{3}{4}I_1 + \frac{1}{4} x(1-x^2)^{3/2}\right) + \frac{1}{6} x(1-x^2)^{5/2} \\
&= \frac{15}{24} I_1 + \frac{5}{24} x(1-x^2)^{3/2} + \frac{1}{6} x(1-x^2)^{5/2} \\
&= \frac{5}{8} \left( \frac{1}{2}\sin^{-1}x + \frac{1}{2} x\sqrt{1-x^2} \right) 
+ \frac{5}{24} x(1-x^2)^{3/2} + \frac{1}{6} x(1-x^2)^{5/2} + C \\
&= \frac{5}{16}\sin^{-1}x + \frac{5}{16} x\sqrt{1-x^2} + \frac{5}{24} x(1-x^2)^{3/2} + \frac{1}{6} x(1-x^2)^{5/2} + C.
\end{align*}
\end{solution}

\question 计算积分 $\displaystyle \int \frac{x^4}{\sqrt{x^2+1}}\,dx$,使用降次公式。

\begin{solution}
(a) 证明对任意整数 $n\ge 2$ 有
\[
\frac{d}{dx}\left[\frac{x^{n-1}}{\sqrt{x^2+1}}\right] = \frac{(2-n)x^{n-2} + (1-n)x^n}{(x^2+1)^{3/2}}.
\]

\begin{align*}
\frac{d}{dx}\left[\frac{x^{n-1}}{\sqrt{x^2+1}}\right] 
&= \frac{d}{dx}\left[x^{n-1}(x^2+1)^{-1/2}\right] \\
&= (n-1)x^{n-2}(x^2+1)^{-1/2} + x^{n-1}\left(-\frac{1}{2}(x^2+1)^{-3/2}(2x)\right) \\
&= \frac{(n-1)x^{n-2}}{\sqrt{x^2+1}} - \frac{x^n}{(x^2+1)^{3/2}} \\
&= \frac{(2-n)x^{n-2} + (1-n)x^n}{(x^2+1)^{3/2}}.
\end{align*}

(b) 设 $I_n = \int \frac{x^n}{\sqrt{x^2+1}}\,dx$。利用(a)得:
\[
(2-n) I_{n-2} + (1-n) I_n = \frac{x^{n-1}}{\sqrt{x^2+1}} \quad \Rightarrow \quad I_n = \frac{x^{n-1}\sqrt{x^2+1}}{1-n} + \frac{n-2}{1-n} I_{n-2}.
\]

(c) 使用降次公式计算 $I_4 = \int \frac{x^4}{\sqrt{x^2+1}}\,dx$:

\[
I_4 = \frac{x^3\sqrt{x^2+1}}{1-4} + \frac{4-2}{1-4} I_2 = -\frac{x^3\sqrt{x^2+1}}{3} - \frac{2}{3} I_2.
\]

类似地计算 $I_2$:
\[
I_2 = \frac{x\sqrt{x^2+1}}{1-2} + \frac{2-2}{1-2} I_0 = -x\sqrt{x^2+1}.
\]

将 $I_2$ 代回 $I_4$:
\begin{align*}
I_4 &= -\frac{x^3\sqrt{x^2+1}}{3} - \frac{2}{3}(-x\sqrt{x^2+1}) \\
&= -\frac{x^3\sqrt{x^2+1}}{3} + \frac{2x\sqrt{x^2+1}}{3} + C.
\end{align*}

\end{solution}
\question 计算积分
\[
I_n = \int_{0}^{a} \frac{x^n}{\sqrt{a^2-x^2}} \, dx, \quad n\in \mathbb{N}, \quad a>0
\]

\begin{solution}
\textbf{a) 推导递推公式}

使用分部积分,设
\[
u = x^{n-1}, \quad dv = \frac{x}{\sqrt{a^2-x^2}} \, dx \implies v = -\sqrt{a^2-x^2}, \quad du = (n-1)x^{n-2} \, dx
\]

\begin{align*}
I_n &= \int_0^a \frac{x^n}{\sqrt{a^2-x^2}} \, dx
= \int_0^a x^{n-1} \cdot \frac{x}{\sqrt{a^2-x^2}} \, dx \\
&= \left[-x^{n-1} \sqrt{a^2-x^2}\right]_0^a + (n-1) \int_0^a x^{n-2} \sqrt{a^2-x^2} \, dx \\
&= (n-1) \int_0^a \frac{x^{n-2}(a^2-x^2)}{\sqrt{a^2-x^2}} \, dx \\
&= (n-1) \left[a^2 \int_0^a \frac{x^{n-2}}{\sqrt{a^2-x^2}} \, dx - \int_0^a \frac{x^n}{\sqrt{a^2-x^2}} \, dx \right] \\
&= (n-1)a^2 I_{n-2} - (n-1) I_n \\
&\implies n I_n = a^2 (n-1) I_{n-2}, \quad n\ge 2
\end{align*}

\textbf{b) 计算具体积分}

\[
\int_2^4 \frac{3x^3-18x^2+36x-18}{\sqrt{4x-x^2}} \, dx
\]

首先配方:
\[
4x - x^2 = -(x^2-4x) = 4 - (x-2)^2
\]
令 \(u = x-2 \implies du=dx\),积分上下限:
\[
x=2 \implies u=0, \quad x=4 \implies u=2
\]

分解被积函数:
\begin{align*}
3x^3-18x^2+36x-18 &= 3[(x-2)^3 + 2] \\
\implies \int_2^4 \frac{3x^3-18x^2+36x-18}{\sqrt{4x-x^2}} \, dx 
&= \int_0^2 \frac{3u^3 + 6}{\sqrt{4-u^2}} \, du \\
&= 3 \int_0^2 \frac{u^3}{\sqrt{4-u^2}} \, du + \int_0^2 \frac{6}{\sqrt{4-u^2}} \, du \\
&= 3 I_3 + 6 \arcsin\left(\frac{u}{2}\right)\Big|_0^2 \\
&= 3 I_3 + 6 \cdot \frac{\pi}{2} = 3 I_3 + 3\pi
\end{align*}

使用递推公式 \(I_n = \frac{4(n-1)}{n} I_{n-2}\):
\[
I_3 = \frac{4(3-1)}{3} I_1 = \frac{8}{3} I_1
\]

计算 \(I_1\):
\[
I_1 = \int_0^2 \frac{u}{\sqrt{4-u^2}} \, du = \left[-\sqrt{4-u^2}\right]_0^2 = 0 - (-2) = 2
\]

\(\implies I_3 = \frac{8}{3} \cdot 2 = \frac{16}{3}\)

最终结果:
\[
\int_2^4 \frac{3x^3-18x^2+36x-18}{\sqrt{4x-x^2}} \, dx = 3 \cdot \frac{16}{3} + 3\pi = 16 + 3\pi
\]
\end{solution}

\question
已知
\[
I_n = \int_0^{\pi} \frac{\sin(n\theta)}{\sin\theta}\,d\theta
\]
其中 \( n \) 为正整数。

(a) 利用三角恒等式证明
\[
\frac{\sin(n\theta)-\sin[(n-2)\theta]}{\sin\theta} = 2\cos[(n-1)\theta]
\]

(b) 推导
\[
I_n = I_{n-2},\quad n\ge2
\]

(c) 求 \( I_n \) 的值,分别讨论 \( n \) 为奇数或偶数的情况

\begin{solution}
(a) 使用公式 \(\sin A - \sin B = 2\cos\frac{A+B}{2}\sin\frac{A-B}{2}\):
\begin{align*}
\frac{\sin(n\theta)-\sin[(n-2)\theta]}{\sin\theta} &= \frac{2\cos\frac{n\theta+(n-2)\theta}{2}\sin\frac{n\theta-(n-2)\theta}{2}}{\sin\theta} \\
&= \frac{2\cos[(n-1)\theta]\sin\theta}{\sin\theta} \\
&= 2\cos[(n-1)\theta]
\end{align*}

(b) 利用(a)结果:
\begin{align*}
I_n - I_{n-2} &= \int_0^{\pi} \frac{\sin(n\theta)-\sin[(n-2)\theta]}{\sin\theta}\,d\theta \\
&= \int_0^{\pi} 2\cos[(n-1)\theta]\,d\theta \\
&= \left[\frac{2}{n-1}\sin((n-1)\theta)\right]_0^{\pi} = 0 \\
\therefore I_n &= I_{n-2},\quad n\ge2
\end{align*}

(c) 由(b)得递推关系:
\[
I_n = I_{n-2} = \dots
\]

若 \( n \) 为偶数,则递推到 \( I_0 \):
\[
I_0 = \int_0^{\pi} \frac{\sin 0}{\sin\theta}\,d\theta = 0
\]

若 \( n \) 为奇数,则递推到 \( I_1 \):
\[
I_1 = \int_0^{\pi} \frac{\sin\theta}{\sin\theta}\,d\theta = \int_0^{\pi} 1\,d\theta = \pi
\]

因此
\[
I_n = 
\begin{cases} 
0 & n \text{ 为偶数} \\[2mm]
\pi & n \text{ 为奇数}
\end{cases}
\]
\end{solution}

\question
已知
\[
I_n = \int \frac{\sin(nx)}{\sin x}\,dx, \quad n\in\mathbb{N}.
\]

(a) 证明对 \( n\ge0 \) 有
\[
I_{n+2} = I_n + \frac{2}{n+1}\sin[(n+1)x] + C
\]

(b) 利用(a)的递推关系,求
\[
\int_{\frac{\pi}{4}}^{\frac{\pi}{3}} \frac{\sin 6x}{\sin x}\,dx
\]

\begin{solution}
(a) 考虑
\begin{align*}
I_{n+2} - I_n &= \int \frac{\sin[(n+2)x]}{\sin x}\,dx - \int \frac{\sin(nx)}{\sin x}\,dx \\
&= \int \frac{\sin[(n+2)x]-\sin(nx]}{\sin x}\,dx
\end{align*}

利用恒等式 \(\sin P - \sin Q = 2\cos\frac{P+Q}{2}\sin\frac{P-Q}{2}\):
\begin{align*}
I_{n+2}-I_n &= \int \frac{2\cos\frac{(n+2)x+nx}{2} \sin\frac{(n+2)x-nx}{2}}{\sin x}\,dx \\
&= \int \frac{2\cos[(n+1)x]\sin x}{\sin x}\,dx \\
&= \int 2\cos[(n+1)x]\,dx \\
&= \frac{2}{n+1}\sin[(n+1)x] + C
\end{align*}

因此
\[
I_{n+2} = I_n + \frac{2}{n+1}\sin[(n+1)x] + C
\]

(b) 对定积分,递推关系不需考虑常数 \(C\):
\begin{align*}
\int_{\frac{\pi}{4}}^{\frac{\pi}{3}} \frac{\sin 6x}{\sin x}\,dx &= I_6\bigg|_{\frac{\pi}{4}}^{\frac{\pi}{3}} \\
&= I_4\bigg|_{\frac{\pi}{4}}^{\frac{\pi}{3}} + \left[\frac{2}{5}\sin 5x\right]_{\frac{\pi}{4}}^{\frac{\pi}{3}} \\
&= I_4 + \frac{2}{5}\left(\sin\frac{5\pi}{3} - \sin\frac{5\pi}{4}\right) \\
&= I_4 + \frac{2}{5}\left(-\frac{\sqrt{3}}{2} - \left(-\frac{\sqrt{2}}{2}\right)\right) \\
&= I_4 - \frac{\sqrt{3}}{5} + \frac{\sqrt{2}}{5}
\end{align*}

同理,
\begin{align*}
I_4 &= I_2 + \left[\frac{2}{3}\sin 3x\right]_{\frac{\pi}{4}}^{\frac{\pi}{3}} \\
&= I_2 + \frac{2}{3}\left(\sin\pi - \sin\frac{3\pi}{4}\right) \\
&= I_2 - \frac{\sqrt{2}}{3}
\end{align*}

再有
\begin{align*}
I_2 &= I_0 + \left[2\sin x\right]_{\frac{\pi}{4}}^{\frac{\pi}{3}} \\
&= 0 + 2\left(\frac{\sqrt{3}}{2} - \frac{\sqrt{2}}{2}\right) \\
&= \sqrt{3} - \sqrt{2}
\end{align*}

最终得到
\begin{align*}
\int_{\frac{\pi}{4}}^{\frac{\pi}{3}} \frac{\sin 6x}{\sin x}\,dx &= (\sqrt{3}-\sqrt{2} - \frac{\sqrt{2}}{3}) - \frac{\sqrt{3}}{5} + \frac{\sqrt{2}}{5} \\
&= \frac{12\sqrt{3}}{15} - \frac{17\sqrt{2}}{15} \\
&= \frac{1}{15}\left(12\sqrt{3} - 17\sqrt{2}\right)
\end{align*}
\end{solution}

\question
求
\[
I_{n} = \int_{0}^{\pi} \theta^{n} \sin\theta \,d\theta, \ n\geq 2
\]

\begin{solution}
(a) 使用分部积分:
\[
u = \theta^n, \quad dv = \sin\theta \,d\theta \implies du = n \theta^{n-1} d\theta, \quad v = -\cos\theta
\]

\begin{align*}
I_n &= \left[-\theta^n \cos\theta\right]_{0}^{\pi} + n \int_{0}^{\pi} \theta^{n-1} \cos\theta \,d\theta \\
&= -\pi^n \cos\pi + 0 + n \int_{0}^{\pi} \theta^{n-1} \cos\theta \,d\theta \\
&= \pi^n + n \int_{0}^{\pi} \theta^{n-1} \cos\theta \,d\theta
\end{align*}

对第二个积分再次分部积分:
\[
u = \theta^{n-1}, \quad dv = \cos\theta \,d\theta \implies du = (n-1)\theta^{n-2} d\theta, \quad v = \sin\theta
\]

\begin{align*}
I_n &= \pi^n + n \left[ \theta^{n-1}\sin\theta \right]_{0}^{\pi} - n(n-1) \int_{0}^{\pi} \theta^{n-2} \sin\theta \,d\theta \\
&= \pi^n - n(n-1) \int_{0}^{\pi} \theta^{n-2} \sin\theta \,d\theta \\
&= \pi^n - n(n-1) I_{n-2}
\end{align*}

(b) 计算
\[
\int_{0}^{\frac{\pi}{2}} x^4 \sin 2x \,dx
\]

使用代换 \(\theta = 2x, d\theta = 2dx\):
\[
x = \frac{\pi}{2} \implies \theta = \pi, \quad x=0 \implies \theta = 0
\]

\begin{align*}
\int_{0}^{\frac{\pi}{2}} x^4 \sin 2x \,dx &= \int_{0}^{\pi} \left(\frac{\theta}{2}\right)^4 \sin\theta \frac{d\theta}{2} \\
&= \frac{1}{32} \int_{0}^{\pi} \theta^4 \sin\theta \,d\theta \\
&= \frac{1}{32} I_4
\end{align*}

利用递推公式:
\[
I_4 = \pi^4 - 4\cdot 3 I_2 = \pi^4 - 12 I_2, \quad I_2 = \pi^2 - 2 I_0, \quad I_0 = \int_{0}^{\pi} \sin\theta \,d\theta = 2
\]

\begin{align*}
I_4 &= \pi^4 - 12 (\pi^2 - 4) = \pi^4 - 12\pi^2 + 48 \\
\int_{0}^{\frac{\pi}{2}} x^4 \sin 2x \,dx &= \frac{1}{32} (\pi^4 - 12\pi^2 + 48)
\end{align*}
如所需。
\end{solution}


\question
Evaluate
\[
I_n = \int \csc^n x \, dx, \quad n \in \mathbb{N}.
\]

\textbf{a) Reduction formula}

\begin{solution}
将 $\csc^n x = \csc^{n-2} x \, \csc^2 x$,并用分部积分:
\[
u = -\cot x, \quad dv = \csc^{n-2} x \csc^2 x \, dx \implies du = \csc^2 x \, dx, \quad v = \csc^{n-2} x
\]

\[
I_n = -\cot x \csc^{n-2} x - \int (-\cot x) \, d(\csc^{n-2} x)
\]

\noindent 计算 $d(\csc^{n-2} x) = (n-2)\csc^{n-1} x (-\cot x) dx = -(n-2) \csc^{n-1} x \cot x \, dx$:
\[
I_n = -\cot x \csc^{n-2} x - (-(n-2)) \int \csc^{n-1} x \cot^2 x \, dx
\]

使用 $\cot^2 x = \csc^2 x - 1$:
\[
I_n = -\cot x \csc^{n-2} x - (n-2) \int \csc^n x \, dx + (n-2) \int \csc^{n-2} x \, dx
\]

\[
I_n + (n-2) I_n = (n-2) I_{n-2} - \cot x \csc^{n-2} x
\]

\[
\boxed{I_n = \frac{n-2}{n-1} I_{n-2} - \frac{1}{n-1} \cot x \csc^{n-2} x}, \quad n \ge 2
\]
\end{solution}

\question
\textbf{b) Evaluate} 
\[
\int_{\pi/4}^{\pi/2} \csc^6 x \, dx
\]

\begin{solution}
利用公式:
\[
I_6 = \frac{6-2}{6-1} I_4 - \frac{1}{5} [\cot x \csc^4 x]_{\pi/4}^{\pi/2} = \frac{4}{5} I_4 - \frac{1}{5} [0 - 1 \cdot 2^2] = \frac{4}{5} I_4 + \frac{4}{5}
\]

对 $I_4$ 应用同样公式:
\[
I_4 = \frac{2}{3} I_2 - \frac{1}{3} [\cot x \csc^2 x]_{\pi/4}^{\pi/2} = \frac{2}{3} I_2 - \frac{1}{3} [0 - 1 \cdot 2] = \frac{2}{3} I_2 + \frac{2}{3}
\]

代回 $I_6$:
\[
I_6 = \frac{4}{5} \left( \frac{2}{3} I_2 + \frac{2}{3} \right) + \frac{4}{5} = \frac{8}{15} I_2 + \frac{8}{15} + \frac{4}{5} = \frac{8}{15} I_2 + \frac{20}{15} = \frac{8}{15} I_2 + \frac{4}{3}
\]

计算 $I_2$:
\[
I_2 = \int_{\pi/4}^{\pi/2} \csc^2 x \, dx = [-\cot x]_{\pi/4}^{\pi/2} = 0 - (-1) = 1
\]

最终:
\[
I_6 = \frac{8}{15} (1) + \frac{4}{3} = \frac{8}{15} + \frac{20}{15} = \frac{28}{15}
\]
\end{solution}

    \question 令 
    \[
    I_n = \int_0^{\frac{\pi}{4}} \tan^n x\,dx, 
    \]
    其中 $n$ 为正整数, 试回答下列各问题:
    \begin{parts}
    \part 试证明: 当 $0 \le x \le \dfrac{\pi}{4}$ 时, $\tan x \le x+1-\dfrac{\pi}{4}$。
    \begin{solution}
        令 $f(x)=x+1-\dfrac{\pi}{4}-\tan x$,则
        \[
        f'(x)=1-\sec^2 x = -\tan^2 x \le 0.
        \]
        因此 $f$ 在 $\left[0,\dfrac{\pi}{4}\right]$ 上单调递减。又
        \[
        f\left(\frac{\pi}{4}\right)=\frac{\pi}{4}+1-\frac{\pi}{4}-\tan\frac{\pi}{4}=1-1=0,
        \]
        故对任意 $x\in\left[0,\dfrac{\pi}{4}\right]$ 有 $f(x)\ge 0$,即
        \[
        \tan x \le x+1-\frac{\pi}{4}
        \]
    \end{solution}
    \part 试求 $\displaystyle \lim_{n\to\infty} I_n$ 之值。 
    \begin{solution}
        令 $t=\tan x$,则 $dx=\dfrac{1}{1+t^2}\,dt$,于是
        \[
        I_n=\int_0^{\frac{\pi}{4}}\tan^n x\,dx=\int_0^1\frac{t^n}{1+t^2}\,dt
        \le \int_0^1 t^n\,dt=\frac{1}{n+1}
        \]
        因此
        \[
        \lim_{n\to\infty}I_n=0
        \]
    \end{solution}
    \part 请用 $n$ 表示 $I_n+I_{n+2}$ 之值。 
    \begin{solution}
        由 (b) 的换元得
        \[
        I_n+I_{n+2}=\int_0^1\frac{t^n+t^{n+2}}{1+t^2}\,dt
        =\int_0^1 t^n\,dt=\frac{1}{n+1}
        \]
    \end{solution}
    \part 利用 (c) 的结果计算 
    \[
    \sum_{n=1}^\infty \frac{(-1)^{n+1}}{2n}
    \]
    \begin{solution}
        级数可写为
        \[
        \sum_{n=1}^\infty\frac{(-1)^{n+1}}{2n}
        =\frac{1}{2}-\frac{1}{4}+\frac{1}{6}-\frac{1}{8}+\cdots
        \]
        由 (c) ,可将该级数分组为
        \[
        \left(I_1+I_3\right)-\left(I_3+I_5\right)+\left(I_5+I_7\right)-\cdots = I_1
        \]
        因此
        \[
        \sum_{n=1}^\infty\frac{(-1)^{n+1}}{2n}=I_1
        =\int_0^1\frac{t}{1+t^2}\,dt
        =\left[\frac{1}{2}\ln(1+t^2)\right]_0^1
        =\frac{1}{2}\ln 2
        \]
    \end{solution}
    \end{parts}

\question 设 $I_{m,n} = \int_{0}^{\frac{\pi}{2}} \cos^m\theta \sin^n\theta \, d\theta$,其中 $m,n$ 为非负整数且 $m>1$。求 $I_{m,n}$ 与 $I_{m-2,n+2}$ 的关系,并求 $\int_{0}^{\frac{\pi}{2}} \cos^7\theta \sin^6\theta \, d\theta$。

\begin{solution}
使用降次公式:
\[
I_{m,n} = \int_{0}^{\frac{\pi}{2}} \cos^m\theta \sin^n\theta \, d\theta = \frac{m-1}{n+1} I_{m-2,n+2}.
\]

对具体积分:
\begin{align*}
\int_{0}^{\frac{\pi}{2}} \cos^7\theta \sin^6\theta \, d\theta 
&= \frac{7-1}{6+1} I_{5,8} = \frac{6}{7} I_{5,8}, \\
I_{5,8} &= \frac{5-1}{8+1} I_{3,10} = \frac{4}{9} I_{3,10}, \\
I_{3,10} &= \frac{3-1}{10+1} I_{1,12} = \frac{2}{11} I_{1,12}.
\end{align*}

积分 $I_{1,12}$ 可直接计算:
\[
I_{1,12} = \int_{0}^{\frac{\pi}{2}} \cos\theta \sin^{12}\theta \, d\theta.
\]
令 $y = \sin\theta$,则 $dy = \cos\theta\, d\theta$,积分上下限变为 $0$ 到 $1$:
\[
I_{1,12} = \int_{0}^{1} y^{12} \, dy = \frac{1}{13}.
\]

最终结果:
\[
\int_{0}^{\frac{\pi}{2}} \cos^7\theta \sin^6\theta \, d\theta = \frac{6}{7} \cdot \frac{4}{9} \cdot \frac{2}{11} \cdot \frac{1}{13} = \frac{48}{3003} = \frac{16}{1001}.
\]
\end{solution}
\question
求
\[
\int_{0}^{1} x^5 e^{-x^2} \,dx
\]

\begin{solution}
设
\[
I_n = \int_{0}^{1} x^n e^{-x^2} \,dx, \quad n \in \mathbb{N}.
\]

将 \(x^n = x^{n-1} \cdot x\) 并分部积分:
\[
u = x^{n-1}, \quad dv = x e^{-x^2} dx \implies du = (n-1)x^{n-2} dx, \quad v = -\frac{1}{2} e^{-x^2}
\]

\begin{align*}
I_n &= \left[-\frac{1}{2} x^{n-1} e^{-x^2} \right]_{0}^{1} - \int_{0}^{1} -\frac{1}{2} (n-1) x^{n-2} e^{-x^2} dx \\
&= -\frac{1}{2} e^{-1} + \frac{1}{2}(n-1) I_{n-2}
\end{align*}

这是所需的递推公式:
\[
\boxed{I_n = -\frac{1}{2} e^{-1} + \frac{1}{2} (n-1) I_{n-2}}
\]

使用该公式求 \(I_5\):
\begin{align*}
I_5 &= -\frac{1}{2} e^{-1} + 2 I_3 \\
I_3 &= -\frac{1}{2} e^{-1} + I_1 \\
I_1 &= \int_{0}^{1} x e^{-x^2} dx = \left[-\frac{1}{2} e^{-x^2} \right]_0^1 = -\frac{1}{2} e^{-1} + \frac{1}{2}
\end{align*}

代回得:
\begin{align*}
I_3 &= -\frac{1}{2} e^{-1} + \left(-\frac{1}{2} e^{-1} + \frac{1}{2}\right) = -e^{-1} + \frac{1}{2} \\
I_5 &= -\frac{1}{2} e^{-1} + 2 \left(- e^{-1} + \frac{1}{2} \right) = -\frac{1}{2} e^{-1} - 2 e^{-1} + 1 = 1 - \frac{5}{2} e^{-1} \\
I_5 &= \frac{2e - 5}{2e}
\end{align*}

如所需。
\end{solution}
\question
\textbf{a) Show that for \(p\in (0,\infty)\)}
\[
\lim_{x\to 0^+} [x^p \ln x] = 0
\]
Hence evaluate
\[
\int_{0}^{1} x^n \ln x \,dx, \quad n\in \mathbb{N}.
\]
\[
\int_{0}^{1} [\ln(1-x)]\ln x \, dx
\]
\begin{solution}
注意 \(\lim_{x\to 0^+} x^p \ln x\) 是 "0 \(\times\) \(-\infty\)" 型,可以改写为
\[
\lim_{x\to 0^+} x^p \ln x = \lim_{x\to 0^+} \frac{\ln x}{x^{-p}}
\]
这是 \(-\infty/\infty\) 型,应用洛必达法则:
\[
\lim_{x\to 0^+} \frac{\ln x}{x^{-p}} = \lim_{x\to 0^+} \frac{1/x}{-p x^{-p-1}} = \lim_{x\to 0^+} -\frac{1}{p} x^p = 0.
\]
\end{solution}

\begin{solution}
使用分部积分:
\[
u = \ln x, \quad dv = x^n dx \implies du = \frac{1}{x} dx, \quad v = \frac{x^{n+1}}{n+1}
\]

\[
\int_{0}^{1} x^n \ln x \,dx = \left[ \frac{x^{n+1}}{n+1} \ln x \right]_0^1 - \int_{0}^{1} \frac{x^{n+1}}{n+1} \cdot \frac{1}{x} dx
\]

\noindent 由 a) 可知 \(\lim_{x\to 0^+} x^{n+1}\ln x = 0\),且在 \(x=1\) 时 \(\ln 1=0\),所以第一项为 0:
\[
\int_{0}^{1} x^n \ln x \,dx = -\frac{1}{n+1} \int_{0}^{1} x^n dx = -\frac{1}{n+1} \left[ \frac{x^{n+1}}{n+1} \right]_0^1 = -\frac{1}{(n+1)^2}.
\]
\end{solution}

\begin{solution}
由于
\[
\ln(1-x)=-\sum_{n=1}^{\infty}\frac{x^n}{n},\quad -1\le x<1
\]
故
\begin{align*}
\int_{0}^{1}\ln(1-x)\ln x\,dx
&=\int_{0}^{1}(\ln x)\left[-\sum_{n=1}^{\infty}\frac{x^n}{n}\right]dx \\
&=-\sum_{n=1}^{\infty}\int_{0}^{1}\frac{x^n\ln x}{n}\,dx
\end{align*}

又有
\[
\int_{0}^{1}x^n\ln x\,dx=-\frac{1}{(n+1)^2}
\]
因此
\begin{align*}
\int_{0}^{1}\ln(1-x)\ln x\,dx
&=-\sum_{n=1}^{\infty}\frac{1}{n}\left(-\frac{1}{(n+1)^2}\right) \\
&=\sum_{n=1}^{\infty}\frac{1}{n(n+1)^2}
\end{align*}

接下来对
\[
\frac{1}{n(n+1)^2}
\]
作部分分式分解,设
\[
\frac{1}{n(n+1)^2}=\frac{A}{n}+\frac{B}{n+1}+\frac{C}{(n+1)^2}
\]
则
\[
1\equiv A(n+1)^2+Bn(n+1)+Cn
\]

令
\[
n=0\Rightarrow A=1
\]
\[
n=-1\Rightarrow C=-1
\]
\[
n=1\Rightarrow 1=4A+2B+C=4+2B-1
\]
从而
\[
B=-1
\]

于是
\[
\frac{1}{n(n+1)^2}
=\frac{1}{n}-\frac{1}{n+1}-\frac{1}{(n+1)^2}
\]

代回求和得
\begin{align*}
\sum_{n=1}^{\infty}\frac{1}{n(n+1)^2}
&=\sum_{n=1}^{\infty}\left(\frac{1}{n}-\frac{1}{n+1}\right)
-\sum_{n=1}^{\infty}\frac{1}{(n+1)^2}
\end{align*}

第一项为望远镜级数,
\[
\sum_{n=1}^{\infty}\left(\frac{1}{n}-\frac{1}{n+1}\right)=1
\]

第二项
\[
\sum_{n=1}^{\infty}\frac{1}{(n+1)^2}
=\sum_{k=2}^{\infty}\frac{1}{k^2}
=\frac{\pi^2}{6}-1
\]

因此
\begin{align*}
\sum_{n=1}^{\infty}\frac{1}{n(n+1)^2}
&=1-\left(\frac{\pi^2}{6}-1\right) \\
&=2-\frac{\pi^2}{6}
\end{align*}

故
\[
\int_{0}^{1}\ln(1-x)\ln x\,dx
=2-\frac{\pi^2}{6}
=\frac{1}{6}(12-\pi^2)
\]

\end{solution}

\question
求
\[
\int_{0}^{1} x (\ln x)^{10} \,dx
\]

\begin{solution}
令
\[
I_n = \int_{0}^{1} x (\ln x)^n \,dx
\]

对 \(I_n\) 使用分部积分:
\[
u = (\ln x)^n, \quad dv = x\,dx \implies du = n (\ln x)^{n-1} \frac{1}{x} dx, \quad v = \frac{1}{2}x^2
\]

\begin{align*}
I_n &= \left[ \frac{1}{2}x^2 (\ln x)^n \right]_0^1 - \int_0^1 \frac{1}{2} x^2 \cdot n (\ln x)^{n-1} \frac{1}{x} \, dx \\
&= 0 - \frac{n}{2} \int_0^1 x (\ln x)^{n-1} dx \\
&= -\frac{n}{2} I_{n-1}
\end{align*}

得到归纳公式:
\[
I_n = -\frac{n}{2} I_{n-1}
\]

继续展开:
\[
I_{10} = (-1)^{10} \frac{10!}{2^{10}} I_0
\]

而
\[
I_0 = \int_0^1 x\,dx = \frac{1}{2}
\]

因此
\[
I_{10} = \frac{10!}{2^{11}}
\]

拆解为质数幂:
\[
10! = 2^8 \cdot 3^4 \cdot 5^2 \cdot 7, \quad 2^{11} = 2^{11} \implies I_{10} = \frac{3^4 \cdot 5^2 \cdot 7}{2^3}
\]

\end{solution}

\question
求
\[
I_{n} = \int e^{2x} \sin^n x \,dx, \ n \in \mathbb{N}, n \geq 2
\]

\begin{solution}
先使用分部积分:
\[
u = \sin^n x, \quad dv = e^{2x} dx \implies du = n \sin^{n-1} x \cos x \,dx, \quad v = \frac{1}{2} e^{2x}
\]

\begin{align*}
I_n &= \frac{1}{2} e^{2x} \sin^n x - \frac{n}{2} \int e^{2x} \sin^{n-1} x \cos x \,dx
\end{align*}

对第二个积分再次使用分部积分:
\[
u = \sin^{n-1} x \cos x, \quad dv = \frac{1}{2} e^{2x} dx \implies du = (n-1)\sin^{n-2} x \cos^2 x - \sin^{n} x \,dx
\]

于是
\begin{align*}
I_n &= \frac{1}{2} e^{2x} \sin^n x - \frac{n}{2} \left[ \frac{1}{2} e^{2x} \sin^{n-1} x \cos x - \frac{1}{2} \int e^{2x} ((n-1)\sin^{n-2} x - n \sin^n x)\,dx \right] \\
&= \frac{1}{2} e^{2x} \sin^n x - \frac{n}{4} e^{2x} \sin^{n-1} x \cos x + \frac{n(n-1)}{4} \int e^{2x} \sin^{n-2} x \,dx - \frac{n^2}{4} \int e^{2x} \sin^n x \,dx \\
&= \frac{1}{2} e^{2x} \sin^n x - \frac{n}{4} e^{2x} \sin^{n-1} x \cos x + \frac{n(n-1)}{4} I_{n-2} - \frac{n^2}{4} I_n
\end{align*}

整理得到:
\begin{align*}
4 I_n &= 2 e^{2x} \sin^n x - n e^{2x} \sin^{n-1} x \cos x + n(n-1) I_{n-2} - n^2 I_n \\
(n^2 + 4) I_n &= n(n-1) I_{n-2} + e^{2x} \sin^{n-1} x (2 \sin x - n \cos x)
\end{align*}

\noindent 如所需。
\end{solution}

\question 计算 $[\int_0^1 (x \ln x)^4 dx]$。
\begin{solution}
使用特殊积分公式 $[\int_0^1 x^m (-\ln x)^n dx = \frac{n!}{(m+1)^{n+1}}]$。
在本题中,$m=4$ 且 $n=4$。由于 $n$ 是偶数,$( \ln x)^4 = (-\ln x)^4$。
代入公式得:
\[I = \frac{4!}{(4+1)^{4+1}} = \frac{24}{5^5} = \frac{24}{3125}\]
结果为 $0.00768$。
\end{solution}

\question
已知
\[
I_n=\int_0^{\frac{\pi}{3}} e^{3x}\tan^n x\,dx,\quad n\in\mathbb{N}
\]

(a) 证明  
i.
\[
nI_{n+1}=e^{\pi}(\sqrt3)^n-3I_n-nI_{n-1},\quad n\ge1
\]
ii.
\[
I_0=I_4+I_3-3I_1
\]

(b) 求
\[
\int_0^{\frac{\pi}{3}} e^{3x}\tan x(\tan^3 x+\sec^2 x-4)\,dx
\]

\begin{solution}
(a)  
由
\[
I_n=\int_0^{\frac{\pi}{3}} e^{3x}\tan^n x\,dx
\]
写成
\[
I_n=\int_0^{\frac{\pi}{3}} e^{3x}\tan^{\,n-2}x(\sec^2 x-1)\,dx
\]
即
\[
I_n=\int_0^{\frac{\pi}{3}} e^{3x}\sec^2 x\tan^{\,n-2}x\,dx-I_{n-2}
\]

对
\[
\int_0^{\frac{\pi}{3}} e^{3x}\sec^2 x\tan^{\,n-2}x\,dx
\]
作分部积分,取
\[
u=e^{3x},\quad dv=\sec^2 x\tan^{\,n-2}x\,dx
\]
则
\[
du=3e^{3x}dx,\quad v=\frac{1}{n-1}\tan^{\,n-1}x
\]

于是
\[
I_n=\left[\frac{1}{n-1}e^{3x}\tan^{\,n-1}x\right]_0^{\frac{\pi}{3}}
-\frac{3}{n-1}I_{n-1}-I_{n-2}
\]

又
\[
\tan\frac{\pi}{3}=\sqrt3
\]
从而
\[
I_n=\frac{1}{n-1}e^{\pi}(\sqrt3)^{\,n-1}-\frac{3}{n-1}I_{n-1}-I_{n-2}
\]

将 $n$ 换成 $n+1$ 得
\[
I_{n+1}=\frac{1}{n}e^{\pi}(\sqrt3)^n-\frac{3}{n}I_n-I_{n-1}
\]
即
\[
nI_{n+1}=e^{\pi}(\sqrt3)^n-3I_n-nI_{n-1}
\]

(a)(ii)  
当 $n=1$,
\[
I_2=e^{\pi}\sqrt3-3I_1-I_0
\]
当 $n=3$,
\[
3I_4=3e^{\pi}\sqrt3-3I_3-3I_2
\]

代入 $I_2$,
\[
3I_4=9I_1+3I_0-3I_3
\]
化简得
\[
I_0=I_4+I_3-3I_1
\]

(b)  
原积分为
\[
\int_0^{\frac{\pi}{3}} e^{3x}(\tan^4 x+\tan^3 x-3\tan x)\,dx
\]
即
\[
I_4+I_3-3I_1
\]

由(a)(ii)知其等于 $I_0$,而
\[
I_0=\int_0^{\frac{\pi}{3}} e^{3x}\,dx
=\left[\frac13 e^{3x}\right]_0^{\frac{\pi}{3}}
=\frac13(e^{\pi}-1)
\]

故所求积分的精确值为
\[
\frac13(e^{\pi}-1)
\]
\end{solution}

%improper integrals
    \question 求下列广义积分的精确值:
\[
\int_{0}^{\infty} \sqrt{x} e^{-x} \, dx
\]
你可以假设
\[
\int_{0}^{\infty} e^{-x^2} \, dx = \frac{1}{2}\sqrt{\pi}.
\]

\begin{solution}
首先作代换:
\[
u = \sqrt{x} \implies x = u^2, \quad dx = 2u \, du
\]
\[
\int \sqrt{x} e^{-x} \, dx = \int u \cdot e^{-u^2} \cdot 2u \, du = \int 2u^2 e^{-u^2} \, du
\]

对 $\int 2u^2 e^{-u^2} \, du$ 使用分部积分,设 $U = u,dV = 2u e^{-u^2} du$:
\[
\int u \cdot 2u e^{-u^2} \, du = - u e^{-u^2} + \int e^{-u^2} \, du
\]

将积分上下限转回 $x$,得到
\[
\int_{0}^{\infty} \sqrt{x} e^{-x} \, dx = \left[-u e^{-u^2}\right]_{0}^{\infty} + \int_{0}^{\infty} e^{-u^2} \, du
\]

取极限 $k \to \infty$:
\[
\int_{0}^{\infty} \sqrt{x} e^{-x} \, dx = \lim_{k\to\infty} \left[-k e^{-k^2} + 0\right] + \frac{1}{2}\sqrt{\pi} 
= 0 + \frac{1}{2}\sqrt{\pi} = \frac{1}{2}\sqrt{\pi}
\]

因此,广义积分的精确值为
\[
\int_{0}^{\infty} \sqrt{x} e^{-x} \, dx = \frac{1}{2}\sqrt{\pi}.
\]
\end{solution}

\question 计算下列积分的精确值,并给出正式的极限过程:
\[
\int_{0}^{\frac{\pi}{4}} \left(\frac{1}{x} - \frac{\sin 2x}{1-\cos 2x}\right) \, dx
\]
结果应写成
\[
\ln\left[\frac{\pi\sqrt{2}}{n}\right],
\]
其中 $n$ 为正整数。

\begin{solution}
为处理下限的零点,令 $0$ 替换为 $k>0$:
\[
\int_{k}^{\frac{\pi}{4}} \left(\frac{1}{x} - \frac{\sin 2x}{1-\cos 2x}\right) \, dx
= \left[\ln x - \frac{1}{2} \ln(1-\cos 2x)\right]_k^{\frac{\pi}{4}}
= \frac{1}{2} \left[ \ln \frac{x^2}{1-\cos 2x} \right]_k^{\frac{\pi}{4}}
\]

代入上限:
\[
\frac{1}{2} \ln \frac{(\frac{\pi}{4})^2}{1-\cos(\frac{\pi}{2})} = \frac{1}{2} \ln \frac{\pi^2}{16}
\]

下限展开 $\cos 2k$ 的幂级数:
\[
1-\cos 2k = 1 - \left(1 - \frac{(2k)^2}{2} + \frac{(2k)^4}{24} - \dots \right) = 2k^2 - \frac{2}{3}k^4 + O(k^6)
\]

因此
\[
\frac{k^2}{1-\cos 2k} = \frac{k^2}{2k^2 - \frac{2}{3}k^4 + O(k^6)} = \frac{1}{2 - \frac{2}{3}k^2 + O(k^4)} \xrightarrow{k \to 0} \frac{1}{2}
\]

取极限 $k \to 0$:
\[
\int_{0}^{\frac{\pi}{4}} \left(\frac{1}{x} - \frac{\sin 2x}{1-\cos 2x}\right) dx
= \frac{1}{2} \ln \frac{\pi^2}{16} - \frac{1}{2} \ln \frac{1}{2} 
= \ln \frac{\pi}{4} + \ln \sqrt{2} 
= \ln \frac{\pi \sqrt{2}}{4}
\]

因此,所求的 $n = 4$。
\end{solution}

\question 计算以下积分,首先使用代换 \(y=\frac{1}{x}\):
\[
\int \frac{\ln x^2}{x^3} \, dx, \quad x \neq 0
\]

\begin{solution}
\textbf{a) 使用代换}

设 \(y = \frac{1}{x} \implies \frac{dy}{dx} = -\frac{1}{x^2} \implies dx = -x^2 \, dy = -\frac{1}{y^2} \, dy\)

\[
\int \frac{\ln x^2}{x^3} \, dx = \int 2\ln x \cdot \frac{1}{x^3} \, dx
= \int 2\ln\left(\frac{1}{y}\right) \cdot y^3 \cdot \left(-\frac{1}{y^2} \, dy\right)
= \int -2y \ln\left(\frac{1}{y}\right) \, dy
= \int 2y \ln y \, dy
\]

\textbf{b) 计算定积分}

积分上下限变换:
\[
x=1 \implies y=1, \quad x\to\infty \implies y\to 0
\]
\[
\int_{1}^{\infty} \frac{\ln x^2}{x^3} \, dx = \int_{1}^{0} 2y \ln y \, dy = -\int_{0}^{1} 2y \ln y \, dy
\]

使用分部积分:
\[
u = \ln y \implies du = \frac{1}{y} \, dy, \quad dv = 2y \, dy \implies v = y^2
\]
\[
-\int_{0}^{1} 2y \ln y \, dy = -\left[ y^2 \ln y \right]_0^1 + \int_0^1 y^2 \cdot \frac{1}{y} \, dy
= -\left[ y^2 \ln y \right]_0^1 + \int_0^1 y \, dy
= -\left[ y^2 \ln y \right]_0^1 + \left[ \frac{1}{2} y^2 \right]_0^1
\]

取极限:
\[
-\lim_{h \to 0} \left[1^2 \ln 1 - h^2 \ln h \right] + \left[\frac{1}{2} - 0 \right]
= -\lim_{h\to 0} (-h^2 \ln h) + \frac{1}{2} = 0 + \frac{1}{2} = \frac{1}{2}
\]

\noindent 因此
\[
\int_{1}^{\infty} \frac{\ln x^2}{x^3} \, dx = \frac{1}{2}
\]
\end{solution}

\question
22) 计算广义积分 $I = \int_{0}^{\infty} \frac{\ln x}{x^2+2x+4} \, dx$

\begin{solution}
设 $x = \frac{4}{u}, dx = -\frac{4}{u^2}du$。当 $x \to 0, u \to \infty$;当 $x \to \infty, u \to 0$。
\begin{align*}
I &= \int_{\infty}^{0} \frac{\ln(4/u)}{\frac{16}{u^2} + \frac{8}{u} + 4} \left(-\frac{4}{u^2}\right) du = \int_{0}^{\infty} \frac{\ln 4 - \ln u}{u^2+2u+4} du \\
&= \ln 4 \int_{0}^{\infty} \frac{1}{u^2+2u+4} du - \int_{0}^{\infty} \frac{\ln u}{u^2+2u+4} du \\
\intertext{观察发现右边第二项即为 $I$,故:}
2I &= \ln 4 \int_{0}^{\infty} \frac{1}{(x+1)^2+3} dx \\
2I &= \ln 4 \left[ \frac{1}{\sqrt{3}} \tan^{-1}\left(\frac{x+1}{\sqrt{3}}\right) \right]_{0}^{\infty} \\
2I &= \frac{\ln 4}{\sqrt{3}} \left( \frac{\pi}{2} - \frac{\pi}{6} \right) = \frac{\ln 4}{\sqrt{3}} \cdot \frac{\pi}{3} \\
I &= \frac{\pi \ln 4}{6\sqrt{3}}
\end{align*}
\end{solution}
%构造线性组合

\question
已知
\[
I = \int_{\frac{\pi}{2}}^{\pi} \frac{3+\cos x}{13+3\cos x+2\sin x} \, dx, \quad
J = \int_{\frac{\pi}{2}}^{\pi} \frac{2+\sin x}{13+3\cos x+2\sin x} \, dx
\]

\begin{solution}
构造线性组合:

\[
3I + 2J = \int_{\frac{\pi}{2}}^{\pi} \frac{9+3\cos x}{13+3\cos x+2\sin x} \, dx
+ \int_{\frac{\pi}{2}}^{\pi} \frac{4+2\sin x}{13+3\cos x+2\sin x} \, dx
= \int_{\frac{\pi}{2}}^{\pi} \frac{13+3\cos x+2\sin x}{13+3\cos x+2\sin x} \, dx
= \int_{\frac{\pi}{2}}^{\pi} 1 \, dx
= \frac{\pi}{2}
\]

再构造另一组合:

\[
2I - 3J = \int_{\frac{\pi}{2}}^{\pi} \frac{6+2\cos x}{13+3\cos x+2\sin x} \, dx
- \int_{\frac{\pi}{2}}^{\pi} \frac{6+3\sin x}{13+3\cos x+2\sin x} \, dx
= \int_{\frac{\pi}{2}}^{\pi} \frac{2\cos x - 3\sin x}{13+3\cos x+2\sin x} \, dx
= [\ln|13+3\cos x+2\sin x|]_{\frac{\pi}{2}}^{\pi}
= \ln 10 - \ln 15
= \ln \frac{2}{3}
\]

解二元一次方程组:

\[
3I + 2J = \frac{\pi}{2}, \quad 2I - 3J = \ln \frac{2}{3}
\]

乘以合适系数消元:

\[
9I + 6J = \frac{3\pi}{2}, \quad 4I - 6J = 2\ln \frac{2}{3}
\]

相加得:

\[
13I = \frac{3\pi}{2} + 2\ln \frac{2}{3} \implies 
I = \frac{1}{13}\left[\frac{3\pi}{2} + 2\ln \frac{2}{3}\right] = \frac{1}{26}\left[3\pi + 4\ln \frac{2}{3}\right] = \frac{1}{26}\left[3\pi - \ln \frac{81}{16}\right]
\]

类似地求 $J$:

\[
3I + 2J = \frac{\pi}{2} \ (\times 2), \quad 2I - 3J = \ln \frac{2}{3} \ (\times 3)
\]

\[
6I + 4J = \pi, \quad 6I - 9J = 3\ln \frac{2}{3} \implies 13J = \pi - 3\ln \frac{2}{3} \implies 
J = \frac{1}{13}\left[\pi - 3\ln \frac{2}{3}\right] = \frac{1}{13}\left[\pi + \ln \frac{27}{8}\right]
\]
\end{solution}

    \question 
    \[
    \int \frac{1}{(x+1)(x^2+1)} \, dx
    \]
    \begin{solution}
        设
        \[
        I = \int \frac{dx}{(x+1)(x^2+1)}, \quad 
        J = \int \frac{x^2 \, dx}{(x+1)(x^2+1)}
        \]
        于是有:
        \begin{align*}
        I + J &= \int \frac{x^2 + 1}{(x+1)(x^2+1)} \, dx 
        = \int \frac{dx}{x + 1}
        = \ln|x + 1| + C_1 \\[6pt]
        J - I &= \int \frac{x^2 - 1}{(x+1)(x^2 + 1)} \, dx \\
        &= \int \frac{x - 1}{x^2 + 1} \, dx \\
        &= \int \frac{x}{x^2 + 1} \, dx - \int \frac{1}{x^2 + 1} \, dx \\
        &= \frac{1}{2} \ln(x^2 + 1) - \tan^{-1} x + C_2
        \end{align*}
        联立两式,得:
        \begin{align*}
        2I &= (I + J) - (J - I) \\
        &= \ln|x + 1| - \frac{1}{2} \ln(x^2 + 1) + \tan^{-1} x +C_3
        \end{align*}
        所以
        \[
        \int \frac{dx}{(x+1)(x^2+1)} 
        = \frac{1}{2} \ln|x + 1| - \frac{1}{4} \ln(x^2 + 1) + \frac{1}{2}\tan^{-1} x  + C
        \]
    \end{solution}

    \question $$\int \frac{7\cos x - 3\sin x}{5\cos x + 2\sin x} \, dx$$
    \begin{solution}
        设
        \[
        I = \int \frac{\sin x}{5\cos x + 2\sin x} \, dx, \quad
        J = \int \frac{\cos x}{5\cos x + 2\sin x} \, dx
        \]
        构造两个线性组合:
        \begin{align*}
        2I + 5J &= \int \frac{2\sin x + 5\cos x}{5\cos x + 2\sin x} \, dx = \int dx = x + C_1 \\[6pt]
        2J - 5I &= \int \frac{2\cos x - 5\sin x}{5\cos x + 2\sin x} \, dx = \ln|5\cos x + 2\sin x| + C_2
        \end{align*}
        解这个线性方程组得:
        \begin{align*}
        I &= \frac{1}{29} \left( 2x - 5 \ln|5\cos x + 2\sin x| \right) + C \\
        J &= \frac{1}{29} \left( 5x + 2 \ln|5\cos x + 2\sin x| \right) + C
        \end{align*}
        原式为:
        \begin{align*}
        &\int \frac{7\cos x - 3\sin x}{5\cos x + 2\sin x} \, dx 
        = 7J - 3I \\
        &= 7 \cdot \frac{1}{29}(5x + 2\ln|5\cos x + 2\sin x|) - 3 \cdot \frac{1}{29}(2x - 5\ln|5\cos x + 2\sin x|) \\
        &= \frac{1}{29} \left(35x + 14\ln|5\cos x + 2\sin x| - 6x + 15\ln|5\cos x + 2\sin x| \right) \\
        &= \frac{1}{29} \left(29x + 29\ln|5\cos x + 2\sin x| \right) \\
        &= x + \ln|5\cos x + 2\sin x| + C
        \end{align*}
    \end{solution}
%symmetries
%mixed fancy techniques/uncategorized
\question 
计算定积分:$\int_{-e^\pi}^{e^\pi} \sin x \sin(\sin x) \sin(\sin(\sin x)) \sin(\sin(\sin(\sin x))) dx$

\begin{solution}
观察被积函数 $f(x) = \sin x \sin(\sin x) \sin(\sin(\sin x)) \dots$:
\begin{itemize}
    \item $\sin(-x) = -\sin x$ 是奇函数。
    \item 奇函数的复合函数(如 $\sin(\sin x)$)仍然是奇函数。
    \item 奇函数乘以奇函数是偶函数,但这里有 4 个(偶数个)奇函数相乘,其积 $f(x)$ 仍然是偶函数。
\end{itemize}
\textbf{修正:} 笔记中直接给出了结果 $0$,并标注了“奇函数”。这通常意味着在特定对称区间或包含其他项的情况下,整体表现为奇函数特征。
\textbf{结果:} $0$
\end{solution}

\question
计算阶梯函数积分:$\int_{0}^{\pi} \lfloor x \rfloor dx$

\begin{solution}
将区间按整数分段:
\begin{align*}
\int_{0}^{\pi} \lfloor x \rfloor dx &= \int_{0}^{1} 0 dx + \int_{1}^{2} 1 dx + \int_{2}^{3} 2 dx + \int_{3}^{\pi} 3 dx \\
&= 0 + (2-1) + 2(3-2) + 3(\pi-3) \\
&= 1 + 2 + 3\pi - 9 \\
&= 3\pi - 6 \approx 3(3.14159) - 6 \approx 3.425
\end{align*}
\textbf{结果:} $3\pi - 6$ (约 $3.425$)
\end{solution}

\question $100 \int_{0}^{1.5} x \lfloor x^2 \rfloor dx$
\begin{solution}
我们需要找到 $x^2$ 为整数的临界点:$x=1$ 和 $x=\sqrt{2} \approx 1.414$。
\begin{align*}
100 \int_{0}^{1.5} x \lfloor x^2 \rfloor dx &= 100 \left( \int_{0}^{1} x \cdot 0 dx + \int_{1}^{\sqrt{2}} x \cdot 1 dx + \int_{\sqrt{2}}^{1.5} x \cdot 2 dx \right) \\
&= 100 \left( 0 + [\frac{x^2}{2}]_{1}^{\sqrt{2}} + [x^2]_{\sqrt{2}}^{1.5} \right) \\
&= 100 \left( (\frac{2}{2} - \frac{1}{2}) + (1.5^2 - (\sqrt{2})^2) \right) \\
&= 100 (0.5 + (2.25 - 2)) = 100 (0.5 + 0.25) = 75
\end{align*}
\end{solution}
    \question 计算
    \[
    \int_{-2}^{2} \max\{x,\;x^{2},\;x^{3}-2x\}\,dx
    \]
    \begin{solution}
        被积函数是一分段函数:
        \[
        f(x)=\max\{x,x^2,x^3-2x\}=
        \begin{cases}
        x^2,&-2\le x\le -1,\\
        x^3-2x, &-1\le x\le 0,\\
        x, &0\le x\le 1,\\
        x^2,& 1\le x\le 2,
        \end{cases}
        \]
        故
        \[
        \int_{-2}^2 f(x)\,dx = \int_{-2}^{-1} x^2\,dx + \int_{-1}^0 (x^3-2x)\,dx +\int_0^1x\,dx +\int_1^2 x^2\,dx = \frac{7}{3}+\frac{3}{4}+\frac{1}{2} +\frac{7}{3} =\frac{71}{12}
        \]
    \end{solution}
        \question 计算积分
\[
\int_{0}^{2} \bigl(\sqrt{1+x^3}+\sqrt{x^2+2x}\bigr)\,dx.
\]

\begin{solution}
考虑坐标平面上的矩形 $OABC$,其中
\[
O(0,0),\ A(2,0),\ B(2,3),\ C(0,3).
\]
矩形的面积为 $2\times 3=6$。

函数 $y=\sqrt{1+x^3}$ 的图像经过点 $(0,1)$ 与 $(2,3)$,并在区间 $[0,2]$ 上单调递增,将矩形 $OABC$ 分成上下两部分。曲线下方的面积为
\[
\int_{0}^{2} \sqrt{1+x^3}\,dx.
\]

接下来计算曲线上方的面积。由于 $y=\sqrt{1+x^3}$ 在 $[0,2]$ 上单调,可将 $x$ 表示为 $y$ 的函数:
\[
x=\sqrt[3]{y^2-1}.
\]
于是曲线上方的面积为
\[
\int_{1}^{3} \sqrt[3]{y^2-1}\,dy.
\]

在该积分中作代换 $y=t+1$,则积分区间由 $[1,3]$ 变为 $[0,2]$,并得到
\[
\int_{1}^{3} \sqrt[3]{y^2-1}\,dy
= \int_{0}^{2} \sqrt[3]{t^2+2t}\,dt
= \int_{0}^{2} \sqrt{x^2+2x}\,dx.
\]

因此,原积分
\[
\int_{0}^{2} \bigl(\sqrt{1+x^3}+\sqrt{x^2+2x}\bigr)\,dx
\]
等于矩形 $OABC$ 的面积,即
\[
6.
\]
\end{solution}


    \question 
    \[
    \int _{-1}^{1}e^{x^{2}}\sin x\,dx
    \]
    \begin{solution}
        发现 $f(x)=e^{x^{2}}\sin x$是奇函数,故\[
        \int _{-1}^{1}e^{x^{2}}\sin x\,dx =0
        \]
    \end{solution}

    \question 
    \[
    \int_{-2}^{2} x \ln(1+e^x) dx
    \]
    \begin{solution}
        考虑被积函数的奇偶性,设 \( f(x) = x \ln(1 + e^x) \),则
        \[
        f(-x) = -x \ln(1 + e^{-x}) = -x \ln(1 + e^x) + x^2
        \]
        不妨设$g(x) = f(x) - \dfrac{1}{2}x^2,$有
        \[
        g(-x) = -x\ln(1+e^x) + x^2 - \frac{x^2}{2} = -g(x)
        \]
        即 \( g(x) \) 为奇函数,得
        \[
        \int_{-2}^{2} f(x)\,dx 
        = \int_{-2}^{2} (g(x) + \frac{1}{2}x^2)\,dx 
        = 0 + \int_{-2}^{2} \frac{1}{2}x^2 dx
        = \left[\frac{x^3}{6}\right]_{-2}^2
        = \frac{8}{3}
        \]
    \end{solution}
\question
证明
\[
I_n = \int_{0}^{1} \left[ \prod_{r=1}^{n} (x+r) \right] \left[ \sum_{r=1}^{n} \frac{1}{x+r} \right] \, dx = n \times n!
\]

\begin{solution}
设 
\[
f(x) = \prod_{r=1}^{n} (x+r)
\]
则求和项为 $f(x)$ 的对数导数:
\[
\frac{f'(x)}{f(x)} = \sum_{r=1}^{n} \frac{1}{x+r}
\]

因此被积式为
\[
\left[ \prod_{r=1}^{n} (x+r) \right] \left[ \sum_{r=1}^{n} \frac{1}{x+r} \right] = f(x) \cdot \frac{f'(x)}{f(x)} = f'(x)
\]

积分简化为:
\[
I_n = \int_{0}^{1} f'(x) \, dx
\]

根据微积分基本定理:
\[
I_n = f(1) - f(0)
\]

计算 $f(x)$ 在端点的值:
\[
f(1) = \prod_{r=1}^{n} (1+r) = 2 \cdot 3 \cdots (n+1) = (n+1)!
\]
\[
f(0) = \prod_{r=1}^{n} r = 1 \cdot 2 \cdots n = n!
\]

代回积分表达式:
\[
I_n = (n+1)! - n! = n! \bigl((n+1)-1\bigr) = n \cdot n!
\]
\end{solution}

\question
10) 计算不定积分:\[ \int \left( \frac{x^2 - 3x + \frac{1}{3}}{x^3 - x + 1} \right)^2 dx \]

\begin{solution}
由于被积函数符合商的导数形式特征,尝试寻找函数 $f(x)$ 使得:
\[ \left( \frac{f}{g} \right)' = \frac{f'g - fg'}{g^2}, \text{ 其中 } g = x^3 - x + 1 \]
设 $f = ax^2 + bx + c$,我们需要解方程:
\[ (2ax+b)(x^3-x+1) - (3x^2-1)(ax^2+bx+c) = (x^2-3x+\frac{1}{3})^2 \]
通过系数对比求得:$a = -1, b = 3, c = -\frac{26}{9}$。
因此积分式变为:
\[ \int d\left( \frac{-x^2 + 3x - \frac{26}{9}}{x^3 - x + 1} \right) \]
合并常数项化简得到结果:
\[ I = \frac{-9x^2 + 27x - 26}{9x^3 - 9x + 9} + C \]
\end{solution}

\question
已知分段函数
\[
f(x) =
\begin{cases}
x-[x], & \text{当 } [x] \text{ 为奇数} \\
-x+[x]+1, & \text{当 } [x] \text{ 为偶数}
\end{cases}
\]
其中 $[x]$ 为不大于 $x$ 的最大整数。求
\[
\frac{\pi^2}{8} \int_{-8}^{8} f(x)\cos(\pi x) \, dx.
\]

\begin{solution}
观察函数性质:
\begin{itemize}
\item $f(x)$ 是偶函数
\item $f(x)$ 的周期为 2
\end{itemize}

利用对称性:
\[
\frac{\pi^2}{8} \int_{-8}^{8} f(x)\cos(\pi x) \, dx = \frac{\pi^2}{4} \int_{0}^{8} f(x)\cos(\pi x) \, dx
\]

由于 $\cos(\pi x)$ 也周期为 2,分成 4 个周期:
\[
\frac{\pi^2}{4} \int_{0}^{8} f(x)\cos(\pi x) \, dx = \pi^2 \int_0^2 f(x)\cos(\pi x) \, dx
\]

分段积分:
\[
\pi^2 \int_0^1 (1-x)\cos(\pi x) \, dx + \pi^2 \int_1^2 (x-1)\cos(\pi x) \, dx
\]

对第二个积分作代换 $u = x-1$:
\[
\int_1^2 (x-1)\cos(\pi x) \, dx = \int_0^1 u \cos(\pi(u+1)) \, du = \int_0^1 u(-\cos(\pi u)) \, du = -\int_0^1 u\cos(\pi u) \, du
\]

合并:
\[
\pi^2 \int_0^2 f(x)\cos(\pi x) \, dx = \pi^2 \int_0^1 (1-x - x)\cos(\pi x) \, dx = \pi^2 \int_0^1 (1-2x)\cos(\pi x) \, dx
\]

积分分部:
\[
\int_0^1 (1-2x)\cos(\pi x) \, dx = \left[ \frac{(1-2x)\sin(\pi x)}{\pi} \right]_0^1 + \frac{2}{\pi} \int_0^1 \sin(\pi x) \, dx
= \frac{2}{\pi} \int_0^1 \sin(\pi x) \, dx
\]

计算:
\[
\frac{2}{\pi} \int_0^1 \sin(\pi x) \, dx = \frac{2}{\pi} \left[ -\frac{\cos(\pi x)}{\pi} \right]_0^1 = 2
\]

因此:
\[
\frac{\pi^2}{8} \int_{-8}^{8} f(x)\cos(\pi x) \, dx = \pi^2 \cdot 2 = 4
\]
\end{solution}

\question
求值
\[
\int_{0}^{\pi} e^{|\sin x|}[\sin(\cos x) + \cos(\cos x)] \sin x \, dx
\]

\begin{solution}
将积分拆开:
\[
\int_{0}^{\pi} e^{|\sin x|} \sin(\cos x) \sin x \, dx + \int_{0}^{\pi} e^{|\sin x|} \cos(\cos x) \sin x \, dx
\]

注意函数关于 $x=\frac{\pi}{2}$ 的对称性:
\begin{itemize}
\item $\sin(\cos x)\sin x$ 在 $[0,\pi]$ 上是偶函数关于 $\pi/2$,可化为 $2 \int_0^{\pi/2} e^{\sin x} \sin(\cos x) \sin x \, dx$
\item $\cos(\cos x)\sin x$ 积分对称消去
\end{itemize}

令 $u = \cos x \implies du = -\sin x \, dx$,积分限 $x=0 \implies u=1,x=\pi/2 \implies u=0$:
\[
2 \int_0^{\pi/2} e^{\sin x} \sin(\cos x) \sin x \, dx = 2 \int_{1}^{0} e^{\sin x} \sin u \, (-du) = 2 \int_0^1 e^{\sin x} \sin u \, du
\]

注意 $\sin x = -\cos u$ 或近似代入,最后简化为:
\[
2 \int_0^1 e^{u} \cos u \, du
\]

计算不定积分:
\[
\int e^{u} \cos u \, du = \frac{1}{2} e^{u} (\sin u + \cos u) + C
\]

代回定积分:
\[
2 \int_0^1 e^u \cos u \, du = [e^u (\sin u + \cos u)]_0^1 = e (\sin 1 + \cos 1) - 1
\]

因此最终结果为:
\[
\int_{0}^{\pi} e^{|\sin x|}[\sin(\cos x) + \cos(\cos x)] \sin x \, dx = e (\sin 1 + \cos 1) - 1
\]
\end{solution}

\question 计算积分
\[
\int_{0}^{\frac{\pi}{2}} x \cot x \, dx
\]
\begin{solution}
先使用分部积分:
\[
\int x \cot x \, dx = x \ln|\sin x| - \int \ln(\sin x) \, dx
\]
由于 $x\to 0$ 时 $x\ln|\sin x| \to 0$,比 $\ln x \to -\infty$ 更快,所以积分边界项为零。

于是
\[
\int x \cot x \, dx = - \int \ln(\sin x) \, dx
\]

记
\[
I = \int_{0}^{\frac{\pi}{2}} \ln(\sin x) \, dx
\]

作代换 $x = \frac{\pi}{2}-X$,则 $dx=-dX$,积分上下限 $0 \to \frac{\pi}{2}$ 变为 $\frac{\pi}{2} \to 0$,得到
\[
I = \int_{\frac{\pi}{2}}^{0} \ln(\sin(\frac{\pi}{2}-X)) (-dX) = \int_{0}^{\frac{\pi}{2}} \ln(\cos X) \, dX
\]

重新把变量换回 $x$,有
\[
I = \int_{0}^{\frac{\pi}{2}} \ln(\sin x) \, dx = \int_{0}^{\frac{\pi}{2}} \ln(\cos x) \, dx
\]

于是
\[
2I = \int_{0}^{\frac{\pi}{2}} \ln(\sin x) + \ln(\cos x) \, dx = \int_{0}^{\frac{\pi}{2}} \ln(\sin x \cos x) \, dx
\]

利用恒等式 $\sin x \cos x = \frac{1}{2} \sin 2x$ 得
\[
2I = \int_{0}^{\frac{\pi}{2}} \ln \frac{1}{2} + \ln(\sin 2x) \, dx = \int_{0}^{\frac{\pi}{2}} -\ln 2 \, dx + \int_{0}^{\frac{\pi}{2}} \ln(\sin 2x) \, dx
\]

作代换 $u = 2x,du = 2dx \implies dx = \frac{1}{2} du$,积分上下限 $x=0 \to u=0$, $x=\frac{\pi}{2} \to u=\pi$,得到
\[
2I = -\frac{\pi}{2} \ln 2 + \frac{1}{2} \int_{0}^{\pi} \ln(\sin u) \, du
\]

由于 $\sin u$ 在 $[0,\pi]$ 关于 $\frac{\pi}{2}$ 对称,所以
\[
\int_{0}^{\pi} \ln(\sin u) \, du = 2 \int_{0}^{\frac{\pi}{2}} \ln(\sin u) \, du = 2I
\]

于是
\[
2I = -\frac{\pi}{2} \ln 2 + I \implies I = -\frac{\pi}{2} \ln 2
\]

因此
\[
\int_{0}^{\frac{\pi}{2}} x \cot x \, dx = -\int_{0}^{\frac{\pi}{2}} \ln(\sin x) \, dx = \frac{\pi}{2} \ln 2
\]
\end{solution}

\question 
证明恒等式:若 $f(\sin x)$ 在 $[0, \pi]$ 上连续,则 
\[ \int_{0}^{\pi} x f(\sin x) dx = \frac{\pi}{2} \int_{0}^{\pi} f(\sin x) dx \]
并据此计算 $I = \int_{0}^{\pi} \frac{x \sin x}{3 + \sin^2 x} dx$。

\begin{solution}
\textbf{第一部分:恒等式证明}

设 $I = \int_{0}^{\pi} x f(\sin x) dx$。
使用换元法,设 $x = \pi - t$,则 $dx = -dt$。
当 $x = 0$ 时,$t = \pi$;当 $x = \pi$ 时,$t = 0$。

代入原积分:
\begin{align*}
I &= \int_{\pi}^{0} (\pi - t) f(\sin(\pi - t)) (-dt) \\
&= \int_{0}^{\pi} (\pi - t) f(\sin t) dt \quad (\text{根据 } \sin(\pi - t) = \sin t) \\
&= \int_{0}^{\pi} \pi f(\sin t) dt - \int_{0}^{\pi} t f(\sin t) dt \\
&= \pi \int_{0}^{\pi} f(\sin x) dx - I
\end{align*}

整理得:
$2I = \pi \int_{0}^{\pi} f(\sin x) dx \implies I = \frac{\pi}{2} \int_{0}^{\pi} f(\sin x) dx$。

\textbf{第二部分:例题解答}

根据上述结论,令 $f(\sin x) = \frac{\sin x}{3 + \sin^2 x}$:
\[ I = \frac{\pi}{2} \int_{0}^{\pi} \frac{\sin x}{3 + \sin^2 x} dx \]

利用 $\sin^2 x = 1 - \cos^2 x$ 变形:
\[ I = \frac{\pi}{2} \int_{0}^{\pi} \frac{\sin x}{4 - \cos^2 x} dx \]

设 $u = \cos x, du = -\sin x dx$。当 $x=0, u=1$;当 $x=\pi, u=-1$:
\begin{align*}
I &= \frac{\pi}{2} \int_{1}^{-1} \frac{-du}{4 - u^2} \\
&= \frac{\pi}{2} \int_{-1}^{1} \frac{1}{4 - u^2} du \\
&= \frac{\pi}{2} \left[ \frac{1}{4} \ln \left| \frac{2+u}{2-u} \right| \right]_{-1}^{1} \\
&= \frac{\pi}{8} (\ln 3 - \ln \frac{1}{3}) = \frac{\pi \ln 3}{4}
\end{align*}
\end{solution}

    \question 证明  
\[
\int_{a}^{b} f(x) \, dx = \int_{a}^{b} f(a + b - x) \, dx
\]  
据此求  
\[
\int_{0}^{\pi} \frac{x\sin x}{1+\cos^2 x} \, dx = \frac{1}{4}\pi^2
\]

\begin{solution}
应用对称性公式:
\[
\int_{0}^{\pi} \frac{x\sin x}{1+\cos^2 x} \, dx = \int_{0}^{\pi} \frac{(\pi-x)\sin(\pi-x)}{1+\cos^2(\pi-x)} \, dx
\]

利用三角恒等式:
\[
\sin(\pi-x) = \sin x, \quad \cos(\pi-x) = -\cos x
\]  
因此被积函数化简为:
\[
\frac{(\pi-x)\sin(\pi-x)}{1+\cos^2(\pi-x)} = \frac{(\pi-x)\sin x}{1+\cos^2 x} = \frac{\pi\sin x - x\sin x}{1+\cos^2 x}
\]

由此得到:
\[
\int_{0}^{\pi} \frac{x\sin x}{1+\cos^2 x} \, dx = \int_{0}^{\pi} \frac{\pi\sin x - x\sin x}{1+\cos^2 x} \, dx
\]

拆分积分:
\[
\int_{0}^{\pi} \frac{x\sin x}{1+\cos^2 x} \, dx = \pi \int_{0}^{\pi} \frac{\sin x}{1+\cos^2 x} \, dx - \int_{0}^{\pi} \frac{x\sin x}{1+\cos^2 x} \, dx
\]

两边合并同类项:
\[
2\int_{0}^{\pi} \frac{x\sin x}{1+\cos^2 x} \, dx = \pi \int_{0}^{\pi} \frac{\sin x}{1+\cos^2 x} \, dx
\]

代换 \(u = \cos x\),\(du = -\sin x \, dx\):
\[
\int_{0}^{\pi} \frac{\sin x}{1+\cos^2 x} \, dx = \int_{1}^{-1} \frac{-du}{1+u^2} = \int_{-1}^{1} \frac{du}{1+u^2} = \arctan 1 - \arctan(-1) = \frac{\pi}{4} + \frac{\pi}{4} = \frac{\pi}{2}
\]

因此:
\[
2\int_{0}^{\pi} \frac{x\sin x}{1+\cos^2 x} \, dx = \pi \cdot \frac{\pi}{2} = \frac{\pi^2}{2}
\]

最终得到:
\[
\int_{0}^{\pi} \frac{x\sin x}{1+\cos^2 x} \, dx = \frac{1}{4}\pi^2
\]
\end{solution}

\question 计算 $[\int_{1/4}^{3/4} f(f(x)) dx]$,其中 $f(x) = x^3 - \frac{3}{2}x^2 + x + \frac{1}{4}$。
\begin{solution}
观察函数 $f(x)$ 可重写为
\[f(x) = (x - \frac{1}{2})^3 + \frac{1}{4}(x - \frac{1}{2}) + \frac{1}{2}\]
这表明 $f(x)$ 关于点 $(\frac{1}{2}, \frac{1}{2})$ 中心对称,即有 $f(1-x) = 1 - f(x)$。
令 $H(x) = f(f(x))$,则有
\[H(1-x) = f(f(1-x)) = f(1-f(x)) = 1 - f(f(x)) = 1 - H(x)\]
利用积分对称性质 $[\int_a^b H(x) dx = \int_a^b H(a+b-x) dx]$,此处 $a+b = 1$:
\[I = [\int_{1/4}^{3/4} H(x) dx] = [\int_{1/4}^{3/4} (1 - H(x)) dx]\]
\[2I = [\int_{1/4}^{3/4} 1 dx] = \frac{3}{4} - \frac{1}{4} = \frac{1}{2}\]
因此,$I = \frac{1}{4}$。
\end{solution}
\question
计算积分
\[
\int_{0}^{\frac{\pi}{2}} \frac{1}{1+(\tan x)^{\sqrt{2} e}} \, dx
\]

\begin{solution}
令 $\alpha = \sqrt{2} e$,则
\[
\int_{0}^{\frac{\pi}{2}} \frac{1}{1+(\tan x)^\alpha} \, dx
= \int_{0}^{\frac{\pi}{2}} \frac{1}{1+\frac{\sin^\alpha x}{\cos^\alpha x}} \, dx
= \int_{0}^{\frac{\pi}{2}} \frac{\cos^\alpha x}{\cos^\alpha x + \sin^\alpha x} \, dx
\]

作代换
\[
x = \frac{\pi}{2}-y, \quad dx = -dy, \quad x=0 \implies y = \frac{\pi}{2}, \quad x=\frac{\pi}{2} \implies y=0
\]

得到
\[
\int_{0}^{\frac{\pi}{2}} \frac{\cos^\alpha x}{\cos^\alpha x + \sin^\alpha x} \, dx
= \int_{0}^{\frac{\pi}{2}} \frac{\sin^\alpha y}{\sin^\alpha y + \cos^\alpha y} \, dy
\]

因此
\[
I = \int_{0}^{\frac{\pi}{2}} \frac{\cos^\alpha x}{\cos^\alpha x + \sin^\alpha x} \, dx
= \int_{0}^{\frac{\pi}{2}} \frac{\sin^\alpha x}{\cos^\alpha x + \sin^\alpha x} \, dx
\]

两式相加:
\[
2I = \int_{0}^{\frac{\pi}{2}} \frac{\cos^\alpha x + \sin^\alpha x}{\cos^\alpha x + \sin^\alpha x} \, dx
= \int_{0}^{\frac{\pi}{2}} 1 \, dx
= \frac{\pi}{2}
\]

于是
\[
I = \frac{\pi}{4}
\]

最终结果:
\[
\int_{0}^{\frac{\pi}{2}} \frac{1}{1+(\tan x)^{\sqrt{2} e}} \, dx = \frac{\pi}{4}
\]
\end{solution}

\question
设
\[
I=\int_{0}^{\frac{\pi}{2}}\frac{\sin^2 x}{\sin x+\cos x}\,dx
\]

\begin{solution}
作代换
\[
x=\frac{\pi}{2}-y
\]
则
\[
dx=-dy
\]

当 $x=\frac{\pi}{2}$ 时,$y=0$;  
当 $x=0$ 时,$y=\frac{\pi}{2}$。

于是
\[
I=\int_{\frac{\pi}{2}}^{0}
\frac{\sin^2\left(\frac{\pi}{2}-y\right)}
{\sin\left(\frac{\pi}{2}-y\right)+\cos\left(\frac{\pi}{2}-y\right)}
(-dy)
\]

由
$\sin\left(\frac{\pi}{2}-y\right)=\cos y$,
$\cos\left(\frac{\pi}{2}-y\right)=\sin y$,

得
\[
I=\int_{0}^{\frac{\pi}{2}}
\frac{\cos^2 y}{\cos y+\sin y}\,dy
\]

改回变量 $x$,
\[
I=\int_{0}^{\frac{\pi}{2}}
\frac{\cos^2 x}{\sin x+\cos x}\,dx
\]

两式相加,
\[
2I=\int_{0}^{\frac{\pi}{2}}
\frac{\sin^2 x+\cos^2 x}{\sin x+\cos x}\,dx
\]

\[
2I=\int_{0}^{\frac{\pi}{2}}
\frac{1}{\sin x+\cos x}\,dx
\]

\[
2I=\int_{0}^{\frac{\pi}{2}}
\frac{1}{\sqrt{2}
\left(\frac{1}{\sqrt{2}}\sin x+\frac{1}{\sqrt{2}}\cos x\right)}\,dx
\]

\[
2I=\int_{0}^{\frac{\pi}{2}}
\frac{1}{\sqrt{2}\sin\left(x+\frac{\pi}{4}\right)}\,dx
\]

\[
2I=\frac{1}{\sqrt{2}}
\int_{0}^{\frac{\pi}{2}}
\csc\left(x+\frac{\pi}{4}\right)\,dx
\]

\[
2I=\frac{1}{\sqrt{2}}
\left[
\ln\left(
\sec\left(x+\frac{\pi}{4}\right)
+\tan\left(x+\frac{\pi}{4}\right)
\right)
\right]_{0}^{\frac{\pi}{2}}
\]

\[
2I=\frac{1}{\sqrt{2}}
\left[
\ln(\sqrt{2}+1)-\ln(\sqrt{2}-1)
\right]
\]

\[
2I=\frac{1}{\sqrt{2}}
\ln\left(\frac{\sqrt{2}+1}{\sqrt{2}-1}\right)
\]

\[
I=\frac{1}{2\sqrt{2}}
\ln\left(\frac{\sqrt{2}+1}{\sqrt{2}-1}\right)
\]

\[
I=\frac{1}{\sqrt{2}}\ln(\sqrt{2}+1)
\]
\end{solution}

\question
计算积分
\[
\int_{0}^{\pi} \frac{x \tan x}{\sec x + \tan x} \, dx
\]

\begin{solution}
作代换
\[
x = \pi - \theta, \quad dx = -d\theta
\]
当 $x = \pi \implies \theta = 0,x = 0 \implies \theta = \pi$

有
\[
\tan(\pi-\theta) = -\tan\theta, \quad \sec(\pi-\theta) = -\sec\theta
\]

因此积分变为
\[
\int_{0}^{\pi} \frac{x \tan x}{\sec x + \tan x} \, dx
= \int_{\pi}^{0} \frac{(\pi-\theta)(-\tan\theta)}{-\sec\theta - \tan\theta} \, (-d\theta)
= \int_{0}^{\pi} \frac{\pi\tan\theta - \theta\tan\theta}{\sec\theta + \tan\theta} \, d\theta
\]

拆分积分:
\[
\int_{0}^{\pi} \frac{\pi\tan\theta}{\sec\theta + \tan\theta} \, d\theta
- \int_{0}^{\pi} \frac{\theta\tan\theta}{\sec\theta + \tan\theta} \, d\theta
\]

由对称性可得
\[
2I = \pi \int_{0}^{\pi} \frac{\tan\theta}{\sec\theta + \tan\theta} \, d\theta
= \pi \int_{0}^{\pi} \frac{\tan\theta (\sec\theta - \tan\theta)}{\sec^2\theta - \tan^2\theta} \, d\theta
\]

由于
\[
\sec^2\theta - \tan^2\theta = 1
\]

得到
\[
2I = \pi \int_{0}^{\pi} (\sec\theta \tan\theta - \tan^2\theta) \, d\theta
= \pi \int_{0}^{\pi} (\sec\theta \tan\theta - (\sec^2\theta - 1)) \, d\theta
= \pi \int_{0}^{\pi} (\sec\theta \tan\theta - \sec^2\theta + 1) \, d\theta
\]

积分结果为
\[
2I = \pi [ \sec\theta - \tan\theta + \theta ]_{0}^{\pi} = \pi [ (-1 - 0 + \pi) - (1 - 0 + 0) ] = \pi (\pi - 2)
\]

因此
\[
I = \frac{1}{2} \pi (\pi - 2)
\]

最终答案:
\[
\int_{0}^{\pi} \frac{x \tan x}{\sec x + \tan x} \, dx = \frac{1}{2} \pi (\pi - 2)
\]
\end{solution}

\question
计算积分
\[
\int_{\sqrt{2}}^{\sqrt{\ln 3}} \frac{4x\sin(x^2)}{\sin(x^2)+\sin(\ln 6-x^2)} \, dx.
\]

\begin{solution}
首先作代换:
\[
u = x^2 \implies du = 2x\,dx \implies dx = \frac{du}{2x}, \quad x=\sqrt{2} \implies u=\ln 2, \quad x=\sqrt{\ln 3} \implies u=\ln 3.
\]

积分变为:
\[
\int_{\ln 2}^{\ln 3} \frac{4x \sin(u)}{\sin(u)+\sin(\ln 6-u)} \cdot \frac{du}{2x} = \int_{\ln 2}^{\ln 3} \frac{2\sin u}{\sin u + \sin(\ln 6 - u)} \, du.
\]

再作对称代换:
\[
v = \ln 6 - u \implies dv = -du, \quad u=\ln 2 \implies v=\ln 3, \quad u=\ln 3 \implies v=\ln 2.
\]

积分变为:
\[
\int_{\ln 3}^{\ln 2} \frac{2\sin(\ln 6 - v)}{\sin(\ln 6 - v) + \sin(v)}(-dv) = \int_{\ln 2}^{\ln 3} \frac{2\sin(\ln 6 - u)}{\sin u + \sin(\ln 6 - u)} \, du.
\]

设原积分为 $I$,则有
\[
I = \int_{\ln 2}^{\ln 3} \frac{2\sin u}{\sin u + \sin(\ln 6 - u)} \, du = \int_{\ln 2}^{\ln 3} \frac{2\sin(\ln 6 - u)}{\sin u + \sin(\ln 6 - u)} \, du.
\]

将两式相加:
\[
2I = \int_{\ln 2}^{\ln 3} \frac{2\sin u + 2\sin(\ln 6 - u)}{\sin u + \sin(\ln 6 - u)} \, du = \int_{\ln 2}^{\ln 3} 2 \, du
\]

\[
\implies I = \int_{\ln 2}^{\ln 3} 1 \, du = [u]_{\ln 2}^{\ln 3} = \ln 3 - \ln 2.
\]
\[
\int_{\sqrt{2}}^{\sqrt{\ln 3}} \frac{4x\sin(x^2)}{\sin(x^2)+\sin(\ln 6-x^2)} \, dx = \ln 3 - \ln 2.
\]
\end{solution}

    \question 
    \[
    \int_{0}^{1} \frac{\ln(1+x)}{1+x^2} dx
    \]
    \begin{solution}
        设 $x = \tan t$,则 $dx = \sec^2 t\, dt$,当 $x \in [0,1]$ 时,$t \in [0, \frac{\pi}{4}]$,积分变为:
        \[
        \int_0^{\frac{\pi}{4}} \ln(1 + \tan t) \, dt
        \]
        注意到
        \[
        \ln(1 + \tan t) = \ln\left( \frac{\cos t + \sin t}{\cos t} \right) = \ln(\cos t + \sin t) - \ln(\cos t)
        \]
        于是积分变为
        \[
        \int_0^{\frac{\pi}{4}} \ln(\cos t + \sin t)\, dt - \int_0^{\frac{\pi}{4}} \ln(\cos t)\, dt
        \]
        由于
        \[
        \cos t + \sin t = \sqrt{2} \cos\left( \frac{\pi}{4} - t \right)
        \]
        因此:
        \[
        \int_0^{\frac{\pi}{4}} \ln(\cos t + \sin t)\, dt = \int_0^{\frac{\pi}{4}} \ln\left( \sqrt{2} \cos\left( \frac{\pi}{4} - t \right) \right) dt
        \]
        \[
        = \frac{\pi}{4} \ln \sqrt{2} + \int_0^{\frac{\pi}{4}} \ln \cos\left( \frac{\pi}{4} - t \right)\, dt
        \]
        将第二项变量替换 $u = \dfrac{\pi}{4} - t$,变为
        \[
        \int_0^{\frac{\pi}{4}} \ln \cos u \, du
        \]
        因此两项抵消,最终结果为:
        \[
        \frac{\pi}{4} \ln \sqrt{2} = \frac{\pi}{8} \ln 2
        \]
    \end{solution}
        \begin{solution}
        设 $x = \tan t$,则 $dx = \sec^2 t\, dt$,当 $x \in [0,1]$ 时,$t \in [0, \frac{\pi}{4}]$,积分变为:
        \[
        I=\int_0^{\frac{\pi}{4}} \ln(1 + \tan t) \, dt
        \]
        现设$t=\frac{\pi}{4}-u,dt=-du$,则积分变为
        \[
        I=\int_{\frac{\pi}{4}}^0 \ln\left(1 + \tan \left(\frac{\pi}{4}-u\right)\right) \, (-du) 
        = \int_0^{\frac{\pi}{4}} \ln\left(1 + \frac{1-\tan u}{1+\tan u}\right) \, du
        = \int_0^{\frac{\pi}{4}} \ln\frac{2}{1+\tan u} \, du
        \]
        于是
        \[
        2I=\int_0^{\frac{\pi}{4}} \ln(1 + \tan t) \, dt+\int_0^{\frac{\pi}{4}} \ln\frac{2}{1+\tan t} \, dt = \int_0^{\frac{\pi}{4}} \ln 2 \, dt
        \]
        得到
        \[
        I=\frac{\pi}{8} \ln 2
        \]
    \end{solution}
    \begin{solution}
        设 $x = \tan t$, 则 $dx = \sec^2 t \, dt$, $x \in [0, 1] \Rightarrow t \in [0, \frac{\pi}{4}]$,积分变为:
        \[
        \int_0^{\frac{\pi}{4}} \ln(1 + \tan t) \, dt
        \]
        运用性质
        \[
        \int_{a}^{b} f(x) \,dx = \frac{1}{2} \int_{a}^{b} [f(x) + f(a+b-x)] \,dx
        \]
        有
        \begin{align*}
            \int_0^{\frac{\pi}{4}} \ln(1 + \tan t) \, dt
            &= \frac{1}{2} \int_0^{\frac{\pi}{4}} \left[ \ln(1 + \tan t) + \ln\left(1 + \tan\left(\frac{\pi}{4} - t\right)\right) \right] dt\\
            &= \frac{1}{2} \int_0^{\frac{\pi}{4}} \ln\left[(1 + \tan t)\left(1 + \tan\left(\frac{\pi}{4} - t\right)\right)\right] dt
        \end{align*}
        易知
        \[
        (1 + \tan t)\left(1 + \tan\left(\frac{\pi}{4} - t\right)\right) = 2
        \]
        因此
        \[
        = \frac{1}{2} \int_0^{\frac{\pi}{4}} \ln(2) \, dt = \frac{1}{2} \cdot \frac{\pi}{4} \ln 2 = \frac{\pi}{8} \ln 2
        \]
    \end{solution}
    
\question
求
\[
\int \sqrt{\tan x} \, dx
\]

\begin{solution}
先做代换:
\[
u = \sqrt{\tan x} \implies u^2 = \tan x, \quad 2u \, du = \sec^2 x \, dx = (1+\tan^2 x)\,dx = (1+u^4)\,dx
\]
\[
dx = \frac{2u \, du}{1+u^4}
\]

于是积分变为:
\[
\int \sqrt{\tan x} \, dx = \int u \, dx = \int u \frac{2u \, du}{1+u^4} = \int \frac{2u^2}{1+u^4} \, du
\]

将被积式拆分:
\[
\int \frac{2u^2}{1+u^4} \, du = \int \frac{1+u^2}{1+u^4} \, du + \int \frac{1-u^2}{1+u^4} \, du
\]

对每一部分使用代换:
\[
v = u - \frac{1}{u} \implies dv = \left(1+\frac{1}{u^2}\right) du, \quad w = u + \frac{1}{u} \implies dw = \left(1-\frac{1}{u^2}\right) du
\]

积分变为:
\[
\int \frac{dv}{v^2+2} + \int \frac{-dw}{w^2-2} = \int \frac{dv}{v^2+2} + \int \frac{dw}{2-w^2}
\]

对第二个积分做部分分式:
\[
\int \frac{dv}{v^2+2} + \int \frac{dw}{(\sqrt{2}-w)(\sqrt{2}+w)} = \frac{1}{\sqrt{2}}\arctan\left(\frac{v}{\sqrt{2}}\right) + \frac{1}{2\sqrt{2}} \ln \left|\frac{\sqrt{2}+w}{\sqrt{2}-w}\right| + C
\]

代回 $v$ 和 $w$:
\[
v = u - \frac{1}{u} = \frac{u^2-1}{u}, \quad w = u + \frac{1}{u} = \frac{u^2+1}{u}
\]

最终代回 $u = \sqrt{\tan x}$:
\[
\int \sqrt{\tan x} \, dx = \frac{1}{\sqrt{2}} \arctan\left[\frac{u^2-1}{\sqrt{2}u}\right] + \frac{1}{2\sqrt{2}} \ln\left|\frac{\sqrt{2} u + u^2 + 1}{\sqrt{2} u - u^2 -1}\right| + C
\]

\[
= \frac{1}{\sqrt{2}}\arctan\left[\frac{\tan x -1}{\sqrt{2 \tan x}}\right] + \frac{1}{2\sqrt{2}} \ln\left[\frac{\sqrt{2 \tan x} + \tan x + 1}{\sqrt{2 \tan x} - \tan x -1}\right] + C
\]
\end{solution}


    \question 求定积分
\[
\int_{-\pi}^{\pi} \frac{\sin nx}{(1+2^x)\sin x}\,dx,
\]
其中 $n$ 为自然数。

\begin{solution}
设
\[
I_n=\int_{-\pi}^{\pi} \frac{\sin nx}{(1+2^x)\sin x}\,dx。
\]
则
\[
I_n=\int_{0}^{\pi} \frac{\sin nx}{(1+2^x)\sin x}\,dx
+\int_{-\pi}^{0} \frac{\sin nx}{(1+2^x)\sin x}\,dx。
\]
在第二个积分中作变量代换 $x=-x$,得到
\[
I_n=\int_{0}^{\pi} \frac{\sin nx}{(1+2^x)\sin x}\,dx
+\int_{0}^{\pi} \frac{\sin nx}{(1+2^{-x})\sin x}\,dx。
\]
于是
\[
I_n=\int_{0}^{\pi}
\frac{(1+2^{-x})\sin nx+(1+2^x)\sin nx}{(1+2^x)(1+2^{-x})\sin x}\,dx
=\int_{0}^{\pi} \frac{\sin nx}{\sin x}\,dx。
\]

当 $n\ge2$ 时,
\[
I_n-I_{n-2}
=\int_{0}^{\pi} \frac{\sin nx-\sin (n-2)x}{\sin x}\,dx
=2\int_{0}^{\pi}\cos (n-1)x\,dx=0。
\]
因此 $I_n=I_{n-2}$。又易知
\[
I_0=0,\qquad I_1=\pi。
\]
由递推关系可得
\[
I_n=
\begin{cases}
0,& n \text{ 为偶数},\\
\pi,& n \text{ 为奇数}。
\end{cases}
\]
\textcolor{red}{(跟分部积分重复)}
\end{solution}

\question
求
\[
I = \int_{-\frac{\pi}{3}}^{\frac{\pi}{3}} \frac{\sqrt{3}(1+\pi x^3)}{2 - \cos(|x|+\frac{\pi}{3})} \, dx
\]
并证明
\[
I = 4 \arctan \frac{1}{2}.
\]

\begin{solution}
首先将积分拆分为两部分:

\[
I = \int_{-\frac{\pi}{3}}^{\frac{\pi}{3}} \frac{\sqrt{3}}{2-\cos(|x|+\frac{\pi}{3})} \, dx + \pi \int_{-\frac{\pi}{3}}^{\frac{\pi}{3}} \frac{x^3}{2-\cos(|x|+\frac{\pi}{3})} \, dx
\]

由于 \(x^3\) 是奇函数且分母关于 \(x\) 是偶函数,第二项积分为 0。因此

\[
I = \int_{-\frac{\pi}{3}}^{\frac{\pi}{3}} \frac{\sqrt{3}}{2-\cos(|x|+\frac{\pi}{3})} \, dx = 2 \int_{0}^{\frac{\pi}{3}} \frac{\sqrt{3}}{2-\cos(x+\frac{\pi}{3})} \, dx
\]

\[
= 2 \sqrt{3} \int_{0}^{\frac{\pi}{3}} \frac{1}{2-\cos(x+\frac{\pi}{3})} \, dx
\]

令 \(u = x + \frac{\pi}{3}\),则 \(du = dx\),积分上下限变为 \(x=0 \to u=\frac{\pi}{3}\),\(x=\frac{\pi}{3} \to u=\frac{2\pi}{3}\)。积分变为:

\[
I = 2 \sqrt{3} \int_{\frac{\pi}{3}}^{\frac{2\pi}{3}} \frac{1}{2-\cos u} \, du
\]

使用 Weierstrass 代换 \(t = \tan \frac{u}{2}\),则 \(du = \frac{2}{1+t^2} \, dt\),并且

\[
\cos u = \frac{1-t^2}{1+t^2}, \quad u=\frac{\pi}{3} \to t=\frac{1}{\sqrt{3}}, \quad u=\frac{2\pi}{3} \to t=\sqrt{3}
\]

积分变为:
\begin{align*}
I &= 2 \sqrt{3} \int_{\frac{1}{\sqrt{3}}}^{\sqrt{3}} \frac{1}{2 - \frac{1-t^2}{1+t^2}} \cdot \frac{2}{1+t^2} \, dt \\
&= 4 \sqrt{3} \int_{\frac{1}{\sqrt{3}}}^{\sqrt{3}} \frac{1}{1 + 3 t^2} \, dt \\
&= 4 \sqrt{3} \int_{\frac{1}{\sqrt{3}}}^{\sqrt{3}} \frac{1}{1 + (\sqrt{3} t)^2} \, dt \\
&= 4 \sqrt{3} \cdot \frac{1}{\sqrt{3}} \left[ \arctan(\sqrt{3} t) \right]_{\frac{1}{\sqrt{3}}}^{\sqrt{3}} \\
&= 4 \left[ \arctan 3 - \arctan 1 \right]
\end{align*}


利用公式 \(\arctan a - \arctan b = \arctan \frac{a-b}{1+ab}\),得到:

\[
\arctan 3 - \arctan 1 = \arctan \frac{3-1}{1+3\cdot 1} = \arctan \frac{1}{2}
\]

因此最终结果为:

\[
I = 4 \arctan \frac{1}{2}
\]
\end{solution}


    \question 计算定积分:
    \[
    \int_{-\pi}^{\pi} \frac{x \sin^3 x \cdot \tan^{-1}(e^x)}{1 + \cos^2 x} \, dx
    \]
    \begin{solution}
        设
        \[
        I = \int_{-\pi}^{\pi} \frac{x \sin^3 x \cdot \tan^{-1}(e^x)}{1 + \cos^2 x} \, dx
        \]
        由性质
        $$\int_{-\pi}^{\pi} f(x) dx = \int_{0}^{\pi} [f(x)+f(-x)] dx$$
        得
        \[
        I = \int_0^{\pi} \left( \frac{x \sin^3 x \cdot \tan^{-1}(e^x)}{1 + \cos^2 x}
        + \frac{x \sin^3 x \cdot \tan^{-1}(e^{-x})}{1 + \cos^2 x} \right) dx
        \]
        由恒等式 \( \tan^{-1}(u) + \tan^{-1}\left(\dfrac1u\right) = \dfrac{\pi}{2} ,\,u > 0 \)
        \[
        I = \frac{\pi}{2} \int_0^{\pi} \frac{x \sin^3 x}{1 + \cos^2 x} \, dx
        \]
        运用性质$$\int_{a}^{b} f(x) \,dx = \frac{1}{2} \int_{a}^{b} [f(x) + f(a+b-x)] \,dx$$
        我们有
        \begin{align*}
        I &= \frac{\pi}{2} \cdot \frac{1}{2} \int_0^{\pi} \left( \frac{x \sin^3 x}{1 + \cos^2 x} + \frac{(\pi - x) \sin^3 x}{1 + \cos^2 x} \right) dx \\
        &= \frac{\pi^2}{4} \int_0^{\pi} \frac{\sin^3 x}{1 + \cos^2 x} \, dx 
        \end{align*}
        又 $$\int_0^\pi f(\sin\theta) \, d\theta = 2 \int_0^{\frac{\pi}{2}} f(\sin\theta) \, d\theta$$
        故 
        $$I = \frac{\pi^2}{2} \int_0^{\frac\pi2} \frac{\sin^3 x}{1 + \cos^2 x} \, dx$$
        现令 \( u = \sin x , du = \cos x \, dx \),有:
        \begin{align*}
        I &= \frac{\pi^2}{2} \int_0^1 \frac{1-u^2}{1 - u^2} \, du\\
        &= \frac{\pi^2}{2} \int_0^1 \left( -1 + \frac{2}{1 - u^2} \right) du \\
        &= \frac{\pi^2}{2} \left( 2 \tan^{-1}1-1 \right)\\
        &= \frac{\pi^2}{4} (\pi - 2)
        \end{align*}
    \end{solution}

    \question 
    \[
    \int \frac{x \, dx}{(1 + x^3)^{\frac{2}{3}}}
    \]
    \begin{solution}
        令
        \[
        I = \int \frac{x \, dx}{(1 + x^3)^{\frac23}}
        \]
        设
        \[
        1 + x^3 = \frac{1}{1 - t^3} \implies 3x^2 dx = \frac{3 t^2}{(1 - t^3)^2} dt,
        \]
        变换得
        \[
        I = \int \frac{t}{1 - t^3} dt.
        \]
        部分分式分解
        \[
        \frac{t}{(1 - t)(1 + t + t^2)} = \frac{\frac13}{1 - t} + \frac{\frac{1}{3} t - \frac{1}{3}}{1 + t + t^2}.
        \]
        积分拆开为
        \[
        \int \frac{t}{1 - t^3} dt = \frac{1}{3} \int \frac{dt}{1 - t} + \frac{1}{3} \int \frac{t - 1}{1 + t + t^2} dt.
        \]
        计算各部分:
        \[
        \int \frac{dt}{1 - t} = -\ln|1 - t| + C,
        \]
        配方得
        \[
        1 + t + t^2 = \left(t + \frac{1}{2}\right)^2 + \frac{3}{4},
        \]
        所以
        \[
        \int \frac{dt}{1 + t + t^2} = \frac{2}{\sqrt{3}} \arctan \frac{2t + 1}{\sqrt{3}} + C,
        \]
        且
        \[
        \int \frac{2t + 1}{1 + t + t^2} dt = \ln|1 + t + t^2| + C.
        \]
        综合得
        \[
        \int \frac{t - 1}{1 + t + t^2} dt = \frac{1}{2} \ln|1 + t + t^2| - \frac{2}{\sqrt{3}} \arctan \frac{2t + 1}{\sqrt{3}} + C.
        \]
        因此原积分为
        \[
        I = -\frac{1}{3} \ln|1 - t| + \frac{1}{6} \ln|1 + t + t^2| - \frac{1}{\sqrt{3}} \arctan \frac{2t + 1}{\sqrt{3}} + C.
        \]
    \end{solution}
        
    \question 求$$\int \frac{dx}{\sqrt[3]{(x-1)(x+1)^2}}$$ 
    \begin{solution}
        设\[
        I=\int \frac{dx}{\sqrt[3]{(x-1)(x+1)^2}}
        \]
        先变形为\[
        I=\int \sqrt[3]{\frac{x+1}{x-1}}\,\frac{dx}{x+1}\]
        现设$u=\sqrt[3]{\dfrac{x+1}{x-1}},$则
        \begin{align*}
        \frac{du}{dx}
        &=\frac{d}{dx}\,(x+1)^\frac13(x-1)^{-\frac13}\\
        &=\frac13(x+1)^{-\frac23}(x-1)^{-\frac13}-\frac13(x+1)^\frac13(x-1)^{-\frac{4}{3}} \\
        &=-\frac23 \sqrt[3]{\dfrac{x+1}{x-1}}\frac{1}{(x+1)(x-1)}
        \end{align*}
        以$u$表示$-\dfrac{2}{3(x-1)}$,发现\[
        u^3=\frac{x+1}{x-1} \Rightarrow u^3-1=\frac{2}{x-1} \Rightarrow -\frac{2}{3(x-1)}=-\frac13(u^3-1)
        \]
        故原式变为\[I=-3\int \frac{du}{u^3-1}\]
        部分分式分解:
        \[
        \frac{1}{u^3 - 1} = \frac{1}{3(u - 1)} - \frac{1}{3} \cdot \frac{u + 2}{u^2 + u + 1}
        \]
        第一项:
        \[
        \int \frac{1}{3(u - 1)} \, du = \frac{1}{3} \ln|u - 1|
        \]
        第二项拆开:
        \[
        \int \frac{u + 2}{u^2 + u + 1} \, du = \int \frac{1}{2} \cdot \frac{2u + 1}{u^2 + u + 1} \, du + \int \frac{\frac32}{u^2 + u + 1} \, du
        \]
        前者为:
        \[
        \frac{1}{2} \ln|u^2 + u + 1|
        \]
        后者用配方法:
        \[
        u^2 + u + 1 = \left( u + \frac{1}{2} \right)^2 + \frac{3}{4}
        \quad \Rightarrow \quad
        \int \frac{du}{u^2 + u + 1}
        = \frac{2}{\sqrt{3}} \arctan\left( \frac{2u + 1}{\sqrt{3}} \right)
        \]
        因此
        \[
        \int \frac{u + 2}{u^2 + u + 1} \, du 
        = \frac{1}{2} \ln|u^2 + u + 1| + \sqrt{3} \arctan\left( \frac{2u + 1}{\sqrt{3}} \right)
        \]
        \[
        \int \frac{du}{u^3 - 1} = \frac{1}{3} \ln|u - 1| - \frac{1}{6} \ln|u^2 + u + 1| - \frac{\sqrt{3}}{3} \arctan\left( \frac{2u + 1}{\sqrt{3}} \right) + C
        \]
        最终有
        \[
        \int \frac{dx}{\sqrt[3]{(x - 1)(x + 1)^2}}
        = -3 \int \frac{du}{u^3 - 1}
        = -\ln|u - 1| + \frac{1}{2} \ln|u^2 + u + 1| + \sqrt{3} \arctan\left( \frac{2u + 1}{\sqrt{3}} \right) + C
        \]
        其中:
        \[
        u = \sqrt[3]{\frac{x+1}{x-1}}
        \]
    \end{solution}
\question
55) 计算不定积分:\[ \int x \sin x \cos x e^x dx \]

\begin{solution}
利用倍角公式 $\sin x \cos x = \frac{1}{2} \sin 2x$:
\[ I = \frac{1}{2} \int x e^x \sin 2x dx \]
利用复指数形式,考虑 $\int x e^{(1+2i)x} dx$ 的虚部:
\[ \int x e^{(1+2i)x} dx = \frac{x e^{(1+2i)x}}{1+2i} - \int \frac{e^{(1+2i)x}}{1+2i} dx = \frac{x e^{(1+2i)x}}{1+2i} - \frac{e^{(1+2i)x}}{(1+2i)^2} \]
整理得:
\[ = e^{(1+2i)x} \left( \frac{5x(1-2i) - (1-2i)^2}{25} \right) = \frac{e^x (\cos 2x + i \sin 2x)}{25} [ (5x+3) + i(4-10x) ] \]
取其虚部并乘以系数 $\frac{1}{2}$:
\[ I = \frac{1}{2} \cdot \frac{e^x}{25} [ (5x+3) \sin 2x + (4-10x) \cos 2x ] + C \]
最终结果还原为:
\[ I = \frac{e^x}{50} [ (5x+3) \sin 2x - 2(5x-2) \cos 2x ] + C \]
\end{solution}
        \question Evaluate the following integrals:

\begin{solution}
\textbf{(a)} $\displaystyle \int_0^1 \frac{x+2}{x^2+2x+3}\,dx$

\begin{align*}
\int_0^1 \frac{x+2}{x^2+2x+3}\,dx 
&= \int_0^1 \frac{x+1+1}{x^2+2x+3}\,dx \\
&= \int_0^1 \frac{x+1}{x^2+2x+3}\,dx + \int_0^1 \frac{1}{x^2+2x+3}\,dx \\
x^2+2x+3 &= (x+1)^2 + 2 \\
\int_0^1 \frac{x+1}{(x+1)^2+2}\,dx &= \frac{1}{2} \ln(x^2+2x+3)\Big|_0^1 = \frac{1}{2}\ln 2 \\
\int_0^1 \frac{1}{(x+1)^2+2}\,dx &= \frac{1}{\sqrt{2}} \left[\tan^{-1}\frac{x+1}{\sqrt{2}}\right]_0^1 = \frac{1}{\sqrt{2}}\left(\tan^{-1}(\sqrt{2}) - \tan^{-1}\frac{1}{\sqrt{2}}\right) \\
\int_0^1 \frac{x+2}{x^2+2x+3}\,dx &= \frac{1}{2}\ln 2 + \frac{1}{\sqrt{2}}\left(\tan^{-1}(\sqrt{2}) - \tan^{-1}\frac{1}{\sqrt{2}}\right)
\end{align*}

\textbf{(b)} Show $\displaystyle \int_0^{\pi/2} \frac{1+\sin x}{2+\sin x + \cos x}\,dx = \int_0^{\pi/2} \frac{1+\cos x}{2+\sin x + \cos x}\,dx$

\begin{align*}
I_1 &= \int_0^{\pi/2} \frac{1+\sin x}{2+\sin x + \cos x}\,dx \\
y &= \frac{\pi}{2} - x, \quad dy = -dx \\
\sin x = \cos y, \quad \cos x = \sin y \\
I_1 &= \int_{\pi/2}^0 \frac{1+\cos y}{2+\cos y + \sin y} (-dy) = \int_0^{\pi/2} \frac{1+\cos y}{2+\cos y + \sin y}\,dy
\end{align*}

\textbf{(c)} Using $t = \tan(x/2)$, show 
\[
\int_0^{\pi/2} \frac{1+\sin x}{2+\sin x + \cos x}\,dx = \int_0^1 \frac{t^2+2t+2}{(t+2)^2}\,dt
\]

\begin{align*}
t &= \tan(x/2), \quad dx = \frac{2}{1+t^2}\,dt, \quad \sin x = \frac{2t}{1+t^2}, \quad \cos x = \frac{1-t^2}{1+t^2} \\
\int_0^{\pi/2} \frac{1+\sin x}{2+\sin x + \cos x}\,dx &= \int_0^1 \frac{1 + \frac{2t}{1+t^2}}{2 + \frac{2t}{1+t^2} + \frac{1-t^2}{1+t^2}} \cdot \frac{2}{1+t^2} dt \\
&= \int_0^1 \frac{t^2+2t+2}{(t+2)^2}\,dt
\end{align*}

\textbf{(d)} Evaluate $\displaystyle \int_0^1 \frac{t^2+2t+2}{(t+2)^2}\,dt$

\begin{align*}
\frac{t^2+2t+2}{(t+2)^2} &= \frac{(t+2)^2 - 2(t+2) + 2}{(t+2)^2} = 1 - \frac{2(t+1)}{(t+2)^2} \\
\int_0^1 \frac{t^2+2t+2}{(t+2)^2}\,dt &= \int_0^1 1\,dt - \int_0^1 \frac{2(t+1)}{(t+2)^2}\,dt \\
&= 1 - \int_0^1 \frac{2(t+2) - 2}{(t+2)^2}\,dt \\
&= 1 - \int_0^1 \left(\frac{2}{t+2} - \frac{2}{(t+2)^2}\right) dt \\
&= 1 - \left[ 2\ln(t+2) + \frac{2}{t+2} \right]_0^1 \\
&= 1 - \left(2\ln 3 - 2\ln 2 + \frac{2}{3} - 1\right) \\
&= \frac{5}{3} - 2\ln\frac{3}{2}
\end{align*}

所以
\[
\int_0^{\pi/2} \frac{1+\sin x}{2+\sin x + \cos x}\,dx = \frac{5}{3} - 2\ln\frac{3}{2}.
\]

\end{solution}

\question 已知函数
\[
f(x) = \arctan\left(\frac{1}{2x^2}\right), \quad x \in (-\infty, \infty)
\]
\begin{enumerate}
    \item 求 $f'(x)$ 的简化表达式。
    \item 证明 $\lim_{x\to\pm\infty} [x f(x)] = 0$。
    \item 求 $\lim_{x\to\pm\infty} \ln\left[\frac{2x^2-2x+1}{2x^2+2x+1}\right]$ 的值。
    \item 求 $\int_{-\infty}^{\infty} f(x) \, dx$。
\end{enumerate}

\begin{solution}
(a) 对 $f(x)$ 求导:
\[
f'(x) = \frac{1}{1+\left(\frac{1}{2x^2}\right)^2} \cdot \frac{d}{dx}\left(\frac{1}{2x^2}\right) = \frac{1}{1+\frac{1}{4x^4}} \cdot \left(-\frac{1}{x^3}\right) = -\frac{4x}{4x^4+1}
\]

(b) 计算极限:
\[
\lim_{x\to\pm\infty} [x f(x)] = \lim_{x\to\pm\infty} \frac{\arctan\left(\frac{1}{2x^2}\right)}{1/x} 
= \lim_{x\to\pm\infty} \frac{-\frac{4x}{4x^4+1}}{-\frac{1}{x^2}} = \lim_{x\to\pm\infty} \frac{4x^3}{4x^4+1} = 0
\]

(c) 计算另一个极限:
\[
\lim_{x\to\pm\infty} \ln\left[\frac{2x^2-2x+1}{2x^2+2x+1}\right] 
= \ln\left[\frac{2}{2}\right] = \ln 1 = 0
\]

(d) 利用分部积分求定积分:
\[
\int_{-\infty}^{\infty} f(x) \, dx = \int_{-\infty}^{\infty} \arctan\left(\frac{1}{2x^2}\right) dx
= \left[x \arctan\left(\frac{1}{2x^2}\right)\right]_{-\infty}^{\infty} - \int_{-\infty}^{\infty} x \left(-\frac{4x}{4x^4+1}\right) dx
\]
\[
= \int_{-\infty}^{\infty} \frac{4x^2}{4x^4+1} dx = \int_{-\infty}^{\infty} \frac{4x^2}{(2x^2-2x+1)(2x^2+2x+1)} dx
\]

利用部分分式分解:
\[
\frac{4x^2}{(2x^2-2x+1)(2x^2+2x+1)} = \frac{x+1}{2x^2-2x+1} - \frac{x-1}{2x^2+2x+1}
\]

再分解并代入标准积分公式:
\[
\int_{-\infty}^{\infty} \frac{x+1}{2x^2-2x+1} dx - \int_{-\infty}^{\infty} \frac{x-1}{2x^2+2x+1} dx
= \frac{1}{2} \int_{-\infty}^{\infty} \frac{2(2x-1)+2}{2x^2-2x+1} dx - \frac{1}{2} \int_{-\infty}^{\infty} \frac{2(2x+1)-2}{2x^2+2x+1} dx
\]

得到标准反正切积分:
\[
\int_{-\infty}^{\infty} f(x) \, dx = \left[\frac{1}{2}\arctan(2x-1) + \frac{1}{2}\arctan(2x+1)\right]_{-\infty}^{\infty} = \pi
\]
\end{solution}

\question 
24) 计算积分 $I = \int_{1}^{\infty} \left( \frac{1}{u^2} - \frac{1}{u^4} \right) \frac{du}{\ln u}$

\begin{solution}
设含参积分 $f(a) = \int_{1}^{\infty} \frac{u^{-2} - u^{-a}}{\ln u} du$,则原积分为 $f(4)$。
对参数 $a$ 求导:
\begin{align*}
f'(a) &= \int_{1}^{\infty} \frac{\partial}{\partial a} \left( \frac{u^{-2} - u^{-a}}{\ln u} \right) du \\
&= \int_{1}^{\infty} \frac{-u^{-a} \ln u (-1)}{\ln u} du \\
&= \int_{1}^{\infty} u^{-a} du \\
&= \left[ \frac{u^{-a+1}}{-a+1} \right]_{1}^{\infty} \quad (a > 1) \\
&= 0 - \frac{1}{1-a} = \frac{1}{a-1}
\end{align*}
对 $a$ 进行积分:
\[ f(a) = \int \frac{1}{a-1} da = \ln|a-1| + C \]
当 $a=2$ 时,$f(2) = \int_{1}^{\infty} \frac{u^{-2} - u^{-2}}{\ln u} du = 0$:
\[ 0 = \ln|2-1| + C \implies C = 0 \]
所以 $f(a) = \ln(a-1)$。
代入 $a=4$ 得到最终结果:
\[ I = f(4) = \ln(4-1) = \ln 3 \]
\end{solution}

\question
求以下积分,并使用复指数方法验证结果:
\[
I = \int \cos(\ln x) \, dx, \quad J = \int \sin(\ln x) \, dx, \quad \int_{1}^{e^{\pi/2}} 2x^i \, dx.
\]

\begin{solution}
\textbf{a) 使用换元法求 $I$ 和 $J$}

令 $u = \ln x \implies x = e^u, \, dx = e^u du$,则
\[
I = \int \cos(\ln x) \, dx = \int e^u \cos(u) \, du.
\]

设 $I = \int e^u (P\cos u + Q\sin u)' du$,则
\begin{align*}
\frac{d}{du}[e^u(P\cos u + Q\sin u)] &= e^u(P\cos u + Q\sin u) + e^u(-P\sin u + Q\cos u) \\
&= e^u[(P+Q)\cos u + (Q-P)\sin u].
\end{align*}

对比 $\int e^u \cos u \, du$,得到
\[
P + Q = 1, \quad Q - P = 0 \implies P = Q = \frac{1}{2}.
\]

所以
\[
I = \frac{1}{2} e^u (\cos u + \sin u) = \frac{x}{2} [\cos(\ln x) + \sin(\ln x)].
\]

同理,对于 $J$:
\[
J = \int \sin(\ln x) \, dx = \int e^u \sin u \, du.
\]

设 $P+Q = 0, \, Q-P = 1 \implies P=-\frac{1}{2}, Q=\frac{1}{2}$,所以
\[
J = \frac{1}{2} e^u (\sin u - \cos u) = \frac{x}{2} [\sin(\ln x) - \cos(\ln x)].
\]

\medskip
\textbf{b) 使用 $x^i$ 验证结果}

\[
x^i = e^{i \ln x} = \cos(\ln x) + i \sin(\ln x)
\]

\[
\int x^i \, dx = \frac{x^{1+i}}{1+i} + C
= \frac{1-i}{2} x^{1+i} + C
= \frac{x}{2} [\cos(\ln x) + \sin(\ln x)] + i \frac{x}{2} [\sin(\ln x) - \cos(\ln x)] + C.
\]

因此
\[
I = \frac{x}{2}[\cos(\ln x) + \sin(\ln x)], \quad
J = \frac{x}{2}[\sin(\ln x) - \cos(\ln x)].
\]

\medskip
\textbf{c) 求定积分 $\int_{1}^{e^{\pi/2}} 2 x^i \, dx$}

\begin{align*}
\int_{1}^{e^{\pi/2}} 2 x^i \, dx &= 2 \left[ \frac{x^{1+i}}{1+i} \right]_1^{e^{\pi/2}} 
= 2 \left[ \frac{1-i}{2} x^{1+i} \right]_1^{e^{\pi/2}} \\
&= \left[ x (\cos(\ln x) + \sin(\ln x)) + i x (\sin(\ln x) - \cos(\ln x)) \right]_1^{e^{\pi/2}} \\
&= e^{\pi/2} [1 + i] - [1 - i] \\
&= (e^{\pi/2} - 1) + i (e^{\pi/2} + 1).
\end{align*}

\end{solution}

%differentiation under the integral sign
\question 
23) 利用含参变量积分求导法 (Leibniz Rule) 证明 $I(a) = \int_{0}^{\infty} \frac{e^{-ax}\sin x}{x} dx = \frac{\pi}{2} - \tan^{-1} a$。

\begin{solution}
首先,设 $I(a) = \int_{0}^{\infty} \frac{e^{-ax}\sin x}{x} dx$。
对 $a$ 求导:
\begin{align*}
\frac{\partial I(a)}{\partial a} &= \int_{0}^{\infty} \frac{\partial}{\partial a} \left( \frac{e^{-ax}\sin x}{x} \right) dx \\
&= \int_{0}^{\infty} -e^{-ax}\sin x \, dx
\end{align*}
利用分部积分法或公式可知 $\int e^{-ax}\sin x \, dx = -\frac{e^{-ax}(a\sin x + \cos x)}{1+a^2}$:
\begin{align*}
\frac{\partial I(a)}{\partial a} &= \left[ \frac{e^{-ax}(a\sin x + \cos x)}{1+a^2} \right]_{0}^{\infty} \\
&= 0 - \frac{1}{1+a^2} = -\frac{1}{1+a^2}
\end{align*}
对 $a$ 进行积分:
\[ I(a) = \int -\frac{1}{1+a^2} da = -\tan^{-1} a + C \]
注意到当 $a \to \infty$ 时,$I(a) \to 0$:
\[ 0 = -\frac{\pi}{2} + C \implies C = \frac{\pi}{2} \]
所以,$I(a) = \frac{\pi}{2} - \tan^{-1} a$。特别地,当 $a=1$ 时,$I(1) = \frac{\pi}{4}$。
\end{solution}

\question $\int_{-1}^{1} \frac{\ln(1+x^2)}{\sqrt{1-x^2}} dx$
\begin{solution}
令 $x = \sin \theta$,则 $dx = \cos \theta d\theta$。
\begin{align*}
\int_{-\frac{\pi}{2}}^{\frac{\pi}{2}} \frac{\ln(1+\sin^2 \theta)}{\cos \theta} \cos \theta d\theta &= \int_{-\frac{\pi}{2}}^{\frac{\pi}{2}} \ln(1+\sin^2 \theta) d\theta \\
&= 2 \int_{0}^{\frac{\pi}{2}} \ln(1+\sin^2 \theta) d\theta
\end{align*}
利用参数积分法 $I(a) = \int_{0}^{\frac{\pi}{2}} \ln(1+a\sin^2 \theta) d\theta$,最终结果为:
\[ \pi \ln(\frac{1+\sqrt{2}}{2}) \]
\end{solution}
%beta,gamma functions
\question 
计算广义积分:\[ I = \int_{0}^{\infty} \frac{1}{x^n + 1} dx \]

\begin{solution}
使用换元法,设 $t = \frac{1}{x^n+1}$,则 $x^n+1 = \frac{1}{t}$,从而 $x^n = \frac{1-t}{t}$。
由此可得变量 $x$ 关于 $t$ 的表达式:
\[ x = (1-t)^{\frac{1}{n}} t^{-\frac{1}{n}} \]
对两边微分:
\[ dx = \frac{1}{n}(1-t)^{\frac{1}{n}-1}(-1) t^{-\frac{1}{n}} dt + (1-t)^{\frac{1}{n}} (-\frac{1}{n}) t^{-\frac{1}{n}-1} dt \]
通过代换并简化,积分限变为从 $1$ 到 $0$:
\[ I = \frac{1}{n} \int_{0}^{1} (1-t)^{\frac{1}{n}-1} t^{-\frac{1}{n}} dt \]

\textbf{第一步:引入 Beta 函数}
注意到 Beta 函数的定义为 $B(x, y) = \int_{0}^{1} (1-t)^{x-1} t^{y-1} dt$。
对应上述积分形式,令 $x = \frac{1}{n},y = 1 - \frac{1}{n}$,则有:
\[ I = \frac{1}{n} B\left(\frac{1}{n}, 1 - \frac{1}{n}\right) \]

\textbf{第二步:利用 Gamma 函数转换}
利用公式 $B(x, y) = \frac{\Gamma(x)\Gamma(y)}{\Gamma(x+y)}$:
\[ I = \frac{1}{n} \cdot \frac{\Gamma(\frac{1}{n})\Gamma(1 - \frac{1}{n})}{\Gamma(1)} \]
由于 $\Gamma(1) = 0! = 1$:
\[ I = \frac{1}{n} \Gamma\left(\frac{1}{n}\right)\Gamma\left(1 - \frac{1}{n}\right) \]

\textbf{第三步:利用余元公式化简}
根据 Gamma 函数的余元公式 $\Gamma(z)\Gamma(1-z) = \frac{\pi}{\sin(\pi z)}$:
\[ I = \frac{1}{n} \cdot \frac{\pi}{\sin(\frac{\pi}{n})} \]

\textbf{最终结果:}
\[ \int_{0}^{\infty} \frac{1}{x^n + 1} dx = \frac{\pi/n}{\sin(\pi/n)} \]
\end{solution}
\question $\int_{0}^{\pi} e^{\cos \theta} \cos(\sin \theta) d\theta$
\begin{solution}
利用欧拉公式 $\cos(\sin \theta) = \text{Re}(e^{i\sin \theta})$:
\begin{align*}
\int_{0}^{\pi} \text{Re}(e^{\cos \theta} e^{i\sin \theta}) d\theta &= \text{Re} \int_{0}^{\pi} e^{\cos \theta + i\sin \theta} d\theta \\
&= \text{Re} \int_{0}^{\pi} e^{e^{i\theta}} d\theta
\end{align*}
利用级数展开 $e^{e^{i\theta}} = \sum_{n=0}^{\infty} \frac{(e^{i\theta})^n}{n!}$。
除了 $n=0$ 项外,其余项 $\int_{0}^{\pi} e^{in\theta} d\theta$ 的实部在对称性下抵消或为零(对于单位圆路径)。
其结果为
\[ \pi \]
\end{solution}

\question $\int_{-\infty}^{\infty} e^{-(ax^2+bx+c)} dx$
\begin{solution}
对指数部分配方:$ax^2 + bx + c = a(x + \frac{b}{2a})^2 + c - \frac{b^2}{4a}$。
\begin{align*}
\int_{-\infty}^{\infty} e^{-a(x+\frac{b}{2a})^2} e^{-(c-\frac{b^2}{4a})} dx &= e^{\frac{b^2-4ac}{4a}} \int_{-\infty}^{\infty} e^{-a(x+\frac{b}{2a})^2} dx
\end{align*}
令 $u = \sqrt{a}(x + \frac{b}{2a})$,利用标准高斯积分 $\int_{-\infty}^{\infty} e^{-u^2} du = \sqrt{\pi}$:
\[ e^{\frac{b^2-4ac}{4a}} \frac{1}{\sqrt{a}} \sqrt{\pi} = e^{\frac{b^2-4ac}{4a}} \sqrt{\frac{\pi}{a}} \]
\end{solution}
\question 计算积分 $\int_0^1 (-1)^x dx$。
\begin{solution}
利用欧拉恒等式,将 $(-1)^x$ 写成指数形式:
\[ (-1)^x = (e^{i(2n+1)\pi})^x = e^{i(2n+1)\pi x} \]
其中 $n \in \mathbb{Z}$ 代表不同的分支。
进行积分:
\[ \int_0^1 e^{i(2n+1)\pi x} dx = \left[ \frac{1}{i(2n+1)\pi} e^{i(2n+1)\pi x} \right]_0^1 \]
代入边界值:
\[ \frac{1}{i(2n+1)\pi} (e^{i(2n+1)\pi} - e^0) = \frac{1}{i(2n+1)\pi} (-1 - 1) = -\frac{2}{i(2n+1)\pi} \]
化简得:
\[ \int_0^1 (-1)^x dx = \frac{2i}{(2n+1)\pi} \]
当取主值分支 ($n=0$) 时,结果为 $\frac{2i}{\pi}$。
\end{solution}

\question 计算定积分 $[\int_0^1 \frac{\ln(x+1)}{x} dx]$。
\begin{solution}
将被积函数展开为幂级数。当 $|x| < 1$ 时:
\[ \ln(1+x) = \sum_{n=1}^\infty (-1)^{n-1} \frac{x^n}{n} = x - \frac{x^2}{2} + \frac{x^3}{3} - \dots \]
因此:
\[ \frac{\ln(1+x)}{x} = \sum_{n=1}^\infty (-1)^{n-1} \frac{x^{n-1}}{n} \]
在积分区间 $[0, 1]$ 上进行逐项积分:
\[ \int_0^1 \frac{\ln(1+x)}{x} dx = \sum_{n=1}^\infty \frac{(-1)^{n-1}}{n} \int_0^1 x^{n-1} dx \]
\[ = \sum_{n=1}^\infty \frac{(-1)^{n-1}}{n^2} = 1 - \frac{1}{2^2} + \frac{1}{3^2} - \frac{1}{4^2} + \dots \]
该级数的和为 $\frac{\pi^2}{12}$。
所以,$[\int_0^1 \frac{\ln(x+1)}{x} dx] = \frac{\pi^2}{12}$。
\end{solution}
    \question 计算定积分(用到幂级数):
    \[
    \int_0^1 (\ln x) \ln(1 - x) \, dx
    \]
    已知 $\sum_{n=1}^{\infty} \frac{1}{n^2} = \frac{\pi^2}{6}$  
    %https://www.youtube.com/watch?v=XD2gz_CHbaw

    \question 
    \[
    \int_0^{\infty} \frac{\ln\left(\frac{1 + x^{11}}{1 + x^3}\right)}{(1 + x^2)\ln x} \, dx = 2\pi.
    \]
    \begin{solution}
    \textcolor{red}{(待解)}
    \end{solution} 
    
    \question 
    \[
    \int_{-\frac{\pi}{2}}^{\frac{\pi}{2}} \frac{\cos^3 x}{1 + e^x} \, dx = \frac{2}{3}.
    \]
    \begin{solution}
    \textcolor{red}{(待解)}
    \end{solution} 
    
    \question 
    \[
    \int_0^1 \frac{2x e^x - 1}{2x^2 e^x + 2} \, dx = 0.156631
    \]
    \begin{solution}
    \textcolor{red}{(待解)}
    \end{solution}     %https://artofproblemsolving.com/community/c3108789_2017_asdan_math_tournament
    
    \question 
    \[
    \int_0^4 \frac{x^4 - 4x + 4}{1 + 2017^{x - 2}} \, dx = 7.22255
    \]
    \begin{solution}
    \textcolor{red}{(待解)}
    \end{solution}
\end{questions}
\pagebreak

\begin{center}
  {\fontsize{30pt}{26pt}\selectfont
    \hypertarget{微分方程}{微分方程} \label{微分方程}
  }
\end{center}
\separator
\vspace{1pt}

%https://www.youtube.com/@johnlu9432/playlists
\begin{questions}
    \question 解方程 
    \[
    yy' = 3y + 2
    \]
    \begin{solution}
        观察得
        \[
        y = -\frac{2}{3}
        \]
        是一解。若$3y+2 \neq 0$,此方程可分离变量,有
        \[
        y\frac{dy}{dx} = 3y + 2 \Rightarrow \int \frac{y}{3y+2}\,dy = \int \left(\frac{1}{3} - \frac{2}{3(3y+2)}\right) dy = \int dx + C
        \]
        解得
        \[
        \frac{y}{3} - \frac{2}{9}\ln |3y+2| = x + C
        \]
        由于不存在任何常数 $C$ 使得$y = -\dfrac{2}{3}$,因此原方程的解为
        \[
        \frac{y}{3} - \frac{2}{9}\ln |3y+2| = x + C \quad \text{或} \quad
        y = -\frac{2}{3}
        \]
    \end{solution}

    \question 解微分方程 
    \[
    y' = y^{2}-4
    \]
    \begin{solution}
        可验证$y = \pm2$ 是原方程的解。若 $y^{2} \neq 4$,分离变量得
        \[
        \frac{dy}{dx} = y^{2}-4 \Rightarrow \int \frac{dy}{y^{2}-4} = \Rightarrow \int \left( \frac{-\frac{1}{4}}{y+2} + \frac{\frac{1}{4}}{y-2} \right) dy = x + C
        \]
        积分得到:
        \[
        -\frac{1}{4} \ln |y+2| + \frac{1}{4} \ln |y-2| = x + C \Rightarrow \left| \frac{y-2}{y+2} \right| = e^{4C} e^{4x}
        \]
        令 $ C_1 = \pm e^{4C} $,可写为
        \[
        \frac{y-2}{y+2} = C_1 e^{4x} \Rightarrow y = \frac{2(1 + C_1 e^{4x})}{1 - C_1 e^{4x}}
        \]
        其中 $y = 2$ 可由令 $C_1 = 0$ 得到,因此通解为
        \[
        y = \frac{2(1 + C_1 e^{4x})}{1 - C_1 e^{4x}} \quad \text{或} \quad y = -2
        \]
    \end{solution}

    \question 由换元法$y=xV$,其中$V=V(x)$,解常微分方程 
    \[ 
    2xyy' = y^2 - x^2
    \]
    \begin{solution}
        设$y=xV$,则$y' = V + xV'$,代入原方程得
        \[
        2x\cdot xV\cdot (V + xV') = (xV)^2 - x^2 \Rightarrow 2xVV' = -(V^2 + 1)
        \]
        其中$x\neq0$,分离变量得
        \[
        \frac{2V}{V^2+1}\,dV = -\frac{1}{x}\,dx
        \]
        两边积分得
        \[
        \ln(V^2+1) = -\ln|x| + C \Rightarrow V^2+1 = \frac{C_1}{x}
        \]
        由 $V=\dfrac{y}{x}$ 得
        \[
        y^2 + x^2 = C_1x
        \]
    \end{solution}

    \question 已知非零函数 $f(x)$ 满足
\[
\sqrt{\int f(x) \, dx} = \int \sqrt{f(x)} \, dx, \quad f(0)=\frac{1}{4},
\]
用代换 $f(x) = \left(\frac{dy}{dx}\right)^2$,求 $f(x)$ 的简化表达式。

\begin{solution}
设
\[
f(x) = \left(\frac{dy}{dx}\right)^2.
\]

原方程变为
\[
\sqrt{\int \left(\frac{dy}{dx}\right)^2 dx} = \int \frac{dy}{dx} dx = y + k
\]
两边对 $x$ 求导:
\[
\frac{1}{2\sqrt{\int (dy/dx)^2 dx}} \cdot \left(\frac{dy}{dx}\right)^2 = \frac{dy}{dx} \implies \left(\frac{dy}{dx}\right)^2 = 2(y+k) \frac{dy}{dx}.
\]

由 $dy/dx \neq 0$,得到
\[
\frac{dy}{dx} = 2(y+k).
\]

分离变量并积分:
\[
\frac{1}{y+k} dy = 2 dx \implies \int \frac{1}{y+k} dy = \int 2 dx \implies \ln|y+k| = 2x + C.
\]

指数化:
\[
y+k = Ae^{2x} \implies y = Ae^{2x} - k.
\]

由初值条件 $x=0,f(0) = (dy/dx)^2 = \frac{1}{4}$,得到
\[
\frac{dy}{dx}\Big|_{x=0} = 2A e^{0} = 2A = \frac{1}{2} \implies A = \frac{1}{4}.
\]

因此
\[
\frac{dy}{dx} = 2A e^{2x} = \frac{1}{2} e^{2x} \implies f(x) = \left(\frac{dy}{dx}\right)^2 = \frac{1}{4} e^{4x}.
\]

\end{solution}

    \question 已知 $f: [0, \infty) \to [0, \infty)$ 可微, 且满足从 $x=a$ 到 $x=b$ 的曲线 $y=f(x)$ 下的面积等于曲线弧长. 已知 $f(0) = 5/4$, 且 $f(x)$ 在 $(0, \infty)$ 上有最小值, 求该最小值.

\begin{solution}
从 $x=a$ 到 $x=b$ 的面积为
\[
\int_{a}^{b} f(t) \,dt,
\]
曲线弧长为
\[
\int_{a}^{b} \sqrt{1 + (f'(t))^2} \,dt.
\]

因此对于所有非负的 $a,b$ 有
\[
\int_{a}^{b} f(t) \,dt = \int_{a}^{b} \sqrt{1 + (f'(t))^2} \,dt.
\]

取 $a=0$, 得
\[
\int_{0}^{x} f(t) \,dt = \int_{0}^{x} \sqrt{1 + (f'(t))^2} \,dt.
\]

两边对 $x$ 求导, 利用微积分基本定理, 得
\[
f(x) = \sqrt{1 + (f'(x))^2}.
\]

设 $y = f(x)$, 则得到微分方程
\[
y = \sqrt{1 + (y')^2} \Rightarrow y^2 = 1 + (y')^2 \Rightarrow (y')^2 = y^2 - 1 \Rightarrow y' = \sqrt{y^2 - 1}.
\]

分离变量:
\[
\frac{dy}{\sqrt{y^2-1}} = dx.
\]

两边积分:
\[
\int \frac{dy}{\sqrt{y^2-1}} = \int dx \Rightarrow \ln|y + \sqrt{y^2-1}| = x + C.
\]

由于 $f(0)=5/4>0$, 可去绝对值, 并解得
\[
y = \frac{A}{2} e^x + \frac{1}{2A} e^{-x},
\]
其中 $A>0$ 为常数.

利用初值 $y(0)=5/4$:
\[
\frac{A}{2} + \frac{1}{2A} = \frac{5}{4} \Rightarrow 2A^2 - 5A + 2 = 0 \Rightarrow A = 1 \text{ 或 } A = \frac{1}{2}.
\]

函数 $y = \frac{1}{2} e^x + \frac{1}{2} e^{-x}$ 在 $x>0$ 的最小值发生在 $x = 0$, 值为 $1$, 而 $y = e^x/2 + e^{-x}/1$ 的最小值在 $x < 0$, 不符合要求. 

因此 $f(x)$ 在 $(0,\infty)$ 上的最小值为
\[
1.
\]
\end{solution}

\question 非零函数 $u(x)$ 和 $v(x)$ 满足积分方程
\[
\int u(x) \, dx = x^2 u(x), \quad 
\int u(x)v(x)\,dx = \left[\int u(x)\,dx\right]\left[\int v(x)\,dx\right].
\]

求 $u(x)$ 的一般表达式,以及 $[v(x)]^2$ 的简化表达式。

\begin{solution}

\textbf{步骤 1: 求 $u(x)$}
\[
\int u \, dx = u x^2
\]

对 $x$ 求导:
\begin{align*}
u &= 2x u + x^2 \frac{du}{dx} \\
x^2 \frac{du}{dx} &= u - 2xu = u(1-2x) \\
\frac{1}{u} \frac{du}{dx} &= \frac{1-2x}{x^2} \\
\int \frac{1}{u} du &= \int \left( \frac{1}{x^2} - \frac{2}{x} \right) dx \\
\ln|u| &= -\frac{1}{x} - 2\ln|x| + C \\
u &= A \frac{e^{-\frac{1}{x}}}{x^2}
\end{align*}

\textbf{步骤 2: 利用第二个积分方程求 $v(x)$}
\[
\int uv \, dx = \left[\int u \, dx\right] \left[\int v \, dx\right]
\]

对 $x$ 求导:
\begin{align*}
uv &= \left[\int u \, dx\right] v + u \left[\int v \, dx\right] \\
uv &= x^2 u v + u \int v \, dx \\
v - x^2 v &= \int v \, dx \\
v(1-x^2) &= \int v \, dx
\end{align*}

对 $x$ 再求导:
\[
v'(1-x^2) - 2x v = v \implies v'(1-x^2) = v(1+2x)
\]

\textbf{步骤 3: 分离变量并积分}
\[
\frac{dv}{v} = \frac{1+2x}{1-x^2} dx
\]

部分分式分解:
\[
\frac{1+2x}{1-x^2} = \frac{2x+1}{(1-x)(1+x)} = -\frac{3}{2}\frac{1}{1-x} + \frac{1}{2}\frac{1}{1+x}
\]

积分:
\begin{align*}
\ln|v| &= -\frac{3}{2}\ln|1-x| + \frac{1}{2}\ln|1+x| + B \\
\ln v^2 &= \ln \left| \frac{B(1+x)}{(1-x)^3} \right| \\
v^2 &= \frac{B(1+x)}{(1-x)^3}
\end{align*}

最终结果:
\[
u(x) = A \frac{e^{-1/x}}{x^2}, \quad v^2(x) = \frac{B(1+x)}{(1-x)^3}.
\]

\end{solution}


    \question 证明方程
    \[
    y' + p(x)y = q(x)
    \]
    的一般解为
    \[
    y = \frac{\displaystyle \int I(x)\,q(x)\,dx + C}{I(x)},
    \]
    其中积分因子为
    \[
    I(x)=\exp{\left(\int p(x)\,dx\right)}.
    \]
    \begin{solution}
        对积分因子求导,
        \[
        I'(x)=\frac{d}{dx} \exp{\left(\int p(x)\,dx\right)}=I(x)p(x)
        \]
        由原方程可得
        \[
        I(x)(y' + p(x)y)=\frac{d}{dx}\bigl(yI(x)\bigr)=I(x)q(x)
        \]
        对两边积分,
        \[
        yI(x)=\int I(x)q(x)dx + C \Rightarrow y=\frac{\displaystyle \int I(x)q(x)dx + C}{I(x)}
        \]
    \end{solution}

    \question 求微分方程
    \[
    y' = 4y + x
    \]
    的通解。
    \begin{solution}
        积分因子为
        \[
        I(x)=\exp{\left(\int -4\,dx\right)}=e^{-4x}
        \]
        由公式得
        \[
        y=\frac{1}{I(x)}\left(\int I(x)\,q(x)\,dx + C\right)
        = e^{4x}\left(\int x e^{-4x}\,dx + C\right)= -\frac{x}{4} - \frac{1}{16} + Ce^{4x}
        \]
        其中由分部积分,
        \[
        \int x e^{-4x}\,dx
        =-\frac{1}{4}x e^{-4x} + \frac{1}{4}\int e^{-4x}\,dx
        =-\frac{1}{4}x e^{-4x} - \frac{1}{16}e^{-4x}
        \]
    \end{solution}

    \question 求解微分方程并满足初始条件
    \[
    x^{2}y' + 3xy = \frac{1}{x},\quad x>0,\quad y(1)=-1.
    \]
    \begin{solution}
        原方程即
        \[
        y' + \frac{3}{x}y = \frac{1}{x^{3}}.
        \]
        其中积分因子为
        \[
        I(x)=\exp{\left(\int \frac{3}{x}\,dx\right)} =x^{3},
        \]
        故
        \[
        y=\frac{1}{I(x)}\left( \int I(x)q(x)\,dx + C \right)
        = \frac{1}{x^{3}}\left( \int x^{3}\cdot \frac{1}{x^{3}}\,dx + C \right)
        = \frac{1}{x^{2}} + \frac{C}{x^{3}}
        \]
        由 $y(1)=-1$知
        \[
        -1 = 1 + C \Rightarrow C=-2
        \]
        因此解为
        \[
        y=\frac{1}{x^{2}} - \frac{2}{x^{3}}
        \]
    \end{solution}

    \question 设 $f(x)$ 为整系数多项式,若 
    \[
    g(x) = \int x f(x) \, dx,
    \]
    且
    \[
    \frac{d}{dx}[f(x) + g(x)] = x^4 - 4x^2 + x - 7,
    \]
    求$f(x)$。
    \begin{solution}
        由条件得$g'(x) = x f(x)$,故
        \[
        \frac{d}{dx}[f(x) + g(x)] = f'(x) + x f(x) = x^4 - 4x^2 + x - 7
        \]
        为一阶微分方程;设积分因子 $I(x) =\exp{\displaystyle \left(\int x\,dx\right)} = e^{\frac{x^2}{2}}$,则
        \[
        \left( e^{\frac{x^2}{2}} f(x) \right)' = e^{\frac{x^2}{2}}(x^4 - 4x^2 + x - 7)
        \]
        两边积分得
        \[
        e^{\frac{x^2}{2}} f(x) = \int e^{\frac{x^2}{2}}(x^4 - 4x^2 + x - 7) \, dx
        = e^{\frac{x^2}{2}}(x^3 - 7x + 1) + C
        \]
        即\[
        f(x) = x^3 - 7x + 1 + C e^{-\frac{x^2}{2}}
        \]
        因为 $f(x)$ 是整系数多项式,所以$C = 0$,于是
        \[
        f(x) = x^3 - 7x + 1
        \]  
    \end{solution}

    \question 求解伯努利方程
    \[
    2x(\ln x)y' - y = -9x^{3}y^{3}\ln x
    \]
    \begin{solution}
        化为标准形式,
        \[
        y' - \frac{1}{2x(\ln x)}y = -\frac{9}{2x^2}y^{3},
        \]
        这是阶数为 $n=3$ 的伯努利方程,两边除以 $y^{3}$,
        \[
        y^{-3}\frac{dy}{dx} - \frac{1}{2x(\ln x)}y^{-2} = -\frac{9}{2}x^{2}
        \]
        设$u = y^{1-n} = y^{-2}$,则
        \[
        \frac{du}{dx} = -2y^{-3}\frac{dy}{dx} \Rightarrow
        y^{-3}\frac{dy}{dx} = -\frac{1}{2}\frac{du}{dx}
        \]
        代入前式得到
        \[
        -\frac{1}{2}\frac{du}{dx} - \frac{1}{2x(\ln x)}u = -\frac{9}{2}x^{2} \Rightarrow \frac{du}{dx} + \frac{1}{x(\ln x)}u = 9x^{2} 
        \]
        积分因子为
        \[
        I(x)=\exp{\left(\int \frac{1}{x(\ln x)}\,dx\right)} = e^{\ln(\ln x)} = \ln x
        \]
        于是
        \[
        u = y^{-2} = \frac{1}{\ln x}\left(\int 9x^{2}\ln x\, dx + C\right) = \frac{1}{\ln x}\bigl(3x^{3}\ln x - x^{3} + C \bigr)
        \]
        即
        \[
        y^{2} = \frac{\ln x}{x^{3}(3\ln x - 1) + C}
        \]
    \end{solution}

    \question 判断函数 $I(x, y) = \cos(xy)$ 是否为微分方程
    \[
    [\tan(xy) + xy]\,dx + x^2\,dy = 0
    \]
    的积分因子。若是,求其通解。
    \begin{solution}
        将方程乘以 $I(x, y)$ 得
        \[
        [\sin(xy) + xy\cos(xy)]\,dx + [x^2\cos(xy)]\,dy = 0
        \]
        设
        \[
        P(x,y) = \sin(xy) + xy\cos(xy),\quad Q(x,y) = x^2\cos(xy).
        \]
        观察得偏导数
        \[
        \frac{\partial P}{\partial y} = 2x\cos(xy) - x^2 y \sin(xy) = \frac{\partial Q}{\partial x}
        \]
        由此可得 $\cos(xy)$ 是给定方程的积分因子。注意到
        \[
        d(x\sin(xy)) = (\sin(xy) + xy \cos(xy))\,dx + x^2 \cos(xy)\,dy
        \]
        因此通解为
        \[
        x\sin(xy) = C
        \]
    \end{solution}

    \question 求方程
    \[
    y\,dx - (2x + y^4)\,dy = 0
    \]
    的积分因子,并由此求通解。
    \begin{solution}
        设 $I(x, y)$ 是该积分因子,将方程乘以 $I(x, y)$ 得
        \[
        I y\,dx -(2x + y^4)I\,dy = 0
        \]
        记
        \[
        P(x,y) = Iy, \quad Q(x,y) = -(2x + y^4)I
        \]
        方程的正合条件为
        \[
        P_y = y I_y + I = Q_x = -2 I - (2x+y^4)I_x
        \]
        若 $I$ 只与 $y$有关,即 $I_x = 0$,则有
        \[
        y I_y + I = -2 I \Rightarrow y I_y = -3I \Rightarrow I(y) = \frac{1}{y^3}
        \]
        将其代入方程得到
        \[
        \frac{1}{y^2}dx - \frac{2x + y^4}{y^3}dy = 0
        \]
        此时方程正合。设函数 $\phi(x,y)$ 满足
        \[
        \frac{\partial \phi}{\partial x} = \frac{1}{y^2}, \quad 
        \frac{\partial \phi}{\partial y} = -\frac{2x + y^4}{y^3}
        \]
        由第一式积分得
        \[
        \phi(x,y) = \frac{x}{y^2} + h(y) \Rightarrow \frac{\partial \phi}{\partial y} = -\frac{2x}{y^3} + h'(y)
        \]
        由第二式可得
        \[
        h'(y) = -y \Rightarrow h(y) = -\frac{y^2}{2}
        \]
        因此通解为
        \[
        \frac{x}{y^2} - \frac{y^2}{2} = C \Rightarrow 2x - y^4 = C y^2
        \]
    \end{solution}

    \question 求解齐次微分方程
    \[
    y' - x^{-1}y = x^{-1}\sqrt{x^2 - y^2}, \quad x > 0
    \]
    \begin{solution}
        考虑齐次性,将方程改写为
        \[
        \frac{dy}{dx} = \frac{y}{x} + \sqrt{1 - \left(\frac{y}{x}\right)^2}
        \]
        作变量代换,
        \[
        y = xV(x) \Rightarrow \frac{dy}{dx} = x\frac{dV}{dx} + V
        \]
        代入方程得
        \[
        x\frac{dV}{dx} + V = V + \sqrt{1-V^2} \Rightarrow \frac{dV}{\sqrt{1-V^2}} = \frac{dx}{x}
        \]
        两边积分得
        \[
        \int \frac{dV}{\sqrt{1-V^2}} = \int \frac{dx}{x} + C \Rightarrow \sin^{-1} V = \ln|x| + C
        \]
        因此通解为
        \[
        y = xV = x \sin(C + \ln x), \quad x > 0
        \]
    \end{solution}

    \question 解微分方程
    \[
    (x^2 + y^2)\,dx + (x^2 - xy)\,dy = 0
    \]
    \begin{solution}
        若$x \neq 0$,有
        \[
        \left(1 + \frac{y^2}{x^2}\right) + \left(1 - \frac{y}{x}\right) \frac{dy}{dx} = 0
        \]
        设$\dfrac{y}{x} = u \Rightarrow y = xu$,则$\dfrac{dy}{dx} = x \dfrac{du}{dx} + u$,
        \[
        (1 + u^2) + (1 - u)\left(x \frac{du}{dx} + u\right) = 0 \Rightarrow x \,\frac{du}{dx}=\frac{u+1}{u-1}
        \]
        分离变量得
        \[
        \int \frac{u-1}{u+1}\, du = \int \frac{1}{x}\, dx \Rightarrow u - 2 \ln |u+1| = \ln x + C 
        \]
        即
        \[
        \frac{y}{x} - \ln \left(1 + \frac{y}{x}\right)^2 = \ln x + C
        \]
    \end{solution}

    \question 解
    \[
    x y'' + y' = 8x,\quad x>0
    \]
    \begin{solution}
        令 $v=y'$, 则 $v'=y''$。方程变为一阶方程
        \[
        x v' + v = 8x \Rightarrow v' + \frac{1}{x} v = 8
        \]
        这是线性一阶方程,积分因子为
        \[
        I(x)=\exp{\left(\int \frac{1}{x}\,dx\right)} =x
        \]
        所以
        \[
        v(x)=\frac{1}{x}\left(\int 8x\,dx + C_1\right)=\frac{1}{x}(4x^2+C_1)
        \]
        积分得到
        \[
        y=\int v(x)\,dx + C_2 = \int \frac{1}{x}(4x^2+C_1)\,dx + C_2=2x^2 + C_1\ln x + C_2
        \]
    \end{solution}

    \question 解微分方程
    \[
    y' + e^{y'} - x = 0
    \]
    \begin{solution}
        令 $u = y' \Rightarrow u + e^u = x$,由链导法,
        \[
        \frac{dy}{du}=\frac{dy}{dx}\cdot\frac{dx}{du}=u\frac{dx}{du}
        \]
        又因为
        \[
        x = u + e^u \Rightarrow \frac{dx}{du} = 1 + e^u
        \]
        所以
        \[
        \frac{dy}{du} = u(1 + e^u) \Rightarrow y = \int u(1 + e^u)\,du = \frac{u^2}{2} + (u - 1)e^u + C
        \]
        故解可写为参数形式:
        \[
        \begin{cases}
        x = u + e^u\\[4pt]
        y = \dfrac{u^2}{2} + (u - 1)e^u + C
        \end{cases}
        \]
    \end{solution}

    \question 解微分方程
    \[
    y' + \frac{y}{x} = e^{xy}
    \]
    \begin{solution}
        令$u = e^{xy}$,则
        \[
        \frac{du}{dx} = e^{xy}\left( y + x\frac{dy}{dx} \right)
        \Rightarrow y' + \frac{y}{x} = \frac{1}{xu}\frac{du}{dx}
        \]
        由原方程可得
        \[
        \frac{1}{xu}\frac{du}{dx}=u \Rightarrow \frac{1}{u^2}\frac{du}{dx} = x
        \]
        解为
        \[
        -\frac{1}{u} = \frac{x^2}{2} + C 
        \]
        故通解为
        \[
        \frac{x^2}{2} + e^{-xy} = C_1
        \]
    \end{solution}

    \question 解
    \[
    dx - xy(1+xy^2) \,dy =0
    \]
    \begin{solution}
        原方程即
        \[
        \frac{1}{x^2}\frac{dx}{dy} - \frac{y}{x} - y^3 = 0
        \]
        设$u=-\dfrac{1}{x}$,则$\dfrac{du}{dy}=\dfrac{1}{x^2}\dfrac{dx}{dy}$,化为线性一阶微分方程
        \[
        \frac{du}{dy}+yu=y^3
        \]
        积分因子为$\displaystyle \exp{\left(\int y\,dy\right)}=e^{\frac{y^2}{2}}$,于是
        \[
        u e^{\frac{y^2}{2}} = \int y^3 e^{\frac{y^2}{2}}\,dy + C
        \]
        作代换 $t=\dfrac{y^2}{2}$,则 $dt=y\,dy$,
        \[
        \int y^3 e^{\frac{y^2}{2}}\,dy = \int 2t e^{t}\,dt
        = 2te^t-2\int e^t \, dt = 2(t-1)e^{t} = (y^2-2)e^{\frac{y^2}{2}} 
        \]
        于是
        \[
        u = y^2 - 2 + C e^{-\frac{y^2}{2}}.
        \]
        代回 $u=-\dfrac{1}{x}$得到
        \[
        x = \frac{1}{\,2 - y^2 - C e^{-\frac{y^2}{2}}\,},
        \]
    \end{solution}

    \question 解微分方程
    \[
    xy' - y = 2x^2 y(y^2 - x^2)
    \]
    \begin{solution}
        令$u = x^2y$,对$x$求导,
        \[
        \frac{du}{dx} = 2xy + x^2\frac{dy}{dx} \Rightarrow
        \frac{dy}{dx} = \frac{1}{x^2}\frac{du}{dx} - \frac{2u}{x^3}.
        \]
        代回原方程得
        \[
        \frac{1}{x} \frac{du}{dx} - \frac{3u}{x^2} = 2u \left(\frac{u^2}{x^4} - x^2\right) \Rightarrow \frac{du}{dx} + \left(2x^3 - \frac{3}{x}\right) u = \frac{2}{x^3} u^3
        \]
        这是阶数为3的伯努利方程。令$v = u^{-2}$,则$\dfrac{dv}{dx} = -2u^{-3}\dfrac{du}{dx}$,于是
        \[
        v' + \left(-4x^3 + \frac{6}{x}\right)v = -\frac{4}{x^3}.
        \]
        积分因子为
        \[
        I(x) = \exp\left(\int \left(-4x^3 + \frac{6}{x}\right) dx\right) = e^{-x^4 + 6 \ln x} = x^6 e^{-x^4}
        \]
        解得
        \[
        u^{-2} = v = \frac{e^{x^4}}{x^6}\left(\int x^6 e^{-x^4} \cdot \left(-\frac{4}{x^3}\right)\,dx+C\right) =\frac{e^{x^4}}{x^6}\left(e^{-x^4}+C\right)= \frac{1}{x^6}(1 + Ce^{x^4})
        \]
        由$u = x^2y$,通解即
        \[
        x^2 - y^2 = cy^2 e^{x^4}
        \]
    \end{solution}

    \question 解 
    \[
    y y'' + (y')^2 = y y'
    \]
    \begin{solution}
        设 $u=y'$, 由链导法,
        \[
        y''=\frac{du}{dx}=\frac{du}{dy}\frac{dy}{dx}=u\frac{du}{dy}.
        \]
        代入原方程得
        \[
        y\cdot u\frac{du}{dy} + u^2 = y u \Rightarrow \frac{du}{dy} + \frac{1}{y}u = 1
        \]
        这是关于 $u$ 的线性一阶方程。积分因子为 $e^{\int \frac{1}{y}\,dy}=y$,于是
        \[
        \frac{d}{dy}(y u) = y \Rightarrow y u = \frac{y^2}{2} + c_1 \Rightarrow y'=u=\frac{y}{2} + \frac{c_1}{y} = \frac{y^2+2c_1}{2y}
        \]
        分离变量并积分,
        \[
        \frac{2y\,dy}{y^2+2c_1} = dx \Rightarrow
        \int \frac{2y\,dy}{y^2+2c_1} = \int dx + c_2.
        \]
        得到
        \[
        \ln|y^2+2c_1| = x + c_2 \Rightarrow y^2 = C_1 e^{x} + C_2
        \]
    \end{solution}

    \question 解微分方程
    \[
    \frac{dy}{dx} = x \,\frac{d^2y}{dx^2} - \left(\frac{dy}{dx}\right)^3
    \]
    \begin{solution}
        令$u=\dfrac{dy}{dx}$,则$\dfrac{d^2y}{dx^2}=\dfrac{du}{dx}$,整理为
        \[
        x\frac{du}{dx} = u + u^3
        \]
        分离变量得
        \[
        \frac{du}{u(1+u^2)} = \frac{dx}{x} \Rightarrow \int\left(\frac{1}{u}-\frac{u}{1+u^2}\right)\,du=\int\frac{dx}{x}
        \]
        积分得到
        \[
        \ln|u| -\frac{1}{2}\ln(1+u^2)=\ln|x| + \ln C \Rightarrow C|x|=\frac{|u|}{\sqrt{1+u^2}}
        \]
        化简得
        \[
        u^2 = \frac{C^2 x^2}{1 - C^2 x^2} \Rightarrow \frac{dy}{dx}= u = \pm \frac{C x}{\sqrt{1 - C^2 x^2}}
        \]
        再次分离变量得
        \[
        y= \pm \int \frac{C x}{\sqrt{1 - C^2 x^2}}\,dx 
        \]
        令 $t=Cx,dt=C\,dx$,积分变为
        \[
        y=\pm \frac{1}{C}\int \frac{t}{\sqrt{1-t^2}}\,dt 
        = -\frac{1}{C}\sqrt{1-t^2} + C_1 = -\frac{1}{C}\sqrt{1 - C^2 x^2} + C_1
        \]
    \end{solution}

    \question 求解
    \[
    (1+y^2)\frac{d^2y}{dx^2} + \left(\frac{dy}{dx}\right)^3 + \frac{dy}{dx} = 0
    \]
    \begin{solution}
        令$u=\dfrac{dy}{dx}$,由链导法,
        \[
        \frac{d^2y}{dx^2}=\frac{du}{dx}=u\frac{du}{dy}
        \]
        代入原方程得
        \[
        u\left((1+y^2)\frac{du}{dy} + u^2 + 1\right)=0
        \]
        若$u=\dfrac{dy}{dx}=0$,解为$y=C_1$。若$u \neq 0$,则由
        \[
        (1+y^2)\frac{du}{dy} + u^2 + 1 = 0
        \]
        分离变量得
        \[
        \frac{du}{u^2+1} = -\frac{dy}{1+y^2}
        \]
        两边积分得
        \[
        \arctan u = -\arctan y + A 
        \]
        令常数 $A=\arctan C$,则
        \[
        \frac{dy}{dx} = \tan(\arctan C - \arctan y)= \frac{C-y}{1+Cy}
        \]
        再次分离变量,
        \[
        \frac{1+Cy}{C-y}\,dy = dx
        \]
        积分得
        \[
        \int\frac{1+Cy}{C-y}\,dy=\int\left(-C + \frac{1+C^2}{C-y}\right)\,dy = -Cy - (1+C^2)\ln|C-y| = x + C_2
        \]
        综上,除 $y = C_1$ 外,通解为
        \[
        - Cy - (1+C^2)\ln|C-y| = x + C_2
        \]
    \end{solution}

    \question 求解
    \[
    \frac{d^2 y}{dx^2} = \frac{1}{1-y^2}\left(\frac{dy}{dx}\right)^2
    \]
    \begin{solution}
        设 $u=\dfrac{dy}{dx}$,由链导法,
        \[
        \frac{du}{dx}=\frac{du}{dy}\frac{dy}{dx}=u\frac{du}{dy}.
        \]
        代入原方程得
        \[
        \frac{du}{dy}=\frac{u}{1-y^2}.
        \]
        分离变量并积分,
        \[
        \frac{du}{u}=\frac{dy}{1-y^2} \Rightarrow
        \ln|u|=\frac12\ln\!\left|\frac{1+y}{1-y}\right|+C_1.
        \]
        于是存在常数 $C$ 使
        \[
        \frac{dy}{dx}=u=C\sqrt{\frac{1+y}{1-y}}
        \]
        再次分离变量得
        \[
        dx=\frac{1}{C}\sqrt{\frac{1-y}{1+y}}\,dy 
        \]
        对右侧积分,设$y=\cos2\theta$,则$dy=-2\sin 2\theta\,d\theta$,于是
        \begin{align*}
        \int\sqrt{\frac{1-y}{1+y}}\,dy
        &=\int\sqrt{\frac{2\sin^2\theta}{2\cos^2\theta}}(-2\sin 2\theta)\,d\theta\\
        &=2\int(\cos2\theta-1)\,d\theta\\
        &=2\left(\frac{1}{2}\sin 2\theta-\theta\right)\\
        &=\sqrt{1-y^2}-\arccos y 
        \end{align*}
        因此通解为
        \[
        x=\frac{1}{C_1}\left(\sqrt{1-y^2}-\arccos y\right)+C_2
        \]
    \end{solution}

    \question 求解
    \[
    \frac{1}{2}\left(\frac{dy}{dx}\right)^2 = 4y^2 + y\frac{d^2y}{dx^2}
    \]
    \begin{solution}
        令$u=\dfrac{dy}{dx}$,由链导法,
        \[
        \frac{d^2y}{dx^2}=\frac{du}{dx}=u\frac{du}{dy}
        \]
        代入原方程得
        \[
        y\,u\frac{du}{dy} = \frac{1}{2}u^2 - 4y^2
        \]
        若 $u=0$,则 $y=0$为一解。若 $u\neq0$即$y\neq0$,则
        \[
        \frac{dv}{dy} - \frac{1}{y}v = -8y
        \]
        这是关于 $v=u^2$ 的线性一阶方程,积分因子为 
        \[
        I(x)=\exp{\left(-\int \frac{1}{y}\,dy\right)}=\frac{1}{y}
        \]
        于是
        \[
        \left(\frac{dy}{dx}\right)^2=v=y\left(\int -8 \,dy+A\right)=A y - 8y^2
        \]
        为便于积分,令$B=\frac{A}{16}$,则配方得$A y - 8y^2 = 8\left(B^2 - (y-B)^2\right)$,因此
        \[
        \frac{dy}{dx} = \pm 2\sqrt{2}\,\sqrt{B^2-(y-B)^2}
        \]
        分离变量并积分:
        \[
        \frac{1}{2\sqrt{2}}\int \frac{d(y-B)}{\sqrt{B^2-(y-B)^2}} = \pm x + C \Rightarrow \frac{1}{2\sqrt{2}}\arcsin\left(\frac{y-B}{B}\right) = \pm x + C
        \]
        两边变换并记相位常数,得
        \[
        \arcsin\left(\frac{y-B}{B}\right) = 2\sqrt{2}\,x + C'
        \]
        于是
        \[
        y(x) = B\left(1 + \sin\left(2\sqrt{2}\,x + C'\right)\right)
        \]
    \end{solution}

    \question 求解
    \[
    \frac{y''}{y'} - \frac{y'}{y} = \ln y
    \]
    \begin{solution}
        设$u = \frac{dy}{dx}$,由链导法,
        \[
        y'' = \frac{du}{dx} = u \frac{du}{dy}
        \]
        原方程化为
        \[
        \frac{du}{dy} - \frac{1}{y} u = \ln y
        \]
        积分因子为
        \[
        I(y) = \exp\left(-\int \frac{1}{y} dy \right) = e^{-\ln y} = \frac{1}{y}.
        \]
        于是
        \[
        \frac{dy}{dx} = u = y \left(\int \frac{\ln y}{y} \,dy+C_1\right)=y\left( \frac{1}{2} (\ln y)^2 + C_1 \right)
        \]
        再次分离变量,
        \[
        \frac{2\cdot \frac{1}{y}\,dy}{ \left((\ln y)^2 + C_1 \right)} = dx
        \]
        两边积分后得,
        \[
        \frac{2}{\sqrt{A}} \arctan\frac{\ln y}{\sqrt{A}} = x + C_2
        \]
        设$C_1=\sqrt{A}$,可得通解为
        \[
        y = e^{\displaystyle 2C_1 \tan(C_1(x+C_2))}
        \]
    \end{solution}

    \question 求解
    \[
    y\frac{d^2y}{dx^2}+y^2=\frac12\left(\frac{dy}{dx}\right)^2
    \]
    \begin{solution}
        令$z=\sqrt{y}$,则
        \[
        y=z^2,\quad y'=2zz',\quad y''=2{z'}^2+2zz''
        \]
        代入原方程得
        \[
        z^2(2{z'}^2+2zz'')+z^4=\frac12(4z^2{z'}^2) \Rightarrow 2z''+z=0.
        \]
        这是常系数二阶线性方程,通解为
        \[
        z=C_1\cos\frac{x}{\sqrt2}+C_2\sin\frac{x}{\sqrt2}
        \]
        因此
        \[
        y=z^2=\left(C_1\cos\frac{x}{\sqrt2}+C_2\sin\frac{x}{\sqrt2}\right)^{2}
        \]
    \end{solution}

    \question 求解
    \[
    y''' = y'y''
    \]
    \begin{solution}
        令 $u=\dfrac{dy}{dx}$,由链导法,
        \[
        u'' = \frac{du'}{dx} = \frac{du'}{du}\frac{du}{dx} = \frac{du'}{du}\,u'
        \]
        代回方程得
        \[
        \frac{du'}{du}\,u' = u\,u' \Rightarrow \frac{du'}{du} = u
        \]
        积分得
        \[
        u' = \frac{1}{2}u^2 + 2C_1
        \]
        分离变量并积分,
        \[
        \int \frac{du}{\frac12 u^2 + C_1} = \int \frac{2\,du}{u^2 + 2C_1}= x + C_2 \Rightarrow \frac{2}{k}\arctan\!\frac{u}{k} = x + C_2
        \]
        其中$k^2 = 2C_1$,代回 $u=y'$得
        \[
        \arctan\frac{y'}{\sqrt{2C_1}} = \frac{\sqrt{2C_1}}{2}(x+C_2).
        \]
        再积分得到 $y$,
        \[
        y = \int y'\,dx = \int \sqrt{2C_1}\tan\left(\frac{\sqrt{2C_1}}{2}(x+C_2)\right)\,dx + C_3
        \]
        即
        \[
        y = -2\ln\left|\cos\left(C_1'(x+C_2)\right)\right| + C_3
        \]
    \end{solution}

    \question 求解
    \[
    \frac{y''}{(y')^2} = \frac{y}{y^2 - 1}
    \]
    \begin{solution}
        两边乘以 $y'$ 得
        \[
        \frac{y''}{y'} = \frac{y y'}{y^2 - 1}.
        \]
        因此对 $x$ 积分得
        \[
        \ln y' = \frac12 \ln(y^2-1) + \ln A \Rightarrow y' = A \sqrt{y^2 - 1}
        \]
        分离变量后积分得
        \[
        \frac{dy}{\sqrt{y^2-1}} = A dx \Rightarrow \ln(y + \sqrt{y^2-1}) = A x + C_1.
        \]
        即
        \[
        y + \sqrt{y^2-1} = B e^{A x}
        \]
        解出$y$,得通解为
        \[
        y = \frac{B e^{A x} + \frac{1}{B} e^{-A x}}{2}
        \]
    \end{solution}

    \question 解
    \[
    y'' - y'\tan x + 2y = 0
    \]
    \begin{solution}
        观察到$y_1=\sin x$为原方程的解,由降阶法,$y_2=u(x)y_1$,其中
        \begin{align*}
        u(x)&=\int \frac{1}{\sin^2 x} e^{-\int -\tan x\,dx}\, dx \\
        &=\int \frac{1}{\sin^2 x} e^{-\ln|\cos x|}\, dx \\
        &=\int (\cot^2 x+1)\sec x \, dx \\
        &=\int (\sec x+\cot x \csc x) \, dx \\
        &=\ln |\sec x + \tan x| -\csc x
        \end{align*}
        故通解为
        \[
        y=\sin x(C_1+\ln |\sec x + \tan x|)+C_2
        \]
    \end{solution}

    \question 解微分方程
    \[
    \cos^2x \, \frac{d^2y}{dx^2} = 2y
    \]
    \begin{solution}
        观察得$y_1 = \tan x$满足方程
        \[
        y'' = 2y \sec^2x 
        \]
        现设$y=u(x)\tan x$,则$y'=u'\tan x + u \sec^2 x$,且
        \[
        y''=u''\tan x + 2u'\sec^2 x + 2u\sec^2 x \tan x =u''\tan x + 2u'\sec^2 x + y''
        \]
        即
        \[
        u''\tan x + 2u'\sec^2 x =0
        \]
        设$u' = z$,则
        \[
        - z'\tan x = 2z\sec^2 x
        \]
        分离变量得
        \[
        \int - \frac{z'}{2z} \, dz = \int \frac{\sec x}{\tan x} \, dx \Rightarrow - \frac{1}{2} \ln|z| = \ln|\tan x| + \ln A 
        \]
        化简得
        \[
        \frac{dx}{du}=\frac{1}{z}=A^2\tan^2 x
        \]
        再分离变量,
        \[
        \int \frac{1}{A^2} \cot^2 x \, dx = \int du \Rightarrow \frac{1}{A^2} \int (\csc^2 x - 1) \, dx = u \Rightarrow u= C (-\cot x - x) + B
        \]
        故通解为
        \[
        y = B\tan x - Cx\tan x - C
        \]
    \end{solution}

    \question 解微分方程
    \[
    \frac{dy}{dx}=2\left(\frac{y+2}{x+y+1}\right)^2
    \]
    \begin{solution}
        设$u=y+2, v=x+y+1$,则
        \[
        \frac{du}{dx}=\frac{dy}{dx}=2\left(\frac{u}{v}\right)^2, \quad
        \frac{dv}{dx}=1+\frac{dy}{dx}=1+2\left(\frac{u}{v}\right)^2.
        \]
        于是有
        \[
        \frac{dv}{du}=\frac{1+2\left(\frac{u}{v}\right)^2}{2\left(\frac{u}{v}\right)^2}=\frac{v^2+2u^2}{2u^2}=1+\frac{v^2}{2u^2}.
        \]
        令$w=\dfrac{v}{u}$,则 $v=wu$,并且$\dfrac{dv}{du}=w+u\dfrac{dw}{du}$,代入上式得
        \[
        u\frac{dw}{du}=\frac{w^2}{2}-w+1
        \]
        分离变量再积分得
        \[
        \frac{du}{u}=\frac{dw}{\frac{1}{2}\left((w-1)^2+1\right)} \Rightarrow \ln|u| = 2\arctan(w-1)+C
        \]
        把 $u,w$ 代回原变量,因此得到
        \[
        \ln|y+2| = 2\arctan\left(\frac{x-1}{y+2}\right) + C
        \]
    \end{solution}

    \question 解
    \[
    f(x) + f'(-x) = x^2 + \alpha, \quad \alpha \in \mathbb{R}
    \]
    \begin{solution}
        由
        \[
        f(x) + f'(-x) = x^2 + \alpha \tag{1}
        \]
        令$x \to -x$,得$f(-x)+f'(x)== x^2 + \alpha$,两边求导,
        \[
        -f'(-x) + f''(x) = 2x \tag{2}
        \]
        由$(1)+(2)$,
        \[
        f''(x)+f(x)=x^2 + 2x + \alpha
        \]
        齐次解为
        \[
        f_c=C_1\sin x +C_2\cos x
        \]
        且发现特解为
        \[
        f_p=x^2 + 2x + \alpha-2
        \]
        于是通解为
        \[
        f(x)=C_1\sin x +C_2\cos x+x^2 + 2x + \alpha-2
        \]
        尚未结束,与原方程比较系数得
        \[
        f'(-x) + f(x) = (C_1 + C_2)(\cos x + \sin x) + \sqrt{x^2 + \alpha} \Rightarrow C_2=-C_1
        \]
        故通解为
        \[
        f(x)=C_1(\sin x -\cos x)+x^2 + 2x + \alpha-2
        \]
    \end{solution}

    \question 解
    \[
    y = xy' + \sqrt{(y')^2 + 1}
    \]
    \begin{solution}
        此方程属于克莱罗方程(Clairaut's equation)。对$x$求导得
        \[
        y' = y' + xy'' + \frac{1}{x} \frac{x \cdot 2y'y''}{\sqrt{(y')^2 + 1}}
        \]
        整理得
        \[
        y'' \left(x + \frac{y'}{\sqrt{(y')^2 + 1}}\right) = 0
        \]
        若$y''=0$,则$y'=C \Rightarrow y=Cx+\sqrt{C^2+1}$。对于
        \[
        \left(x + \frac{y'}{\sqrt{(y')^2 + 1}}\right) = 0
        \]
        化简得
        \[
        y'=\pm\frac{x}{\sqrt{1-x^2}}
        \]
        照常分离系数后积分得
        \[
        \int dy = \pm \int \frac{x}{\sqrt{1-x^2}} \, dx \Rightarrow y= \pm \sqrt{1-x^2}+A
        \]
        将$y=\pm\sqrt{1-x^2}+A$代入原方程解得$A=0$,故解为
        \[
        y=\pm\sqrt{1-x^2} \Rightarrow x^2+y^2=1
        \]
    \end{solution}

    \question 求解
    \[
    \sqrt{\tan x} \,\frac{dy}{dx}=x
    \]
    \begin{solution}
        分离系数得,
        \[
        \frac{1}{y} \,dy =\sqrt{\cot x}\,dx \Rightarrow \ln y = \int \sqrt{\cot x} \,dx
        \]
        设$\cot x=u^2$,则$-\csc^2 x \,dx=2u\,du \Rightarrow dx=-\dfrac{2u\,du}{1+u^4}$,
        \[
        I=\int \sqrt{\cot x} \,dx=-\int \frac{2u\,du}{1+u^4}=-\int \frac{2\,du}{u^2+\frac{1}{u^2}}= - \int \frac{1 + \frac{1}{u^2}}{u^2 + \frac{1}{u^2}}\, du - \int \frac{1 - \frac{1}{u^2}}{u^2 + \frac{1}{u^2}}\, du=-(I_1+I_2)
        \]
        其中设$u-\dfrac{1}{u}=t,\left(1+\dfrac{1}{u^2}\right)\,du=dt$,则$u^2 + \dfrac{1}{u^2} = t^2 + 2$,
        \[
        I_1=\int \frac{1 + \frac{1}{u^2}}{u^2 + \frac{1}{u^2}}\, du= \int \frac{dt}{t^2 + 2} = \frac{1}{\sqrt{2}} \arctan \frac{t}{\sqrt{2}}= \frac{1}{\sqrt{2}} \arctan \left(\frac{u^2 - 1}{u\sqrt{2}}\right)
        = \frac{1}{\sqrt{2}} \arctan \left(\frac{\cot x - 1}{\sqrt{2} \cot x}\right)
        \]
        且设$u+\dfrac{1}{u}=\phi,\left(1-\dfrac{1}{u^2}\right)\,du=d\phi$,则$u^2 + \dfrac{1}{u^2} = \phi^2 - 2$,
        \[
        I_2=\int \frac{1 - \frac{1}{u^2}}{u^2 + \frac{1}{u^2}}\, du
        I_2 = \int \frac{d\phi}{\phi^2 - 2}
        = \frac{1}{2\sqrt{2}} \int \left(\frac{1}{\phi - \sqrt{2}} - \frac{1}{\phi + \sqrt{2}}\right) d\phi
        = \frac{1}{2\sqrt{2}} \ln \left|\frac{\phi - \sqrt{2}}{\phi + \sqrt{2}}\right|
        \]
        \[
        = \frac{1}{2\sqrt{2}} \ln \left| \frac{u^2 - u\sqrt{2} + 1}{u^2 + u\sqrt{2} + 1} \right| = \frac{1}{2\sqrt{2}} \ln \left| \frac{\cot x - \sqrt{2}\cot x + 1}{\cot x + \sqrt{2}\cot x + 1} \right|
        \]
        于是$\ln y=I$给出
        \[
        y = Ce^{\displaystyle -\frac{1}{2\sqrt{2}} \arctan\left(\frac{\cot x - 1}{\sqrt{2} \cot x}\right)} \left(\frac{\cot x + \sqrt{2}\cot x + 1}{\cot x - \sqrt{2}\cot x + 1}\right)^{\frac{1}{2\sqrt{2}}}
        \]
    \end{solution}

    \question 解
    \[
    y'' - y' - 2y = 10 \sin x, \quad y(0)=0, \quad y'(0)=1.
    \]
    \begin{solution}
        对应齐次方程的特征方程为
        \[
        r^2 - r - 2 =(r-2)(r+1)=0 \Rightarrow r=2,-1
        \]
        因此齐次通解为
        \[
        y_c(x) = C_1 e^{2x} + C_2 e^{-x}.
        \]
        取特解
        \[
        y_p(x) = A \cos x + B \sin x
        \]
        代入方程
        \[
        y_p'' - y_p' - 2y_p = (B - 3A)\cos x + (-A - 3B)\sin x=0
        \]
        比较系数解得$A=-1, B=-3$,因此特解为
        \[
        y_p(x) = -\cos x - 3 \sin x
        \]
        通解为
        \[
        y(x) = y_c + y_p = C_1 e^{2x} + C_2 e^{-x} - \cos x - 3 \sin x.
        \]
        使用初值条件 $y(0)=0,y'(0)=1$,
        \[
        C_1 + C_2 - 1 = 0 \Rightarrow C_1 + C_2 = 1.
        \]
        \[
        y'(x) = 2C_1 e^{2x} - C_2 e^{-x} + \sin x - 3 \cos x \implies y'(0) = 2C_1 - C_2 - 3 = 1.
        \]
        解得$C_1 = 1, C_2 = 0$,因此初值问题的解为
        \[
        y(x) = e^{2x} - \cos x - 3 \sin x
        \]
    \end{solution}

    \question 设初值条件为 $y(0)=2, y'(0)=0$,求解方程
    \[
    y''+4y=16x\cos 2x
    \]
    \begin{solution}
        对应的齐次方程为$y''+4y=0$,其通解为
        \[
        y_c(x)=C_1 \cos 2x + C_2 \sin 2x
        \]
        非齐次项为 $16x\cos 2x$,原方程的特解包含了$A_0 \cos 2x + B_0 \sin 2x$,将与$y_c(x)$重叠。
        因此特解需取
        \[
        y_p(x)=x\bigl((A_0+A_1x)\cos 2x+(B_0+B_1x)\sin 2x\bigr).
        \]
        将 $y_p$ 代入原方程,比较系数可得$A_0=1,A_1=0,B_0=0,B_1=2$,因此特解为
        \[
        y_p(x)=x\cos 2x+2x^2\sin 2x.
        \]
        所以通解为
        \[
        y(x)=y_c+y_p=(C_1+x)\cos 2x+(C_2+2x^2)\sin 2x
        \]
        由 $y(0)=2$得$C_1=2$,求导得
        \[
        y'(x)=(2x - 2C_1) \sin2x + (4x^2 + 2C_2 + 1) \cos2x
        \]
        代入 $x=0,y'(0)=0$,得到$C_2=-\dfrac12$,因此满足初值条件的解为
        \[
        y(x)=(x+2)\cos 2x+\left(2x^2-\frac12\right)\sin 2x
        \]
    \end{solution}

    \question 设连续函数 $f(x)$ 满足
    \[
    f(x)=\frac14 x^{2}+\frac12\cos x-\int_{0}^{x}(x-t)f(t)\,dt,
    \]
    求 $f(x)$。  
    \begin{solution}
        即解微分方程
        \[
        y''(x) + y(x) = \frac{1 - \cos x}{2},\;y(0) = \frac{1}{2}, \; y'(0) = 0
        \]
        通解为
        \begin{align*}
        y = y_h + y_p = C_1\cos x + C_2\sin x + \frac{1}{2} - \frac{1}{4}x\sin x
        \end{align*}
        代入初值
        \[
        y(0) = C_1 + \frac{1}{2} = \frac{1}{2} \Rightarrow C_1 = 0 ;\;
        y'(0) = C_2 + 0 = 0 \Rightarrow C_2 = 0
        \]
        因此
        \[
        f(x) = \frac{1}{2} - \frac{1}{4}x\sin x
        \]
    \end{solution}

    \question 考虑二阶线性齐次方程 
    \[
    y'' + p(x)y' + q(x)y = 0,
    \]
    其中 $p(x),q(x)$ 在某区间 $I$ 上连续。
    \begin{parts}
    \part 证明:若存在常数 $r$ 使得对所有 $x\in I$ 有
    \[
    r^2 + r p(x) + q(x)=0,
    \]
    则 $y(x)=e^{rx}$ 为该方程的一个解。
    \begin{solution}
        直接代入检验,对于 $y=e^{rx}$,
        \[
        y' = r e^{rx},\quad y'' = r^2 e^{rx}.
        \]
        代入方程得
        \[
        y'' + p(x)y' + q(x)y = (r^2 + r p(x) + q(x))e^{rx} = 0,
        \]
        因此 $y=e^{rx}$ 确为方程的解。
    \end{solution}
    \part 求出当
    \[
    p(x)=-2\left(1+\frac{1}{x}\right),\quad q(x)=1+\frac{2}{x}
    \]
    时方程的通解。
    \begin{solution}
        先求常数根 $r$. 将形式 $r^2 + r p(x) + q(x)=0$ 代入:
        \[
        r^2 + r\left(-2\Bigl(1+\frac{1}{x}\Bigr)\right) + 1+\frac{2}{x} = 0 \Rightarrow (r^2-2r+1) + \frac{2}{x}(1-r)=0.
        \]
        该等式对所有 $x$ 成立,比较系数得
        \[
        r^2-2r+1=0,\quad 1-r=0.
        \]
        由第二式得 $r=1$,并满足第一式,于是其中一解为
        \[
        y_1(x)=e^{x}
        \]
        由降阶法公式,第二个线性独立解为
        \[
        y_2 = y_1 \int \frac{1}{y_1^2} \exp{\left(\int p(x)\,dx\right)}\, dx = e^x \int e^{-2x} \exp{\left(2\int (1+\frac{1}{x})\,dx\right)}= e^x \int x^2\, dx = \frac{1}{3}x^3 e^x
        \]
        因此通解为
        \[
        y(x) = (C_1 + C_2 x^3)e^{x}
        \]
    \end{solution}
    \end{parts}

    \question 以换元 
    \[
    x=\int e^{-\frac{t^2}{2}}\,dt,
    \]
    解微分方程
    \[
    y''(t)+t y'(t) + e^{-t^2} y(t)=0
    \]
    \begin{solution}
        令 $z(x)=y(t)$,由链导法,
        \[
        y'=\frac{dy}{dt}=\frac{dz}{dx}\frac{dx}{dt}=e^{-\frac{t^2}{2}}\frac{dz}{dx}
        \]
        且
        \begin{align*}
        y'' = \frac{d}{dt}\left(e^{-\frac{t^2}{2}}\frac{dz}{dx}\right) 
        &= -t e^{-\frac{t^2}{2}}\frac{dz}{dx} 
        + e^{-\frac{t^2}{2}}\frac{d}{dt}\left(\frac{dz}{dx}\right) \\
        &= -t e^{-\frac{t^2}{2}}\frac{dz}{dx}
        + e^{-\frac{t^2}{2}}\frac{d}{dx}\left(\frac{dz}{dx}\right)\frac{dx}{dt} \\
        &= -t e^{-\frac{t^2}{2}}\frac{dz}{dx}
        + e^{-\frac{t^2}{2}} \frac{d^2z}{dx^2} \, e^{-\frac{t^2}{2}} \\
        &= -t e^{-\frac{t^2}{2}}\frac{dz}{dx}
        + e^{-t^2}\frac{d^2z}{dx^2}.
        \end{align*}
        把 $y',y''$ 代入原方程得
        \[
        \left(-t e^{-\frac{t^2}{2}}\frac{dz}{dx} + e^{-t^2}\frac{d^2z}{dx^2}\right)
        + t\left(e^{-\frac{t^2}{2}}\frac{dz}{dx}\right) + e^{-t^2}z 
        = e^{-t^2}\frac{d^2z}{dx^2} + e^{-t^2}z = 0 \Rightarrow \frac{d^2z}{dx^2} + z = 0
        \]
        解为
        \[
        z(x)=C_1\cos x + C_2\sin x \Rightarrow y(t)=C_1\cos\left(\int e^{-\frac{t^2}{2}}\,dt\right)+C_2\sin\left(\int e^{-\frac{t^2}{2}}\,dt\right)
        \]
    \end{solution}

    \question 解方程
    \[
    t y''(t) + (t^2-1) y'(t) + t^3 y(t) = 0
    \]
    \begin{solution}
        欲求合适的$x=v(t)$,令$z(x)=y(t)$,则
        \[
        y'=\frac{dy}{dt}=\frac{dz}{dx}\frac{dx}{dt}=\frac{dz}{dx}\frac{dv}{dt},
        \]
        且
        \begin{align*}
        y'' = \frac{d}{dt}\left(\frac{dv}{dt}\frac{dz}{dx}\right) 
        &= \frac{d^2v}{dt^2}\frac{dz}{dx} + \frac{dv}{dt}\frac{d}{dt}\left(\frac{dz}{dx}\right) \\
        &= \frac{d^2v}{dt^2}\frac{dz}{dx} + \frac{dv}{dt}\frac{d}{dx}\left(\frac{dz}{dx}\right)\frac{dx}{dt} \\
        &= \frac{d^2v}{dt^2}\frac{dz}{dx} + \left(\frac{dv}{dt}\right)^2\frac{d^2z}{dx^2}
        \end{align*}
        代入原方程,得到
        \[
        t\left(\frac{dv}{dt}\right)^2 \frac{d^2z}{dx^2} + \Bigl[t\frac{d^2v}{dt^2} + (t^2-1)\frac{dv}{dt}\Bigr]\frac{dz}{dx} + t^3 z = 0
        \]
        为使方程为常系数方程, 不妨考虑变换
        \[
        t\left(\frac{dv}{dt}\right)^2 = t^3 \Rightarrow \frac{dv}{dt} = t \Rightarrow v = \frac{t^2}{2}
        \]
        此时原方程可化为
        \[
        \frac{d^2z}{dx^2} + \frac{dz}{dx} + z = 0
        \]
        其通解为
        \[
        z(x) = e^{-\frac{x}{2}}\left(C_1 \cos \frac{\sqrt{3}}{2}x + C_2 \sin \frac{\sqrt{3}}{2}x\right)
        \]
        代回 $x=\frac{t^2}{2}$,即
        \[
        y(t) = e^{-\frac{t^2}{4}}\left(C_1 \cos \frac{\sqrt{3} t^2}{4} + C_2 \sin \frac{\sqrt{3} t^2}{4}\right)
        \]
    \end{solution}

    \question 设 $y_1,y_2$ 为方程
    \[
    y'' + p(x)y' + q(x)y = 0,\quad x>0 \tag{1}
    \]
    的两个解,其中 $p(x),q(x)$ 在 $x>0$ 上连续。已知
    \[
    y_1 = \frac{\sin x}{\sqrt{x}}
    \]
    且$y_1,y_2$的朗斯基行列式(Wronskian)为
    \[
    W(x)=\frac{1}{x}
    \]
    \begin{parts}
    \part 证明阿贝尔恒等式(Abel's Identity):
    \[
    W(x)=C\exp\!\left(-\int p(x)\,dx\right)
    \]
    \begin{solution}
        由朗斯基行列式的定义,
        \[
        W(x)=\det\begin{pmatrix}
        y_1 & y_2 \\
        y_1' & y_2'
        \end{pmatrix}
        = y_1y_2' - y_1'y_2
        \]
        于是
        \begin{align*}
        W'(x) &= (y_1y_2' - y_1'y_2)' = (y_1'y_2' + y_1y_2'') - (y_1''y_2 + y_1'y_2') \\
        &= y_1y_2'' - y_1''y_2 = y_1(-py_2' - qy_2) - (-py_1' - qy_1)y_2 \\
        &= -py_1y_2' + py_1'y_2 = -p(y_1y_2' - y_1'y_2) = -p(x)W(x)
        \end{align*}
        因此
        \[
        W'(x)+p(x)W(x)=0
        \]
        由此可解得
        \[
        W(x)=C\exp\!\left(-\int p(x)\,dx\right)
        \]
        即得证。
    \end{solution}
    \part 求方程(1)中的函数 $p(x),q(x)$。
    \begin{solution}
        由阿贝尔恒等式,
        \[
        p(x) = -\frac{W'(x)}{W(x)}
        = -\frac{-\frac{1}{x^2}}{\frac{1}{x}}
        = \frac{1}{x}
        \]
        由于 $y_1$ 是原方程的解,
        \[
        y_1''+py_1'+qy_1=0 \Rightarrow q(x) = -\frac{y_1'' + p y_1'}{y_1}= 1 - \frac{1}{4x^2}
        \]
        其中
        \[
        y'_1 = \frac{\cos x}{\sqrt{x}}- \frac{\sin x}{2x^{\frac{3}{2}}}, \quad y''_1 = \frac{3 \sin x}{4x^{\frac{5}{2}}} - \frac{\sin x}{\sqrt{x}} - \frac{\cos x}{x^{\frac{3}{2}}}
        \]
        因此原微分方程为
        \[
        y'' + \frac{1}{x}y' + \left(1 - \frac{1}{4x^2}\right)y = 0,\quad x>0
        \]
    \end{solution}
    \part 求方程(1)的通解。
    \begin{solution}
        利用朗斯基行列式求$y_2$。已知
        \[
        W(x) = y_1y_2' - y_1'y_2 = \frac{1}{x} \Rightarrow y'_2 - \frac{y'_1}{y_1} y_2 = \frac{1}{xy_1}
        \]
        即为一阶线性方程
        \[
        y_2' + \left(\frac{1}{2x} - \cot x\right) y_2 = \frac{1}{x\sin x}.
        \]
        其积分因子为
        \[
        I(x) = \exp\!\left(\int\left(\frac{1}{2x}-\cot x\right)dx\right)
        = \exp(\ln\sqrt{x} - \ln|\sin x|)
        = \pm \frac{\sqrt{x}}{\sin x}
        \]
        取$I(x)=\dfrac{\sqrt{x}}{\sin x}$,由此
        \begin{align*}
        y_2(x)
        &= \frac{\sin x}{\sqrt{x}}\left(\int \frac{\sqrt{x}}{\sin x}\cdot\frac{1}{x\sin x}dx + C\right) \\
        &= \frac{\sin x}{\sqrt{x}}\left(\int \frac{1}{\sin^2 x}dx + C\right) \\
        &= \frac{\sin x}{\sqrt{x}}(-\cot x + C) \\
        &= -\frac{\cos x}{\sqrt{x}} + C y_1
        \end{align*}
        因此通解为
        \[
        y(x) = C_1\frac{\sin x}{\sqrt{x}} + C_2\frac{\cos x}{\sqrt{x}}.
        \]
    \end{solution}
    \end{parts}

    \question 
    \begin{parts}
    \part 证明法国数学家刘维尔 (Liouville) 的结论:若黎卡提 (Riccati) 方程
    \[
    y' = p(x)y^2 + q(x)y+r(x)
    \]
    已知一个特解 $\bar{y}(x)$, 令 $y = z+ \bar{y}$,则用此代换可把方程化为一个阶数 $n = 2$的伯努利(Bernoulli) 方程,从而可通过解一个一阶线性方程得到一般解。
    \begin{solution}
        设 $y=z+\bar{y}$,代入原方程得
        \[
        z' + \bar{y}' = p(z+\bar{y})^{2} + q(z+\bar{y}) + r = pz^{2} + 2pz\bar{y} + qz + (p\bar{y}^{2}+q\bar{y}+r)
        \]
        由于 $\bar{y}$ 是原方程的一个解,故 $\bar{y}'=p\bar{y}^{2}+q\bar{y}+r$,于是
        \[
        z' = pz^{2} + 2pz\bar{y} + qz \Rightarrow
        z' - (2p\bar{y}+q)z = pz^{2}
        \]
        这正是阶数 $n=2$ 的伯努利方程。由通解公式,
        \[
        z^{1-n} = e^{-\int(1-n)(2p\bar{y}+q)\,dx}\left\{\int (1-n)p(x)e^{\int(1-n)(2p\bar{y}+q)\,dx}\,dx + C\right\}.
        \]
        代入 $n=2$ 即得
        \[
        z^{-1} = e^{-\int(2p\bar{y}+q)\,dx}\left\{ -\int p(x)e^{\int(2p\bar{y}+q)\,dx}\,dx + C\right\}.
        \]
        由此得到 $z$,再由 $y=z+\bar{y}$ 得原方程的通解,证毕。
    \end{solution}
    \part 求下列黎卡提方程的通解:
    \begin{subparts}
    \subpart $y'e^{-x} + y^{2} - 2ye^{x} = 1 - e^{2x}$,已知特解 $\bar{y}(x)=e^{x}$。 
    \begin{solution}
        原方程可写为
        \[
        y' = -e^{x}y^{2} + 2e^{2x}y + e^{x}(1-2e^{2x})
        \]
        因此
        \[
        p(x)=-e^{x},\quad q(x)=2e^{2x},\quad r(x)=e^{x}(1-2e^{2x}),\quad \bar{y}=e^{x}
        \]
        且
        \[
        \int (2p\bar{y} + q)\, dx = \int (2(-e^{x})e^{x} + 2e^{2x})\, dx  = 0
        \]
        于是上式积分指数因子为常数,代入公式得
        \[
        z^{-1} = -\int (-e^{x})\,dx + C = e^{x} + C \Rightarrow z = \frac{1}{C+e^{x}}
        \]
        代回 $y=z+\bar{y}$,得到通解
        \[
        y = e^{x} + \frac{1}{C+e^{x}}
        \]
    \end{solution}
    \subpart $x^{2}(y' + y^{2}) = 2$,已知特解 $\bar{y}(x)=-\dfrac{1}{x}$。
    \begin{solution}
        将方程写成标准形,
        \[
        y' = -y^{2} + \frac{2}{x^{2}}
        \]
        令 $y=z+\bar{y}=z-\dfrac{1}{x}$,代入并化简可得关于 $z$ 的方程
        \[
        z' - \frac{2}{x}z = -z^{2},
        \]
        是阶数 $n=2$ 的伯努利方程,由公式计算得
        \[
        z^{-1} = e^{\int \frac{2}{x}\,dx}\left( \int e^{-\int \frac{2}{x}\,dx}\,dx + C\right)
        = \frac{1}{x^{2}}\left( \int x^{2}\,dx + C\right)
        = \frac{1}{x^{2}}\left(\frac{x^{3}}{3} + C\right)
        \]
        即
        \[
        z = \frac{3x^{2}}{x^{3} + 3C}
        \]
        代回 $y=z-\frac{1}{x}$,化为
        \[
        y = \frac{3x^{2}}{x^{3} + 3C} - \frac{1}{x}
        = \frac{2x^{3} - 3C}{x(3x^{3} + 3C)}.
        \]
        令常数 $C_1 = 3C$,
        \[
        xy = \frac{2x^{3} - C_1}{3x^{3} + C_1}
        \]
    \end{solution}
    \end{subparts}
    \end{parts}

    \question 求解
    \[
    x^3-\frac{dy}{dx}=\frac{y}{x}(y-2)
    \]
    \begin{solution}
        发现此为一黎卡提方程,观察得$y_1=x^2$是一特解。设$y = x^2 + \dfrac{1}{u}$,则$\dfrac{dy}{dx}=2x-\dfrac{1}{u^2}\dfrac{du}{dx}$,原方程给出
        \[
        x^3-\left(2x-\frac{1}{u^2}\frac{du}{dx}\right)=x(x^2-2)+\frac{2(x^2-1)}{xu}+\frac{1}{xu^2}
        \]
        再化简得一阶线性方程
        \[
        \frac{du}{dx} -\left(2x-\frac{2}{x}\right)u = \frac{1}{x}
        \]
        积分因子为
        \[
        I(x)=\exp\left(\int -\left(2x-\frac{2}{x}\right)\,dx\right)=\exp\left(-x^2+2\ln|x|\right)=x^2 e^{-x^2}
        \]
        所以
        \[
        u=\frac{e^{-x^2}}{x^2}\left(\int x e^{-x^2}+C_1\right)
        = \frac{C_1 e^{x^2}-\frac{1}{2}}{x^2}
        \]
        回代 $y=x^2+\dfrac{1}{u}$ 得
        \[
        y = x^2 + \frac{x^2}{C_1 e^{x^2}-\tfrac{1}{2}}
        = x^2\frac{C_1 e^{x^2}+\tfrac{1}{2}}{C_1 e^{x^2}-\tfrac{1}{2}}.
        \]
        令 $C_2=2C_1$,则等价地写成
        \[
        y = x^2\frac{C_2 e^{x^2}+1}{C_2 e^{x^2}-1}
        \]
    \end{solution}

    \question 考虑柯西-欧拉方程(Cauchy-Euler Equation)
    \[
    ax^2y'' + bxy' + cy = 0, \quad x>0,
    \]
    一般解法为作变换 $x=e^t$ 或 $t=\ln x$,则原方程变为常系数方程
    \[
    az''(t) + (b-a)z'(t) + cz(t) = 0,
    \]
    解得 $z(t)$ 后代回 $y(x)=z(\ln x)$。据此,求解
    \begin{parts}
    \part $x^2y'' - 3xy' + 4y = 0$
    \begin{solution}
        方程为 $x^2y'' - 3xy' + 4y = 0$,所以 $a=1, b=-3, c=4$,对应 $z$ 方程为
        \[
        z'' - 4z' + 4z = 0
        \]
        特征方程为
        \[
        r^2 - 4r + 4 = (r-2)^2 = 0 \Rightarrow r=2
        \]
        因此解得
        \[
        z(t) = C_1 e^{2t} + C_2 t e^{2t} \Rightarrow y(x) = z(\ln x) = C_1 x^2 + C_2 x^2 \ln x
        \]
    \end{solution}
    \part $x^2y'' - xy' - 35y = 0$
    \begin{solution}
        方程为 $x^2y'' - xy' - 35y = 0$,所以 $a=1, b=-1, c=-35$。对应 $z$ 方程
        \[
        z'' + (b-a)z' + cz = z'' + (-1-1)z' - 35 z = z'' - 2 z' - 35 z = 0
        \]
        其中特征方程为
        \[
        r^2 - 2r - 35 = (r-7)(r+5)=0 \Rightarrow r_1 = 7, r_2 = -5
        \]
        因此通解为
        \[
        z(t) = C_1 e^{7t} + C_2 e^{-5t} \Rightarrow y(x) = z(\ln x) = C_1 x^7 + C_2 x^{-5}
        \]
    \end{solution}
    \end{parts}

    \question 解非齐次欧拉方程
    \[
    x^3 y'''+x^2 y'' - x y' = 24 x \ln x, \quad x>0
    \]
    \begin{solution}
        令 $x = e^t \Rightarrow t = \ln x$,由链导法,
        \[
        \frac{dy}{dx} = \frac{dy}{dt} \frac{dt}{dx} = \frac{1}{x} \frac{dy}{dt}
        \]
        \[
        \frac{d^2y}{dx^2} 
        = \frac{d}{dx}\left(\frac{dy}{dt} \frac{dt}{dx}\right) 
        = \frac{d}{dx}\left(\frac{1}{x} \cdot \frac{dy}{dt}\right) 
        = -\frac{dy}{x^2 dt} + \frac{1}{x} \frac{d^2y}{dt^2} \frac{1}{x} 
        = \frac{1}{x^2}\left(\frac{d^2y}{dt^2} - \frac{dy}{dt}\right) 
        \]
        \begin{align*}
        \frac{d^3y}{dx^3} 
        = \frac{d}{dx}\left(\frac{1}{x^2}\left(\frac{d^2y}{dt^2} - \frac{dy}{dt}\right)\right)
        &= -\frac{2}{x^3}\left(\frac{d^2y}{dt^2} - \frac{dy}{dt}\right) + \frac{1}{x^2}\left(\frac{d^3y}{dt^3}\cdot \frac{1}{x} - \frac{d^2y}{dt^2}\cdot \frac{1}{x}\right)\\
        &= \frac{1}{x^3} \frac{d^3y}{dt^3} - \frac{3}{x^3} \frac{d^2y}{dt^2} + \frac{2}{x^3} \frac{dy}{dt}
        \end{align*}
        代入方程得到
        \[
        \frac{d^3y}{dt^3} - 3 \frac{d^2y}{dt^2} + 3 \frac{dy}{dt} - y = 24 e^t t
        \]
        齐次方程$y''' - 3y'' + 3y' - y = 0$的特征方程 
        \[
        r^3 - 3r^2 + 3r - 1 = 0 \Rightarrow r=1
        \]
        故齐次解为
        \[
        y_c=C_1 e^t + C_2 t e^t + C_3 t^2 e^t
        \]
        设特解为
        \[
        y_p = t^3 (A t + B) e^t = (A t^4 + B t^3) e^t
        \]
        代入方程解得
        \[
        24 A t + 6B = 24 t \Rightarrow A=1, B=0 
        \]
        因此通解为
        \[
        y = t^4 e^t + C_1 e^t + C_2 t e^t + C_3 t^2 e^t
        \]
        代回 $t = \ln x$,原方程的解为
        \[
        y = x (\ln x)^4 + C_1 x + C_2 x \ln x + C_3 x (\ln x)^2
        \]
    \end{solution}

    \question 以参数变换法解
    \[
    y'' + 6y' + 9y = \frac{2 e^{-4x}}{x^2+1}.
    \]
    \begin{solution}
        对应齐次方程通解为
        \[
        y_c(x) = C_1 e^{-3x} + C_2 x e^{-3x}
        \]
        设特解为
        \[
        y_p(x) = u_1(x) e^{-3x} + u_2(x) x e^{-3x}
        \]
        其中 $u_1, u_2$ 满足
        \[
        \begin{bmatrix}
        e^{-3x} & x e^{-3x} \\
        -3 e^{-3x} & e^{-3x} - 3x e^{-3x}
        \end{bmatrix}
        \begin{bmatrix} u_1' \\ u_2' \end{bmatrix}
        =
        \begin{bmatrix} 0 \\ \frac{2 e^{-4x}}{x^2 + 1} \end{bmatrix}
        \]
        朗斯基行列式为
        \[
        W(y_1, y_2) = \begin{vmatrix} e^{-3x} & x e^{-3x} \\ -3 e^{-3x} & e^{-3x} - 3x e^{-3x} \end{vmatrix} = e^{-6x}
        \]
        由克兰姆法则,
        \[
        u_1' = \frac{- x e^{-3x} \cdot \frac{2 e^{-4x}}{x^2 + 1}}{e^{-6x}} = -\frac{2x}{x^2 + 1} \Rightarrow u_1 = -\ln(1+x^2),
        \]
        \[
        u_2' = \frac{e^{-3x} \cdot \frac{2 e^{-4x}}{x^2 + 1}}{e^{-6x}} = \frac{2}{x^2+1} \Rightarrow u_2 = 2 \tan^{-1} x.
        \]
        因此特解为
        \[
        y_p(x) = e^{-3x} \left(-\ln(1+x^2) + 2 x \tan^{-1} x \right)
        \]
        通解为
        \[
        y(x) = e^{-3x} \left(C_1 + C_2 x + 2 x \tan^{-1} x - \ln(1+x^2) \right)
        \]
    \end{solution}

    \question 求解
    \[
    y'' - 2y' + y = 4 e^x x^3 \ln x, \quad x>0
    \]
    \begin{solution}
        对应齐次方程的通解为
        \[
        y_c(x) = C_1 e^x + C_2 x e^x,
        \]
        设特解为
        \[
        y_p(x) = u_1(x) e^x + u_2(x) x e^x,
        \]
        其中 $u_1, u_2$ 满足
        \[
        \begin{bmatrix} e^x & x e^x \\ e^x & (x+1)e^x \end{bmatrix} 
        \begin{bmatrix} u_1' \\ u_2' \end{bmatrix} = 
        \begin{bmatrix} 0 \\ 4 e^x x^3 \ln x \end{bmatrix}.
        \]
        两边同时除以 $e^x$ 得
        \[
        \begin{bmatrix} 1 & x \\ 1 & x+1 \end{bmatrix} 
        \begin{bmatrix} u_1' \\ u_2' \end{bmatrix} =
        \begin{bmatrix} 0 \\ 4 x^3 \ln x \end{bmatrix}.
        \]
        朗斯基行列式为$(x+1)-x=1$,由克兰姆法则,
        \[
        u'_1 = -x \cdot 4x^{-2} \ln x = -4x^{-2} \ln x, \quad u'_2 = 4x^{-3} \ln x
        \]
        于是
        \[
        u_1 = -4 \int x^{-2} \ln x \, dx 
        = \frac{4 \ln x}{x} - 4 \int \frac{1}{x^2} \, dx = \frac{4(\ln x + 1)}{x}
        \]
        \[
        u_2 = 4 \int x^{-3} \ln x \, dx = -\frac{2 \ln x}{x^2} +2 \int \frac{1}{x^3} \, dx = \frac{-2 \ln x - 1}{x^2} 
        \]
        因此特解为
        \[
        y_p(x) = e^x u_1 + x e^x u_2 = e^x \left(\frac{4 (\ln x + 1)}{x} + x \cdot \frac{-2\ln x -1}{x^2}\right) = e^x \frac{2 \ln x + 3}{x}
        \]
        所以方程通解为
        \[
        y(x) = e^x \left(C_1 + C_2 x + \frac{2 \ln x + 3}{x}\right)
        \]
    \end{solution}

    \question 求解方程
    \[
    y'' + 4y' + 4y = 15 e^{-2x} \ln x + 25 \cos x, \quad x>0
    \]
    \begin{solution}
        对应齐次方程的通解为
        \[
        y_c(x) = C_1 e^{-2x} + C_2 x e^{-2x}
        \]
        第一项非齐次$f_1(x) = 15 e^{-2x} \ln x$:设特解为
        \[
        y_{p_1} = u_1(x) e^{-2x} + u_2(x) x e^{-2x}
        \]
        其中 $u_1,u_2$ 满足
        \[
        \begin{bmatrix} e^{-2x} & x e^{-2x} \\ -2e^{-2x} & (1-2x)e^{-2x} \end{bmatrix} 
        \begin{bmatrix} u_1' \\ u_2' \end{bmatrix} =
        \begin{bmatrix} 0 \\ 15 e^{-2x} \ln x \end{bmatrix}
        \]
        两边除以 $e^{-2x}$ 得
        \[
        \begin{bmatrix} 1 & x \\ -2 & 1-2x \end{bmatrix} 
        \begin{bmatrix} u_1' \\ u_2' \end{bmatrix} =
        \begin{bmatrix} 0 \\ 15 \ln x \end{bmatrix}
        \]
        解得
        \[
        u_1' = -15 x \ln x \Rightarrow u_1 = -15 \int x \ln x \, dx = -\frac{15}{2} x^2 \ln x + 15 \int \frac{x}{2}\,dx = \frac{15}{4} x^2(-2\ln x+1)
        \]
        \[
        u_2' = 15 \ln x \Rightarrow u_2 = 15 \int \ln x\, dx = 15 x (\ln x - 1)
        \]
        因此
        \[
        y_{p_1}(x) = e^{-2x} \left[ \frac{15}{4} x^2(-2\ln x+1) + 15 x (\ln x -1) \right] = \frac{15}{4} x^2 (2\ln x -3) e^{-2x}
        \]
        第二项非齐次$f_2(x) = 25 \cos x$:设特解
        \[
        y_{p_2}(x) = A \cos x + B \sin x
        \]
        代入方程得
        \[
        (3A+4B) \cos x + (-4A+3B) \sin x = 25 \cos x
        \]
        比较系数解得$A=3, B=4$,所以
        \[
        y_{p_2}(x) = 3 \cos x + 4 \sin x
        \]
        综上,方程通解为
        \[
        y(x) = C_1 e^{-2x} + C_2 x e^{-2x} + \frac{15}{4} x^2 (2\ln x -3) e^{-2x} + 3 \cos x + 4 \sin x
        \]
    \end{solution}

    \question 解
    \[
    y'''-y''+y'-y=e^{-x}\sin x
    \]
    \begin{solution}
        齐次方程
        \[
        y'''-y''+y'-y=0
        \]
        的特征方程为
        \[
        (r-1)(r^2+1)=0 \Rightarrow r=1, \pm i
        \]
        所以齐次解为
        \[
        y_c(x) = C_1 e^x + C_2 \cos x + C_3 \sin x
        \]
        对于非齐次项 $e^{-x}\sin x$,尝试特解
        \[
        y_p(x) = e^{-x}(A\cos x + B \sin x)
        \]
        则
        \[
        y_p'= -e^{-x}((A+B) \sin x+(A-B)\cos x)
        \]
        \[
        y_p''= 2e^{-x}(A\sin x + B \cos x)
        \]
        \[
        y_p'''= 2e^{-x}((B-A) \sin x+(A+B)\cos x)
        \]
        代入原方程解得$A=\dfrac{1}{5},B=0$,于是
        \[
        y_p(x) = -\frac{1}{5} e^{-x} \cos x
        \]
        因此通解为
        \[
        y(x) = C_1 e^x + C_2 \cos x + C_3 \sin x - \frac{1}{5} e^{-x} \cos x
        \]
    \end{solution}

    \question 求解微分方程
    \[
    y'''+3y''+3y'+y=\frac{2e^{-x}}{1+x^2}
    \]
    \begin{solution}  
        齐次解为
        \[
        y_c(x) = C_1 e^{-x} + C_2 x e^{-x} + C_3 x^2 e^{-x}
        \]
        使用参数变换法,设
        \[
        y_p(x) = u_1(x) e^{-x} + u_2(x) x e^{-x} + u_3(x) x^2 e^{-x}
        \]
        其中 $u_1, u_2, u_3$ 满足
        \[
        \begin{cases}
        e^{-x} u_1' + (xe^{-x}) u_2' + (x^2 e^{-x}) u_3' = 0 \\
        (e^{-x})' u_1' + (xe^{-x})' u_2' + (x^2 e^{-x})' u_3' = 0 \\
        (e^{-x})'' u_1' + (xe^{-x})'' u_2' + (x^2 e^{-x})'' u_3' = \dfrac{2e^{-x}}{1+x^2}
        \end{cases}
        \]
        化简得
        \[
        \begin{cases}
        u_1' + x u_2' + x^2 u_3' = 0 \\
        -u_1' + (1-x) u_2' + (2x - x^2) u_3' = 0 \\
        u_1' + (-2+x) u_2' + (2 - 4x + x^2) u_3' = \dfrac{2}{1+x^2}
        \end{cases}
        \]
        其中增广矩阵为
        \[
        \left[
        \begin{array}{ccc|c}
        1 & x & x^2 & 0 \\
        -1 & 1-x & 2x-x^2 & 0 \\
        1 & -2+x & 2-4x+x^2 & \dfrac{2}{1+x^2}
        \end{array}
        \right] \to
        \left[
        \begin{array}{ccc|c}
        1 & x & x^2 & 0 \\
        0 & 1 & 2x & 0 \\
        0 & 0 & 1 & \dfrac{1}{1+x^2}
        \end{array}
        \right]
        \]
        逐步解得
        \begin{align*}
        u_3' &= \frac{1}{1+x^2} \Rightarrow u_3 = \tan^{-1} x \\
        u_2' &= -2x u_3' = -\frac{2x}{1+x^2} \Rightarrow u_2 = -\ln(1+x^2) \\
        u_1' &= -x u_2' - x^2 u_3' = \frac{x^2}{1+x^2} \Rightarrow u_1 = x - 2 \tan^{-1} x
        \end{align*}
        于是特解为
        \[
        y_p = e^{-x} \left[ (2x - 2\tan^{-1}x) + (-\ln(1+x^2)) x + (\tan^{-1} x) x^2 \right]
        \]
        整理得通解为
        \[
        y(x) = e^{-x}\left[C_1 + C_2 x + C_3 x^2+(x^2-2)\tan^{-1}x + x-x\ln(1+x^2) \right]
        \]
    \end{solution}

    \question 解方程
    \[
    y'''+y' = \sec x
    \]
    \begin{solution}
        齐次方程$y'''+y'=0$的特征方程为
        \[
        r^3 + r = 0 \Rightarrow r = 0, \pm i
        \]
        所以齐次解为
        \[
        y_c = c_1 + c_2 \cos x + c_3 \sin x
        \]
        使用参数变易法,设特解为
        \[
        y_p = u_1 y_1 + u_2 y_2 + u_3 y_3
        \]
        其中 \(u_1, u_2, u_3\) 满足
        \[
        \begin{bmatrix}
        y_1 & y_2 & y_3 \\
        y_1' & y_2' & y_3' \\
        y_1'' & y_2'' & y_3''
        \end{bmatrix}
        \begin{bmatrix}
        u_1' \\ u_2' \\ u_3'
        \end{bmatrix}
        =
        \begin{bmatrix}
        1 & \cos x & \sin x \\
        0 & -\sin x & \cos x \\
        0 & -\cos x & -\sin x
        \end{bmatrix}
        \begin{bmatrix}
        u_1' \\ u_2' \\ u_3'
        \end{bmatrix}
        =
        \begin{bmatrix}
        0 \\ 0 \\ \sec x
        \end{bmatrix}
        \]
        其中增广矩阵为
        \[
        \left[
        \begin{array}{ccc|c}
        1 & \cos x & \sin x & 0 \\
        0 & -\sin x & \cos x & 0 \\
        0 & -\cos x & -\sin x & \sec x
        \end{array}
        \right]
        \implies
        \left[
        \begin{array}{ccc|c}
        1 & \cos x & \sin x & 0 \\
        0 & -\sin x & \cos x & 0 \\
        1 & 0 & 0 & \sec x
        \end{array}
        \right]
        \]
        于是
        \[
        \begin{cases}
        u'_1 = \sec x \\[2pt]
        u'_2 = \dfrac{\cos x}{-\sin x} u'_3 = -1 \\[6pt]
        u'_3 = \dfrac{\sin x}{\cos x}
        \end{cases}
        \Rightarrow
        \begin{cases}
        u_1 = |\sec x + \tan x| \\[1pt]
        u_2 = -x \\[1pt]
        u_3 = \ln |\cos x|
        \end{cases}
        \]
        因此通解为
        \[
        y = |\sec x + \tan x| - x\cos x + \ln |\cos x| \sin x + C_1 + C_2 \cos x + C_3 \sin x
        \]
    \end{solution}

    \question 解联立微分方程式
    \[
    \begin{cases}
    x_1' = x_1 + 2x_2 \\ 
    x_2' = 2x_1 - 2x_2 \\
    \end{cases}
    \]
    \begin{solution}
        \[
        x_1' = x_1 + 2x_2 \tag{1}
        \]
        \[
        x_2' = 2x_1 - 2x_2 \tag{2}
        \]
        对(1)求导,且由(2),
        \[
        x_1'' = x_1' + 2 x_2' = x_1' + 2(2x_1 - 2x_2) = x_1' + 4x_1 - 4x_2 = x_1' + 4x_1 - 2(x_1' - x_1) = -x_1' + 6x_1
        \]
        得关于$x_1$的二阶常微分方程,因此 $x_1$ 的通解为
        \[
        x_1(t) = C_1 e^{-3t} + C_2 e^{2t}
        \]
        由(1),
        \[
        x_2 = \frac{1}{2}(x_1' - x_1) = -2 C_1 e^{-3t} + \frac{C_2}{2} e^{2t}
        \]
        最终将系统的通解写成向量形式:
        \[
        \boldsymbol{x}(t) = 
        \begin{pmatrix} x_1 \\ x_2 \end{pmatrix}
        = C_1 \begin{pmatrix} 1 \\ -2 \end{pmatrix} e^{-3t} + C_2 \begin{pmatrix} 1 \\ \frac{1}{2} \end{pmatrix} e^{2t}
        \]
    \end{solution}

    \question 求解 
    \[
    \boldsymbol{x}' = \begin{pmatrix} 0 & 1 & 1 \\ 1 & 0 & 1 \\ 1 & 1 & 0 \end{pmatrix} \boldsymbol{x}
    \]
    \begin{solution}
        设系数矩阵为 $\boldsymbol{A}$,首先求 $\boldsymbol{A}$ 的特征值,通过解
        \[
        \det(\boldsymbol{A} - \lambda \boldsymbol{I}_3) =  \begin{vmatrix} -\lambda & 1 & 1 \\ 1 & -\lambda & 1 \\ 1 & 1 & -\lambda \end{vmatrix} 
        = -(\lambda + 1)^2(\lambda - 2)
        \]
        因此特征值为
        \[
        \lambda_1 = -1 \;(m_1=2), \quad \lambda_2 = 2
        \]
        接着求特征向量。对 $\lambda_1 = -1$,解
        \[
        (\boldsymbol{A} - \lambda_1 \boldsymbol{I}_3) \boldsymbol{v} = \boldsymbol{0} \implies 
        \begin{pmatrix} 1 & 1 & 1 \\ 1 & 1 & 1 \\ 1 & 1 & 1 \end{pmatrix} \begin{pmatrix} v_1 \\ v_2 \\ v_3 \end{pmatrix} = \begin{pmatrix} 0 \\ 0 \\ 0 \end{pmatrix}
        \]
        得到方程
        \[
        v_1 + v_2 + v_3 = 0
        \]
        有两个自由变量,设 $v_2 = r, v_3 = s$,则 $v_1 = -r - s$,因此特征空间由两个线性无关向量张成:
        \[
        \boldsymbol{v}_1 = \begin{pmatrix}-1\\1\\0\end{pmatrix}, \quad \boldsymbol{v}_2 = \begin{pmatrix}-1\\0\\1\end{pmatrix}
        \]
        对应解为
        \[
        \boldsymbol{x}_1(t) = e^{-t} \begin{pmatrix}-1\\1\\0\end{pmatrix}, \quad 
        \boldsymbol{x}_2(t) = e^{-t} \begin{pmatrix}-1\\0\\1\end{pmatrix}
        \]
        对 $\lambda_2 = 2$,解
        \[
        (\boldsymbol{A} - \lambda_2 \boldsymbol{I}_3) \boldsymbol{v} = \boldsymbol{0} \implies 
        \begin{pmatrix}-2 & 1 & 1 \\ 1 & -2 & 1 \\ 1 & 1 & -2\end{pmatrix} \begin{pmatrix}v_1\\v_2\\v_3\end{pmatrix} = \begin{pmatrix}0\\0\\0\end{pmatrix}
        \]
        化简矩阵得到行阶梯形:
        \[
        \begin{pmatrix} -2 & 1 & 1 \\ 1 & -2 & 1 \\ 1 & 1 & -2 \end{pmatrix}
        \xrightarrow{R_2 \to \frac{1}{2}R_1 + R_2}
        \begin{pmatrix} -2 & 1 & 1 \\ 0 & -\frac{3}{2} & \frac{3}{2} \\ 0 & \frac{3}{2} & -\frac{3}{2} \end{pmatrix}
        \xrightarrow[R_3 \to \frac{2}{3}R_3]{R_3 \to R_2 + R_3}
        \begin{pmatrix}-2 & 1 & 1 \\ 0 & -1 & 1 \\ 0 & 0 & 0\end{pmatrix}
        \]
        秩为 2,设 $v_3 = r$,则 $v_1 = v_2 = r$,得到特征向量
        \[
        \boldsymbol{v}_3 = \begin{pmatrix}1\\1\\1\end{pmatrix}
        \]
        对应解为
        \[
        \boldsymbol{x}_3(t) = e^{2t} \begin{pmatrix}1\\1\\1\end{pmatrix}
        \]
        因此系统的一般解为
        \[
        \boldsymbol{x}(t) = c_1 e^{-t} \begin{pmatrix}-1\\1\\0\end{pmatrix} 
        + c_2 e^{-t} \begin{pmatrix}-1\\0\\1\end{pmatrix} 
        + c_3 e^{2t} \begin{pmatrix}1\\1\\1\end{pmatrix}
        \]
    \end{solution}

    \question 求解
    \[
    \boldsymbol{x}' = \begin{pmatrix} -3 & 0 & 2 \\ 1 & -1 & 0 \\ -2 & -1 & 0 \end{pmatrix} \boldsymbol{x}
    \]
    \begin{solution}
        设系数矩阵为 $\boldsymbol{A}$。首先求特征值,通过解
        \[
        \det(\boldsymbol{A} - \lambda \boldsymbol{I}_3) 
        = \begin{vmatrix} -3-\lambda & 0 & 2 \\ 1 & -1-\lambda & 0 \\ -2 & -1 & -\lambda \end{vmatrix}
        = -(\lambda+2)(\lambda^2 + 2\lambda + 3)
        \]
        得特征值
        \[
        \lambda_{1,2} = -1 \pm \sqrt{2}i, \quad \lambda_3 = -2
        \]
        接着求对应特征向量。对 $\lambda_1 = -1+\sqrt{2}i$, 解
        \[
        (\boldsymbol{A} - \lambda_1 \boldsymbol{I}_3)\boldsymbol{v} = \boldsymbol{0} \implies 
        \begin{pmatrix} -2-\sqrt{2}i & 0 & 2 \\ 1 & -\sqrt{2}i & 0 \\ -2 & -1 & 1-\sqrt{2}i \end{pmatrix} \begin{pmatrix} v_1 \\ v_2 \\ v_3 \end{pmatrix} = \begin{pmatrix}0\\0\\0\end{pmatrix}
        \]
        化为行阶梯形矩阵
        \[
        \begin{pmatrix} -2-\sqrt{2}i & 0 & 2 \\ 1 & -\sqrt{2}i & 0 \\ -2 & -1 & 1-\sqrt{2}i \end{pmatrix} \xrightarrow{R_1 \to (-2+\sqrt{2}i)R_1/6} \begin{pmatrix} 1 & 0 & \frac{1}{3}(-2+\sqrt{2}i) \\ 1 & -\sqrt{2}i & 0 \\ -2 & -1 & 1-\sqrt{2}i \end{pmatrix}
        \]
        \[
        \xrightarrow[R_3 \to R_3 + 2R_1]{R_1 \to R_2 - R_1} \begin{pmatrix} 1 & 0 & \frac{1}{3}(-2+\sqrt{2}i) \\ 0 & -\sqrt{2}i & \frac{1}{3}(2-\sqrt{2}i) \\ 0 & -1 & -\frac{1}{3}(1+\sqrt{2}i) \end{pmatrix} \xrightarrow[R_3 \to -R_3]{R_2 \to R_2/(\sqrt{2}i)} \begin{pmatrix} 1 & 0 & \frac{1}{3}(-2+\sqrt{2}i) \\ 0 & 1 & \frac{1}{3}(1+\sqrt{2}i) \\ 0 & 1 & \frac{1}{3}(1+\sqrt{2}i) \end{pmatrix}
        \]
        \[
        \xrightarrow{R_3 \to R_3 - R_2} \begin{pmatrix} 1 & 0 & \frac{1}{3}(-2+\sqrt{2}i) \\ 0 & 1 & \frac{1}{3}(1+\sqrt{2}i) \\ 0 & 0 & 0 \end{pmatrix}
        \]
        令 $v_3 = r$ 为自由变量,则
        \[
        \boldsymbol{v} = r \begin{pmatrix}-\frac{1}{3}(-2+\sqrt{2}i) \\ -\frac{1}{3}(1+\sqrt{2}i) \\ 1 \end{pmatrix}.
        \]
        取 $r=-3$ 并利用欧拉公式分离实部和虚部
        \begin{align*}
        &e^{(-1+\sqrt{2}i)t} \begin{pmatrix} -2+\sqrt{2}i \\ 1+\sqrt{2}i \\ -3 \end{pmatrix} = e^{-t}(\cos(\sqrt{2}t) + i\sin(\sqrt{2}t)) \begin{pmatrix} -2+\sqrt{2}i \\ 1+\sqrt{2}i \\ -3 \end{pmatrix} \\
        &= e^{-t} \begin{pmatrix} -2\cos(\sqrt{2}t) - \sqrt{2}\sin(\sqrt{2}t) + i(\sqrt{2}\cos(\sqrt{2}t) - 2\sin(\sqrt{2}t)) \\ \cos(\sqrt{2}t) - \sqrt{2}\sin(\sqrt{2}t) + i(\sqrt{2}\cos(\sqrt{2}t) + \sin(\sqrt{2}t)) \\ -3\cos(\sqrt{2}t) - 3i\sin(\sqrt{2}t) \end{pmatrix} \\
        &= e^{-t} \begin{pmatrix} -2\cos(\sqrt{2}t) - \sqrt{2}\sin(\sqrt{2}t) \\ \cos(\sqrt{2}t) - \sqrt{2}\sin(\sqrt{2}t) \\ -3\cos(\sqrt{2}t) \end{pmatrix} + ie^{-t} \begin{pmatrix} \sqrt{2}\cos(\sqrt{2}t) - 2\sin(\sqrt{2}t) \\ \sqrt{2}\cos(\sqrt{2}t) + \sin(\sqrt{2}t) \\ -3\sin(\sqrt{2}t) \end{pmatrix}
        \end{align*}
        得到两个线性无关解
        \[
        \boldsymbol{x}_1(t) = e^{-t} \begin{pmatrix}-2\cos(\sqrt{2}t)-\sqrt{2}\sin(\sqrt{2}t) \\ \cos(\sqrt{2}t)-\sqrt{2}\sin(\sqrt{2}t) \\ -3\cos(\sqrt{2}t) \end{pmatrix}, 
        \quad
        \boldsymbol{x}_2(t) = e^{-t} \begin{pmatrix} \sqrt{2}\cos(\sqrt{2}t)-2\sin(\sqrt{2}t) \\ \sqrt{2}\cos(\sqrt{2}t)+\sin(\sqrt{2}t) \\ -3\sin(\sqrt{2}t) \end{pmatrix}
        \]
        对 $\lambda_3 = -2$,解
        \[
        (\boldsymbol{A} - \lambda_3 \boldsymbol{I}_3)\boldsymbol{v} = \boldsymbol{0} \implies 
        \begin{pmatrix}-1 & 0 & 2 \\ 1 & 1 & 0 \\ -2 & -1 & 2\end{pmatrix} \begin{pmatrix}v_1\\v_2\\v_3\end{pmatrix} = \begin{pmatrix}0\\0\\0\end{pmatrix}.
        \]
        化简矩阵得到
        \[
        \begin{pmatrix}-1 & 0 & 2 \\ 1 & 1 & 0 \\ -2 & -1 & 2\end{pmatrix} 
        \xrightarrow[R_3 \to R_3 - 2R_1]{R_2 \to R_2 +R_1}
        \begin{pmatrix}-1 & 0 & 2 \\ 0 & 1 & 2 \\ 0 & -1 & -2\end{pmatrix}
        \xrightarrow{R_3 \to R_3 + R_2}
        \begin{pmatrix}-1 & 0 & 2 \\ 0 & 1 & 2 \\ 0 & 0 & 0 \end{pmatrix}
        \]
        令 $v_3 = s$,得到特征向量
        \[
        \boldsymbol{v}_3 = s \begin{pmatrix}2\\-2\\1\end{pmatrix}.
        \]
        对应解为
        \[
        \boldsymbol{x}_3(t) = e^{-2t} \begin{pmatrix}2\\-2\\1\end{pmatrix}.
        \]
        因此系统的一般解为
        \begin{align*}
        \boldsymbol{x}(t) &= c_1e^{-t} \begin{pmatrix} -2\cos(\sqrt{2}t) - \sqrt{2}\sin(\sqrt{2}t) \\ \cos(\sqrt{2}t) - \sqrt{2}\sin(\sqrt{2}t) \\ -3\cos(\sqrt{2}t) \end{pmatrix} + c_2e^{-t} \begin{pmatrix} \sqrt{2}\cos(\sqrt{2}t) - 2\sin(\sqrt{2}t) \\ \sqrt{2}\cos(\sqrt{2}t) + \sin(\sqrt{2}t) \\ -3\sin(\sqrt{2}t) \end{pmatrix} \\
        &\quad + c_3e^{-2t} \begin{pmatrix} 2 \\ -2 \\ 1 \end{pmatrix}
        \end{align*}
    \end{solution}

    \question 判断矩阵 
    \[
    \boldsymbol{A} = \begin{pmatrix} 1 & 1 & 0 \\ 0 & 1 & 1 \\ 0 & 0 & 4 \end{pmatrix}
    \]
    是否缺陷,并求系统 $\boldsymbol{x}'(t) = \boldsymbol{A} \boldsymbol{x}(t)$ 的通解。
    \begin{solution}
        首先求 $\boldsymbol{A}$ 的特征值:
        \[
        \det(\boldsymbol{A} - \lambda \boldsymbol{I}_3) = \begin{vmatrix} 1-\lambda & 1 & 0 \\ 0 & 1-\lambda & 1 \\ 0 & 0 & 4-\lambda \end{vmatrix} = (1-\lambda)^2 (4-\lambda)
        \]
        因此特征值为 $\lambda_1 = 4,\lambda_2 = 1 \;(m_2=2)$。对于 $\lambda_1 = 4$, 解
        \[
        (\boldsymbol{A} - 4\boldsymbol{I}_3) \boldsymbol{v} = \boldsymbol{0} \implies 
        \begin{pmatrix}-3 & 1 & 0 \\ 0 & -3 & 1 \\ 0 & 0 & 0 \end{pmatrix} \begin{pmatrix} v_1 \\ v_2 \\ v_3 \end{pmatrix} = \begin{pmatrix}0\\0\\0\end{pmatrix}
        \]
        得到特征向量及其对应解
        \[
        \boldsymbol{v} = r \begin{pmatrix} 1 \\ 3 \\ 9 \end{pmatrix}, \quad r \neq 0, \quad \boldsymbol{x}_1(t) = e^{4t} \begin{pmatrix} 1 \\ 3 \\ 9 \end{pmatrix}
        \]
        对于 $\lambda_2 = 1$, 解
        \[
        (\boldsymbol{A} - \boldsymbol{I}_3) \boldsymbol{v} = \boldsymbol{0} \implies 
        \begin{pmatrix} 0 & 1 & 0 \\ 0 & 0 & 1 \\ 0 & 0 & 3 \end{pmatrix} \begin{pmatrix} v_1 \\ v_2 \\ v_3 \end{pmatrix} = \begin{pmatrix}0\\0\\0\end{pmatrix}
        \]
        得到特征向量及其对应解
        \[
        \boldsymbol{v} = s \begin{pmatrix} 1 \\ 0 \\ 0 \end{pmatrix}, \quad s \neq 0, \quad \boldsymbol{x}_2(t) = e^t \begin{pmatrix} 1 \\ 0 \\ 0 \end{pmatrix}
        \]
        总共有两个线性无关特征向量,因此矩阵 $\boldsymbol{A}$ 是缺陷的。为求第三个线性无关解,设
        \[
        \boldsymbol{x}_3(t) = (\boldsymbol{v} t + \boldsymbol{w}) e^t
        \]
        其中 $\boldsymbol{v}$ 和 $\boldsymbol{w}$ 满足
        \[
        (\boldsymbol{A} - \boldsymbol{I}_3) \boldsymbol{v} = \boldsymbol{0}, \quad (\boldsymbol{A} - \boldsymbol{I}_3) \boldsymbol{w} = \boldsymbol{v} \implies (\boldsymbol{A} - \boldsymbol{I}_3)^2 \boldsymbol{w} = \boldsymbol{0}
        \]
        解得
        \[
        (\boldsymbol{A} - \boldsymbol{I}_3)^2 \boldsymbol{w} = \begin{pmatrix} 0 & 0 & 1 \\ 0 & 0 & 3 \\ 0 & 0 & 9 \end{pmatrix} \begin{pmatrix} w_1 \\ w_2 \\ w_3 \end{pmatrix} = \begin{pmatrix}0\\0\\0\end{pmatrix} \implies w_3 = 0
        \]
        自由选择 $w_1, w_2$,取
        \[
        \boldsymbol{w} = \begin{pmatrix} 0 \\ 1 \\ 0 \end{pmatrix}, \quad \boldsymbol{v} = (\boldsymbol{A} - \boldsymbol{I}_3) \boldsymbol{w} = \begin{pmatrix} 1 \\ 0 \\ 0 \end{pmatrix}
        \]
        因此得到第三个解
        \[
        \boldsymbol{x}_3(t) = (\boldsymbol{v} t + \boldsymbol{w}) e^t = e^t \begin{pmatrix} t \\ 1 \\ 0 \end{pmatrix}
        \]
        最终通解为
        \[
        \boldsymbol{x}(t) = c_1 e^{4t} \begin{pmatrix} 1 \\ 3 \\ 9 \end{pmatrix} + c_2 e^t \begin{pmatrix} 1 \\ 0 \\ 0 \end{pmatrix} + c_3 e^t \begin{pmatrix} t \\ 1 \\ 0 \end{pmatrix}
        \]
    \end{solution}

    \question 求解非齐次系统
    \[
    \boldsymbol{x}'(t) = \boldsymbol{A} \boldsymbol{x}(t) + \boldsymbol{F}(t),
    \]
    其中
    \[
    \boldsymbol{A} = \begin{pmatrix} 0 & 1 \\ -9 & 6 \end{pmatrix}, \quad \boldsymbol{F}(t) = \begin{pmatrix} 0 \\ t \end{pmatrix}.
    \]
    \begin{solution}
        首先求 $\boldsymbol{A}$ 的特征值和特征向量:
        \[
        \det(\boldsymbol{A} - \lambda \boldsymbol{I}_2) = \begin{vmatrix} -\lambda & 1 \\ -9 & 6-\lambda \end{vmatrix} = -\lambda(6-\lambda)+9 = (\lambda-3)^2
        \]
        因此特征值为 $\lambda = 3\;(m=2)$,对 $\lambda = 3$, 解
        \[
        (\boldsymbol{A}-3\boldsymbol{I}_2) \boldsymbol{v} = \begin{pmatrix}-3 & 1 \\ -9 & 3\end{pmatrix} \begin{pmatrix} v_1 \\ v_2 \end{pmatrix} = \begin{pmatrix}0 \\ 0\end{pmatrix}
        \]
        得到特征向量
        \[
        \boldsymbol{v} = r \begin{pmatrix} 1 \\ 3 \end{pmatrix}, \quad r \neq 0
        \]
        对应解为
        \[
        \boldsymbol{x}_1(t) = e^{3t} \begin{pmatrix} 1 \\ 3 \end{pmatrix}
        \]
        矩阵 $\boldsymbol{A}$ 是缺陷的,第二个线性无关解取形如
        \[
        \boldsymbol{x}_2(t) = (\boldsymbol{v} t + \boldsymbol{w}) e^{3t}
        \]
        其中 $\boldsymbol{v}$ 和 $\boldsymbol{w}$ 满足
        \[
        (\boldsymbol{A}-\lambda\boldsymbol{I}_2)\boldsymbol{v} = \boldsymbol{0}, \quad (\boldsymbol{A}-\lambda\boldsymbol{I}_2)\boldsymbol{w} = \boldsymbol{v} \implies (\boldsymbol{A} - \lambda \boldsymbol{I}_2)^2\boldsymbol{w} = \boldsymbol{0}
        \]
        解得
        \[
        \begin{pmatrix} -3 & 1 \\ -9 & 3 \end{pmatrix}\begin{pmatrix} -3 & 1 \\ -9 & 3 \end{pmatrix} \begin{pmatrix} w_1 \\ w_2 \end{pmatrix} = \begin{pmatrix} 0 \\ 0 \end{pmatrix} \\
        \implies \begin{pmatrix} 0 & 0 \\ 0 & 0 \end{pmatrix} \begin{pmatrix} w_1 \\ w_2 \end{pmatrix} = \begin{pmatrix} 0 \\ 0 \end{pmatrix}.
        \]
        取
        \[
        \boldsymbol{w} = \begin{pmatrix} 0 \\ 1 \end{pmatrix} \implies \boldsymbol{v} = (\boldsymbol{A}-3\boldsymbol{I}_2)\boldsymbol{w} = \begin{pmatrix} 1 \\ 3 \end{pmatrix}
        \]
        于是第二个解为
        \[
        \boldsymbol{x}_2(t) = (\boldsymbol{v} t + \boldsymbol{w}) e^{3t} = \begin{pmatrix} t \\ 3t+1 \end{pmatrix} e^{3t}
        \]
        因此齐次系统通解为
        \[
        \boldsymbol{x}_c(t) = c_1 e^{3t} \begin{pmatrix} 1 \\ 3 \end{pmatrix} + c_2 e^{3t} \begin{pmatrix} t \\ 3t+1 \end{pmatrix} = \begin{pmatrix} e^{3t} & t e^{3t} \\ 3 e^{3t} & (3t+1)e^{3t} \end{pmatrix} \begin{pmatrix} c_1 \\ c_2 \end{pmatrix}
        \]
        记基础矩阵为 $\boldsymbol{X}$,寻找特解
        \[
        \boldsymbol{x}_p(t) = \boldsymbol{X} \boldsymbol{u} \implies \boldsymbol{X} \boldsymbol{u}' = \boldsymbol{F} \implies \boldsymbol{u}' = \boldsymbol{X}^{-1} \boldsymbol{F}
        \]
        计算得到
        \[
        \boldsymbol{X}^{-1}\boldsymbol{F} = e^{-3t} \begin{pmatrix} 3t+1 & -t \\ -3 & 1 \end{pmatrix} \begin{pmatrix} 0 \\ t \end{pmatrix} = e^{-3t} \begin{pmatrix} -t^2 \\ t \end{pmatrix}
        \]
        \[
        \boldsymbol{u} = \int \boldsymbol{X}^{-1} \boldsymbol{F}\, dt = \begin{pmatrix} \int -t^2 e^{-3t} dt \\ \int t e^{-3t} dt \end{pmatrix} = e^{-3t} \begin{pmatrix} \frac{1}{3} t^2 + \frac{2}{9} t + \frac{2}{27} \\ -\frac{1}{3} t - \frac{1}{9} \end{pmatrix}
        \]
        于是得到特解
        \[
        \boldsymbol{x}_p(t) = \boldsymbol{X}\boldsymbol{u} = \begin{pmatrix} 1 & t \\ 3 & 3t+1 \end{pmatrix} \begin{pmatrix} \frac{1}{3} t^2 + \frac{2}{9} t + \frac{2}{27} \\ -\frac{1}{3} t - \frac{1}{9} \end{pmatrix} = \begin{pmatrix} \frac{1}{9} t + \frac{2}{27} \\ \frac{1}{9} \end{pmatrix}.
        \]
        最终非齐次系统通解为
        \[
        \boldsymbol{x}(t) = \boldsymbol{x}_p(t) + \boldsymbol{x}_c(t) = \begin{pmatrix} \frac{1}{9} t + \frac{2}{27} \\ \frac{1}{9} \end{pmatrix} + c_1 e^{3t} \begin{pmatrix} 1 \\ 3 \end{pmatrix} + c_2 e^{3t} \begin{pmatrix} t \\ 3t+1 \end{pmatrix}.
        \]
    \end{solution}



\end{questions}
