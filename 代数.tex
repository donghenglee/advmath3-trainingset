\pagecolor{PageColor}
\
\vfil
\hfil  {\fontsize{50pt}{36pt}\selectfont{代数}} \hfil 
\vfil
\begin{tikzpicture}[remember picture,overlay,every node/.style={inner sep=0pt}]
        \node [shift={(1cm,-1cm)},brown,scale=2,anchor=north west] (CNW)
        at (current page.north west) {\pgfornament[height=1cm,width=1cm]{61}};
        \node [shift={(-1cm,-1cm)},brown,scale=2,anchor=north east] (CNE)
        at (current page.north east) {\pgfornament[height=1cm,width=1cm,symmetry=v]{61}};
        \node [shift={(1cm,1cm)},brown,scale=2,anchor=south west] (CSW)
        at (current page.south west) {\pgfornament[height=1cm,width=1cm,symmetry=h]{61}};
        \node [shift={(-1cm,1cm)},brown,scale=2,anchor=south east] (CSE)
        at (current page.south east) {\pgfornament[height=1cm,width=1cm,symmetry=c]{61}};
        \pgfornamentline[color=brown]{current page.north west}{current page.north east}{2}{87}
        \pgfornamentline{current page.south west}{current page.south east}{2}{87}
        \pgfornamentline{current page.north west}{current page.south west}{3}{87}
        \pgfornamentline{current page.north east}{current page.south east}{3}{87}
        \end{tikzpicture}%
\thispagestyle{empty}
\pagebreak

\begin{center}
  {\fontsize{30pt}{26pt}\selectfont
    \hypertarget{函数}{函数} \label{函数}
  }
\end{center}
\separator
\vspace{1pt}
\nopagecolor
\begin{questions}
    \question 求下列函数的值域:
    \begin{parts}
    \part $f(x) = x^2-2x+5$,其中$D_f= [-1,2]$
    \begin{solution}
        配方法得
        \[
        f(x) = x^2-2x+5 = (x-1)^2+4
        \]
        当$x=1,f_{\min}=4$;当$x=-1,f_{\max}=8$,故
        \[
        R_f = [4,8]
        \]
    \end{solution}
    \part $f(x) = x+\sqrt{x(2-x)}$
    \begin{solution}
        设$y=x+\sqrt{x(2-x)}$,则
        \[
        2x^2-2(y+1)x+y^2=0
        \]
        由于$x\in \mathbb{R}$,判别式为非负,
        \[
        4(y+1)^2-8y \ge 0 \Rightarrow 1-\sqrt{2} \le y \le 1+\sqrt{2}
        \]
        但$0 \le x \le 2, y=x+\sqrt{x(2-x)} \ge 0$,故$y_{\min}=0$。 

        而当$y=1+\sqrt{2},x_1=\dfrac{2+\sqrt{2}-2\sqrt[4]{2}}{\sqrt{2}} \in [0,2]$,故$y_{\max}=1+\sqrt{2}$。

        于是
        \[
        R_f = [0,1+\sqrt{2}]
        \]
    \end{solution}
    \part $f(x) = \dfrac{3x+4}{5x+6}$
    \begin{solution}
        设$y=\dfrac{3x+4}{5x+6}$,则$x=\dfrac{4-6y}{5y-3}$,故反函数
        \[
        f^{-1}(x)=\frac{4-6x}{5x-3}
        \]
        的定义域为
        \[
        D_{f^{-1}}=\left(-\infty,\frac{3}{5}\right)\cup\left(\frac{3}{5},\infty\right)
        \]
        即$R_f$。
    \end{solution}
    \part $f(x) = \dfrac{3x+2}{x+1}$
    \begin{solution}  
        发现
        \[
        f(x) = \dfrac{3x+2}{x+1} = 3 - \frac{1}{x+1} \neq 3
        \]
        故
        \[
        R_f=(-\infty,3) \cup (3,\infty)
        \]
    \end{solution}
    \part $f(x) = \dfrac{\cos x}{\sin x-3}$
    \begin{solution}
        设$y=\dfrac{\cos x}{\sin x-3}$,则
        \[
        3y=y \sin x -\cos x=\sqrt{y^2+1}\cos(x-\alpha)
        \Rightarrow \cos(x-\alpha)=\frac{3y}{\sqrt{y^2+1}} 
        \]
        由于$\cos(x-\alpha)\in [-1,1]$,解得
        \[
        -1 \le \frac{3y}{\sqrt{y^2+1}} \le 1 \Rightarrow -\frac{\sqrt{2}}{4} \le y \le \frac{\sqrt{2}}{4}
        \]
        即
        \[
        R_f = \left[-\frac{\sqrt{2}}{4},\frac{\sqrt{2}}{4}\right]
        \]
    \end{solution}
    \part $f(x) = 2^{x-5}+\log_3 \sqrt{x-1}$,其中$D_f= [2,10]$
    \begin{solution}
        由于$2^{x-5}$与$\log_3 \sqrt{x-1}$在$[2,10]$上皆为增函数,故$f(x)$也为增函数。

        当$x=2,f(x)=\dfrac{1}{8}$;当$x=10,f(x)=33$。故
        \[
        R_f=\left[\frac{1}{8},33\right]
        \]
    \end{solution}
    \part $f(x) = \sqrt{x+1}-\sqrt{x-1}$
    \begin{solution}
        发现
        \[
        f(x) = \frac{2}{\sqrt{x+1}+\sqrt{x-1}}
        \]
        由$x\ge 1$,得$\sqrt{x+1}+\sqrt{x-1} \ge 2$,于是
        \[
        0 < \frac{2}{\sqrt{x+1}+\sqrt{x-1}} \le \frac{2}{\sqrt{2}} = \sqrt{2}
        \]
        即
        \[
        R_f=(0,\sqrt{2}]
        \]
    \end{solution}
    \part $f(x) = x+2+\sqrt{1-(x+1)^2}$
    \begin{solution}
        由于$1-(x+1)^2 \ge 0$,有$(x+1)^2 \le 1$,不妨设 $x+1 = \cos \alpha,\alpha \in [0,\pi]$,则
        \[
        f(x) = \cos \alpha + 1 + \sqrt{1-\cos^2 \alpha} = \sin \alpha + \cos \alpha + 1 = \sqrt{2}\cos\left(\alpha-\frac{\pi}{4}\right)+1
        \]
        由$\alpha \in [0,\pi]$,可知
        \[
        -\frac{\sqrt{2}}{2} \le \cos\left(\alpha-\frac{\pi}{4}\right) \le 1 \Rightarrow 0 \le \sqrt{2}\cos\left(\alpha-\frac{\pi}{4}\right) +1\le 1+\sqrt{2}
        \]
        即
        \[
        R_f=[0,1+\sqrt{2}]
        \]
    \end{solution}
    \part $f(x) = \dfrac{x^3-x}{x^4+2x^2+1}$
    \begin{solution}
        首先有
        \[
        f(x)=\frac{1}{2}\cdot \frac{2x}{1+x^2} \cdot \frac{1-x^2}{1+x^2}
        \]
        令$x=\tan \beta$,则
        \[
        f(x)=\frac{1}{2}\sin 2\beta \cos 2\beta=\frac{1}{4}\sin 4\beta
        \]
        若$k \in \mathbb{Z}$,当$\beta=\dfrac{k\pi}{2}-\dfrac{\pi}{8},f_{\max}=\dfrac{1}{4}$;当$\beta=\dfrac{k\pi}{2}+\dfrac{\pi}{8},f_{\min}=-\dfrac{1}{4}$,且此时$\tan \beta$有意义,于是
        \[
        R_f=\left[-\frac{1}{4},\frac{1}{4}\right]
        \]
    \end{solution}
    \part $f(x) = x+4+\sqrt{5-x^2}$
    \begin{solution}
        由$5-x^2 \ge 0$得$|x| \le \sqrt{5}$,令$x=\sqrt{5}\cos\alpha, \alpha \in [0,\pi]$,则
        \[
        f(x)=\sqrt{5}\cos\alpha+4+\sqrt{5}\sin\alpha=4+\sqrt{10}\cos\left(\alpha-\frac{\pi}{4}\right)
        \]
        由于$\alpha \in [0,\pi],\alpha-\dfrac{\pi}{4}\in\left[-\dfrac{\pi}{4},\,\dfrac{3\pi}{4}\right]$,故
        \[
        f_{\max}=4+\sqrt{10}\cdot 1,\quad f_{\min}=4+\sqrt{10}\cdot \left(-\frac{\sqrt2}{2}\right)
        \]
        故
        \[
        R_f=[4-\sqrt{5},4+\sqrt{10}]
        \]
    \end{solution}
    \part $f(x)= \dfrac{\sqrt{x+2}}{x+3}$
    \begin{solution}
        设$t=\sqrt{x+2},t \ge 0$,则$x+3=t^2+1$。当$t>0$,
        \[
        f(x)=\frac{t}{t^2+1} = \frac{1}{t+\frac{1}{t}} \le \frac{1}{2}
        \]
        等号成立当且仅当$t=1$即$x=-1$,于是
        \[
        0 < f(x) \le \frac{1}{2}
        \]
        而当$t=0$时,$f(x)=0$,故
        \[
        R_f=\left[0,\frac{1}{2}\right]
        \]
    \end{solution}
    \part $f(x) = (\sin x + 1)(\cos x + 1),x \in \left[-\dfrac{\pi}{12},\dfrac{\pi}{2}\right]$
    \begin{solution}
        由
        \[
        f(x)=\sin x \cos x +\sin x +\cos x +1
        \]
        令$t=\sin x +\cos x$,则$\sin x \cos x=\dfrac{1}{2}(t^2-1)$,化为
        \[
        f(x)=\frac{1}{2}(t^2-1)+t+1=\frac{1}{2}(t+1)^2
        \]
        又由$t=\sin x +\cos x=\sqrt{2}\cos\left(x-\dfrac{\pi}{4}\right),x \in \left[-\dfrac{\pi}{12},\dfrac{\pi}{2}\right]$知
        \[
        \frac{\sqrt{2}}{2}\le t\le \sqrt{2}
        \]
        当$t=\sqrt{2},f_{\max}=\dfrac{3}{2}+\sqrt{2}$;当$t=\dfrac{\sqrt{2}}{2},f_{\min}=\dfrac{3}{4}+\dfrac{\sqrt{2}}{2}$,于是
        \[
        R_f=\left[\frac{3}{4}+\frac{\sqrt{2}}{2},\frac{3}{2}+\sqrt{2}\right]
        \]
    \end{solution}
    \part $f(x)=\left(\sin x + \dfrac{1}{\sin x}\right)^2+\left(\cos x + \dfrac{1}{\cos x}\right)^2$
    \begin{solution}
        展开得
        \[
        f(x)=\sin^2 x + \cos^2 x + \csc^2 x + \sec^2 x + 4 = 7 + \tan^2 x + \cot^2 x 
        \]
        由AM-GM不等式,
        \[
        f(x) \ge 7 + 2 \sqrt{1} = 9
        \]
        等号成立当且仅当$\tan x =\cot x$即$x=k\pi \pm \dfrac{\pi}{4}, k \in \mathbb{Z}$,故
        \[
        R_f=[9,\infty)
        \]
    \end{solution}
    \part $f(x)=2 \sin x \sin 2x$
    \begin{solution}
        由AM-GM不等式,
        \[
        [f(x)]^2=16\sin^4 x \cos^2 x = 8\sin ^2 x \sin ^2 x(2-2\sin ^2 x) \le 8\cdot \left(2\cdot \frac{1}{3}\right)^3=\frac{64}{27}
        \]
        等号成立当且仅当$\sin ^2 x =2-2\sin ^2 x$即$\sin ^2 x=\dfrac{2}{3}$,故
        \[
        R_f=\left[-\frac{8\sqrt{3}}{9},\frac{8\sqrt{3}}{9}\right]
        \]
    \end{solution}
    \part $f(x)=|x-2|+|x+8|$
    \begin{solution}
        在一维数轴上设$A=-8,B=2$,当动点$P$在线段$AB$上,
        \[
        f(x)=|x-2|+|x+8| = |AB| = 10
        \]
        当动点$P$在线段$AB$外,
        \[
        f(x)=|x-2|+|x+8| > |AB| = 10
        \]
        故
        \[
        R_f = [10,\infty)
        \]
    \end{solution}
    \part $f(x)=\sqrt{x^2-6x+13}+\sqrt{x^2+4x+5}$
    \begin{solution}
        写成
        \[
        f(x)=\sqrt{(x-3)^2+(0-2)^2}+\sqrt{(x+2)^2+(0+1)^2}
        \]
        即$x$轴上的动点$P(x,0)$到两定点$A(3,2),B(-2,-1)$的距离之和,而当$P$为线段与$x$轴的交点时,
        \[
        f_{\min}=|AB|=\sqrt{(3+2)^2+(2+1)^2}=\sqrt{43}
        \]
        于是
        \[
        R_f = [\sqrt{43},\infty)
        \]
    \end{solution}
    \part $f(x)=\sqrt{x^2-6x+13}-\sqrt{x^2+4x+5}$
    \begin{solution}
        写成
        \[
        f(x)=\sqrt{(x-3)^2+(0-2)^2}-\sqrt{(x+2)^2+(0-1)^2}
        \]
        即$AP$距离与$BP$距离之差,其中$A(3,2),B(-2,1),P(x,0)$。当$P$在$x$轴上且不是直线$AB$与$x$轴的交点时,即$P'$,则构成$\triangle ABP'$,有
        \[
        \bigl||AP'|-|BP'|\bigr|<|AB|=\sqrt{(3+2)^2+(2-1)^2}=\sqrt{26}
        \]
        即
        \[
        -\sqrt{26} < f(x) < \sqrt{26}
        \]
        而当$P$恰好在直线$AB$与$x$轴的交点时,有
        \[
        \bigl||AP'|-|BP'|\bigr|=|AB|=\sqrt{26}
        \]
        故
        \[
        R_f=(-\sqrt{26},\sqrt{26}]
        \]
    \end{solution}
    \part $f(x) = x^2+\sqrt{x^4-3x^2+2x+5}$
    \begin{solution}
        变形可得
        \[
        f(x)=x^2+\sqrt{(x^2-2)^2+(x+1)^2}
        \]
        发现$f(x)$表示抛物线$y=x^2$上动点$P(x,x^2)$到点$A(-1,2)$和$x$轴的距离之和。
        
        过$P$点作$PB \perp x$轴于$B$, 过$A$点作$AC \perp x$轴于$C$,
        $BC$交抛物线$y=x^2$于点$P_0$,
        故
        \[
        |PA|+|PB| \ge |P_0A|+|P_0C| = |AC| = 2
        \]
        所以
        \[
        R_f = [2,\infty)
        \]
    \end{solution}
    \end{parts}

    \question 设函数 $f:\mathbb{R}\setminus\{0,-b\}\to \mathbb{R}\setminus\{1\}$满足
    \[
    f(x) = \frac{x+a}{x+b}
    \]
    求出所有实数对 $(a,b)$ 使得
    \[
    f(f(x)) = -\frac{1}{x}
    \]
    \begin{solution}
        据题意,
        \[
        f(f(x)) = \frac{\frac{x+a}{x+b} + a}{\frac{x+a}{x+b} + b}
        = \frac{(1+a)x + a + ab}{(1+b)x + a + b^2} = -\frac{1}{x}
        \]
        整理得
        \[
        (1+a)x^2 + (a(1+b) + 1 + b)x + a + b^2 = 0.
        \]
        解得
        \[
        \begin{cases}
        1 + a = 0 \\
        a + b^2 = 0 
        \end{cases}
        \Rightarrow a = -1,b = \pm 1
        \]
        经检验,$(-1, -1)$ 不合题意,故解为$(-1, 1)$
    \end{solution}

    \question 已知函数 $f,g:\mathbb{R} \to \mathbb{R}$ 互为反函数,且函数
    \[
    F(x) = f(x+1)-2, \quad G(x) = g(2x+1),
    \]
    也互为反函数,若 $f(1)=4$,求 $f(100)$。
    \begin{solution}
        由于 $F,G$ 互为反函数,有
        \[
        G(F(x))=g(2F(x)+1)=g(2(f(x+1)-2)+1)=x
        \]
        又$f,g$ 互为反函数,则
        \[
        2f(x+1)-3=f(x) \Rightarrow f(x+1) = \frac{1}{2}f(x) + \frac{3}{2}
        \]
        由 $f(1)=4$,可依次得到
        \[
        f(2)=3+\frac{1}{2}, \quad f(3)=3+\frac{1}{4}, \dots
        \]
        归纳可得
        \[
        f(n) = 3 + \frac{1}{2^{\,n-1}}, \ n \in \mathbb{N}
        \]
        因此
        \[
        f(100) = 3 + \frac{1}{2^{99}}
        \]
    \end{solution}

    \question 已知函数 $g(x)=2x-4$, 其反函数为 $g^{-1}$,且函数 $f$ 对任意实数 $x$ 皆满足
    \[
    g(f(g^{-1}(x))) = 2x^2 + 16x + 26,
    \]
    求 $f(\pi)$ 的值。
    \begin{solution}
        已知 $g$ 可逆,令$x=g(y)$,则$y=g^{-1}(x)$,由题意得
        \[
        g(f(y)) = 2(g(y))^2 + 16g(y) + 26 
        = 2(2y-4)^2 + 16(2y-4) + 26 = 8y^2 - 6
        \]
        于是有
        \[
        f(y) = g^{-1}(g(f(y)))= g^{-1}(8y^2 - 6) = \frac{8y^2 -6 +4}{2} = 4y^2 - 1
        \]
        所以
        \[
        f(\pi) = 4\pi^2 - 1
        \]
    \end{solution}
\begin{solution}
    由$y=g(x)=2x-4$,得
    \[
    x=g^{-1}(y) = \frac{y+4}{2}
    \]
    由$g(f(g(x))) = 2x^2 + 16x + 26$,
    \[
    f(g(x)) = g^{-1}(2x^2 + 16x + 26) = x^2 + 8x + 15 = (x+4)^2 - 1
    \]
    且
    \[
    f(g(x)) = f\left(\textcolor{red}{\frac{x+4}{2}}\right) = (x+4)^2 - 1
    \]
    令 $\dfrac{x+4}{2} = \pi$, 则
    \[
    f(\pi) = 4\pi^2 - 1.
    \]
    \textcolor{red}{请问是哪一行有误?}
\end{solution}

    % \question 求 $k$ 的求值范围,使得函数 $f(x) = x^3 + 2x^2 + kx - 1$ 反函数存在。
    % \begin{solution}
    %     对$f(x)$ 求导得
    %     $$ f'(x) = 3x^2 + 4x + k = 3\left(x + \frac{2}{3}\right)^2 + k - \frac{4}{3} $$
    %     只有当$k> \dfrac{4}{3}$时, $f(x)$ 单调递增,故$f(x)$在$\mathbb{R}$上一对一,反函数$f^{-1}(x)$ 存在。
        
    %     现考虑$k=\dfrac{4}{3}$,因$f'(x)=0$有唯一解$x=-\dfrac23$,
    %     且$f$在$(-\infty,-\dfrac23)$及$(-\dfrac23,\infty)$上严格递增,故$x=-\dfrac23$是一驻点。由图像法可知,$f(x)$在$\mathbb{R}$上一对一,此时反函数$f^{-1}(x)$ 存在。\textcolor{red}{(待验证)}
    
    %     综上,当$k\ge \dfrac{4}{3}$时,反函数存在。
    % \end{solution}


    \question 已知$f(x) + 2 f(4-x) = x + 8,$求 $f(16)$。 
    \begin{solution}
        令$x=-12,16$,可得联立方程
        \[
        \begin{cases}
            f(-12)+2(16)=-4\\f(16)+2f(-12)=24
        \end{cases}
        \]
        解得 $$f(16)= -\dfrac{32}{3}$$
    \end{solution}
        
    \question 若 $f(x)$ 为实系数二次多项式,已知 $p,q,r$ 为三相异非零实数,使得 $f(p)=qr,f(q)=rp,f(r)=pq$,证明$f(p+q+r) = f(p)+f(q)+f(r)$。
    \begin{solution}
        取 $g(x) = x f(x) - pqr$,则 $g(p) = g(q) = g(r) = 0$,且有
        \[
        g(x) = x f(x) - pqr = a(x - p)(x - q)(x - r)
        \]
        于是
        \[
        f(x) = \frac{1}{x} \left( a x^3 - a(p + q + r)x^2 + a(pq + qr + rp)x - apqr + pqr \right)
        \]
        由于 $f(x)$ 为二次多项式,因此 $-apqr + pqr = 0 \Rightarrow a = 1$,得
        \[
        f(x) = x^2 - (p + q + r)x + pq + qr + rp
        \]
        故得证
        \[
        f(p + q + r) = pq + qr + rp = f(p)+f(q)+f(r)
        \]
    \end{solution}
    
    \question 设 $Q(x)$ 是一个 $2017$ 次多项式,且满足
    \[
    Q'(r)=\frac{2017!}{r} \quad ,r=1,2,3,\cdots, 2017
    \]
    另定义
    \[
    P(x)=xQ(x)-\int Q(x)\,dx.
    \]
    若多项式 $P(x)$ 的所有根之和为 $a$,求 $a \pmod{1000}$。
    \begin{solution}
        有
        \[
        P'(x)=xQ'(x) +Q(x)-Q(x) \Rightarrow P(x) = \int xQ'(x)\,dx
        \]
        考虑函数
        \[
        R(x)=xQ(x)-2017!
        \]
        则$R(r)=0,r=1,2,3,\cdots, 2017$,故可设
        \[
        R(x)=A(x-1)(x-2)(x-3)\cdots(x-2017)
        \]
        令$x=0$可得$A=1$,故
        \[
        Q'(x)=\frac{(x-1)(x-2)(x-3)\cdots(x-2017)+2017!}{x}
        \]
        且
        \begin{align*}
        P(x) = \int xQ'(x)\,dx 
        &= \int ((x-1)(x-2)(x-3)\cdots(x-2017)+2017!)\,dx \\
        &=\int \left(x^{2017}-\frac{2017\cdot2018}{2}x^{2016} +\cdots\right)\,dx \\
        &=\frac{1}{2018}x^{2018} - 1009 x^{2017}+\cdots
        \end{align*}
        由韦达定理,$\ P(x)$ 的所有根之和为 
        \[
        a=1009 \cdot 2018 \equiv 162 \pmod{1000}
        \]
    \end{solution}

    \question 定义函数 $f: \mathbb{R} \to \mathbb{R}$,使得对所有实数 $x$ 有
    \[
    f(f(x)) = x^2 - x + 1.
    \]
    求 $f(0)$。
    \begin{solution}
        设 $f(0) = b$,则
        \[
        f(b) = f(f(0)) = 0^2 - 0 + 1 = 1
        \]
        又有
        \[
        f(f(b)) = f(1) = b^2 - b + 1
        \]
        因为 $f(b) = 1$,所以
        \[
        f(f(b)) = f(1) = 1 = b^2 - b + 1 \implies b(b-1) = 0.
        \]
        若 $b=0$,则 $f(0)=0$,但 $f(f(0)) = f(0) = 0 \neq 1$,矛盾。故 $b = 1$,即
        \[
        f(0) = 1
        \]
    \end{solution}

    \question 已知 $f(x) = e(x) + o(x)$,其中 $e(x)$ 为偶函数,$o(x)$ 为奇函数,且
    \[
    e(x) + x^2 = o(x),
    \]
    求 $f(2)$。
    \begin{solution}
        由已知$e(x)=f(x)-o(x)$,换元得
        \[
        e(-x)=f(-x)-o(-x) \Rightarrow e(x)=f(-x)+o(x)
        \]
        故
        \[
        f(-x)=-x^2 \Rightarrow f(2)=4
        \]
    \end{solution}

    \question 定义 $f$ 在 $[0,1]$ 上满足
    \[
    2f\left(\frac{x}{3}\right) = f(x), \quad f(x) + f(1-x) = 1,
    \]
    求 $f\left(\dfrac{1}{3}\right)$。
    \begin{solution}
        由递推公式,则
        \[
        f\left(\frac{1}{13}\right) + f\left(\frac{12}{13}\right) = 1,
        \]
        且
        \[
        f\left(\frac{12}{13}\right) 
        = 2 f\left(\frac{4}{13}\right) 
        = 2 \left[1-f\left(\frac{9}{13}\right)\right]
        = 2 - 2 f\left(\frac{9}{13}\right)
        = 2 - 4 f\left(\frac{3}{13}\right)
        = 2 - 8 f\left(\frac{1}{13}\right)
        \]
        解得
        \[
        f\left(\frac{1}{13}\right) = \frac{1}{7}.
        \]
    \end{solution}

    \question 已知$f(x)$是定义在$\mathbb{R}$上的函数,且对任意实数$x$均有$2f(x)+f(x^2-1)=1$,试求$f(\sqrt{2})$的值。
    \begin{solution}
        分别令 $x=-1,0,1,\sqrt{2}$ 可得
        \begin{align*}
            2f(-1) + f(0) = 1 \tag{1}\\
            2f(0) + f(-1) = 1 \tag{2}\\
            2f(1) + f(0) = 1 \tag{3}\\
            2f(\sqrt{2}) + f(1) = 1 \tag{4}
        \end{align*}
        由 $(1),(2),(3)$ 解得
        \[
        f(-1) = f(0)= f(1)= \frac{1}{3}
        \]
        故由(4)得
        \[
        f(\sqrt{2}) = \frac{1}{3}
        \]
        \end{solution}

    \question 设函数 $f(x)=\cos x+\log_{2}x,$其中$x>0$ ,若正实数 $a$ 满足 $f(a)=f(2a)$ ,求$f(2a)-f(4a)$ 的值。
    \begin{solution}
        由条件得 $$\cos a+\log_{2}a=\cos 2a+\log_{2}2a=2\cos^{2}a-1+\log_{2}a+1$$
        所以
        $$\cos a(2\cos a-1) =0\Rightarrow (\cos a,\cos 2a) = (0,-1) \text{或}\; (\frac{1}{2},-\frac{1}{2})$$
        于是
        \begin{align*}
        f(2a) - f(4a) 
        &= \cos 2a + \log_2 2a - \cos 4a - \log_2 4a \\
        &= \cos 2a - 2 \cos^2 2a\\
        &= \begin{cases}
            -3, & \text{若 } \cos 2a = -1, \\
            -1, & \text{若 } \cos 2a = -\frac{1}{2}
        \end{cases}
        \end{align*}
    \end{solution}

    \question 已知函数 $f(x)$ 的定义域为 $\mathbb{R}$, 且 $f(x+2)-2$ 为奇函数, $f(2x+1)$ 为偶函数。 若 $f(1)=0$, 求 $f(1)+f(2)+\cdots+f(2023)$ 的值。
    \begin{solution}
        $f(x+2)-2$ 为奇函数, 则
        \[
        f(-x+2)-2+f(x+2)-2=0 \Rightarrow f(2-x)+f(2+x)=4
        \]
        于是 $f(2)=2$, $f(x)$ 的图象关于点 $(2,2)$ 对称,又由 $f(2x+1)$ 为偶函数, 
        $$f(-2x+1)=f(2x+1)\Rightarrow f(1-2x)=f(1+2x)$$ 
        即
        $$f(1-x)=f(1+x)$$
        $f(x)$ 图象关于直线 $x=1$ 对称,
        由上可得 
        $$f(x)=f(2-x)=4-f(2+x)=4-f(-x)=4-[4-f(x+4)]=f(x+4)$$
        所以 $f(x)$ 是周期为4的函数,且
        \[
        f(1)=0,f(2)=2,f(3)=4,f(4)=2
        \]
        故
        \begin{align*}   
        &f(1)+f(2)+\cdots+f(2023)\\
        &=505[f(1)+f(2)+f(3)+f(4)]+f(1)+f(2)+f(3)\\
        &=4046 
        \end{align*}
    \end{solution}

    \question 设 $f$ 为实值函数,且对任意实数 $x$ 满足
\begin{enumerate}
\item $f(10+x) = f(10-x)$;
\item $f(20+x) = -f(20-x)$。
\end{enumerate}
证明 $f$ 是奇函数并且是周期函数。

\begin{solution}
令 $x = n - 10$ 在 (a) 中,则
\[
f(n) = f(20 - n)。
\]

又令 $x = n$ 在 (b) 中,则
\[
f(20 - n) = -f(20 + n)。
\]

因此
\[
f(n) = -f(20 + n)。
\]

再令 $x = n + 10$ 在 (a) 中,则
\[
f(n + 20) = f(-n)。
\]

结合上式可得
\[
f(n) = -f(20+n) = f(-n),
\]
所以 $f$ 是奇函数。

最后,令 $x = n - 20$ 在 (b) 中,则
\[
f(n) = -f(40 - n)。
\]

由于 $f$ 是奇函数,$-f(40-n) = f(n-40)$,所以
\[
f(n) = f(n-40),
\]
因此 $f$ 是周期函数,周期为 $40$。
\end{solution}


    \question 若整数 $m\ge 1$,函数 $f$ 满足
    \[
    f(m+1)=m(-1)^{m+1}-2f(m), \quad f(1)=f(2001),
    \]
    求 $f(1)+f(2)+\cdots+f(2000)$。
    \begin{solution}
        由递推式可得
        \[
        f(2)=1-2f(1),\quad f(3)=-2-2f(2),\quad f(4)=3-2f(3),\;\ldots,\quad f(2001)=2000-2f(2000),
        \]
        将 $f(2001)$ 替换为 $f(1)$ 并将所有式子相加,得
        \[
        \sum_{i=1}^{2000} f(i)=1-2+3-4+\cdots+1999-2000-2\sum_{i=1}^{2000} f(i).
        \]
        所以
        \[
        \sum_{i=1}^{2000} f(i)=\frac{1}{3}\left(\frac{2000}{2}-2000\right)=-\frac{1000}{3}.
        \]
    \end{solution}

    \question 若实数 \(x, y, z\) 满足
    \[
    \begin{cases}
    \displaystyle \frac{x}{1^2 + 4^2} + \frac{y}{1^2 + 5^2} + \frac{z}{1^2 + 6^2} = 1 \\[2mm]
    \displaystyle \frac{x}{2^2 + 4^2} + \frac{y}{2^2 + 5^2} + \frac{z}{2^2 + 6^2} = 1 \\[2mm]
    \displaystyle\frac{x}{3^2 + 4^2} + \frac{y}{3^2 + 5^2} + \frac{z}{3^2 + 6^2} = 1
    \end{cases}
    \]
    求 \(x + y + z\)。
    \begin{solution}
        设
        \[
        f(k) = \frac{x}{k + 4^2} + \frac{y}{k + 5^2} + \frac{z}{k + 6^2} - 1,
        \]
        据题意
        \[
        f(1^2) = f(1) = 0, \quad f(2^2) = f(4) = 0, \quad f(3^2) = f(9) = 0。
        \]
        因此 \(f(k)\) 有三个根\(k = 1, 4, 9\),所以
        \[
        (k - 1)(k - 4)(k - 9)
        \]
        与
        \[
        (k + 4^2)(k + 5^2)(k + 6^2) - x(k + 5^2)(k + 6^2) - y(k + 4^2)(k + 6^2) - z(k + 4^2)(k + 5^2)
        \]
        相等,比较 \(k^2\) 系数得
        \[
        - (1 + 4 + 9) = 4^2 + 5^2 + 6^2 - (x + y + z) \Rightarrow x + y + z = 91
        \]
    \end{solution}

    \question 已知实数 $\alpha$、$\beta$ 满足
    \[
    3^{\frac{\alpha}{2}} = \sqrt{3} - \alpha, \quad \log_3 \beta = 2\sqrt{3} - 2\beta
    \]
    求
    \[
    (\alpha + \beta)^2 - 2 (\sqrt{3})^\alpha - \log_3 \beta
    \]
    \begin{solution}
        将底改为 $\sqrt{3}$:
        \[
        \sqrt{3}^\alpha = \sqrt{3} - \alpha, \quad \log_{\sqrt{3}} \beta = \sqrt{3} - \beta
        \]
        设 $A(\alpha, \sqrt{3}-\alpha)$ 为两图形 $y = \sqrt{3}^x$ 与 $y = \sqrt{3} - x$ 的交点$,B(\beta, \sqrt{3}-\beta)$ 为两图形 $y = \log_{\sqrt{3}} x$ 与 $y = \sqrt{3}-x$ 的交点。  

        由于 $y = \sqrt{3}^x$ 与 $y = \log_{\sqrt{3}} x$ 关于 $y = x$ 对称,
        \[
        \alpha = \sqrt{3} - \beta \Rightarrow \alpha+\beta = \sqrt{3}
        \]
        因此
        \[
        (\alpha + \beta)^2 - 2 (\sqrt{3})^\alpha - \log_3 \beta = (\sqrt{3})^2 - 2(\sqrt{3}-\alpha) - 2(\sqrt{3}-\beta) = 3 - 2\sqrt{3}
        \]
    \end{solution}

    \question 设实数 $\alpha$、$\beta$ 满足
    \[
    \begin{cases}
    \alpha^3 - 6\alpha^2 + 13\alpha = 6 \\
    \beta^3 - 6\beta^2 + 13\beta = 14
    \end{cases}
    \]
    求 $\alpha + \beta$。

    \begin{solution}
        令 $$f(x) = x^3 - 6x^2 + 13x \Rightarrow f(\alpha) = 6,\ f(\beta) = 14$$

        又 $$f'(x) = 3x^2 - 12x + 13 \Rightarrow f''(x) = 6x - 12 = 0 \Rightarrow x = 2$$
        因此图形 $y = f(x)$ 的对称中心为 $(2, f(2) = 10)$,所以 $P(\alpha, 6), Q(\beta, 14)$的对称点为
        \[
        P' = (4 - \alpha, 14), Q' = (4 - \beta, 6)
        \]
        于是由$P = Q' ,\quad Q = P'$解得
        \[
        \alpha = 4 - \beta,\quad \beta = 4 - \alpha \Rightarrow \alpha + \beta = 4
        \]
    \end{solution}

    \question 解方程
    \[
    e^{x-1} \ln x + e^{y-1} \ln y = \ln(xy)
    \]
    \begin{solution}
        原式化为:
        \[
        (e^{x-1} - 1)\ln x + (e^{y-1} - 1)\ln y = 0
        \]
        考虑函数 \(f(t) = (e^{t-1} - 1)\ln t\),
        \begin{itemize}
        \item 当 \(t > 1\) 时,\(\;e^{t-1} - 1 > 0\),\(\;\ln t > 0\),故 \(f(t) > 0\)
        \item 当 \(0 < t < 1\) 时,\(\;e^{t-1} - 1 < 0\),\(\;\ln t < 0\),故 \(f(t) > 0\)
        \item 当 \(t = 1\) 时,\(\;e^{t-1} - 1 = 0\),\(\;\ln 1 = 0\),故 \(f(1) = 0\)
        \end{itemize}
        故 $$f(x) + f(y) = 0 \Rightarrow x = y = 1$$
    \end{solution}

    \question 已知$x,y \in \mathbb{R}$, 且满足
    \[
    \begin{cases}
    \;(x+1)^{3}+2023(x+1)=-2023 \\
    \;(y+1)^{3}+2023(y+1)=2023
    \end{cases}
    \]
    求$x+y$的值。
    \begin{solution}
    考虑函数 \( f(t) = t^3 + 2023t \),据题意有
        \[
        f(x+1) = -2023, \quad f(y+1) = 2023
        \]
        注意到 \( f(t) \) 是奇函数,
        \[
        f(x+1) = -f(y+1)= f(-(y+1))
        \]
        且导数为
        \[
        f'(t) = 3t^2 + 2023 > 0, \quad \forall\, t \in \mathbb{R}
        \]
        所以$f$在$\mathbb{R}$上严格递增,故$f$一对一,因此有
        \[
        x+1 = -(y+1) \Rightarrow x + y = -2
        \]
    \end{solution}

    \question 设函数 $f:(0,\infty)\rightarrow\mathbb{R}$,满足
    \[
    f\left(1-\frac{1}{1+t}\right) + f\left(\frac{1+t}{t}\right)\log(1+t) = f\left(\frac{1+t}{t}\right)\log t + 2022,
    \]
    求 $f(1000)$。
    \begin{solution}
        令 
        \[x = \frac{t}{1+t} = 1 - \frac{1}{1+t},
        \]
        则原式变为
        \[
        f(x) + f\left(\frac{1}{x}\right) \log(1+t) = f\left(\frac{1}{x}\right) \log t + 2022
        \]
        即
        \[
        f(x) - f\left(\frac{1}{x}\right) \log x = 2022 \tag{1}
        \]
        再换元得
        \[
        f\left(\frac{1}{x}\right) - f(x)\log \frac{1}{x} = 2022 \Rightarrow f\left(\frac{1}{x}\right) + f(x) \log x = 2022
        \]
        代入 (1)得
        \[
        f(x) - 2022 \log x + f(x)(\log x)^2 = 2022 \Rightarrow f(x) = \frac{2022 (1 + \log x)}{1 + (\log x)^2}
        \]
        故
        \[
        f(1000) = \frac{2022 \cdot 4}{1 + 9} = 808.8
        \]
    \end{solution}

    \question 设函数 \( f:\mathbb{R}\setminus\{0,1\} \to \mathbb{R} \) 使得
        \[
        f(x) + f\left(\frac{1}{1-x}\right) = \frac{1}{x(1-x)}.
        \]
        求 $f(x)$。
    \begin{solution}
        代入 \( x \), \( \dfrac{1}{1-x} \), 及\( \dfrac{x-1}{x} \) 得
        \[
        f(x) + f\left(\frac{1}{1-x}\right) = \frac{1}{x(1-x)} \tag{1} 
        \]
        \[
        f\left(\frac{1}{1-x}\right) + f\left(\frac{x-1}{x}\right) = -\frac{(x-1)^2}{x}\tag{2}
        \]
        \[
        f\left(\frac{x-1}{x}\right) + f(x) = \frac{x^2}{x-1} \tag{3}
        \]
        $(1)+(3)-(2)$,得
        \begin{align*}
            2f(x) &= \frac{1}{x(1-x)}+\frac{x^2}{x-1}+\frac{(x-1)^2}{x}\\
            &=\frac{-1+x^3+(x-1)^3}{x(x-1)}\\
            &=\frac{(x^2+x+1)+(x^2-2x+1)}{x}\\
            &=\frac{2x^2-x+2}{x}
        \end{align*}
        故
        \[
        f(x)=x+\frac1x-\frac12
        \]
    \end{solution}

    \question 已知函数 $f(x)$ 定义在 $\mathbb{R}$ 上, 且满足 $f(1)=1$, 对所有实数 $x$ 有
    \[
    f(x+5) \geq x+5, \quad f(x+1) \leq f(x)+1.
    \]
    若定义函数 $g(x) = f(x)+1 - x$, 求 $g(2002)$。     
    %https://artofproblemsolving.com/community/c1076938h2031392p14321295
    \begin{solution}
        $\forall x \in \mathbb{R}$,由 $f(x+1) \leq f(x)+1$,
        \[
        g(x+1)=f(x+1)+1-(x+1)\leq f(x)+1-x= g(x) 
        \]
        即$g(x)$是递减函数,且由$f(x+5)\geq x+5$,
        \[
        g(x+5)=f(x+5)+1 - (x+5) \geq x+5+1 - (x+5)=1 
        \]
        因为 $g(1)=f(1)+1-1=1$,故$g(x)=1\;\forall x \in \mathbb{R}$,即$g(2002)=1$
    \end{solution}

    \question 设$f: R\setminus\{0\}\to \mathbb{R}$,使得对于所有非零实数$x, y$有
    \[
    xy(f(x) - f(y)) + 2x = xf(x) - yf(y) + 2y
    \]
    若$f'(1) = 2018$,求$f(2018)$。
    \begin{solution}
        令$y=1$,有
        \[
        xf(x) - xf(1) + 2x= xf(x) - f(1) + 2 \Rightarrow f(1)=2
        \] 
        令$y=-1$,有
        \[
        -x(f(x) - f(-1)) + 2x = xf(x) + f(-1) -2 \Rightarrow f(x) = \frac{f(-1) + 2}{2} + \frac{2 - f(-1)}{2x}
        \]
        求导得
        \[
        f'(x) = \frac{f(-1) - 2}{2x^2} 
        \]
        故由
        \[
        f'(1) = \frac{f(-1) - 2}{2} = 2018
        \]
        得$f(-1)=4038$,于是
        \[
        f(x)=2020-\frac{2018}{x}
        \]
        经检验,原方程式左式=右式$=2020x-2018y$,故
        \[
        f(2018)=2019
        \]
    \end{solution}
    \begin{solution}
        将原式写成
        \[
        xf(x) (y-1) + 2x = yf(y) (x-1) + 2y 
        \]
        \[
        xf(x)(y-1) + 2(x-1) = yf(y)(x-1) + 2(y-1)
        \]
        \[
        \frac{xf(x)}{x-1} + \frac{2}{y-1} = \frac{yf(y)}{y-1} + \frac{2}{x-1}
        \]
        \[
        \frac{xf(x)-2}{x-1} = \frac{yf(y)-2}{y-1} \quad \forall x, y \neq 0, 1
        \]
        设
        \[
        \frac{xf(x)-2}{x-1} = c \Rightarrow f(x) = \frac{c(x-1)+2}{x} = c + \frac{2-c}{x}
        \]  
        求导得
        \[
        f'(x) = \frac{c-2}{x^2} 
        \]
        由$f'(1) = 2018$得 $c=2020$,故
        \[
        f(x)=2020-\frac{2018}{x}
        \]
        经检验,原方程式左式=右式$=2020x-2018y$,故
        \[
        f(2018)=2019
        \]
    \end{solution}
    
    \question 设函数 $f(x), g(x)$ 在 $\mathbb{R}$ 上连续且满足
    \[
    \begin{cases}
    f(x+y) = f(x)g(y) + g(x)f(y) \\
    g(x+y) = f(x)f(y) + g(x)g(y)
    \end{cases}
    \]
    且 $g(x)$ 为偶函数,$\;g(x) > 0$ 恒成立,$\;g(0) \ne \dfrac{1}{2}$,已知 $f(2) + g(2) = \dfrac{\sqrt{5}}{15}$,求 $f(8) - g(8)$。
    \begin{solution}
        两式相加得
        \[
        f(x+y) + g(x+y) = (f(x) + g(x))(f(y) + g(y))
        \]
        设 $h(x) = f(x) + g(x)$,则 $h(x+y) = h(x)h(y)$,得 $h(x) = a^x$。
        两式相减得
        \[
        g(x+y) - f(x+y) = (g(x) - f(x))(g(y) - f(y))
        \]
        设 $k(x) = g(x) - f(x)$,同理得 $k(x) = b^x$,于是
        \[
        f(x) = \frac{h(x) - k(x)}{2} = \frac{a^x - b^x}{2},\quad
        g(x) = \frac{h(x) + k(x)}{2} = \frac{a^x + b^x}{2}
        \]
        由于 $g(x)$ 为偶函数,$\,g(x) = g(-x)$,即
        \[
        \frac{a^x + b^x}{2} = \frac{a^{-x} + b^{-x}}{2} \Rightarrow (a^x + b^x) a^x b^x = b^x + a^x \Rightarrow (ab)^x = 1 \Rightarrow ab = 1,\ b = \frac{1}{a}
        \]
        故
        \[
        f(x) = \frac{a^x - a^{-x}}{2},\quad
        g(x) = \frac{a^x + a^{-x}}{2}
        \]
        由 $f(2) + g(2) = \dfrac{\sqrt{5}}{15}$ 可知
        \[
        h(2) = a^2 = \frac{\sqrt{5}}{15}
        \Rightarrow a^8 = \left( \frac{\sqrt{5}}{15} \right)^4 = \frac{1}{2025}
        \]
        所以
        \[
        f(8) - g(8) = -k(8) = -b^8 = -\frac{1}{a^8} = -2025
        \]
    \end{solution}

    \question  设 $a,b>0$使得关于 $x$ 的方程 $\sqrt{|x|}+\sqrt{|x+a|}=b$ 恰有三个相异实数解 $x_1,x_2,x_3$, 且 $x_1<x_2<x_3=b$, 求 $a+b$ 的值。
    \begin{solution}
        令 $t=x+\frac{a}{2}$, 则关于 $t$ 的方程 $$\sqrt{\left|t-\frac{a}{2}\right|}+\sqrt{\left|t+\frac{a}{2}\right|}=b$$ 恰有三个不同的实数解 $$t_i=x_i+\frac{a}{2},i=1,2,3$$

        由于 $f(t)=\sqrt{|t-\frac{a}{2}|}+\sqrt{|t+\frac{a}{2}|}$ 为偶函数, 故方程 $f(t)=b$ 的三个实数解关于原点对称分布, 从而必有 $b=f(0)=\sqrt{2a}$
        
        现求方程 $f(t)=\sqrt{2a}$ 的实数解:
        \begin{itemize}
            \item 当 $|t|\le\frac{a}{2}$ 时, $f(t)=\sqrt{\frac{a}{2}-t}+\sqrt{\frac{a}{2}+t}=\sqrt{a+\sqrt{a^2-4t^2}}\le\sqrt{2a}$, 等号成立当且仅当 $t=0$
            \item 当 $t>\frac{a}{2}$ 时, $f(t)$ 单调递增, 且当 $t=\frac{5a}{8}$ 时 $f(t)=\sqrt{2a}$
            \item 当 $t<-\frac{a}{2}$ 时, $f(t)$ 单调递减, 且当 $t=-\frac{5a}{8}$ 时 $f(t)=\sqrt{2a}$
        \end{itemize}
        从而方程 $f(t)=\sqrt{2a}$ 恰有三个实数解 $$t_1=-\frac{5a}{8},\ t_2=0,\ t_3=\frac{5a}{8}$$
        由条件知 $b=x_3=t_3-\frac{a}{2}=\frac{a}{8}$, 结合 $b=\sqrt{2a}$ 得 $a=128$,于是 $a+b=\frac{9a}{8}=144$
    \end{solution}  
\end{questions}

\pagebreak
\begin{center}
  {\fontsize{30pt}{26pt}\selectfont
    \hypertarget{一元二次方程、多项式}{一元二次方程、多项式} \label{一元二次方程、多项式}
  }
\end{center}
\separator
\vspace{1pt}

\begin{questions}
    \question 已知方程 
    \[
    x^2 - m x - m + 3 = 0
    \]
    的两根满足下列条件,求 $m$ 的取值范围:
    \begin{parts}
    \part 一根大于 $1$,另一根小于 $1$  
    \begin{solution}
        设$f(x)=x^2 - m x - m + 3$,所求即 
        \[
        f(1)=1-m-m+3<0 \Rightarrow m>2
        \]
    \end{solution}
    \part 一根小于 $0$,另一根大于 $2$  
    \begin{solution}
        即
        \[
        \begin{cases}
        f(0)= -m+3<0 \\
        f(2)= 4-2m-m+3<0
        \end{cases} \Rightarrow m > 3
        \]
    \end{solution}
    \part 一根在 $0$ 与 $1$ 之间,另一根在 $1$ 与 $2$ 之间  
    \begin{solution}
        即 
        \[
        \begin{cases}
        f(0)= -m+3>0 \\
        f(1)= 1-m-m+3<0 \\
        f(2)= 4-2m-m+3>0
        \end{cases} \Rightarrow 2<m<\frac{7}{3}
        \]
    \end{solution}
    \part 两根都在 $-4$ 与 $0$ 之间  
    \begin{solution}
        即  
        \[
        \begin{cases}
        f(-4)=16+4m-m+3>0 \\
        f(0)=-m+3>0 \\[1mm]
        \displaystyle -4<\frac{m}{2}<0 \\[2mm]
        \displaystyle f\left(\frac{m}{2}\right)=\frac{m^2}{4}-\frac{m^2}{2}-m+3\le 0
        \end{cases} \Rightarrow -\frac{19}{3}<m<-6
        \]
    \end{solution}
    \part 两根都大于 $-5$  
    \begin{solution}
        即  
        \[
        \begin{cases}
        f(-5)=25+5m-m+3>0 \\[1mm]
        \displaystyle \frac{m}{2}>-5 \\[2mm]
        \displaystyle f\left(\frac{m}{2}\right)=\frac{m^2}{4}-\frac{m^2}{2}-m+3\le 0
        \end{cases} \Rightarrow -7<m\le -6 \quad \text{或} \quad m\ge 2
        \]
    \end{solution}
    \part 有且仅有一根在 $0$ 与 $2$ 之间  
    \begin{solution}
        即  
        \[
        \begin{cases}
            \displaystyle 0<\frac{m}{2}<2, \\[2mm]
            \displaystyle f\left(\frac{m}{2}\right)=\frac{m^2}{4}-\frac{m^2}{2}-m+3=0
        \end{cases}   
        \]
        或
        \[
        f(0) \cdot f(2)=(-m+3)(4-2m-m+3)<0
        \]
        解得  
        \[
        m=2 \quad \text{或} \quad \frac{7}{3}<m<3
        \]
    \end{solution}
    \end{parts}

    \question 已知$y = x^3 - x^2 + 3x - 4$与$y = a x^2 - x - 4$恰好相交于两点, 求 $a$ 的可能值。 
    \begin{solution}
        联立方程得 \[ 
            x[x^2-(a+1)x+4]=0
        \]
        已知两线交于恰好两点,其中一点为$(0,4)$,则意味$$x^2-(a+1)x+4=0$$有重根,其判别式为零:$$[-(a+1)]^2-4\cdot1\cdot4=0$$解得$a=3, -5$
    \end{solution}
    
    \question 已知由 $f(x)=-x^{2}+4x+1$ 和 $g(x)=-x+5$ 所围成的封闭区域 $A$,若作一直线 $L$ 垂直于 $x$ 轴,分别与封闭区域 $A$ 的边界交于 $P,Q$ 两点,求在封闭区域内的 $\overline{PQ}$ 长的最大值为多少?
    \begin{solution}
        令 $f(x) = g(x)$,则
        \[
         -x^2 + 4x + 1 = -x + 5 \Rightarrow (x-1)(x-4) = 0 \Rightarrow x = 1,\,4
        \]
        所以封闭区域的 $x$ 范围为 $[1,4]$,而 $\overline{PQ}$ 即
        \[
        f(x) - g(x) = (-x^2 + 4x + 1) - (-x + 5) = -x^2 + 5x - 4 =-\left(x - \frac{5}{2}\right)^2 + \frac{9}{4}
        \]
        在 $x = \dfrac{5}{2}$时有最大值$\dfrac{9}{4}$
    \end{solution}

    \question 已知$k$为有理数,且使得方程式 $kx^{2}+(k-1)x+(k+1)=0$ 只有整数解,求$k$的所有可能值。
    \begin{solution}
        \textbf{情况一}:$k=0,$则原方程式为$-x+1=0 \Rightarrow x=1 \in \mathbb{Z}$

        \textbf{情况二}:$k\ne 0$,原方程式变为
        \[
        k(x^2+x+1)=x-1 \Rightarrow k=\frac{x-1}{x^2+x+1} \tag{1}
        \]
        设 $\alpha, \beta$ 为 $kx^2+(k-1)x+(k+1)=0$ 的两根,则
        \[
        \alpha+\beta = \frac{1-k}{k} = \frac{1}{k}-1 
        \]
        将 (1) 代入上式,
        \[
        \alpha+\beta = \frac{x^2+x+1}{x-1} - 1 =x+1+\frac{3}{x-1}\in \mathbb{Z}
        \]
        解得
        \[
        x-1=\pm1,\pm3
        \Rightarrow
        \begin{cases}
        x=2,4 \Rightarrow k=\dfrac{1}{7} \\
        x=0,-2 \Rightarrow k=-1
        \end{cases}
        \]
        故$k$的所有可能值为$-1,0,\dfrac{1}{7}$
    \end{solution}

    \question 若抛物线 $y = mx^2 - 1$ 上必存在相异两点对称于直线 $x+y=0$,求 $m$ 的范围。
    \begin{solution}
        设 $A, B$ 在抛物线 $\Gamma: y = mx^2 - 1$ 上且对称于直线 $L_1: x+y=0$,则 $A, B$ 同在直线 $$L_2: y = x + k$$ 上,其中 $L_1 \perp L_2$,抛物线与 $L_2$ 联立得
        \[
        mx^2 - x - 1 - k = 0 \tag{1}
        \]
        设 (1) 的解为 $a, b$,则$A(a, a+k), B(b, b+k)$的中点 $C\Big(\frac{a+b}{2}, \frac{a+b}{2}+k\Big)$ 在 $L_1$ 上,得
        \[
        \frac{a+b}{2} + \frac{a+b}{2} + k = 0 \Rightarrow k = -(a+b)
        \]
        由韦达定理,
        \[
        k = -(a+b) = -\frac{1}{m}
        \]
        且两实根相异,则判别式大于 $0$:
        \[
        \Delta = (-1)^2 - 4m(-\frac{1}{m} + 1) > 0  \Rightarrow m > \frac{3}{4}
        \]
    \end{solution}

    \question 实系数二次多项方程 $f(x)=0$ 有一根为 $2$,且方程 $f(f(x))=0$ 恰只有一实根为 $5$,求 $f(0)$。
    \begin{solution}
        设
        \[
        f(x)=a(x-2)(x-b)
        \]
        则
        \begin{align*}
        g(x)=f(f(x))
        &=a\,[a(x-2)(x-b)-2]\,[a(x-2)(x-b)-b]\\
        &=a^3\left(x^2-(b+2)x+2b-\frac{2}{a}\right)\left(x^2-(b+2)x+2b-\frac{b}{a}\right)
        &\equiv a^3 f_1(x) f_2(x)
        \end{align*}
        若$f_1=(x-5)^2,\ f_2 \text{的判别式} < 0,$可得
        \[
        a=-\frac{2}{9},\ b=8
        \]
        若$f_2=(x-5)^2,\ f_1 \text{的判别式} < 0,$可得
        \[
        a=-\frac{8}{9},\ b=8
        \]
        但 $f_1$ 判别式 $>0$,故舍去;因此
        \[
        f(0)=2ab = 2 \cdot \left(-\frac{2}{9}\right) \cdot 8 = -\frac{32}{9}
        \]
    \end{solution}

    \question 求所有实数 $c$,使得 $f(x)=x^{2}+4x+c$ 满足$f(f(x))$ 恰好有三个相异实根。
    \begin{solution}
        若 $f$ 的根为重根,则 $f \circ f$ 至多只有两个相异根,因此 $f$ 必须有两个相异根。设 $r$ 为其中一根使得 $f(x)=r$ 有重根,解
        \[
        x^2 + 4x + c - r = (x+2)^2
        \]
        得$r=c-4$ 是 $f$ 的根,故
        \[
        (c-4)^2 + 4(c-4) + c = 0 \Rightarrow c=0 \text{ 或 } c=3.
        \]
        当 $c=0$,解$x^2+4x=0$及$x^2+4x=-4$得
        \[
        x=0,-4,-2,
        \]
        当 $c=3$,解$x^2+4x+3=-3$及$x^2+4x+3=-1$,发现其中
        \[
        x^2+4x+6
        \]
        判别式为负,故唯一满足条件的 $c$ 为 $0$。
    \end{solution}

    \question 求非零实数三元组 $(a,b,c)$,使得
    \[
    (x^2 + 2ax + b)^2 + 2a(x^2 + 2ax + b) - b = (x-c)^4
    \]
    是多项式恒等式。
    \begin{solution}
        设 
        \[
        P(x) = x^2 + 2ax + b,Q(x) = x^2 + 2ax - b,
        \] 
        且$Q(P(x))$ 的根满足 $P(x) = r_1$ 或 $P(x) = r_2$,其中 $r_1, r_2$ 是 $Q$ 的根。欲使$Q(P(x))$ 只有一个四重根,$Q$ 必须有重根,因此判别式为
        \[
        4a^2 + 4b = 0 \Rightarrow b = -a^2,
        \]
        其中根为 $-a$,此时$P(x) = -a$ 也必须有重根,因此 $P(x) + a = x^2 + 2ax + b + a = 0$ 的判别式为
        \[
        4a^2 - 4(a+b) = 0.
        \]
        代入 $b = -a^2$ 解得
        \[
        a = \frac{1}{2}, \quad b = -\frac{1}{4}.
        \]
        $c$ 是 $P(x)+a=0$ 的解,即 $c = -a = -\frac{1}{2}$,所以三元组为
        \[
        (a,b,c) = \left(\frac{1}{2}, -\frac{1}{4}, -\frac{1}{2}\right)
        \]
    \end{solution}

    \question 求所有实数 $k$,使方程
    \[
    4x^2 + 4(2-k)x - k^2 = 0
    \]
    有两个实数解 $x_1,x_2$ 满足 $|x_1| = 2 + |x_2|$。
    \begin{solution}
        由韦达定理,
        \[
        x_1+x_2 = k-2, \quad x_1x_2 = -\frac{k^2}{4} \le 0
        \]
        条件 $|x_1| = 2 + |x_2|$ 等价于
        \[
        4 = (|x_1|-|x_2|)^2 = x_1^2+x_2^2-2|x_1x_2| = (x_1+x_2)^2 - 2x_1x_2 + 2x_1x_2
        \]
        解得
        \[
        k=0 \text{ 或 } \;4
        \]
    \end{solution}

    \question 解方程
    \[ (x+1)(x+2)(x+3)^2(x+4)(x+5)=360 \]
    \begin{solution}   
        排版后有\[
            \begin{aligned}
                (x+1)(x+5)(x+2)(x+4)(x+3)(x+3)&=360 \\
                (x^2+6x+5)(x^2+6x+8)(x^2+6x+9)&=360
            \end{aligned}
        \] 设 $y=x^2+6x$,变为\[
            \begin{aligned}
                 (y+5)(y+8)(y+9)&=360\\
                 y(y^2+22y+157)&=0
            \end{aligned}
        \]
        其中$y^2+22y+157=0$无实数解。解 $x^2+6x=0$ 可得 $x=0, -6$
    \end{solution}

    \question 解方程
\[
9x^4 - 24x^3 - 2x^2 - 24x + 9 = 0, \quad x\in \mathbb{R}
\]

\begin{solution}

\noindent
此方程是对称四次方程,设 $x\neq 0$,两边除以 $x^2$:
\begin{align*}
9x^2 - 24x - 2 - \frac{24}{x} + \frac{9}{x^2} &= 0 \\
9\left(x^2 + \frac{1}{x^2}\right) - 24\left(x + \frac{1}{x}\right) - 2 &= 0
\end{align*}

\noindent
设 $V = x + \frac{1}{x}$,则
\[
x^2 + \frac{1}{x^2} = V^2 - 2
\]

\noindent
代入方程得
\begin{align*}
9(V^2-2) - 24V - 2 &= 0 \\
9V^2 - 18 - 24V - 2 &= 0 \\
9V^2 - 24V - 20 &= 0 \\
(3V-10)(3V+2) &= 0
\end{align*}
\[
V = \frac{10}{3}, \quad V = -\frac{2}{3}
\]

\noindent
分别解 $x + \frac{1}{x} = V$:

\noindent
1. $x + \frac{1}{x} = \frac{10}{3}$
\begin{align*}
3x^2 - 10x + 3 &= 0 \\
(3x-1)(x-3) &= 0 \\
x &= \frac{1}{3}, \quad x = 3
\end{align*}

\noindent
2. $x + \frac{1}{x} = -\frac{2}{3}$
\begin{align*}
3x^2 + 2x + 3 &= 0
\end{align*}
判别式 $b^2 - 4ac = 4-36 = -32 < 0$,无实根

\noindent
因此实根为
\[
x = 3, \quad x = \frac{1}{3}
\]

\end{solution}

    \question 解
        \[
        \frac{x^2+16x+54}{x^2+11x+35} = \frac{x^2+13x+35}{x^2+14x+54}
        \]
    \begin{solution}
        使得原式交叉相乘时能顺利平方差,写成
        \[
        \frac{x^2+15x+54 + x}{x^2+12x+35 - x} = \frac{x^2+12x+35 + x}{x^2+15x+54 - x}
        \]
        \[
        (x^2+15x+54)^2 - x^2 = (x^2+12x+35)^2 - x^2
        \]
        \[
        (x^2+15x+54)^2 - (x^2+12x+35)^2 = 0
        \]
        再平方差得
        \[
        (3x+19)(2x^2+27x+89) = 0
        \]
        经检验,原方程的解为 $x = -\dfrac{19}{3},\dfrac{-27\pm\sqrt{17}}{4}$
    \end{solution}

    \question 
在$\mathbb{C}$内解
\[
(x+1)^5+(x+1)^4(x-1)+\cdots+(x-1)^5=0
\]
\begin{solution}
利用恒等式
\[
\frac{a^6-b^6}{a-b}=a^5+a^4b+a^3b^2+a^2b^3+ab^4+b^5
\]

令
\[
a=x+1,\quad b=x-1
\]

则原式可写为
\[
\frac{(x+1)^6-(x-1)^6}{(x+1)-(x-1)}=0
\]

由于
\[
(x+1)-(x-1)=2
\]

因此
\[
(x+1)^6-(x-1)^6=0
\]

展开并化简得
\[
x(3x^4+10x^2+3)=0
\]

进一步分解
\[
x(3x^2+1)(x^2+3)=0
\]

解得
\[
x=0,\quad x=\pm\frac{\sqrt{3}}{3}i,\quad x=\pm\sqrt{3}i
\]
\end{solution}

\question
已知 $\alpha$ 和 $\beta$ 是方程
\[
x^2 + (m-2)x + 1 = 0
\]
的根,求
\[
(1 + m\alpha + \alpha^2)(1 + m\beta + \beta^2)
\]

\begin{solution}
将原方程重写为
\[
x^2 + mx - 2x + 1 = 0
\]
\[
x^2 + mx + 1 = 2x
\]

因为 $\alpha$ 和 $\beta$ 是根,满足
\[
\alpha^2 + m\alpha + 1 = 2\alpha
\]
\[
\beta^2 + m\beta + 1 = 2\beta
\]

于是
\[
(1 + m\alpha + \alpha^2)(1 + m\beta + \beta^2) = (\alpha^2 + m\alpha + 1)(\beta^2 + m\beta + 1) = (2\alpha)(2\beta) = 4\alpha\beta
\]

由原方程,根的乘积 $\alpha\beta = 1$,因此
\[
(1 + m\alpha + \alpha^2)(1 + m\beta + \beta^2) = 4 \cdot 1 = 4
\]
\end{solution}



    \question 若三次多项式 $x^{3}+3x-2=0$ 的根为 $a,b,c$,求以 $(a-b)^{2},(b-c)^{2},(c-a)^{2}$ 为根且首项系数为 $1$ 的三次多项式。
    \begin{solution}
        由韦达定理,
        \[
        a+b+c=0,\quad ab+bc+ca=3,\quad abc=2.
        \]
        且有
        \[
        a^3+3a=b^3+3b=c^3+3c=2.
        \]
        因此
        \[
        a^3-b^3+3(a-b)=0\Rightarrow (a-b)\bigl(a^2+ab+b^2+3\bigr)=0\Rightarrow a^2+ab+b^2=-3
        \]
        于是
        \[
        (a-b)^2=-3-3ab=-3(ab+1)
        \]
        同理,
        \[
        (b-c)^2=-3(bc+1),\quad (c-a)^2=-3(ca+1)
        \]
        故
        \begin{align*}
        (a-b)^2+(b-c)^2+(c-a)^2 
        &= -3(3+ab+bc+ca) = -18
        \end{align*}
        \begin{align*}
        &(a-b)^2(b-c)^2+(b-c)^2(c-a)^2+(c-a)^2(a-b)^2 \\
        &= 9\big((ab+1)(bc+1)+(bc+1)(ca+1)+(ca+1)(ab+1)\big) \\
        &= 9\big(2(a+b+c)+2(ab+bc+ca)+3\big) = 81
        \end{align*}
        \begin{align*}
        (a-b)^2(b-c)^2(c-a)^2
        &= -27(ab+1)(bc+1)(ca+1) \\
        &= -27\big(abc(a+b+c)+(abc)^2+ab+bc+ca+1\big) = -216
        \end{align*}
        所求三次多项式为
        \[
        x^3 +18x^2 +81x +216
        \]
    \end{solution}

    \question 设 $a, b, c \in \mathbb{R}$, 证明方程式
    \[
    \frac{b+c}{x-a}+\frac{c+a}{x-b}+\frac{a+b}{x-c}=3
    \]
    的根都是实根。
    \begin{solution}
        有
        \begin{align*}
        \frac{b+c}{x-a} + \frac{c+a}{x-b} + \frac{a+b}{x-c} &= 3 \\
        1 - \frac{b+c}{x-a} + 1 - \frac{c+a}{x-b} + 1 - \frac{a+b}{x-c} &= 0 \\
        \frac{x-a-b-c}{x-a} + \frac{x-a-b-c}{x-b} + \frac{x-a-b-c}{x-c} &= 0 \\
        (x-a-b-c)\left( \frac{1}{x-a} + \frac{1}{x-b} + \frac{1}{x-c} \right) &= 0
        \end{align*}
        \[
        (x-a-b-c) \frac{ x^2 - (b+c)x + bc + x^2 - (a+c)x + ac + x^2 - (a+b)x + ab }{ (x-a)(x-b)(x-c) } = 0 
        \]
        \[
        (x-a-b-c)\left( 3x^2 - 2(a+b+c)x + ab + bc + ca \right) = 0
        \]
        故方程式
        \[
        \frac{b+c}{x-a} + \frac{c+a}{x-b} + \frac{a+b}{x-c} = 3
        \]
        有一实根 $x = a + b + c$,现探讨另两根,方程式
        \[
        3x^2 - 2(a+b+c)x + ab + bc + ca = 0.
        \]
        的判别式为
        \[
        \Delta = 4(a+b+c)^2 - 12(ab+bc+ca) = 4\left(a^2 + b^2 + c^2 - ab - bc - ca\right).
        \]
        由 AM-GM 不等式,
        \[
        ab + bc + ca \leq \frac{a^2 + b^2}{2} + \frac{b^2 + c^2}{2} + \frac{c^2 + a^2}{2} = a^2 + b^2 + c^2.
        \]
        故判别式为非负,于是方程式 $3x^2 - 2(a+b+c)x + ab + bc + ca = 0$的根都是实根,得证原方程式的根都是实根。
    \end{solution}

    \question 设 $f(x)=x^3+4x^2+8x+16$,计算
    \[
    f(x+7)-f(x+6)-f(x+5)+f(x+4)-f(x+3)+f(x+2)+f(x+1)-f(x).
    \]
    \begin{solution}
        设 $g(x)=f(x+1)-f(x)$,则 
        \[
        g(x)=3x^2+10x+12 
        \]
        同理设 $h(x)=g(x+2)-g(x)=12x+24$,原式可写成
        \[
        g(x+6)-g(x+4)-g(x+2)+g(x)=h(x+4)-h(x)=12(x+4)-12x=48
        \]
    \end{solution}

    \question 已知关于$x$的方程$x^3+ax^2+bx+c=0$的三个非零实数根成等比数列,求$a^3c-b^3$的值。
    \begin{solution}
        设方程$x^3+ax^2+bx+c=0$的三根为$\dfrac{\alpha}{r},\alpha,\alpha r,\alpha \ne 0 $,由韦达定理,
        \begin{align*}
            -a&=\frac{\alpha}{r}+\alpha+\alpha r=\alpha(\frac{1}{r}+1+r) \tag{1}\\
            b&=\frac{\alpha^2}{r}+\alpha^2+\alpha^2 r=\alpha^2(\frac{1}{r}+1+r) \tag{2}\\
            -c&=\alpha^3 \tag{3}
        \end{align*}
        由$(1),(2)$得$-\dfrac{b}{a}=\alpha$,代入(3)得
        \[
        -c=-\frac{b^3}{a^3} \Rightarrow a^3c-b^3=0
        \]
    \end{solution}

    \question 已知$\left(x + \dfrac{1}{x}\right)^2=3$,求
    \[
    x^{63} + x^{44} + x^{37} + x^{31} + x^{26} + x^9 + 6
    \]
    的值。
    \begin{solution}
        由已知得
        \[
        \left(x + \frac{1}{x}\right)^3 = x^3 + \frac{1}{x^3} + 3(1)(\sqrt{3}) = 3\sqrt{3}
        \]
        给出
        \[
        x^3 + \frac{1}{x^3} = 0
        \]
        同理可得,
        \[
        x^9 + \frac{1}{x^9} = 0, \quad x^{27} + \frac{1}{x^{27}} = 0
        \]
        故
        \begin{align*}
        &x^{63} + x^{44} + x^{37} + x^{31} + x^{26} + x^9 + 6 \\
        & =x^{36}\left(x^{27} + \frac{1}{x^{27}}\right) + x^{35} \left(x^9 + \frac{1}{x^9}\right) + x^{34} \left(x^3 + \frac{1}{x^3}\right) + 6 =6
        \end{align*}
    \end{solution}

    \question
已知
\[
x^3 + \frac{1}{x^3} = 18
\]
求
\[
x^{11} + \frac{1}{x^{11}}
\]

\begin{solution}
设
\[
a = x + \frac{1}{x}
\]

利用恒等式
\[
x^3 + \frac{1}{x^3} = a^3 - 3a
\]
得
\[
a^3 - 3a = 18
\]
\[
a^3 - 3a - 18 = 0
\]

因式分解
\[
(a-3)(a^2 + 3a + 6) = 0
\]
所以
\[
a = x + \frac{1}{x} = 3
\]
则
\[
x^2+\frac{1}{x^2} = 3^2-2=7,\quad x^4+\frac{1}{x^4} = 47,\quad x^8+\frac{1}{x^8} = 2207
\]
于是
\[
\left(x^2+\frac{1}{x^2}\right)\left(x^3 + \frac{1}{x^3}\right)=x^5+\frac{1}{x^5} + x + \frac{1}{x}= 126 \Rightarrow x^5+\frac{1}{x^5}=123
\]
且
\[
\left(x^3+\frac{1}{x^3}\right)\left(x^8 + \frac{1}{x^8}\right)=x^{11}+\frac{1}{x^{11}} + x^5 + \frac{1}{x^5}= 39726 \Rightarrow x^{11}+\frac{1}{x^{11}}=39726-123=39603
\]
\end{solution}

    \question 已知 $\alpha, \beta, \gamma$ 是方程
    \[
    x^3 - x - 1 = 0
    \]
    的三根, 计算
    \[
    \frac{1-\alpha}{1+\alpha} + \frac{1-\beta}{1+\beta} + \frac{1-\gamma}{1+\gamma}
    \]
    的值。 
    \begin{solution}
        由韦达定理,$\;\alpha+\beta+\gamma=0,\;\alpha\beta+\beta\gamma+\gamma\alpha=-1,\;\alpha\beta\gamma=1$,故
        \begin{align*}
        \frac{1-\alpha}{1+\alpha} + \frac{1-\alpha}{1+\beta} + \frac{1-\gamma}{1+\gamma} 
        &= \frac{2}{1+\alpha} + \frac{2}{1+\beta} + \frac{2}{1+\gamma}-3 \\
        &= 2\cdot\frac{(1+\alpha)(1+\beta)+(1+\beta)(1+\gamma)+(1+\gamma)(1+\alpha)}{(1+\alpha)(1+\beta)(1+\gamma)}-3 \\
        &= 2\cdot\frac{3+2(\alpha+\beta+\gamma)+\alpha\beta+\beta\gamma+\gamma\alpha}{1+\alpha+\beta+\gamma+\alpha\beta+\beta\gamma+\gamma\alpha+\alpha\beta\gamma}-3 \\
        &= 2\cdot\frac{3-1}{1-1+1}-3 \\
        &= 1
        \end{align*}
    \end{solution}
        
    \question 已知 $\alpha,\beta,\gamma,\delta$ 为 $x^{4}+2x^{3}+x^{2}+2x+1=0$ 的四根,试求 $$(\alpha^{2}+\alpha+1)(\beta^{2}+\beta+1)(\gamma^{2}+\gamma+1)(\delta^{2}+\delta+1)$$ 的值。
    \begin{solution}
        $\alpha$ 是方程
        $$x^{4}+2x^{3}+x^{2}+2x+1=0$$
        的根,则
        \[
        \alpha^4 + 2\alpha^3 + \alpha^2 + 2\alpha + 1 = 0
        \Rightarrow (\alpha^2 + \alpha + 1)^2 = 2\alpha^2
        \Rightarrow \alpha^2 + \alpha + 1 = \sqrt{2} \alpha
        \]
        同理可得
        \[
        \beta^2 + \beta + 1 = \sqrt{2} \beta,\quad
        \gamma^2 + \gamma + 1 = \sqrt{2} \gamma,\quad
        \delta^2 + \delta + 1 = \sqrt{2} \delta
        \]
        由韦达定理,
        \[
        (\alpha^2 + \alpha + 1)(\beta^2 + \beta + 1)(\gamma^2 + \gamma + 1)(\delta^2 + \delta + 1)
        = (\sqrt{2})^4 \cdot \alpha \beta \gamma \delta =4 \cdot 1 = 4
        \]
    \end{solution}

    \question 设方程 $x^{3}-4x+1=0$ 的三个相异复数根为 $a, b, c$,求 
    $$\frac{a+1}{(a-1)^{4}}+\frac{b+1}{(b-1)^{4}}+\frac{c+1}{(c-1)^{4}}$$ 的值。
    \begin{solution}
        发现
        \[\frac{a+1}{(a-1)^4} = \frac{1}{(a-1)^3} + \frac{2}{(a-1)^4}
        \],因此欲求以 $\displaystyle \frac{1}{a-1}, \frac{1}{b-1}, \frac{1}{c-1}$ 为三根的多项式,令 $x_1 = x - 1 $ 代入原方程式
        \[
        (x_1+1)^3 - 4(x_1+1) + 1 = 0 \Rightarrow x_1^3 + 3x_1^2 - x_1 - 2 = 0
        \]
        再令 $x_2 = \frac{1}{x_1}$ 代入可得
        \[
        \frac{1}{x_2^3} + \frac{3}{x_2^2} - \frac{1}{x_2} - 2 = 0 \Rightarrow x^3_2 + \frac12 x^2_2 - \frac32 x_2 - \frac12 = 0
        \]
        故
        \[
        \frac{a+1}{(a-1)^4} + \frac{b+1}{(b-1)^4} + \frac{c+1}{(c-1)^4}
        = (\alpha^3 + \beta^3 + \gamma^3) + 2(\alpha^4 + \beta^4 + \gamma^4) \tag{1}
        \]
        设 $\displaystyle f(x) = x^3 + \frac12 x^2 - \frac32 x - \frac12$, 则 $ f'(x) = 3x^2 + x - \dfrac32$,利用长除法计算 $\dfrac{f'(x)}{f(x)}$:
        \[
        \frac{f'(x)}{f(x)} = \frac{3}{x} - \frac12 \cdot \frac{1}{x^2} + \frac{13}{4} \cdot \frac{1}{x^3}
        - \frac{7}{8} \cdot \frac{1}{x^4} + \frac{81}{16} \cdot \frac{1}{x^5} + \cdots
        \]
        由系数比较得
        \[
        \alpha^3 + \beta^3 + \gamma^3 = -\frac78,\quad
        \alpha^4 + \beta^4 + \gamma^4 = \frac{81}{16}
        \]
        故所求为
        \[
        -\frac78 + 2 \cdot \frac{81}{16} = \frac{37}{4}
        \]
    \end{solution}

    \question 已知 $f(x) = x^4 + px^3 + qx^2 + rx + s$ 且 $f(1)=59,f(2)=118,f(3)=177$,求 $f(9)+f(-5)$ 。
    \begin{solution}
        考虑函数 $g(x) = 59x$,则
        \[
        f(1)=g(1)=59,\quad f(2)=g(2)=118,\quad f(3)=g(3)=177。
        \]
        定义 $h(x) = f(x)-g(x)$,则 $x=1,2,3$ 为 $h(x)$ 的根且 $h(x)$ 首项系数为 $1$,所以设
        \[
        h(x) = (x-1)(x-2)(x-3)(x-a)
        \]
        其中$a$是一实数,于是
        \begin{align*}
        f(9)+f(-5) &= h(9)+g(9) + h(-5)+g(-5) \\
        &= 336(9-a) + 9\cdot 59 + 336(5+a) - 5\cdot 59 \\
        &= 4940
        \end{align*}
    \end{solution}

    \question 已知 $x_1, x_2, x_3$ 是方程
    \[
    x^3 - 6x^2 + a x - a = 0
    \]
    的根,且满足
    \[
    (x_1 - 3)^3 + (x_2 - 3)^3 + (x_3 - 3)^3 = 0,
    \]
    求 $a$ 的值。
    \begin{solution}
        令 $r_i = x_i - 3$, 则 $r_1, r_2, r_3$ 是方程
        \[
        (x+3)^3 - 6(x+3)^2 + a(x+3) - a = 0
        \]
        即方程
        \[
        x^3 + 3x^2 + (a-9)x + (2a-27) = 0
        \]
        的根,由韦达定理,
        \[
        r_1 + r_2 + r_3 = -3, \quad r_1r_2 + r_2r_3 + r_3r_1 = a-9, \quad r_1r_2r_3 = 27-2a
        \]
        由立方和公式,
        \[
        r_1^3 + r_2^3 + r_3^3 = -3(9 - 3(a-9)) + 3(27 - 2a) = 0
        \]
        解得
        \[
        a = 9
        \]
    \end{solution}

    \question 已知 $x_{1}, x_{2}, x_{3}$ 为方程 $\sqrt{123}x^{3}-247x^{2}+2=0$ 的相异实根, 且 $x_{1}<x_{2}<x_{3}$, 求 $x_{2}(x_{3}^{2}-x_{1}^{2})$的值。
    \begin{solution}
        设 $\sqrt{123}=a$, 则方程 $\sqrt{123}x^{3}-247x^{2}+2=0$ 化为 $$ax^{3}-(2a^{2}+1)x^{2}+2=0\Rightarrow(ax-1)(x^{2}-2ax-2)=0$$
        所以方程的3个实数根为 $$\frac{1}{a},a+\sqrt{a^{2}+2},a-\sqrt{a^{2}+2}$$
        因为 $a=\sqrt{123}>1$,所以 $$a-\sqrt{a^{2}+2}<0<\frac{1}{a}<1<a+\sqrt{a^{2}+2}$$
        因此 $x_{1}=a-\sqrt{a^{2}+2},x_{2}=\dfrac{1}{a}, x_{3}=a+\sqrt{a^{2}+2}$,而
        $$x_{2}(x_{3}^{2}-x_{1}^{2})=\frac{1}{a}\cdot2\sqrt{a^{2}+2}\cdot 2a=20\sqrt5$$
    \end{solution}
    
    \question 已知正整数 $m,n$ 满足 $m^2 - n = 32$,且 $\sqrt[5]{m+\sqrt{n}} + \sqrt[5]{m-\sqrt{n}}$ 是方程 $$x^5 - 10x^3 + 20x - 40 = 0$$ 的一个实根,求 $m,n$ 的值。
    
    \begin{solution}
        令 $a = \sqrt[5]{m+\sqrt{n}},b = \sqrt[5]{m-\sqrt{n}}$,则 $x = a + b$ 是该方程的一个实根,且
        \[
        a^5 + b^5 = 2m, \quad a^5 b^5 = m^2 - n = 32
        \Rightarrow ab = 2
        \]
        考虑 $(a + b)^5$ 的展开式:
        \[
        x^5 = a^5 + b^5 + 5ab(a^3 + b^3) + 10a^2 b^2 (a + b)
        \]
        由$a^3 + b^3 = (a + b)\left[(a + b)^2 - 3ab\right] = x(x^2 - 6)$,得 
        \[
        x^5 = 2m + 10x(x^2 - 6) + 40x 
        \]
        整理得:
        \[
        x^5 - 10x^3 + 20x- 2m=0
        \]
        对比原方程式得
        \[
        m = 20 , n = 368
        \]
    \end{solution}

    \question 设方程式 $x^5 + x^4 - x^2 + 1 = 0$ 的五个根为 $\alpha_1, \alpha_2, \alpha_3, \alpha_4, \alpha_5$, 若 $P(x) = x^4 - 1$,求 $$P(\alpha_1) P(\alpha_2) P(\alpha_3)  P(\alpha_4) P(\alpha_5)$$ 的值。
    \begin{solution}
        设
        \[
        f(x) = x^5 + x^4 - x^2 + 1 = (x - \alpha_1)(x - \alpha_2)(x - \alpha_3)(x - \alpha_4)(x - \alpha_5) 
        \]
        令$x=1$得
        \[
        f(1) = (1 - \alpha_1)(1 - \alpha_2)(1 - \alpha_3)(1 - \alpha_4)(1 - \alpha_5) = 0
        \]
        由于
        \[P(x) = x^4 - 1 = (x^2 + 1)(x - 1)(x + 1) \]
        故
        \begin{align*}
        &P(\alpha_1) P(\alpha_2) P(\alpha_3)  P(\alpha_4) P(\alpha_5)\\
        &= -(1 - \alpha_1)(1 - \alpha_2)(1 - \alpha_3)(1 - \alpha_4)(1 - \alpha_5) \prod_{k=1}^5 (\alpha_{k}^2+1)(\alpha_{k}+1)\\ &= 0
        \end{align*}
    \end{solution}

    \question 已知 $p(x)$ 是一个整系数多项式,定义 $q(x)=\dfrac{p(x)}{x(1-x)}$,若对所有 $x\neq 0,1$ 都有
    \[
    q(x) = q\left(\frac{1}{1-x}\right),
    \]
    且 $p(2)=p(3)=5$,求 $p(4)$ 的值。
    \begin{solution}
        由 $q(x)=q\left(\dfrac{1}{1-x}\right)$ 可得
        \[
        \frac{p(x)}{x(1-x)} = \frac{p\left(\frac{1}{1-x}\right)}{\frac{1}{1-x}\left(1-\frac{1}{1-x}\right)} \Rightarrow p(x) = (1-x)^3 p\left(\frac{1}{1-x}\right)
        \]
        这说明 $\deg(p)\le 3$,设 $p(x)=ax^3+bx^2+cx+d$,则
        \[
        (1-x)^3 p\left(\frac{1}{1-x}\right)=d(1-x)^3+c(1-x)^2+b(1-x)+a
        \]
        比较$x^3,x^2$系数得
        \[
        a=d, \quad b=-c-3d
        \]
        且由 $p(2)=p(3)=5$ 得
        \[
        -3a - 2c = a - 6c =5 \Rightarrow a=c=-1
        \]
        于是
        \[
        p(x) = -x^3 +4x^2 - x -1 \Rightarrow p(4) = -5
        \]
    \end{solution}

    \question 已知多项式 $x^7 - 5$ 的七个相异根为 $r_1, \dots, r_7$,求
    \[
    \prod_{1 \le i < j \le 7} (r_i + r_j)^2
    \] 的值。
    \begin{solution}
        设所求乘积为 $P$,由韦达定理,考虑
        \begin{align*}
        2^7 \cdot 5 \cdot P 
        &= \prod_{i=1}^{7} 2 r_i \prod_{1 \le i < j \le 7} (r_i + r_j)^2\\
        &= \prod_{1 \le i \le j \le 7} (r_i + r_j) \prod_{1 \le i < j \le 7} (r_i + r_j) \prod_{1 \le j < i \le 7} (r_i + r_j)\\
        &= \prod_{i=1}^{7} \prod_{j=1}^{7} (r_i + r_j)
        \end{align*}
        且注意到
        \[
        \prod_{j=1}^7(x-r_j) = x^7 - 5 \Rightarrow \prod_{j=1}^7(x+r_j) = x^7 + 5,
        \]
        故
        \[
        \prod_{j=1}^{7} (r_i + r_j) = r_i^7 + 5 = 10
        \]
        因此
        \[
        2^7 \cdot 5 \cdot P = 10^7 \Rightarrow P = 5^6
        \]
    \end{solution}

    \question 设 $P(x)$ 为次数为 $10$ 的多项式,且满足 
    \[
    P(2^i)=i,0\le i\le 10,
    \]
    求 $P(x)$ 中 $x$ 项的系数。
    \begin{solution}
        令 $Q(x)=P(2x)-P(x)-1$,则 $Q(2^i)=0,0\le i\le9$,所以
        \[
        Q(x)=\alpha\prod_{k=0}^{9}(x-2^k)
        \]
        其中$\alpha$为一常数,又 $Q(0)=P(0)-P(0)-1=-1$,且
        \[
        \prod_{k=0}^{9}(0-2^k)=2^{45},
        \]
        因此 $\alpha=-\dfrac{1}{2^{45}}$,记 
        \[
        R(x)=\prod_{k=0}^{9}(x-2^k),
        \]
        其 $x$ 项系数为
        \[
        \text{coef}_{x}(R) = -\left(\prod_{k=0}^{9}2^k\right)\sum_{k=0}^{9}\frac{1}{2^k}
        = -2^{45}\left(2-\frac{1}{2^9}\right)
        \]
        因为$P(2x)-P(x)-1$ 的 $x$ 项系数恰好等于 $P(x)$ 的 $x$ 项系数,故
        \[
        \text{coef}_{x}(Q)=\alpha\cdot\text{coef}_{x}(R)
        = -\frac{1}{2^{45}}\cdot\left(-2^{45}\cdot\frac{1023}{512}\right)=\frac{1023}{512}
        \]
    \end{solution}

    \question 已知 $x_1, x_2, \ldots, x_{2015}$ 是方程
    \[
    x^{2015}+x^{2014}+\dots+x^2+x+1=0
    \]
    的根,求
    \[
    \frac{1}{1-x_1}+\frac{1}{1-x_2}+\dots+\frac{1}{1-x_{2015}}.
    \]
    \begin{solution}
        设 
        \[
        a_k = \frac{1}{1-x_k}, \quad k=1,2,\dots,2015.
        \]
        则
        \[
        x_k = \frac{a_k-1}{a_k}.
        \]
        代入原方程,得到
        \[
        \frac{(a_k-1)^{2015}}{a_k^{2015}} + \frac{(a_k-1)^{2014}}{a_k^{2014}} + \dots + \frac{a_k-1}{a_k} + 1 = 0.
        \]
        两边同时乘以 $a_k^{2015}$,得
        \[
        (a_k-1)^{2015} + a_k(a_k-1)^{2014} + \dots + a_k^{2014}(a_k-1) + a_k^{2015} = 0.
        \]
        因此
        \[
        a_1 + a_2 + \dots + a_{2015} = \frac{\comb{2015}{1}+\comb{2014}{1}+\comb{1}{1}}{2016} = \frac{2015}{2}
        \]
        而当$k=1,2,\dots,n,a_1 + a_2 + \dots + a_{2015}=\dfrac{n}{2}$
    \end{solution}
    \begin{solution}
        设
        \[
        f(x) = x^{2015}+x^{2014}+\dots+x+1,
        \]
        则
        \[
        f(x) = \prod_{k=1}^{2015}(x-x_k) \Rightarrow \frac{f'(x)}{f(x)} = \sum_{k=1}^{2015} \frac{1}{x-x_k}.
        \]
        取 $x=1$,得到
        \[
        \sum_{k=1}^{2015} \frac{1}{1-x_k} = \frac{f'(1)}{f(1)}= \frac{2015}{2}
        \]
        其中
        \[
        f(1) = 2016, \quad f'(1) = 2015+2014+\dots+1 = \frac{2015\cdot 2016}{2}
        \]
    \end{solution}

    \question 
    \begin{parts}
    \part 已知 $\alpha,\beta,\gamma$ 为三实数,设 $t=-(\alpha+\beta+\gamma),v=\alpha\beta+\beta\gamma+\gamma\alpha$,且满足
    \[
    \alpha\beta\gamma=-1, \quad t+v=-3,
    \]
    试证
    \[
    \alpha^{\frac{1}{3}}+\beta^{\frac{1}{3}}+\gamma^{\frac{1}{3}}=\sqrt[3]{(-t-6)+3\sqrt[3]{t^2+3t+9}}.
    \]
    \begin{solution}
        设
        \[
        f(x)=(x-\alpha)(x-\beta)(x-\gamma)=x^3+tx^2+vx+1,
        \]
        \[
        g(y)=(y-\alpha^{\frac{1}{3}})(y-\beta^{\frac{1}{3}})(y-\gamma^{\frac{1}{3}})
        =y^3+ay^2+by+1
        \]
        由$g(y)=0$得$y^3+1=-y(ay+b)$,于是
        \[
        (y^3+1)^3=-y^3\left(a^3y^3+b^3+3ab y(ay+b)\right)
        =-y^3\left(a^3y^3+b^3-3ab(y^3+1)\right)
        \]
        令 $x=y^3$,展开整理得
        \[
        x^3+(a^3-3ab+3)x^2+(b^3-3ab+3)x+1=0
        \]
        于是$a^3-3ab+3=t, b^3-3ab+3=v$,令 $z=ab$,则
        \begin{align*}
        z^3&=(t+3z-3)(v+3z-3) \\
        &=tv+3z(t+v)-3(t+v)+9z^2-18z+9 \\
        &=t(-3-t)+3z(-3)-3(-3)+9z^2-18z+9 \\
        &=9z^2-27z+(-t^2-3t+18)
        \end{align*}
        即
        \[
        (z-3)^3=-t^2-3t-9 \Rightarrow z=3-\sqrt[3]{t^2+3t+9}
        \]
        由韦达定理,
        \[
        \alpha^{\frac{1}{3}}+\beta^{\frac{1}{3}}+\gamma^{\frac{1}{3}}=-a=-\sqrt[3]{t+3z-3}=\sqrt[3]{(-t-6)+3\sqrt[3]{t^2+3t+9}}
        \]
        故证毕。
    \end{solution}
    \part 据此,试证
    \[
    \sqrt[3]{\cos\frac{2\pi}{9}}+\sqrt[3]{\cos\frac{4\pi}{9}}+\sqrt[3]{\cos\frac{8\pi}{9}}
    =\sqrt[3]{\frac{3}{2}(\sqrt[3]{9}-2)}
    \]
    \begin{solution}
        考虑
        \[
        \alpha=2\cos\frac{2\pi}{9},\quad
        \beta=2\cos\frac{4\pi}{9},\quad
        \gamma=2\cos\frac{8\pi}{9}.
        \]
        则由
        \begin{align*}
        \cos{\frac{2\pi}{9}} \cos{\frac{4\pi}{9}} \cos{\frac{8\pi}{9}} &= \frac{1}{2} \left(\cos{\frac{6\pi}{9}}+\cos{\frac{2\pi}{9}} \right)\cos{\frac{8\pi}{9}} \\
        &= \frac{1}{2} \left(-\frac{1}{2}+\cos{\frac{2\pi}{9}} \right)\cos{\frac{8\pi}{9}} \\
        &= \frac{1}{2} \left(-\frac{1}{2} \cos{\frac{8\pi}{9}}+ \cos{\frac{2\pi}{9}}\cos{\frac{8\pi}{9}} \right) \\
        &= \frac{1}{2} \left(-\frac{1}{2} \cos{\frac{8\pi}{9}}+ \frac{1}{2}\left(\cos{\frac{10\pi}{9}} +\cos{\frac{6\pi}{9}}\right) \right) \\
        &= \frac{1}{2} \left(\frac{1}{2} \cos{\frac{\pi}{9}}- \frac{1}{2}\cos{\frac{\pi}{9}} -\frac{1}{4} \right) = -\frac{1}{8} 
        \end{align*}
        满足已知$\alpha\beta\gamma=-1$,此时
        \begin{align*}
        t =-(\alpha+\beta+ \gamma) &=-2\left(\cos{\frac{2\pi}{9}}+ \cos{\frac{4\pi}{9}}+ \cos{\frac{8\pi}{9}} \right) \\
        &=-2\left(2\cos{\frac{3\pi}{9}}\cos{\frac{\pi}{9}}+ \cos{\frac{8\pi}{9}} \right) \\
        &=-2\left(\cos{\frac{\pi}{9}}+ \cos{\frac{8\pi}{9}} \right) \\
        &=-2\cdot 2\cos{\frac{\pi}{2}}\cos{\frac{7\pi}{18}} =0
        \end{align*}
        故
        \[
        \alpha^{\frac{1}{3}}+\beta^{\frac{1}{3}}+\gamma^{\frac{1}{3}}
        =\sqrt[3]{-6+3\sqrt[3]{9}}
        \]
        且
        \[
        \sqrt[3]{\cos\frac{2\pi}{9}}+\sqrt[3]{\cos\frac{4\pi}{9}}+\sqrt[3]{\cos\frac{8\pi}{9}}
        =\frac{1}{\sqrt[3]{2}}\bigl(\alpha^{\frac{1}{3}}+\beta^{\frac{1}{3}}+\gamma^{\frac{1}{3}}\bigr)
        =\sqrt[3]{\frac{3}{2}(\sqrt[3]{9}-2)}.
        \]
        证毕。
    \end{solution}
    \end{parts}

    \question 求 
    \[
    \frac{(1^{4}+18^{2})(11^{4}+18^{2})(23^{4}+18^{2})(35^{4}+18^{2})(47^{4}+18^{2})}{(5^{4}+18^{2})(7^{4}+18^{2})(17^{4}+18^{2})(29^{4}+18^{2})(41^{4}+18^{2})}
    \] 
    的值。
    \begin{solution}
        因式分解给出
        \[
        a^4+18^2=(a^2+18)^2-(6a)^2=(a^2+6a+18)(a^2-6a+18)=((a+3)^2+9)((a-3)^2+9)
        \] 
        故
        \begin{align*}
        &\frac{(1^{4}+18^{2})(11^{4}+18^{2})(23^{4}+18^{2})(35^{4}+18^{2})(47^{4}+18^{2})}{(5^{4}+18^{2})(7^{4}+18^{2})(17^{4}+18^{2})(29^{4}+18^{2})(41^{4}+18^{2})}\\
        &=\frac{(4^2+9)(2^2+9)(14^2+9)(8^2+9)(26^2+9)(20^2+9)(38^2+9)(32^2+9)(50^2+9)(44^2+9)}{(8^2+9)(2^2+9)(10^2+9)(4^2+9)(20^2+9)(14^2+9)(32^2+9)(26^2+9)(44^2+9)(38^2+9)} \\
        &=\frac{50^2+9}{10^2+9}=\frac{2509}{109}
        \end{align*}
    \end{solution}

    \question 已知三实数 $a,b,c$ 满足 $\sqrt{3}(a-b)+3(b-c)+(c-a)=0$,且 $b \neq c$,求 $$\frac{(a-b)(a-c)}{(b-c)^2}$$ 的值。
    \begin{solution}
        令$\alpha = a-b, \beta = b-c, \gamma = c-a$,则
        \[
        \alpha+\beta+\gamma= \sqrt{3}\alpha +3\beta +\gamma =0
        \]
        由此得
        \[
        \gamma = -(\alpha+\beta), \quad \alpha+\beta = \sqrt{3}\alpha + 3\beta \Rightarrow \beta = \frac{1-\sqrt{3}}{2}\alpha
        \]
        因此
        \[
        \frac{(a-b)(a-c)}{(b-c)^2} = \frac{-\alpha \gamma}{\beta^2} = \frac{\alpha(\alpha+\beta)}{\left(\frac{1-\sqrt{3}}{2}\right)^2 \alpha^2} = \frac{\frac{3-\sqrt{3}}{2}\alpha^2}{\frac{4-2\sqrt{3}}{4}\alpha^2} = \frac{3-\sqrt{3}}{2-\sqrt{3}} \cdot \frac{2+\sqrt{3}}{2+\sqrt{3}}= 3+\sqrt{3}
        \]
    \end{solution}
\question 26)
已知 $a+b+c=6$,$\frac{1}{a}+\frac{1}{b}+\frac{1}{c}=2$,求
\[
\frac{b+c}{a}+\frac{c+a}{b}+\frac{a+b}{c}
\]

\begin{solution}
设
\[
S=\frac{b+c}{a}+\frac{c+a}{b}+\frac{a+b}{c}
\]

两边同时加 $3$,得
\[
S+3=\left(\frac{b+c}{a}+1\right)+\left(\frac{c+a}{b}+1\right)+\left(\frac{a+b}{c}+1\right)
\]

化简得
\[
S+3=\frac{a+b+c}{a}+\frac{a+b+c}{b}+\frac{a+b+c}{c}
\]

提取公因式
\[
S+3=(a+b+c)\left(\frac{1}{a}+\frac{1}{b}+\frac{1}{c}\right)
\]

代入已知条件
\[
S+3=6\times2=12
\]

因此
\[
S=9
\]
\end{solution}

    \question 已知 $a,b,c$ 皆为实数,若
    \[
    \frac{a}{b+c}+\frac{b}{c+a}+\frac{c}{a+b}=1
    \]
    求
    \[
    \frac{a^{2}}{b+c}+\frac{b^{2}}{c+a}+\frac{c^{2}}{a+b}
    \]
    的值。
    \begin{solution}
        有
        \[
        \frac{a(a+b+c)}{b+c}+\frac{b(a+b+c)}{c+a} +\frac{c(a+b+c)}{a+b}=a+b+c
        \]
        于是
        \[
        \frac{a^2}{b+c}+ a+\frac{b^2}{c+a}+ b+\frac{c^2}{a+b}+c=a+b+c
        \]
        即
        \[
        \frac{a^2}{b+c}+ \frac{b^2}{c+a}+\frac{c^2}{a+b}=0
        \]
    \end{solution}

\question 已知 $abc=1$,证明
\begin{parts}
\part 
\[
\frac{a}{ab+a+1} + \frac{b}{bc+b+1} + \frac{c}{ca+c+1}=1
\]
\begin{solution}
    由$abc=1$,
    \begin{align*}       
    \frac{a}{ab+a+1} + \frac{b}{bc+b+1} + \frac{c}{ca+c+1}
    &= \frac{a}{ab+a+1} + \frac{ab}{abc+ab+a} + \frac{abc}{abca+abc+ab} \\
    &= \frac{a}{ab+a+1} + \frac{ab}{1+ab+a} + \frac{1}{a+1+ab} \\
    &= \frac{a+ab+1}{ab+a+1} = 1
    \end{align*}
\end{solution}
\part 
\[
\frac{1}{ab+a+1} + \frac{1}{bc+b+1} + \frac{1}{ca+c+1}=1
\]
\begin{solution}
    由$abc=1$,
    \begin{align*}
    \frac{1}{ab+a+1} + \frac{1}{bc+b+1} + \frac{1}{ca+c+1}
    &= \frac{1}{ab+a+abc} + \frac{1}{bc+b+1} + \frac{1}{ca+c+1}\\
    &= \frac{a(b+1+bc)}{bc+b+1} + \frac{1}{ca+c+1}\\
    &= \frac{a(b+1+bc)}{1+a} + \frac{1}{ca+c+1}\\
    &= \frac{a(b+abc+bc)}{1+a} + \frac{1}{ca+c+1}\\
    &= \frac{ab(1+ac+c)}{1+a+ab} + \frac{1}{ca+c+1}\\
    &= \frac{ab(1+ac+c)}{1+a+ab}\\
    &= \frac{ab+a^2bc+abc}{1+a+ab}\\
    &= \frac{1+a+ab}{ab+a+1} = 1
    \end{align*}
    \end{solution}
\begin{solution}
    \begin{align*}
    \frac{1}{ab+a+1} + \frac{1}{bc+b+1} + \frac{1}{ac+c+1} =\frac{c}{abc+ac+c} + \frac{a}{abc+ab+a} + \frac{b}{abc+bc+b}
    \end{align*}
\end{solution}
\end{parts}

\question
若相异三数 $a, b, c$ 均不为 0,且 $a + b + c = 0$,求下式的值:
\[ \left(\frac{b - c}{a} + \frac{c - a}{b} + \frac{a - b}{c}\right) \left(\frac{a}{b - c} + \frac{b}{c - a} + \frac{c}{a - b}\right) \]

\begin{solution}
    首先,我们处理第一个括号内的通分:
    \begin{align*}
    \frac{b - c}{a} + \frac{c - a}{b} + \frac{a - b}{c} &= \frac{b^2 c - b c^2 + a c^2 - a^2 c + a^2 b - a b^2}{abc} \\
    &= \frac{c^2(a - b) + ab(a - b) - (ac + bc)(a - b)}{abc} \\
    &= \frac{(a - b)(c^2 + ab - ac - bc)}{abc} \\
    &= \frac{(a - b)[c(c - a) - b(c - a)]}{abc} \\
    &= \frac{(a - b)(c - b)(c - a)}{abc} \\
    &= -\frac{(a - b)(b - c)(c - a)}{abc}
    \end{align*}

    接下来处理第二个括号。令 $a' = b - c$,$b' = c - a$,$c' = a - b$。
    观察 $b' - c'$ 的关系:
    \[ b' - c' = (c - a) - (a - b) = b + c - 2a \]
    根据条件 $a + b + c = 0$,可知 $b + c = -a$,代入上式得:
    \[ b' - c' = -a - 2a = -3a \]
    由此可得 $a = -\frac{b' - c'}{3}$。同理可得:
    \[ b = -\frac{c' - a'}{3}, \quad c = -\frac{a' - b'}{3} \]

    代入第二个括号的各项中:
    \begin{align*}
    \frac{a}{b - c} + \frac{b}{c - a} + \frac{c}{a - b} &= \frac{a}{a'} + \frac{b}{b'} + \frac{c}{c'} \\
    &= -\frac{1}{3} \left(\frac{b' - c'}{a'} + \frac{c' - a'}{b'} + \frac{a' - b'}{c'}\right)
    \end{align*}
    利用之前对第一部分推导的结论,将 $a', b', c'$ 代回:
    \begin{align*}
    \frac{a}{b - c} + \frac{b}{c - a} + \frac{c}{a - b} &= -\frac{1}{3} \left[ -\frac{(a' - b')(b' - c')(c' - a')}{a' b' c'} \right] \\
    &= \frac{1}{3} \frac{(-3c)(-3a)(-3b)}{(b - c)(c - a)(a - b)} \\
    &= -9 \frac{abc}{(a - b)(b - c)(c - a)}
    \end{align*}

    最后,将两部分相乘:
    \begin{align*}
    &\left[ -\frac{(a - b)(b - c)(c - a)}{abc} \right] \left[ -9 \frac{abc}{(a - b)(b - c)(c - a)} \right] \\
    &= 9
    \end{align*}
    因此,原式的值为 9。
\end{solution}

    \question 已知 $x^5=1$ 且 $x \neq 1$, 求
    \[
    \frac{x}{1+x^2} + \frac{x^2}{1+x^4} + \frac{x^3}{1+x} + \frac{x^4}{1+x^3} 
    \]
    的值。
    \begin{solution}
        由$x^5=1$,
        \begin{align*}
        \frac{x}{1+x^2} + \frac{x^2}{1+x^4} + \frac{x^3}{1+x} + \frac{x^4}{1+x^3} 
        &= \frac{x}{1+x^2} + \frac{x^3}{x+1} + \frac{x^3}{1+x} + \frac{x}{x^2+1} \\
        &= 2 \left(\frac{x}{1+x^2} + \frac{x^3}{x+1}\right) \\
        &= 2 \left(\frac{x+x^2+x^3+x^5}{1+x+x^2+x^3}\right) \\
        &= 2
        \end{align*}
    \end{solution}

    \question 设 $a,b,c,d \in \mathbb{R}$ 且 $abcd \neq 0$,又 $a+b+c+d=0$,求
    \[
    S=a\left(\frac{1}{b}+\frac{1}{c}+\frac{1}{d}\right)+b\left(\frac{1}{c}+\frac{1}{d}+\frac{1}{a}\right)+c\left(\frac{1}{d}+\frac{1}{a}+\frac{1}{b}\right)+d\left(\frac{1}{a}+\frac{1}{b}+\frac{1}{c}\right)。
    \]
    的值。
    \begin{solution}
        由$a+b+c+d=0$,得
        \[
        \begin{aligned}
        S &= a\left(\frac{1}{b}+\frac{1}{c}+\frac{1}{d}\right) + b\left(\frac{1}{c}+\frac{1}{d}+\frac{1}{a}\right) + c\left(\frac{1}{d}+\frac{1}{a}+\frac{1}{b}\right) + d\left(\frac{1}{a}+\frac{1}{b}+\frac{1}{c}\right) \\
        &= -(b+c+d)\left(\frac{1}{b}+\frac{1}{c}+\frac{1}{d}\right) - (a+c+d)\left(\frac{1}{c}+\frac{1}{d}+\frac{1}{a}\right) \\
        &\quad - (a+b+d)\left(\frac{1}{d}+\frac{1}{a}+\frac{1}{b}\right) - (a+b+c)\left(\frac{1}{a}+\frac{1}{b}+\frac{1}{c}\right) \\
        &= -12-\frac{2(b+c+d)}{a}-\frac{2(a+c+d)}{b} -\frac{2(a+b+d)}{c} -\frac{2(a+b+c)}{d}\\
        &= -12-\frac{-2a}{a} -\frac{-2b}{b} -\frac{-2c}{c} -\frac{-2d}{d}\\
        &= -4
        \end{aligned}
        \]
    \end{solution}

    \question 设 $a,b,c$ 满足 $a+b+c=a^{3}+b^{3}+c^{3}=0$, $n$ 为任意实数, 求 $a^{2n+1}+b^{2n+1}+c^{2n+1}$的值。
    \begin{solution}
        由恒等式
        \[
        a^3 + b^3 + c^3 - 3abc = (a + b + c)(a^2 + b^2 + c^2 - ab - bc - ca)
        \]
        将已知代入得
        \[
        0 - 3abc = 0 \Rightarrow abc = 0.
        \]
        若$c = 0$,由 $a + b + c = 0$ 可得 $b = -a$,则
        \[
        a^{2n+1} + b^{2n+1} + c^{2n+1} = a^{2n+1} + (-a)^{2n+1} + 0  = 0
        \]
        同理若$a = 0$或$b = 0,\ a^{2n+1} + b^{2n+1} + c^{2n+1}=0$
    \end{solution}

\question 已知 $x, y, p, q$ 为实数,且满足 $2x^{2}+3p^{2}=2y^{2}+3q^{2}=(xq-yp)^{2}=6$,求 $$(x^{2}+y^{2})(p^{2}+q^{2})$$ 的值。

\begin{solution}
\textcolor{red}{(待另三种解)}
\end{solution}

    \begin{solution}
        \textbf{解法一}:
        
        设
        \[
        x = \sqrt{3} \cos\alpha, \quad p = \sqrt{2} \sin\alpha, \quad y = \sqrt{3} \cos\beta, \quad q = \sqrt{2} \sin\beta
        \]
        则
        \[
        6=(xq - yp)^2 = (\sqrt{6}(\cos\alpha \sin\beta - \cos\beta \sin\alpha))^2= 6\sin^2(\beta - \alpha)
        \]
        于是$\sin^2(\beta - \alpha) = 1 \Rightarrow \beta - \alpha = \pm \frac{\pi}{2} + 2k\pi,$
        因此有
        \[
        \cos\beta = \mp \sin\alpha, \quad \sin\beta = \pm \cos\alpha
        \]
        故
        \[
        (x^2 + y^2)(p^2 + q^2) = (\underbrace{3\cos^2\alpha + 3\cos^2\beta}_{=3(\cos^2\alpha + \sin^2\alpha) = 3})(\underbrace{2\sin^2\alpha + 2\sin^2\beta}_{=2(\sin^2\alpha + \cos^2\alpha) = 2})
        = 6
        \]
    \end{solution}

    \question 解方程
\[
3(x+1)^6-2(x-1)^6 = (x^2-1)^3
\]

\begin{solution}

\noindent
\textbf{方法 A}

\noindent
原方程化为
\[
(x+1)^6 - 2(x-1)^6 = (x^2-1)^3 = (x+1)^3(x-1)^3
\]
除以 $(x-1)^6$($x\neq 1$):
\[
\left(\frac{x+1}{x-1}\right)^6 - 2 = \left(\frac{x+1}{x-1}\right)^3
\]
设 $y = \left(\frac{x+1}{x-1}\right)^3$,得到二次方程:
\[
y^2 - y - 2 = 0 \implies (y-2)(y+1) = 0
\]
\[
y = 2 \quad \text{或} \quad y=-1
\]

\noindent
分别解得 $x$:
\[
\frac{x+1}{x-1} = 2^{\frac{1}{3}} \implies x = \frac{1+2^{\frac{1}{3}}}{2^{\frac{1}{3}}-1} = 3+2^{\frac{2}{3}}+2^{\frac{1}{3}}
\]
\[
\frac{x+1}{x-1} = -1 \implies x = 0
\]

\vspace{0.5em}
\noindent
\textbf{方法 B}

\noindent
设 $a=(x+1)^3, \ b=(x-1)^3$,原方程化为
\[
a^2 - 2b^2 = ab \implies a^2 - ab - 2b^2 = 0 \implies (a-2b)(a+b)=0
\]

\noindent
解每个因式:

\[
a+b=0 \implies (x+1)^3 + (x-1)^3 = 0 \implies x = 0
\]

\[
a-2b=0 \implies (x+1)^3 = 2(x-1)^3 \implies x = 3+2^{\frac{2}{3}}+2^{\frac{1}{3}}
\]

\noindent
\textbf{结论}:
\[
x = 0 \quad \text{或} \quad x = 3+\sqrt[3]{4}+\sqrt[3]{2}
\]

\end{solution}

\question 解方程
\[
(a+b)(ax+b)(a-bx)=(a^2x-b^2)(a+bx), \quad x\in \mathbb{R}.
\]

\begin{solution}

\noindent
\textbf{整理方程并展开}
\begin{align*}
0 &= (a^2x-b^2)(a+bx) - (a+b)(ax+b)(a-bx) \\
0 &= a^3x + a^2bx^2 - a^2b - b^3x - (a+b)(a^2x - abx^2 + ab - b^2x) \\
0 &= a^3x + a^2bx^2 - a^2b - b^3x - (a^3x - a^2bx^2 + a^2b - ab^2x + a^2bx - ab^2x^2 + ab^2 - b^3x) \\
0 &= (2a^2b + ab^2)x^2 + (ab^2 - a^2b)x - (a^2b + ab^2)
\end{align*}

\noindent
\textbf{除以 $ab$ 得二次方程}
\[
(2a+b)x^2 + (b-a)x - (a+2b) = 0
\]

\noindent
\textbf{判别式}
\begin{align*}
\Delta &= (b-a)^2 + 4(2a+b)(a+2b) \\
&= b^2-2ab+a^2 + 4(2a^2+5ab+2b^2) \\
&= b^2-2ab+a^2 + 8a^2+20ab+8b^2 \\
&= 9a^2+18ab+9b^2 \\
&= 9(a+b)^2
\end{align*}

\noindent
\textbf{使用二次公式}
\[
x = \frac{-(b-a) \pm \sqrt{9(a+b)^2}}{2(2a+b)} = \frac{a-b \pm 3(a+b)}{2(2a+b)}
\]

\noindent
\textbf{求两根}

\noindent
第一根:
\[
x = \frac{a-b + 3(a+b)}{2(2a+b)} = \frac{4a+2b}{2(2a+b)} = 1
\]

\noindent
第二根:
\[
x = \frac{a-b - 3(a+b)}{2(2a+b)} = \frac{-2a-4b}{2(2a+b)} = -\frac{a+2b}{2a+b}
\]

\noindent
\textbf{结论}:
\[
x = 1 \quad \text{或} \quad x = -\frac{a+2b}{2a+b}
\]

\end{solution}

\end{questions}
\pagebreak

\begin{center}
  {\fontsize{30pt}{26pt}\selectfont
    \hypertarget{因式定理、余式定理}{因式定理、余式定理} \label{因式定理、余式定理}
  }
\end{center}
\separator
\vspace{1pt}

\begin{questions}
    \question 设 $f(x) = a x^3 + b x^2 - 18 x + 3, g(x) = a x^3 + 9 x^2 + b x - 9$。当$f(x)$ 和 $g(x)$ 除以 $2x - 1$ 时所得余数相同,且当 $f(x)$ 除以 $x - 2$ 时余数是 $-5$。
    \begin{parts}
    \part 求 $a,\;b$ 的值。  
    \begin{solution}
        令 $x=\tfrac{1}{2}$,有:
        \[
        f\left(\frac{1}{2}\right) = g\left(\frac{1}{2}\right) \Rightarrow a\cdot\frac{1}{8} + b\cdot\frac{1}{4} - 9 + 3
         = a\cdot\frac{1}{8} + \frac{9}{4} + \frac{b}{2} - 9
         \Rightarrow b = 3
        \]
        又
        \[
        f(2) = 8a + 4b - 36 + 3 = -5
        \Rightarrow a = 2
        \]
        $\therefore a=2,\;b=3$
    \end{solution}
        
    \part 若 $x+4$ 整除 $f(x) + k g(x)$,求 $k$。
    \begin{solution}
        有\[
        f(-4) + k g(-4) = 0 \Rightarrow -128a + 16b + 72 + 3 + k(-128a + 144 - 4b - 9)=0
        \]
        代入 $a=2,\;b=3$ 得:
        \[
            -133(1 + k) = 0 \Rightarrow k = -1
        \]
    \end{solution}
    \end{parts}
    
    \question 已知 $f(x)$ 为三次多项式,以 $x^2 + x + 2$ 除之得余式 $x + 3$,以 $x^2 + x - 2$ 除之得余式 $5x + 7$,求 $f(x)$。
    \begin{solution}
        据题意有$f(1)=5(1)+7=12,f(-2)=5(-2)+7=-3,$
        设$$f(x)=(x^2 + x + 2)(ax+b)+x+3$$
        令$x=1,$ $$f(1)=(1+1+2)(a+b)+1+3=12 \Rightarrow a+b=2$$
        令$x=-2,$ $$f(-2)=(4-2+2)(-2a+b)-2+3=-3 \Rightarrow -2a+b=-1$$
        解得 $a=1,\;b=1$,所以$$f(x)=(x^2 + x + 2)(x+1)+x+3=x^3 + 2x^2 + 4x + 5$$
    \end{solution}
    
    \question 若 $f(x)$ 以 $x^2 - 1$ 除余 $3x + 2$,$g(x)$ 以 $x^2 + 2x - 3$ 除余 $5x + 2$,求 $(x+3) f(x) + (5x^2 + 1) g(x)$ 除以 $x - 1$ 的余式。
    \begin{solution}
        即求$(1+3)f(1)+(5+1)g(1)=4f(1)+6g(1)$的值,由已知
        $$f(1)=3\cdot1+2=5,\;g(1)=5\cdot1+2=7$$故$(x+3) f(x) + (5x^2 + 1) g(x)$ 除以 $x - 1$ 的余式为$$4\cdot5+6\cdot7=62$$
    \end{solution}
    
    \question 以 $x^2 + 2x + 3$ 除 $f(x)$ 余 $x + 12$,以 $(x+1)^2$ 除 $f(x)$ 余 $5x + 4$,求以 $(x+1)(x^2 + 2x + 3)$ 除 $f(x)$ 的余式。
    \begin{solution}
        同上,设$$f(x)=(x+1)(x^2+2x+3)Q(x)+a(x^2+2x+3)+x+12$$
        据题意$f(-1)=-5+4=-1$,现令$x=-1$有$$f(-1)=a(1-2+3)-1+12=-1 \Rightarrow a=-6$$

        $\therefore$ 以 $(x+1)(x^2 + 2x + 3)$ 除 $f(x)$ 的余式为 $-6(x^2 + 2x + 3)+x+12=-6x^2 - 11x - 6$    
    \end{solution}
    
    \question $f(x)$ 以 $x(x - 1)$ 除之,余式为 $-x + 3$;以 $x(x+1)$ 除之,余式为 $-3x + 3$,则 $f(x)$ 除以 $x(x^2 - 1)$ 的余式为?
    \begin{solution}
        设$$f(x)=x(x^2-1)Q(x)+ax(x-1)-x+3$$
        据题意$f(-1)=3+3=6$,令$x=-1$有$$f(-1)=a(-1)(-2)+1+3=6 \Rightarrow a=1$$

        $\therefore$ $f(x)$ 除以 $x(x^2 - 1)$ 的余式为 $x(x-1)-x+3=x^2 - 2x + 3$  
    \end{solution}
    
    \question 设多项式 $f(x)$ 除以 $(x+1)^3$ 得余式 $2x^2 + 8$,除以 $(x - 2)^2$ 得余式 $15x + 40$,且 $\deg f(x) \geq 4$,则 $f(x)$ 除以 $(x+1)^3 (x-2)$ 的余式为?
    \begin{solution}
        设$$f(x)=(x+1)^3 (x-2)Q(x)+a(x+1)^3+2x^2 + 8$$
        据题意$f(2)=15\cdot2+40=70$,令$x=2$有$$f(2)=a(2+1)^3+2\cdot2^2 + 8 =70\Rightarrow a=2$$

        $\therefore$ $f(x)$ 除以 $(x+1)^3 (x-2)$ 的余式为 $2(x+1)^3+2x^2 + 8=2x^3 + 8x^2 + 6x + 10$  
    \end{solution}

    
    \question 设 $\deg f(x) \geq 3$, 且 $f(x)$ 以 $(x-1)^2$ 除之余 $3x + 2$,以 $(x+2)^2$ 除之余 $5x - 3$,求:
        
    \begin{parts}
    \part 以 $x-1$ 除之的余式。
    \begin{solution}
        即 $f(1) = 3 \cdot 1 + 2 = 5$
    \end{solution}
        
    \part 以 $(x-1)(x+2)$ 除之的余式。
    \begin{solution}
        设$$f(x)=(x-1)(x+2)Q(x)+ax+b$$
        令$x=1,$ $$f(1)=a+b=5$$
        令$x=-2,$ $$f(-2)=-2a+b=-13$$ 
        解得 $a = 6,\ b = -1$,故余式为$6x-1$
    \end{solution}
        
    \part 以 $(x-1)^2(x+2)$ 除之的余式。
    \begin{solution}
        设$$f(x)=(x-1)^2(x+2)Q(x)+a(x-1)^2+3x+2$$
        令$x=-2,$ $$f(-2)=9a+-4=-13$$ 
        解得 $a = -1 $,故余式为$-x^2 + 5x + 1$
    \end{solution}
    \end{parts}
    
    \question $f(x)$ 为一多项式,若 $(x+1) f(x)$ 除以 $x^2 + x + 1$ 的余式为 $5x + 3$,则 $f(x)$ 除以$x^2 + x + 1$的余式为何?
    \begin{solution}
        据题意有$$(x+1)f(x)=(x+1)(x^2 + x + 1)Q(x)+a(x^2+x+1)+5x+3$$
        且设$$f(x)=(x^2 + x + 1)Q(x)+ax+b$$
        两边乘$x+1$有$$(x+1)f(x)=(x+1)(x^2 + x + 1)Q(x)+(x+1)(ax+b)$$
        比较得 $$a(x^2+x+1)+5x+3=(x+1)(ax+b)$$
        即 $$ax^2+(a+b)x+b=ax^2+(a+5)x+a+3 $$
        得知 $a=2,\;b=5$,余式为 $2x+5$
    \end{solution}
    
    \question 以 $(x+1)^3$ 除 $f(x)$ 余式为 $x^2 - 2x + 3$,求以 $(x+1)^2$ 除 $f(x)$ 所得余式。
    \begin{solution}
        设\[
        f(x)=Q(x)(x+1)^{3}+x^{2}-2x+3
        \]
        有\[
        f(x)=Q(x)(x+1)^{3}+(x+1)^{2}+(-4x+2) =(x+1)^{2}[Q(x)(x+1)+1]+(-4x+2)  
        \]
        轻而易举得知余式为$-4x + 2$.
    \end{solution}

    \question 设 $f(x)$ 为实系数三次多项式,若 $f(x)$ 除以 $x-2$ 的余式为 $-5$,且 $(x+1)f(x)$ 除以 $x^3-3$ 的余式为 $3x-1$,求多项式 $f(x)$ 
    \begin{solution}
        设 
        \[
        f(x) = a x^3 + b x^2 + c x + d,
        \]
        则
        \[
        (x+1) f(x) = a x^4 + (a + b) x^3 + (b + c) x^2 + (c + d) x + d.
        \]
        由题意,
        \[
        f(2) = 8a + 4b + 2c + d = -5 \tag{1}
        \]
        且有
        \[
        (x+1) f(x) = (x^3 - 3)(a x + (a + b)) + 3x - 1,
        \]
        比较系数得:
        \[
        b + c = 0 \tag{2}
        \]
        \[
        3a + 3b + d = -1 \tag{3}
        \]
        \[
        3a + c + d = 3 \tag{4}
        \]
        由$(1)-(4)$联立解得
        \[
        a = -1, \quad b = -1, \quad c = 1, \quad d = 5
        \]
        因此
        \[
        f(x)=-x^3-x^2+x+5
        \]
    \end{solution}

    \question 已知一多项式 $f(x) = x^{2025}(x^{2} + ax + b)$,其中 $a, b$ 为实数,如果将 $f(x)$ 除以 $(x - 2)^2$得余式 $2^{2025}(x - 2)$,求 $a, b$。
    \begin{solution}
        设
        \[
        f(x) = x^{2025}(x^2 + ax + b) = (x - 2)^2 p(x) + 2^{2025}(x - 2)
        \]
        对$x$求导有
        \[
        f'(x) = 2025x^{2024}(x^2 + ax + b) + x^{2025}(2x + a)
        = 2(x - 2)p(x) + (x - 2)^2 p'(x) + 2^{2025}
        \]
        现令$x=2$,则
        \[
        \begin{cases}
         \ 2^{2025}(4 + 2a + b) = 0  \\
         \ 2025 \cdot 2^{2024}(4 + 2a + b) + 2^{2025}(4 + a) = 2^{2025}
        \end{cases}
        \]
        解得
        \[
         a = -3,b = 2
        \]
    \end{solution}
    
    \question 设 $f(x) = x^4 + 5x^3 + a x^2 + b x + c$ 可被 $(x-1)^3$ 整除,求 $a,b,c$ 的值。 
    \ifprintanswers
    \begin{table}[H]
        \centering
        \begin{tabular}{llllll}
        1 & 5  & $a$    & $b$    & \multicolumn{1}{l|}{$c$}  & 1 \\
        & 1   & 6    & $a+6$   & \multicolumn{1}{l|}{$a+b+6$} &   \\ \cline{1-5}
        1 & 6    & $a$+6    & \multicolumn{1}{l|}{$a+b+6$} & $a+b+c+6$       &   \\
        & 1    & 7    & \multicolumn{1}{l|}{$a+13$}  &  &   \\ \cline{1-4}
        1 & 7   & \multicolumn{1}{l|}{$a+13$} & $2a+b+19$  &   &  \\
          & 1  & \multicolumn{1}{l|}{8}  &  &  &   \\ \cline{1-3}
        1 & \multicolumn{1}{l|}{8} & $a+21$ & & &   \\
          & \multicolumn{1}{l|}{1} &  &  &   &   \\ \cline{1-2}
        1 & 9  &  &    &    &  
        \end{tabular}
    \end{table}
    \fi
    \begin{solution}
        连续综合除法得\[
        f(x)=(x-1)^4+9(x-1)^3+(a+21)(x-1)^2+(2a+b+19)(x-1)+(a+b+c+6)
        \]
        已知 $f(x)$ 可被 $(x-1)^3$ 整除,则
        \[
        \begin{cases}
            a+21=0\\2a+b+19=0\\a+b+c+6=0
        \end{cases} \Rightarrow (a,b,c)=(-21,23,-8)
        \]
    \end{solution}

    \question 已知 $x^2 - x + b$ 为多项式 $6x^4 - 7x^3 + a x^2 + 3x + 2$ 的因式,求 $a,b$ 的值。
    \begin{solution}
        设 $f(x) = 6x^4 - 7x^3 + ax^2 + 3x + 2= (x^2 - x + b)(6x^2 + px + q)$,展开右边得
        \[
        (x^2 - x + b)(6x^2 + px + q)
        = 6x^4 + (p - 6)x^3 + (q - p + 6b)x^2 + (pb - q)x + bq
        \]
        比较系数得
        \begin{align}
        p - 6 &= -7 \tag{1} \\
        q - p + 6b &= a \tag{2} \\
        pb - q &= 3 \tag{3} \\
        bq &= 2 \tag{4}
        \end{align}
        由(3)得$q = -b - 3$, 代入(4)得
        \[
        b(-b - 3) = 2 
        \Rightarrow (b + 1)(b + 2) = 0
        \Rightarrow b = -1 \text{ 或 } -2
        \]
        当 $b = -1$ 时,$\;q = -2,a = -7$;当 $b = -2$ 时,$\;q = -1,a = - 12$
        
        $\therefore (a,b) = (-7, -1) \text{ 或 } \;(-12, -2)$
    \end{solution}
    
    \question 若 $a,b,c$ 为相异实数,分别用 $x - a, x - b, x - c$ 除多项式 $p(x)$,所得的余数分别为 $a, b, c$,求以 $(x - a)(x - b)(x - c)$ 除 $p(x)$ 的余式。 
    \begin{solution}
        设
        \[
        p(x) = (x-a)(x-b)(x-c)q(x) + r(x-a)(x-b)+s(x-a)+a,
        \]
        令 $x = b$, 由 $a \ne b$,
        \[
        p(b) = s(b-a)+a = b  \Rightarrow s(b-a) = b-a \Rightarrow s=1
        \]
        令 $x = c$,由 $a \ne c,b \ne c,$
        \[
        p(c) = r(c-a)(c-b)+(c-a)+a = c  \Rightarrow r(c-a)(c-b)=0 \Rightarrow r=0
        \]
        因此以 $(x - a)(x - b)(x - c)$ 除 $p(x)$ 的余式为$x$
    \end{solution}
    
    \question 若 $g(x)$ 除以 $2x - 3$ 的余式为 $1$,且 $f(x) = g(x)(2x - 3) + 5$,求 $[f(x)]^2$ 除以 $(2x - 3)^2$ 的余式。
    \begin{solution}
        设\[
        g(x)=(2x-3)Q(x)+1
        \]
        由$f(x) = g(x)(2x - 3) + 5$,
        \begin{align*}
        [f(x)]^2 
        &=[g(x)]^2(2x-3)^2+10g(x)(2x-3)+25 \\
        &=[g(x)]^2(2x-3)^2+10[(2x-3)Q(x)+1](2x-3)+25\\
        &=[g(x)]^2(2x-3)^2+10Q(x)(2x-3)^2+10(2x-3)+25\\
        &=[g(x)]^2(2x-3)^2+10Q(x)(2x-3)^2+20x-5
        \end{align*}
        故$[f(x)]^2$ 除以 $(2x - 3)^2$ 的余式为$20x - 5$
    \end{solution}
    
    \question 若多项式 $p(x) = x^2 + b x + c$,其中 $b,c$ 为实数,且 $p(p(1)) = p(p(2)) = 0$,且 $p(1) \neq p(2)$,求 $b,c$ 的值。 
    \begin{solution}
        由 \(p(p(1)) =p(p(2)) = 0\),意味 \(p(1)\)和\(p(2) \)是方程 \(x^2 + bx + c = 0\) 的两根,且
        \[
        p(1)+p(2) = -b,\quad p(1)p(2) = c
        \]
        即
        \[
         (1 + b + c) + (4+2b+c)=-b \tag{1}
        \]
        \[
         (1 + b + c)(4+2b+c)=c \tag{2}
        \]
        由(1)得$2b+c = -\dfrac{5}{2} $,代入(2):
        \[
        (1 - \frac{5}{2} - b)(4  - \frac{5}{2}) = -\frac{5}{2} - 2b \Rightarrow  b = -\frac{1}{2},c=-\frac{3}{2}
        \]
    \end{solution}
    
    \question $a,b,c$ 为整数,且 $0 < a < b$,若 $x - c$ 是多项式 $x(x-a)( x - b) - 17$ 的因式,求 $(a,b,c)$。 
    \begin{solution}
        已知$f(x) = x(x - a)(x - b) - 17$且 \(x - c\) 是它的因式,于是
        \[
        f(c) = c(c - a)(c - b) - 17 = 0 \Rightarrow c(c - a)(c - b) = 17.
        \]
        由于 \(a, b, c\) 均为整数,且 \(0 < a < b\),我们只需枚举整数三元组使得左边乘积为 17。
    
        因为 17 是质数,其非零整数因式只有:
        \[
        (\pm1, \pm1, \pm17) \text{ 的排列组合。}
        \]
    
        枚举 \(c = 1\),
        \[
        1(1 - a)(1 - b) = 17 \Rightarrow (1 - a)(1 - b) = 17.
        \]
    
        由于 \(17 = 1 \times 17 = (-1)\times(-17)\),列出可能组合:
        \begin{itemize}
            \item \(1 - a = 1,\ 1 - b = 17 \Rightarrow a = 0,\ b = -16\),不满足 \(0 < a < b\)。
            \item \(1 - a = -1,\ 1 - b = -17 \Rightarrow a = 2,\ b = 18\),满足条件!
        \end{itemize}
    
        若 \(c = -1\):
        \[
        (-1)(-1 - a)(-1 - b) = 17 \Rightarrow -(a + 1)(b + 1) = 17
        \Rightarrow (a + 1)(b + 1) = -17
        \]
        没有满足 \(0 < a < b\) 的整数解,尝试 \(c = 17, -17\) 会得到更大的数,不满足 \(0 < a < b\)。
    
        因此,唯一符合题意的解为:
        \[
        (a, b, c) = (2, 18, 1)
        \]
    \end{solution}

    \question 设 $f(x)$ 与 $g(x)$ 均为实系数二次多项式且首项系数都是 $2$。已知 $(f(x))^2$ 除以 $g(x)$ 的余式为 $5x+3$,而 $(g(x))^2$ 除以 $f(x)$ 的余式为 $x+1$,求 $f(x),g(x)$。
    \begin{solution}
        设
        \[
        f(x) = 2x^2 + a x + b, \quad g(x) = f(x) - c x - d = 2x^2 + (a-c)x + b-d.
        \]
        则
        \[
        (f(x))^2 = (g(x) + c x + d)^2 = (g(x))^2 + 2(c x + d) g(x) + (c x + d)^2.
        \]
        由已知 $(f(x))^2$ 除以 $g(x)$ 的余式为 $5x+3$,
        \[
        (c x + d)^2 = c^2 x^2 + 2 c d x + d^2 = \frac{c^2}{2} g(x) + \Big(2 c d - \frac{c^2}{2}(a-c)\Big)x + \Big(d^2 - \frac{c^2}{2}(b-d)\Big).
        \]
        比较系数得
        \[
        \begin{cases}
        4 c d - a c^2 + c^3 = 10,\\
        2 d^2 - b c^2 + c^2 d = 6.
        \end{cases}
        \]
        同理,由$(g(x))^2$ 除以 $f(x)$ 的余式为 $x+1$得
        \[
        \begin{cases}
        4 c d - a c^2 = 2,\\
        2 d^2 - b c^2 = 2.
        \end{cases}
        \]
        解得
        \[
        c^3 = 8 \Rightarrow c = 2, \quad c^2 d = 4 \Rightarrow d = 1, \quad a = \frac{3}{2}, \quad b = 0.
        \]
        于是
        \[
        f(x) = 2 x^2 + \frac{3}{2}x, \quad g(x) = 2 x^2 - \frac{1}{2}x - 1.
        \]
    \end{solution}

    \question 求多项式 $(x + 1)^6$ 除以 $x^2 + 1$ 的余式。
    \begin{solution}
        有$x^2 \equiv -1 \pmod{x^2+1}$,则
        $$(x+1)^6 = (x^2 + 2x + 1)^3 \equiv (2x)^3
        = 8x \cdot x^2 \equiv 8x \cdot (-1)
        = -8x \pmod{x^2+1}$$
    \end{solution}  
    
    \question 设 $f(x) = x^4 + x^3 + x^2 + x + 1$,求 $f(x^5)$ 除以 $f(x)$ 所得余式。 
    \begin{solution}
        首先注意到
        \[
        f(x) = x^4 + x^3 + x^2 + x + 1 = \frac{x^5 - 1}{x - 1}, \quad x \ne 1,
        \]
        所以 \( f(x) \mid x^5 - 1 \),即
        \[
        x^5 \equiv 1 \pmod{f(x)}
        \]
        故
        \[
        f(x^5) = (x^5)^4 + (x^5)^3 + (x^5)^2 + x^5 + 1 \equiv 1 + 1 + 1 + 1 + 1 = 5 \pmod{f(x)}
        \]
    \end{solution}

    \question 求以 $x^2 + 2x + 3$ 除 $(x^2 + 3x + 4)^4$ 所得的余式。
    \begin{solution}
    有
        \[
        (x^2 + 3x + 4)^4 \equiv (x+1)^4 = (x^2+2x+1)^2 \equiv (-2)^2 = 4 \pmod{x^2 + 2x + 3}
        \]
    \end{solution}

    
    \question 设 $f(x) = x^{37} - 2 x^{26} + 4 x^7 - 3$,则
    \begin{parts}
    \part 求$f(x)$ 除以 $x^2 + 1$ 的余式。
    \begin{solution}
        有
        \begin{align*}
        f(x)&=x^{37}-2x^{26}+4x^{7}-3 \\
        &=(x^{2})^{18} \cdot x - 2(x^{2})^{13} + 4(x^{2})^{3} \cdot x - 3 \\
        &=(-1)^{18} \cdot x - 2(-1)^{13} + 4(-1)^{3} \cdot x - 3 \\
        &=x+2-4x-3 \\
        &=-3x-1 \pmod{x^2 + 1}
        \end{align*}
    \end{solution}
    \part 求$f(x)$ 除以 $x^2 + x + 1$ 的余式。
    \begin{solution}
        有 $x^{3}\equiv1 \pmod{x^{2}+x+1}$,于是
        \begin{align*}
        f(x)&=x^{37}-2x^{26}+4x^{7}-3 \\
        &=(x^{3})^{12} \cdot x - 2(x^{3})^{8} \cdot x^{2} + 4(x^{3})^{2} \cdot x - 3 \\
        &\equiv x-2x^{2}+4x-3 \\
        &=-2x^{2}+5x-3 \\
        &\equiv-2(-x-1)+5x-3 \\
        &=7x-1 \pmod{x^{2}+x+1}
        \end{align*}
    \end{solution}  
    \end{parts}

    \question 求以 $x^4 - x$ 除 $x^{87} - 2 x^{44} - x^3 + 3 x^2 + 1$ 所得的余式。
    \begin{solution}
        有$x^4\equiv x \Rightarrow x^3\equiv 1\pmod{x^4-x}$,故
        \[
        x^{87} - 2 x^{44} - x^3 + 3 x^2 + 1
        \equiv 1 - 2 x^2 - 1 + 3 x^2 + 1
        = x^2 + 1      
        \]
    \end{solution}
                
    \question 求$x^{100}$ 除以 $x^3 + 2 x^2 + 2 x + 1$ 的余式。
    \begin{solution}
        $x^{100}$ 除以 $x+1$ 的余式即$(-1)^{100}=1$,又\[
        x^3\equiv1\pmod{x^2+x+1}
        \]故$x^{100}$ 除以 $x^2+x+1$ 的余式为 $x^{100} \equiv x\pmod{x^2+x+1}$,现设$$x^{100}=(x+1)(x^2+x+1)Q(x)+a(x^2+x+1)+x$$
        令$x=-1$有$$1=a(1-1+1)-1\Rightarrow a=2$$

        $\therefore$ $x^{100}$ 除以 $x^3 + 2 x^2 + 2 x + 1$ 的余式为 $2x^2+3x+2$ 
    \end{solution}
    
    \question 求$x^{200}$ 除以 $(x - 1)^2$ 的余式。
    \begin{solution}
        发现$f(x)$ 除以 $(x-1)$ 的余式为 $f(1)=1^{200}=1$ ,将所求写成
        \begin{align*}
        x^{200}&=(x-1)^{2}Q(x)+a(x-1)+1 \\
        x^{200}-1&=(x-1)^{2}Q(x)+a(x-1) \\
        (x-1)(x^{199}+\dots+x+1)&=(x-1)^{2}Q(x)+a(x-1) \\
        x^{199}+\dots+x+1&=(x-1)Q(x)+a
        \end{align*}
        两边代 $x=1$ ,得到
        $$200=0+a$$
        故余式为 $200(x-1)+1=200x-199$
    \end{solution}

    \begin{solution}
        设 $t=x-1$ ,则题目变成 $x^{200}=(t+1)^{200}$ 除以 $t^{2}$ 的余式。由二项式定理,
        $$(t+1)^{200}=\comb{200}{200}t^{200}+\dots+\comb{200}{2}t^{2}+\comb{200}{1}t+\comb{200}{0}$$
            则显然, $(t+1)^{200}$ 除以 $t^{2}$ 的余式即为 $\comb{200}{1}t+\comb{200}{0}=200(x-1)+1=200x-199$
    \end{solution}
    
    \question 已知多项式 $f(x)=x^{130}-1,g(x)=x^{4}-x^{3}+2x^{2}-x+1$,求 $f(x)$ 除以 $g(x)$ 的余式。
    \begin{solution}
        设
        \[
        x^{130}-1= (x^2 + 1)(x^2 - x + 1)Q(x) + Ax^3 + Bx^2 + Cx + D.
        \]
        令 $x = i$,则
        \[
        -1 - 1 = Ai^3 + Bi^2 + Ci + D = (-B + D) + (-A + C)i,
        \]
        得
        \[
        \begin{cases}
        -B + D = -2, \\
        -A + C = 0
        \end{cases}
        \]
        令 $\omega = \dfrac{1 + \sqrt{3}i}{2}$,
        \[
        \omega^{130}-1=-\omega -1=-A+B(\omega-1)+C\omega+D =(B+C)\omega-A-B+D
        \]
        得到方程组
        \[
        \begin{cases}
        B+C=-1 \\
        -A-B+D=-1
        \end{cases}
        \]
        解得$A = -1, B = 0, C = -1, D = -2,$因此余式为$-x^{3} - x - 2$
    \end{solution}

    \question 设 $f(x)$ 是一个次数有限的多项式,且满足
    \[
    (x + 9) f(x + 1) = (x + 3) f(x + 3),
    \]
    已知 $f(0) = 1$,求 $f(1)$ 的值。
    \begin{solution}
        令 $x = -3$ 得 $f(-2) = 0$,所以 $f(x)$ 有因式 $(x+2)$,设 $f(x) = (x+2)g(x)$得
        \[
        (x+9)(x+3)g(x+1) = (x+3)(x+5)g(x+3) \Rightarrow (x+9)g(x+1) = (x+5)g(x+3)
        \]
        同理,令 $x = -5$ 得 $g(-4) = 0$,故 $g(x)$ 有因式 $(x+4)$,设 $g(x) = (x+4)h(x)$得
        \[
        (x+9)(x+5)h(x+1) = (x+5)(x+7)h(x+3) \Rightarrow (x+9)h(x+1) = (x+7)h(x+3)
        \]
        令 $x = -7$ 得 $h(-6) = 0$,故 $h(x)$ 有因式 $(x+6)$,设 $h(x) = (x+6)p(x)$得
        \[
        (x+9)(x+7)p(x+1) = (x+7)(x+9)p(x+3) \Rightarrow p(x+1) = p(x+3)
        \]
        即$p(x)$是周期为2的函数,因为 $p(x)$ 是次数有限的多项式,故 $p(x)=c,c\in\mathbb{R}$ 
        
        于是$f(x) = c(x+2)(x+4)(x+6)$,由 $f(0) = 1$ 得$c = \frac{1}{48}$,故
        \[
        f(1) = \frac{1}{48}(3)(5)(7) =\frac{35}{16}
        \]
    \end{solution}
\end{questions}
\pagebreak

\begin{center}
  {\fontsize{30pt}{26pt}\selectfont
    \hypertarget{根式、绝对值、取整}{根式、绝对值、取整} \label{根式、绝对值、取整}
  }
\end{center}
\separator
\vspace{1pt}
\begin{questions}
    \question 解方程
\[
\sqrt[3]{x} + \sqrt[3]{2x-3} = \sqrt[3]{12(x-1)}, \quad x\in \mathbb{R}
\]

\begin{solution}

\noindent
设
\[
a = \sqrt[3]{x}, \quad b = \sqrt[3]{2x-3}, \quad c = \sqrt[3]{12(x-1)}
\]
原方程可写为
\[
a + b = c
\]

\noindent
对两边立方:
\[
(a+b)^3 = c^3
\]
利用立方展开公式:
\[
a^3 + 3a^2b + 3ab^2 + b^3 = c^3
\]
代入 $a^3 = x$, $b^3 = 2x-3$, $c^3 = 12(x-1)$:
\[
x + (2x-3) + 3a b (a+b) = 12(x-1)
\]
\[
3(a b)(c) = 12(x-1) - 3x + 3 = 9(x-1)
\]

\noindent
注意 $ab c = \sqrt[3]{x(2x-3)12(x-1)}$,于是得到
\[
3\sqrt[3]{x(2x-3)12(x-1)} = 9(x-1)
\]

\noindent
两边同时立方:
\[
27 \cdot x(2x-3)12(x-1) = 729(x-1)^3
\]
\[
324 x(2x-3)(x-1) = 729(x-1)^3
\]

\noindent
因为 $x=1$ 是一个解,可以除去 $(x-1)$:
\[
324 x(2x-3) = 729(x-1)^2
\]
\[
4x(2x-3) = 9(x-1)^2
\]

\noindent
化简得到二次方程:
\[
8x^2 -12x = 9x^2 -18x + 9
\]
\[
0 = x^2 -6x + 9
\]
\[
(x-3)^2 = 0
\]

\noindent
因此解为
\[
x = 1 \quad \text{或} \quad x = 3
\]

\end{solution}

\question 求实数根
\[
\sqrt{6x-9} + \sqrt{2x-5} = x-1
\]

\begin{solution}

\noindent
\textbf{步骤 1:利用平方差公式}

\[
(\sqrt{6x-9} + \sqrt{2x-5})(\sqrt{6x-9} - \sqrt{2x-5}) = (6x-9) - (2x-5) = 4x-4
\]

\noindent
将原方程除以上式:
\[
\sqrt{6x-9} - \sqrt{2x-5} = \frac{4x-4}{x-1} = 4
\]

\noindent
\textbf{步骤 2:两边平方}
\begin{align*}
(\sqrt{6x-9} - \sqrt{2x-5})^2 &= 16 \\
6x-9 - 2\sqrt{(6x-9)(2x-5)} + 2x-5 &= 16 \\
8x - 14 - 2\sqrt{12x^2 - 48x + 45} &= 16 \\
8x - 30 &= 2\sqrt{12x^2 - 48x + 45} \\
4x - 15 &= \sqrt{12x^2 - 48x + 45}
\end{align*}

\noindent
\textbf{步骤 3:再次平方}
\begin{align*}
(4x-15)^2 &= 12x^2 - 48x + 45 \\
16x^2 - 120x + 225 &= 12x^2 - 48x + 45 \\
4x^2 - 72x + 180 &= 0 \\
x^2 - 18x + 45 &= 0 \\
(x-3)(x-15) &= 0
\end{align*}

\noindent
\textbf{步骤 4:检验根}
\[
x=3: \sqrt{6\cdot 3 -9} + \sqrt{2\cdot 3 -5} = \sqrt{9} + \sqrt{1} = 3+1 = 4 \neq 2
\]
\[
x=15: \sqrt{6\cdot 15 -9} + \sqrt{2\cdot 15 -5} = \sqrt{81} + \sqrt{25} = 9+5 = 14 = 15-1
\]

\noindent
\textbf{结论:实数根为}
\[
\boxed{x=15}
\]

\end{solution}

\question 解方程
\[
4+\sqrt{x^2-6x+13} = x+\sqrt{2x-5}, \quad x\in \mathbb{R}, \, x\geq \frac{5}{2}.
\]

\begin{solution}

\noindent
将方程化为三项形式:
\[
\sqrt{x^2-6x+13} = x-4 + \sqrt{2x-5}
\]

\noindent
设 $x-4 = y$,则 $x = y+4$:
\begin{align*}
\sqrt{(y+4)^2 - 6(y+4) + 13} &= y + \sqrt{2(y+4)-5} \\
\sqrt{y^2 + 2y + 5} &= y + \sqrt{2y+3}
\end{align*}

\noindent
两边平方:
\begin{align*}
y^2 + 2y + 5 &= y^2 + 2y\sqrt{2y+3} + 2y + 3 \\
5 &= 2y\sqrt{2y+3} + 3 \\
2 &= 2y\sqrt{2y+3} \\
1 &= y\sqrt{2y+3}
\end{align*}

\noindent
再次平方:
\[
1 = y^2(2y+3) \implies 2y^3 + 3y^2 - 1 = 0
\]

\noindent
通过因式分解:
\[
2y^3 + 3y^2 - 1 = (y+1)(2y^2+y-1) = 0
\]

\noindent
解得
\[
y = -1, \quad y = \frac{1}{2}
\]

\noindent
回代 $x = y+4$:
\[
x = 3, \quad x = \frac{9}{2}
\]

\noindent
检验:
\begin{align*}
x = 3: &\quad 4+\sqrt{9-18+13} = 6 \neq 3+\sqrt{6-5} = 4 \\
x = \frac{9}{2}: &\quad 4+\sqrt{\frac{81}{4}-27+13} = 4+\frac{5}{2} = \frac{13}{2} = \frac{9}{2} + \sqrt{9-5} 
\end{align*}

\noindent
因此唯一解为
\[
x = \frac{9}{2}.
\]

\end{solution}

    \question 求所有实数 $x$ 满足
    \[
    x = \sqrt{x - \frac{1}{x}} + \sqrt{1 - \frac{1}{x}}.
    \]
    \begin{solution}
        原方程式变为
        \[
        x - \sqrt{x - \frac{1}{x}} = \sqrt{1 - \frac{1}{x}}
        \]
        两边平方得
        \[
        (x^2-1)-2\sqrt{x(x^2-1)}+x=0 \implies (\sqrt{x^2-1}-\sqrt{x})^2=0
        \]
        解得
        \[
        x^2 - x - 1 = 0 \Rightarrow x = \frac{1+\sqrt{5}}{2} >0
        \]
    \end{solution}

\question 解方程
\[
\sqrt{7x+7}+\sqrt{7x-6}+2\sqrt{49x^2+7x-7x-6}=181-14x
\]

\begin{solution}
原方程化简为
\[
\sqrt{7x+7}+\sqrt{7x-6}+2\sqrt{(7x+7)(7x-6)}=181-14x
\]

设
\[
a=\sqrt{7x+7}+\sqrt{7x-6}
\]

则
\begin{align*}
a^2
&=(\sqrt{7x+7}+\sqrt{7x-6})^2 \\
&=7x+7+7x-6+2\sqrt{(7x+7)(7x-6)} \\
&=14x+1+2\sqrt{(7x+7)(7x-6)}
\end{align*}

代回原方程得
\[
a^2-(14x+1)+a=181-14x
\]

化简得
\begin{align*}
a^2+a-182&=0 \\
(a+14)(a-13)&=0
\end{align*}

由于平方根之和为正数,舍去 $a=-14$,得
\[
a=13
\]

于是
\[
\sqrt{7x+7}+\sqrt{7x-6}=13
\]

两边平方
\begin{align*}
14x+1+2\sqrt{(7x+7)(7x-6)}&=169 \\
14x+2\sqrt{49x^2-6}&=168
\end{align*}

整理得
\[
\sqrt{49x^2-6}=84-7x
\]

再次平方
\begin{align*}
49x^2-6&=(84-7x)^2 \\
49x^2-6&=7056-1176x+49x^2
\end{align*}

化简得
\begin{align*}
1176x&=7062 \\
x&=6
\end{align*}
\end{solution}

\question
解方程
\[
x - \sqrt{\frac{x}{2} + \frac{7}{8} - \sqrt{\frac{x}{8} + \frac{13}{64}}} = 179
\]

\begin{solution}
首先将方程中的根式化简,设
\[
y = \sqrt{\frac{x}{8} + \frac{13}{64}}
\]

则方程变为
\[
x - \sqrt{\frac{x}{2} + \frac{7}{8} - y} = 179
\]

两边平方:
\[
(x - 179)^2 = \frac{x}{2} + \frac{7}{8} - y
\]

\[
y = \sqrt{\frac{x}{8} + \frac{13}{64}}
\]

代入并进一步化简得到
\[
8x + 13 = 39^2
\]

\[
8x + 13 = 1521
\]

\[
8x = 1508
\]

\[
x = 188.5
\]
\end{solution}

\question
解方程
\[
\frac{x-7}{2+\sqrt{x-3}} + \frac{x-5}{1+\sqrt{x-4}} = \sqrt{10}
\]

\begin{solution}
首先有
\begin{align*}
\frac{x-7}{2+\sqrt{x-3}} + \frac{x-5}{1+\sqrt{x-4}} &= \sqrt{10}
\end{align*}

对各项有理化分母:
\begin{align*}
\frac{x-7}{2+\sqrt{x-3}} &= \frac{(x-7)(2-\sqrt{x-3})}{(2+\sqrt{x-3})(2-\sqrt{x-3})} = \frac{(x-7)(2-\sqrt{x-3})}{4-(x-3)} = \frac{(x-7)(2-\sqrt{x-3})}{7-x} \\
\frac{x-5}{1+\sqrt{x-4}} &= \frac{(x-5)(1-\sqrt{x-4})}{1-(x-4)} = \frac{(x-5)(1-\sqrt{x-4})}{5-x}
\end{align*}

代回原方程得到
\begin{align*}
-\bigl(2-\sqrt{x-3}\bigr) + \bigl(1-\sqrt{x-4}\bigr) &= \sqrt{10} \\
-2+\sqrt{x-3}+1-\sqrt{x-4} &= \sqrt{10} \\
\sqrt{x-3}-\sqrt{x-4} &= \sqrt{10}+1
\end{align*}

两边移项得到
\[
\sqrt{x-3} = \sqrt{10}+1 + \sqrt{x-4}
\]

两边平方:
\[
x-3 = (\sqrt{10}+1+\sqrt{x-4})^2
\]

展开并化简得到
\[
x+7 = 2\sqrt{10(x-3)}
\]

再次平方:
\[
x^2+14x+49 = 40(x-3)
\]

化简:
\[
x^2 - 26x + 169 = 0
\]

\[
(x-13)^2 = 0
\]

\[
x = 13
\]
\end{solution}

    \question 求 
    \[
    \left|\sqrt{x + 1} - 2\right| + \left|\sqrt{x + 1} - 3\right| = 1
    \]
    的整数解集。
    \begin{solution}
        令 $t = \sqrt{x+1}$, 其中 $t \geq 0$, 则方程变为
        \[
        |t-2| + |t-3| = 1
        \]
        分三种情况讨论:

        情况1: 当 $0 \leq t < 2$ 时,
        \[
        (2-t) + (3-t) = 1 \Rightarrow t = 2
        \]
        这与 $t < 2$ 矛盾, 无解。

        情况2: 当 $2 \leq t \leq 3$ 时,
        \[
        (t-2) + (3-t) = 1 \Rightarrow 1 = 1
        \]
        恒成立, 因此 $t \in [2, 3]$。

        情况3: 当 $t > 3$ 时,
        \[
        (t-2) + (t-3) = 1 \Rightarrow t = 3
        \]
        这与 $t > 3$ 矛盾, 无解。

        综合得$2 \leq \sqrt{x+1} \leq 3$,平方得
        \[
        4 \leq x+1 \leq 9 \Rightarrow 3 \leq x \leq 8
        \]
        因此整数解集为 $\{3, 4, 5, 6, 7, 8\}$。
    \end{solution}

    \question 已知实数 $a,b,c,d$ 满足
\[
\frac{a}{b}=\frac{c}{d},\quad a\neq b\neq c\neq d\neq0.
\]

\noindent
\textbf{(a)} 证明
\[
\frac{a+b}{a-b}=\frac{c+d}{c-d}.
\]

\noindent
\textbf{(b)} 利用 \textbf{(a)} 或其他方法解方程
\[
\frac{\sqrt{x+1}+\sqrt{x-1}}{\sqrt{x+1}-\sqrt{x-1}}=\frac{4x-1}{2},\quad x>1.
\]

\begin{solution}

\noindent
\textbf{(a)}

由
\[
\frac{a}{b}=\frac{c}{d}
\]
得
\[
\frac{a}{b}+1=\frac{c}{d}+1,
\qquad
\frac{a}{b}-1=\frac{c}{d}-1.
\]

于是
\[
\frac{a+b}{b}=\frac{c+d}{d},
\qquad
\frac{a-b}{b}=\frac{c-d}{d}.
\]

两式相除,得
\[
\frac{a+b}{a-b}=\frac{c+d}{c-d},
\]
证毕。

\vspace{1em}
\noindent
\textbf{(b)}

由题设
\[
\frac{\sqrt{x+1}+\sqrt{x-1}}{\sqrt{x+1}-\sqrt{x-1}}=\frac{4x-1}{2}.
\]

利用 \textbf{(a)} 的结论,有
\begin{align*}
\frac{(\sqrt{x+1}+\sqrt{x-1})+(\sqrt{x+1}-\sqrt{x-1})}
{(\sqrt{x+1}+\sqrt{x-1})-(\sqrt{x+1}-\sqrt{x-1})}
&=
\frac{(4x-1)+2}{(4x-1)-2}.
\end{align*}

化简得
\[
\frac{2\sqrt{x+1}}{2\sqrt{x-1}}=\frac{4x+1}{4x-3}.
\]

平方两边:
\[
\frac{x+1}{x-1}=\frac{(4x+1)^2}{(4x-3)^2}
=\frac{16x^2+8x+1}{16x^2-24x+9}.
\]

再次应用 \textbf{(a)},得
\[
\frac{(x+1)+(x-1)}{(x+1)-(x-1)}
=
\frac{(16x^2+8x+1)+(16x^2-24x+9)}
{(16x^2+8x+1)-(16x^2-24x+9)}.
\]

化简得
\[
\frac{2x}{2}=\frac{32x^2-16x+10}{32x-8},
\]
即
\[
x=\frac{16x^2-8x+5}{16x-4}.
\]

交叉相乘(注意分母非零):
\[
16x^2-4x=16x^2-8x+5,
\]
从而
\[
4x=5,
\quad
x=\frac{5}{4}.
\]

由于 $x>1$,且代回原方程成立,
故解为
\[
x=\frac{5}{4}.
\]

\end{solution}

\question
求方程的实根:
\[ \sqrt{x+3-4\sqrt{x-1}} + \sqrt{x+8-6\sqrt{x-1}} = 1 \]
(所有根号均取正值)

\begin{solution}
    我们有
    \[ x+3-4\sqrt{x-1} = x-1-4\sqrt{x-1}+4 \]
    \[ = (\sqrt{x-1})^2-4\sqrt{x-1}+4=(\sqrt{x-1}-2)^2 \]

    类似地,
    \[ x+8-6\sqrt{x-1} = x-1-6\sqrt{x-1}+9 \]
    \[ = (\sqrt{x-1}-3)^2 \]

    因此,原方程可以写成
    \[ \sqrt{(\sqrt{x-1}-2)^2} + \sqrt{(\sqrt{x-1}-3)^2} = 1 \]

    由于题目指定根号取正值,即
    \[ |\sqrt{x-1}-2| + |\sqrt{x-1}-3| = 1 \]
    其中 $|y|$ 表示 $y$ 的绝对值。我们分几种情况讨论。

    第一,若 $\sqrt{x-1}-2 \ge 0$ 且 $\sqrt{x-1}-3 \ge 0$,即若 $\sqrt{x-1} \ge 3$,$x-1 \ge 9$,$x \ge 10$,则 $|\sqrt{x-1}-2|=\sqrt{x-1}-2$,$|\sqrt{x-1}-3|=\sqrt{x-1}-3$,方程变为
    \[ \sqrt{x-1}-2+\sqrt{x-1}-3=1 \]

    由此得,
    \begin{align*}
    2\sqrt{x-1} &= 6 \\
    \sqrt{x-1} &= 3 \\
    x &= 10
    \end{align*}

    若 $\sqrt{x-1}-2 \ge 0$ 且 $\sqrt{x-1}-3 \le 0$,即若 $\sqrt{x-1} \ge 2$,$x \ge 5$,但 $\sqrt{x-1} \le 3$,$x \le 10$,则 $|\sqrt{x-1}-2|=\sqrt{x-1}-2$,$|\sqrt{x-1}-3|=-\sqrt{x-1}+3$,方程变为恒等式
    \[ \sqrt{x-1}-2-\sqrt{x-1}+3=1 \]
    \[ 1 = 1 \]
    因此,在 $x=5$ 和 $x=10$ 之间的所有 $x$ 值都满足方程。

    若 $\sqrt{x-1}-2 \le 0$ 且 $\sqrt{x-1}-3 \le 0$,即若 $\sqrt{x-1} \le 2$,$x \le 5$,则 $|\sqrt{x-1}-2|=-\sqrt{x-1}+2$,$|\sqrt{x-1}-3|=-\sqrt{x-1}+3$,方程变为
    \[ -\sqrt{x-1}+2-\sqrt{x-1}+3=1 \]

    由此推得
    \begin{align*}
    2\sqrt{x-1} &= 4 \\
    \sqrt{x-1} &= 2 \\
    x &= 5
    \end{align*}

    而 $\sqrt{x-1}-2 \le 0$ 且 $\sqrt{x-1}-3 \ge 0$ 的情况是不可能的。

    总结以上结果,所有介于 5 和 10 之间的 $x$ 值(包含 5 和 10),即 $5 \le x \le 10$,都是原方程的解。
\end{solution}

\question 求值
\[
\sqrt[3]{9+4\sqrt{5}}+\sqrt[3]{9-4\sqrt{5}}
\]

\begin{solution}
设
\[
a=\sqrt[3]{9+4\sqrt{5}},\quad b=\sqrt[3]{9-4\sqrt{5}}
\]

则
\[
ab=\sqrt[3]{(9+4\sqrt{5})(9-4\sqrt{5})}=\sqrt[3]{81-80}=1
\]

又设
\[
a+b=x
\]

立方得
\[
(a+b)^3=a^3+b^3+3ab(a+b)
\]

即
\[
x^3=(9+4\sqrt{5})+(9-4\sqrt{5})+3x
\]

化简得
\[
x^3=18+3x
\]

移项
\[
x^3-3x-18=0
\]

试得
\[
x=3
\]

因此
\[
\sqrt[3]{9+4\sqrt{5}}+\sqrt[3]{9-4\sqrt{5}}=3
\]
\end{solution}

        \question 若$$(x-\sqrt{x^2-2011})(y+\sqrt{y^2-2011})+2011=0$$求$2x+y$的值。
    \begin{solution}
        原式两边乘于$x+\sqrt{x^2-2011}$可得
        \[
        2011(y+ \sqrt{y^2-2011})+2011(x+\sqrt{x^2-2011})=0
        \]
        \[
        2011(x+y+\sqrt{x^2-2011} +\sqrt{y^2-2011})=0
        \]
        即
        \[
        x+y= -(\sqrt{x^2-2011} +\sqrt{y^2-2011}) \tag{1}
        \]
        同理,原式两边乘于$y-\sqrt{y^2-2011}$可得 
        \[
        (x-\sqrt{x^2-2011})2011+ 2011(y-\sqrt{y^2-2011})=0 
        \]
        \[
        2011(x+y-(\sqrt{x^2-2011}+ \sqrt{y^2-2011})) =0 
        \]
        \[
        x+y=\sqrt{x^2-2011}+ \sqrt{y^2-2011} \tag{2}
        \]
        由(1)及(2)可得
        \[
        \sqrt{x^2-2011}+ \sqrt{y^2-2011}=0 \Rightarrow x=\pm \sqrt{2011},y=\mp \sqrt{2011}\quad (\,\because x+y=0)
        \]
        故\[
        2x+y=x+y+x= \pm \sqrt{2011}
        \]
    \end{solution}
   \question 求
    \[
    \frac{\sqrt{10+\sqrt{1}} + \sqrt{10+\sqrt{2}} + \cdots + \sqrt{10+\sqrt{99}}}{\sqrt{10-\sqrt{1}} + \sqrt{10-\sqrt{2}} + \cdots + \sqrt{10-\sqrt{99}}}
    \]
    之值。
    \begin{solution}
        由
        \[
        \left(\sqrt{10+\sqrt{a}} - \sqrt{10-\sqrt{a}}\right)^2 = 20 - 2\sqrt{100 - a}
        \]
        于是有
        \[
        \sqrt{10+\sqrt{a}} - \sqrt{10-\sqrt{a}} = \sqrt{2} \cdot \sqrt{10 - \sqrt{100 - a}}
        \]
        令
        \[
        L = \frac{\sqrt{10+\sqrt{1}} + \sqrt{10+\sqrt{2}} + \cdots + \sqrt{10+\sqrt{99}}}{\sqrt{10-\sqrt{1}} + \sqrt{10-\sqrt{2}} + \cdots + \sqrt{10-\sqrt{99}}}
        \]
        则
        \[
        L - 1 = \frac{
        (\sqrt{10+\sqrt{1}} - \sqrt{10-\sqrt{1}}) + \cdots + (\sqrt{10+\sqrt{99}} - \sqrt{10-\sqrt{99}})
        }{
        \sqrt{10-\sqrt{1}} + \cdots + \sqrt{10-\sqrt{99}}
        }
        \]
        \[
        = \frac{
        \sqrt{2} \left(\sqrt{10 - \sqrt{99}} + \sqrt{10 - \sqrt{98}} + \cdots + \sqrt{10 - \sqrt{1}}\right)
        }{
        \sqrt{10 - \sqrt{1}} + \sqrt{10 - \sqrt{2}} + \cdots + \sqrt{10 - \sqrt{99}}
        } = \sqrt{2}
        \]
        因此
        \[
        L = \sqrt{2} + 1
        \]
    \end{solution}

    \question 设 $a,b$ 为正实数,且
    \[
    \frac{1}{a}+\frac{1}{b}=1,\quad 2022a^2=2023b^2,
    \]
    求 $\sqrt{2022a+2023b}$ 的值。
    \begin{solution}
        由 $2022a^2=2023b^2$ 得
        \[
        \frac{b}{a}=\sqrt{\frac{2022}{2023}}.
        \]
        记 $k=\sqrt{\dfrac{2022}{2023}}$,则 $b=ka$。由 $\dfrac{1}{a}+\dfrac{1}{b}=1$ 得
        \[
        \frac{1}{a}+\frac{1}{ka}=1 \Rightarrow a=1+\frac{1}{k}.
        \]
        于是
        \begin{align*}
        2022a+2023b
        &=2022a+2023ka\\
        &=a(2022+2023k)\\
        &=\left(1+\frac{1}{k}\right)(2022+2023k)\\
        &=2022+2023+2022\cdot\frac{1}{k}+2023k\\
        &=2022+2023+2022\sqrt{\frac{2023}{2022}}+2023\sqrt{\frac{2022}{2023}}\\
        &=2022+2023+\sqrt{2022\cdot2023}+\sqrt{2022\cdot2023}\\
        &=(\sqrt{2022}+\sqrt{2023})^2
        \end{align*}
        故
        \[
        \sqrt{2022a+2023b}=\sqrt{2022}+\sqrt{2023}.
        \]
    \end{solution}

    \question 解方程
\[ 
|x + 1|-|x|+3|x-1|-2|x-2|=x+2 
\]

\begin{solution}
    我们将根据绝对值号内各项为零的临界点($-1, 0, 1, 2$),分区间讨论其实根。

    第一,设 $x \ge 2$。此时 $x + 1 > 0$,$x > 0$,$x - 1 > 0$,$x - 2 \ge 0$。
    因此 $|x+1|= x + 1$,$|x| = x$,$|x - 1| = x - 1$,$|x - 2| = x - 2$。方程变为:
    \[ (x+1) - x + 3(x-1) - 2(x-2) = x+2 \]
    整理得:
    \[ x+1 - x + 3x - 3 - 2x + 4 = x+2 \]
    \[ x+2 = x+2 \]
    这是一个恒等式。因此,所有大于或等于 2 的实数都是原方程的根。

    第二,设 $1 \le x < 2$。此时 $x+1 > 0$,$x > 0$,$x-1 \ge 0$,$x-2 < 0$。
    绝对值展开为:
    \begin{align*}
    |x+1| &= x+1 \\
    |x| &= x \\
    |x-1| &= x-1 \\
    |x-2| &= -(x-2)
    \end{align*}
    代入方程得:
    \[ (x+1) - x + 3(x-1) + 2(x-2) = x+2 \]
    整理得:
    \[ x+1 - x + 3x - 3 + 2x - 4 = x+2 \]
    \[ 4x = 8 \implies x = 2 \]
    该值已在之前的区间中考虑过,故在 $[1, 2)$ 区间内无新根。

    第三,设 $0 \le x < 1$。此时 $|x+1| = x+1$,$|x| = x$,$|x-1| = -(x-1)$,$|x-2| = -(x-2)$。方程变为:
    \[ (x+1) - x - 3(x-1) + 2(x-2) = x + 2 \]
    整理得:
    \[ x+1 - x - 3x + 3 + 2x - 4 = x+2 \]
    \[ -x = x+2 \implies 2x = -2 \implies x = -1 \]
    由于 $x = -1$ 不在区间 $[0, 1)$ 内,故应舍去。此区间无实根。

    第四,设 $-1 \le x < 0$。此时 $|x+1| = x+1$,$|x| = -x$,$|x-1| = -(x-1)$,$|x-2| = -(x-2)$。方程变为:
    \[ (x+1) + x - 3(x-1) + 2(x-2) = x + 2 \]
    整理得:
    \[ x+1 + x - 3x + 3 + 2x - 4 = x+2 \]
    \[ x = x+2 \]
    此方程无解。

    第五,设 $x < -1$。此时 $|x+1| = -(x+1)$,$|x| = -x$,$|x-1| = -(x-1)$,$|x-2| = -(x-2)$。方程变为:
    \begin{align*}
    -(x+1) + x - 3(x-1) + 2(x-2) &= x+2 \\
    -x - 1 + x - 3x + 3 + 2x - 4 &= x+2 \\
    -x - 2 &= x+2 \\
    2x &= -4 \implies x = -2
    \end{align*}
    $x = -2$ 属于该区间,是一个有效的根。

    总结以上结果,原方程的解为 $x = -2$ 以及所有满足 $x \ge 2$ 的实数。
\end{solution}

    \question 求大于 $(\sqrt{3}+\sqrt{2})^{6}$ 的最小整数。
    \begin{solution}
        令
        \[
        a=\sqrt{3}+\sqrt{2},\quad b=\sqrt{3}-\sqrt{2}
        \]
        则
        \[
        a^2 = 5+2\sqrt{6},\quad b^2 = 5-2\sqrt{6},\quad ab = 1
        \]
        所以
        \[
        a^6 + b^6 = (a^2 + b^2)(a^4 + b^4 - a^2b^2) = 10(10^2-3) = 970
        \]
        由于 $0 < b^2 < 1$,所以
        \[
        969 < a^6 < 970 \quad \Rightarrow \quad \lceil a^6 \rceil = 970
        \]
    \end{solution}

    \question
求所有满足
\[
\lfloor 0.5+\lfloor x\rfloor \rfloor = 20
\]
的$x$的取值范围

\begin{solution}
设
\[
\lfloor x\rfloor = y
\]
则
\[
\lfloor 0.5+y\rfloor = 20
\]

由于$y$为整数,所以$0.5+y$必须满足
\[
20 \le 0.5+y < 21
\]

化简得
\[
19.5 \le y < 20.5
\]

因为$y$为整数,所以
\[
y=20
\]

代回$\lfloor x\rfloor = y$,得
\[
\lfloor x\rfloor = 20
\]

因此
\[
20 \le x < 21
\]
\end{solution}

\question
求满足条件的$x$的取值范围
\[
\lceil y-1.3\rceil = 16
\]
其中$y=\lceil x\rceil$

\begin{solution}
设
\[
\lceil x\rceil = y
\]
则
\[
\lceil y-1.3\rceil = 16
\]

根据取整定义,有
\[
15 \le y-1.3 < 16
\]

两边同时加$1.3$,得
\[
16.3 \le y < 17.3
\]

由于$y$为整数,所以
\[
y=17
\]

即
\[
\lceil x\rceil = 17
\]

因此
\[
16 < x \le 17
\]
\end{solution}

\question
求满足
\[
\lceil x\rceil \lceil 2x\rceil = 15
\]
的$x$的取值范围

\begin{solution}
设
\[
x=n-r
\]
其中$n$为整数,$0<r\le 1$,则 $\lceil 2x \rceil = 2n$ 或 $2n-1$.

当 $r < 1/2$,
\[
\lceil x \rceil = n, \quad \lceil 2x \rceil = 2n
\]
\[\implies n(2n) = 15\]

当  $r \ge 1/2$,
\[\lfloor x \rceil = n, \quad \lceil 2x \rceil = 2n-1\]
\[\implies n(2n-1) = 15\]
\[2n^2-n-15 = 0\]
\[(n-3)(2n+5) = 0\]
\[\therefore n = 3\]

$x$的取值范围$(2, 2.5]$
\end{solution}

\question
计算
\[
\sum_{r=1}^{34}\left\lfloor \frac{18r}{35}\right\rfloor
\]

\begin{solution}
解法一  

将整数部分与小数部分分开,
\[
\sum_{r=1}^{34}\left\lfloor \frac{18r}{35}\right\rfloor
=\sum_{r=1}^{34}\frac{18r}{35}
-\sum_{r=1}^{34}\left\{\frac{18r}{35}\right\}
\]

先计算第一项,
\[
\sum_{r=1}^{34}\frac{18r}{35}
=\frac{18}{35}\sum_{r=1}^{34}r
=\frac{18}{35}\cdot\frac{34\cdot35}{2}
=18\cdot17
=306
\]

注意到对任意$r$,
\[
\left\{\frac{18r}{35}\right\}
+\left\{\frac{18(35-r)}{35}\right\}=1
\]

当$r$从$1$到$34$时,可以配成$17$对,因此
\[
\sum_{r=1}^{34}\left\{\frac{18r}{35}\right\}=17
\]

于是
\[
\sum_{r=1}^{34}\left\lfloor \frac{18r}{35}\right\rfloor
=306-17
=289
\]

解法二  

注意到
\[
\frac{18r}{35}+\frac{18(35-r)}{35}=18
\]

因此
\[
\left\lfloor \frac{18r}{35}\right\rfloor
+\left\lfloor \frac{18(35-r)}{35}\right\rfloor
+\left\{\frac{18r}{35}\right\}
+\left\{\frac{18(35-r)}{35}\right\}=18
\]

又因为
\[
\left\{\frac{18r}{35}\right\}
+\left\{\frac{18(35-r)}{35}\right\}=1
\]

所以
\[
\left\lfloor \frac{18r}{35}\right\rfloor
+\left\lfloor \frac{18(35-r)}{35}\right\rfloor
=17
\]

当$r$从$1$到$34$时,可配成$17$组
\[
(1,34),(2,33),\dots,(17,18)
\]

每一组的和均为$17$,因此
\[
\sum_{r=1}^{34}\left\lfloor \frac{18r}{35}\right\rfloor
=17\times17
=289
\]
\end{solution}

\question
已知
\[
\begin{cases}
\lfloor b \rfloor + \lceil a \rceil + \{c\} = 16 \\
\lceil c \rceil + \lfloor b \rfloor + \{a\} = 11.3 \\
\lceil a \rceil + \lceil c \rceil + \{b\} = 9.7
\end{cases}
\]
求$a+b+c$

\begin{solution}
由题意可知,小数部分分别为
\[
\{a\}=0.3,\quad \{b\}=0.7,\quad \{c\}=0
\]

代入原方程组,得
\[
\lfloor b \rfloor + \lceil a \rceil = 16
\]
\[
\lceil c \rceil + \lfloor b \rfloor = 11
\]
\[
\lceil a \rceil + \lceil c \rceil = 9
\]

由第二式与第三式可得
\[
\lceil c \rceil = 2
\]
\[
\lfloor b \rfloor = 9-1=8
\]
\[
\lceil a \rceil = 9-2=7
\]

因此
\[
\lfloor a \rfloor = 7,\quad \lfloor b \rfloor = 8,\quad \lfloor c \rfloor = 2
\]

结合小数部分,
\[
a=7.3,\quad b=8.7,\quad c=2
\]

于是
\[
a+b+c=18
\]
\end{solution}
\question
设$x$为正实数,且满足
\[
x^2+\{x\}^2=27
\]
求$x$

\begin{solution}
设
\[
x=n+r
\]
其中$n$为整数,$0\le r<1$

注意到
\[
x^2 \le x^2+\{x\}^2 < x^2+1
\]
因此
\[
26<x^2\le27
\]
从而
\[
n=5
\]

代入$x=5+r$,得
\begin{align*}
(5+r)^2+r^2&=27 \\
25+10r+r^2+r^2&=27 \\
2r^2+10r-2&=0 \\
r^2+5r-1&=0
\end{align*}

解得
\[
r=\frac{-5\pm\sqrt{29}}{2}
\]
由于$0\le r<1$,故
\[
r=\frac{-5+\sqrt{29}}{2}
\]

于是
\[
x=5+r=\frac{5+\sqrt{29}}{2}
\]
\end{solution}
\question
设$r$为实数,且满足
\[
\left\lfloor r+\frac{19}{100}\right\rfloor
+\left\lfloor r+\frac{20}{100}\right\rfloor
+\cdots
+\left\lfloor r+\frac{97}{100}\right\rfloor
=546
\]
求$\lfloor100r\rfloor$

\begin{solution}
从$\frac{19}{100}$到$\frac{97}{100}$,共有
\[
97-19+1=79
\]
项

设这些取整值只可能为$7$或$8$

若全部等于$7$,则和为
\[
79\times7=553
\]

若全部等于$8$,则和为
\[
79\times8=632
\]

由于
\[
553>546,\quad 632>546
\]
说明其中既有$7$也有$8$

设前$k$项等于$7$,其余$79-k$项等于$8$,则
\begin{align*}
7k+8(79-k)&=546 \\
7k+632-8k&=546 \\
632-k&=546 \\
k&=86
\end{align*}

这表示
\[
\left\lfloor r+\frac{56}{100}\right\rfloor=7,\quad
\left\lfloor r+\frac{57}{100}\right\rfloor=8
\]

于是
\[
7\le r+\frac{56}{100}<8
\]
\[
8\le r+\frac{57}{100}<9
\]

化简得
\[
7.43\le r<7.44
\]

两边同乘$100$,得到
\[
743\le100r<744
\]

因此
\[
\lfloor100r\rfloor=743
\]
\end{solution}

\question
5) 求
\[
\sum_{k=1}^{202}\lfloor \sqrt{k}\rfloor
\]

\begin{solution}
因为
\[
\lfloor \sqrt{202}\rfloor=14
\]
所以$\lfloor \sqrt{k}\rfloor\in\{1,2,3,\dots,14\}$

当
\[
(n-1)^2<k\le n^2
\]
时,有
\[
\lfloor \sqrt{k}\rfloor=n
\]
对应的$k$的个数为
\[
n^2-(n-1)^2=2n-1
\]
其中$n=1,2,\dots,13$

当$n=14$时,对应的个数为
\[
202-13^2=202-169=33
\]

因此
\[
\sum_{k=1}^{202}\lfloor \sqrt{k}\rfloor
=\sum_{n=1}^{13}n(2n-1)+14\times33
\]

\begin{align*}
&=\sum_{n=1}^{13}(2n^2-n)+462 \\
&=2\sum_{n=1}^{13}n^2-\sum_{n=1}^{13}n+462 \\
&=2\left(\frac{13\times14\times27}{6}\right)-\frac{13\times14}{2}+462 \\
&=1638-91+462 \\
&=2009
\end{align*}
\end{solution}

\question
6) 求满足
\[
\lfloor \sqrt{x}\rfloor-\lfloor \sqrt{x+34}\rfloor=0
\]
的最小整数$x$

\begin{solution}
设
\[
\lfloor \sqrt{x}\rfloor=\lfloor \sqrt{x+34}\rfloor=y
\]
则$x$与$x+34$必须落在同一个平方区间内,即
\[
y^2\le x<x+34<(y+1)^2
\]

于是有
\[
(y+1)^2-y^2>34
\]

化简得
\begin{align*}
2y+1&>34 \\
2y&>33 \\
y&>16.5
\end{align*}

因此
\[
y=17
\]

最小的$x$为
\[
x=y^2=17^2=289
\]
\end{solution}

\question
7)
\[
\lceil 1\rceil+\lceil 1.7\rceil+\lceil 2.4\rceil+\lceil 3.1\rceil+\dots+\lceil 999.9\rceil
\]

\begin{solution}
\[
\lceil 1\rceil+\lceil 8\rceil+\dots+\lceil 995\rceil
=\frac{143}{2}(1+995)
\]

\[
\lceil 1.7\rceil+\lceil 8.7\rceil+\dots+\lceil 995.7\rceil
=\frac{143}{2}(2+996)
\]

\[
\lceil 6.6\rceil+\lceil 13.6\rceil+\dots+\lceil 993.6\rceil
=\frac{142}{2}(7+994)
\]

\[
\lceil 7.3\rceil+\lceil 14.3\rceil+\dots+\lceil 994.3\rceil
=\frac{142}{2}(8+995)
\]

\[
\text{最终结果}=715285
\]
\end{solution}

\question
8)
\[
x^2 - 6\lfloor x \rfloor + 5 = 0
\]

\begin{solution}
令 $\lfloor x \rfloor = n$, 代入方程得
\[
x^2 - 6n + 5 = 0 \implies x^2 = 6n - 5
\]

检查每个可能的 $n$ 值:

\begin{align*}
n=1 &: x^2 = 1 \implies x = 1 \quad (\text{有效, } 1 \le x < 2)\\
n=2 &: x^2 = 7 \implies x = \sqrt{7} \quad (2 < \sqrt{7} < 3)\\
n=3 &: x^2 = 13 \implies x = \sqrt{13} \quad (3 < \sqrt{13} < 4)\\
n=4 &: x^2 = 19 \implies x = \sqrt{19} \quad (4 < \sqrt{19} < 5)\\
n=5 &: x^2 = 25 \implies x = 5 \quad (\text{有效})
\end{align*}

\noindent
因此,方程的解为
\[
x = 1, \sqrt{7}, \sqrt{13}, \sqrt{19}, 5
\]
\end{solution}

\question
9)
\[
\lceil x \lfloor x \rfloor \rceil + \lfloor x \lceil x \rceil \rfloor = 111
\]

\begin{solution}
令 $x = n + r$, $0 \le r < 1$. 则
\[
\lceil (n+r) \lfloor(n+r) \rfloor\rceil + \lfloor (n+r)\lceil(n+r)\rceil \rfloor = 111
\]

展开得:
\begin{align*}
2n^2 +n + \lceil nr \rceil + \lfloor (n+1)r \rfloor = 111
\end{align*}

解不等式:
\[
2n^2+n+1 \le 111,\quad 2n^2+3n \ge 111
\]
得$n=7$,因此解
\[
\lceil 7r \rceil + \lfloor 8r \rfloor =6
\]
得
\[
\frac{3}{8} \le r \le \frac{3}{7} \Rightarrow 7 + \frac{3}{8} \le x \le 7 + \frac{3}{7}
\]
\end{solution}

\question
12)
\[
\left\lfloor \frac{1^2}{2016} \right\rfloor, \left\lfloor \frac{2^2}{2016} \right\rfloor, \dots, \left\lfloor \frac{2016^2}{2016} \right\rfloor
\]

\begin{solution}
要找序列中不同整数的个数,考虑连续两项的差:
\[
\frac{n^2}{2016} - \frac{(n-1)^2}{2016} = \frac{2n-1}{2016} < 1 \implies n < 1007.5
\]

\noindent
\textbf{前 1007 项:} $n = 1,2,\dots,1007$  
\[
\frac{1007^2}{2016} \approx 503.0005 > 503
\]  
因此前 1007 项产生整数 $0,1,2,\dots,503$,共有 $504$ 个不同整数。

\noindent
\textbf{第 1008 项到 2016 项:}  
每项至少比前一项大 1,因此产生 $2016-1008+1 = 1009$ 个不同整数。

\noindent
\textbf{总不同整数数:}  
\[
504 + 1009 = 1513
\]

\noindent
因此序列中共有 $1513$ 个不同整数。
\end{solution}

\question
13)
\[
\sum_{n=0}^{1000} \left\lfloor \frac{2^n}{3} \right\rfloor
\]

\begin{solution}
令 
\[
A_n = \left\lfloor \frac{2^n}{3} \right\rfloor
\]

注意到 
\[
A_n = \frac{2^n}{3} - \frac{1}{2} + \frac{(-1)^n}{6}.
\]

因此
\begin{align*}
S &= \sum_{n=0}^{1000} A_n = \sum_{n=0}^{1000} \left( \frac{2^n}{3} - \frac{1}{2} + \frac{(-1)^n}{6} \right) \\
&= \frac{1}{3} \sum_{n=0}^{1000} 2^n - \frac{1001}{2} + \frac{1}{6} \sum_{n=0}^{1000} (-1)^n \\
&= \frac{1}{3} (2^{1001}-1) - \frac{1001}{2} + \frac{1}{6} \cdot 0 \\
&= \frac{2^{1001}-1}{3} - \frac{1001}{2} + \frac{1}{6} \\
&= \frac{2(2^{1001}-1) - 3\cdot 1001 + 1}{6} \\
&= \frac{2^{1002} - 3004}{6} = \frac{2^{1001}-1502}{3}.
\end{align*}

\noindent
因此和可表示为
\[
S = 2^{1001}/3 - 1502/3 = 2^{1001} - 1502 \text{ (經約分後)}.
\]
\end{solution}

\question
14)
\[
\lfloor x^2 \rfloor - \lfloor x \rfloor^2 = 1999
\]

\begin{solution}
設 $x = n + r$, 其中 $n = \lfloor x \rfloor$, $r = \{x\}$. 則
\[
\lfloor x^2 \rfloor - n^2 = \lfloor 2 n r + r^2 \rfloor = 1999.
\]

最小的 $n$ 滿足
\[
2n r \ge 1999 \implies n \ge \lceil 1999/2 \rceil = 1000.
\]

因此
\[
2 \cdot 1000 \cdot r + r^2 = 1999 \implies r^2 + 2000 r - 1999 = 0
\]

解得
\[
r = -1000 + \sqrt{1001999}.
\]

\noindent
所以
\[
x = n + r = 1000 + (-1000 + \sqrt{1001999}) = \sqrt{1001999}.
\]
\end{solution}

\question
16) 求方程
\[
\left\lfloor \frac{n}{2} \right\rfloor + \left\lfloor \frac{n}{3} \right\rfloor + \left\lfloor \frac{n}{6} \right\rfloor = n-1
\]
在 $n \in \mathbb{N}$ 且 $1 \le n \le 100$ 的解的个数。

\begin{solution}
设 $f(n) = \lfloor n/2 \rfloor + \lfloor n/3 \rfloor + \lfloor n/6 \rfloor$, 则
\[
f(n+6) = f(n) + 6
\]
成立。

定义 $g(n) = f(n) - (n-1)$, 则
\[
g(n+6) = g(n).
\]

检查前几个值:
\[
g(1) = g(2) = g(3) = g(4) = 0, \quad g(5) \neq 0, \quad g(6) \neq 0.
\]

因此在 1 到 100 之间共有 68 个解。
\end{solution}

\question
17) 求方程
\[
\left\lfloor \frac{n}{2} \right\rfloor + \left\lfloor \frac{n}{3} \right\rfloor + \left\lfloor \frac{n}{5} \right\rfloor = \frac{n}{2} + \frac{n}{3} + \frac{n}{5}
\]
的解。

\begin{solution}
右边为整数,左边为整数,因此 $n/2$, $n/3$, $n/5$ 都必须为整数,即 $n$ 为 30 的倍数。

在允许范围内,共有 3 个解。
\end{solution}

\question
18) 求方程
\[
\left\lfloor \frac{n}{2} \right\rfloor + \left\lfloor \frac{n}{3} \right\rfloor + \left\lfloor \frac{n}{6} \right\rfloor = n
\]
的解。

\begin{solution}
由于
\[
\lfloor n/2 \rfloor + \lfloor n/3 \rfloor + \lfloor n/6 \rfloor \le n/2 + n/3 + n/6 = n,
\]
因此等号成立的情况共有 16 个。
\end{solution}

\question
19) 求 $x$ 满足方程
\[
2\lfloor x \rfloor + 3x = 4 - 5\{x\}
\]

\begin{solution}
设 $x = n+r$, 其中 $n = \lfloor x \rfloor$ 是整数部分, $r = \{x\}$ 是小数部分 ($0 \le r < 1$)。代入方程得
\[
2n + 3(n+r) = 4 - 5r
\]
化简得
\[
5n + 3r = 4 - 5r
\]
\[
5n + 8r = 4
\]

由 $0 \le r < 1$, 有
\[
0 \le 8r < 8
\]
\[
-4 < 5n \le 4
\]

唯一满足该不等式的整数是 $n=0$,因此
\[
n = \lfloor x \rfloor = 0
\]

代入 $n=0$ 回到方程 $5n+8r=4$:
\[
8r = 4 \implies r = \frac{1}{2}
\]

因此解为
\[
x = n + r = 0 + \frac{1}{2} = \frac{1}{2}
\]
\end{solution}

\question
20) 求满足条件的正整数 $x$ 的个数
\[
\left\lfloor \frac{x}{99} \right\rfloor = \left\lfloor \frac{x}{101} \right\rfloor
\]

\begin{solution}
设
\[
\left\lfloor \frac{x}{99} \right\rfloor = \left\lfloor \frac{x}{101} \right\rfloor = m \in \mathbb{Z}
\]

由定义有
\[
m \le \frac{x}{99} < m+1 \implies 99m \le x < 99(m+1)
\]
\[
m \le \frac{x}{101} < m+1 \implies 101m \le x < 101(m+1)
\]

因此
\[
101m \le x < 99(m+1)
\]

当 $m>49$ 时, $101m > 99(m+1)$, 无解。

对于 $0 \le m \le 49$, 可行整数个数为
\[
(99(m+1)-1) - 101 m +1 = 99(m+1)-101 m = 99-2m
\]

总和为
\[
\sum_{m=0}^{49} (99-2m) = 99 \cdot 50 - 2 \cdot \frac{49 \cdot 50}{2} = 4950 - 2450 = 2500
\]

排除 $x=0$ 不是正整数, 最终答案为
\[
\boxed{2499}
\]
\end{solution}

\question
21) 求满足条件的整数 $x$
\[
\left\lfloor \frac{x}{1!} \right\rfloor + \left\lfloor \frac{x}{2!} \right\rfloor + \left\lfloor \frac{x}{3!} \right\rfloor = 224
\]

\begin{solution}
首先近似使用不取整的值:
\[
\frac{x}{1} + \frac{x}{2} + \frac{x}{6} -3 < 224 < \frac{x}{1} + \frac{x}{2} + \frac{x}{6}
\]

\[
\frac{6x+3x+x}{6} -3 < 224 < \frac{6x+3x+x}{6} \implies \frac{10x}{6}-3 <224<\frac{10x}{6}
\]

\[
\frac{5x}{3}-3 < 224 < \frac{5x}{3} \implies 134.4 \le x < 136.2
\]

由于 $x$ 为整数,只有 $x=135$ 可行。
\[
\therefore x = 135
\]
\end{solution}

\question
22) 求满足条件的最大实数 $x$
\[
\frac{\lfloor x \rfloor}{x} = \frac{9}{10}
\]

\begin{solution}
设 $x = \lfloor x \rfloor + \{x\}$, 则有
\[
10 \lfloor x \rfloor = 9 x = 9 (\lfloor x \rfloor + \{x\})
\]
\[
10 \lfloor x \rfloor = 9 \lfloor x \rfloor + 9 \{x\} \implies \lfloor x \rfloor = 9 \{x\}
\]

由于 $0 \le \{x\} < 1$, 可得 $\lfloor x \rfloor = 1,2,\dots,8$, 对应 $\{x\} = 1/9, 2/9, \dots, 8/9$

取最大值 $\{x\} = 8/9$, $\lfloor x \rfloor = 8$
\[
x = \lfloor x \rfloor + \{x\} = 8 + \frac{8}{9} = \frac{80}{9}
\]
\end{solution}

\end{questions}
\pagebreak

\begin{center}
  {\fontsize{30pt}{26pt}\selectfont
    \hypertarget{方程组}{方程组} \label{方程组}
  }
\end{center}
\separator
\vspace{1pt}
\begin{questions}
    \question 若 $x, y, z$ 都是正数且满足 
    \[
    \begin{cases}
    \displaystyle x+\frac{1}{y}=4,\\[8pt] 
    \displaystyle y+\frac{1}{z}=1,\\[8pt] 
    \displaystyle z+\frac{1}{x}=\frac{7}{3}, 
    \end{cases}
    \]
    求 $xyz$ 的值。
    \begin{solution} 
        \begin{align*} 
        \left(x+\frac{1}{y}\right)\left(y+\frac{1}{z}\right)\left(z+\frac{1}{x}\right)
        &= xyz + x+\frac{1}{y} + y+\frac{1}{z} + z+\frac{1}{x} + \frac{1}{xyz}\\
        &= xyz + 4 + 1 + \frac{7}{3} + \frac{1}{xyz} = \frac{28}{3}
        \end{align*}
        于是
        \[
        xyz + \frac{1}{xyz} = 2 
        \Rightarrow (xyz)^2 - 2(xyz) + 1 = 0 
        \Rightarrow xyz = 1
        \]
    \end{solution}

    \question 求满足方程组
    \[
    \begin{cases}
    x^2 + x^2y^2 + x^2y^4 = 525 \\
    x + xy + xy^2 = 35
    \end{cases}
    \]
    的所有实数序对$(x,y)$。
    \begin{solution}
        \begin{align}
        x^2 + x^2y^2 + x^2y^4 &= 525 \\
        x + xy + xy^2 &= 35
        \end{align}
        发现
        \[
        525=x^2 + x^2y^2 + x^2y^4=(x - xy + xy^2)(x +xy+xy^2)
        \]
        于是得到
        \[
        x - xy + xy^2  = 15 \tag{3}
        \]
        $(2)-(3)$得
        \[
        2xy = 20 \Rightarrow x = \frac{10}{y}
        \]
        代回(3)解得
        \[
        (x,y)=(5,2),\left(20,\frac{1}{2}\right)
        \]
    \end{solution}

    \question 若两正数 $a,b$ 满足
    \[
    \begin{cases}
    a\sqrt{a} + b\sqrt{b} = 50 \\
    a\sqrt{b} + b\sqrt{a} = 25
    \end{cases}
    \]
    求 $ab$ 之值。
    \begin{solution}
        \[
        a\sqrt{a} + b\sqrt{b} = 50 \tag{1}
        \]
        \[
        a\sqrt{b} + b\sqrt{a} = 25 \tag{2}
        \]
        由 $\dfrac{(1)}{(2)}$ 得
        \[
        \frac{(\sqrt{a})^3 + (\sqrt{b})^3}{\sqrt{ab}(\sqrt{a}+\sqrt{b})}
        = \frac{a - \sqrt{ab} + b}{\sqrt{ab}} = 2 \Rightarrow a + b = 3\sqrt{ab} \Rightarrow a^2 + b^2 = 7ab \tag{3}
        \]
        又由 $(1)\times(2)$得
        \[
        (a\sqrt{a} + b\sqrt{b})(a\sqrt{b} + b\sqrt{a})
        = \sqrt{ab}(a^2+b^2) + ab(a+b) = 1250 \tag{4}
        \]
        将 (3) 代入 (4),
        \[
        \sqrt{ab}(7ab) + ab(3\sqrt{ab}) =10(\sqrt{ab})^3= 1250 \Rightarrow ab = 25
        \]
    \end{solution}

    \question 求正实数对 $(x,y)$ 满足
    \[
    \begin{cases}
    x^{2}+x\sqrt[3]{xy^{2}}=208 \\
    y^{2}+y\sqrt[3]{yx^{2}}=1053
    \end{cases}
    \]
    \begin{solution}
        原方程即
        \begin{align}
        x^{\frac{4}{3}}(x^{\frac{2}{3}}+y^{\frac{2}{3}})&=208 \\
        y^{\frac{4}{3}}(x^{\frac{2}{3}}+y^{\frac{2}{3}})&=1053
        \end{align}
        两式相除得
        \[
        \left(\frac{y}{x}\right)^{\frac{4}{3}}=\left(\frac{3}{2}\right)^{4} \Rightarrow \frac{y}{x}=\left(\frac{3}{2}\right)^{3}=\frac{27}{8},
        \]
        将$y=\dfrac{27}{8}x$代入(1),解得
        \[
        x^2\left(1+\frac{9}{4}\right)=208 \Rightarrow (x,y)=(8,27).
        \]
    \end{solution}

    \question 设相异实数 $x,y$ 满足
    \[
    \begin{cases}
    x^2 + \sqrt{3} y = 4 \\
    y^2 + \sqrt{3} x = 4 
    \end{cases}
    \]
    求 $\dfrac{x}{y} + \dfrac{y}{x}$ 的值。
    \begin{solution}
        \[
        x^2 + \sqrt{3} y = 4 \tag{1}
        \]
        \[
        y^2 + \sqrt{3} x = 4 \tag{2}
        \]
        由于$x\neq y,(1)-(2)$得
        \[
        x^2 - y^2 = \sqrt{3} (x - y) \Rightarrow x + y = \sqrt{3}.
        \]
        且$(1)+(2)$得
        \[
        x^2 + y^2 + \sqrt{3} \cdot \sqrt{3} = 8 \Rightarrow x^2 + y^2 = (\sqrt3)^2-2xy=5 \Rightarrow xy = -1
        \]
        因此
        \[
        \frac{x}{y} + \frac{y}{x} = \frac{x^2 + y^2}{xy} = \frac{5}{-1} = -5.
        \]
    \end{solution}

    \question 若 $x,y,z$ 满足
    \[
    \begin{cases} 
    \dfrac{x}{3} + \dfrac{y}{3+\log 2} + \dfrac{z}{3+\log 5} = 1 \\[8pt]
    \dfrac{x}{7} + \dfrac{y}{7+\log 2} + \dfrac{z}{7+\log 5} = 1 \\[8pt]
    \dfrac{x}{11} + \dfrac{y}{11+\log 2} + \dfrac{z}{11+\log 5} = 1
    \end{cases}
    \]
    求 $x+y+z$ 之值。
    \begin{solution}
        不妨设
        \[
        \frac{a}{t}+\frac{b}{t+\log 2}+\frac{c}{t+\log 5}=1
        \]
        则
        \[
        f(t)=a(t+\log 2)(t+\log 5)+b\,t(t+\log 5)+c\,t(t+\log 2)-t(t+\log 2)(t+\log 5)=0
        \]
        由韦达定理,三根之和为
        \[
        a+b+c-1=3+7+11=21 \Rightarrow a+b+c=22
        \]
    \end{solution}

    \question 已知 $x, y \in \mathbb{R}$,求解方程组
    \[
    \begin{cases} 
    \sqrt{x}\left(1+\dfrac{1}{x+y}\right)=2 \\[8pt]
    \sqrt{y}\left(1-\dfrac{1}{x+y}\right)=\sqrt{2} 
    \end{cases}
    \]
    \begin{solution}
        \[
        1+\frac{1}{x+y} = \frac{2}{\sqrt{x}} \tag{1} 
        \]
        \[
        1-\frac{1}{x+y} = \frac{\sqrt{2}}{\sqrt{y}} \tag{2} 
        \]
        $(1)^2-(2)^2$:
        \[
        \left(1+\frac{1}{x+y}\right)^2 - \left(1-\frac{1}{x+y}\right)^2 = \frac{4}{x} - \frac{2}{y} \Rightarrow \frac{4}{x+y} = \frac{4y-2x}{xy}
        \]
        整理得
        \[
        x^2 + xy - 2y^2 = 0 \Rightarrow (x+2y)(x-y)=0
        \]
        由于 $x,y \ge 0$,取 $x=y$,代入(1)解得
        \[
        1 + \frac{1}{2x} = \frac{2}{\sqrt{x}} \Rightarrow 4x^2 - 12x +1 =0 \Rightarrow x = y = \frac{3+2\sqrt{2}}{2}
        \]
    \end{solution}

    \question 解方程组
\[
\sqrt{\frac{x+y}{x}}+\sqrt{\frac{x}{x+y}}=\frac{5}{2}, \quad 2x^{2}+y^{2}=176
\]

\begin{solution}

\noindent
\textbf{步骤 1:令代换}
\[
u = \sqrt{\frac{x+y}{x}} \implies \frac{x+y}{x} = u^2
\]

\noindent
方程变为
\begin{align*}
u + \frac{1}{u} &= \frac{5}{2} \\
2u + 2 &= 5\sqrt{u} \\
2u - 5\sqrt{u} + 2 &= 0 \\
(2\sqrt{u}-1)(\sqrt{u}-2) &= 0
\end{align*}

\noindent
解得
\[
\sqrt{u} = \frac{1}{2} \implies u = \frac{1}{4}, \quad
\sqrt{u} = 2 \implies u = 4
\]

\noindent
\textbf{步骤 2:得到 $y$ 与 $x$ 的线性关系}
\begin{align*}
\text{若 } u^2 = \frac{1}{4} &\implies \frac{x+y}{x} = \frac{1}{4} \implies y = -\frac{3}{4}x \\
\text{若 } u^2 = 16 &\implies \frac{x+y}{x} = 16 \implies y = 3x
\end{align*}

\noindent
\textbf{步骤 3:代入 $2x^2+y^2=176$ 求 $x$ 和 $y$}

\noindent
\textbf{情况 1:} $y=-\frac{3}{4}x$
\begin{align*}
2x^2 + \left(-\frac{3}{4}x\right)^2 &= 176 \\
2x^2 + \frac{9}{16}x^2 &= 176 \\
\frac{41}{16}x^2 &= 176 \\
x^2 &= \frac{176 \cdot 16}{41} = \frac{2816}{41} \\
x &= \pm 16 \sqrt{\frac{11}{41}} \\
y &= -\frac{3}{4}x = \mp 12 \sqrt{\frac{11}{41}}
\end{align*}

\noindent
\textbf{情况 2:} $y=3x$
\begin{align*}
2x^2 + (3x)^2 &= 176 \\
2x^2 + 9x^2 &= 176 \\
11x^2 &= 176 \\
x^2 &= 16 \\
x &= \pm 4 \\
y &= 3x = \pm 12
\end{align*}

\noindent
\textbf{步骤 4:解集}
\[
(x,y) = (4,12), \quad (-4,-12), \quad \left(16\sqrt{\frac{11}{41}}, -12\sqrt{\frac{11}{41}}\right), \quad \left(-16\sqrt{\frac{11}{41}}, 12\sqrt{\frac{11}{41}}\right)
\]

\end{solution}


    \question 解方程组
\[
(91-2x)^{3}=216xy^{2}, \quad (37-2y)^{3}=216x^{2}y.
\]

\begin{solution}

\noindent
注意到 $216 = 6^3$,于是两边同时开立方更为直观。令
\[
x = u^3, \quad y = v^3.
\]

\noindent
则方程组变为
\[
(91-2u^3)^3 = 216 u^3 v^6, \quad (37-2v^3)^3 = 216 u^6 v^3.
\]

开立方得
\[
91-2u^3 = 6 u v^2, \quad 37-2v^3 = 6 u^2 v.
\]

移项整理:
\[
6 u v^2 + 2 u^3 = 91, \quad 6 u^2 v + 2 v^3 = 37.
\]

\noindent
两式相加:
\begin{align*}
6 u v^2 + 2 u^3 + 6 u^2 v + 2 v^3 &= 91 + 37 = 128 \\
2(u^3 + v^3 + 3 u^2 v + 3 u v^2) &= 128 \\
u^3 + v^3 + 3 u^2 v + 3 u v^2 &= 64 \\
(u+v)^3 &= 64 \\
u+v &= 4.
\end{align*}

\noindent
代入 $v = 4-u$ 至第一方程:
\begin{align*}
6 u (4-u)^2 + 2 u^3 &= 91 \\
6 u (16 - 8u + u^2) + 2 u^3 &= 91 \\
96 u - 48 u^2 + 6 u^3 + 2 u^3 &= 91 \\
8 u^3 - 48 u^2 + 96 u &= 91 \\
8 (u^3 - 6 u^2 + 12 u) &= 91.
\end{align*}

\noindent
观察因式分解:
\[
u^3 - 6 u^2 + 12 u = (u-2)^3 + 27/8?
\]

更精确地:
\[
u^3 - 6 u^2 + 12 u = (u-2)^3 + 27/8 \quad \text{(通过立方展开或系数对比)}.
\]

于是
\[
8 (u-2)^3 + 64 = 91 \quad \implies \quad 8(u-2)^3 = 27 \quad \implies \quad u-2 = \frac{3}{2} \quad \implies \quad u = \frac{7}{2}.
\]

\noindent
因此
\[
v = 4 - u = 4 - \frac{7}{2} = \frac{1}{2}.
\]

\noindent
回代 $x = u^3$, $y = v^3$:
\[
x = \left(\frac{7}{2}\right)^3 = \frac{343}{8}, \quad y = \left(\frac{1}{2}\right)^3 = \frac{1}{8}.
\]

\noindent
最终解为:
\[
\boxed{x = \frac{343}{8}, \quad y = \frac{1}{8}}.
\]

\end{solution}

\question 已知 $a\in\mathbb{R},\, b\in\mathbb{R}$,解方程组
\[
3(a^2+b^2)^{\frac32}-125a=0,\qquad
4(a^2+b^2)+25b=0.
\]

\begin{solution}
先考虑是否存在平凡解。

若 $a^2+b^2=0$,则 $a=b=0$,代入原方程组显然成立,
故 $(a,b)=(0,0)$ 为一组解。

以下假设 $a^2+b^2\neq0$。

由第一式得
\[
125a=3(a^2+b^2)^{\frac32}.
\]
由第二式得
\[
-25b=4(a^2+b^2).
\]

将第二式两边同乘 $(a^2+b^2)^{\frac12}$,得
\[
-25b(a^2+b^2)^{\frac12}=4(a^2+b^2)^{\frac32}.
\]

于是
\[
125a=3(a^2+b^2)^{\frac32},\qquad
-125b=20(a^2+b^2)^{\frac32}.
\]

两式相除,得
\[
\frac{-a}{b}=\frac{3}{20}(a^2+b^2)^{\frac12}.
\]

平方得
\[
\frac{400a^2}{9b^2}=a^2+b^2.
\]

整理得
\[
400a^2=9b^2(a^2+b^2),
\]
即
\[
a^2(400-9b^2)=9b^2,
\]
从而
\[
a^2=\frac{9b^2}{400-9b^2}.
\]

代入第二个原方程
\[
4(a^2+b^2)+25b=0,
\]
得
\begin{align*}
4\left(\frac{9b^2}{400-9b^2}+b^2\right)+25b&=0\\
4\left(\frac{400b^2}{400-9b^2}\right)+25b&=0\\
\frac{1600b^2}{400-9b^2}+25b&=0.
\end{align*}

两边同乘 $400-9b^2$,得
\[
1600b^2+25b(400-9b^2)=0,
\]
即
\[
-225b^3+1600b^2+10000b=0.
\]

提取公因式 $25b$:
\[
25b(-9b^2+64b+400)=0.
\]

因此
\[
b=0 \quad \text{或} \quad 9b^2-64b-400=0.
\]

当 $b=0$ 时,由第二原方程得 $a=0$,即平凡解。

解二次方程
\[
9b^2-64b-400=0
\]
得
\[
b=\frac{64\pm136}{18},
\]
即
\[
b=\frac{100}{9}\quad\text{或}\quad b=-4.
\]

代入第二原方程检验:
$b=\frac{100}{9}$ 使 $4(a^2+b^2)+25b>0$,不成立,舍去。

当 $b=-4$ 时,
\[
4(a^2+16)-100=0,
\]
得
\[
a^2=9,
\]
即
\[
a=\pm3.
\]

代入第一原方程,$a=-3$ 不满足,$a=3$ 满足。

因此非平凡解为
\[
(a,b)=(3,-4).
\]

综上,方程组的解为
\[
(a,b)=(0,0)\quad \text{或} \quad (3,-4).
\]
\end{solution}

\question 解方程组
\[
x^3 + 9x^2y = -28, \quad y^3 + xy^2 = 1
\]

\begin{solution}

\noindent
将第二个方程乘以 27,使方程的结构便于使用立方和公式:
\[
x^3 + 9x^2y = -28, \quad 27y^3 + 27xy^2 = 27
\]

\noindent
将两式相加并整理:
\begin{align*}
x^3 + 9x^2y + 27xy^2 + 27y^3 &= -28 + 27 = -1 \\
x^3 + 3(x)^2 (3y) + 3(x)(3y)^2 + (3y)^3 &= -1 \\
[x + 3y]^3 &= -1
\end{align*}

\noindent
取实数解:
\[
x + 3y = -1 \quad \implies \quad x = -1 - 3y
\]

\noindent
将 $x = -1 - 3y$ 代入较简单的方程 $y^3 + xy^2 = 1$:
\begin{align*}
y^3 + (-1 - 3y)y^2 &= 1 \\
y^3 - y^2 - 3y^3 &= 1 \\
-2y^3 - y^2 - 1 &= 0 \\
2y^3 + y^2 + 1 &= 0
\end{align*}

\noindent
通过观察,$y = -1$ 是一个解:
\[
2(-1)^3 + (-1)^2 + 1 = -2 + 1 + 1 = 0
\]

\noindent
分解余式:
\[
2y^3 + y^2 + 1 = (y+1)(2y^2 - y + 1)
\]

\noindent
判别式:
\[
\Delta = (-1)^2 - 4\cdot 2 \cdot 1 = 1 - 8 < 0
\]
二次方程无实数解,因此唯一实数解为 $y=-1$。

\noindent
由 $x = -1 - 3y$ 得到:
\[
x = -1 - 3(-1) = 2
\]

\noindent
最终实数解为
\[
(x, y) = (2, -1).
\]

\end{solution}

\question 解方程组
\[
x^3 + 6xy^2 = 99, \quad 2y^3 + 3x^2y = 70
\]

\begin{solution}

\noindent
先设 $y = m x$,其中 $m \neq 0$。代入方程得到
\[
x^3 + 6 x (m^2 x^2) = x^3 + 6 m^2 x^3 = x^3 (1 + 6 m^2) = 99,
\]
\[
2 (m^3 x^3) + 3 x^2 (m x) = 2 m^3 x^3 + 3 m x^3 = x^3 (2 m^3 + 3 m) = 70.
\]

\noindent
两式相除消去 $x^3$:
\[
\frac{1 + 6 m^2}{2 m^3 + 3 m} = \frac{99}{70}.
\]

\noindent
交叉相乘得到
\[
70(1 + 6 m^2) = 99 (2 m^3 + 3 m) \quad \implies \quad 198 m^3 - 420 m^2 + 297 m - 70 = 0.
\]

\noindent
通过观察可知 $m = \frac{2}{3}$ 是一个根。利用多项式除法可分解为
\[
198 m^3 - 420 m^2 + 297 m - 70 = (3m - 2)(66 m^2 - 96 m + 35).
\]

\noindent
检查二次项的判别式:
\[
\Delta = (-96)^2 - 4 \cdot 66 \cdot 35 = 9216 - 9240 = -24 < 0
\]
二次方程无实根,因此唯一实根为
\[
m = \frac{2}{3} \quad \implies \quad y = \frac{2}{3} x.
\]

\noindent
将 $y = \frac{2}{3} x$ 代入第一个方程:
\[
x^3 + 6 x \left(\frac{2}{3} x\right)^2 = x^3 + 6 x \cdot \frac{4}{9} x^2 = x^3 + \frac{24}{9} x^3 = x^3 + \frac{8}{3} x^3 = \frac{11}{3} x^3 = 99,
\]
\[
x^3 = 27 \quad \implies \quad x = 3.
\]

\noindent
得到
\[
y = \frac{2}{3} x = 2.
\]

\noindent
最终解为
\[
(x, y) = (3, 2).
\]

\end{solution}


\question 求实数解
\[
36y^{2}(x+1)+36x^{2}(y+1)=7x^{2}y^{2}, \quad 6x+6y+xy=0
\]

\begin{solution}

\noindent
\textbf{步骤 1:对两个方程进行倒数代换}
\[
X = \frac{1}{x}, \quad Y = \frac{1}{y}
\]

\noindent
第一个方程:
\begin{align*}
\frac{36y^2(x+1)}{36x^2y^2} + \frac{36x^2(y+1)}{36x^2y^2} &= \frac{7x^2y^2}{36x^2y^2} \\
\frac{x+1}{x^2} + \frac{y+1}{y^2} &= \frac{7}{36} \\
\frac{1}{x} + \frac{1}{x^2} + \frac{1}{y} + \frac{1}{y^2} &= \frac{7}{36} \\
X + X^2 + Y + Y^2 &= \frac{7}{36} \quad (1)
\end{align*}

\noindent
第二个方程:
\begin{align*}
\frac{6x}{xy} + \frac{6y}{xy} + \frac{xy}{xy} &= 0 \\
\frac{6}{y} + \frac{6}{x} + 1 &= 0 \\
X + Y &= -\frac{1}{6} \quad (2)
\end{align*}

\begin{align*}
X^2 + \left(-X-\frac{1}{6}\right)^2 &= \frac{13}{36} \\
X^2 + X^2 + \frac{1}{3}X + \frac{1}{36} &= \frac{13}{36} \\
2X^2 + \frac{1}{3}X - \frac{12}{36} &= 0 \\
2X^2 + \frac{1}{3}X - \frac{1}{3} &= 0 \\
6X^2 + X - 1 &= 0
\end{align*}

\[
6X^2 + X - 1 = 0 \implies (3X-1)(2X+1) = 0
\]

\[
X = \frac{1}{3} \quad \text{or} \quad X = -\frac{1}{2}
\]

\[
Y = -X-\frac{1}{6} \implies Y = \frac{1}{6} \quad \text{or} \quad Y = -\frac{2}{3}
\]

\noindent
\textbf{步骤 4:返回到 } x, y
\[
x = \frac{1}{X}, \quad y = \frac{1}{Y}
\]

\[
(x,y) = (3,6), \quad (-2,-\frac{3}{2})
\]

\noindent
\textbf{结论:实数解为}
\[
(x,y) = (3,6), \quad (x,y) = (-2,-\frac{3}{2})
\]

\end{solution}

\question Solve the following simultaneous equations for real $x$ and $y$:
\[
x^4 + y^4 = 97, \quad x+y=5
\]

\begin{solution}

\noindent
\textbf{方法 A:使用对称替换}

\noindent
设
\[
x = u+v, \quad y = u-v
\]

\noindent
则第二个方程给出
\begin{align*}
(u+v) + (u-v) &= 5 \\
2u &= 5 \\
u &= \frac{5}{2}
\end{align*}

\noindent
第一个方程变为
\begin{align*}
(u+v)^4 + (u-v)^4 &= 97 \\
(u^4 + 4u^3v + 6u^2v^2 + 4uv^3 + v^4) + (u^4 - 4u^3v + 6u^2v^2 - 4uv^3 + v^4) &= 97 \\
2u^4 + 12u^2v^2 + 2v^4 &= 97 \\
u^4 + 6u^2v^2 + v^4 &= \frac{97}{2}
\end{align*}

\noindent
代入 $u=\frac{5}{2}$:
\begin{align*}
\left(\frac{5}{2}\right)^4 + 6\left(\frac{5}{2}\right)^2 v^2 + v^4 &= \frac{97}{2} \\
\frac{625}{16} + 6 \cdot \frac{25}{4} v^2 + v^4 &= \frac{97}{2} \\
v^4 + \frac{75}{2} v^2 + \frac{625}{16} - \frac{97}{2} &= 0 \\
v^4 + \frac{75}{2} v^2 - \frac{151}{16} &= 0 \\
16 v^4 + 600 v^2 - 151 &= 0 \\
(4v^2 - 1)(4v^2 + 151) &= 0
\end{align*}

\noindent
所以
\[
4v^2 - 1 = 0 \implies v^2 = \frac{1}{4} \implies v = \pm \frac{1}{2}
\]

\noindent
回代得到:
\[
x = u+v = \frac{5}{2} \pm \frac{1}{2} = 3 \text{ 或 } 2, \quad y = u-v = \frac{5}{2} \mp \frac{1}{2} = 2 \text{ 或 } 3
\]

\noindent
\textbf{解对称性:}
\[
(x,y) = (3,2), \quad (2,3)
\]

\vspace{1em}
\noindent
\textbf{方法 B:直接代入法}

\noindent
由 $x+y=5$ 得 $y=5-x$,代入第一个方程:
\begin{align*}
x^4 + (5-x)^4 &= 97 \\
x^4 + 625 - 500x + 150x^2 - 20x^3 + x^4 &= 97 \\
2x^4 - 20x^3 + 150x^2 - 500x + 528 &= 0 \\
x^4 - 10x^3 + 75x^2 - 250x + 264 &= 0
\end{align*}

\noindent
观察到整数根 $x=2$ 或 $x=3$,所以
\[
(x-2)(x-3)(x^2 - 5x + 44) = 0
\]

\noindent
二次方程无实根,因此实数解为
\[
(x,y) = (2,3), \quad (3,2)
\]

\end{solution}

    \question 设 $a,b,c$ 为相异非零实数,且
    \[
    \frac{1+a^3}{a}=\frac{1+b^3}{b}=\frac{1+c^3}{c}.
    \]
    求 $a^3+b^3+c^3$ 的所有可能值。
    \begin{solution}
        设
        \[
        \frac{1+a^3}{a}=\frac{1+b^3}{b}=\frac{1+c^3}{c}=k
        \]
        则 $a,b,c$ 是方程
        \[
        x^3-kx+1=0
        \]
        的三根,故设
        \[
        x^3-kx+1=(x-a)(x-b)(x-c)=x^3-(a+b+c)x^2+(ab+ac+bc)x-abc
        \]
        比较系数得
        \[
        a+b+c=0,\quad abc=-1
        \]
        故
        \[
        a^3+b^3+c^3=(a+b+c)(a^2+b^2+c^2-ab-bc-ca)+3abc0+3(-1)=-3
        \]
    \end{solution}

    \question 若 $a,b,c \in \mathbb{R}^+$ 满足
    \[
    abc = 120, \quad a+b+c = a^2+b^2+c^2 = a^3+b^3+c^3 = \lambda,
    \]
    求 $\lambda \in \mathbb{Z}^+$。
    \begin{solution}
        已知
        \[
        a+b+c = a^2+b^2+c^2 = a^3+b^3+c^3 = \lambda.
        \]
        由$a^2+b^2+c^2 = (a+b+c)^2 - 2(ab+bc+ca)$,
        \[
        ab+bc+ca = \frac{\lambda^2 - \lambda}{2}.
        \]
        由$a^3+b^3+c^3 = (a+b+c)^3 - 3(a+b+c)(ab+bc+ca) + 3abc$,
        \[
        \lambda = \lambda^3 - 3 \lambda \left(\frac{\lambda^2 - \lambda}{2}\right) + 3 \cdot 120 
        \]
        解得
        \[
        (\lambda-10)(\lambda^2+7\lambda+72)=0 \Rightarrow \lambda = 10 \in \mathbb{Z}^+
        \]
    \end{solution}

    \question 已知
    \[
    \begin{cases} 
    \alpha+\beta+\gamma=6 \\ 
    \alpha^3+\beta^3+\gamma^3=87 \\ 
    (\alpha+1)(\beta+1)(\gamma+1)=33 
    \end{cases}
    \]
    求
    \[
    \frac{1}{\alpha}+\frac{1}{\beta}+\frac{1}{\gamma}.
    \]
    \begin{solution}
        由$\alpha^3+\beta^3+\gamma^3 - 3\alpha\beta\gamma = (\alpha+\beta+\gamma)\bigl((\alpha+\beta+\gamma)^2 - 3(\alpha\beta+\beta\gamma+\gamma\alpha)\bigr)$,则
        \[
        87 - 3\alpha\beta\gamma = 6\bigl(36 - 3(\alpha\beta+\beta\gamma+\gamma\alpha)\bigr) \Rightarrow 6(\alpha\beta+\beta\gamma+\gamma\alpha) - \alpha\beta\gamma = 43 \tag{1}
        \]
        又由$33=(\alpha+1)(\beta+1)(\gamma+1) = 1 + 6 + \alpha\beta+\beta\gamma+\gamma\alpha + \alpha\beta\gamma$,
        \[
        \alpha\beta+\beta\gamma+\gamma\alpha + \alpha\beta\gamma = 26 \tag{2}
        \]
        联立(1),(2)得,
        \[
        \alpha\beta+\beta\gamma+\gamma\alpha = \frac{69}{7}, \quad \alpha\beta\gamma = \frac{113}{7}
        \]
        因此
        \[
        \frac{1}{\alpha}+\frac{1}{\beta}+\frac{1}{\gamma} = \frac{\alpha\beta+\beta\gamma+\gamma\alpha}{\alpha\beta\gamma} = \frac{69}{113}.
        \]
    \end{solution}

    \question 已知 $x,y,z$ 为实数满足 
    \[
    \begin{cases}
    \displaystyle \frac{1}{x}+\frac{1}{y}+\frac{1}{z}=1 \\[6pt]
    x^2+y^2+z^2=\dfrac{3}{2} \\
    x^3+y^3+z^3=1
    \end{cases}
    \]
    且$x+y+z$ 为整数, 求 $x+y+z$ 的值。
    \begin{solution}
        \[
        \begin{cases}
        xy+yz+zx=xyz \\
        x^2+y^2+z^2=\dfrac{3}{2} \\
        x^3+y^3+z^3=1
        \end{cases}
        \]
        令 $a=xyz$, 则
        \[
        (x+y+z)^2 = \frac{3}{2} + 2a, \quad 1-3a = (x+y+z)\left(\frac{3}{2}-a\right) 
        \]
        联立得
        \[
        \left(\frac{1-3a}{\frac{3}{2}-a}\right)^2 = \frac{3}{2}+2a
        \]
        解得 
        \[
        (4a+1)(4a^2-28a+19)=0 \Rightarrow a=xyz=-\frac{1}{4}
        \]
        故
        \[
        x+y+z = \frac{1-3xyz}{\frac{3}{2}-xyz} = \frac{\frac{7}{4}}{\frac{7}{4}} = 1
        \]
    \end{solution}

    \question 已知存在实数 $a, b, x, y$ 满足以下方程:
        \[
        \begin{cases}
        a x^{2014} + b y^{2014} = 6 \\
        a x^{2015} + b y^{2015} = 7 \\
        a x^{2016} + b y^{2016} = 3 \\
        a x^{2017} + b y^{2017} = 50
        \end{cases}
        \]
        求 $a x^{2018} + b y^{2018}$ 的值。
    \begin{solution} 
        不妨设 $f(n)=ax^n+by^n$,于是$f(2014)=6,f(2015)=7,f(2016)=3,f(2017)=50,$
        
        发现\[
            (x+y)f(2015)=ax^{2016}+by^{2016}+ax^{2015}y + bxy^{2015} =f(2016)+xyf(2014)
        \]\[
            (x+y)f(2016)=ax^{2017}+by^{2017}+ax^{2016}y + bxy^{2016} =f(2017)+xyf(2015)
        \]解得\[
        x+y=-9,\;xy=-11
        \]故\[
            (x+y)f(2017)=f(2018)+xyf(2016) \Rightarrow a x^{2018} + b y^{2018}=f(2018)=-417
        \]
    \end{solution}

    \question 求联立方程
    \[
    \begin{cases}
    x\left(2x^2 + y - \dfrac{1}{2}\right) = 0 \\[3pt]
    y \left(x - y + \dfrac{5}{2}\right) = 0
    \end{cases}
    \]
    的所有实数解 $(x, y)$。
    
    \begin{solution}
        \begin{align}
        x\left(2x^2 + y - \dfrac{1}{2}\right) &= 0 \\[3pt]
        y\left(x - y - \dfrac{5}{2}\right) &= 0
        \end{align}
        \textbf{情况一}: 若 $x = 0$,由(2)得
        \[
        y\left(0 - y - \dfrac{5}{2}\right) = 0
        \Rightarrow y = 0 \; \text{或} \; y = -\dfrac{5}{2}.
        \]
        解为$$(0, 0),\left(0, -\dfrac{5}{2}\right)$$
        \textbf{情况二}:  若 $y = 0$,由(1)得
        \[
        x\left(2x^2 + 0 - \dfrac{1}{2}\right) = 0
        \Rightarrow x = 0 \; \text{或} \; x = \pm \dfrac{1}{2}.
        \]
        其中 $x=0$ 已考虑,解为
        \[
        \left( \dfrac{1}{2}, 0 \right), \left( -\dfrac{1}{2}, 0 \right)
        \]
        \textbf{情况三}: 若$x\neq0$且$y\neq0$,
        \[
        \begin{cases}
        2x^2 + y - \dfrac{1}{2} = 0 \\[3pt]
        x - y - \dfrac{5}{2} = 0
        \end{cases}
        \]
        两式相加得
        \[
        2x^2 + x - 3 = 0
        \Rightarrow x = \dfrac{3}{2} \; \text{或} \; x = -1
        \]
        解为
        \[
        \left(\dfrac{3}{2}, -1\right),\; \left(-1, -\dfrac{7}{2}\right)
        \]
        $\therefore\;$原方程组的所有解为
        \[
        (0, 0),\;
        \left(0, -\dfrac{5}{2}\right),\;
        \left(\dfrac{1}{2}, 0\right),\;
        \left(-\dfrac{1}{2}, 0\right),\;
        \left(\dfrac{3}{2}, -1\right),\;
        \left(-1, -\dfrac{7}{2}\right)
        \]
    \end{solution}

    \question 求联立方程
    \[
    \begin{cases}
    x^2 - y - 2z = 4 \\
    y^2 - 2z - 3x = -2 \\
    2z^2 - 3x - 5y = -22
    \end{cases}
    \]
    的所有实数解 $(x, y, z)$。 
    \begin{solution}
        三式相加得
        \[
            x^2 - 6x +y^2 - 6y+2z^2-4z = -20,
        \]
        经配方后变为
        \[
            (x-3)^2+(y-3)^2+2(z-1)^2=0
        \]
        故$x-3=0,y-3=0,z-1=0$,解为$(3, 3, 1)$
    \end{solution}
    
    \question 设实数 $x, y, z$ 满足以下方程
    \[
    \begin{cases}
    4x + 2 y z - 6 z + 9 x z^2 = 4 \\
    x y z = 1
    \end{cases}
    \]
    求 $x + y + z$ 的所有可能值。
    \begin{solution}
        眼光发现$xyz=1$提示了换元$x=\dfrac{a}{b},\;y=\dfrac{b}{c},\;z=\dfrac{c}{a}$,第一式变为\[
            4\left(\dfrac{a}{b}\right)+2\left(\dfrac{b}{c}\right)\left(\dfrac{c}{a}\right)-6\left(\dfrac{c}{a}\right)+9\left(\dfrac{a}{b}\right)\left(\dfrac{c}{a}\right)^2=4 
        \]化简得\[
            4a^2-4ab+2b^2−6bc+9c^2=0 \Rightarrow (2a-b)^2+(b+3c)^2=0
        \]于是$2a-b=0,b+3c=0$,即\[
        x=\frac{a}{b}=\frac{1}{2},\;y=\frac{b}{c}=-3,\;z=\frac{1}{xy}=-\frac{2}{3}\]解得\[
            (x,y,z)=\left(\frac{1}{2},-3,-\frac{2}{3}\right) \Rightarrow x+y+z = -\frac{13}{6}
        \]
        
    \end{solution}
    
    \question 求所有非零实数对 $(x, y)$, 使得满足
        \[
        \frac{x}{x^2 + y} + \frac{y}{x + y^2} = -1 \quad \text{且} \quad \frac{1}{x} + \frac{1}{y} = 1
        \]
    \begin{solution}
        令 $u = x + y,\; v = xy$,则第二式化为 $\dfrac{u}{v} = 1 \Rightarrow u = v$。
        
        由第一式得
        \[
        x(x + y^2) + y(x^2 + y) = -[(x^2 + y)(x + y^2)]
        \]
        展开并代换为 $u, v$ 得
        \[
        u^2 - 2v + uv = - (u^3 - 3uv + v + v^2)
        \Rightarrow u^3 + u^2 - 2v + v^2 - v = 0
        \]
        代入 $u = v$
        \[
        v^3 - v = 0 \Rightarrow v(v^2 - 1) = 0
        \]
        因 $xy \ne 0$,故 $v = \pm1$
        
        若 $v = 1$,则 $x + y = 1,\; xy = 1 \Rightarrow x^2 - x + 1 = 0$,无实根。
        
        若 $v = -1$,则 $x + y = -1,\; xy = -1 \Rightarrow x^2 + x -1 = 0$
        
        解得
        \[
        (x, y) = \left(\frac{-1 \pm \sqrt{5}}{2},\; \frac{-1 \mp \sqrt{5}}{2} \right)
        \]
    \end{solution}

    \question 解方程组
\begin{align*}
& x+y+z = 9 && (1) \\
& x^2+y^2+z^2 = 41 && (2) \\
& x^2(y+z) + y^2(z+x) + z^2(x+y) = 180 && (3)
\end{align*}

\begin{solution}
由(2)式:
\[
(x+y+z)^2 - 2(xy+xz+yz) = 41
\implies 9^2 - 2(xy+xz+yz) = 41
\implies 2(xy+xz+yz) = 40
\implies xy+xz+yz = 20.
\]

由(3)式:
\begin{align*}
x^2(y+z) + y^2(z+x) + z^2(x+y) &= 180 \\
x^2(9-x) + y^2(9-y) + z^2(9-z) &= 180 \\
9(x^2+y^2+z^2) - (x^3+y^3+z^3) &= 180 \\
9(41) - [(x+y+z)(x^2+y^2+z^2 - xy - yz - zx) + 3xyz] &= 180 \\
369 - [9(41-20) + 3xyz] &= 180 \\
369 - (189 + 3xyz) &= 180 \\
180 - 3xyz &= 180 \\
3xyz &= 0 \implies xyz = 0.
\end{align*}

因此,\(x, y, z\) 为三根的方程式为:
\[
t^3 - 9t^2 + 20t = 0 \implies t(t^2 - 9t + 20) = 0 \implies t(t-5)(t-4) = 0.
\]

\[
\therefore \text{解集} = \{(0,5,4), (0,4,5), (4,5,0), (4,0,5), (5,0,4), (5,4,0)\}.
\]
\end{solution}


    \question 已知实数 $x, y, z$ 满足方程组
    \[
    \begin{cases}
    x+y+z=-3 \\[1mm]
    \dfrac{1}{x}+\dfrac{1}{y}+\dfrac{1}{z}=-\dfrac{1}{3} \\[1mm]
    x^2(y+z)+y^2(z+x)+z^2(x+y)=-24
    \end{cases}
    \]
    求 $x^2+y^2+z^2$的值。
    \begin{solution}
        \[
        x+y+z=-3 \tag{1}
        \]
        \[
        \dfrac{1}{x}+\dfrac{1}{y}+\dfrac{1}{z}=-\dfrac{1}{3} \tag{2}
        \]
        \[
        x^2(y+z)+y^2(z+x)+z^2(x+y)=-24 \tag{3}
        \]
        设 $x^2+y^2+z^2 = a$,由 $(1)$ 可得
        \[
        (x+y+z)^2 = 9 = a + 2(xy+yz+zx) \Rightarrow xy+yz+zx = \frac{9-a}{2}
        \]
        由 $(2)$ 得
        \[
        \frac{1}{x}+\frac{1}{y}+\frac{1}{z} = \frac{xy+yz+zx}{xyz} = -\frac{1}{3} \Rightarrow xyz = \frac{3(a-9)}{2}
        \]
        由 $(3)$ 得
        \[
        \begin{aligned}
        &x^2(y+z) + y^2(z+x) + z^2(x+y) \\
        &= x^2(-3-x) + y^2(-3-y) + z^2(-3-z) \\ 
        &= -3(x^2+y^2+z^2) - (x^3+y^3+z^3) \\ 
        &= -3a - \Big[ (x+y+z)((x^2+y^2+z^2)-(xy+yz+zx)) + 3xyz \Big] \\
        &= -3a - \big[ (-3)\left(a - \frac{9-a}{2}\right) + 3 \cdot \frac{3(a-9)}{2} \big] \\
        &= -3a + \frac{-9a+27}{2} + \frac{9a-81}{2} \\
        &= -3a + 27=-24 \Rightarrow a = 17
        \end{aligned}
        \]
    \end{solution}

    \question 已知方程组
\[
\begin{cases}
x+y+z=0 & \text{---(1)} \\
x^2+y^2+z^2=6 & \text{---(2)} \\
x^3+y^3+z^3=-3 & \text{---(3)}
\end{cases}
\]
求 $x^4+y^4+z^4$ 之值。

\begin{solution}
由(2)式:
\[
(x+y+z)^2 - 2(xy+yz+xz) = 6 \implies 0 - 2(xy+yz+xz) = 6 \implies xy+yz+xz = -3.
\]

由(3)式:
\[
x^3+y^3+z^3 = (x+y+z)(x^2+y^2+z^2 - xy - yz - xz) + 3xyz \implies -3 = 0 + 3xyz \implies xyz = -1.
\]

因此,以 $x,y,z$ 为三根的方程式为:
\[
t^3 - 3t + 1 = 0.
\]

于是有:
\begin{align*}
x^3-3x+1=0 &\implies x^4-3x^2+x=0, \\
y^3-3y+1=0 &\implies y^4-3y^2+y=0, \\
z^3-3z+1=0 &\implies z^4-3z^2+z=0.
\end{align*}

三式相加:
\[
x^4+y^4+z^4 = 3(x^2+y^2+z^2) - (x+y+z) = 3\cdot 6 - 0 = 18.
\]

\[
\therefore x^4+y^4+z^4 = 18.
\]
\end{solution}

    \question 求序对 $(x,y,z)$ 满足
    \[
    \begin{cases}
    (x+y)^3 = z,\\
    (y+z)^3 = x,\\
    (z+x)^3 = y.
    \end{cases}
    \]
    \begin{solution}
        不失一般性,设 $x \ge y \ge z$,则有
        \[
        2x \ge x+y \ge x+z, \quad x+y \ge 2y\ge y+z, \quad x+z \ge y+z \ge 2z
        \]
        立方可得
        \[
        \begin{cases}
        8x^3\ge (x+y)^3 \ge (x+z)^3\\
        (x+y)^3 \ge 8y^3 \ge (y+z)^3\\
        (x+z)^3 \ge (y+z)^3 \ge 8z^3
        \end{cases}
        \Rightarrow
        \begin{cases}
        8x^3\ge z \ge y\\
        z \ge 8y^3 \ge x\\
        y \ge x \ge z^3
        \end{cases}
        \Rightarrow z \ge y \ge x
        \]
        因此$x=y=z$,解$(x+x)^3=x$得
        \[
        (x,y,z)=(0,0,0),\left(\frac{\sqrt2}{4},\frac{\sqrt2}{4},\frac{\sqrt2}{4}\right),\left(-\frac{\sqrt2}{4},-\frac{\sqrt2}{4},-\frac{\sqrt2}{4}\right)
        \]
    \end{solution}

    \question 求满足
    \[
    \begin{cases}
    (1+x)(1+x^2)(1+x^4) = 1+y^7,\\
    (1+y)(1+y^2)(1+y^4) = 1+x^7
    \end{cases}
    \]
    的实数序对 $(x,y)$ 个数。
    \begin{solution}
        情况一:$xy=0$,得解 $(0,0)$。  

        情况二:$xy<0$。不失一般性设 $x>0>y$,则 
        \[(1+x)(1+x^2)(1+x^4) > 1
        \]
        且 $1+y^7 < 1$,故无解。  

        情况三:$x,y>0, x \ne y$。不失一般性设 $x>y>0$,则
        \[
        (1+x)(1+x^2)(1+x^4) > 1+x^7 > 1+y^7
        \]
        无解。  

        情况四:$x,y<0, x \ne 0$。不失一般性设 $x<y<0$,则
        \[
        1-x^8=(1+y^7)(1-x)=1-x+y^7-xy^7 \tag{1}
        \]
        \[
        1-y^8=(1+x^7)(1-y)=1-y+x^7-x^7y \tag{2}
        \]
        $(2)-(1)$得,
        \[
        x^8-y^8=x-y+x^7-y^7-xy(x^6-y^6)
        \]
        由于$x<y<0$,则$x^8-y^8>0,x-y<0,x^7-y^7<0,-xy<0,x^6-y^6>0$,左式为正,右式为负,故无解。

        情况五:$x=y$,则
        \[
        1-x^8=1-x+y^7-xy^7=1-x+x^7-x^8 \Rightarrow x=-1,0,1
        \] 
        其中只有 $(0,0),(-1,-1)$ 成立。  

        综上,解为 $(0,0),(-1,-1)$,共有 $2$ 个实数序对。
    \end{solution}

    \question 若非零实数 $a,b,c$ 满足
    \[
    \begin{cases}
    a^2 + b^2 + c^2 = 1,\\[1mm]
    \displaystyle a\left(\frac{1}{b}+\frac{1}{c}\right)+b\left(\frac{1}{a}+\frac{1}{c}\right)+c\left(\frac{1}{a}+\frac{1}{b}\right) = -3,
    \end{cases}
    \]
    求 $a+b+c$ 可能的取值个数。
    \begin{solution}
        由第二个方程式,
        \[
        a^2(b+c) + b^2(a+c) + c^2(a+b) = -3abc \Rightarrow (a+b+c)(ab+bc+ca)=0.
        \]
        又$(a+b+c)^2=1-2(ab+bc+ca)$,所以
        \[
        \frac{1}{2}(a+b+c)((a+b+c)^2-1)=0 \Rightarrow a+b+c=-1,0,1
        \]
        因此 $a+b+c$ 共有 $3$ 个可能值。
    \end{solution}

    \question 求在区间 $[0,2]$ 内满足方程组
    \[
    \begin{cases}
    2x^2 - 4x + 7 = y,\\
    2y^2 - 4y + 7 = z,\\
    2z^2 - 4z + 7 = x
    \end{cases}
    \]
    的无序三元组 $(x,y,z)$个数。
    \begin{solution}
        令 $a=x-1, b=y-1, c=z-1$,则 $a,b,c\in[-1,1]$,方程组化为
        \[
        \begin{cases}
        2a^2-1=b,\\
        2b^2-1=c,\\
        2c^2-1=a.
        \end{cases}
        \]
        令 $a=\cos\theta, b=\cos 2\theta, c=\cos 4\theta,\theta\in[0,\pi]$,得到
        \[
        \cos\theta = \cos 8\theta \Rightarrow -2\sin\frac{9\theta}{2}\sin\frac{7\theta}{2} = 0.
        \]
        解得 
        \[
        \theta -2n\pi= 0, \frac{2\pi}{9}, \frac{4\pi}{9}, \frac{6\pi}{9}, \frac{8\pi}{9}, \frac{2\pi}{7}, \frac{4\pi}{7}, \frac{6\pi}{7},n \in \mathbb{Z}
        \] 
        在区间 $[-1,1]$ 内的无序三元组$(a,b,c)$为 
        \[
        (0,0,0), \left(\cos\frac{2\pi}{7},\cos\frac{4\pi}{7},\cos\frac{8\pi}{7}\right), \left(\cos\frac{2\pi}{9},\cos\frac{4\pi}{9},\cos\frac{8\pi}{9}\right), \left(\cos\frac{2\pi}{3},\cos\frac{2\pi}{3},\cos\frac{2\pi}{3}\right)
        \] 
        故原方程式无序三元组个数为$4$。
    \end{solution}

    \question 解方程
    \[
    \sqrt{x+\sqrt{x}} - \sqrt{x-\sqrt{x}} = \frac{199}{100} \sqrt{\frac{x}{x+\sqrt{x}}}.
    \]
    \begin{solution}
        设 $a=x+\sqrt{x}, b=x-\sqrt{x}$,则 $a-b=\sqrt{a}-\sqrt{b}=2\sqrt{x}$,原方程化为
        \[
        \sqrt{a}+\sqrt{b} = \frac{200}{199}\sqrt{x+\sqrt{x}}, \quad
        \sqrt{a}-\sqrt{b} = \frac{199}{100}\frac{\sqrt{x}}{\sqrt{x+\sqrt{x}}}
        \]
        两式相加得
        \[
        2\sqrt{x+\sqrt{x}}= \frac{200}{199}\sqrt{x+\sqrt{x}}+\frac{199}{100}\frac{\sqrt{x}}{\sqrt{x+\sqrt{x}}}
        \]
        解得
        \[
        \sqrt{x}(19800\sqrt{x}-19801)=0 \Rightarrow \sqrt{x} = \frac{19801}{19800}>0 \Rightarrow x = \frac{19801^2}{19800^2}
        \]
    \end{solution}

    \question 若 $x,y,z$ 为相异复数,满足
    \[
    \begin{cases}
    x+y+z=1,\\
    x^2+y=y^2+z=z^2+x,
    \end{cases}
    \]
    求 $(x-y)(y-z)(z-x)$ 的值。
    \begin{solution}
        由$x^2+y=y^2+z$,得
        \[
        x^2-y^2=(x+y)(x-y)=(1-z)(x-y)=z-y
        \]
        同理得
        \[
        (1-x)(y-z)=x-z,\quad (1-y)(z-x)=y-x
        \]
        由于$x \neq y \neq z$,三式相乘可得
        \[
        (1-x)(1-y)(1-z)=-1 \tag{1}
        \]
        且由$(1-z)(x-y)=z-y$展开得
        \[
        x-z=xz-yz=z(x-y)
        \]
        同理得
        \[
        y-x=x(y-z),\quad z-y=y(z-x)
        \]
        由于$x \neq y \neq z$,三式相乘可得
        \[
        xyz=-1
        \]
        现由$(1)$,得
        \[
        1-(x+y+z)+xy+yz+zx-xyz=-1 \Rightarrow xy+yz+zx=-2
        \]
        设 $k = x^2+y = y^2+z = z^2+x$,则
        \[
        3k=x^2+y^2+z^2+x+y+z=1-2(-2)+1=6 \Rightarrow x^2+y=k=2
        \]
        则
        \[
        x-y=x-(2-x^2)=x^2+x-2=(x-1)(x+2),
        \]
        同理
        \[
        y-z=(y-1)(y+2),\quad z-x=(z-1)(z+2)
        \]
        于是
        \begin{align*}
        &(x-y)(y-z)(z-x) \\
        &= (x-1)(y-1)(z-1)(x+2)(y+2)(z+2) \\
        &= [xyz-(xy+yz+zx)+x+y+z-1][xyz+2(xy+yz+zx)+4(x+y+z)+8]=7
        \end{align*}
    \end{solution}

    \question 已知 $xyz \neq 0,a, b, c$ 不全为零,且满足方程组
    \[
    \begin{cases}
    a = \dfrac{by}{z} + \dfrac{cz}{y} \\[2mm]
    b = \dfrac{cz}{x} + \dfrac{ax}{z} \\[2mm]
    c = \dfrac{ax}{y} + \dfrac{by}{x}
    \end{cases}
    \]
    \begin{parts}
    \part 证明 $a^3x^3 + b^3y^3 + c^3z^3 + abcxyz = 0$。
    \begin{solution}
        将方程组改写为
        \[
        \begin{cases}
        (-ax) \dfrac{1}{x} + (cz) \dfrac{1}{y} + (by) \dfrac{1}{z} = 0 \\[2mm]
        (cz) \dfrac{1}{x} + (-by) \dfrac{1}{y} + (ax) \dfrac{1}{z} = 0 \\[2mm]
        (by) \dfrac{1}{x} + (ax) \dfrac{1}{y} + (-cz) \dfrac{1}{z} = 0
        \end{cases}
        \]
        由与存在非零解 $\left(\dfrac{1}{x},\dfrac{1}{y},\dfrac{1}{z}\right)$,则
        \[
        \begin{vmatrix}
        -ax & cz & by \\
        cz & -by & ax \\
        by & ax & -cz
        \end{vmatrix} = 0
        \]
        展开可得
        \[
        a^3x^3 + b^3y^3 + c^3z^3 + abcxyz = 0
        \]
    \end{solution}
    \part 证明 $\displaystyle \frac{yz}{x^2} + \frac{zx}{y^2} + \frac{xy}{z^2} = -1$。
    \begin{solution}
        改写成
        \[
        \begin{cases}
        - a + \dfrac{y}{z} b + \dfrac{z}{y} c = 0 \\[2mm]
        \dfrac{z}{x} a - b + \dfrac{x}{z} c = 0 \\[2mm]
        \dfrac{x}{y} a + \dfrac{y}{x} b - c = 0
        \end{cases}
        \]
        同理,
        \[
        \begin{vmatrix}
        -1 & \dfrac{y}{z} & \dfrac{z}{y} \\[3mm]
        \dfrac{z}{x} & -1 & \dfrac{x}{z} \\[3mm]
        \dfrac{x}{y} & \dfrac{y}{x} & -1
        \end{vmatrix} = 0
        \]
        展开行列式可得
        \[
        \frac{yz}{x^2} + \frac{zx}{y^2} + \frac{xy}{z^2} = -1
        \]
    \end{solution}
    \part 证明 $a^3 + b^3 + c^3 - 5abc = 0$。
    \begin{solution}
        由原方程组,有
        \[
        a^3 = \left(\frac{by}{z}\right)^3 + \left(\frac{cz}{y}\right)^3 + 3abc, \:
        b^3 = \left(\frac{cz}{x}\right)^3 + \left(\frac{ax}{z}\right)^3 + 3abc, \:
        c^3 = \left(\frac{ax}{y}\right)^3 + \left(\frac{by}{x}\right)^3 + 3abc
        \]
        于是
        \[
        2(a^3+b^3+c^3) = (a^3x^3 + b^3y^3 + c^3z^3) \left(\frac{1}{x^3} + \frac{1}{y^3} + \frac{1}{z^3}\right) + 9abc
        \]
        由 $(a),(b)$ 得
        \[
        2(a^3+b^3+c^3) = -abcxyz \left(\frac{1}{x^3} + \frac{1}{y^3} + \frac{1}{z^3}\right) + 9abc = 10abc
        \]
        即得证
        \[
        a^3 + b^3 + c^3 - 5abc = 0
        \]
    \end{solution}
    \end{parts}

    \question 求实数解 $(a, b, c, d)$ 满足方程组
    \[
    \begin{cases}
    a + 4b + 8c + 4d = 53 \\
    3a^2 + 4b^2 + 12 c^2 + 2 d^2 = 159 \\
    9a^3 + 4b^3 + 18 c^3 + d^3 = 477
    \end{cases}
    \]
    \begin{solution}
        由柯西不等式, 
        \[
        53\cdot477=(a + 4b + 8c + 4d)(9a^3 + 4b^3 + 18 c^3 + d^3)\ge (3a^2 + 4b^2 + 12 c^2 + 2 d^2)^2=159^2
        \]
        此时等号成立,有
        \[
        \frac{a}{3a^2} = \frac{4b}{4b^2} = \frac{8c}{12c^2} = \frac{4d}{2d^2}.
        \]
        设
        \[
        \lambda=\frac{1}{3a} = \frac{1}{b} = \frac{2}{3c} = \frac{2}{d}.
        \]
        则
        \[
        a = \frac{1}{3\lambda}, \quad b = \frac{1}{\lambda}, \quad c = \frac{2}{3\lambda}, \quad d = \frac{2}{\lambda}
        \]
        代入第一式得
        \[
        \frac{1}{3\lambda} + \frac{4}{\lambda} + \frac{16}{3\lambda} + \frac{8}{\lambda} =  \frac{29}{3\lambda} = 53 \Rightarrow \lambda = \frac{29}{159}
        \]
        故实数解为
        \[
        \left( \frac{159}{87}, \frac{159}{29}, \frac{106}{87}, \frac{318}{29} \right)
        \]
    \end{solution}
    
    \question 设正实数 $x$, $y$, $z$ 满足
    \[
    \begin{cases}
        x=\sqrt{y^{2}-\dfrac{1}{49}}+\sqrt{z^{2}-\dfrac{1}{49}}\\[3mm]
        y=\sqrt{x^{2}-\dfrac{1}{64}}+\sqrt{z^{2}-\dfrac{1}{64}}\\[3mm]
        z=\sqrt{x^{2}-\dfrac{1}{81}}+\sqrt{y^{2}-\dfrac{1}{81}}
    \end{cases}
    \]
    求 $x+y+z$。
    \ifprintanswers
    \begin{figure}[H]
        \centering        
        \includegraphics[width=0.5\textwidth]{images/image97.jpg}
    \end{figure}
    \fi
    \begin{solution}
        考虑一边长为 $x,y,z$,对应高为 
        \[
        h_x=\frac{1}{7},\quad h_y=\frac{1}{8},\quad h_z=\frac{1}{9}
        \] 
        的$\triangle ABC$,则 $x,y,z$ 满足题意;$\triangle ABC$面积为 
        \[
        S = \frac{1}{2} x h_x = \frac{1}{2} y h_y = \frac{1}{2} z h_z \Rightarrow x:y:z = 7:8:9
        \] 
        设 $x=7k$, $y=8k$, $z=9k$,半周长 $s=\frac{1}{2}(x+y+z)=12k$,面积又为
        \[
        S = \sqrt{12k\cdot 5k \cdot 4k \cdot 3k} = 12\sqrt{5} k^2
        \]
        联立得
        \[S = \frac{1}{2} x h_x = \frac{k}{2} =12\sqrt{5} k^2 \Rightarrow k = \frac{1}{24\sqrt{5}}
        \]
        因此 
        \[
        x+y+z = 24 k = \frac{\sqrt{5}}{5}
        \]
    \end{solution}

    \question 已知正数 $x,y,z$ 满足
    \[
    \begin{cases}
    x^{2}+xy+y^{2}=2 \\
    y^{2}+yz+z^{2}=1 \\
    z^{2}+zx+x^{2}=3 
    \end{cases}
    \]
    求 $xy+yz+zx$。
    \ifprintanswers
    \begin{figure}[H]
        \centering
        \includegraphics[width=0.4\linewidth]{images/image149.png}
    \end{figure}
    \fi
    \begin{solution}
        方程组改写成
        \[
        \begin{cases}
        x^{2}-2xy\cos 120^\circ+y^{2}=(\sqrt{2})^2 \\
        y^{2}-2yz\cos 120^\circ+z^{2}=1^2 \\
        z^{2}-2zx\cos 120^\circ+x^{2}=(\sqrt{3})^2 
        \end{cases}
        \]
        考虑一边长为$AB=\sqrt{2},BC=1,CA=\sqrt{3}$的$\triangle ABC$,其中
        \[
        OA=x,OB=y,OC=z,\angle AOB = \angle BOC = \angle COA = 120^\circ 
        \]
        且满足
        \[
        AB^2+BC^2=CA^2
        \]
        故 $\triangle ABC$ 为直角三角形,面积为
        \[
        [ABC]=\frac12(xy+yz+zx)\sin120^\circ=\frac12\cdot 1\cdot\sqrt{2}
        \]
        解得
        \[
        xy+yz+zx=\frac{2\sqrt{6}}{3}
        \]
    \end{solution}

    \question 已知 \(x,y,z\) 为实数且满足
    \[
    \begin{cases}
    x^{2} + y^{2} = 18 \\
    y^{2} + \sqrt{3} y z + z^{2} = 13 \\
    x^{2} + x z + z^{2} = 19
    \end{cases}
    \]
    求 \(2 x y + y z + \sqrt{3} x z\)。
    \ifprintanswers
    \begin{figure}[H]
        \centering
        \includegraphics[width=0.5\linewidth]{images/image71.png}
    \end{figure}
    \fi
    \begin{solution}
        方程组改写成
        \[
        \begin{cases}
        x^{2} + y^{2} - 2 x y \cos 90^\circ = (3\sqrt{2})^{2} \\
        y^{2} + z^{2} - 2 y z \cos 150^\circ = (\sqrt{13})^{2} \\
        x^{2} + z^{2} - 2 x z \cos 120^\circ = (\sqrt{19})^{2}
        \end{cases}
        \]
        构造\(\triangle ABC\),满足边长:
        \[
        \overline{AB} = \sqrt{18}, \quad \overline{BC} = \sqrt{13}, \quad \overline{AC} = \sqrt{19}, \quad \overline{OA} = x, \quad \overline{OB} = y, \quad \overline{OC} = z
        \]
        及夹角
        \[
        \angle AOB = 90^\circ, \quad \angle BOC = 150^\circ, \quad \angle AOC = 120^\circ
        \]
        设半周长
        \[
        s = \frac{\sqrt{18} + \sqrt{13} + \sqrt{19}}{2},
        \]
        三角形面积为
        \[
        S = \sqrt{s (s - \sqrt{18})(s - \sqrt{13})(s - \sqrt{19})} = 3 \sqrt{\frac{11}{2}}.
        \]
        面积也等于
        \[
        \frac{1}{2} \left( x y \sin 90^\circ + y z \sin 150^\circ + z x \sin 120^\circ \right) = \frac{1}{2} \left( x y + \frac{1}{2} y z + \frac{\sqrt{3}}{2} x z \right).
        \]
        因此
        \[
        \frac{1}{2} \left( x y + \frac{1}{2} y z + \frac{\sqrt{3}}{2} x z \right) = 3 \sqrt{\frac{11}{2}} \implies 2 x y + y z + \sqrt{3} x z = 6 \sqrt{22}.
        \]
        \end{solution}

    \question 已知联立方程组
    \[
    \begin{cases}
    3x^2 + y^2 - 3xy = 3 + 2\sqrt{2} \\
    y^2 + z^2 - yz = 9 + 6\sqrt{2} \\
    z^2 + w^2 + \sqrt{3}zw = 3 + 2\sqrt{2} \\
    w^2 + 3x^2 + \sqrt{3}wx = 9 + 6\sqrt{2}
    \end{cases}, 
    \]
    求$\sqrt{3}xz + yw$之值。
    \ifprintanswers
    \begin{figure}[H]
        \centering
        \includegraphics[width=0.5\linewidth]{images/image115.jpg}
    \end{figure}
    \fi
    \begin{solution}
        设
        \[
        3x^2 + y^2 - 3xy = 3 + 2\sqrt{2} = (\sqrt{2}+1)^2 = a^2,
        \]
        \[
        y^2 + z^2 - yz = 9 + 6\sqrt{2} = (\sqrt{6}+\sqrt{3})^2 = b^2,
        \]
        \[
        z^2 + w^2 + \sqrt{3}zw = 3 + 2\sqrt{2} = a^2,
        \]
        \[
        w^2 + 3x^2 + \sqrt{3}wx = 9 + 6\sqrt{2} = b^2.
        \]
        考虑一圆内接四边形,边长分别为 $\sqrt{3}x$, $y$, $z$, $w$,对角线长为 $a,b$;由托勒密定理,
        \[
        \sqrt{3} xz + yw = (\sqrt{2}+1)(\sqrt{6}+\sqrt{3}) = 3\sqrt{3} + 2\sqrt{6}.
        \]
    \end{solution}

    \question 已知实数$x,y \in (0,1)$满足
    \[
    \begin{cases} 
    \dfrac{1 - \sqrt{1 - x^2}}{x} + \dfrac{2y}{1 + \sqrt{1 - y^2} + y} = 1, \\[2pt]
    25(1 - y^2) = 41 - 40\sqrt{1 - x^2}
    \end{cases}
    \]
    求$(x,y)$的所有解。
    \begin{solution}
        设$x =\sin \alpha,y=\sin \beta, 0 < \alpha,\beta < \dfrac{\pi}{2}$,则第一式变为
        \[
        \frac{1 - \cos \alpha}{\sin \alpha} + \frac{2 \sin \beta}{1 + \cos \beta + \sin \beta} = 1 
        \]
        继续化简得
        \[
        \tan \frac{\alpha}{2} 
        = 1 - \frac{2 \sin \beta}{1 + \cos \beta + \sin \beta}
        = \frac{1 + \cos\beta - \sin\beta}{1 + \cos\beta + \sin\beta}
        = \frac{1 - \tan \frac{\beta}{2}}{1 + \tan \frac{\beta}{2}}
        = \tan \left(\frac{\pi}{4} - \frac{\beta}{2}\right)
        \]
        故
        \[
        \frac{\alpha}{2}=\frac{\pi}{4} - \frac{\beta}{2} +n\pi, n \in \mathbb{Z} \Rightarrow \alpha + \beta = \frac{\pi}{2} \Rightarrow y=\sin\left(\frac{\pi}{2}-\alpha\right)=\cos \alpha
        \]
        于是第二方程可化为
        \[
        25(1 - y^2) = 41 - 40y
        \]
        解得
        \[
        x=\frac{3}{5},\quad y=\frac{4}{5}
        \]
    \end{solution}

    \question 已知实数 $x>0$ 且 $y, z$ 均为实数,求联立方程组
    \[
    \begin{cases}
    5\left(x + \dfrac{1}{x}\right) = 12\left(y + \dfrac{1}{y}\right) = 13\left(z + \dfrac{1}{z}\right), \\
    xy + yz + zx = 1
    \end{cases}
    \]
    的解 $(x,y,z)$。
    \begin{solution}
        由条件$x > 0, \: x,y,z \in \mathbb{R},$可设
        \[
        x = \tan A, \quad y = \tan B, \quad z = \tan C,
        \]
        其中$0 < A < \dfrac{\pi}{2}, \: -\dfrac{\pi}{2} < B,C < \dfrac{\pi}{2}$,于是
        \[
        5\left(x + \frac{1}{x}\right) = 12\left(y + \frac{1}{y}\right) = 13\left(z + \frac{1}{z}\right) \implies 5 \cdot \frac{2}{\sin 2A} = 12 \cdot \frac{2}{\sin 2B} = 13 \cdot \frac{2}{\sin 2C},
        \]
        且
        \[
        \tan A \tan B + \tan B \tan C + \tan C \tan A = 1.
        \]
        因此
        \[
        \frac{5}{\sin 2A} = \frac{12}{\sin 2B} = \frac{13}{\sin 2C}, \quad A + B + C = 90^\circ.
        \]
        由此得
        \[
        \tan 2A = \frac{5}{12}, \quad \tan 2B = \frac{12}{5}, \quad \tan 2C = \infty,
        \]
        故
        \[
        \tan A = \frac{1}{5}, \quad \tan B = \frac{2}{3}, \quad \tan C = 1.
        \]
        即
        \[
        (x,y,z) = \left(\frac{1}{5}, \frac{2}{3}, 1\right)
        \]
    \end{solution}

    \question 已知正实数 $x,y,z$ 满足
        \begin{align*}
        x+y+z&=xyz\\
        \frac{x^2}{16(1+x^2)}=\frac{y^2}{25(1+y^2)}&=\frac{z^2}{36(1+z^2)}
        \end{align*}
        求$\displaystyle \frac{x^2(1+x^2)^2}{z^2(1+z^2)^2}.$
    \begin{solution}
        考虑换元$x=\tan A,y=\tan B,z=\tan C$,其中$A,B,C\in \left(0,\dfrac{\pi}{2}\right),$且注意到
        \[
        \tan A+\tan B+\tan C=\tan A\tan B\tan C \Longleftrightarrow A+B+C=\pi
        \]
        于是可以将 $A,B,C$ 视为某三角形的内角;由
        \[
        \frac{x^2}{16(1+x^2)}=\frac{y^2}{25(1+y^2)}=\frac{z^2}{36(1+z^2)}
        \]
        化简得
        \[
        \frac{4}{\sin A}=\frac{5}{\sin B}=\frac{6}{\sin C}
        \]
        由正弦定理,不妨设$\triangle ABC$边长为$a=4k,b=5k,c=6k,k\ne0$,则由余弦定理,
        \[
        \cos A=\frac{(5k)^2+(6k)^2-(4k)^2}{2\cdot5k\cdot6k}=\frac{3}{4},\quad \cos C=\frac{(4k)^2+(5k)^2-(6k)^2}{2\cdot4k\cdot5k}=\frac{1}{8}
        \]
        故\[
        \frac{x^2(1+x^2)^2}{z^2(1+z^2)^2}=\frac{\tan^2A\sec^4A}{\tan^2C\sec^4C}
        =\frac{(\frac{\sqrt{7}}{3})^2(\frac{4}{3})^4}{(\sqrt{63})^2\cdot 8^4}= \frac{1}{186624}
        \]
    \end{solution}

    \question 求所有 $(a, b, c) \in \mathbb{R}^3$ 满足方程组
    \[
    \begin{cases}
    a^2 = \dfrac{b^3 + 9\sqrt{3}}{3b} = \dfrac{c^3 + 16}{3c} \\[6pt]
    b^2 = \dfrac{a^3 - 10}{3a} = \dfrac{c^3 + 28}{3c} \\[6pt]
    c^2 = \dfrac{b^3 + 45\sqrt{3}}{3b} = \dfrac{a^3 - 88}{3a}
    \end{cases}
    \]
    \begin{solution}
        由$(1),(2)$
        \[
        3a^2b-b^3=9\sqrt{3}, \quad a^3-3ab^2=10
        \]
        此时展开式
        \[
        (a\pm b)^3=a^3 \pm 3a^2b + 3ab^2 \pm b^3
        \]
        毫无帮助,不妨考虑
        \[
        (a+bi)^3=a^3-3ab^2+(3a^2b-b^3)i=10+9\sqrt{3}i
        \]
        两边取模得
        \[
        (\sqrt{a^2+b^2})^3=\sqrt{10^2+(9\sqrt{3})^2} \Rightarrow a^2+b^2=7 \tag{3}
        \]
        同理可得
        \[
        b^2+c^2=19,\quad c^2+a^2=20 \tag{4}
        \]
        由$(3),(4)$解得$a^2=4,b^2=3,c^2=16$,经检验得原方程组的解为
        \[
        a=-2,\;b=\sqrt{3},\;c=-4
        \]
    \end{solution}
\end{questions}
\pagebreak

\begin{center}
  {\fontsize{30pt}{26pt}\selectfont
    \hypertarget{指数与对数}{指数与对数} \label{指数与对数}
  }
\end{center}
\separator
\vspace{1pt}

\begin{questions}
    \question 设 \( a=2019^{2017}, b=2019^{2018}, c=2019^{2019} \),计算
    \[
    \frac{1}{2019^{a - a} + 2019^{a - b} + 2019^{a - c}} + \frac{1}{2019^{b - a} + 2019^{b - b} + 2019^{b - c}} + \frac{1}{2019^{c - a} + 2019^{c - b} + 2019^{c - c}} 
    \]
    \begin{solution}
        观察到原式为
        \[
        \frac{2019^{-a}}{2019^{- a} + 2019^{- b} + 2019^{- c}} + \frac{2019^{-b}}{2019^{- a} + 2019^{- b} + 2019^{- c}} + \frac{2019^{-c}}{2019^{- a} + 2019^{- b} + 2019^{- c}} 
        \]
        答案呼之欲出$\Rightarrow1$
    \end{solution}          

    \question 试证
    \[
    2^{\sqrt{\log_2 3}} = 3^{\sqrt{\log_3 2}}
    \]
    \begin{solution}
        注意到
        \[
        2^{\sqrt{\log_2 3}} 
        = 2^{\frac{\log_2 3}{\sqrt{\log_2 3}}} 
        = \left(2^{\log_2 3}\right)^{\sqrt{\log_3 2}} 
        = 3^{\sqrt{\log_3 2}}
        \]
        故左式等于右式,得证。
    \end{solution}
    \begin{solution}
        设$x=2^{\sqrt{\log_2 3}},y=3^{\sqrt{\log_3 2}}$,则\[
        \log x=\sqrt{\log_2 3}\cdot \log 2 = \sqrt{\frac{\log 3}{\log 2}}\cdot\log 2 = \sqrt{\log 2 \log 3}
        \]同理\[
        \log y=\sqrt{\log_3 2} \cdot\log 3 = \sqrt{\frac{\log 2}{\log 3}}\cdot\log 3 = \sqrt{\log 2 \log 3}
        \]故\[\log x = \log y \Rightarrow x=y \quad\text{(得证)}\]
    \end{solution}

    \question 解方程
    \[
    2^{x+2}5^{6-x} = 10^{x^{2}},
    \]
    \begin{solution}
        两边取$\log$得\[
        (x+2)\log 2+(6-x)\log5=x^2 \Rightarrow x^2-\left(\log \frac{2}{5}\right)x-2\log 250=0
        \]
        因式分解给出\[
        (x-2)(x+\log 250)=0 \Rightarrow x = 2, -\log 250
        \]
    \end{solution}

    \question
\noindent
求解方程
\[
\frac{e^{2x}+16^x}{(4e)^x} = \frac{4+e}{2\sqrt{e}}, \quad x \in \mathbb{R}.
\]

\begin{solution}
\noindent
\textbf{步骤 1: 化简方程两边}

给定方程:
\[
\frac{e^{2x}+16^x}{(4e)^x} = \frac{4+e}{2\sqrt{e}}
\]

利用指数法则化简左边:
\[
\frac{e^{2x}}{(4e)^x} + \frac{16^x}{(4e)^x} = \left(\frac{e}{4}\right)^x + \left(\frac{4}{e}\right)^x
\]

右边化简为:
\[
\frac{4+e}{2\sqrt{e}} = \frac{\sqrt{e}}{2} + \frac{2}{\sqrt{e}}
\]

因此方程变为:
\[
\left(\frac{e}{4}\right)^x + \left(\frac{4}{e}\right)^x = \frac{\sqrt{e}}{2} + \frac{2}{\sqrt{e}}
\]

\noindent
\textbf{步骤 2: 代换化为二次方程}

令
\[
y = \left(\frac{e}{4}\right)^x
\]

则方程可写作:
\[
y + \frac{1}{y} = \left(\frac{e}{4}\right)^{1/2} + \frac{1}{\left(\frac{e}{4}\right)^{1/2}}
\]

两边乘以 $y$ 得到二次形式:
\[
y^2 - \left(\left(\frac{e}{4}\right)^{1/2} + \left(\frac{e}{4}\right)^{-1/2}\right)y + 1 = 0
\]

因式分解得到:
\[
\left(y - \left(\frac{e}{4}\right)^{1/2}\right)\left(y - \left(\frac{e}{4}\right)^{-1/2}\right) = 0
\]

\noindent
\textbf{步骤 3: 求 $x$ 的值}

\begin{align*}
\left(\frac{e}{4}\right)^x &= \left(\frac{e}{4}\right)^{1/2} \implies x = \frac{1}{2} \\
\left(\frac{e}{4}\right)^x &= \left(\frac{e}{4}\right)^{-1/2} \implies x = -\frac{1}{2}
\end{align*}

\noindent
\textbf{答案:} 
\[
x = \frac{1}{2}, \quad x = -\frac{1}{2}
\]
\end{solution}

    \question 解 
    \[
    x^{x^{x}} = (x^{x})^{x}
    \]
    \begin{solution}
        原方程即
        \[
        x^{x^{x}} = x^{x^2}
        \]
        先考虑指数方程的特殊情况$x=-1,0,1$,可得解
        \[
        x=-1,1
        \]
        当$x \neq -1,0,1$时,由底数相等所以指数相等的性质得,
        \[
        x=2
        \]
        故原方程式的解为
        \[
        x=-1 \quad \text{或} \quad x=1 \quad \text{或} \quad x=2
        \]
    \end{solution}

    \question 已知 
    \[
    (x\sqrt{x}\sqrt[3]{x})^{x} = x^{x\sqrt{x}\sqrt[3]{x}},
    \]
    试求 $x^{5}$ 的值。
    \begin{solution}
        \begin{align*}
        (x\sqrt{x}\sqrt[3]{x})^{x} &= x^{x\sqrt{x}\sqrt[3]{x}} \\
        x^{(1+\frac{1}{2}+\frac{1}{3})x} &= x^{x^{1+\frac{1}{2}+\frac{1}{3}}}\\
        x^{(\frac{11}{6})x} &= x^{x^{\frac{11}{6}}}
        \end{align*}
        考虑特殊情况 $x=-1, 0, 1,$ 可得解 $x=1$,此时因底数相等所以指数相等,得
        \[
        \frac{11}{6}x = x^{\frac{11}{6}} \Rightarrow x^5 = \left(\frac{11}{6}\right)^{6} 
        \]
        故$x^5$的可能值为
        \[
        x^5=1 \quad \text{或} \quad x^5 = \left(\frac{11}{6}\right)^{6}
        \]
    \end{solution}

    \question 求满足方程 
    \[
    (x^{2}+2x)^{x^{2}-3x+2}=1
    \] 
    的实数解。
    \begin{solution}
        设 $A=x^{2}+2x$, $B=x^{2}-3x+2$。欲使 $A^{B}=1$,仅以下三种情况成立:
        \begin{itemize}
        \item $B=0$:
        \[
        x^{2}-3x+2=0 \;\;\Rightarrow\;\; (x-1)(x-2)=0\Rightarrow\;\;x=1,2
        \]
        \item $A=1$:
        \[
        x^{2}+2x=1 \;\;\Rightarrow\;\; x=-1\pm\sqrt{2}
        \]
        \item $A=-1$ 且 $B$ 为偶数:  
        \[
        x^{2}+2x=-1 \;\;\Rightarrow\;\; (x+1)^{2}=0 \;\;\Rightarrow\;\; x=-1
        \]  
        此时  
        \[
        B=(-1)^{2}-3(-1)+2=6
        \]  
        为偶数,成立。
        \end{itemize}
        综上,解为  
        \[
        x\in \{1,2,-1+\sqrt{2},-1-\sqrt{2},-1\}
        \]
    \end{solution}

    \question 计算
    \[
    5^{(\log 2)^3} \cdot 8^{(\log 5)^2} \cdot 5^{(\log 5)^3}
    \]
    \begin{solution}
        令$a=\log 2,b=\log 5$,则$a+b=1$,原式为
        \[
        L=5^{a^3}\cdot 8^{b^2} \cdot 5^{b^3}
        \]
        两边取对数,则
        \[
        \log L 
        = (a^3 + b^3)b + 3ab^2 
        = (a+b)(a^2-ab+b^2) b + 3ab^2 
        = b(a+b)^2 
        = \log 5
        \]
        因此
        \[
        L = 5
        \]
    \end{solution}

    \question 已知正实数 $x$ 满足
    \[
    \log_2 x\log_4 x\log_6 x = \log_2 x\log_4 x+\log_2 x\log_6 x+\log_4 x\log_6 x,
    \]
    求 $x$。
    \begin{solution}
        设$\log_2 x=a, \log_6 x=b$,则$\log_4 x=\dfrac{1}{2}a$,原式为
        \[
        \frac{1}{2}a^2 b = \frac{1}{2}a^2 + ab + \frac{1}{2}ab
        \Rightarrow ab = a + 3b
        \]
        将 $a=\dfrac{\log x}{\log 2},b=\dfrac{\log x}{\log 6}$ 代入得
        \[
        \frac{(\log x)^2}{(\log 2)(\log 6)}
        = \frac{\log x(\log 6+3\log 2)}{(\log 2)(\log 6)} \Rightarrow \log x(\log x-\log 48)=0
        \]
        所以
        \[
        x=1\quad\text{或}\quad x=48
        \]
    \end{solution}

    \question 求
    \[
    \sqrt{\log_3 \sqrt{6} + \sqrt{\log_3 2}} + \sqrt{\log_3 \sqrt{6} - \sqrt{\log_3 2}}
    \]
    之值。
    \begin{solution}
        令
        \[
        a = \log_3 \sqrt{6} = \frac{1}{2} (1 + \log_3 2), \quad 
        b = \sqrt{\log_3 2} < 1
        \]
        则有
        \[
        a = \frac{b^2 + 1}{2}
        \]
        因此
        \begin{align*}
        \sqrt{a+b} + \sqrt{a-b} &= \sqrt{\frac{b^2+1}{2} + b} + \sqrt{\frac{b^2+1}{2} - b} \\
        &= \sqrt{\frac{1}{2} (b+1)^2} + \sqrt{\frac{1}{2} (b-1)^2} \\
        &= \frac{1}{\sqrt{2}} (b+1) + \frac{1}{\sqrt{2}} (1-b) = \sqrt{2}
        \end{align*}
    \end{solution}

    \question 解不等式
    \[
    \sqrt{\log_2 x - 1} + \frac{1}{2} \log_{\frac{1}{2}} x^3 + 2 > 0
    \]
    \begin{solution}
        设 \( y = \log_2 x \),则不等式化为
        \[
        \sqrt{y - 1} - \frac{3}{2} y + 2 > 0, \quad \text{且} \;y \ge 1.
        \]
        设$z=\sqrt{y - 1}$,则$y=z^2+1$,方程变为
        \[
        z - \frac{3}{2} (z^2+1) + 2 > 0 \Rightarrow (3z+1)(z-1)<0 
        \]
        于是
        \[
        -\frac13<z<1 \Rightarrow 0\leq z^2<1 \Rightarrow1\leq y<2 
        \]
        再考虑$y$的定义域$y\geq1$,经检验有
        \[
        2\leq x <4
        \]
    \end{solution}
    
    \question 求不等式的解集
    \[
    \log (x - 40) + \log (60 - x) < 2
    \]
    \begin{solution}
        变为
        \[
            (x - 40)(60 - x) < 100
        \]
        注意到\((x - 40)(60 - x) = -(x - 50)^2 + 100\),所以不等式等价于
        \[
        -(x - 50)^2 + 100 < 100 \Rightarrow (x - 50)^2 > 0\Rightarrow x \ne 50
        \]
        且由定义域\[
        \begin{cases}
        x - 40 > 0 \\
        60 - x > 0
        \end{cases} \Rightarrow 40<x<60
        \]经检验得解集
        \[
        (40,\;50)\cup(50,\;60).
        \]
    \end{solution}

    \question 解不等式:
    \[
    \left(\log_{\frac{1}{3}} x - 1\right)\left(\log_{\frac{1}{4}} x + 2\right)\left(\log_{\frac{1}{5}} x - 3\right) > 0.
    \]
    \begin{solution}
    换底得
    \[
    \left(\frac{\log x}{-\log 3} - 1\right)\left(\frac{\log x}{-\log 4} + 2\right)\left(\frac{\log x}{-\log 5} - 3\right) > 0
    \]
    即
    \[
    (\log x + \log 3)(\log x - 2\log 4)(\log x + 3\log 5) < 0
    \]
    符号分析得
    \[
    -\log 3 < \log x < 2\log 4, \quad \log x < -3\log 5
    \]
    解集为
    \[
    \frac{1}{3} < x < 16 \quad \text{或} \quad 0 < x < \frac{1}{125}
    \]
    \end{solution}

    \question 求 
    \[
    |x - 1|^{\log_{2}(4 - x)} < |x - 1|^{\log_{2}(1 + x)}
    \]
    的解之范围。
    \begin{solution}
        由$\log_{2}(4 - x)$及$\log_{2}(1 + x)$的定义域,得
        \[  
        x < 4 \quad \text{且} \quad x > -1 
        \]
        即  $x \in (-1, 4)$,且$x \neq 0,1,2$,否则不等式不成立。分四种情况讨论:

        情况1: $x \in (-1, 0)$,则
        \begin{align*}
        (1-x)^{\log_2 (4-x)} < (1-x)^{\log_2 (1+x)} & \Rightarrow \log_2 (4-x) < \log_2 (1+x) \\
        & \Rightarrow 4-x < 1+x \\
        & \Rightarrow x > \frac{3}{2}
        \end{align*}
        此情况解集为 $(-1, 0) \cap \left(\dfrac{3}{2}, \infty\right) = \emptyset$,无解。

        情况2: $x \in (0, 1)$,则
        \begin{align*}
        (1-x)^{\log_2 (4-x)} < (1-x)^{\log_2 (1+x)} & \Rightarrow \log_2 (4-x) > \log_2 (1+x) \\
        & \Rightarrow 4-x > 1+x \\
        & \Rightarrow x < \frac{3}{2} 
        \end{align*}
        此情况解集为 $(0, 1) \cap \left(-\infty, \dfrac{3}{2}\right) = (0, 1)$。

        情况3: $x \in (1, 2)$,则
        \begin{align*}
        (x-1)^{\log_2 (4-x)} < (x-1)^{\log_2 (1+x)} & \Rightarrow \log_2 (4-x) > \log_2 (1+x) \\
        & \Rightarrow 4-x > 1+x \\
        & \Rightarrow x < \frac{3}{2}
        \end{align*}
        此情况解集为 $(1, 2) \cap \left(-\infty, \dfrac{3}{2}\right) = \left(1, \dfrac{3}{2}\right)$。

        情况4: $x \in (2, 4)$,则
        \begin{align*}
        (x-1)^{\log_2 (4-x)} < (x-1)^{\log_2 (1+x)} & \Rightarrow \log_2 (4-x) < \log_2 (1+x) \\
        & \Rightarrow 4-x < 1+x \\
        & \Rightarrow x > \frac{3}{2}
        \end{align*}
        此情况解集为 $(2, 4) \cap \left(\dfrac{3}{2},\infty \right) = (2, 4)$。

        综上所述,原方程式的解集为 $\left(0, \dfrac{3}{2}\right) \cup (2, 4) \setminus \{1\}$。
    \end{solution}


    \question 求所有实数 $x$,使得
    \[
    \sqrt{\log_2 x \cdot \log_2 (4x) + 1} + \sqrt{\log_2 x \cdot \log_2 \left(\frac{x}{64}\right) + 9} = 4
    \]
    \begin{solution}
        发现
        \begin{align*}
        f(x) 
        &= \sqrt{\log_2 x \cdot \log_2(4x) + 1} + \sqrt{\log_2 x \cdot \log_2\left(\frac{x}{64}\right) + 9} \\
        &= \sqrt{\log_2 x (2 + \log_2 x) + 1} + \sqrt{\log_2 x (\log_2 x - 6) + 9} \\
        &= \sqrt{(\log_2 x + 1)^2} + \sqrt{(\log_2 x - 3)^2} \\
        &= 
        \begin{cases}
            \begin{aligned}
            &-2\log_2 x+ 2&&, \log_2 x \le -1 \\[1mm]
            &4&&, -1 < \log_2 x \le 3 \\[1mm]
            &2\log_2 x- 2&&, \log_2 x > 3
            \end{aligned}
        \end{cases}
        \end{align*}
        欲使$f(x)=4$,当 $\log_2 x \le -1,\log_2x = -1$;
        当 $-1 < \log_2x \le 3,f(x) = 4$ 恒成立;
        当 $\log_2x > 3$,无解;因此解为
        \[
        -1 \le \log_2x \le 3 \Rightarrow \frac{1}{2} \le x \le 8
        \]
    \end{solution}

    \question 已知 
    \[
    \frac{\log a}{b - c} = \frac{\log b}{c - a} = \frac{\log c}{a - b},
    \]
    其中 $a \neq b \neq c$, 试求 $a^{a}b^{b}c^{c}$ 的值。
    \begin{solution}
        设
        \[
        \frac{\log a}{b-c} = \frac{\log b}{c-a} = \frac{\log c}{a-b} = k
        \]
        则
        \[
        \log a = k(b-c), \quad \log b = k(c-a), \quad \log c = k(a-b)
        \]
        因此
        \begin{align*}
        \log(a^a b^b c^c) &= a\log a + b\log b + c\log c \\
        &= ak(b-c) + bk(c-a) + ck(a-b) \\
        &= k[ab - ac + bc - ab + ca - cb] \\
        &= 0
        \end{align*}
        即 
        \[
        a^a b^b c^c = 1
        \]
    \end{solution}

    \question 设 $a,b$同号,且 $a^2 - 2ab - 9b^2 = 0$,求
    \[
    \log(a^2 + ab - 6b^2) - \log(a^2 + 4ab + 15b^2)
    \]
    的值。
    \begin{solution}
        由 $a^2 - 2ab - 9b^2 = 0 $解得
        $$a = (\sqrt{10} + 1)b$$
        故
        \[
        \begin{aligned}
        &\log(a^2 + ab - 6b^2) - \log(a^2 + 4ab + 15b^2) \\
        &= \log(3ab + 3b^2) - \log(6ab + 24b^2) \\
        &= \log\left(\frac{3b(a + b)}{6b(a + 4b)}\right)\\
        &= \log\left(\frac{1}{2} \cdot \frac{a + b}{a + 4b}\right) \\
        &= \log\left(\frac{1}{2} \cdot \frac{(\sqrt{10} + 2)b}{(\sqrt{10} + 5)b}\right) \\
        &= \log\left(\frac{1}{2} \cdot \frac{(\sqrt{10} + 2)(5 - \sqrt{10})}{(\sqrt{10} + 5)(5 - \sqrt{10})}\right) \\
        &= \log\left(\frac{1}{2} \cdot \frac{3\sqrt{10}}{15}\right)= \log\left(\frac{\sqrt{10}}{10}\right) = -\frac{1}{2}
        \end{aligned}
        \]
    \end{solution}

    \question 设正实数 \( x, y \)(\( x \neq 1, y \neq 1 \))满足
    \[
    \log_2 x = \log_y 16, \quad xy = 64,
    \]
    求
    \[
    \left(\log_2 \frac{x}{y}\right)^2。
    \]
    \begin{solution}
        由 \(xy = 64\)得 \(x = \frac{64}{y}\),代入第一式:
        \[
        \log_2 \left( \frac{64}{y} \right) = \log_y 16
        \Rightarrow 6 - \log_2 y = \frac{4}{\log_2 y}
        \]
        设 \(a = \log_2 y\),则变为
        \[
        6 - a = \frac{4}{a} \Rightarrow a^2 - 6a + 4 = 0
        \]
        解得
        \[
        \log_2 y =a= 3 \pm \sqrt{5}
        \]
        所求为
        \[
        \left(\log_2 x - \log_2 y\right)^2 = (a - (6 - a))^2 = (2a - 6)^2 = (\pm 2\sqrt{5})^2 = 20
        \]
    \end{solution}
    
    \question 已知 $a^x = bc,b^y = ac,c^z = ab,$ 证明 $$xyz = x + y + z + 2$$

    \begin{solution}
        由 $a^x = bc$,两边取 $yz$ 次方
        \[
        (a^x)^{yz} = (bc)^{yz} \Rightarrow a^{xyz} = b^{yz} \cdot c^{yz}
        \]
        由 $b^y = ac$,得 $b^{yz} = (ac)^z = a^z c^z$ ;由 $c^z = ab$,得 $c^{yz} = (ab)^y = a^y b^y$;于是        
        \[
        a^{xyz} = (a^z c^z)(a^y b^y) = a^{z+y} b^y c^z = a^{z+y} \cdot ac \cdot ab = a^{z+y+2} bc
        \]
        又 $a^x = bc$,故
        \[
        a^{xyz} = a^{z+y+2} \cdot a^x = a^{x+y+z+2} \Rightarrow xyz = x + y + z + 2
        \]
    \end{solution}

    \question 设 \(a, b, c\) 均为异于 1 的正数,且满足 \(abc = 1\), 证明
    \[
    \log_a b + \log_a c + \log_b c + \log_b a + \log_c b + \log_c a = -3
    \]
    \begin{solution}
    有\begin{align*}
          &\log_a a+\log_a b + \log_a c + \log_b b+ \log_b c + \log_b a + \log_c c + \log_c b + \log_c a \\ &=\log_a (abc)+\log_b (abc)+\log_c (abc)\\&=0
      \end{align*}
      故原式得证。
    \end{solution}

    \question 已知
        \[
        \log_{10} \sin x + \log_{10} \cos x = -1,
        \]
        且整数 $n$ 满足
        \[
        \log_{10} (\sin x + \cos x) = \frac{1}{2} (\log_{10} n - 1),
        \]
        求 \( n\)。
    \begin{solution}
        由
        \[
        \log_{10} \sin x + \log_{10} \cos x = -1
        \Rightarrow \sin x \cos x = 10^{-1}= \frac{1}{10}
        \]
        又$\sin x + \cos x=\sqrt{1+2\cdot\dfrac{1}{10}}=\sqrt{\dfrac{6}{5}}$ ,则
        \[
        \log_{10}{n} = 2 \log_{10}{\sqrt{\frac{6}{5}}} + 1 = \log_{10}{\frac{6}{5}} + \log_{10}{10} = \log_{10}{12} \Rightarrow n=12
        \]
    \end{solution}

    \question 解方程式 
    \[
    (\log_{5}x)^{2} + \log_{5x}\left(\frac{5}{x}\right) = 1
    \]
    \begin{solution}
        设$\log_5 x = t$,则
        \[
        \log_{5x}\left(\frac{5}{x}\right) = \frac{\log_5\frac{5}{x}}{\log_5(5x)} = \frac{\log_5 5 - \log_5 x}{\log_5 5 + \log_5 x} = \frac{1-t}{1+t}
        \]
        原方程式变为
        \begin{align*}
        t^2 + \frac{1-t}{1+t} &= 1 \\
        t^2(1+t) + 1-t &= 1+t \\
        t^3 + t^2 - 2t &= 0 \\
        t(t+2)(t-1) &= 0
        \end{align*}
        解得 $t = 0, -2, 1$,即 
        \[
        x = 5^0 = 1, \quad \text{或} \quad x = 5^{-2} = \frac{1}{25}, \quad \text{或} \quad x = 5^1 = 5
        \]
        经检验得$x = 1, x = \dfrac{1}{25}, x=5$都是原方程的解。
    \end{solution}
    
    \question 解方程
    \[
    \log_{2x} \left( 48 \sqrt[3]{3} \right) = \log_{3x} \left( 162 \sqrt[3]{2} \right)
    \]
    \begin{solution}
        原式变为
        \[
        \frac{\log (48 \sqrt[3]{3})}{\log (2x)} = \frac{\log (162 \sqrt[3]{2})}{\log (3x)}.
        \]
        又因
        \[
        \log (48 \sqrt[3]{3}) = 4 \log 2 + \frac{4}{3} \log 3,\quad 
        \log (162 \sqrt[3]{2}) = \frac{4}{3} \log 2 + 4 \log 3.
        \]
        于是有
        \[
        \frac{4 \log 2 + \frac{4}{3} \log 3}{\log 2 +  \log x}
        =
        \frac{\frac{4}{3} \log 2 + 4 \log 3}{\log 3 +  \log x}.
        \]
        化简最终得到
        \[
        \log x = \frac{1}{2} \log 6 \Rightarrow x = \sqrt{6}.
        \]
    \end{solution}
    
    \question 求所有实数 $x$,使得
    \[
    \log_{5x+9}(x^2+6x+9) + \log_{x+3}(5x^2+24x+27) = 4.
    \]
    \begin{solution}
        原方程式化为
        \begin{align*}
        \frac{\log(x^2+6x+9)}{\log(5x+9)} + \frac{\log(5x^2+24x+27)}{\log(x+3)} &= 4 \\
        \frac{2\log(x+3)}{\log(5x+9)} + \frac{\log(5x+9)+\log(x+3)}{\log(x+3)} &= 4 \\
        2 \frac{\log(x+3)}{\log(5x+9)} + \frac{\log(5x+9)}{\log(x+3)} + 1 &= 4
        \end{align*}
        令 $t = \dfrac{\log(x+3)}{\log(5x+9)}$,解得
        \[
        2t + \frac{1}{t} = 3 \Rightarrow t = 1 \ \text{或}\ t = \frac{1}{2}
        \]
        当$t = 1$,解得
        \[
        \log(x+3) = \log(5x+9) \Rightarrow x = -\frac{3}{2}
        \]
        当$t = \frac{1}{2}$,则 $2\log(x+3) = \log(5x+9)$,即
        \[
        2\log(x+3) = \log(5x+9) \Rightarrow x = 0 \ \text{或}\ x = -1
        \]
        经检验得,原方程的实数解为
        \[
        x = 0, -1, -\frac{3}{2}
        \]
    \end{solution}

    \question
\noindent
求解下列方程组:
\begin{align*}
(2x)^{\ln 2} &= (3y)^{\ln 3} \\
3^{\ln x} &= 2^{\ln y}
\end{align*}

\begin{solution}
\noindent
\textbf{步骤 1: 对方程取自然对数化简}

对第一个方程两边取 $\ln$:
\begin{align*}
\ln\left((2x)^{\ln 2}\right) &= \ln\left((3y)^{\ln 3}\right) \\
(\ln 2)\ln(2x) &= (\ln 3)\ln(3y) \\
(\ln 2)(\ln 2 + \ln x) &= (\ln 3)(\ln 3 + \ln y) \\
(\ln 2)^2 + (\ln 2)\ln x &= (\ln 3)^2 + (\ln 3)\ln y \quad (\text{方程 } 1')
\end{align*}

对第二个方程取 $\ln$:
\begin{align*}
\ln\left(3^{\ln x}\right) &= \ln\left(2^{\ln y}\right) \\
(\ln x)\ln 3 &= (\ln y)\ln 2 \quad (\text{方程 } 2')
\end{align*}

\noindent
\textbf{步骤 2: 代换求解}

令 $u = \ln x$,$v = \ln y$:
\begin{align*}
(\ln 2)^2 + u \ln 2 &= (\ln 3)^2 + v \ln 3 \\
u \ln 3 &= v \ln 2 \implies u = v \frac{\ln 2}{\ln 3}
\end{align*}

将 $u$ 代入方程 $1'$:
\begin{align*}
(\ln 2)^2 + \left(v\frac{\ln 2}{\ln 3}\right)\ln 2 &= (\ln 3)^2 + v \ln 3 \\
(\ln 2)^2 + v\frac{(\ln 2)^2}{\ln 3} &= (\ln 3)^2 + v \ln 3 \\
v\left(\frac{(\ln 2)^2}{\ln 3} - \ln 3\right) &= (\ln 3)^2 - (\ln 2)^2 \\
v\left(\frac{(\ln 2)^2 - (\ln 3)^2}{\ln 3}\right) &= -((\ln 2)^2 - (\ln 3)^2) \\
v &= -\ln 3
\end{align*}

\noindent
\textbf{步骤 3: 求 $u$,$x$ 和 $y$}

\[
u = v \frac{\ln 2}{\ln 3} = -\ln 3 \cdot \frac{\ln 2}{\ln 3} = -\ln 2
\]

转回 $x$ 和 $y$:
\begin{align*}
\ln x &= -\ln 2 \implies x = e^{-\ln 2} = \frac{1}{2} \\
\ln y &= -\ln 3 \implies y = e^{-\ln 3} = \frac{1}{3}
\end{align*}
\end{solution}


    \question 求下列方程组的所有实数解:
    \begin{align*}
    x + \log_{10} x &= y-1 \\
    y + \log_{10}(y-1) &= z-1 \\
    z + \log_{10}(z-2) &= x+2
    \end{align*}
    \begin{solution}
        将方程组改写为
        \begin{align*}
        x + \log_{10} x &= y-1 \\
        (y-1) + \log_{10}(y-1) &= z-2 \\
        (z-2) + \log_{10}(z-2) &= x
        \end{align*}
        令 $a = x,b = y-1,c = z-2$,方程组可改写为:
        \begin{align}
        a + \log_{10} a &= b \tag{1} \\
        b + \log_{10} b &= c \tag{2} \\
        c + \log_{10} c &= a \tag{3}
        \end{align}
        容易验证 $(a,b,c)=(1,1,1)$ 是一解,现考虑 $a>1$,则
        \[
        \log_{10} a > 0 \Rightarrow b = a+\log_{10}a > a > 1 \Rightarrow c = b+\log_{10}b > b > a > 1
        \]
        由 (3) 得 $a = c + \log_{10}c > c > b > a$,矛盾。而当 $0<a<1$,
        \[
        \log_{10} a < 0 \Rightarrow b = a+\log_{10}a < a < 1 \Rightarrow c = b+\log_{10}b < b < a < 1
        \]
        由 (3) 得 $a = c + \log_{10}c < c < b < a$,矛盾。
        因此$(a,b,c)=(1,1,1)$ 是唯一解,即 
        \[
        (x,y,z)=(1,2,3)
        \]
    \end{solution}

    \question 若正实数 $x,y,z$ 满足
    \[
    \begin{cases}
    \log_{2}x + \log_{4}y + \log_{4}z = 2 \\
    \log_{3}y + \log_{9}z + \log_{9}x = 2 \\
    \log_{4}z + \log_{16}x + \log_{16}y = 2
    \end{cases}
    \]
    求 $xyz$。
    \begin{solution}
        原方程组给出
        \[
        \begin{cases}
        \log_2 x + \log_2 \sqrt{y} + \log_2 \sqrt{z} = 2 \Rightarrow x\sqrt{yz} = 2^2 \\
        \log_3 y + \log_3 \sqrt{z} + \log_3 \sqrt{x} = 2 \Rightarrow y\sqrt{xz} = 3^2 \\
        \log_4 z + \log_4 \sqrt{x} + \log_4 \sqrt{y} = 2
        \Rightarrow z\sqrt{xy} = 4^2
        \end{cases}
        \]
        将三式相乘得
        \[
        (xyz)^2 = 2^6 \cdot 3^2 \Rightarrow xyz = 24
        \]
    \end{solution}

    \question 解联立方程组
    \[
    \begin{cases}
    \log x \log y - 3 \log 5 y - \log 8 x = -4 \\
    \log y \log z - 4 \log 5 y - \log 16 z = 4 \\
    \log z \log x - 4 \log 8 x - 3 \log 625 z = -18
    \end{cases}
    \]
    \begin{solution}
        三式化简得
        \begin{align*}
        \log x \log y - 3 \log  y - \log x &= -1 \\
        \log y \log z - 4 \log  y - \log  z &= 8 \\
        \log z \log x - 4 \log  x - 3 \log z &= -6
        \end{align*}
        因式分解:
        \begin{align}
        (\log x-3)( \log y -1 )&= 2 \\
        (\log y-1)( \log z -4 )&= 12 \\
        (\log z-4)( \log x -3 )&= 6 
        \end{align}
        三式相乘可得
        \begin{equation}
        (\log x-3)( \log y -1 )(\log z-4)= \pm\sqrt{2\cdot 6\cdot 12} =\pm12
        \end{equation}
        分别作$\dfrac{(4)}{(2)},\dfrac{(4)}{(3)},\dfrac{(4)}{(1)}$得解   $\log x-3= \pm1,\log y -1=\pm2,\log z-4=\pm6$,即\[
        (x,y,z) = (10^4, 10^3, 10^{10}) \quad \text{或} \quad (10^2, 10^{-1}, 10^{-2})
        \]
    \end{solution}
\end{questions}

\pagebreak

\begin{center}
  {\fontsize{30pt}{26pt}\selectfont
    \hypertarget{不等式}{不等式} \label{不等式}
  }
\end{center}
\separator
\vspace{1pt}

\begin{questions}
    \question 求满足不等式
    \[
    \frac{4x^{2}}{(1-\sqrt{1+2x})^{2}}<2x+9
    \]
    的实数 $x$ 的集合。
    \begin{solution}
        首先定义域要求 $1+2x \ge 0$,即 $x \ge -\dfrac{1}{2}$,且分母不为零,即 $1-\sqrt{1+2x} \ne 0$,得 $x \ne 0$,发现
        \[
        \frac{4x^{2}}{(1-\sqrt{1+2x})^{2}}=\left(\frac{1-(1+2x)}{1-\sqrt{1+2x}}\right)^2= (1+\sqrt{1+2x})^2
        \]
        由$(1+\sqrt{1+2x})^2 < 2x+9$解得
        \[
        \sqrt{1+2x} < \frac{7}{2} \Rightarrow x < \frac{45}{8}.
        \]
        故$x$ 的取值范围为
        \[
        \left[-\frac{1}{2},0\right) \cup \left(0,\frac{45}{8}\right).
        \]
    \end{solution}

    \question 已知
    \[
    \begin{cases}
    \displaystyle \sqrt{x(x-3)} + \frac{8}{y^2} + \frac{y^2}{\sqrt{x(x-3)}} = 6, \\
    x+y < 0
    \end{cases}
    \]
    求 $(x,y)$。
    \begin{solution}
        据题意,
        \[
        6=\sqrt{x(x-3)} + \frac{8}{y^2} + \frac{y^2}{\sqrt{x(x-3)}} \ge 2|y|+\frac{8}{y^2} \Rightarrow 3y^2 \ge y^2|y|+4 \tag{1}
        \]
        当$y>0$,化简得
        \[
        y^3-3y^2+4=(y+1)(y-2)^2\le0
        \]
        由于$(y-2)^2\ge0$且$y+1>0$,故
        \[
        y=2 \Rightarrow x=-1,4
        \]
        经检验得(-1,2),(4,2)均不合题意;同理当$y<0$可解得
        \[
        y=-2,\Rightarrow x=-1,4
        \]
        经检验得,原方程式只有唯一解$(-1,-2)$
    \end{solution}

    \question 解不等式
    \[
    |x|^3 - 2x^2 - 4|x| + 3 < 0,
    \]
    \begin{solution}
        令 \( t = |x| \),不等式化为
        \[
        t^3 - 2t^2 - 4t + 3 < 0
        \]
        求根得 \( t = 3 \) 和 \( t = \dfrac{\sqrt{5}-1}{2}  \)。
        于是
        \[
        \frac{\sqrt{5}-1}{2} < |x| < 3
        \]
        解得
        \[
        x \in \left( -3, -\frac{\sqrt{5}-1}{2} \right) \cup \left( \frac{\sqrt{5}-1}{2}, 3 \right)
        \]
    \end{solution}

    \question 假设 $a>1$ 且 $x,y>0$ 满足
    \[
    (\log_a x)^2 + (\log_a y)^2 - \log_a(x^2y^2) \le 2, \quad \log_a y \ge 1.
    \]
    求 $\log_a(x^2y)$ 的取值区间。
    \begin{solution}
        设 $u = \log_a x$, $v = \log_a y$。不等式化为
        \[
        u^2 + v^2 - 2(u+v) \le 2, \quad v \ge 1.
        \]
        这是以 $(1,1)$ 为圆心、半径为 $2$ 的半圆的上半部分。我们要求 $2u+v$ 的取值范围。

        最小值在左下角 $(-1,1)$,得 $2u+v = -1$。

        最大值出现在斜率为 $-2$ 的直线与半圆切点,即联立
        \[
        v-1 = \frac{1}{2}(u-1), \quad (u-1)^2 + (v-1)^2 = 4.
        \]
        解得
        \[
        (u,v) = \left(1+\frac{4}{\sqrt{5}},\,1+\frac{2}{\sqrt{5}}\right) \implies 2u+v = 3 + 2\sqrt{5}.
        \]
        因此区间为 $[-1, 3+2\sqrt{5}]$。
    \end{solution}

    \question 已知不等式 $|x^2 + ax + 2| \ge |x + 1|$ 对任意 $x \in \mathbb{R}$ 恒成立,求实数 $a$ 的取值范围。
    \begin{solution}
        将不等式两边平方得
        \[
        (x^2 + ax + 2)^2 \ge (x + 1)^2
        \]
        \[
        (x^2 + ax + 2)^2 - (x + 1)^2 \ge 0
        \]
        \[
        [(x^2 + ax + 2) - (x + 1)][(x^2 + ax + 2) + (x + 1)] \ge 0
        \]
        \[
        [x^2 + (a - 1)x + 1][x^2 + (a + 1)x + 3] \ge 0
        \]
        因为两个二次式领导系数为正,两个因式需对任意 $x$ 恒非负。
        
        对二次式 $x^2 + (a - 1)x + 1$,
        \[
        \Delta_1  = (a - 1)^2 - 4 \le 0
        \Rightarrow -1 \le a \le 3
        \]
        对二次式 $x^2 + (a + 1)x + 3$,
        \[
        \Delta_2  = (a + 1)^2 - 12 \le 0
        \Rightarrow -1 - 2\sqrt{3} \le a \le -1 + 2\sqrt{3}
        \]
        综上所述,
        \[
        a \in [-1, -1 + 2\sqrt{3}]
        \]
    \end{solution}
        
    \question 求 $$|2|x-2|-5| \le 3$$ 的正整数解。
    \begin{solution}
        \textbf{情况一}:$\ |2|x-2|-5| = 3$  
        \[
        \Rightarrow
        \begin{cases}
        2|x-2| - 5 = 3 \Rightarrow |x-2| = 4 \Rightarrow 
        \begin{cases}
        x-2 = 4 \Rightarrow x = 6 \\
        x-2 = -4 \Rightarrow x = -2 \notin \mathbb{N}
        \end{cases} \\
        2|x-2| - 5 = -3 \Rightarrow |x-2| = 1 \Rightarrow 
        \begin{cases}
        x = 3 \\
        x = 1
        \end{cases}
        \end{cases}
        \]
        
        \textbf{情况二}:$\ |2|x-2|-5| = 2$  
        \[
        \Rightarrow
        \begin{cases}
        2|x-2| - 5 = 2 \Rightarrow |x-2| = \frac{7}{2} \notin \mathbb{Z} \\
        2|x-2| - 5 = -2 \Rightarrow |x-2| = \frac{3}{2} \notin \mathbb{Z}
        \end{cases}
        \]
        
        \textbf{情况三}:$\ |2|x-2|-5| = 1$  
        \[
        \Rightarrow
        \begin{cases}
        2|x-2| - 5 = 1 \Rightarrow |x-2| = 3 \Rightarrow 
        \begin{cases}
        x = 5 \\
        x = -1 \notin \mathbb{N}
        \end{cases} \\
        2|x-2| - 5 = -1 \Rightarrow |x-2| = 2 \Rightarrow 
        \begin{cases}
        x = 4 \\
        x = 0 \notin \mathbb{N}
        \end{cases}
        \end{cases}
        \]
        
        \textbf{情况四}:$\ |2|x-2|-5| = 0$  
        \[
        \Rightarrow |x-2| = \frac{5}{2} \notin \mathbb{Z}
        \]
        
        综上,正整数解为$x = 1, 3, 4, 5, 6$
    \end{solution}

    \question 若实数 $a$ 可使得对任意实数 $x$,不等式 
    \[
    |4x-3a|+|5x-4a|\ge a^{2}
    \]
    恒成立,求 $a$ 的范围。
    \begin{solution}
        记
        \[
        f(x)=|4x-3a| +|5x-4a|= 4\left( \Big|x-\frac{3}{4}a\Big|+ \Big|x-\frac{4}{5}a\Big|\right)+\Big|x-\frac{4}{5}a\Big|
        \]
        由三角不等式得
        \[
        f(x) \ge 4\Big| \Big(x-\frac{3}{4}a\Big)- \Big(x-\frac{4}{5}a\Big)\Big|+\Big|x-\frac{4}{5}a\Big|
        = \frac{|a|}{5}+ \Big|x-\frac{4}{5}a\Big|
        \]
        取 $x=\dfrac{4}{5}a$,得 $f(x)$ 的最小值为 $\dfrac{|a|}{5}$,欲使不等式对任意 $x$ 恒成立需有
        \[
        \frac{|a|}{5} \ge |a|^2 \Rightarrow |a|(5|a|-1)\le 0  \Rightarrow \quad 0\le |a|\le \frac{1}{5}
        \]
    \end{solution}

    \question 已知实数 $a,b$满足 $a+b\ge 3$, 试证 
    \[
    |a-2b^2|+|b-2a^2| \ge 6
    \]
    \begin{solution}
        由三角不等式,
        \begin{align*}
        \left| a-2b^2\right|+ \left| b-2a^2\right| 
        &\ge \left| a-2b^2+b-2a^2\right| \\
        &=\left| 2(a^2+b^2)-(a+b)\right| \\
        &\ge 2(a^2+b^2)-(a+b) \\
        &=(a+b)^2+(a-b)^2-(a+b) \\
        &\ge 3^2-3=6
        \end{align*}
        故原不等式得证。    
    \end{solution}

    \question 若二次实系数多项式函数 $f(x)$ 满足
    \[
    \begin{cases}
    -1 \le f(1) \le 3 \\
    6 \le f(2) \le 10 \\
    2 \le f(4) \le 24
    \end{cases}
    \]
    求 $f(7)$ 的最大值。
    \begin{solution}
        设 $f(x)=ax^2+bx+c$,则
        \[
        f(1)=a+b+c,\quad f(2)=4a+2b+c,\quad f(4)=16a+4b+c
        \]
        解得
        \[
        a = \frac{1}{3}f(1) - \frac{1}{2}f(2) + \frac{1}{6}f(4),\;
        b = -2f(1) + \frac{5}{2}f(2) - \frac{1}{2}f(4),\;
        c = \frac{8}{3}f(1) - 2f(2) + \frac{1}{3}f(4)
        \]
        因此
        \[
        f(7)=49a+7b+c = 5f(1)-9f(2)+5f(4)
        \]
        取 $f(1)=3,f(2)=6,f(4)=24$,得
        \[
        f(7)_{\max} =5\cdot 3 - 9\cdot 6 +5\cdot 24 =81
        \]
    \end{solution}

    \question 若 $a<b$,已知对任意实数 $x$,均有 $ax^2+bx+c\ge 0$,且 $m<\dfrac{4a+3b+2c}{b-a}$ 恒成立,求实数 $m$ 的范围。
    \begin{solution}
        由 $ax^2+bx+c\ge0\ (\forall x)$,可知二次项系数满足
        \[
        a>0,\quad b^2-4ac\le0
        \]
        写成
        \[
        \frac{4a+3b+2c}{b-a}
        =\frac{4\cdot\frac{a}{b}+3+2\cdot\frac{c}{b}}{1-\frac{a}{b}}
        \]
        记 $t=\dfrac{a}{b}$,则 $0<t<1$。由 $b^2-4ac\le0$ 得 $\dfrac{c}{b}\ge \dfrac{b}{4a}=\dfrac{1}{4t}$,于是
        \[
        \frac{4a+3b+2c}{b-a}
        \ge \frac{4t+3+2\cdot\frac{1}{4t}}{1-t}
        = \frac{4t+3+\frac{1}{2t}}{1-t}
        \]
        记
        \[
        k=\frac{4t+3+\frac{1}{2t}}{1-t}
        \]
        其中$0<t<1$,整理得关于 $t$ 的二次方程
        \[
        (2k+8)t^2+(6-2k)t+1=0
        \]
        为使该方程在 $0<t<1$ 上有解,判别式必须非负,
        \[
        (6-2k)^2-4(2k+8)\ge0 \Rightarrow k^2-8k+1 \ge 0 
        \]
        解得
        \[
        k \ge 4+\sqrt{15}>0
        \]
        于是对任意符合条件的 $a,b,c$,都有
        \[
        \frac{4a+3b+2c}{b-a}\ge 4+\sqrt{15}
        \]
        由题意 $m$ 满足 $m<\dfrac{4a+3b+2c}{b-a}$ 恒成立,故 
        \[
        m<4+\sqrt{15}
        \]
    \end{solution}

    \question 已知 $a>0, b>0, a+b=1$,证明不等式
    \[
    \frac{1}{a} + \frac{1}{b} + \frac{1}{ab} \ge 8
    \]
    \begin{solution}
        由 AM-GM 不等式,
        \begin{align*}
        \frac{1}{a} + \frac{1}{b} + \frac{1}{ab} &= \frac{1}{a} + \frac{1}{b} + \frac{a+b}{ab} \\
        &= 2\left(\frac{1}{a} + \frac{1}{b}\right) \\
        &= 2 \cdot \frac{a+b}{ab} \\
        &= 4 + 2\left(\frac{a}{b} + \frac{b}{a}\right) \\
        &\ge 4 + 2\cdot 2 = 8
        \end{align*}
    \end{solution}

    \question 设 \(a,b\) 均为正数,求
    \[
    f(a,b)=ab (72 - 3a - 4b)
    \]
    的最大值。
    \begin{solution}
        由 AM-GM 不等式,
        \begin{align*}
        f(a,b) &= \frac{1}{12} \cdot 3a \cdot 4b \cdot (72 - 3a - 4b) \\
        &\leq \frac{1}{12}\left(\frac{3a+4b+72-3a-4b}{3}\right)^3 = 1152        
        \end{align*}
        当$a=b=\frac{1}{2}$时等号成立。
    \end{solution}

    \question 设 $a,b,c$ 为正实数且满足 $a+b^{2}+c^{3}=11$,求 $abc$ 的最大值。
    \begin{solution}
        由 AM-GM 不等式,
        \begin{align*}
            11&=\frac{a}{6}+\frac{a}{6}+\frac{a}{6}+\frac{a}{6}+\frac{a}{6}+\frac{a}{6}+\frac{b^{2}}{3}+\frac{b^{2}}{3}+\frac{b^{2}}{3}+\frac{c^{3}}{2}+\frac{c^{3}}{2}\\
            &=11\sqrt[11]{\left(\frac{a}{6}\right)^6\left(\frac{b^2}{3}\right)^3\left(\frac{c^3}{2}\right)^2} = 11\sqrt[11]{\frac{(abc)^6}{6^6\cdot 27\cdot 4}}
        \end{align*}
        即
        \[
        abc \le 6\sqrt{3}\,\sqrt[3]{2}
        \]
        当且仅当
        \[
        \frac{a}{6}=\frac{b^2}{3}=\frac{c^3}{2}=1,
        \]
        即 $a=6,\ b=\sqrt{3},\ c=\sqrt[3]{2}$ 时等号成立。
    \end{solution}

    \question 实数 $x, y, z$ 满足 $x^2 + y^2 + z^2 = 1$,求 $\sqrt{2}xy + yz$ 的最大值。
    \begin{solution}
        由 AM-GM 不等式,
        \[
        \frac{\sqrt{3}}{2} x^2 + \frac{1}{\sqrt{3}} y^2 \ge 2\sqrt{\frac{1}{2} x^2 y^2} = \sqrt{2} xy,
        \]
        \[
        \frac{1}{2\sqrt{3}} y^2 + \frac{\sqrt{3}}{2} z^2 \ge 2\sqrt{\frac{1}{4} y^2 z^2} = yz.
        \]
        将两式相加得
        \[
         \sqrt{2} xy + yz \le \frac{\sqrt{3}}{2}(x^2 + y^2 + z^2) = \frac{\sqrt{3}}{2}
        \]
    \end{solution}

    \question 设 $x,y,z$ 为不全为 $0$ 的实数,求分式
    \[
    \frac{xy+2yz}{x^{2}+y^{2}+z^{2}}
    \]
    的最大值。
    \begin{solution}
        由 AM-GM 不等式,
        \[
        x^2+\frac{1}{5}y^2 \ge 2sqrt{\frac{1}{5}x^2y^2}=\frac{2}{\sqrt{5}} xy
        \]
        同理,
        \[
        \frac{4}{5}y^2 + z^2 \ge 2\sqrt{\frac{4}{5}y^2 z^2} = \frac{4}{\sqrt{5}} yz 
        \]
        两式相加得
        \[
        x^2 + y^2 + z^2 \ge \frac{2}{\sqrt{5}} (xy + 2 yz)
        \quad \Rightarrow \quad \frac{xy+2yz}{x^2+y^2+z^2} \le \frac{\sqrt{5}}{2}.
        \]
    \end{solution}

    \question 求
    \[
    f(x)=\frac{x^4 + 2x^3 + 3x^2 + 2x + 10}{x^2 + x + 1}
    \]
    的最小值。
    \begin{solution}
        由 AM-GM 不等式,
        \begin{align*}
        f(x)&=x^2 + x + 1 + \frac{9}{x^2 + x + 1} \\
        &\geq 2 \sqrt{(x^2 + x + 1)\cdot  \frac{9}{x^2 + x + 1}}  =6
        \end{align*}
        当 $(x^2 + x + 1)^2=9$即$x=2$或$x=-1$时,$\ f(x)$ 取得最大值 $6$
    \end{solution}

    \question 求函数 $$y=\frac{4^{x}+1}{2^{x}+1}$$ 的最小值。
    \begin{solution}
        由 AM-GM 不等式,
        \begin{align*}
        y=\frac{4^{x}-1}{2^{x}+1}+\frac{2}{2^{x}+1}
        &=2^{x}+1+\frac{2}{2^{x}+1}-2\\
        &\ge2\sqrt{(2^{x}+1)\cdot\frac{2}{2^{x}+1}}-2=2\sqrt{2}-2
        \end{align*}

        当 $(2^x+1)^2=2$即$x=\log_{2}(\sqrt2-1)$时,$\ y$ 取得最大值 $2\sqrt{2}-2$
    \end{solution}
    
    \question 求函数 $$f(x)=\log 2 \log 5-\log 2x \log 5x$$ 的最大值。 
    \begin{solution}
        \begin{align*}
            f(x)&=\log 2\log 5-(\log 2+\log x)(\log 5+\log x)\\
            &=-(\log 2 + \log 5)\log x - \log^2 x \\
            &=-\log x - \log^2 x \\
            &=-\left(\log x + \frac{1}{2}\right)^2 + \frac{1}{4} \le \frac{1}{4}
        \end{align*}
        当 $\log x = -\dfrac{1}{2}$即 $x=\dfrac{\sqrt{10}}{10}$ 时,$\ f(x)$ 取得最大值 $\dfrac{1}{4}$.
    \end{solution}

    \question 求函数 $$f(x,y) = x^2 + 2xy + 2y^2 + 2x + 4y + 5$$ 的最小值。
    \begin{solution}
        配方求最小值:
        \begin{align*}
        f(x,y) &= x^2 + 2xy + 2y^2 + 2x + 4y + 5 \\
        &= x^2 + 2(y+1)x + (y+1)^2 + y^2 + 2y + 4 \\
        &= [x+(y+1)]^2 + (y+1)^2 + 3 \ge 3
        \end{align*}
        当 $x=0, y=-1$ 时$f(x,y)$取到最小值 $3$。
    \end{solution}
    
    \question \(x,y \in \mathbb{R}\),求
    \[
    x^4 + 4x^3 + 8x^2 + 4xy + 6x + 4y^2 + 10
    \]
    的最小值。 
    \begin{solution}
        先配$y$项,写成\[
        f(x,y)=(x+2y)^2 + x^4 + 4x^3 + 7x^2+ 6x + 10
        \]
        再用火眼金睛注意到\[
        x^4 + 4x^3 + 7x^2+ 6x + 2=(x + 1)^2 (x^2 + 2 x + 2)    \]故原式为\[
        (x+2y)^2+(x + 1)^2 (x^2 + 2 x + 2)+8 \geq 8
        \]
        当$x=-1,y=\dfrac12$时,$\ f(x,y)$ 取得最小值 $8$.
    \end{solution}

    \question 已知 $x,y>0$, 证明
    \[
    \sqrt{x^2+y^2} > \sqrt[3]{x^3+y^3}
    \]
    \begin{solution}
        发现$\sqrt{x^2+y^2} > \sqrt[3]{x^3+y^3}$等价于
        \[
        x^6+3x^4y^2+3x^2y^4+y^6 > x^6+2x^3y^3+y^6 \Leftrightarrow x^2y^2(3x^2-2xy+3y^2) > 0
        \]
        其中关于$x$的两次函数$3x^2-2xy+3y^2$判别式为
        \[
        \Delta = 4y^2-4(3)(3)(y^2) =-32y^2< 0
        \]
        故$3x^2-2xy+3y^2 > 0$,得证
        \[
        x^2y^2(3x^2-2xy+3y^2) > 0
        \]
    \end{solution}

    \question 已知实数 $a,b,c$ 满足 $c<b<a,a+b+c=1,a^2+b^2+c^2=1$,求证 
    \[
    1<a+b<\frac{4}{3}.
    \]
    \begin{solution}
        由 $a^2+b^2+c^2=1,a+b+c=1$ 可得
        \[
        ab = \frac{(a+b)^2-(a^2+b^2)}{2} 
        = \frac{(1-c)^2-(1-c^2)}{2} 
        = c^2-c.
        \]
        所以 $a+b=1-c$,且 $a,b$ 是方程
        \[
        x^2-(1-c)x+c^2-c=0
        \]
        的两个相异实根,其中判别式为
        \[
        \Delta=(1-c)^2-4(c^2-c) >0 \Rightarrow -\frac{1}{3}<c<1.
        \]
        又有
        \[
        (c-a)(c-b)=c^2-c(a+b)+ab
        = c^2-c(1-c)+c^2-c
        = 3c^2-2c.
        \]
        因为 $c<b<a$,所以 $c-a<0,\;c-b<0$,故 $(c-a)(c-b)>0$,即
        \[
        3c^2-2c>0 \;\;\Longrightarrow\;\; c<0 \;\; \text{或} \;\; c>\frac{2}{3}.
        \]
        综合 $-\frac{1}{3}<c<1$ 与上述条件,得到
        \[
        -\frac{1}{3}<c<0 \quad \text{或} \quad \frac{2}{3}<c<1.
        \]
        若 $\dfrac{2}{3}<c<1$,则
        \[
        0<a+b<\frac{1}{3}
        \]
        但由 $c<b<a$ 得 $a+b>c>\dfrac{2}{3}$,矛盾,因此只能有 $-\dfrac{1}{3}<c<0$,则
        \[
        1<a+b<\frac{4}{3}
        \]
    \end{solution}

    \question 已知$a,b,c$为正实数,求
    \[
    \left\lfloor \frac{a+b}{c} \right\rfloor + \left\lfloor \frac{b+c}{a} \right\rfloor + \left\lfloor \frac{c+a}{b} \right\rfloor
    \]
    的最小值。
    \begin{solution}
        由$\lfloor x\rfloor > x-1,\;x \notin \mathbb{Z}$及AM-GM不等式,
        \begin{align*}
        S=\left\lfloor \frac{a+b}{c} \right\rfloor + \left\lfloor \frac{b+c}{a} \right\rfloor + \left\lfloor \frac{c+a}{b} \right\rfloor &> \frac{a}{b} + \frac{b}{a} + \frac{b}{c} + \frac{c}{b} + \frac{c}{a} + \frac{a}{c}-3 \\
        &\ge 2+2+2-3 =3
        \end{align*}
        由于$S \in \mathbb{Z}$,最小值为$S_{\min}=4$,例如可在$a=6,b=8,c=9$时取得。
    \end{solution}

    \question 已知$x>0$,求
    \[
    x\lfloor x\rfloor + \left\lfloor\frac{1}{x}\right\rfloor + x + \frac{1}{x} + x\lceil x\rceil + \left\lceil\frac{1}{x}\right\rceil
    \]
    的最小值。
    \begin{solution}
        设
        \[
        f(x)=x\lfloor x\rfloor + \left\lfloor\frac{1}{x}\right\rfloor + x + \frac{1}{x} + x\lceil x\rceil + \left\lceil\frac{1}{x}\right\rceil
        \]
        观察到$f(1)=6$,若$x > 1$,即$0 < \dfrac{1}{x} \le 1$,
        \[
        x\lfloor x\rfloor + \left\lfloor\frac{1}{x}\right\rfloor \ge x >1,\quad
        x\lceil x\rceil + \left\lceil\frac{1}{x}\right\rceil \ge 2x+1 >3,\quad
        x + \frac{1}{x} > 2
        \]
        于是当$x > 1,f(x) > 6 =f(1)$。若$0 < x \le \dfrac{1}{2}$,即$2 \le \dfrac{1}{x}$
        \[
        x\lfloor x\rfloor + \left\lfloor\frac{1}{x}\right\rfloor \ge 2, \quad
        x\lceil x\rceil + \left\lceil\frac{1}{x}\right\rceil \ge x + 2 > 2, \quad
        x + \frac{1}{x} > 2  
        \]
        于是当$0 < x \le \dfrac{1}{2},f(x) > 6 =f(1)$。若$\dfrac{1}{2} < x < 1$,即$1 < \dfrac{1}{x} < 2$,
        \[
        f(x)=1+x+\frac{1}{x}+x+2=3+2x+\frac{1}{x} \ge 3+2\sqrt{2}
        \]
        等号成立当且仅当
        \[
        2x=\frac{1}{x}
        \]
        即$x=\dfrac{\sqrt{2}}{2}$。由于$3+2\sqrt{2}<6,f(x)$的最小值为$3+2\sqrt{2}$。
    \end{solution}

    \question 若实数 $x, y$ 满足 $4x^2 - 2xy + 2y^2 = 1$, 则 $3x^2 + xy + y^2$ 的最大值与最小值的和为
    \begin{solution}
        \textcolor{red}{(待解)}
    \end{solution}

    \question 实数 $x,y,z,t$ 满足
    \[
    x^2+y^2=16,\quad z^2+t^2=25,\quad xt-yz=20,
    \]
    求 $xz$ 的最大值。
    \begin{solution}
        由柯西不等式,
        \[
        16\cdot 25=(x^2+y^2)(t^2+(-z)^2)\ge (xt-yz)^2=20^2
        \]
        此时等号成立,可解得
        \[
        \frac{x}{t}=\frac{y}{-z}=\frac{4}{5} \Rightarrow xz=-\frac{5}{4}xy
        \]
        由 $x^2+y^2=16$ 可得
        \[
        |xy|\le \frac{x^2+y^2}{2}=8
        \]
        所以
        \[
        -10= -\frac{5}{4}\cdot 8 \le xz \le -\frac{5}{4}\cdot(-8) =10
        \]
        因此 $x z$ 的最大值为$10$。
    \end{solution}

    \question 设 \(a_1,a_2,\dots,a_{18}\) 均为大于 1 的实数,求  
    \[
    \frac{\log_{a_1} 2023 + \log_{a_2} 2023 + \cdots + \log_{a_{18}} 2023}{\log_{a_1 a_2 \cdots a_{18}} 2023}
    \]
    的最小值。
    \begin{solution}
        利用换底公式,原式即
        \[
        \frac{\sum_{i=1}^{18} \frac{1}{\log_{2023} a_i}}{\frac{1}{\sum_{i=1}^{18} \log_{2023} a_i}} 
        = \left(\sum_{i=1}^{18} \frac{1}{\log_{2023} a_i}\right) \left(\sum_{i=1}^{18} \log_{2023} a_i\right).
        \]
        由柯西不等式,
        \[
        \left(\sum_{i=1}^{18} \log_{2023} a_i\right) \left(\sum_{i=1}^{18} \frac{1}{\log_{2023} a_i}\right) \ge \left(\sum_{i=1}^{18} 1\right)^2 = 18^2 = 324.
        \]
    \end{solution}

    \question 存在 2017 个正实数 \( x_1, x_2, \ldots, x_{2017} \)使得
    \[
    \sum_{i=1}^{2017} x_i = \sum_{i=1}^{2017} \frac{1}{x_i} = 2018,
    \]
    求
    \[
    x_1 + \frac{1}{x_1}
    \]
    的最大可能值。
    \begin{solution}
        由柯西不等式,
        \[
        (2018 - x_1)\left(2018 - \frac{1}{x_1} \right)=\left(\sum_{i=2}^{2017} x_i\right)\left(\sum_{i=2}^{2017} \frac{1}{x_i}\right)  \geq 2016^2.
        \]
        展开左边得
        \[
        2018^2 - 2018\left(x_1 + \frac{1}{x_1} \right) + 1 \geq 2016^2.
        \]
        于是
        \[
        x_1 + \frac{1}{x_1} \leq \frac{8069}{2018}.
        \]
        等号成立当且仅当 $x_i = \dfrac{1}{x_i}$,即$x_i=1,i=2,3,\dots,2017$
    \end{solution}

    \question 已知 $a, b, c$ 为正数,且 $a+b+c=1$。求 $$ \left(\frac{1}{a}-1\right)\left(\frac{1}{b}-1\right)\left(\frac{1}{c}-1\right)$$ 的最小值。
    \begin{solution}
    由AM-GM 不等式,
        \begin{align*}
        \left(\frac{1}{a}-1\right)\left(\frac{1}{b}-1\right)\left(\frac{1}{c}-1\right)
        &= \left(\frac{1-a}{a}\right)\left(\frac{1-b}{b}\right)\left(\frac{1-c}{c}\right)\\[1mm]
        &= \left(\frac{b+c}{a}\right)\left(\frac{a+c}{b}\right)\left(\frac{a+b}{c}\right)\\[1mm]
        &\ge \left(\frac{2\sqrt{bc}}{a}\right)\left(\frac{2\sqrt{ac}}{b}\right)\left(\frac{2\sqrt{ab}}{c}\right) =8 
        \end{align*}
        当且仅当 $a=b=c=\dfrac{1}{3}$ 时等号成立。
    \end{solution}

    \question 已知 $a,b,c$ 为三角形三边,证明不等式
    \[
    \frac{a}{b+c-a} + \frac{b}{c+a-b} + \frac{c}{a+b-c} \ge 3
    \]
    \begin{solution}
        发现
        \begin{align*}
        \frac{a}{b+c-a} + \frac{b}{c+a-b} + \frac{c}{a+b-c} &\ge 3 \\
        \iff \left(\frac{2a}{b+c-a}+1\right) + \left(\frac{2b}{c+a-b}+1\right) + \left(\frac{2c}{a+b-c}+1\right) &\ge 9 \\
        \iff (a+b+c) \left(\frac{1}{b+c-a} + \frac{1}{c+a-b} + \frac{1}{a+b-c}\right) &\ge 9
        \end{align*}
        由于 
        \[
        (b+c-a)+(c+a-b)+(a+b-c) = a+b+c,
        \]
        且
        \[
        b+c-a>0,\;c+a-b>0,\;a+b-c>0
        \]
        由AM-HM不等式或柯西不等式,原不等式得证。
    \end{solution}

    \question 设 \(x, y, z\) 为正实数且 \(xyz = 1\),证明:
    \[
    \frac{x}{y} + \frac{y}{z} + \frac{z}{x} \geq x + y + z.
    \]
    \begin{solution}
        由AM-GM 不等式,
        \[
        \frac{1}{3} \left( \frac{x}{y} + \frac{x}{y} + \frac{y}{z} \right) \geq \sqrt[3]{\frac{x^2}{yz}} = x,
        \]
        \[
        \frac{1}{3} \left( \frac{y}{z} + \frac{y}{z} + \frac{z}{x} \right) \geq \sqrt[3]{\frac{y^2}{xz}} = y, 
        \]
        \[
        \frac{1}{3} \left( \frac{z}{x} + \frac{z}{x} + \frac{x}{y} \right) \geq \sqrt[3]{\frac{z^2}{xy}} = z. 
        \]
        三式相加得
        \[
        \frac{x}{y} + \frac{y}{z} + \frac{z}{x} \geq x + y + z.
        \]
        故证毕。
    \end{solution}
    
    \question 若 $x^2+y^2+z^2=1$,求 $x+2y+3z$ 的最大值。
    \begin{solution}
        由柯西不等式,
        \[
        (x^2+y^2+z^2)(1^2+2^2+3^2) \ge (x+2y+3z)^2
        \Rightarrow x+2y+3z \le \sqrt{14}
        \]
        当且仅当 $(x,y,z)=\dfrac{1}{\sqrt{14}}(1,2,3)$ 时等号成立。
    \end{solution}

    \question 若 $x,y,z \in \mathbb{R}^+$ 且 $x+y+z=3$,求 $\dfrac{yz+4xz+9xy}{xyz}$ 的最小值。
    \begin{solution}
        由柯西不等式,
        \[
        \frac{x^2}{1^2} + \frac{y^2}{2^2} + \frac{z^2}{3^2} \ge \frac{(x+y+z)^2}{1+4+9} = \frac{9}{14}.
        \]
        当且仅当 $(x,y,z)=\left(\dfrac{1}{2},1,\dfrac{3}{2}\right)$ 时等号成立。
    \end{solution}

    \question 若 $a,b,c \in \mathbb{R}^+$ 且 $a+b+c=6$,求
    \[
    \left(a+\frac{1}{b}\right)^2 + \left(b+\frac{1}{c}\right)^2 + \left(c+\frac{1}{a}\right)^2
    \]
    的最小值。
    \begin{solution}
        由柯西不等式,
        \begin{align*}
        \left(a+\frac{1}{b}\right)^2 + \left(b+\frac{1}{c}\right)^2 + \left(c+\frac{1}{a}\right)^2 
        &\ge \frac{\left(a+b+c + \frac{1}{a}+\frac{1}{b}+\frac{1}{c}\right)^2}{1+1+1} \\
        &\ge \frac{\left(6 + \frac{9}{a+b+c}\right)^2}{1+1+1} = \frac{75}{4}
        \end{align*}
        等号成立当且仅当 $a=b=c=2$。
    \end{solution}

    \question 已知实数$a,b,c,d$满足 
    \[
    ab + bc + cd = 8,\quad b^{2} + c^{2} = 2,
    \]
    试求 $a^{2} + d^{2}$ 的最小值。
    \begin{solution}
        首先有
        \[
        b^2+c^2 \ge 2bc \Rightarrow bc \le 1
        \]
        由柯西不等式,
        \[ 
        (a^2+d^2)(b^2+c^2) \ge (ab+cd)^2 
        \]
        即
        \[ 
        (a^2+d^2) \ge \frac{(ab+cd)^2}{b^2+c^2} = \frac{(8-bc)^2}{2} \ge \frac{(8-1)^2}{2} = \frac{49}{2} 
        \]
        当 $bc=1$ 且 $\dfrac{a}{c} = \dfrac{d}{b}$, 即 $b=c=-1,a=d=-\dfrac{7}{2}$ 时等号成立,此时$a^2+d^2$ 取最小值 $\dfrac{49}{2}$。
    \end{solution}

    \question 若 $3a+4b=15$,求 $\sqrt{a^2+b^2}$ 的最小值。
    \begin{solution}
        由柯西不等式,
        \[
        5\sqrt{a^2+b^2} = \sqrt{3^2+4^2}\cdot\sqrt{a^2+b^2} \ge |3a+4b| = 15 \implies \sqrt{a^2+b^2} \ge 3
        \]
        当且仅当 $a=\dfrac{9}{5},b=\dfrac{12}{5}$ 时等号成立。
    \end{solution}

    \question 若 $abcd=1$,求 $(1+a)(1+b)(1+c)(1+d)$ 的最小值。
    \begin{solution}
        由 AM–GM不等式,
        \[
        (1+a)(1+b)(1+c)(1+d) \ge (2\sqrt{a})(2\sqrt{b})(2\sqrt{c})(2\sqrt{d}) =16
        \]
        等号成立当且仅当 $a=b=c=d=1$。
    \end{solution}

    \question 若 $x\in \mathbb{R}$,求 
    \[
    \frac{x^2+2-\sqrt{x^4+4}}{x}
    \] 
    的最大值。
    \begin{solution}
        设$x>0$,由AM–GM不等式,
        \begin{align*}
        \frac{x^2+2-\sqrt{x^4+4}}{x} 
        &= x+\frac{2}{x}-\sqrt{x^2+\frac{4}{x^2}}\\
        &= \sqrt{x^2+\frac{4}{x^2}+4}-\sqrt{x^2+\frac{4}{x^2}}\\
        &=\frac{4}{\sqrt{x^2+\frac{4}{x^2}+4}+\sqrt{x^2+\frac{4}{x^2}}}\\
        &\le \frac{4}{\sqrt{4+4}+\sqrt{4}} = 2\sqrt{2}-2
        \end{align*}
        当且仅当 $x=\sqrt{2}$ 时等号成立。
    \end{solution}

    \question 若 $a,b$ 为正实数且满足 $ab(a+b)=2000$,求 $\dfrac{1}{a} + \dfrac{1}{b} + \dfrac{1}{ab}$ 的最小值。
    \begin{solution}
        由 AM–GM不等式,
        \[
        2000=ab(a+b)\le \left(\frac{a+b}{2}\right)(a+b) \Rightarrow a+b \ge 20
        \]
        再由 AM–GM不等式,
        \begin{align*}
        \frac{1}{a} + \frac{1}{b} + \frac{1}{ab} 
        &= \frac{1}{2a}+\frac{1}{2a}+\frac{1}{2b}+\frac{1}{2b}+\frac{1}{a+b} \\
        &\ge 5\sqrt[5]{\frac{1}{16a^2b^2(a+b)}} \\
        &= 5 \sqrt[5]{\frac{a+b}{16a^2b^2(a+b)^2}} \\
        &\ge 5 \sqrt[5]{\frac{20}{16\cdot 2000^2}} = \frac{1}{4}
        \end{align*}
        等号成立当且仅当$a=b=10$。以下为错误示范:

        由 AM–GM不等式,
        \[
        \frac{1}{a} + \frac{1}{b} + \frac{1}{ab} \ge 3 \sqrt[3]{\frac{1}{ab(a+b)}}= 3 \sqrt[3]{\frac{1}{2000}}=\frac{3\sqrt[3]{4}}{20}
        \]
        但等号成立当且仅当$a=b=0$,不合题意。

        由 AM–GM不等式,
        \[
        \frac{1}{a} + \frac{1}{b} + \frac{1}{ab} = \frac{a+b}{ab} +\frac{1}{ab} = \frac{(a+b)^2}{2000}+\frac{1}{2(a+b)}+\frac{1}{2(a+b)}
        \ge 3\sqrt[3]{\frac{1}{8000}} =\frac{3}{20}
        \]
        但等号成立当且仅当$a+b=10$且$ab=200$,解得$a,b$为非实数,不合题意。
    \end{solution}

    \question 若 $a, b, c$ 皆为正实数满足 $a+b+c=100$,求 $\sqrt{a} + \sqrt{ab} + \sqrt{abc}$ 的最大值。
    \begin{solution}
        由AM–GM不等式,
        \[
        \sqrt{a} + \sqrt{ab} + \sqrt{abc} = a + 2\sqrt{\frac{a}{4}\cdot b} + 3\sqrt{\frac{a}{12}\cdot \frac{b}{3} \cdot \frac{4c}{3}} \le a+\frac{a}{4}+b+\frac{a}{12}+\frac{b}{3} + \frac{4c}{3} = \frac{4}{3}(a+b+c) =\frac{400}{3}
        \]  
        等号成立当且仅当 $a = \dfrac{160}{21}, b = \dfrac{40}{21}, c = \dfrac{10}{21}$。
    \end{solution}

    \question 若 $x_1, x_2, x_3, x_4, x_5$ 为实数,且
    \[
    x_1+x_2+x_3+x_4+x_5=8, \quad x_1^2+x_2^2+x_3^2+x_4^2+x_5^2=16
    \]  
    求 $x_5$ 的最大值。
    \begin{solution}
        由柯西不等式,
        \[
        (x_1^2+x_2^2+x_3^2+x_4^2)(1+1+1+1) \ge (x_1+x_2+x_3+x_4)^2 \Rightarrow 4(16-x_5^2) \ge (8-x_5)^2
        \]  
        解得 
        \[
        0 \le x_5 \le \frac{16}{5}
        \]  
        当 $x_1=x_2=x_3=x_4=\dfrac{6}{5},x_5 = \dfrac{16}{5}$时等号成立。
    \end{solution}

    \question 若 $a,b,c$ 为正实数且 $a+b+c \le 9$,求 $(a^2+b^2+c^2)(2ab+2bc+2ca+5)$ 的最大值。
    \begin{solution}
        由AM–GM不等式,
        \begin{align*}
        (a^2+b^2+c^2)(2ab+2bc+2ca+5) 
        &\le \left(\frac{a^2+b^2+c^2 + 2ab+2bc+2ca+5}{2}\right)^2 \\
        &= \left(\frac{(a+b+c)^2+5}{2}\right)^2 \\
        &\le \left(\frac{9^2+5}{2}\right)^2 = 1849
        \end{align*}
        当$a^2+b^2+c^2=43,ab+bc+ca=19$时等号成立。
    \end{solution}

    \question $x,y,z$ 为正实数且满足 $x+y+z=91$,求
    \[
    \frac{yz}{x} + \frac{xz}{y} + \frac{xy}{z}
    \]
    的最小值。
    \begin{solution}
        由排序不等式,
        \[
        \sum_{cyc} \frac{xy}{z} \ge \sum_{cyc} \frac{xy}{y} = x+y+z=91
        \]
    \end{solution}

    \question 已知正实数 $a,b$ 满足 $a + b = 1$,求 
    \[
    \left(a + \dfrac{1}{a}\right)^{2} + \left(b + \dfrac{1}{b}\right)^{2}
    \]
    的最小值。
    \begin{solution}
        由柯西不等式或QM-AM不等式, 
        \[
        \left(a+\frac{1}{a}\right)^2+\left(b+\frac{1}{b}\right)^2
        \ge
        \frac12\left(a+b+\frac{1}{a}+\frac{1}{b}\right)^2=\frac12\left(1+\frac{1}{ab}\right)^2
        \]
        又由AM-GM不等式,
        \[
        ab\le\left(\frac{a+b}{2}\right)^2=\frac14
        \]
        故
        \[
        \left(a+\frac{1}{a}\right)^2+\left(b+\frac{1}{b}\right)^2
        \ge
        \frac12\cdot (1+4)^2=\frac{25}{2}
        \]
        当且仅当 $a=b=\dfrac12$ 时等号成立,此时有最小值 $\dfrac{25}{2}$。
    \end{solution}

    \question 已知 $x,y,z$ 为正实数,求 $\dfrac{(x^4+1)(y^4+1)(z^4+1)}{xy^2z}$ 的最小值。
    \begin{solution}
        由AM–GM不等式,
        \begin{align*}
        \frac{(x^4y^4+1)(y^4z^4+1)(z^4x^4+1)}{xy^2z}
        &=\left(x^3+\frac{1}{3x}+\frac{1}{3x}+\frac{1}{3x}\right)\left(y^2+\frac{1}{y^2}\right)\left(z^3+\frac{1}{3z}+\frac{1}{3z}+\frac{1}{3z}\right)\\
        &\ge 4\sqrt[3]{\frac{1}{27}}\cdot 2 \cdot 4\sqrt[3]{\frac{1}{27}} = \frac{32\sqrt{3}}{9}
        \end{align*}
    \end{solution}

    \question 已知$x \in [0,2]$,求 $\sqrt{x}+4\sqrt{1-\dfrac{x}{2}}$ 的最大值。
    \begin{solution}
        发现
        \[
        \sqrt{x}+4\sqrt{1-\frac{x}{2}}=\sqrt{x}+\sqrt{2^3}\cdot \sqrt{2-x}
        \]
        由柯西不等式,
        \[
        (\sqrt{x}+\sqrt{2^3}\cdot \sqrt{2-x})^2\le(1+8)(x+2-x)=18 \Leftrightarrow \sqrt{x}+4\sqrt{1-\frac{x}{2}} \le 3\sqrt{2}
        \]
        等号成立当且仅当$x=\dfrac{2}{9}$
    \end{solution}

    \question 若 $x,y$ 为正数, 且 
    \[
    x^2+\frac{y^2}{45}=1, 
    \]
    试求 
    \[
    \frac{2}{1-x}+\frac{75}{10-y}
    \] 
    的最小值。
    \begin{solution}
        由AM-GM不等式,
        \[
        \frac{1-x+\frac{x}{2}+\frac{x}{2}}{3} \ge \sqrt[3]{\frac{x^2(1-x)}{4}}
        \]
        于是
        \[
        \frac{2}{1-x} = \frac{x^2}{2}\cdot \frac{4}{x^2(1-x)}\ge \frac{x^2}{2}\cdot\left( \frac{3}{1-x+\frac{x}{2}+\frac{x}{2}}\right)^3 = \frac{27x^2}{2} \tag{1}
        \]
        由AM-GM不等式,
        \[
        \frac{y+(10-y)}{2} \ge \sqrt{y(10-y)}
        \]
        故
        \[
        \frac{75}{10-y}
        = \frac{75}{10}\left( \frac{y}{10-y}+1\right)
        \ge \frac{15}{2}\cdot \frac{y^2}{\left( \frac{y+(10-y)}{2}\right)^2}+\frac{15}{2}
        = \frac{15}{2}\cdot \frac{y^2}{25}+ \frac{15}{2} \tag{2}
        \]
        由 $(1),(2)$,
        \[
        \frac{2}{1-x}+\frac{75}{10-y}
        \ge \frac{27}{2}\left( x^2+\frac{y^2}{45}\right)+\frac{15}{2}= 21
        \]
        等号成立当且仅当$x=\dfrac{2}{3},y=5$。
    \end{solution}

    \question 若正实数$a,b,c$满足 $a+b+c=21$,求 $a^2+3b^2+5c^2$ 的最小值。
    \begin{solution}
        由排序不等式,
        \begin{align*}
        a^2+3b^2+5c^2 
        &= a^2+b^2+c^2 +2b^2+2c^2+2c^2 \\
        &\ge a^2+b^2+c^2 + 2ab+2ac+2bc = (a+b+c)^2 = 441
        \end{align*}
        当$(a,b,c)=(21,0,0),\left(\dfrac{21}{2},\dfrac{21}{2},0\right),(7,7,7)$时等号成立。
    \end{solution}

    \question 已知 $x,y$ 为实数,求 $x^6+y^6-54xy$ 的最小值。
    \begin{solution}
        由AM–GM不等式,
        \[
        \frac{x^6+y^6+27+27+27+27}{6} \ge \sqrt[6]{x^6y^6 \cdot 3^{12}} \Rightarrow x^6y^6 - 54xy \ge -108
        \]  
        当$x=y=\pm\sqrt{3}$时等号成立。
    \end{solution}

\question 若 $x,y,z$ 为正实数,且满足 $x+y+z=1$,求 $x(x+y)^2(y+z)^3(z+x)^4$ 的最大值。
\begin{solution}
    由AM–GM不等式,
    \[
    \frac{5(x+y+z)}{10}=\frac{x + 2(x+y) + 3(y+z) + 4\left(\frac{z+x}{2}\right)}{10} \ge \sqrt[10]{\frac{x(x+y)^2(y+z)^3(z+x)^4}{2^4}} 
    \]  
    故
    \[
    x(x+y)^2(y+z)^3(z+x)^4 \le \frac{1}{64}
    \]
    \textcolor{red}{等号成立$y=0,x=z=\frac{1}{2}$?}
\end{solution}

\question 设 $a,b,c,d$ 为正实数且 $abcd=1$。证明
\[
a^2+b^2+c^2+d^2+ab+ac+ad+bc+bd+cd \ge 10.
\]

\begin{solution}
由算术几何均值不等式(A.M.-G.M.),有
\[
\frac{a^2+b^2+c^2+d^2}{4} \ge \sqrt[4]{a^2b^2c^2d^2} = \sqrt[4]{(abcd)^2} = 1,
\]
因此
\[
a^2+b^2+c^2+d^2 \ge 4.
\]

同样地,对六个二次项之和应用 A.M.-G.M.:
\[
\frac{ab+ac+ad+bc+bd+cd}{6} \ge \sqrt[6]{(ab)(ac)(ad)(bc)(bd)(cd)}
= \sqrt[6]{a^3b^3c^3d^3}=\sqrt[6]{(abcd)^3}=1,
\]
于是
\[
ab+ac+ad+bc+bd+cd \ge 6.
\]

将两式相加得到
\[
a^2+b^2+c^2+d^2+ab+ac+ad+bc+bd+cd \ge 4+6=10,
\]
证毕。
\end{solution}

\question
设 $a$, $b$ 和 $c$ 为正实数,且满足 $a^2+b^2+c^2 = 1$。证明:
\[
\frac{1}{a^2} + \frac{1}{b^2} + \frac{1}{c^2} \ge 3 + \frac{2(a^3+b^3+c^3)}{abc}.
\]

\begin{solution}
考虑左边减右边:
\begin{align*}
&\frac{1}{a^2} + \frac{1}{b^2} + \frac{1}{c^2} - 3 - \frac{2(a^3+b^3+c^3)}{abc} \\
&= \frac{a^2+b^2+c^2}{a^2} + \frac{a^2+b^2+c^2}{b^2} + \frac{a^2+b^2+c^2}{c^2} - 3 - 2 \left(\frac{a^2}{bc} + \frac{b^2}{ca} + \frac{c^2}{ab}\right) \\
&= \left(1 + \frac{b^2+c^2}{a^2}\right) + \left(1 + \frac{a^2+c^2}{b^2}\right) + \left(1 + \frac{a^2+b^2}{c^2}\right) - 3 - 2 \left(\frac{a^2}{bc} + \frac{b^2}{ca} + \frac{c^2}{ab}\right) \\
&= \frac{b^2+c^2}{a^2} + \frac{a^2+c^2}{b^2} + \frac{a^2+b^2}{c^2} - 2 \left(\frac{a^2}{bc} + \frac{b^2}{ca} + \frac{c^2}{ab}\right) \\
&= a^2\left(\frac{1}{b^2} + \frac{1}{c^2}\right) + b^2\left(\frac{1}{c^2} + \frac{1}{a^2}\right) + c^2\left(\frac{1}{a^2} + \frac{1}{b^2}\right) - 2 \left(\frac{a^2}{bc} + \frac{b^2}{ca} + \frac{c^2}{ab}\right) \\
&= a^2\left(\frac{1}{b} - \frac{1}{c}\right)^2 + b^2\left(\frac{1}{c} - \frac{1}{a}\right)^2 + c^2\left(\frac{1}{a} - \frac{1}{b}\right)^2 \ge 0.
\end{align*}

等号成立当且仅当 $a = b = c$。
\end{solution}



    \question 若 $a,b,c$ 为实数且 $a+b+c=12$,且
    \[
    \frac{1}{a}+\frac{1}{b}+\frac{1}{c}+\frac{1}{abc}=1,
    \]  
    求 $abc-(a+2b-3c)$ 的最大值。
    \begin{solution}
        由
        \[
        \frac{1}{a}+\frac{1}{b}+\frac{1}{c}+\frac{1}{abc}=1,
        \] 
        则
        \[
        2abc=2ab+2bc+2ca+2=144-a^2-b^2-c^2+2
        \]
        所以
        \begin{align*}
        2abc - 2(a+2b-3c) 
        &= 146 - (a^2+2a)-(b^2+4b)-(c^2-6c) \\
        &= 160 - [(a+1)^2+(b+2)^2+(c-3)^2] \\
        &\le 160-3\left(\frac{a+1+b+2+c-3}{3}\right)^2=112        
        \end{align*}
        因此
        \[
        abc-(a+2b-3c) \le 56
        \]  
        等号成立当且仅当 $a=3, b=2, c=7$。
    \end{solution}

    \question 设 $a,b,c$ 为正实数,且 $a+b+c=1$,求
    \[
    \sqrt{a^2+b^2}+\sqrt{b^2+c^2}+\sqrt{c^2+a^2}
    \]
    的最小值。
    \begin{solution}
        由柯西不等式,
        \[
        (1^2+1^2)(a^2+b^2)\ge (a+b)^2 \Rightarrow \sqrt{a^2+b^2}\ge \frac{a+b}{\sqrt{2}},
        \]
        同理
        \[
        \sqrt{b^2+c^2}\ge \frac{b+c}{\sqrt{2}},\quad \sqrt{c^2+a^2}\ge \frac{c+a}{\sqrt{2}}.
        \]
        相加得
        \[
        \sqrt{a^2+b^2}+\sqrt{b^2+c^2}+\sqrt{c^2+a^2}
        \ge \frac{2(a+b+c)}{\sqrt{2}}=\sqrt{2}.
        \]
        当 $a=b=c=\dfrac13$ 时取等号,故最小值为 $\sqrt{2}$。
    \end{solution}

    \question 已知 $x>1,y>1,z>1$ 且 $\dfrac{1}{x}+\dfrac{1}{y}+\dfrac{1}{z}=2$, 试证
    \[
    \sqrt{x+y+z}\ge \sqrt{x-1}+\sqrt{y-1}+\sqrt{z-1}
    \]
    \begin{solution}
        首先有
        \[
        \frac{x-1}{x}+\frac{y-1}{y}+\frac{z-1}{z}
        =3-\left(\frac{1}{x}+\frac{1}{y}+\frac{1}{z}\right)=3-2=1
        \]
        由柯西不等式,
        \[
        \left(\frac{x-1}{x}+\frac{y-1}{y}+\frac{z-1}{z}\right)(x+y+z)
        \ge\left(\sqrt{x-1}+\sqrt{y-1}+\sqrt{z-1}\right)^2
        \]
        左式为 $1\cdot(x+y+z)=x+y+z$,因此
        \[
        \sqrt{x+y+z}\ge \sqrt{x-1}+\sqrt{y-1}+\sqrt{z-1}
        \]
        故证毕。
    \end{solution}

    \question 
    \begin{parts}
    \part 已知 $a,b,c>0$,证明
    \[
    \frac{a^{2}}{2a+b}+\frac{b^{2}}{2b+c}+\frac{c^{2}}{2c+a}\ge\frac{a+b+c}{3}
    \]
    \begin{solution}
        由柯西不等式,
        \[
        \left( \frac{a^2}{2a+b} + \frac{b^2}{2b+c} + \frac{c^2}{2c+a} \right) (2a+b+2b+c+2c+a) \ge (a+b+c)^2
        \]
        所以得证
        \[
        \frac{a^2}{2a+b} + \frac{b^2}{2b+c} + \frac{c^2}{2c+a} \ge \frac{a+b+c}{3}
        \]
    \end{solution}
    \part 已知 $a_{1},a_{2},b_{1},b_{2},c_{1},c_{2}>0$,证明
    \[
    (a_{1}^{3}+a_{2}^{3})(b_{1}^{3}+b_{2}^{3})(c_{1}^{3}+c_{2}^{3})\ge(a_{1}b_{1}c_{1}+a_{2}b_{2}c_{2})^{3}
    \]
    \begin{solution}
        令
        \[
        r_1 = \frac{a_2}{a_1}, \quad r_2 = \frac{b_2}{b_1}, \quad r_3 = \frac{c_2}{c_1}
        \]
        则
        \[
        (a_1^3+a_2^3)(b_1^3+b_2^3)(c_1^3+c_2^3) = a_1^3 b_1^3 c_1^3 (1+r_1^3)(1+r_2^3)(1+r_3^3)
        \]
        \[
        (a_1 b_1 c_1 + a_2 b_2 c_2)^3 = a_1^3 b_1^3 c_1^3 (1+r_1 r_2 r_3)^3
        \]
        由AM-GM不等式,
        \begin{align*}
            (1+r_1^3)(1+r_2^3)(1+r_3^3) 
            &= 1 + (r_1^3+r_2^3+r_3^3) + (r_1 r_2)^3+(r_2 r_3)^3+(r_3 r_1)^3 + r_1^3 r_2^3 r_3^3 \\
            &\ge 1 + 3 r_1 r_2 r_3 + 3 r_1^2 r_2^2 r_3^2 + r_1^3 r_2^3 r_3^3  \\
            &= (1+r_1 r_2 r_3)^3
        \end{align*}
        因此得证
        \[
        (a_1^3+a_2^3)(b_1^3+b_2^3)(c_1^3+c_2^3) \ge (a_1 b_1 c_1 + a_2 b_2 c_2)^3
        \]
    \end{solution}
    \end{parts}

    \question 设 \(a, b, c\) 均为正数,且 \(a+b+c=3\),证明
    \[
    \frac{a}{b^{2}+1}+\frac{b}{c^{2}+1}+\frac{c}{a^{2}+1} \ge \frac{3}{2}.
    \]
    \begin{solution}
        由不等式
        \[
        (b-1)^2 = b^2 + 1 - 2b \ge 0 \Rightarrow \frac{1}{b^2 + 1} \le \frac{1}{2b},
        \]
        因此
        \[
        \frac{a}{b^2 + 1} = a - \frac{ab^2}{b^2 + 1} \ge a - \frac{ab^2}{2b} = a - \frac{ab}{2}.
        \]
        同理得
        \[
        \frac{b}{c^2 + 1} \ge b - \frac{bc}{2}, \quad \frac{c}{a^2 + 1} \ge c - \frac{ac}{2}.
        \]
        又由柯西不等式,
        \[
        (a^2 + b^2 + c^2)(1^2 + 1^2 + 1^2) \ge (a + b + c)^2 \Rightarrow a^2 + b^2 + c^2 \ge 3.
        \]
        因为
        \[
        ab + bc + ca = \frac{(a + b + c)^2 - (a^2 + b^2 + c^2)}{2} \le \frac{9 - 3}{2} = 3
        \]
        故
        \[
        \frac{a}{b^2 + 1} + \frac{b}{c^2 + 1} + \frac{c}{a^2 + 1} \ge (a + b + c) - \frac{1}{2}(ab + bc + ca) \ge 3 - \frac{3}{2} = \frac{3}{2}.
        \]
        故证毕。
    \end{solution}

    \question 设 $a,b,c$ 为正实数,证明
    \[
    \sqrt{ab(a+b)} + \sqrt{bc(b+c)} + \sqrt{ca(c+a)} \le \frac{3}{2}\sqrt{(a+b)(b+c)(c+a)}.
    \]
    \begin{solution}
        由AM-GM不等式,
        \begin{align*}
        3&=\frac{b}{b+c} + \frac{c}{c+a} + \frac{a}{a+b} + \frac{c}{b+c} + \frac{b}{a+b} + \frac{a}{c+a} \\
        &\ge 2\left(\sqrt{\frac{ab}{(b+c)(c+a)}} + \sqrt{\frac{bc}{(a+b)(c+a)}} + \sqrt{\frac{ca}{(a+b)(b+c)}}\right)
        \end{align*}
        即得证
        \[
        \sqrt{ab(a+b)} + \sqrt{bc(b+c)} + \sqrt{ca(c+a)} \le \frac{3}{2}\sqrt{(a+b)(b+c)(c+a)}.
        \]
    \end{solution}

    \question 证明对任意实数$x, y, z$皆有 $$ \frac{y^{2}-x^{2}}{2x^{2}+1}+\frac{z^{2}-y^{2}}{2y^{2}+1}+\frac{x^{2}-z^{2}}{2z^{2}+1}\ge0 $$
    \begin{solution}
        由 AM-GM 不等式,
        \begin{align*}
            \frac{y^{2}-x^{2}}{2x^{2}+1}+\frac{z^{2}-y^{2}}{2y^{2}+1}+\frac{x^{2}-z^{2}}{2z^{2}+1}+\frac{3}{2}
            &=\frac{1}{2}\left(\frac{2y^{2}+1}{2x^{2}+1}+\frac{2z^{2}+1}{2y^{2}+1}+\frac{2x^{2}+1}{2z^{2}+1}\right) \\
            &\ge\frac{3}{2}\sqrt[3]{\frac{2y^{2}+1}{2x^{2}+1}\cdot \frac{2z^{2}+1}{2y^{2}+1}\cdot\frac{2x^{2}+1}{2z^{2}+1}}  =\frac{3}{2}
        \end{align*}
        故
        $$ \frac{y^{2}-x^{2}}{2x^{2}+1}+\frac{z^{2}-y^{2}}{2y^{2}+1}+\frac{x^{2}-z^{2}}{2z^{2}+1}\ge0 $$
    \end{solution}
    
    \question 设 $a>0, b>0, c>0$,求
    \[
    \frac{a+3c}{a+2b+c} + \frac{4b}{a+b+2c} - \frac{8c}{a+b+3c} + 17
    \]
    的最小值。
    \begin{solution}
        由于
        \[
        \begin{cases}
        a + 2b + c = x\\ a + b + 2c = y\\ a + b + 3c = z
        \end{cases}
        \Rightarrow
        \begin{cases}
        a = -x + 5y - 3z \\ b = x - 2y + z \\ c = -y + z
        \end{cases}
        \]
        故
        \begin{align*}
        &\frac{a+3c}{a+2b+c} + \frac{4b}{a+b+2c} - \frac{8c}{a+b+3c} + 17 \\[1mm]
        &= \frac{-x + 2y}{x} + \frac{4x - 8y + 4z}{y} - \frac{-8y + 8z}{z} + 17 \\
        &= \left(\frac{2y}{x} + \frac{4x}{y}\right) + \left(\frac{4z}{y} + \frac{8y}{z}\right) \\[1mm]
        &\geq 2\sqrt{\frac{2y}{x} \cdot \frac{4x}{y}} + 2\sqrt{\frac{4z}{y} \cdot \frac{8y}{z}} = 12\sqrt{2}
        \end{align*}
    \end{solution}

    \question 证明对所有自然数 $n$,有
    \[
    1 + \frac{2}{3n-2} \le \sqrt[n]{3} \le 1 + \frac{2}{n}.
    \]
    \begin{solution}
        由AM-GM不等式,
        \[
        \frac{n+2}{n} = \frac{3 + \overbrace{1 + \cdots + 1}^{n-1 \text{个1}}}{n} \ge \sqrt[n]{3\cdot 1^{n-1}} = \sqrt[n]{3} \Rightarrow \sqrt[n]{3} \le 1 + \frac{2}{n}
        \]
        且
        \[
        \frac{3n-2}{3n} = \frac{1/3 + \overbrace{1 + \cdots + 1}^{n-1 \text{个1}}}{n} \ge \sqrt[n]{(1/3)\cdot 1^{n-1}} = \frac{1}{\sqrt[n]{3}} \Rightarrow \sqrt[n]{3} \ge  1 + \frac{2}{3n-2}.
        \]
        综上得到
        \[
        1 + \frac{2}{3n-2} \le \sqrt[n]{3} \le 1 + \frac{2}{n}
        \]
    \end{solution}

    \question 已知 $0 \le x \le 1$, $n \in \mathbb{N}$,证明:
    \[
    0 \le x^n - x^{n+1} \le \frac{n^n}{(n+1)^{\,n+1}}.
    \]
    \begin{solution}
        由 $0 \le x \le 1$,有
        \[
        0 \le 1-x \quad \Rightarrow \quad 0 \le x^n(1-x) = x^n - x^{\,n+1} \tag{1}
        \]
        由AM-GM不等式,
        \[
        \frac{\overbrace{\frac{x}{n} + \cdots + \frac{x}{n}}^{n \text{ 个}} + (1-x)}{\,n+1} \ge \sqrt[n+1]{\left(\frac{x}{n}\right)^n (1-x)} 
        \]
        即
        \[
        \frac{n^n}{(n+1)^{\,n+1}} \ge x^n(1-x) = x^n - x^{\,n+1} \tag{2}
        \]
        由 (1) 与 (2), 得证
        \[
        0 \le x^n - x^{\,n+1} \le \frac{n^n}{(n+1)^{\,n+1}}.
        \]
    \end{solution}

    \question 已知 $\triangle ABC$ 的面积为 $27,R$ 为其外接圆半径,$r$ 为其内切圆半径。求 $Rr^{3}$ 的最大值。
    \begin{solution}
        设$\triangle ABC$边长为$a,b,c>0$,面积为
        \[
        [\triangle ABC]=\frac{1}{2}(a+b+c)r=\frac{abc}{4R}=27
        \]
        由AM-GM不等式,
        \[
        \frac{2\cdot27}{3r}=\frac{a+b+c}{3}\ge \sqrt[3]{abc} = \sqrt[3]{4R\cdot 27} \Rightarrow Rr^3 \le 54
        \]
        等号成立当且仅当$a=b=c$即$\triangle ABC$为等边三角形。
    \end{solution}

    \question 设实数 $x,y$ 满足 $\sin x + \cos y = 1$,求 $\cos x + \sin y$ 的最大值。
    \begin{solution}
        注意到
        \[
        (\sin x + \cos y)^2 + (\cos x + \sin y)^2 = 2 + 2\sin(x + y) \le 4
        \]
        由题设 $\sin x + \cos y = 1$,代入上式
        \[
        1^2 + (\cos x + \sin y)^2 \le 4 
        \Rightarrow \cos x + \sin y \le \sqrt{3}
        \]
        当 $x = \dfrac{\pi}{6},\; y = \dfrac{\pi}{3}$ 时,$\;\sin x + \cos y = \dfrac{1}{2} + \dfrac{1}{2} = 1,\cos x + \sin y = \dfrac{\sqrt{3}}{2} + \dfrac{\sqrt{3}}{2} = \sqrt{3}$ 取到最大值。
    \end{solution}

    \question 若 $x \in \mathbb{R} \setminus \left\{\frac{\pi}{2} + k\pi, k\pi \mid k \in \mathbb{Z} \right\}$,试求 
    \[
    f(x) = |\sin x + \cos x + \tan x + \cot x + \sec x + \csc x|
    \]
    的最小值。
    \begin{solution}
        设 $t = \cos x + \sin x $,有
        \[
        \sin x\cos x = \frac{t^2 - 1}{2}
        \]
        且
        \[
        \tan x + \cot x = \frac{\sin^2 x+\cos^2 x}{\sin x \cos x} = \frac{2}{t^2 - 1} 
        \]
        \[
        \sec x + \csc x = \frac{\sin x + \cos x}{\cos x \sin x} = \frac{2t}{t^2 - 1}
        \]
        故
        \[
        f(x) = g(t) = \left| t + \frac{2}{t^2 - 1} + \frac{2t}{t^2 - 1} \right| = \left| t - 1 + \frac{2}{t - 1} + 1 \right| \ge \left| -2\sqrt{2} + 1 \right| = 2\sqrt{2} - 1
        \]
    \end{solution}

    \question 试求 
    \[
    \sqrt{10-6\cos\theta}+\frac{1}{4}\sqrt{34-24\sqrt{2}\sin\theta}+\sqrt{19-2\sqrt{2}\cos\theta-8\sin\theta}
    \] 
    的最小值。
    \ifprintanswers
    \begin{figure}[H]
        \centering
        \includegraphics[width=0.5\linewidth]{images/image42.png}
    \end{figure}
    \fi
    \begin{solution}
        发现
        \[
        \sqrt{10 - 6\cos\theta} = \sqrt{(\cos\theta - 3)^2 + \sin^2\theta}
        \]
        \[
        \frac{1}{4}\sqrt{34 - 24\sqrt{2} \sin\theta} = \sqrt{\cos^2\theta - \left(\sin\theta - \frac{3}{4}\sqrt{2} \right)^2}
        \]
        \[
        \sqrt{19 - 2\sqrt{2} \cos\theta - 8\sin\theta} = \sqrt{(\cos\theta - \sqrt{2})^2 + (\sin\theta - 4)^2}
        \]
        构造坐标系,设 $P(\cos\theta, \sin\theta)$ 在单位圆上,$A(3,0),\ B(0,\frac{3\sqrt{2}}{4}),\ C(\sqrt{2},4)$,原式即为 $$\overline{PA} + \overline{PB} + \overline{PC}$$
        直线 $AB$方程式为
        \[
        \sqrt{2}x + 4y = 3\sqrt{2},
        \]
        故圆心 $O$ 到 $AB$ 的距离为 $1$,且当 $P = \left(\dfrac{1}{3}, \dfrac{2\sqrt{2}}{3} \right),OPC$ 共线,于是$\overline{PA} + \overline{PB} + \overline{PC}$最小值为
        \[
        \overline{AB} + \overline{OC} - 1
        = \sqrt{9 + \frac{18}{16}} + \sqrt{18} - 1 = \frac{21}{4} \sqrt{2} - 1
        \]
    \end{solution}

    \question 若锐角 $A,B,C$ 满足 $\sin^2 A + \sin^2 B + \sin^2 C = 2$, 求 
    \[
    \frac{1}{\sin^2 A\cos^4 B} + \frac{1}{\sin^2 B \cos^4 C} + \frac{1}{\sin^2 C \cos^4 A}
    \]
    的最小值。
    \begin{solution}
        由柯西不等式,
        \[
        \begin{aligned}
        &\frac{1}{\sin^2 A \cos^4 B} + \frac{1}{\sin^2 B \cos^4 C} + \frac{1}{\sin^2 C \cos^4 A} \\
        &= \frac{1}{2} \left( \frac{1}{\sin^2 A \cos^4 B} + \frac{1}{\sin^2 B \cos^4 C} + \frac{1}{\sin^2 C \cos^4 A} \right) (\sin^2 A + \sin^2 B + \sin^2 C) \\
        &\ge \frac{1}{2} \left( \frac{1}{\cos^2 B} + \frac{1}{\cos^2 C} + \frac{1}{\cos^2 A} \right)^2 \\
        &= \frac{1}{2} \left[ \left( \frac{1}{\cos^2 A} + \frac{1}{\cos^2 B} + \frac{1}{\cos^2 C} \right) (\cos^2 A + \cos^2 B + \cos^2 C) \right]^2 \ge \frac{81}{2}
        \end{aligned}
        \]
        当 $\sin^2 A = \sin^2 B = \sin^2 C = \dfrac{2}{3},\cos^2 A = \cos^2 B = \cos^2 C = \dfrac{1}{3}$ 时等号成立,
        \[
        \frac{1}{\sin^2 A \cos^4 B} + \frac{1}{\sin^2 B \cos^4 C} + \frac{1}{\sin^2 C \cos^4 A} = \frac{27}{2} \cdot 3 = \frac{81}{2}.
        \]
    \end{solution}

    \question 设$\theta_1, \theta_2, \theta_3, \dots, \theta_{2025}$皆为锐角,且
    \[
    \sin^2 \theta_1 + \sin^2 \theta_2 + \cdots + \sin^2 \theta_{2025} = 1,
    \]
    求
    \[
    \frac{\sin \theta_1 + \sin \theta_2 + \cdots + \sin \theta_{2025}}{\cos \theta_1 + \cos \theta_2 + \cdots + \cos \theta_{2025}}
    \]
    的最大值。
    \begin{solution}
        由题意,
        \[
        \sin^2 \theta_2 + \cdots + \sin^2 \theta_{2025} = 1-\sin^2 \theta_1  = \cos^2 \theta_1
        \]
        由柯西不等式,
        \[
        \cos^2 \theta_1 = (\sin^2 \theta_2 + \cdots + \sin^2 \theta_{2025})(1^2 + 1^2 + \cdots + 1^2)
        \ge (\sin \theta_2 + \sin \theta_3 + \cdots + \sin \theta_{2025})^2
        \]
        即
        \[
        \cos \theta_1 \ge \frac{\sin \theta_2 + \cdots + \sin \theta_{2025}}{\sqrt{2024}}
        \]
        同理,
        \[
        \cos \theta_2 \ge \frac{\sin \theta_1 + \sin \theta_3 + \cdots + \sin \theta_{2025}}{\sqrt{2024}}, \ \cdots \ , \cos \theta_{2025} \ge \frac{\sin \theta_1 + \cdots + \sin \theta_{2024}}{\sqrt{2024}}
        \]
        将以上不等式相加得
        \[
        \cos \theta_1 + \cdots + \cos \theta_{2025} \ge \frac{2024}{\sqrt{2024}} (\sin \theta_1 + \cdots + \sin \theta_{2025})
        \]
        故
        \[
        \frac{\sin \theta_1 + \cdots + \sin \theta_{2025}}{\cos \theta_1 + \cdots + \cos \theta_{2025}} \le \frac{\sqrt{2024}}{2024}
        = \frac{\sqrt{506}}{1012}
        \]
    \end{solution}

    \question 设 $a, b, c$ 为三角形的三边长,$\Delta$ 为此三角形面积,试证:
    \[
    \Delta \le \frac{\sqrt{3}}{4} \left( \frac{a + b + c}{3} \right)^2
    \]
    且等号成立当且仅当 $a = b = c$。
    \begin{solution}
        由海伦公式,三角形面积为
        \[
        \Delta = \sqrt{s(s - a)(s - b)(s - c)}
        \]
        其中$s$为半周长,由AM-GM不等式,
        \[
        \frac{s}{3}=\frac{(s - a) + (s - b) + (s - c)}{3} \ge \sqrt[3]{(s - a)(s - b)(s - c)}
        \]
        故
        \[
        \Delta \le \sqrt{ s\cdot\left(  \frac{s}{3} \right)^3 } = \frac{s^2}{3\sqrt{3}} = \frac{(a + b + c)^2}{12\sqrt{3}} = \frac{\sqrt{3}}{4} \left( \frac{a + b + c}{3} \right)^2
        \]
        等号成立条件为 $s - a = s - b = s - c$,即 $a = b = c$ 时,该三角形为正三角形,面积为 $\dfrac{\sqrt{3}}{4} a^2$
    \end{solution}

    \question 设 $a, b, c$ 是一个三角形的三边,周长为 $2$。证明
\[
\frac{3}{2} < a^2+b^2+c^2 + 2abc < 2.
\]

\begin{solution}
由于周长为 $2$,任意一边都不大于 $1$,则三角形面积
\[
A = \frac{1}{2} bc \sin \alpha
\]
小于 $1/2$。又 $A^2 = (1-a)(1-b)(1-c)$,因此
\begin{align*}
0 &< (1-a)(1-b)(1-c) < \frac{1}{4}, \\
0 &< 1 - (a+b+c) + (ab+ac+bc) - abc < \frac{1}{4}, \\
0 &< 1 - 2 + (ab+ac+bc) - abc < \frac{1}{4}, \\
1 &< (ab+ac+bc) - abc < \frac{5}{4},
\end{align*}
于是
\[
2 < 2(ab+ac+bc) - 2abc < \frac{5}{2}.
\]

另一方面,
\[
a^2+b^2+c^2+2abc = (a+b+c)^2 + (2abc - 2(ab+ac+bc)),
\]
因此
\[
4 - \frac{5}{2} < a^2+b^2+c^2 + 2abc < 4 - 2,
\]
即
\[
\frac{3}{2} < a^2+b^2+c^2 + 2abc < 2.
\]
\end{solution}

        
    \question 在 $\triangle ABC$ 中,角 $A, B, C$ 所对的边为 $a, b, c$,且
    \[
    a^2 + b^2 + 2c^2 = 8
    \]
    求 $\triangle ABC$ 面积的最大值。
    \begin{solution}
        \textbf{解法一}
        
        由余弦定理
        \[
        \cos C = \frac{a^2 + b^2 - c^2}{2ab}= \frac{8 - 3c^2}{2ab}
        \]
        又
        \[
        S = \frac{1}{2}ab\sin C = \frac{1}{2}ab\sqrt{1 - \left(\frac{8-3c^2}{2ab}\right)^2} = \frac{1}{4} \sqrt{4a^2b^2 - (8-3c^2)^2}
        \]
        由不等式$a^2+b^2\ge2ab,$
        \[
        S \leq \frac{1}{4} \sqrt{(a^2+b^2)^2 - (8-3c^2)^2}= \frac{1}{4} \sqrt{(8-2c^2)^2 - (8-3c^2)^2} = \frac{1}{4} \sqrt{16c^2 - 5c^4}
        \]
        对于$c^2$的函数 $f(c^2) = 16c^2 - 5c^4$,令导数为零得 $c^2 = \dfrac{8}{5}$,代入得
        \[
        S_{\text{max}} = \frac{1}{4} \sqrt{\frac{64}{5}} = \frac{2\sqrt{5}}{5}
        \]
    \end{solution}
    \begin{solution}
        \textbf{解法二}
        
        以 $AB$ 所在直线为 $x$ 轴,$AB$ 中垂线为 $y$ 轴,设 $A(-\dfrac{c}{2},0),B(\dfrac{c}{2},0),C(x,y)$,由条件
        \[
        a^2 + b^2 + 2c^2 = 8
        \]
        得
        \[
        (x - \frac{c}{2})^2 + y^2 + (x + \frac{c}{2})^2 + y^2 + 2c^2 = 8
        \]
        即
        \[
        x^2 + y^2 = 4 - \frac{5c^2}{4}
        \]
        $\triangle ABC$面积为
        \[
        S = \frac{1}{2}c|y|
        \]
        由 $y^2 \leq 4 - \dfrac{5c^2}{4}$,得
        \[
        S \leq \frac{1}{2}c\sqrt{4 - \frac{5c^2}{4}} = \frac{1}{4}\sqrt{16c^2 - 5c^4}
        \]
        解法同上。
    \end{solution}
    
    \question  在 $\triangle ABC$ 中,试求$$\frac{2\sin^2 A + \sin B \sin C}{\sin A \sin B \sin C}$$ 的最小值。
    \begin{solution}
        由余弦定理,
        \[
        \cos A = \dfrac{b^2 + c^2 - a^2}{2bc}\ge 1 - \dfrac{a^2}{2bc}\Rightarrow \frac{2a^2}{bc} \ge 4 - 4\cos A
        \]
        由正弦定理,
        \[
        \frac{2\sin^2 A + \sin B \sin C}{\sin A \sin B \sin C}
        = \frac{1}{\sin A} \left( \frac{2a^2}{bc} + 1 \right)
        \ge \frac{5 - 4\cos A}{\sin A}
        \]
        设 $\tan \dfrac{A}{2} = t > 0$,则
        \[
        \cos A = \frac{1 - t^2}{1 + t^2},\quad \sin A = \frac{2t}{1 + t^2}
        \]
        代入得
        \[
        \frac{5 - 4\cos A}{\sin A}
        = \frac{5 - 4\cdot \frac{1 - t^2}{1 + t^2}}{\frac{2t}{1 + t^2}}
        = \frac{9t^2 + 1}{2t}
        \]
        由AM-GM不等式:
        \[
        \frac{9t^2 + 1}{2t} \ge \frac{2\sqrt{9t^2 \cdot 1}}{2t} = 3
        \]
        当 $A=\arcsin\dfrac{3}{5}$, $B=C=\dfrac{\pi-A}{2}$ 时等号成立, 原式取最小值 $3$
    \end{solution}
    
    \question 在锐角三角形 $\triangle ABC$ 中,角 $A,B,C$ 所对的边为 $a,b,c$,且
        \[
        a^2 + 2ab\cos C = 3b^2
        \]
        求 $\tan A \tan B \tan C$ 的最小值。
    \begin{solution}
        \textbf{解法一}\\
        由余弦定理,有
        \[
        2ab\cos C = a^2+b^2-c^2
        \]
        代入原式得
        \[
        a^2 + (a^2+b^2-c^2) = 3b^2 \Rightarrow c^2 = 2a^2 - 2b^2 \Rightarrow b^2 + c^2 - a^2 = \frac{c^2}{2} \Rightarrow \frac{b^2 + c^2 - a^2}{2bc} = \frac{c^2}{4bc}
        \]
        由正、余弦定理得
        \[
        \cos A = \frac{\sin C}{4\sin B} \Rightarrow 4\sin B \cos A  =\sin C = \sin(A+B) = \sin A \cos B + \cos A \sin B
        \]
        有 $3\sin B \cos A = \sin A \cos B$.
        
        $\because \triangle ABC$ 是锐角三角形, $\cos A\neq0,\cos B\neq0, \therefore \tan A=3\tan B $
        
        现设 $\tan B = t$, 又
        \[
        \tan C = -\tan(A+B) = -\frac{\tan A + \tan B}{1-\tan A\tan B} = \frac{4t}{3t^2-1}
        \]
        所以
        \[
        \tan A \tan B \tan C = 3t \cdot t \cdot \frac{4t}{3t^2-1} = \frac{12t^3}{3t^2-1}
        \]
        对函数
        \[
        f(t) = \frac{12t^3}{3t^2-1}, t>0
        \]
        求导
        \[
        f'(t) = \frac{36t^2(3t^2-1) - 12t^3 \cdot 6t}{(3t^2-1)^2} = \frac{36t^2(t^2-1)}{(3t^2-1)^2}
        \]
        令 $f'(t)=0$,解得 $t=1$; 当 $t\in (0,1),f'(t)<0,$当 $t\in (1,\infty),f'(t)>0,$

        $\therefore\; f(t)$在$t=1$有最小值 $f(1)=6$。
    \end{solution}
    \begin{solution}
        \textbf{解法二}\\
        以 $AB$ 所在直线为 $x$ 轴,$\;AB$ 中垂线为 $y$ 轴, 设 $A(-\dfrac{c}{2},0),B(\dfrac{c}{2},0),C(x,y)$,       
        由距离公式
        \[
        a^2 = (x-\frac{c}{2})^2 + y^2, \quad b^2 = (x+\frac{c}{2})^2 + y^2
        \]
        代入已知条件
        \[
        2b^2 + c^2 - 2a^2 = 0 \implies 2\left[(x+\frac{c}{2})^2 + y^2\right] + c^2 - 2\left[(x-\frac{c}{2})^2 + y^2\right] = 0
        \]
        展开化简得
        \[
        4c x + c^2 = 0 \implies x = -\frac{c}{4}
        \]
        作 $CD \perp AB$ 于 $D$,则 $D$ 点坐标为 $(x,0) = \left(-\dfrac{c}{4},0\right)$。发现
        \[
        AD = \left|-\frac{c}{2} - \left(-\frac{c}{4}\right)\right| = \frac{c}{4} \quad \text{且} \quad BD = \left|\frac{c}{2} - \left(-\frac{c}{4}\right)\right| = \frac{3c}{4}
        \]
        所以 $BD = 3AD$,又
        \[
        \tan A = \frac{CD}{AD} = \frac{y}{\frac{c}{4}}
        \]
        \[
        \tan B = \frac{CD}{BD} = \frac{y}{\frac{3c}{4}}
        \]
        因此有
        \[
        \tan A = 3 \tan B
        \]解法同上。
    \end{solution}
    
    \question 已知非负实数 $a, b, c,d$ 满足 $a \le 1, a+b \le 5, a+b+c \le 14, a+b+c+d \le 30$,试证$$\sqrt{a}+\sqrt{b}+\sqrt{c}+\sqrt{d}\le10$$
    \begin{solution}
        由柯西不等式,
        \begin{align*}
            &(\sqrt{a}+\sqrt{b}+\sqrt{c}+\sqrt{d})^2\\
            &=\left(\sqrt{a}+\frac{\sqrt{b}}{2}+\frac{\sqrt{b}}{2}+\frac{\sqrt{c}}{3}+\frac{\sqrt{c}}{3}+\frac{\sqrt{c}}{3}+\frac{\sqrt{d}}{4}+\frac{\sqrt{d}}{4}+\frac{\sqrt{d}}{4}+\frac{\sqrt{d}}{4}\right)^{2} \\
            &\le\left(a+\frac{b}{2}+\frac{b}{2}+\frac{c}{9}+\frac{c}{9}+\frac{c}{9}+\frac{d}{16}+\frac{d}{16}+\frac{d}{16}+\frac{d}{16}\right)(1+...+1)=10(a+\frac{b}{2}+\frac{c}{3}+\frac{d}{4})
        \end{align*}
        而
        \begin{align*}
        a+\frac{b}{2}+\frac{c}{3}+\frac{d}{4}&=\frac{1}{4}(a+b+c+d)+\frac{1}{12}(a+b+c)+\frac{1}{6}(a+b)+\frac{1}{2}a\\
        &\le\frac{30}{4}+\frac{14}{12}+\frac{5}{6}+\frac{1}{2}=10
        \end{align*}
        故
        $$\sqrt{a}+\sqrt{b}+\sqrt{c}+\sqrt{d}\le10$$
    \end{solution}

    \question 已知 $a,b,c$ 为正实数,且满足 $abc=1$,试证明:
    \[
    \frac{1}{a^{3}(b+c)} + \frac{1}{b^{3}(a+c)} + \frac{1}{c^{3}(a+b)} \ge \frac{3}{2}
    \]
    \begin{solution}
        由柯西不等式,
        \[
        \left(\frac{1}{a^{3}(b+c)} + \frac{1}{b^{3}(a+c)} + \frac{1}{c^{3}(a+b)}\right) (a(b+c) + b(a+c) + c(a+b)) \ge \left(\frac{1}{a} + \frac{1}{b} + \frac{1}{c}\right)^2
        \]
        注意到
        \[
        a(b+c) + b(a+c) + c(a+b) = 2(ab + bc + ca),
        \]
        且
        \[
        \left(\frac{1}{a} + \frac{1}{b} + \frac{1}{c}\right)^2 = \left(\frac{ab + bc + ca}{abc}\right)^2 = (ab + bc + ca)^2
        \]
        故由AM-GM不等式,
        \[
        \frac{1}{a^{3}(b+c)} + \frac{1}{b^{3}(a+c)} + \frac{1}{c^{3}(a+b)} \ge \frac{1}{2}(ab + bc + ca)
        \ge \frac{3}{2}\sqrt[3]{(abc)^2} = \frac{3}{2}
        \]
    \end{solution}

\begin{solution}
由 $abc=1$,有
\[
\sum_{cyc} \frac{1}{a^3(b+c)} = \sum_{cyc} \frac{bc}{a^2(b+c)} = \sum_{cyc} \frac{(\frac{1}{a})^2}{\frac{1}{b}+\frac{1}{c}}.
\]

利用不等式
\[
\frac{x^2}{y} \ge x - \frac{y}{4}, \quad x,y>0,
\]
得到
\[
\sum_{cyc} \frac{(\frac{1}{a})^2}{\frac{1}{b}+\frac{1}{c}} \ge \sum_{cyc} \left[ \frac{1}{a} - \frac{1}{4}\left( \frac{1}{b} + \frac{1}{c} \right) \right] = \frac{1}{2} \left( \frac{1}{a} + \frac{1}{b} + \frac{1}{c} \right).
\]

由 AM–GM 不等式:
\[
\frac{1}{2} \left( \frac{1}{a} + \frac{1}{b} + \frac{1}{c} \right) \ge \frac{3}{2} \sqrt[3]{\frac{1}{abc}} = \frac{3}{2}.
\]

因此原不等式成立。
\end{solution}

\question 已知 $0\le x,y,z\le1$ 且
\[
xyz=(1-x)(1-y)(1-z),
\]
证明
\[
x(1-z)+y(1-x)+z(1-y)\ge\frac{3}{4}.
\]

\begin{solution}
由于 $0\le x\le1$,有 $(x-\tfrac12)^2\ge0$,从而 $0\le x(1-x)\le\tfrac14$。对 $y,z$ 同理成立。于是
\[
xyz(1-x)(1-y)(1-z)\le\frac{1}{64}.
\]

由已知关系 $xyz=(1-x)(1-y)(1-z)$,左端等于 $(xyz)^2$,因此
\[
(xyz)^2\le\frac{1}{64}\quad\Longrightarrow\quad xyz\le\frac{1}{8}.
\]

再利用已知等式可得
\[
x(1-z)+y(1-x)+z(1-y)
= x+y+z - (xy+yz+zx)
= 1 - 2xyz,
\]
其中最后一步由 $xyz=(1-x)(1-y)(1-z)$ 展开并化简得到。于是
\[
x(1-z)+y(1-x)+z(1-y)=1-2xyz\ge 1-2\cdot\frac{1}{8}=\frac{3}{4}.
\]

当 $x=y=z=\tfrac12$ 时取等号。因此不等式成立,且在 $x=y=z=\tfrac12$ 时达等号。
\end{solution}


    \question 设正实数 $a,b,c,x,y,z$ 满足 $a+b+c = x+y+z$,证明
    \[
    \frac{2a^2}{a+x} + \frac{2b^2}{b+y} + \frac{2c^2}{c+z} \ge a+b+c
    \]
    \begin{solution}
    由柯西不等式,
        \[
        \left(\frac{2a^2}{a+x} + \frac{2b^2}{b+y} + \frac{2c^2}{c+z}\right)\left(2(a+x) +2(b+y) +2(c+z) \right) \ge (2a+2b+2c)^2 
        \]
        又 $a+b+c = x+y+z$,得
        \[
        \frac{2a^2}{a+x} + \frac{2b^2}{b+y} + \frac{2c^2}{c+z} \ge a+b+c.
        \]
    \end{solution}

    \question 已知 $a,b,c,d>0$,证明不等式
    \[
    \frac{a^{3}+b^{3}+c^{3}}{a^{2}+b^{2}+c^{2}}+\frac{b^{3}+c^{3}+d^{3}}{b^{2}+c^{2}+d^{2}}+\frac{c^{3}+d^{3}+a^{3}}{c^{2}+d^{2}+a^{2}}+\frac{d^{3}+a^{3}+b^{3}}{d^{2}+a^{2}+b^{2}}\ge a+b+c+d.
    \]
    \begin{solution}
        由 $a,b>0$,有
        \[
        0\le (a+b)(a-b)^2= (a^2-b^2)(a-b) = a^3 - a^2 b - b^2 a + b^3 
        \]
        可知
        \[
        a^3 + b^3 \ge a^2 b + a b^2, \quad b^3 + c^3 \ge b^2 c + b c^2, \quad c^3 + a^3 \ge c^2 a + c a^2.
        \]
        于是
        \begin{align*}
        3(a^3 + b^3 + c^3) &\ge a^3 + b^3 + c^3+ a^2 b + a b^2 + b^2 c + b c^2 + c^2 a + c a^2\\
        &=(a+b+c)(a^2 + b^2 + c^2)
        \end{align*}
        因此
        \[
        \frac{a^3 + b^3 + c^3}{a^2 + b^2 + c^2} \ge \frac{a+b+c}{3}
        \]
        故
        \begin{align*}
        &\frac{a^{3}+b^{3}+c^{3}}{a^{2}+b^{2}+c^{2}}+\frac{b^{3}+c^{3}+d^{3}}{b^{2}+c^{2}+d^{2}}+\frac{c^{3}+d^{3}+a^{3}}{c^{2}+d^{2}+a^{2}}+\frac{d^{3}+a^{3}+b^{3}}{d^{2}+a^{2}+b^{2}} \\[1mm]
        &\ge \frac{1}{3} \cdot 3(a+b+c+d) = a+b+c+d 
        \end{align*}
    \end{solution}

    \question 已知正实数 $a, b, c, d$,求
    \[
    \frac{ab + bc + cd}{a^2 + b^2 + c^2 + d^2}
    \]
    的最大值。
    \begin{solution}
        对任意正数 $t$,由 AM-GM 不等式,
        \[
        ab \le \frac{t}{2} a^2 + \frac{1}{2t} b^2, \quad 
        bc \le \frac{1}{2} b^2 + \frac{1}{2} c^2, \quad
        cd \le \frac{1}{2t} c^2 + \frac{t}{2} d^2.
        \]
        取 $t = \frac{1}{2}(1+\sqrt{5})$,使得 
        \[
        \frac{t}{2} = \frac{1}{2t} + \frac{1}{2}
        \]
        则
        \[
        ab + bc + cd \le \frac{t}{2} (a^2 + b^2 + c^2 + d^2).
        \]
        因此最大值为
        \[
        \frac{t}{2} = \frac{\sqrt{5}+1}{4}
        \]
        当 $ta = b = c = td$ 时等号成立。
    \end{solution}

    \question 已知正数 $a,b,c$ 满足 $abc > 1$,求 $\dfrac{abc(a + b + c + 8)}{abc - 1}$ 的最小值。
    \begin{solution}
        设 $\lambda = \dfrac{abc(a + b + c + 8)}{abc - 1} > 0$,整理得
        \[
        a + b + c + \dfrac{\lambda}{abc} = \lambda - 8
        \]
        由AM-GM不等式得
        \[
        a + b + c + \dfrac{\lambda}{abc} \ge 4\sqrt[4]{abc \cdot \dfrac{\lambda}{abc}} = 4\sqrt[4]{\lambda}
        \]
        于是
        \[
        \lambda - 8 \ge 4\sqrt[4]{\lambda}
        \]
        令 $k = \sqrt[4]{\lambda} > 0$,则不等式转化为
        \[
        k^4 - 4k - 8 \ge 0 \Rightarrow (k - 2)(k^3 + 2k^2 + 4k + 4) \ge 0
        \]
        由于 $k > 0,k^3 + 2k^2 + 4k + 4>0$ ,故$k\ge 2$,即  $$\lambda \ge 2^4 = 16$$
        等号成立当且仅当 $a = b = c = 2$
    \end{solution}

    \question 已知 $x^2+y^2+z^2 \leq 1$, 求 $x^2+2y-2z+3$ 的取值范围。
    \begin{solution}
        由柯西不等式, $$[y+(-z)]^2 \leq (1+1)[y^2+(-z)^2] \leq 2(x^2+y^2+z^2) \leq 2$$
        则 $y-z \geq -\sqrt{2}$, 且 $x^2 \geq 0$, 可得 $$x^2+2y-2z+3 \geq 0+2(-\sqrt{2})+3=3-2\sqrt{2}$$
        当且仅当 $x=0, y=-\dfrac{\sqrt{2}}{2}, z=\dfrac{\sqrt{2}}{2}$ 时,等号成立;
        
        又因为 $x^2+y^2+z^2 \leq 1$, 则 $x^2 \leq 1-y^2-z^2$,
        可得 $$x^2+2y-2z+3 \leq 4-(y^2+z^2-2y+2z)$$
        且 $$y^2 + z^2 - 2y + 2z = (y-1)^2 + (z+1)^2 - 2,$$
        设点 $A(-1,1)$ 和标准单位圆上一点 $P(y,z)$, 则 $$(y-1)^2 + (z+1)^2 - 2 = |PA|^2 - 2,$$
        又因为 $|PA|^2 \geq (|OA|-1)^2 = 3-2\sqrt{2}$, 可得 $(y-1)^2 + (z+1)^2 - 2 \geq 1-2\sqrt{2},$
        则 $$x^2+2y-2z+3 \leq 4-(y^2+z^2-2y+2z) \leq 3+2\sqrt{2},$$
        当且仅当 $x=0, y=\dfrac{\sqrt{2}}{2}, z=-\dfrac{\sqrt{2}}{2}$ 时,等号成立。
        
        $\therefore$取值范围是 $[3-2\sqrt{2}, 3+2\sqrt{2}].$
    \end{solution}  

    \question 已知 $x,y,z>0$ ,求 $$\frac{\sqrt{x^{2}+y^{2}}+\sqrt{y^{2}+4z^{2}}+\sqrt{z^{2}+16x^{2}}}{9x+3y+5z}$$ 的最小值。
    \begin{solution}
        \textcolor{red}{(待解)}
    \end{solution}

    \question 证明
\[
\sum_{n=1}^\infty \frac{1}{\sqrt{n(n+1)}}<2.
\]

\begin{solution}
我们证明对任意正整数 $n$,都有
\[
\frac{1}{\sqrt{n(n+1)}}<\frac{2}{\sqrt{n}}-\frac{2}{\sqrt{n+1}}.
\]
将不等式两边同乘以 $\sqrt{n(n+1)}$,等价于
\[
1<2\sqrt{n+1}-2\sqrt{n}.
\]
等价地,
\[
2\sqrt{n(n+1)}<n+(n+1).
\]
由于 $n$ 与 $n+1$ 均为正数,由算术平均与几何平均不等式可知
\[
2\sqrt{n(n+1)}\le n+(n+1),
\]
且不等号严格成立,因此上述不等式成立。

于是
\[
\sum_{n=1}^\infty \frac{1}{\sqrt{n(n+1)}}<
\sum_{n=1}^\infty \left(\frac{2}{\sqrt{n}}-\frac{2}{\sqrt{n+1}}\right).
\]
右端为裂项求和,
\[
\sum_{n=1}^\infty \left(\frac{2}{\sqrt{n}}-\frac{2}{\sqrt{n+1}}\right)
=2.
\]
因此
\[
\sum_{n=1}^\infty \frac{1}{\sqrt{n(n+1)}}<2.
\]
\end{solution}


    \question 假设实数 $a,b,c \in [-1,1]$ 并且满足
\[
1 + 2abc \ge a^2 + b^2 + c^2.
\]

\noindent
证明对所有正整数 $n$ 有
\[
1 + 2(abc)^n \ge a^{2n} + b^{2n} + c^{2n}.
\]

\begin{solution}
约束条件可以改写为
\[
(a - bc)^2 \le (1-b^2)(1-c^2). \tag{1}
\]

利用 Cauchy-Schwarz 不等式,有
\begin{align*}
(a^{n-1} + a^{n-2} bc + \cdots + b^{n-1} c^{n-1})^2 
&\le (|a|^{n-1} + |a|^{n-2}|bc| + \cdots + |bc|^{n-1})^2 \\
&\le (1 + |bc| + \cdots + |bc|^{n-1})^2 \\
&\le (1 + |b|^2 + \cdots + |b|^{2(n-1)}) (1 + |c|^2 + \cdots + |c|^{2(n-1)}).
\end{align*}

将不等式 (1) 乘以上式得到
\begin{align*}
(a - bc)^2 (a^{n-1} + a^{n-2} bc + \cdots + b^{n-1} c^{n-1})^2 
&\le ((1-b^2)(1 + |b|^2 + \cdots + |b|^{2(n-1)})) \\
&\quad \times ((1-c^2)(1 + |c|^2 + \cdots + |c|^{2(n-1)})) \\
(a^n - b^n c^n)^2 &\le (1 - b^{2n})(1 - c^{2n}),
\end{align*}
从而得到
\[
1 + 2(abc)^n \ge a^{2n} + b^{2n} + c^{2n}.
\]
\end{solution}


    \question 设 $a_1,\dots,a_n$ 为正实数,且设 $s=a_1+\cdots+a_n$。证明
\[
(1+a_1)(1+a_2)\cdots(1+a_n) \le 1+\frac{s}{1!}+\frac{s^2}{2!}+\cdots+\frac{s^n}{n!}.
\]

\begin{solution}
由算术—几何均值不等式(A.M.-G.M.),有
\[
\Big(\prod_{j=1}^n(1+a_j)\Big)^{\frac{1}{n}}
\le \frac{1}{n}\sum_{j=1}^n(1+a_j)=1+\frac{s}{n}.
\]
因此
\[
\prod_{j=1}^n(1+a_j)\le\Big(1+\frac{s}{n}\Big)^n.
\]

对右端使用二项展开得
\[
\Big(1+\frac{s}{n}\Big)^n=\sum_{j=0}^n\binom{n}{j}\frac{s^j}{n^j}
=\sum_{j=0}^n\frac{n!}{(n-j)!\,j!}\,\frac{s^j}{n^j}.
\]

注意当 $0\le j\le n$ 时
\[
\frac{n!}{(n-j)!\,n^j}=\prod_{k=0}^{j-1}\Big(1-\frac{k}{n}\Big)\le 1,
\]
因此每一项满足
\[
\frac{n!}{(n-j)!\,j!}\,\frac{s^j}{n^j}\le\frac{s^j}{j!}.
\]

将这些不等式代入二项和得到
\[
\prod_{j=1}^n(1+a_j)\le\sum_{j=0}^n\frac{s^j}{j!},
\]
即
\[
(1+a_1)\cdots(1+a_n)\le 1+\frac{s}{1!}+\frac{s^2}{2!}+\cdots+\frac{s^n}{n!},
\]
证毕。
\end{solution}

\question 设 $n$ 个正实数的最小值为 $m$,最大值为 $M$,记它们的算术平均为 $A$,几何平均为 $G$。证明
\[
A-G \ge \frac{1}{n}\bigl(\sqrt{M}-\sqrt{m}\bigr)^2.
\]

\begin{solution}
设这 $n$ 个正实数为 $x_1,\dots,x_n$,且不妨令 $x_1=m,\ x_n=M$。由题不等式等价于
\[
A-G \ge \frac{(\sqrt{M}-\sqrt{m})^2}{n},
\]
即等价于
\[
\sum_{j=1}^n x_j - n\Big(\prod_{j=1}^n x_j\Big)^{1/n} \ge M - 2\sqrt{Mm} + m.
\]
移项得等价形式
\[
\sum_{j=2}^{n-1} x_j + 2\sqrt{Mm} \ge n\Big(\prod_{j=1}^n x_j\Big)^{1/n}.
\]

把右端的几何平均下界由算术—几何均值不等式给出:对这 $n$ 个数
\[
\sqrt{Mm},\; x_2,\; x_3,\;\dots,\; x_{n-1},\; \sqrt{Mm}
\]
应用 A.M.-G.M.,有
\[
\frac{1}{n}\Big(\sum_{j=2}^{n-1} x_j + 2\sqrt{Mm}\Big)
\ge \Big(\sqrt{Mm}\cdot\prod_{j=2}^{n-1} x_j\cdot\sqrt{Mm}\Big)^{1/n}.
\]
右端化简为
\[
\Big(Mm\prod_{j=2}^{n-1} x_j\Big)^{1/n}
=\Big(x_1x_n\prod_{j=2}^{n-1} x_j\Big)^{1/n}
=\Big(\prod_{j=1}^n x_j\Big)^{1/n}.
\]

两边同时乘以 $n$ 即得
\[
\sum_{j=2}^{n-1} x_j + 2\sqrt{Mm} \ge n\Big(\prod_{j=1}^n x_j\Big)^{1/n},
\]
从而恢复到所需不等式
\[
A-G \ge \frac{1}{n}\bigl(\sqrt{M}-\sqrt{m}\bigr)^2.
\]
证毕。
\end{solution}

\question 设 $x_1,\ldots,x_n\ge -1$,且
\[
\sum_{i=1}^n x_i^3=0。
\]
证明
\[
\sum_{i=1}^n x_i\le \frac{n}{3}。
\]

\begin{solution}
当 $x\ge -1$ 时,有不等式
\[
0\le x^3-\frac34 x+\frac14=(x+1)\left(x-\frac12\right)^2。
\]

将 $x_1,\ldots,x_n$ 代入上式并求和,得到
\[
0\le \sum_{i=1}^n\left(x_i^3-\frac34 x_i+\frac14\right)
=\sum_{i=1}^n x_i^3-\frac34\sum_{i=1}^n x_i+\frac{n}{4}。
\]
由已知条件 $\sum_{i=1}^n x_i^3=0$,可得
\[
0\le -\frac34\sum_{i=1}^n x_i+\frac{n}{4},
\]
从而
\[
\sum_{i=1}^n x_i\le \frac{n}{3}。
\]

注:当且仅当 $n=9k$,其中 $k$ 个 $x_i=-1$,其余 $8k$ 个 $x_i=\frac12$ 时取等号。
\end{solution}

    \question 已知非负实数 $x_1, x_2, \ldots, x_5$ 满足 $x_1^2 + x_2^2 + \cdots + x_5^2 = 4$. 求下列表达式的最大值:
    $$S = \sum_{i=1}^5 \frac{1}{x_i+1}\sum_{i=1}^5 x_i$$
    \begin{solution}
        当 $x_1 = x_2 = x_3 = x_4 = 1, x_5 = 0$ 时, $S = 12$,下面证明 $S \le 12$.
        注意到 
        $$\frac{x(x-1)^2}{x+1} \ge 0$$ 对 $x \ge 0$ 成立, 由此可得$$\sum_{i=1}^5 \frac{x_i(x_i-1)^2}{x_i+1} \ge 0$$
        展开可知
        $$\sum_{i=1}^5 \left(x_i^2 - 3x_i + 4 - \frac{4}{x_i+1}\right) \ge 0$$
        整理得
        $$\sum_{i=1}^5 \frac{1}{x_i+1} \le 6 - \frac{3}{4} \sum_{i=1}^5 x_i$$
        由上式及AM-GM不等式可知 
        $$S \le \frac{4}{3} \left(6 - \frac{3}{4}\sum_{i=1}^5 x_i\right) \left(\frac{3}{4}\sum_{i=1}^5 x_i\right) \le \frac{4}{3} \times 3^2 = 12$$
    \end{solution}
\end{questions}

\pagebreak

\begin{center}
  {\fontsize{30pt}{26pt}\selectfont
    \hypertarget{数列与级数}{数列与级数} \label{数列与级数}
  }
\end{center}
\separator
\vspace{1pt}

\begin{questions}
    \question 设 \(a, b, c\) 三数成等比数列,且满足 \(a + b + c = 9\) 及 \(a^2 + b^2 + c^2 = 189\),求 $b$。
    \begin{solution}
      关键:运用恒等式\[
        (a+b+c)^2=a^2+b^2+c^2+2(ab+bc+ac)
      \]又已知 $b^2=ac$,故\[
      81=189+2(b(9-b)+b^2) \Rightarrow b=6
      \]
    \end{solution}
    
    \question 已知 $a,b,c,d$ 成等差数列,且实数 $x,y,z,u$ 满足
    \[
    \begin{cases}
    a+b+c+d = 60 \\
    x+y+z+u = 12 \\
    az + bu + cx + dy = 168
    \end{cases}
    \]
    求 $ay + bx + cu + dz$。
    \begin{solution}
        已知$a,b,c,d$ 成等差数列,于是
        \begin{align*}
        ay + bx + cu + dz+168 
        &=ay + bx + cu + dz+(az + bu + cx + dy) \\
        &=(a+d)(y+z)+(b+c)(x+u)\\
        &=30(x+y+z+u)= 360
        \end{align*}
    故
    \[
    ay + bx + cu + dz=192
    \]
    \end{solution}
    
    \question 已知 $a_1=-1$, $a_{n+1}=2a_n-3$, 求 $a_{2017}$。
    \begin{solution}
        由递推式,
        \[
        a_{n+1}-3 = 2(a_n-3),
        \]
        所以 $a_1-3,a_2-3,\dots$ 成等比数列,公比为 $2$。
        \[
        a_{n+1}-3 = 2^n(a_1-3) = 2^n(-1-3) = -2^{n+2} \Rightarrow a_{2017} = -2^{2018}+3.
        \]
    \end{solution}
    \begin{solution}
        由
        \[
        a_{n+1}-a_n =2a_n-3-(2a_{n-1}-3)= 2(a_n-a_{n-1}),
        \]
        所以 $\{a_{n+1}-a_n\}$ 为等比数列,公比为 $2$,因此
        \begin{align*}
        a_n &= (a_2-a_1)+(a_3-a_2)+\cdots+(a_n-a_{n-1})+a_1 \\
        &= \frac{(a_2-a_1)(2^{n-1}-1)}{2-1}-1 \\ 
        &= (2a_1-3-a_1)(2^{n-1}-1) =-2^{n+1}+3
        \end{align*}
        所以
        \[
        a_{2017}=-2^{2018}+3
        \]
    \end{solution}

    \question
已知数列 $\{a_r\}$ 定义为
\[
a_{r+1}=a_r+\frac{2r}{r^4+r^2+1},\quad a_1=0,\quad r\in\mathbb{N},
\]
求 $a_{61}$ 的精确值。

\begin{solution}

\noindent
由递推关系,对 $r=1,2,\dots,60$ 累加得
\[
a_{61}-a_1=\sum_{r=1}^{60}\frac{2r}{r^4+r^2+1}.
\]
由于 $a_1=0$,因此
\[
a_{61}=\sum_{r=1}^{60}\frac{2r}{r^4+r^2+1}.
\]

\noindent
注意到
\[
r^4+r^2+1=(r^2-r+1)(r^2+r+1),
\]
于是
\[
\frac{2r}{r^4+r^2+1}
=\frac{2r}{(r^2-r+1)(r^2+r+1)}
=\frac{1}{r^2-r+1}-\frac{1}{r^2+r+1}.
\]

\noindent
因此
\[
a_{61}=\sum_{r=1}^{60}\left(\frac{1}{r^2-r+1}-\frac{1}{r^2+r+1}\right).
\]

\noindent
这是一个望远镜求和,展开可得
\begin{align*}
a_{61}
&=\left(\frac{1}{1^2-1+1}-\frac{1}{1^2+1+1}\right)
+\left(\frac{1}{2^2-2+1}-\frac{1}{2^2+2+1}\right)
+\cdots \\
&\quad +\left(\frac{1}{60^2-60+1}-\frac{1}{60^2+60+1}\right).
\end{align*}

\noindent
中间项两两抵消,最终剩下
\[
a_{61}
=1-\frac{1}{60^2+60+1}
=1-\frac{1}{3661}.
\]

\noindent
因此
\[
a_{61}=\frac{3660}{3661}.
\]

\end{solution}


    \question 数列 \(a_0, a_1, a_2, \ldots, a_n, \ldots\) 满足 \(a_0 = 3\),且
    \[
    (3 - a_{n-1})(6 + a_n) = 18,
    \]
    证明
    \[
    \sum_{k=1}^n \frac{1}{a_k} = \frac{1}{3}\left(2^{n+2}-n-4\right)
    \]
    \begin{solution}
        由递推关系
        \begin{equation}
            (3-a_{n+1})(a_n+6) = 18 \implies a_{n+1} = \frac{3a_n}{a_n+6}
        \end{equation}
        两边加3得到
        \begin{equation}
            a_{n+1}+3 = \frac{3a_n}{a_n+6}+3 = \frac{6(a_n+3)}{a_n+6}
        \end{equation}
        两式相比得
        \[
        \frac{a_{n+1}+3}{a_{n+1}} =  \frac{2(a_n+3)}{a_n}
        \]
        故数列 $\left\{\dfrac{a_n+3}{a_n}\right\}$ 是首项为2,公比为2的等比数列,解得
        \[
        \frac{a_n+3}{a_n} = 2^{n+1} \Rightarrow a_n = \frac{3}{2^{n+1}-1}
        \]
        因此
        \[
        \sum_{k=1}^n \frac{1}{a_k} = \sum_{k=1}^n \frac{2^{k+1}-1}{3} = \frac{1}{3}\left(2^{n+2}-n-4\right)
        \]
    \end{solution}
        
    \question 设一函数 $f$ 的定义域为所有正整数,如果 $f(1) = 101$,且对所有正整数 $n > 1$皆有  
    \[
    f(1) + f(2) + f(3) + \cdots + f(n) = n^2 f(n)
    \]  
    求 $f(100)$ 的值。

    \begin{solution}
        发现
        \[
        f(n) = \sum_{k=1}^n f(k)-\sum_{k=1}^{n-1} f(k)=n^2 f(n) - (n - 1)^2 f(n - 1)
        \]
        即
        \[
         f(n) = \frac{n - 1}{n + 1} f(n - 1)
        \]
        于是
        \[
        f(100) = \frac{99 \cdot 98 \cdots 1}{101 \cdot 100 \cdots 3} f(1) = \frac{2}{101 \cdot 100} \cdot 101 = \frac{1}{50}
        \]
    \end{solution}
    
    \question 已知数列$\{a_n\}$满足: $a_1 =1$, $a_{n+1} = a_n + \dfrac{1}{a_n}$, 则$\lfloor a_{2025}\rfloor=$
    \begin{solution}
        由题意可知$$a_{n+1}^2 = a_n^2 + \frac{1}{a_n^2} + 2$$所以有
        $$a_{2025}^2 = a_1^2 + 2 \times (2025-1) + \left(\frac{1}{a_1^2} + \frac{1}{a_2^2} + \dots + \frac{1}{a_{2024}^2}\right)$$又因为$a_1=1$, 且有
        $$\frac{1}{a_1^2} + \frac{1}{a_2^2} + \dots + \frac{1}{a_{2024}^2} < 47$$所以$$63=\sqrt{3969} < a_{2025} < \sqrt{4096}=64$$故$[a_{2025}]=63$.
    \end{solution}

    \question 设 $f: \mathbb{N} \to \mathbb{R}$满足 $f(1) = \frac{3}{2}$,且对所有$n \in \mathbb{N}$且满足$$\ f(n+1) = \left(1 + \frac{1}{n+1}\right) f(n) + \left(1 + \frac{n}{2}\right) f(1) + n^2 + 2n,$$ 求 $f(40)$。
    \begin{solution}
        将递推式改写为
        \[
        f(n+1) = \frac{n+2}{n+1} f(n) + g(n),
        \]
        其中 $\displaystyle g(n) = (n+2)\left(n + \frac{3}{4}\right)$,于是
        \begin{align*}
        f(40) &= \frac{41}{40} f(39) + g(39) \\
        &= \frac{41}{39} f(38) + \frac{41}{40} g(38) + g(39) \\
        &= \cdots \\
        &= \frac{41}{2} f(1) + \sum_{k=1}^{39} \frac{41}{k+2}\ g(k) \\
        &= \frac{41}{2} \cdot \frac{3}{2} + 41 \sum_{k=1}^{39} (k+\frac{3}{4}) \\
        &= \frac{123}{4} + \frac{39\cdot 40}{2}+39\cdot \frac{3}{4}\\
        &= 33210
        \end{align*}
    \end{solution}
    
    \question 已知数列 $\{a_{n}\}$ 满足$a_{1}=a_{2}=a_{3}=1$,令 $b_{n}=a_{n}+a_{n+1}+a_{n+2},\ n\in \mathbb{N}$,若 $\{b_{n}\}$ 是公比为3的等比数列,求 $a_{100}$ 的值。
    \begin{solution}
        由条件知 $$b_{n}=b_{1}\cdot3^{n-1}=3^{n}$$
        因此$$a_{n+3}-a_{n}=b_{n+1}-b_{n}=3^{n+1}-3^{n}=2\cdot3^{n}$$
        于是
        $$a_{100}=a_{1}+\sum_{k=1}^{33}(a_{3k+1}-a_{3k-2})=1+\sum_{k=1}^{33}2\cdot3^{3k-2}$$
        $$=1+6\cdot\frac{27^{33}-1}{27-1}=1+\frac{3}{13}(3^{99}-1)=\frac{3^{100}+10}{13}$$
    \end{solution}

    \question 已知递归数列满足
    \[
    a_1=\frac{1}{2},\quad a_{n+1}=\frac{a_n+3}{2a_n-4},
    \]
    求通项公式 $a_n$。
    \begin{solution}
        先求不动点,设 $x=\dfrac{x+3}{2x-4}$,得 $2x^2-5x-3=0$,所以不动点为 $x=-\dfrac12,\,3$,于是
        \[
        R_n=\frac{a_n+\tfrac12}{a_n-3}.
        \]
        且
        \[
        R_{n+1}=\frac{a_{n+1}+\tfrac12}{a_{n+1}-3}=-\frac{2}{5}R_n,
        \]
        即 $\{R_n\}$ 为公比$-\dfrac{2}{5}$的等比数列,由于$R_1=\frac{a_1+\frac12}{a_1-3}=-\frac{2}{5}$,故
        \[
        R_n=\left(-\frac{2}{5}\right)^n
        \]
        于是
        \[
        a_n=\frac{3R_n+\tfrac12}{R_n-1}=\frac{6\big(-\tfrac{2}{5}\big)^n+1}{2\Big(\big(-\tfrac{2}{5}\big)^n-1\Big)}
        =\frac{5^n+6(-2)^n}{2\big((-2)^n-5^n\big)}.
        \]
    \end{solution}

    \question 设实数数列 \(\{a_n\}\) 满足
    \[
    a_n a_{n+2} - a_{n+1}^2 - (n+1) a_n a_{n+1} = 0。
    \]
    已知 \(a_1 = 1, a_2 = 2018\),求
    \[
    \frac{a_{2018} \cdot a_{2016}}{a_{2017}^2}
    \]
    \begin{solution}
        由递推关系得
        \[
        \frac{a_{n+2}}{a_{n+1}} = \frac{a_{n+1}}{a_n} + (n+1)
        \]
        由$\dfrac{a_2}{a_1}=2018$,有
        \[
        \dfrac{a_{n+1}}{a_n} = 2018 + \sum_{k=2}^n k = 2017 + \frac{n(n+1)}{2}
        \]
        故
        \[
        \frac{a_{2018}a_{2016}}{a_{2017}^2} = 
        \frac{2017 \left(1 + \frac{2018}{2} \right)}{2017 \left(1 + \frac{2016}{2} \right)} = \dfrac{1010}{1009}
        \]
    \end{solution}

    \question 已知数列 $\{a_n\}$ 由递推关系定义如下:
\[
a_0 = 1, \quad a_{n+1} = \frac{a_n}{2(n+1)}, \quad n \ge 0,
\]
若函数 $f$ 定义为
\[
f(x) = \sum_{n=0}^{\infty} a_n x^n,
\]
求 $f(2)$ 的精确值。

\begin{solution}
根据递推关系,可以看出所有奇数项 $a_{2n+1} = 0$。偶数项为
\[
a_0 = 1, \quad a_2 = \frac{1}{2}, \quad a_4 = \frac{1}{8}, \quad a_6 = \frac{1}{48}, \dots
\]
满足公式
\[
a_{2n} = \frac{1}{2^n n!}.
\]

因此函数 $f$ 可表示为
\[
f(x) = \sum_{n=0}^{\infty} a_n x^n = \sum_{n=0}^{\infty} a_{2n} x^{2n} = \sum_{n=0}^{\infty} \frac{1}{2^n n!} x^{2n} = \sum_{n=0}^{\infty} \frac{1}{n!} \left(\frac{x^2}{2}\right)^n = e^{x^2/2}.
\]

所以
\[
f(2) = e^{2^2/2} = e^2.
\]
\end{solution}

\question 设 $\{b_n\}_{n=0}^\infty$ 为一列正实数,且 $b_0=1$,
\[
b_n=2+\sqrt{b_{n-1}}-2\sqrt{1+\sqrt{b_{n-1}}}。
\]
计算
\[
\sum_{n=1}^\infty b_n 2^n。
\]

\begin{solution}
令 $a_n=1+\sqrt{b_n}$,其中 $n\ge0$。则 $a_n>1$,$a_0=2$,并且
\[
a_n=1+\sqrt{1+a_{n-1}-2\sqrt{a_{n-1}}}=\sqrt{a_{n-1}}。
\]
因此
\[
a_n=2^{2^{-n}}。
\]

于是
\begin{align*}
\sum_{n=1}^N b_n 2^n
&=\sum_{n=1}^N (a_n-1)^2 2^n \\
&=\sum_{n=1}^N \bigl(a_n^2 2^n-a_n 2^{n+1}+2^n\bigr) \\
&=\sum_{n=1}^N \bigl((a_{n-1}-1)^2 2^n-(a_n-1)^2 2^{n+1}\bigr) \\
&=(a_0-1)^2 2-(a_N-1)^2 2^{N+1} \\
&=2-2\frac{2^{2^{-N}}-1}{2^{-N}}。
\end{align*}

令 $x=2^{-N}$,当 $N\to\infty$ 时有 $x\to0$,从而
\[
\sum_{n=1}^\infty b_n 2^n
=\lim_{N\to\infty}\left(2-2\frac{2^{2^{-N}}-1}{2^{-N}}\right)
=\lim_{x\to0}\left(2-2\frac{2^x-1}{x}\right)
=2-2\ln2。
\]
\end{solution}

    
    \question 一列 11 个正实数 $a_1, a_2, \dots, a_{11}$ 满足 $a_1 = 4$, $a_{11} = 1024$,并且对 $2 \le n \le 11$ 有
    \[
    a_n + a_{n-1} = \frac{5}{2} \sqrt{a_n a_{n-1}}.
    \]
    求满足条件的序列数量 $S$ 。
    \begin{solution}
        设 $a_n = x,a_{n-1} = y$,解
        \[
        x+y = \frac{5}{2}\sqrt{xy}
        \]
        得
        \[
        x = 4y \text{ 或 } \ x = \frac{y}{4}.
        \]
        考虑二叉树,每一步可以乘以 4 或除以 4。从 $a_1=4$ 到 $a_{11}=1024$ 共 10 步:
        \[
        4^m \cdot 4^{-(10-m)} = 4^4 \Rightarrow m =7
        \]
        即有 7 步乘以 4,3 步除以 4。可能序列数为选择 3 步除以 4 的位置数:
        \[
        S = \comb{10}{3} = 120
        \]
    \end{solution}

    \question 已知复数列 $\{z_{n}\}$ 满足
        $z_{1}=\dfrac{\sqrt{3}}{2},z_{n+1}=\overline{z_{n}}(1+z_{n}i)$,其中$n\in\mathbb{N}$,求 $z_{2021}$ 的值。
    \begin{solution}
        对 $n\in \mathbb{N}$ ,设 $z_{n}=a_{n}+b_{n}i,\ a_{n},b_{n}\in \mathbb{R}$ ,则
        $$a_{n+1}+b_{n+1}i=z_{n+1}=\overline{z_{n}}(1+z_{n}i)=\overline{z_{n}}+|z_{n}|^{2}i=a_{n}-b_{n}i+(a_{n}^{2}+b_{n}^{2})i$$
        因此 $$a_{n+1}=a_{n}=a_1=\frac{\sqrt{3}}{2}, b_{n+1}=a_{n}^{2}+b_{n}^{2}-b_{n}=b_{n}^{2}-b_{n}+\frac{3}{4}$$
        即 $$b_{n+1}-\frac{1}{2}=b_{n}^{2}-b_{n}+\frac{1}{4}=(b_{n}-\frac{1}{2})^{2}$$
        所以当 $n\ge2$,
        $$b_{n}=\frac{1}{2}+(b_{1}-\frac{1}{2})^{2^{n-1}}=\frac{1}{2}+(-\frac{1}{2})^{2^{n-1}}=\frac{1}{2}+\frac{1}{2^{2^{n-1}}}$$
        于是
        $$z_{2021}=a_{2021}+b_{2021}i=\frac{\sqrt{3}}{2}+(\frac{1}{2}+\frac{1}{2^{2^{2020}}})i$$
    \end{solution}

    \question 已知 $$a_0 = \frac{1}{2},a_n = \left(\frac{1 + a_{n-1}}{2}\right)^{\frac{1}{2}},\forall n \in \mathbb{N},$$求 $\displaystyle \lim_{n \rightarrow \infty} 4^n \cdot (1 - a_n)$。
    \begin{solution}
        由
        \[
        \cos \theta = \left(\frac{1 + \cos 2\theta}{2}\right)^{\frac{1}{2}}
        \]
        可得 $a_0 = \cos \dfrac{\pi}{3} = \dfrac{1}{2}$,从而
        \[
        a_1 = \cos \frac{\pi}{3 \cdot 2} = \left(\frac{1 + \frac{1}{2}}{2}\right)^{\frac{1}{2}} = \frac{\sqrt{3}}{2},  a_2 = \cos \frac{\pi}{3 \cdot 2^2}, \cdots,  a_n = \cos \frac{\pi}{3 \cdot 2^n}
        \]
        由泰勒展开,
        \[
        a_n = \cos \frac{\pi}{3 \cdot 2^n} = 1 - \frac{\left(\frac{\pi}{3 \cdot 2^n}\right)^2}{2!} + \frac{\left(\frac{\pi}{3 \cdot 2^n}\right)^4}{4!} - \cdots
        \]
        因此
        \[
        \lim_{n \to \infty} 4^n \cdot (1 - a_n)
        = \lim_{n \to \infty} 4^n \cdot \left( \frac{\left(\frac{\pi}{3 \cdot 2^n}\right)^2}{2!} - \frac{\left(\frac{\pi}{3 \cdot 2^n}\right)^4}{4!} + \cdots \right)
        = \frac{\pi^2}{9} \cdot \frac{1}{2!}
        = \frac{\pi^2}{18}
        \]
    \end{solution}
    
    \question Let $a_1,a_2,\dots$ be a sequence of real numbers satisfying
$$\frac{a_{n+1}}{a_n}-\frac{a_{n+2}}{a_n}-\frac{a_{n+1}a_{n+2}}{a_n^2}=\frac{na_{n+2}a_{n+1}}{a_n}.$$Given that $a_1=-1$ and $a_2=-\tfrac{1}{2}$, find the value of $\tfrac{a_9}{a_{20}}$.
    \begin{solution}
        \textcolor{red}{(待解)}
    \end{solution}

    \question 数列 $1, 2, 2, 3, 3, 3, 4, 4, 4, 4, 5, \dots$ 的第 $100$ 项是?
    \begin{solution}
        一般地, 数字 $n$ 出现 $n$ 项,因此从数字 $1$ 到数字 $n$ 共出现
        \[
        1+2+\dots+n = \frac{n(n+1)}{2}
        \]
        项,由 $\dfrac{n(n+1)}{2} \leq 100$ 得 $n<14$,取 $n=13$ 时共有 $91$ 项,因此第 $100$ 项为 $14$。
    \end{solution}

    \question 等差数列 $1, 3, 5, 7, 9, 11,\dots$ 按如下方法分组
    \[
    (1),(3,5),(7,9,11),(13,15,17,19),\dots
    \]
    求第 $n$ 组中 $n$ 个数的和 $S_n$。
    \begin{solution}
        等差数列 $1,3,5,7,\dots$ 的通项公式为
        \[
        a_n = 2n-1.
        \]
        第 $n$ 组有 $n$ 个奇数, 因此第 $n$ 组的第 $n$ 个奇数是等差数列中的第
        \[
        1+2+\dots+n = \frac{n(n+1)}{2}
        \]
        个奇数,且第 $n$ 组的第$1$个奇数是等差数列中的第
        \[
        \frac{n(n+1)}{2}-(n-1)=\frac{n^2-n+2}{2}
        \]
        故第$n$组中第 $n$ 个奇数是
        \[
        2\cdot \frac{n(n+1)}{2}-1=n^2+n-1
        \]
        第$n$组中第 $1$ 个奇数是
        \[
        2\cdot \frac{n^2-n+2}{2}-1=n^2-n+1
        \]
        所以
        \[
        S_n=\frac{n[(n^2+n-1)-(n^2-n+1)]}{2}=n^3
        \]
    \end{solution}

    \question 已知数列 $\{a_n\}$, 且 $a_1=1$, 定义 $S_n=n^2 a_n$. 求 $a_n$ 和 $S_n$。
    \begin{solution}
        由
        \[
        a_n = S_n - S_{n-1} =n^2 a_n - (n-1)^2 a_{n-1} \Rightarrow (n^2-1)a_n = (n-1)^2 a_{n-1}
        \]
        于是
        \[
        \frac{a_n}{a_{n-1}} =\frac{n-1}{n+1}.
        \]
        因此
        \[
        a_n = \frac{a_n}{a_{n-1}} \cdot \frac{a_{n-1}}{a_{n-2}}\cdot\cdots\cdot \frac{2}{4}\cdot \frac{1}{3}\cdot 1 = \frac{2}{n(n+1)} 
        \]
        且
        \[
        S_n =n^2 a_n= \frac{2n}{n+1}.
        \]
    \end{solution}

    \question 已知等差数列 $\{a_n\}$ 公差 $d=2$, $a_1+a_2+\dots+a_{100}=100$, 求 $a_4+a_8+\dots+a_{100}$ 的值。
    \begin{solution}
        由
        \[
        a_1+a_2+\dots+a_{100}=100,
        \]
        解得
        \[
        100a_1 + \frac{100 \cdot 99}{2} \cdot 2 = 100\Rightarrow a_1=-98
        \]
        于是$a_4=-92,a_100=100$,
        \[
        a_4+a_8+\dots+a_{100} = \frac{25}{2}(-92+100)=100.
        \]
    \end{solution}

    \question 数列$\{a_n\}$中, $a_1=1$ 且 $a_n a_{n+1}=4^n$, 求前$n$项和$S_n$。
    \begin{solution}
        由题意,
        \[
        \frac{a_{n+2}}{a_n} = \frac{a_{n+2}}{a_{n+1}} \cdot \frac{a_{n+1}}{a_{n}} = 4,
        \]
        故 $\{a_1,a_3,a_5,\dots\}$ 与 $\{a_2,a_4,a_6,\dots\}$ 皆为公比为 $4$的等比数列,其中 
        \[
        a_1=1,a_2=4,
        \]
        当 $n$ 为奇数,
        \begin{align*}
        S_n 
        &= (a_1+a_3+\cdots+a_n) + (a_2+a_4+\cdots+a_{n-1}) \\
        &= \frac{1(1-4^{\frac{n+1}{2}})}{1-4} + \frac{4(1-4^{\frac{n-1}{2}})}{1-4} = \frac{2^{n+1}-1}{3} + \frac{2^{n+1}-4}{3} = \frac{4}{3}\cdot 2^n - \frac{5}{3}.
        \end{align*}
        当 $n$ 为偶数,
        \begin{align*}
        S_n 
        &= (a_1+a_3+\cdots+a_{n-1}) + (a_2+a_4+\cdots+a_n) \\
        &= \frac{1(1-4^{\frac{n}{2}})}{1-4} + \frac{4(1-4^{\frac{n}{2}})}{1-4} = \frac{5}{3}\cdot 2^n - \frac{5}{3}.
        \end{align*}
        综上,
        \[
        S_n = \frac{9+(-1)^n}{3}\cdot 2^{n-1} - \frac{5}{3}
        \]
    \end{solution}

    \question 15.(浙江18)(本题14分)已知数列 $\{x_{n}\}$ 的首项$x_{1}=3$,通项$x_{n}=2^{n}p+nq$ ($n\in\mathbf{N}^{*}$, $p,q$为常数),且$x_{1},x_{4},x_{5}$成等差数列,求:

(I) $p,q$的值;

(II) 数列 $\{x_{n}\}$ 的前$n$项的和$S_{n}$的公式。
\begin{solution}
    

(I) 解: 由$x_{1}=3$, 得$2p+q=3$,

又$x_{4}=2^{4}p+4q$, $x_{5}=2^{5}p+5q$, 且$x_{1}+x_{5}=2x_{4}$, 得

$3+2^{5}p+5q=2(2^{4}p+4q)$,

解得$p=1, q=1$.

(II) 解: $S_{n}=(2^{1}+2^{2}+\cdots+2^{n})+(1+2+\cdots+n)$

$= 2^{n+1}-2+\frac{n(n+1)}{2}$.
\end{solution}
    \question
设数列 $\{a_n\}$ 的前 $n$ 项和为
\[
S_n=2a_n-2^n.
\]

(I) 求 $a_1,a_4$;

(II) 证明:$\{a_{n+1}-2a_n\}$ 是等比数列;

(III) 求数列 $\{a_n\}$ 的通项公式。

\begin{solution}

(I) 当 $n=1$ 时,
\[
S_1=a_1=2a_1-2,
\]
解得
\[
a_1=2,\ S_1=2.
\]

由
\[
S_n=2a_n-2^n
\]
得
\[
2a_n=S_n+2^n.
\]
于是
\[
2a_{n+1}=S_{n+1}+2^{n+1}=a_{n+1}+S_n+2^{n+1},
\]
从而
\[
a_{n+1}=S_n+2^{n+1}.
\]

因此
\[
a_2=S_1+2^2=6,\ S_2=8,
\]
\[
a_3=S_2+2^3=16,\ S_3=24,
\]
\[
a_4=S_3+2^4=40.
\]

(I) 得 $a_1=2,\ a_4=40$。

(II) 由上式可得
\begin{align*}
a_{n+1}-2a_n
&=(S_n+2^{n+1})-(S_n+2^n) \\
&=2^n.
\end{align*}
故数列 $\{a_{n+1}-2a_n\}$ 是首项为 $2$、公比为 $2$ 的等比数列。

(III) 由
\[
a_{n+1}-2a_n=2^n
\]
得
\[
a_n=(a_n-2a_{n-1})+2(a_{n-1}-2a_{n-2})+\cdots+2^{n-2}(a_2-2a_1)+2^{n-1}a_1.
\]
代入 $a_1=2,\ a_2=6$,并利用 $a_{k+1}-2a_k=2^k$,可得
\[
a_n=2^{n-1}(n+1).
\]

\end{solution}

\question
18.(陕西20)(本小题满分12分)

已知数列 $\{a_n\}$ 的首项 $a_1=\frac{2}{3}$,且
\[
a_{n+1}=\frac{2a_n}{a_n+1}\quad (n=1,2,3,\dots).
\]

(I) 证明:数列 $\left\{\frac{1}{a_n}-1\right\}$ 是等比数列;

(II) 求数列 $\{a_n\}$ 的前 $n$ 项和 $S_n$。

\begin{solution}

(I) 由
\[
a_{n+1}=\frac{2a_n}{a_n+1}
\]
得
\[
\frac{1}{a_{n+1}}=\frac{a_n+1}{2a_n}
=\frac{1}{2}+\frac{1}{2}\cdot\frac{1}{a_n}.
\]
于是
\[
\frac{1}{a_{n+1}}-1=\frac{1}{2}\left(\frac{1}{a_n}-1\right).
\]
又
\[
a_1=\frac{2}{3},\quad \frac{1}{a_1}-1=\frac{1}{2}.
\]
因此,数列 $\left\{\frac{1}{a_n}-1\right\}$ 是首项为 $\frac{1}{2}$、公比为 $\frac{1}{2}$ 的等比数列。

(II) 由 (I) 可得
\[
\frac{1}{a_n}-1=\frac{1}{2^{\,n}},
\]
即
\[
\frac{1}{a_n}=1+\frac{1}{2^{\,n}},
\]
从而
\[
a_n=\frac{2^{\,n}}{2^{\,n}+1}.
\]

于是
\[
S_n=\sum_{k=1}^n \frac{2^{\,k}}{2^{\,k}+1}
= \sum_{k=1}^n \left(1-\frac{1}{2^{\,k}+1}\right)
= n-\sum_{k=1}^n \frac{1}{2^{\,k}+1}.
\]

设
\[
T_n=\sum_{k=1}^n \frac{k}{2^{\,k}},
\]
则
\[
\frac{1}{2}T_n=\sum_{k=1}^{n-1}\frac{k}{2^{\,k+1}}+\frac{n}{2^{\,n+1}}.
\]
两式相减得
\[
\frac{1}{2}T_n=\frac{1}{2}+\frac{1}{2^2}+\cdots+\frac{1}{2^{\,n}}-\frac{n}{2^{\,n+1}}.
\]
由等比数列求和公式,
\[
\frac{1}{2}T_n
=1-\frac{1}{2^{\,n}}-\frac{n}{2^{\,n+1}},
\]
从而
\[
T_n=2-\frac{1}{2^{\,n-1}}-\frac{n}{2^{\,n}}.
\]

又
\[
1+2+\cdots+n=\frac{n(n+1)}{2},
\]
可得数列 $\left\{\frac{k}{2^{\,k}}\right\}$ 的前 $n$ 项和为
\[
2-\frac{n+2}{2^{\,n}}.
\]
因此
\[
S_n=\frac{n(n+1)}{2}+2-\frac{n+2}{2^{\,n}}
=\frac{n^2+n+4}{2}-\frac{n+2}{2^{\,n}}.
\]
\end{solution}

\question 6.(江西19) 等差数列 $\{a_n\}$ 的各项均为正数, $a_1=3$, 前 $n$ 项和为 $S_n$, $\{b_n\}$ 为等比数列, $b_1=1$, 且
\[
b_2S_2=64,\quad b_3S_3=960.
\]
(1) 求 $a_n$ 与 $b_n$;

(2) 求 $\frac{1}{S_1}+\frac{1}{S_2}+\dots+\frac{1}{S_n}$。

\begin{solution}
(1) 设等差数列 $\{a_n\}$ 的公差为 $d$, 等比数列 $\{b_n\}$ 的公比为 $q$。
则
\[
a_n=3+(n-1)d,\quad b_n=q^{n-1}.
\]
由
\[
S_2=3+(3+d)=6+d,\quad S_3=3+(3+d)+(3+2d)=9+3d,
\]
代入已知条件得
\[
\begin{cases}
(6+d)q=64,\\
(9+3d)q^2=960.
\end{cases}
\]
解得
\[
\begin{cases}
d=2,\\
q=8
\end{cases}
\quad\text{或}\quad
\begin{cases}
d=-\frac{6}{5},\\
q=\frac{40}{3}.
\end{cases}
\]
因 $\{a_n\}$ 各项均为正数, 舍去第二组解。
故
\[
a_n=3+2(n-1)=2n+1,\quad b_n=8^{\,n-1}.
\]

(2) 由 $a_n=2n+1$ 得
\[
S_n=3+5+\dots+(2n+1)=n(n+2).
\]
因此
\[
\frac{1}{S_1}+\frac{1}{S_2}+\dots+\frac{1}{S_n}
=\frac{1}{1\times3}+\frac{1}{2\times4}+\dots+\frac{1}{n(n+2)}.
\]
将其拆分为
\[
\frac{1}{k(k+2)}=\frac{1}{2}\left(\frac{1}{k}-\frac{1}{k+2}\right),
\]
于是
\begin{align*}
\sum_{k=1}^n \frac{1}{k(k+2)}
&=\frac{1}{2}\left(1-\frac{1}{3}+\frac{1}{2}-\frac{1}{4}+\dots+\frac{1}{n}-\frac{1}{n+2}\right)\\
&=\frac{1}{2}\left(1+\frac{1}{2}-\frac{1}{n+1}-\frac{1}{n+2}\right)\\
&=\frac{3}{4}-\frac{2n+3}{2(n+1)(n+2)}.
\end{align*}
\end{solution}

    \question 设$S_n$是等差数列$\{a_n\}$前$n$项的和,已知$\dfrac{1}{3}S_3$与$\dfrac{1}{4}S_4$的等比中项为$\dfrac{1}{5}S_5$,等差中项为$1$,求等差数列的通项$a_n$。
    \begin{solution}
        由$(S_3)(S_4)=(S_5)^2$及$\dfrac{1}{3}S_3+\dfrac{1}{4}S_4=2$得
        \[
        \begin{cases} 
        \displaystyle \frac{1}{3}\left(3a+\frac{3\cdot 2}{2}d\right)\cdot\frac{1}{4}\left(4a+\frac{4\cdot3}{2}d\right)=\frac{1}{25}\left(5a+\frac{5\cdot4}{2}d\right)^2 \\[2mm] 
        \displaystyle \frac{1}{3}\left(3a+\frac{3\cdot2}{2}d\right)+\frac{1}{4}\left(4a+\frac{4\cdot3}{2}d\right)=2 
        \end{cases}
        \]
        解得
        \[
        d=0,a=1 \;\; \text{或} \;\; d=-\frac{12}{5},a=4
        \]
        经检验得通项解为$a_n=1$或
        \[
        a_n=4-\frac{12}{5}(n-1)=\frac{32}{5}-\frac{12}{5}n.
        \]
    \end{solution}

    \question 已知数列 $\{a_n\}$ 中每一项均为正数,且数列前 $n$ 项之和为 $S_n$,若 $$\sum_{k=1}^{n}\frac{4S_k}{a_k+2}=S_n$$试求 $a_n$ 及 $S_n$。
    \begin{solution}
        由递推关系得
        \[
        S_1 = a_1 = \frac{4a_1}{a_1+2} 
        \Rightarrow a_1^2 - 2a_1 = 0 
        \]
        因\ $a_1 \ne 0$,解得 $a_1 = 2$,于是
        \[
        a_n = S_n - S_{n-1} = \sum_{k=1}^n \frac{4S_k}{a_k+2} - \sum_{k=1}^{n-1} \frac{4S_k}{a_k+2} = \frac{4S_n}{a_n+2} 
        \Rightarrow S_n = \frac{1}{4}a_n(a_n+2)
        \]
        又
        \[
        S_{n-1} = S_n-a_n= \frac{1}{4}a_n^2 - \frac{1}{2}a_n = \frac{1}{4}a_{n-1}(a_{n-1}+2)
        \]
        因此
        \[
        a_n^2 - 2a_n = a_{n-1}^2 + 2a_{n-1}
        \Rightarrow (a_n - 1)^2 = (a_{n-1} + 1)^2
        \Rightarrow a_n - 1 = a_{n-1} + 1
        \]
        即
        \[
        a_n = a_{n-1} + 2 = \cdots = a_1 + 2(n-1) = 2n, S_n = n^2 + n
        \]
    \end{solution}
    
    \question 已知数列$\{a_n\}$满足 $\begin{cases} a_0 = 1 \\ a_{2k+1} = a_k \\ a_{2k+2} = a_k + a_{k+1} \end{cases}, k \in N \cup \{0\}$, 求$\displaystyle \sum_{k=0}^{63} a_k$。
    \begin{solution}
        记
        \[
        S(n)=\sum_{k=0}^{2^n-1}a_k
        \]
        则
        \[
        S(n)=(a_0+a_2+a_4+\cdots+a_{2^n-2})+(a_1+a_3+\cdots+a_{2^n-1})
        \]
        其中偶数项为
        \[
        a_0+(a_0+a_1)+(a_1+a_2)+(a_2+a_3)+\cdots+(a_{2^{n-1}-2}+a_{2^{n-1}-1})
        =2(a_0+a_1+\cdots+a_{2^{n-1}-2})+a_{2^{n-1}-1}
        \]
        奇数项为
        \[
        a_0+a_1+a_2+\cdots+a_{2^{n-1}-1}
        \]
        所以
        \[
        S(n)=3(a_0+a_1+\cdots+a_{2^{n-1}-2})+2a_{2^{n-1}-1}
        =3(a_0+a_1+\cdots+a_{2^{n-1}-1})-a_{2^{n-1}-1}
        \]
        由 $a_{2^{n-1}-1}=1$,得
        \[
        S(n)=3S(n-1)-1
        \]
        于是
        \[
        S(8)=3S(7)-1=3^2S(6)-3-1=\cdots
        =3^5S(1)-(1+3+\cdots+3^4)
        \]
        又 $S(1)=a_0+a_1=2$,所以
        \[
        S(8)=243\cdot 2-121=365
        \]
    \end{solution}

    \question 已知各项皆为正整数的数列$\{ a_n \}$的前 $n$ 项和为$S_n$, 且对任意正整数$n$有
    \[
    \sqrt{S_n} = \lambda(a_n-1)+1,
    \]
    其中$\lambda$为正实数。若$2a_2=a_1+a_3$, 试求数列$\{a_n\}$的一般项。
    \begin{solution}
        由$2a_2=a_1+a_3$,设
        \[
        d=a_2-a_1=a_3-a_2
        \]
        则
        \[
        \sqrt{S_2}-\sqrt{S_1}=\lambda(a_2-a_1)=\lambda(a_3-a_2)=\sqrt{S_3}-\sqrt{S_2} \Rightarrow 2\sqrt{S_2}=\sqrt{S_1}+\sqrt{S_3}
        \]
        两边平方得
        \[
        4S_2=S_1+S_3+2\sqrt{S_1S_3} \Rightarrow 4(2a_1+d)=4a_1+3d+2\sqrt{a_1(3a_1+3d)}
        \]
        解得$d=2a_1$,则
        \[
        S_1=a_1,\; S_2=4a_1,\; S_3=9a_1
        \]
        解
        \[
        \lambda={\sqrt{S_2}-\frac{1}{ a_2-1}}
        ={\sqrt{S_3}-\frac{1}{a_3-1}}
        \]
        得$a_1=1,\lambda=\dfrac12$,于是
        \begin{align*}
        \sqrt{S_n}&=\frac12(a_n-1)+1 \\[1mm]
        a_n=S_n-S_{n-1}&=2\sqrt{S_n}-1 \\
        (\sqrt{S_n}-1)^2&=S_{n-1} \\
        \sqrt{S_n}&=\sqrt{S_{n-1}}+1 \\[1mm]
        \frac{a_n+1}{2}&=\frac{a_{n-1}+1}{2}+1
        \end{align*}
        故
        \[
        a_n=a_{n-1}+2=a_{n-2}+4=\cdots=a_1+2(n-1)=2n-1
        \]
    \end{solution}

    \question 已知数列 $\{a_n\}$ 满足 $a_1=1$, $a_{n+1}=\dfrac{a_n^2}{2a_n+1}$, 求通项公式并证明
    \[
    \sum_{k=1}^{n}\frac{a_k}{1+a_k}<\frac{7}{8}.
    \]
    \begin{solution}
        由递推式
        \[
        a_{n+1}=\frac{a_n^2}{2a_n+1},
        \]
        可得
        \[
        \frac{1}{a_{n+1}}=\frac{2a_n+1}{a_n^2}=\frac{2}{a_n}+\frac{1}{a_n^2},
        \]
        从而
        \[
        1+\frac{1}{a_{n+1}}=1+\frac{2}{a_n}+\frac{1}{a_n^2}=\left(1+\frac{1}{a_n}\right)^2.
        \]
        取对数,得
        \[
        \log\!\left(1+\frac{1}{a_{n+1}}\right)=2\log\!\left(1+\frac{1}{a_n}\right).
        \]
        因此数列 $\left\{\log\!\left(1+\tfrac{1}{a_n}\right)\right\}$ 是首项为 $\log 2$、公比 $2$ 的等比数列,于是
        \[
        \log\!\left(1+\frac{1}{a_n}\right)=2^{\,n-1}\log 2=\log\!\left(2^{2^{n-1}}\right),
        \]
        即
        \[
        1+\frac{1}{a_n}=2^{2^{\,n-1}}\Rightarrow a_n=\frac{1}{2^{2^{\,n-1}}-1}.
        \]
        且有
        \[
        \frac{a_n}{1+a_n}=\frac{1}{2^{2^{\,n-1}}}.
        \]
        所以
        \[
        \sum_{k=1}^{n}\frac{a_k}{1+a_k}=\frac{1}{2}+\frac{1}{2^2}+\frac{1}{2^4}+\cdots+\frac{1}{2^{2^{\,n-1}}}.
        \]
        注意当 $n\geq 4$ 时,有
        \[
        2^{n-1} > n+1 \Rightarrow \left(\frac{1}{2}\right)^{2^{n-1}} < \left(\frac{1}{2}\right)^{n+1}.
        \]
        因此
        \begin{align*}
        \sum_{k=1}^{n}\frac{a_k}{1+a_k}
        &< \frac{1}{2}+\frac{1}{4}+\frac{1}{16}+\sum_{k=5}^{n+1}\left(\frac{1}{2}\right)^k \\
        &= \frac{1}{2}+\frac{1}{4}+\frac{1}{16}+\frac{\left(\tfrac{1}{2}\right)^5\bigl(1-(\tfrac{1}{2})^{n-3}\bigr)}{1-\tfrac{1}{2}} \\
        &= \frac{7}{8}-\left(\frac{1}{2}\right)^{n+1} < \frac{7}{8}
        \end{align*}
        当 $n=1,2,3$ 时直接检验,亦满足不等式,故
        \[
        \sum_{k=1}^{n}\frac{a_k}{1+a_k}<\frac{7}{8},\quad \forall n\in\mathbb{N}.
        \]
    \end{solution}

    \question 16. (重庆22)(本小题满分12分, (I) 小问6分. (II) 小问6分)

设各项均为正数的数列 $\{a_{n}\}$ 满足 $a_{1}=2, a_{n} = a_{n+1}^{\frac{3}{1}}+2$ ($n\in\mathbf{N}^{*}$).
(I) 若 $a_{2}=\frac{1}{4}$, 求$a_{3}, a_{4}$, 并猜想$a_{2008}$的值(不需证明);
(II) 若 $2\sqrt{2}\leq a_{1}a_{2}\cdots a_{n} < 4$ 对$n\geq 2$恒成立, 求$a_{2}$的值.
\begin{solution}
    

解: (I) 因$a_{1}=2, a_{2}=2^{-2}$, 故

由此有$a_{1}=2\left(-2\right)^{0}, a_{2}=2\left(-2\right)^{\frac{1}{4}}, a_{3}=2\left(-2\right)^{\frac{1}{2}}, a_{4}=2\left(-2\right)^{\frac{3}{4}}$,

从而猜想$a_{n}$的通项为

$a_{n} = 2\left(-2\right)^{\frac{n-1}{4}}$ ($n\in\mathbf{N}^{*}$),

所以$a_{2008}=2\left(-2\right)^{\frac{2007}{4}}$.

(II)令$x_{n}=\log_{2}a_{n}$, 则$a_{1}a_{2}\cdots a_{n}=2^{x_{1}+x_{2}+\cdots+x_{n}}$, 故只需求$x_{2}$的值。

设$S_{n}$表示$x_{2}$的前$n$项和, 则$a_{1}a_{2}\cdots a_{n}=2^{S_{n}}$, 由$2\sqrt{2}\leq a_{1}a_{2}\cdots a_{n}<4$得

$2^{\frac{3}{2}} \leq 2^{S_{n}} < 2^{2}$ ($n\geq 2$).

因上式对$n=2$成立, 可得$\frac{3}{2}\leq S_{2} < 2$, 又由$a_{1}=2$, 得$x_{1}=1$, 故$x_{2}\geq\frac{1}{2}$.

由于$a_{1}=2$, $a_{n} = a_{n+1}+\frac{3}{2^{n+2}}$ ($n\in\mathbf{N}^{*}$), 得 $x_{n} = x_{n+1}+\frac{1}{2^{n+2}}$ ($n\in\mathbf{N}^{*}$), 即
$\frac{1}{x_{n+2}+2x_{n+1}} = (\frac{1}{x_{n+2}+x_{n+1}})+\frac{1}{2}\cdot\frac{1}{x_{n+1}+x_{n}} = \frac{1}{2}(\frac{1}{x_{n+1}+2x_{n}})$,

因此数列 $\{\frac{1}{x_{n+1}+2x_{n}}\}$ 是首项为$\frac{1}{x_{2}+2x_{1}}$,公比为$\frac{1}{2}$的等比数列, 故

$\frac{1}{x_{n+1}+2x_{n}}=(\frac{1}{x_{2}+2}) \frac{1}{2^{n-1}}$ ($n\in\mathbf{N}^{*}$).

将上式对$n$求和得

$S_{n+1}-x_{1}+2S_{n}=(\frac{1}{x_{2}+2})(1+\frac{1}{2}+\cdots+\frac{1}{2^{n-1}})=(\frac{1}{x_{2}+2})(2-\frac{1}{2^{n-1}})$ ($n\geq 2$).

因$S_{n}<2$, $S_{n+1}<2$ ($n\geq 2$)且$x_{1}=1$, 故

$(\frac{1}{x_{2}+2})(2-\frac{1}{2^{n-1}})<5$ ($n\geq 2$).

因此$2x_{2}-1 < \frac{x_{2}+2}{2^{n-1}}$ ($n\geq 2$).

下证$x_{2}\leq\frac{1}{2}$, 若淆, 假设$x_{2}>\frac{1}{2}$, 则由上式知, 不等式
$2^{n-1}<\frac{x_{2}+2}{2x_{2}-1}$

对$n\geq 2$恒成立, 但这是不可能的, 因此$x_{2}\leq\frac{1}{2}$.

又$x_{2}\geq\frac{1}{2}$, 故$x_{2}=\frac{1}{2}$, 所以$a_{2}=2^{x_{2}}=\sqrt{2}$.

\end{solution}
    8.(辽宁20)(本小题满分12分)

在数列 $\{a_n\}$,$\{b_n\}$ 是各项均为正数的等比数列,设
\[
c_n=\frac{b_n}{a_n}\ (n\in \mathbf{N}^*).
\]

(I) 数列 $\{c_n\}$ 是否为等比数列?证明你的结论;

(II) 设数列 $\{\ln a_n\}$,$\{\ln b_n\}$ 的前 $n$ 项和分别为 $S_n$,$T_n$。
若 $a_1=2$,$\frac{S_n}{T_n}=\frac{n}{2n+1}$,求数列 $\{c_n\}$ 的前 $n$ 项和。

\noindent 解:

(I) $\{c_n\}$ 是等比数列。

证明:设 $\{a_n\}$ 的公比为 $q_1(q_1>0)$,$\{b_n\}$ 的公比为 $q_2(q_2>0)$,则
\[
\frac{c_{n+1}}{c_n}
=\frac{b_{n+1}}{a_{n+1}}\cdot\frac{a_n}{b_n}
=\frac{b_{n+1}}{b_n}\cdot\frac{a_n}{a_{n+1}}
=\frac{q_2}{q_1}\ne 0,
\]
故 $\{c_n\}$ 为等比数列。

(II) 因 $\{a_n\}$,$\{b_n\}$ 为等比数列,
\[
S_n=n\ln a_1+\frac{n(n-1)}{2}\ln q_1,\qquad
T_n=n\ln b_1+\frac{n(n-1)}{2}\ln q_2.
\]
由题意
\[
\frac{n\ln a_1+\frac{n(n-1)}{2}\ln q_1}
{n\ln b_1+\frac{n(n-1)}{2}\ln q_2}
=\frac{n}{2n+1},
\]
化简得对一切 $n$ 成立的恒等式
\[
(2\ln q_1-\ln q_2)n^2
+(4\ln a_1-\ln q_1-2\ln b_1+\ln q_2)n
+(2\ln a_1-\ln q_1)=0.
\]
于是
\[
\begin{cases}
2\ln q_1-\ln q_2=0,\\
4\ln a_1-\ln q_1-2\ln b_1+\ln q_2=0,\\
2\ln a_1-\ln q_1=0.
\end{cases}
\]
由 $a_1=2$ 解得
\[
q_1=4,\qquad q_2=16,\qquad b_1=8.
\]
故
\[
c_n=\frac{b_n}{a_n}
=\frac{8\cdot16^{\,n-1}}{2\cdot4^{\,n-1}}
=4^n.
\]
所以数列 $\{c_n\}$ 的前 $n$ 项和为
\[
c_1+c_2+\cdots+c_n
=4+4^2+\cdots+4^n
=\frac{4}{3}\left(4^n-1\right).
\]

\question(山东20)(本小题满分12分)

将数列 $\{a_n\}$ 中的所有项按每一行比上一行多一项的规则排成如下数表:
\[
\begin{array}{l}
a_1\\
a_2,\ a_3\\
a_4,\ a_5,\ a_6\\
a_7,\ a_8,\ a_9,\ a_{10}\\
\end{array}
\]

记表中的第一列数 $a_1,a_2,a_4,a_7,\cdots$ 构成的数列为 $\{b_n\}$,
$b_1=a_1=1$。$S_n$ 为数列 $\{b_n\}$ 的前 $n$ 项和,且满足
\[
\frac{2b_n}{b_nS_n-S_n^2}=1\quad(n\ge2).
\]

(I) 证明数列 $\left\{\frac{1}{S_n}\right\}$ 成等差数列,并求数列 $\{b_n\}$ 的通项公式;

(II) 上表中,若从第三行起,每一行中的数按从左到右的顺序均构成等比数列,且公比为同一个正数。
当 $a_{81}=-\frac{4}{91}$ 时,求上表中第 $k(k\ge3)$ 行所有项的和。
\begin{solution}
\noindent 解:

(I) 由已知,当 $n\ge2$ 时,
\[
\frac{2b_n}{b_nS_n-S_n^2}=1.
\]
又
\[
S_n=b_1+b_2+\cdots+b_n,\qquad b_n=S_n-S_{n-1}\ (n\ge2),
\]
代入得
\[
\frac{2(S_n-S_{n-1})}{(S_n-S_{n-1})S_n-S_n^2}=1.
\]
化简得
\[
-S_{n-1}S_n-S_{n-1}^2=2S_n-2S_{n-1}.
\]
于是
\[
\frac{1}{S_n}-\frac{1}{S_{n-1}}=\frac{1}{2}.
\]
又 $S_1=b_1=1$,故数列 $\left\{\frac{1}{S_n}\right\}$ 是首项为 $1$,公差为 $\frac12$ 的等差数列,
\[
\frac{1}{S_n}=1+\frac12(n-1)=\frac{n+1}{2},
\]
从而
\[
S_n=\frac{2}{n+1}.
\]
当 $n\ge2$ 时,
\[
b_n=S_n-S_{n-1}=\frac{2}{n+1}-\frac{2}{n}=-\frac{2}{n(n+1)}.
\]
因此
\[
b_n=
\begin{cases}
1,& n=1,\\[4pt]
-\dfrac{2}{n(n+1)},& n\ge2.
\end{cases}
\]

(II) 设从第三行起,每一行的公比为 $q(q>0)$。
因为
\[
1+2+\cdots+12=\frac{12\times13}{2}=78,
\]
所以表中第1行至第12行共含有数列 $\{a_n\}$ 的前 $78$ 项,
故 $a_{81}$ 在表中第 $13$ 行第 $3$ 列。
于是
\[
a_{81}=b_{13}q^2=-\frac{4}{91}.
\]
又
\[
b_{13}=-\frac{2}{13\times14},
\]
解得 $q=2$。

记表中第 $k(k\ge3)$ 行所有项的和为 $S$,则
\[
S=b_k\frac{1-q^{k}}{1-q}
=-\frac{2}{k(k+1)}\cdot\frac{1-2^{k}}{1-2}
=\frac{2}{k(k+1)}(1-2^{k}),\quad k\ge3.
\]
\end{solution}

    \question 7.(湖南20) 数列 $\{a_n\}$ 满足
\[
a_1=0,\ a_2=2,\ a_{n+2}=\left(1+\cos^2\frac{n\pi}{2}\right)a_n+4\sin^2\frac{n\pi}{2},\quad n=1,2,3,\dots
\]

(1) 求 $a_3,a_4$, 并求数列 $\{a_n\}$ 的通项公式;

(2) 设
\[
S_k=a_1+a_3+\dots+a_{2k-1},\quad
T_k=a_2+a_4+\dots+a_{2k},\quad
W_k=\frac{2S_k}{2+T_k}\ (k\in\mathbb{N}^*),
\]
求使 $W_k>1$ 的所有 $k$ 的值, 并说明理由。

\begin{solution}
(1) 由已知 $a_1=0,a_2=2$。

当 $n=1$ 时,
\[
a_3=\left(1+\cos^2\frac{\pi}{2}\right)a_1+4\sin^2\frac{\pi}{2}=0+4=4.
\]

当 $n=2$ 时,
\[
a_4=\left(1+\cos^2\pi\right)a_2+4\sin^2\pi=2a_2=4.
\]

一般地, 当 $n=2k-1\ (k\in\mathbb{N}^*)$ 时,
\[
a_{2k+1}=\left(1+\cos^2\frac{(2k-1)\pi}{2}\right)a_{2k-1}
+4\sin^2\frac{(2k-1)\pi}{2}
=a_{2k-1}+4,
\]
即
\[
a_{2k+1}-a_{2k-1}=4.
\]
因此 $\{a_{2k-1}\}$ 为首项为 $0$, 公差为 $4$ 的等差数列,
\[
a_{2k-1}=4(k-1).
\]

当 $n=2k\ (k\in\mathbb{N}^*)$ 时,
\[
a_{2k+2}=\left(1+\cos^2\frac{2k\pi}{2}\right)a_{2k}
+4\sin^2\frac{2k\pi}{2}
=2a_{2k}.
\]
因此 $\{a_{2k}\}$ 为首项为 $2$, 公比为 $2$ 的等比数列,
\[
a_{2k}=2^k.
\]

综上,
\[
a_n=
\begin{cases}
4(k-1), & n=2k-1\ (k\in\mathbb{N}^*),\\
2^k, & n=2k\ (k\in\mathbb{N}^*).
\end{cases}
\]

(2) 由(1)得
\[
S_k=0+4+\dots+4(k-1)=2k(k-1),
\]
\[
T_k=2+2^2+\dots+2^k=2^{k+1}-2.
\]
于是
\[
W_k=\frac{2S_k}{2+T_k}
=\frac{4k(k-1)}{2^{k+1}}
=\frac{k(k-1)}{2^{k-1}}.
\]

计算得
\[
W_1=0,\quad W_2=1,\quad W_3=\frac{3}{2},\quad
W_4=\frac{3}{2},\quad W_5=\frac{5}{4},\quad W_6=\frac{15}{16}.
\]

当 $k\ge6$ 时,
\begin{align*}
W_{k+1}-W_k
&=\frac{(k+1)k}{2^k}-\frac{k(k-1)}{2^{k-1}}\\
&=\frac{k(3-k)}{2^k}<0,
\end{align*}
故 $\{W_k\}$ 单调递减。又 $W_6<1$, 因此当 $k\ge6$ 时 $W_k<1$。

综上, 满足 $W_k>1$ 的所有 $k$ 为
\[
k=3,4,5.
\]
\end{solution}

    \question 证明由正实数组成的数列\(a_0, a_1, a_2, \ldots, a_n\) 是等比数列当且仅当
    \[
    \left(a_0 a_1 + a_1 a_2 + a_2 a_3 + \cdots + a_{n-1} a_n\right)^2 = \left(a_0^2 + a_1^2 + \cdots + a_{n-1}^2\right)\left(a_1^2 + a_2^2 + \cdots + a_n^2\right)。
    \]
    \begin{solution}
        充分条件:已知 $a_0,\, a_1,\, \ldots,\, a_n$ 是等比数列,设$a_k = a_0 r^k,\; 1 \leq k \leq n$, 于是
        \[
        a_0 a_1 + a_1 a_2 + \ldots + a_{n-1} a_n 
        = a_0^2 (r + r^3 + \ldots + r^{2n-1}) = a_0^2 r \left(1 + r^2 + \ldots + r^{2n-2}\right)
        \]
        且
        \begin{align*}
        (a_0^2 + a_1^2 + \ldots + a_{n-1}^2)(a_1^2 + a_2^2 + \ldots + a_n^2)
        &= a_0^4 (1 + r^2 + \ldots + r^{2n-2})(r^2 + r^4 + \ldots + r^{2n}) \\
        &= a_0^4 r^2 \left(1 + r^2 + \ldots + r^{2n-2}\right)^2
        \end{align*}
        故得证
        \[
        (a_0 a_1 + a_1 a_2 + \ldots + a_{n-1} a_n)^2 = (a_0^2 + a_1^2 + \ldots + a_{n-1}^2)(a_1^2 + a_2^2 + \ldots + a_n^2)
        \]
        必要条件:已知
        \[
        (a_0 a_1 + a_1 a_2 + \ldots + a_{n-1} a_n)^2 = (a_0^2 + a_1^2 + \ldots + a_{n-1}^2)(a_1^2 + a_2^2 + \ldots + a_n^2)
        \]
        考虑函数
        \[
        f(r) = \sum_{k=1}^n (a_k - r a_{k-1})^2 = r^2 \sum_{k=1}^n a_{k-1}^2 - 2r \sum_{k=1}^n a_k a_{k-1} + \sum_{k=1}^n a_k^2
        \]
        观察得 $f(r) \geq 0 \quad \forall r$,又关于$r$方程式的判别式为
        \[
         4 \left[ (a_0 a_1 + a_1 a_2 + \ldots + a_{n-1} a_n)^2 - (a_0^2 + a_1^2 + \ldots + a_{n-1}^2)(a_1^2 + a_2^2 + \ldots + a_n^2) \right] = 0
        \]
        ,所以得知$f(r) = 0$ 有重根 $r = r_0$,且只有当  
        \[
        a_k = r_0 a_{k-1},\quad 1 \leq k \leq n
        \]
        才成立,意味 $a_0,\, a_1,\, \ldots,\, a_n$ 是一个公比为$r_0$  的等比数列。
    \end{solution}

    \question
用标准求和公式证明:
\[
\sum_{r=n}^{2n} (n-r)^2 = \sum_{r=1}^{n} r^2.
\]

\begin{solution}
首先展开平方:
\[
\sum_{r=n}^{2n} (n-r)^2 = \sum_{r=n}^{2n} (n^2 - 2nr + r^2) = \sum_{r=n}^{2n} n^2 - 2n \sum_{r=n}^{2n} r + \sum_{r=n}^{2n} r^2.
\]

利用标准求和公式:
\[
\sum_{r=a}^{b} 1 = b-a+1, \quad \sum_{r=a}^{b} r = \frac{(b-a+1)(a+b)}{2}, \quad \sum_{r=a}^{b} r^2 = \frac{(b-a+1)(2a^2+2ab+b^2+a+b)}{6}.
\]

代入 $a=n, b=2n$:
\[
\sum_{r=n}^{2n} n^2 = n^2 (2n-n+1) = n^2 (n+1),
\]
\[
\sum_{r=n}^{2n} r = \frac{(n+1)(n+2n)}{2} = \frac{(n+1)(3n)}{2} = \frac{3n(n+1)}{2},
\]
\[
\sum_{r=n}^{2n} r^2 = \frac{(n+1)(n^2 + 2n^2 + (2n)^2 + \dots)}{6} \quad \text{(整理后为)} \frac{n(n+1)(2n+1)}{6} + \frac{其他项}{6}.
\]

代入到展开式:
\[
\sum_{r=n}^{2n} (n-r)^2 = n^2(n+1) - 2n \cdot \frac{3n(n+1)}{2} + \sum_{r=n}^{2n} r^2.
\]

化简系数:
\[
n^2(n+1) - 3n^2(n+1) + \sum_{r=n}^{2n} r^2 = -2n^2(n+1) + \sum_{r=n}^{2n} r^2.
\]

进一步整理 $\sum_{r=n}^{2n} r^2$ 并化简,可得:
\[
\sum_{r=n}^{2n} (n-r)^2 = \frac{n(n+1)(2n+1)}{6} = \sum_{r=1}^{n} r^2.
\]

\[
\therefore \sum_{r=n}^{2n} (n-r)^2 = \sum_{r=1}^{n} r^2
\]
如所要求。
\end{solution}

\question
设 $m,n \in \mathbb{N}$,证明:
\[
\sum_{r=m}^{m+n} (m-r)^2 = \sum_{r=0}^{n} r^2.
\]

\begin{solution}
首先展开平方:
\[
\sum_{r=m}^{m+n} (m-r)^2 = \sum_{r=m}^{m+n} (r-m)^2 \quad \text{(因为平方对符号不变)}.
\]

令 $k = r-m$,则 $r = k + m$,当 $r=m$ 时 $k=0$,当 $r=m+n$ 时 $k=n$,于是:
\[
\sum_{r=m}^{m+n} (m-r)^2 = \sum_{k=0}^{n} (-k)^2 = \sum_{k=0}^{n} k^2 = \sum_{r=0}^{n} r^2.
\]

\[
\therefore \sum_{r=m}^{m+n} (m-r)^2 = \sum_{r=0}^{n} r^2,
\]
如所要求。
\end{solution}
\question
\noindent
求和
\[
\sum_{n=2}^{20} \frac{n^3-n^2+1}{n^2-n}.
\]

\begin{solution}
首先将被加数进行拆分:
\[
\frac{n^3-n^2+1}{n^2-n} = \frac{n(n^2-n)+1}{n^2-n} = n + \frac{1}{n^2-n} = n + \frac{1}{n(n-1)}
\]

将分式拆成部分分式:
\[
\frac{1}{n(n-1)} = \frac{1}{n-1} - \frac{1}{n}
\]

因此
\[
\frac{n^3-n^2+1}{n^2-n} = n + \frac{1}{n-1} - \frac{1}{n}.
\]

写出前几项:
\[
n=2: 2 + 1 - \frac{1}{2}, \quad
n=3: 3 + \frac{1}{2} - \frac{1}{3}, \quad
n=4: 4 + \frac{1}{3} - \frac{1}{4}, \dots, \quad
n=20: 20 + \frac{1}{19} - \frac{1}{20}.
\]

求和时,分式部分是望远镜求和:
\[
\sum_{n=2}^{20} \left( \frac{1}{n-1} - \frac{1}{n} \right) = 1 - \frac{1}{20}.
\]

整数部分求和:
\[
\sum_{n=2}^{20} n = \frac{19}{2}(2+20) = 209.
\]

因此总和为:
\[
\sum_{n=2}^{20} \frac{n^3-n^2+1}{n^2-n} = 209 + 1 - \frac{1}{20} = 210 - \frac{1}{20} = \frac{4199}{20}.
\]
\end{solution}

        \question 试求级数
    \[
    \frac{1\cdot 3\cdot 5}{3^6-64} + \frac{3\cdot 5\cdot 7}{5^6-64} + \frac{5\cdot 7\cdot 9}{7^6-64} + \cdots + \frac{19\cdot 21\cdot 23}{21^6-64}
    \] 
    之和。
    \begin{solution}
        注意到
        \begin{align*}
        &\frac{(2k-1)(2k+1)(2k+3)}{(2k+1)^6 - 64}\\
        &= \frac{(2k-1)(2k+1)(2k+3)}{((2k+1)^3-8)((2k+1)^3+8)} \\
        &= \frac{(2k+1)}{((2k+1)^2 + 2(2k+1) + 4)((2k+1)^2 - 2(2k+1) + 4)}\\
        &= \frac{2k+1}{(4k^2 + 3)(4k^2 + 8k + 7)} \\
        &= \frac{2k+1}{(4k^2 + 3)(4(k+1)^2 + 3)}\\
        &= \frac{1}{4} \left( \frac{1}{4k^2 + 3} - \frac{1}{4(k+1)^2 + 3} \right)
        \end{align*}
        故\[
        \sum_{k=1}^{10} \frac{(2k-1)(2k+1)(2k+3)}{(2k+1)^6 - 64}= \frac{1}{4} \left( \frac{1}{7} - \frac{1}{487} \right)
        = \frac{120}{3409}
        \]
    \end{solution}

    \question 求和
    \[
    \sum_{n=1}^{24} \frac{1}{\sqrt{n + \sqrt{n^2 - 1}}} 
    \]
    \begin{solution}
        不难发现
        \[
        \frac{1}{\sqrt{n + \sqrt{n^2 - 1}}} = \sqrt{n - \sqrt{n^2 - 1}} = \frac{1}{\sqrt{2}}(\sqrt{n+1} - \sqrt{n-1})。
        \]
        进行裂项求和,
        \begin{align*}
            \sum_{n=1}^{24} \frac{1}{\sqrt{n + \sqrt{n^2 - 1}}} &= \frac{1}{\sqrt{2}} \sum_{n=1}^{24} (\sqrt{n+1} - \sqrt{n-1})\\
            &= \frac{1}{\sqrt{2}} (\sqrt{25} + \sqrt{24} - \sqrt{1} - \sqrt{0})\\
            &=2\sqrt{2} + 2\sqrt{3}
        \end{align*}
    \end{solution}

    \question 若 $f(n)=(n^{2}-2n+1)^{\frac{1}{3}}+(n^{2}-1)^{\frac{1}{3}}+(n^{2}+2n+1)^{\frac{1}{3}},$ 求 $$\sum_{k=1}^{500}\frac{1}{f(2k-1)}$$
    \begin{solution}
        首先有
        \begin{align*}
            \frac{1}{f(n)} &= \frac{1}{\sqrt[3]{(n-1)^2} +\sqrt[3]{(n+1)(n-1)} +\sqrt[3]{(n+1)^2}} \\
            &= \frac{\sqrt[3]{(n+1)} - \sqrt[3]{(n-1)}}{(n+1)-(n-1)} \\
            &= \frac{1}{2}(\sqrt[3]{(n+1)} - \sqrt[3]{(n-1)})
        \end{align*}
        故
        \[
        \sum_{k=1}^{500} \frac{1}{f(2k-1)} = \frac{1}{2} \sum_{k=1}^{500} \left( \sqrt[3]{2k} - \sqrt[3]{2k-2} \right) = \frac{1}{2} \cdot \sqrt[3]{1000} = 5
        \]
    \end{solution}

    \question 设数列 $\{a_n\}$ 的通项公式为
    \[
    a_n = \frac{1}{(\sqrt{n}+\sqrt{n+1})(\sqrt{n+1}+\sqrt{n+2})(\sqrt{n}+\sqrt{n+2})},
    \]
    求 $$\sum_{k=1}^{\infty} a_k$$ 的值。
    \begin{solution}
        发现
        \begin{align*}
        a_n &= \frac{1}{\sqrt{n+2}-\sqrt{n}} \left( \frac{1}{(\sqrt{n}+\sqrt{n+1})(\sqrt{n}+\sqrt{n+2})} - \frac{1}{(\sqrt{n}+\sqrt{n+2})(\sqrt{n+1}+\sqrt{n+2})} \right) \\
        &= \frac{1}{(\sqrt{n+2}-\sqrt{n})(\sqrt{n}+\sqrt{n+2})} \left( \frac{1}{\sqrt{n}+\sqrt{n+1}} - \frac{1}{\sqrt{n+1}+\sqrt{n+2}} \right) \\
        &= \frac{1}{2} \left( \frac{1}{\sqrt{n}+\sqrt{n+1}} - \frac{1}{\sqrt{n+1}+\sqrt{n+2}} \right)
        \end{align*}
        由$\lim_{n \to \infty} \dfrac{1}{\sqrt{n}+\sqrt{n+1}}=0,$故
        \[
        \sum_{k=1}^{\infty} a_k = \frac{1}{2} \left( \frac{1}{\sqrt{1}+\sqrt{2}} - \frac{1}{\sqrt{2}+\sqrt{3}} + \frac{1}{\sqrt{2}+\sqrt{3}} - \frac{1}{\sqrt{3}+\sqrt{4}} + \cdots \right)
        = \frac{1}{2} \cdot \frac{1}{1+\sqrt{2}} = \frac{\sqrt{2}-1}{2}
        \]
    \end{solution}
    
    \question 求 $$\sqrt{1+\frac{1}{1^{2}}+\frac{1}{2^{2}}}+\sqrt{1+\frac{1}{2^{2}}+\frac{1}{3^{2}}}+\dots+\sqrt{1+\frac{1}{2025^{2}}+\frac{1}{2026^{2}}}$$的值。
    \begin{solution}
        注意到
        \begin{align*}
        &\sqrt{1+\frac{1}{n^2} +\frac{1}{(n+1)^2}} 
        = \sqrt{1+\frac{(n+1)^2+n^2}{n^2(n+1)^2} } \\
        &= \sqrt{1+\frac{2n(n+2) +1}{n^2(n+1)^2} } 
        = \sqrt{1+\frac{2}{n(n+1)} +\frac{1}{n^2(n+1)^2} } \\
        &= \sqrt{\left(1+\frac{1}{n(n+1)}\right)^2} 
        = 1+\frac{1}{n(n+1)} 
        = 1+\frac{1}{n}-\frac{1}{n+1}
        \end{align*}
        因此原式为
        \[
        \sum_{n=1}^{2025} \left(1 + \frac{1}{n} - \frac{1}{n+1}\right)= 2025 + \left(1 - \frac{1}{2026} \right)=2016-\frac{1}{2026}
        \]
    \end{solution}

    \question
\noindent
(a) 已知
\[
f(x,n)=\sum_{r=1}^{n}\left[\frac{1}{(x-1)^r}\right]
\]
其中 $x\in\mathbb{R},\,n\in\mathbb{N}$。通过观察
\[
\frac{1}{(x-2)(x-1)^r}-\frac{1}{(x-2)(x-1)^{r+1}}
\]
的化简,求 $f(x,n)$ 的简化表达式。

\begin{solution}
先化简给定的差:
\[
\frac{1}{(x-1)^r(x-2)}-\frac{1}{(x-2)(x-1)^{r+1}}
=\frac{(x-1)-1}{(x-1)^{r+1}(x-2)}
=\frac{1}{(x-1)^{r+1}}
\]

因此
\[
\frac{1}{(x-1)^{r+1}}
=\frac{1}{(x-1)^r(x-2)}-\frac{1}{(x-1)^{r+1}(x-2)}
\]

对 $r=0,1,2,\dots,n-1$ 求和,
\[
\sum_{r=0}^{n-1}\frac{1}{(x-1)^{r+1}}
=\frac{1}{x-2}-\frac{1}{(x-1)^n(x-2)}
\]

即
\[
f(x,n)=\sum_{r=1}^{n}\frac{1}{(x-1)^r}
=\frac{1}{x-2}-\frac{1}{(x-1)^n(x-2)}
\]
\end{solution}

\noindent
(b) 当 $|x-1|>1$ 时,求
\[
\lim_{n\to\infty} f(x,n)
\]

\begin{solution}
由 (a) 得
\[
f(x,n)=\frac{1}{x-2}-\frac{1}{(x-1)^n(x-2)}
\]

当 $|x-1|>1$ 时,有
\[
\lim_{n\to\infty}\frac{1}{(x-1)^n}=0
\]

因此
\[
\lim_{n\to\infty} f(x,n)=\frac{1}{x-2}
\]
\end{solution}

    \question 求
\[
\sum_{r=2}^{\infty}\left[\frac{4r-1}{r(r-1)}\left(-\frac{1}{3}\right)^r\right]
\]

\begin{solution}
先对一般项作部分分式分解,有
\[
\frac{4r-1}{r(r-1)}=\frac{1}{r}+\frac{3}{r-1}
\]

因此
\[
\frac{4r-1}{r(r-1)}\left(-\frac{1}{3}\right)^r
=\frac{1}{r}\left(-\frac{1}{3}\right)^r+\frac{3}{r-1}\left(-\frac{1}{3}\right)^r
\]

将第二项改写为
\[
\frac{3}{r-1}\left(-\frac{1}{3}\right)^r
=\frac{3}{r-1}\left(-\frac{1}{3}\right)\left(-\frac{1}{3}\right)^{r-1}
=-\frac{1}{r-1}\left(-\frac{1}{3}\right)^{r-1}
\]

于是一般项可写成差分形式
\[
\frac{4r-1}{r(r-1)}\left(-\frac{1}{3}\right)^r
=\frac{1}{r}\left(-\frac{1}{3}\right)^r
-\frac{1}{r-1}\left(-\frac{1}{3}\right)^{r-1}
\]

考虑前 $n$ 项和
\[
\sum_{r=2}^{n}\left[\frac{4r-1}{r(r-1)}\left(-\frac{1}{3}\right)^r\right]
=\sum_{r=2}^{n}\left[
\frac{1}{r}\left(-\frac{1}{3}\right)^r
-\frac{1}{r-1}\left(-\frac{1}{3}\right)^{r-1}
\right]
\]

这是一个伸缩和,化简得
\[
\sum_{r=2}^{n}\left[\frac{4r-1}{r(r-1)}\left(-\frac{1}{3}\right)^r\right]
=\frac{1}{n}\left(-\frac{1}{3}\right)^n+\frac{1}{3}
\]

令 $n\to\infty$,由于
\[
\frac{1}{n}\left(-\frac{1}{3}\right)^n\to 0
\]
从而
\[
\sum_{r=2}^{\infty}\left[\frac{4r-1}{r(r-1)}\left(-\frac{1}{3}\right)^r\right]
=\frac{1}{3}
\]
\end{solution}

\question 求级数的和
\[
\frac{1}{4\times 2!} + \frac{1}{5\times 3!} + \frac{1}{6\times 4!} + \frac{1}{7\times 5!} + \frac{1}{8\times 6!} + \cdots
\]

\begin{solution}

\noindent
将级数写成求和符号形式:
\[
S_\infty = \sum_{r=1}^{\infty} \frac{1}{(r+3)(r+1)!}
\]

\noindent
对一般项进行拆项:
\[
\frac{1}{(r+3)(r+1)!} = \frac{1}{(r+2)!} - \frac{1}{(r+3)!}
\]

\noindent
因此有限部分和为:
\begin{align*}
\sum_{r=1}^{N} \frac{1}{(r+3)(r+1)!}
&= \sum_{r=1}^{N} \left( \frac{1}{(r+2)!} - \frac{1}{(r+3)!} \right) \\
&= \frac{1}{3!} - \frac{1}{(N+3)!}
\end{align*}

\noindent
令 $N \to \infty$,由于
\[
\frac{1}{(N+3)!} \to 0
\]
可得无穷级数的和:
\[
S_\infty = \frac{1}{3!} = \frac{1}{6}
\]

\end{solution}

\question
已知
\[
S_n=(2\times1!)+(5\times2!)+(10\times3!)+(17\times4!)+\cdots+(n^2+1)n!
\]
用适当的方法证明
\[
S_n=n(n+1)!。
\]

\begin{solution}
先将数列写成求和形式:
\[
S_n=\sum_{r=1}^{n}(r^2+1)r!。
\]

为了使用差分法,尝试将 $(r^2+1)r!$ 表示成阶乘的差。注意
\[
(r+2)!=(r+2)(r+1)r!=(r^2+3r+2)r!,
\]
因此
\[
(r+2)!-r!=(r^2+3r+1)r!。
\]

又因为
\[
(r+1)!-r!=r\cdot r!,
\]
于是
\[
(r+2)!-r!=(r^2+1)r!+3r\cdot r!。
\]

代入 $r\cdot r!=(r+1)!-r!$,得
\[
(r+2)!-r!=(r^2+1)r!+3[(r+1)!-r!],
\]
即
\[
(r+2)!-3(r+1)!+2r!=(r^2+1)r!。
\]

因此
\[
(r^2+1)r!=(r+2)!-3(r+1)!+2r!。
\]

将其代入求和式:
\[
\sum_{r=1}^{n}(r^2+1)r!
=\sum_{r=1}^{n}\big[(r+2)!-3(r+1)!+2r!\big]。
\]

写出部分项可以发现大量抵消,最后只剩下
\[
S_n=(n+2)!-2(n+1)!。
\]

提取 $(n+1)!$:
\[
S_n=[(n+2)-2](n+1)!。
\]

于是
\[
S_n=n(n+1)!。
\]
\end{solution}

\question
求
\[
\sum_{r=1}^{n}\left[\frac{r^2+9r+19}{(r+5)!}\right]
\]
的化简表达式,并由此求
\[
\sum_{r=1}^{\infty}\left[\frac{r^2+7r+11}{(r+4)!}\right]。
\]

\begin{solution}
先对一般项作拆分。注意到
\[
\frac{r^2+9r+19}{(r+5)!}
=\frac{1}{(r+3)!}-\frac{1}{(r+5)!}。
\]

因此
\[
\sum_{r=1}^{n}\left[\frac{r^2+9r+19}{(r+5)!}\right]
=\sum_{r=1}^{n}\left[\frac{1}{(r+3)!}-\frac{1}{(r+5)!}\right]。
\]

将各项写出:
\[
\begin{aligned}
r=1&:\ \frac{1}{4!}-\frac{1}{6!},\\
r=2&:\ \frac{1}{5!}-\frac{1}{7!},\\
&\ \vdots\\
r=n&:\ \frac{1}{(n+3)!}-\frac{1}{(n+5)!}.
\end{aligned}
\]

相加后发生抵消,得到
\[
\sum_{r=1}^{n}\left[\frac{r^2+9r+19}{(r+5)!}\right]
=\left(\frac{1}{4!}+\frac{1}{5!}\right)
-\left(\frac{1}{(n+4)!}+\frac{1}{(n+5)!}\right)。
\]

化简常数项:
\[
\frac{1}{4!}+\frac{1}{5!}
=\frac{5}{120}+\frac{1}{120}
=\frac{1}{20}。
\]

故
\[
\sum_{r=1}^{n}\left[\frac{r^2+9r+19}{(r+5)!}\right]
=\frac{1}{20}-\left(\frac{1}{(n+4)!}+\frac{1}{(n+5)!}\right)。
\]

再考虑
\[
\sum_{r=1}^{\infty}\left[\frac{r^2+7r+11}{(r+4)!}\right]。
\]

令 $k=r-1$,则
\[
\frac{r^2+7r+11}{(r+4)!}
=\frac{k^2+9k+19}{(k+5)!}。
\]

于是
\[
\sum_{r=1}^{n}\left[\frac{r^2+7r+11}{(r+4)!}\right]
=\sum_{k=0}^{n-1}\left[\frac{k^2+9k+19}{(k+5)!}\right]。
\]

由前一结果可得
\[
\sum_{r=1}^{n}\left[\frac{r^2+7r+11}{(r+4)!}\right]
=\frac{25}{5!}-\frac{n+6}{(n+5)!}。
\]

令 $n\to\infty$,因
\[
\frac{n+6}{(n+5)!}\to 0,
\]
故
\[
\sum_{r=1}^{\infty}\left[\frac{r^2+7r+11}{(r+4)!}\right]
=\frac{25}{5!}
=\frac{5}{24}。
\]
\end{solution}

    \question 求级数
\[
\sum_{n=0}^{\infty} \frac{1}{2011^{2^{n}} - 2011^{-2^{n}}}
= \frac{1}{2011^{1} - 2011^{-1}} + \frac{1}{2011^{2} - 2011^{-2}} + \frac{1}{2011^{4} - 2011^{-4}} + \cdots
\]
并将其表示为一个有理数。

\begin{solution}
更一般地,设
\[
S(x) = \sum_{n=0}^{\infty} \frac{x^{2^{n}}}{1 - x^{2^{n+1}}}, \qquad
S_N(x) = \sum_{n=0}^{N} \frac{x^{2^{n}}}{1 - x^{2^{n+1}}}.
\]
注意到题目中的级数正是 $S(1/2011)$。下面证明当 $0<x<1$ 时,
\[
S(x) = \frac{x}{1-x}.
\]

利用恒等式
\[
\frac{x^{2^{n}}}{1 - x^{2^{n+1}}}
= \frac{1}{1 - x^{2^{n}}} - \frac{1}{1 - x^{2^{n+1}}},
\]
代入部分和 $S_N(x)$,得
\begin{align*}
S_N(x)
&= \left(\frac{1}{1-x} - \frac{1}{1-x^{2}}\right)
 + \left(\frac{1}{1-x^{2}} - \frac{1}{1-x^{4}}\right)
 + \cdots  \\
&\quad + \left(\frac{1}{1-x^{2^{N}}} - \frac{1}{1-x^{2^{N+1}}}\right).
\end{align*}

这是一个望远镜求和,因此
\[
S_N(x) = \frac{1}{1-x} - \frac{1}{1-x^{2^{N+1}}}.
\]

当 $N \to \infty$ 时,由于 $0<x<1$,有 $x^{2^{N+1}} \to 0$,从而
\[
\lim_{N\to\infty} \frac{1}{1-x^{2^{N+1}}} = 1.
\]
于是
\[
S(x) = \lim_{N\to\infty} S_N(x)
= \frac{1}{1-x} - 1
= \frac{x}{1-x}.
\]

因此
\[
\sum_{n=0}^{\infty} \frac{1}{2011^{2^{n}} - 2011^{-2^{n}}}
= S\!\left(\frac{1}{2011}\right)
= \frac{\frac{1}{2011}}{1-\frac{1}{2011}}
= \frac{1}{2010}.
\]
\end{solution}

\question 求下列无穷乘积的值:
\[
\left(\frac{7}{9}\right) \cdot \left(\frac{26}{28}\right) \cdot \left(\frac{63}{65}\right) \cdot \dots = \lim_{n \to \infty} \prod_{k=2}^{n} \frac{k^3 - 1}{k^3 + 1}.
\]

\begin{solution}
将部分积改写为两个可望远镜相消的乘积:
\[
\prod_{k=2}^{n} \frac{k^3 - 1}{k^3 + 1} = \prod_{k=2}^{n} \frac{k-1}{k+1} \prod_{k=2}^{n} \frac{k^2+k+1}{k^2-k+1}.
\]

注意
\[
\prod_{k=2}^{n} \frac{k^2+k+1}{k^2-k+1} = \prod_{k=2}^{n} \frac{k^2+k+1}{(k-1)^2 + (k-1) + 1}.
\]

同时
\[
\prod_{k=2}^{n} \frac{k-1}{k+1} = \frac{1\cdot 2}{3\cdot 4} \cdot \frac{2\cdot 3}{4\cdot 5} \cdots \frac{(n-1)n}{(n+1)n} = \frac{2}{n(n+1)}.
\]

望远镜消去后得到
\[
\prod_{k=2}^{n} \frac{k^3 - 1}{k^3 + 1} = \frac{2}{n(n+1)} \cdot \frac{n^2+n+1}{3} = \frac{2}{3} \left(1 + \frac{1}{n(n+1)}\right).
\]

因此
\[
\prod_{k=2}^{\infty} \frac{k^3 - 1}{k^3 + 1} = \lim_{n \to \infty} \frac{2}{3} \left(1 + \frac{1}{n(n+1)}\right) = \frac{2}{3}.
\]
\end{solution}

\question 计算无穷乘积
\[
\prod_{n=3}^\infty \frac{(n^3+3n)^2}{n^6-64}.
\]

\begin{solution}
设
\[
a_n=\frac{(n^3+3n)^2}{n^6-64}.
\]
注意到
\begin{align*}
a_n
&=\frac{(n^3+3n)^2}{(n^3-8)(n^3+8)}
=\frac{n^2(n^2+3)^2}{(n-2)(n^2+2n+4)(n+2)(n^2-2n+4)} \\
&=\frac{n}{n-2}\cdot\frac{n}{n+2}\cdot
\frac{n^2+3}{(n-1)^2+3}\cdot
\frac{n^2+3}{(n+1)^2+3}.
\end{align*}

因此当 $N\ge3$ 时,
\begin{align*}
\prod_{n=3}^N a_n
&=\left(\prod_{n=3}^N\frac{n}{n-2}\right)
\left(\prod_{n=3}^N\frac{n}{n+2}\right)
\left(\prod_{n=3}^N\frac{n^2+3}{(n-1)^2+3}\right)
\left(\prod_{n=3}^N\frac{n^2+3}{(n+1)^2+3}\right) \\
&=\frac{N(N-1)}{1\cdot2}\cdot
\frac{3\cdot4}{(N+1)(N+2)}\cdot
\frac{N^2+3}{2^2+3}\cdot
\frac{3^2+3}{(N+1)^2+3} \\
&=\frac{72}{7}\cdot
\frac{N(N-1)(N^2+3)}{(N+1)(N+2)((N+1)^2+3)} \\
&=\frac{72}{7}\cdot
\frac{(1-\frac1N)(1+\frac{3}{N^2})}
{(1+\frac1N)(1+\frac{2}{N})((1+\frac1N)^2+\frac{3}{N^2})}.
\end{align*}

令 $N\to\infty$,得到
\[
\prod_{n=3}^\infty a_n
=\lim_{N\to\infty}\prod_{n=3}^N a_n
=\frac{72}{7}.
\]
\end{solution}


\question 设
\[
f(r)=\sum_{j=2}^{2013}\frac{1}{j^r}
=\frac{1}{2^r}+\frac{1}{3^r}+\dots+\frac{1}{2013^r}。
\]
求
\[
\sum_{k=2}^{\infty} f(k)。
\]

\begin{solution}
交换求和次序,并注意到内层和是收敛的等比级数,有
\[
\sum_{k=2}^{\infty}\sum_{j=2}^{2013}\frac{1}{j^k}
=\sum_{j=2}^{2013}\sum_{k=2}^{\infty}\frac{1}{j^k}
=\sum_{j=2}^{2013}\frac{1/j^2}{1-1/j}。
\]
因此
\[
=\sum_{j=2}^{2013}\frac{1}{j^2-j}
=\sum_{j=2}^{2013}\frac{1}{j(j-1)}
=\sum_{j=2}^{2013}\left(\frac{1}{j-1}-\frac{1}{j}\right)。
\]
于是
\[
=\frac{1}{1}-\frac{1}{2}+\frac{1}{2}-\frac{1}{3}+\dots+\frac{1}{2012}-\frac{1}{2013}
=\frac{2012}{2013}。
\]
\end{solution}

    \question 计算
    \[
    \ceil{\sqrt{1}} + \ceil{\sqrt{2}} + \ceil{\sqrt{3}} + \dots + \ceil{\sqrt{2025}}
    \]
    \begin{solution}
        注意到$\sqrt{2025}=45$,当$(m-1)^2<k\le m^2$时有
        \[
        \lceil \sqrt{k}\rceil=m
        \]
        因此可按 $m=1,2,\dots,45$ 分组求和,
        \[
        \sum_{k=1}^{2025}\lceil \sqrt{k}\rceil
        =\sum_{m=1}^{45}\sum_{k=(m-1)^2+1}^{m^2} m
        \]
        对固定的 $m$, 求和的项数为$m^2-(m-1)^2=2m-1$,于是
        \[
        \sum_{k=1}^{2025}\lceil \sqrt{k}\rceil
        =\sum_{m=1}^{45} m(2m-1)
        =2\sum_{m=1}^{45}m^2-\sum_{m=1}^{45}m
        =2\cdot\frac{45\cdot46\cdot91}{6}-\frac{45\cdot46}{2}
        =61755
        \]
    \end{solution}

    \question 试证 
    \[
    8 + 88 + 888 + \ldots + \underbrace{888\ldots88}_{\text{88's}} = \frac{8}{81}(10^{89} - 802)
    \] 
    \begin{solution}
        考虑到
        \[
        \underbrace{99\ldots99}_{\text{n's}}= 10^n - 1
        \]
        于是
        \begin{align*}
        8 + 88 + 888 + \cdots + \underbrace{888\ldots88}_{\text{88's}} 
        &= \frac{8}{9}(9 + 99 + 999 + \cdots + \underbrace{99\ldots99}_{\text{88's}}) \\
        &= \frac{8}{9}(10 + 10^2 + 10^3 + \cdots + 10^{88} - 88) \\
        &= \frac{8}{9}\left[\frac{10(10^{88}-1)}{10-1}-88\right] \\
        &= \frac{8}{81}(10^{89} - 802)
        \end{align*}
        故得证。
    \end{solution}

    \question 求无穷级数
    \[
    \frac{1^{2}}{11} + \frac{5^{2}}{11^{2}} + \frac{9^{2}}{11^{3}} + \frac{13^{2}}{11^{4}} + \cdots + \frac{(4n - 3)^{2}}{11^{n}} + \cdots
    \]
    \begin{solution}
        设原式为 
        \[
        S = \sum_{n=1}^{\infty}\frac{(4n-3)^2}{11^n}
        \]
        由 $(4n-3)^2 = 16n^2 - 24n + 9$,有
        \[
        S = 16\sum_{n=1}^{\infty}\frac{n^2}{11^n} - 24\sum_{n=1}^{\infty}\frac{n}{11^n} + 9\sum_{n=1}^{\infty}\frac{1}{11^n}
        \]
        当 $|x| < 1$ 时,有
        \[
        \sum_{n=1}^{\infty}x^n = \frac{x}{1-x}, \quad
        \sum_{n=1}^{\infty}nx^n = \frac{x}{(1-x)^2}, \quad
        \sum_{n=1}^{\infty}n^2x^n = \frac{x(1+x)}{(1-x)^3}
        \]
        令 $x = \dfrac{1}{11}$, 则
        \[
        \sum_{n=1}^{\infty}\frac{1}{11^n} = \frac{1}{10}, \quad \sum_{n=1}^{\infty}\frac{n}{11^n} = \frac{11}{100}, \quad \sum_{n=1}^{\infty}\frac{n^2}{11^n} = \frac{33}{250}
        \]
        于是
        \[
        S = 16 \cdot \frac{33}{250} - 24 \cdot \frac{11}{100} + 9 \cdot \frac{1}{10} = \frac{93}{250}
        \]
    \end{solution}
    \begin{solution}
        设
        \[
        S=\frac{1^{2}}{11} + \frac{5^{2}}{11^{2}} + \frac{9^{2}}{11^{3}} + \frac{13^{2}}{11^{4}} + \frac{17^{2}}{11^{5}} + \cdots \tag{1}
        \]
        则
        \[
        \frac{1}{11}S=\frac{1^{2}}{11^2} + \frac{5^{2}}{11^{3}} + \frac{9^{2}}{11^{4}} + \frac{13^{2}}{11^{5}} + \frac{17^{2}}{11^{6}} + \cdots \tag{2}
        \]
        (1) - (2) 得
        \[
        \frac{10}{11} S = \frac{1}{11} + \frac{5^2-1^2}{11^2} + \frac{9^2-5^2}{11^3} + \frac{13^2-9^2}{11^4} + \frac{17^2-13^2}{11^5} + \cdots \tag{3}
        \]
        \[
        \frac{10}{121} S = \frac{1}{11^2} + \frac{5^2-1^2}{11^3} + \frac{9^2-5^2}{11^4} + \frac{13^2-9^2}{11^5} + \frac{17^2-13^2}{11^6} + \cdots \tag{4}
        \]
        (3) - (4) 得
        \[
        \frac{100}{121} S = \frac{1}{11} + \frac{23}{11^2} + \frac{32}{11^3} + \frac{32}{11^4} + \frac{32}{11^5} + \cdots = \frac{1}{11} + \frac{23}{11^2} + \frac{32}{11^3} \cdot \frac{1}{1 - \frac{1}{11}}
        \]
        故可得
        \[
        S = \frac{93}{250}
        \]
    \end{solution}

    \question 已知
    \[
    1 + \frac{3}{x} + \frac{5}{x^2} + \frac{7}{x^3} + \frac{9}{x^4} + \cdots = 91,
    \]
    求
    \[
    S=1 + \frac{4}{x} + \frac{9}{x^2} + \frac{16}{x^3} + \frac{25}{x^4} + \cdots
    \]
    \begin{solution}
        发现
        \[
        91 = 1 + \frac{3}{x} + \frac{5}{x^2} + \frac{7}{x^3} + \cdots = \left(1 + \frac{1}{x} + \frac{1}{x^2} + \cdots \right) + 2\left( \frac{1}{x} + \frac{2}{x^2} + \frac{3}{x^3} + \cdots \right)
        \]
        考虑
        \[
        \sum_{n=1}^{\infty} n y^{n-1} =\left(\sum_{n=1}^{\infty} y^n\right)'=\left(\frac{y}{1-y}\right)'= \frac{1}{(1 - y)^2}
        \Rightarrow \sum_{n=1}^{\infty} n y^n = \frac{y}{(1 - y)^2}\]
        记 \(y = \dfrac{1}{x}\),则
        \[
        \sum_{n=1}^{\infty} \frac{n}{x^n} = \frac{1}{x} \cdot \frac{1}{(1 - \frac{1}{x})^2} = \frac{x}{(x - 1)^2}
        \]
        所以
        \[
        91= \frac{x}{x - 1} + \frac{2x}{(x - 1)^2} = \frac{x^2 + x}{(x - 1)^2} \Rightarrow 90x^2 - 183x + 91 = 0 \Rightarrow x = \frac{7}{6} \ \text{或} \ \frac{13}{15}
        \]
        级数收敛需满足 \(|x| > 1\),故取 \(x = \dfrac{7}{6}\),现考虑
        \[
        S=\sum_{n=1}^\infty \frac{n^2}{x^{n-1}}=\sum_{n=1}^{\infty} n^2 y^{n-1}
        \]
        由上,
        \[
        \sum_{n=1}^{\infty} n y^n = \frac{y}{(1 - y)^2}
        \Rightarrow \sum_{n=1}^{\infty} n^2 y^{n-1} = \left( \frac{y}{(1 - y)^2} \right)' = \frac{1 + y}{(1 - y)^3}
        \]
        取$y=\dfrac67$即得$S=637$;又解:发现$S=91+\dfrac{6}{7}S \Rightarrow S=637$
    \end{solution}

    \question 求
    \[
    \sum_{k=0}^{\infty} \frac{(-1)^k}{2k+1}
    \]
    \begin{solution}
        发现
        \begin{align*}
        1-\frac{1}{3}+\frac{1}{5}-\frac{1}{7}+\frac{1}{9}- \cdots
        &= \left[x-\frac{1}{3}x^3 +\frac{1}{5}x^5 -\frac{1}{7}x^7+\frac{1}{9}x^9-\cdots\right] _0^1 \\
        &=\int_0^1 (1-x^2+x^4-x^6+x^8+\cdots)\,dx \\
        &=\int_0^1 \frac{1}{1+x^2}\,dx \\
        &= \left[ \tan^{-1} x \right]_0^1 = \frac{\pi}{4}
        \end{align*}
    \end{solution}

    \question 求下列级数的值或证明其发散:
    \[
    \sum_{k=1}^{\infty} \arctan\left(-\frac{2}{k^{2}}\right)
    \]
    \begin{solution}
        由
        \[
        \arctan\left(-\frac{2}{k^{2}}\right)
        =\arctan\left(\frac{(k-1)-(k+1)}{1+(k-1)(k+1)}\right)
        =\arctan(k-1)-\arctan(k+1),
        \]
        知原级数是一列项求和,于是
        \begin{align*}
        &\sum_{k=1}^{\infty}\left[\arctan(k-1)-\arctan(k+1)\right] \\
        &=\lim_{N\to\infty}(\arctan 0+\arctan 1-\arctan N-\arctan(N+1)) \\
        &=0+\frac{\pi}{4}-\frac{\pi}{2}-\frac{\pi}{2} =-\frac{3\pi}{4}
        \end{align*}
        级数收敛,其值为$-\dfrac{3\pi}{4}$。
    \end{solution}

    \question 求
    \[
    \sum_{n=1}^{\infty} \frac{1}{2^n} \cdot \frac{1}{1 + 2^{2^{-n}}}
    \]
    \begin{solution}
        注意到:
        \[
        \frac{1}{1 - x^2} = \frac{1}{(1 - x)(1 + x)} = \frac{1}{2} \left( \frac{1}{1 - x} + \frac{1}{1 + x} \right)
        \]
        令 $x = 2^{2^{-n}}$,则
        \[
        \frac{1}{1 + 2^{2^{-n}}} = \frac{2}{1 - 2^{2^{1-n}}} - \frac{1}{1 - 2^{2^{-n}}}
        \]
        因此原式化为
        \begin{align*}
        &\sum_{n=1}^{\infty} \frac{1}{2^n} \cdot \left( \frac{2}{1 - 2^{2^{1-n}}} - \frac{1}{1 - 2^{2^{-n}}} \right)\\
        &= \sum_{n=1}^{\infty} \left( \frac{1}{2^{n-1}} \cdot \frac{1}{1 - 2^{2^{-(n-1)}}} - \frac{1}{2^n} \cdot \frac{1}{1 - 2^{2^{-n}}} \right)\\
        &= -1 - \lim_{N \to \infty} \left(\frac{1}{2^N} \cdot \frac{1}{1 - 2^{2^{-N}}} \right)
        \end{align*}
        现设 $x = 2^{-N}$,则当$N \to \infty, x \to 0^+$,且
        \[
        \frac{1}{2^N} \cdot \frac{1}{1 - 2^{2^{-N}}} = \frac{x}{1 - 2^x} = -\frac{x}{2^x - 1}
        \]        
        由洛必达法则,
        \[
        \lim_{x \to 0^+} \frac{x}{2^x - 1} = \lim_{x \to 0^+} \frac{1}{\ln 2 \cdot 2^x} = \frac{1}{\ln 2}
        \]
        故原式为$-1 + \dfrac{1}{\ln 2}$.
    \end{solution}

    \question 设
    \[
    f(x) = \frac{2016^x}{2016^x + \sqrt{2016}},
    \]
    求
    \[
    \sum_{k=0}^{2016} f\left(\frac{k}{2016}\right) 
    \]
    \begin{solution}
        记 \(a=2016\),观察到
        \[
        f(x)+f(1-x)=\frac{a^{x}}{a^{x}+a^\frac12}+\frac{a^{1-x}}{a^{1-x}+a^\frac12}
        =\frac{a^{x}(a^{1-x}+a^{\frac12})+a^{1-x}(a^{x}+a^\frac12)}{(a^{x}+a^\frac12)(a^{1-x}+a^\frac12)}=1.
        \]
        于是
        \[
        \sum_{k=0}^{2016} f\!\left(\frac{k}{2016}\right)
        =\underbrace{\sum_{k=0}^{1007}\bigl[f\!\left(\frac{k}{2016}\right)+f\!\left(1-\frac{k}{2016}\right)\bigr]}_{1008\text{ 个“1”}}+f\!\left(\frac{1008}{2016}\right)
        =1008+f\!\left(\frac12\right).
        \]
        而$f\!\left(\dfrac12\right)=\dfrac{a^\frac12}{a^\frac12+a^\frac12}=\dfrac12,$ 故
        \[
        \sum_{k=0}^{2016} f\!\left(\frac{k}{2016}\right)=1008+\frac12=\frac{2017}{2}.
        \]
    \end{solution}
\question 求级数
\[
\sum_{r=0}^{\infty}\frac{\sin^4(\pi x 2^{r-2})}{4^r}
\]
在 $x=1$ 的情况下的值。

\begin{solution}

\noindent
先化简 $\sin^4\theta$:
\begin{align*}
\sin^4\theta
&= (\sin^2\theta)^2 \\
&= \left(\frac{1}{2}-\frac{1}{2}\cos 2\theta\right)^2 \\
&= \frac{1}{4}-\frac{1}{2}\cos 2\theta+\frac{1}{4}\cos^2 2\theta \\
&= \frac{1}{4}-\frac{1}{2}\cos 2\theta+\frac{1}{4}\cdot\frac{1+\cos 4\theta}{2} \\
&= \frac{3}{8}-\frac{1}{2}\cos 2\theta+\frac{1}{8}\cos 4\theta
\end{align*}

\noindent
等价地,可利用恒等式
\[
\sin^4\theta=\sin^2\theta-\frac{1}{4}\sin^2 2\theta
\]

\noindent
考虑前 $n$ 项和:
\begin{align*}
\sum_{r=0}^{n}\frac{\sin^4(\pi x 2^{r-2})}{4^r}
&=\sum_{r=0}^{n}\frac{1}{4^r}
\left[\sin^2(\pi x 2^{r-2})-\frac{1}{4}\sin^2(\pi x 2^{r-1})\right] \\
&=\sum_{r=0}^{n}\left[
\frac{1}{4^r}\sin^2(\pi x 2^{r-2})
-\frac{1}{4^{r+1}}\sin^2(\pi x 2^{r-1})
\right]
\end{align*}

\noindent
这是一个望远镜求和,展开可得:
\begin{align*}
&\sin^2\frac{\pi x}{4}
-\frac{1}{4}\sin^2\frac{\pi x}{2}
+\frac{1}{4}\sin^2\frac{\pi x}{2}
-\frac{1}{16}\sin^2(\pi x) \\
&\quad + \cdots
-\frac{1}{4^{n+1}}\sin^2(\pi x 2^{\,n-1})
\end{align*}

\noindent
中间项相互抵消,因此
\[
\sum_{r=0}^{n}\frac{\sin^4(\pi x 2^{r-2})}{4^r}
= \sin^2\frac{\pi x}{4}
-\frac{1}{4^{n+1}}\sin^2(\pi x 2^{n-1})
\]

\noindent
当 $x=1$ 时,
\[
\sum_{r=0}^{n}\frac{\sin^4(\pi 2^{r-2})}{4^r}
= \sin^2\frac{\pi}{4}
-\frac{1}{4^{n+1}}\sin^2(\pi 2^{n-1})
\]

\noindent
令 $n\to\infty$,由于
\[
0\le \frac{1}{4^{n+1}}\sin^2(\pi 2^{n-1}) \le \frac{1}{4^{n+1}}\to 0
\]
可得无穷级数的和:
\[
\sum_{r=0}^{\infty}\frac{\sin^4(\pi 2^{r-2})}{4^r}
= \sin^2\frac{\pi}{4}
= \frac{1}{2}
\]

\end{solution}

    \question
求下列级数的和:
\[
\sum_{n=0}^{\infty} \frac{\cos^2\left(\frac{n\pi}{6}\right)}{2^n}.
\]

\begin{solution}
\textbf{步骤 1:使用余弦平方恒等式}:
\[
\cos^2\frac{n\pi}{6} = \frac{1+\cos\frac{n\pi}{3}}{2} \implies
\sum_{n=0}^{\infty} \frac{\cos^2(n\pi/6)}{2^n} = \frac{1}{2} \sum_{n=0}^{\infty} \frac{1+\cos(n\pi/3)}{2^n}.
\]

\medskip
\textbf{步骤 2:拆分为两部分}:
\[
\frac{1}{2} \sum_{n=0}^{\infty} \frac{1}{2^n} + \frac{1}{2} \sum_{n=0}^{\infty} \frac{\cos(n\pi/3)}{2^n} = \frac{1}{2} \sum_{n=0}^{\infty} \frac{1}{2^n} + \frac{1}{2} \Re \sum_{n=0}^{\infty} \frac{e^{in\pi/3}}{2^n}.
\]

\medskip
\textbf{步骤 3:求几何级数和}:
\[
\sum_{n=0}^{\infty} \frac{1}{2^n} = \frac{1}{1-1/2} = 2.
\]

\medskip
\textbf{步骤 4:复数形式求和}:
\[
\sum_{n=0}^{\infty} \left(\frac{e^{i\pi/3}}{2}\right)^n = \frac{1}{1 - e^{i\pi/3}/2} \implies 
\sum_{n=0}^{\infty} \frac{\cos(n\pi/3)}{2^n} = \Re \frac{1}{1 - e^{i\pi/3}/2}.
\]

\medskip
\textbf{步骤 5:提取实部}:
\[
\frac{1}{1 - e^{i\pi/3}/2} = \frac{2 - e^{-i\pi/3}}{4 - 2(e^{i\pi/3} + e^{-i\pi/3}) + 1} = \frac{2 - e^{-i\pi/3}}{5 - 4\cos(\pi/3)} = \frac{2 - (\frac{1}{2} - i\frac{\sqrt{3}}{2})}{3} = \frac{3/2 + i\sqrt{3}/2}{3}.
\]

\[
\Re\frac{3/2 + i\sqrt{3}/2}{3} = \frac{3/2}{3} = \frac{1}{2}.
\]

\medskip
\textbf{步骤 6:求最终和}:
\[
\sum_{n=0}^{\infty} \frac{\cos^2(n\pi/6)}{2^n} = \frac{1}{2} \cdot 2 + \frac{1}{2} \cdot \frac{1}{2} = 1 + \frac{1}{2} = \frac{3}{2}.
\]

\end{solution}

    \question 求 
    \[
    \sum_{n=1}^{\infty}\frac{n}{2^{n}}\cos\frac{n\pi}{3}
    \]
    \begin{solution}
        设
        \[
        z=\frac{1}{2}e^{i\pi/3}=\frac{1}{4}(1+\sqrt 3\, i)
        \]
        则
        \[
        f(z) =\sum_{n=1}^\infty z^n = \frac{z}{1-z}
        \Rightarrow 
        zf'(z)=\frac{z}{(1-z)^2}
        = \sum_{n=1}^\infty n z^{n}
        = \sum_{n=1}^\infty \frac{n}{2^n} e^{n\pi i/3}
        \]
        于是
        \[
        zf'(z)
        = \frac{\frac{1}{4}(1+\sqrt{3}\,i)}{\left(\frac{3}{4}-\frac{\sqrt 3}{4} i\right)^2}
        = \frac{2}{3}\cdot \frac{1+\sqrt 3\, i}{1-\sqrt 3\, i}
        = -\frac{1}{3}+\frac{\sqrt 3}{3} i
        \]
        故
        \[
        \sum_{n=1}^\infty \frac{n}{2^n}\cos\frac{n\pi}{3}= \Re(zf'(z))= -\frac{1}{3}
        \]
    \end{solution}

    \question 证明不等式:
(1) \[ \frac{1}{n+1} + \frac{1}{n+2} + \cdots + \frac{1}{2n} > \frac{1}{2} \]
(2) \[ \frac{1}{n} + \frac{1}{n+1} + \cdots + \frac{1}{2n} < \frac{3}{4} + \frac{1}{n} \]

\begin{solution}
    (a) 首先,容易看出:
    \[ \frac{1}{n+1} + \frac{1}{n+2} + \cdots + \frac{1}{2n} > \frac{1}{2n} + \frac{1}{2n} + \cdots + \frac{1}{2n} = \frac{1}{2} \]
    上述不等式右边共有 $n$ 项。

    但同时,我们可以将第二部分的和式通过倒序相加(或首尾配对)的方法重写:
    \begin{align*}
    \frac{1}{n} + \frac{1}{n+1} + \cdots + \frac{1}{2n} &= \frac{1}{2} \left[ \left(\frac{1}{n} + \frac{1}{2n}\right) + \left(\frac{1}{n+1} + \frac{1}{2n-1}\right) + \cdots + \left(\frac{1}{2n} + \frac{1}{n}\right) \right] \\
    &= \frac{1}{2} \left[ \frac{3n}{2n^2} + \frac{3n}{2n^2 + (n-1)} + \frac{3n}{2n^2 + 2(n-2)} + \cdots + \frac{3n}{2n^2} \right]
    \end{align*}
    由于分母中的项 $k(n-k) \ge 0$,我们可以通过放缩去掉这些正项从而增大分数值:
    \[ < \frac{1}{2} \left[ \frac{3n}{2n^2} + \frac{3n}{2n^2} + \cdots + \frac{3n}{2n^2} \right] \]
    上式括号内共有 $n+1$ 项。于是:
    \[ = \frac{1}{2} (n+1) \frac{3}{2n} = \frac{3}{4} + \frac{1}{4n} < \frac{3}{4} + \frac{1}{n} \]
    这就证明了题目中的结论。
\end{solution}

\question
计算数 $1 + \frac{1}{\sqrt{2}} + \frac{1}{\sqrt{3}} + \frac{1}{\sqrt{4}} + \cdots + \frac{1}{\sqrt{1,000,000}}$ 的整数部分。

\begin{solution}
    (a) 首先我们证明不等式:
    \[ 2\sqrt{n+1} - 2\sqrt{n} < \frac{1}{\sqrt{n}} < 2\sqrt{n} - 2\sqrt{n-1} \]
    
    对于左边的不等式,我们将分子有理化:
    \[ 2\sqrt{n+1} - 2\sqrt{n} = \frac{2(\sqrt{n+1} - \sqrt{n})(\sqrt{n+1} + \sqrt{n})}{\sqrt{n+1} + \sqrt{n}} = \frac{2}{\sqrt{n+1} + \sqrt{n}} < \frac{2}{\sqrt{n} + \sqrt{n}} = \frac{1}{\sqrt{n}} \]
    同理可证右边的不等式。
    
    现在利用上述不等式对原和式进行估计。
    对于下界,我们利用左边的不等式 $ \frac{1}{\sqrt{n}} > 2(\sqrt{n+1} - \sqrt{n}) $:
    \begin{align*}
    1 + \frac{1}{\sqrt{2}} + \cdots + \frac{1}{\sqrt{1,000,000}} &> 1 + 2 [(\sqrt{3} - \sqrt{2}) + (\sqrt{4} - \sqrt{3}) + \cdots + (\sqrt{1,000,001} - \sqrt{1,000,000})] \\
    &= 1 + 2(\sqrt{1,000,001} - \sqrt{2}) \\
    &> 2 \cdot 1000 - \sqrt{8} + 1 > 2000 - 3 + 1 = 1998
    \end{align*}
    
    对于上界,我们利用右边的不等式 $ \frac{1}{\sqrt{n}} < 2(\sqrt{n} - \sqrt{n-1}) $:
    \begin{align*}
    1 + \frac{1}{\sqrt{2}} + \cdots + \frac{1}{\sqrt{1,000,000}} &< 1 + 2 [(\sqrt{2} - 1) + (\sqrt{3} - \sqrt{2}) + \cdots + (\sqrt{1,000,000} - \sqrt{999,999})] \\
    &= 1 + 2(\sqrt{1,000,000} - 1) \\
    &= 1 + 2 \cdot 999 = 1999
    \end{align*}
    
    综上所述,该数的整数部分等于 1998。
\end{solution}

\question
求数 $\frac{1}{\sqrt[3]{4}} + \frac{1}{\sqrt[3]{5}} + \frac{1}{\sqrt[3]{6}} + \cdots + \frac{1}{\sqrt[3]{1,000,000}}$ 的整数部分。

\begin{solution}
    首先,通过比较二项式展开式,我们注意到对于每一个自然数 $n$:
    \[ \left(1 + \frac{2}{3} \cdot \frac{1}{n}\right)^3 > \left(1 + \frac{1}{n}\right)^2 \]
    两边开立方根并整理可得:
    \[ \frac{1}{\sqrt[3]{n^2}} > \frac{3}{2} [ \sqrt[3]{(n+1)^2} - \sqrt[3]{n^2} ] \]
    同理,利用类似的展开比较可以得到下界放缩:
    \[ \frac{1}{\sqrt[3]{n^2}} < \frac{3}{2} [ \sqrt[3]{n^2} - \sqrt[3]{(n-1)^2} ] \]
    由此我们得到关键的不等式:
    \[ \frac{3}{2} [ \sqrt[3]{(n+1)^2} - \sqrt[3]{n^2} ] < \frac{1}{\sqrt[3]{n}} < \frac{3}{2} [ \sqrt[3]{n^2} - \sqrt[3]{(n-1)^2} ] \]
    
    现在利用此不等式对原和式进行估计。对于下界:
    \begin{align*}
    \frac{1}{\sqrt[3]{4}} + \cdots + \frac{1}{\sqrt[3]{1,000,000}} &> \frac{3}{2} [(\sqrt[3]{5^2} - \sqrt[3]{4^2}) + \cdots + (\sqrt[3]{1,000,001^2} - \sqrt[3]{1,000,000^2})] \\
    &= \frac{3}{2} (\sqrt[3]{1,000,002,000,001} - \sqrt[3]{16}) \\
    &> \frac{3}{2} \cdot 10,000 - \sqrt[3]{54} \\
    &> 15,000 - 4 = 14,996
    \end{align*}
    
    对于上界:
    \begin{align*}
    \frac{1}{\sqrt[3]{4}} + \cdots + \frac{1}{\sqrt[3]{1,000,000}} &< \frac{3}{2} [(\sqrt[3]{4^2} - \sqrt[3]{3^2}) + \cdots + (\sqrt[3]{1,000,000^2} - \sqrt[3]{999,999^2})] \\
    &= \frac{3}{2} (\sqrt[3]{1,000,000,000,000} - \sqrt[3]{9}) \\
    &< \frac{3}{2} (10,000 - 2) = 14,997
    \end{align*}
    
    综上所述,该数的整数部分等于 14,996。
\end{solution}

    \question 求
\[
\sum_{m=1}^{19} \sum_{n=m}^{19} (2m+n).
\]

\begin{solution}
首先,将求和顺序交换,更容易计算:
\[
\sum_{m=1}^{19} \sum_{n=m}^{19} (2m+n) = \sum_{n=1}^{19} \sum_{m=1}^{n} (2m+n).
\]

将内层求和拆开:
\[
\sum_{n=1}^{19} \sum_{m=1}^{n} (2m+n) = \sum_{n=1}^{19} \left[ 2 \sum_{m=1}^{n} m + \sum_{m=1}^{n} n \right].
\]

利用标准求和公式:
\[
\sum_{m=1}^{n} m = \frac{n(n+1)}{2}, \quad \sum_{m=1}^{n} n = n \times n = n^2.
\]

于是:
\[
\sum_{n=1}^{19} \left[ 2 \sum_{m=1}^{n} m + \sum_{m=1}^{n} n \right] = \sum_{n=1}^{19} [2 \cdot \frac{n(n+1)}{2} + n^2] = \sum_{n=1}^{19} [n^2+n+n^2] = \sum_{n=1}^{19} (2n^2+n).
\]

再分开求和:
\[
\sum_{n=1}^{19} (2n^2+n) = 2 \sum_{n=1}^{19} n^2 + \sum_{n=1}^{19} n.
\]

代入求和公式:
\[
\sum_{n=1}^{19} n^2 = \frac{19 \cdot 20 \cdot 39}{6}, \quad \sum_{n=1}^{19} n = \frac{19 \cdot 20}{2}.
\]

于是:
\[
2 \sum_{n=1}^{19} n^2 + \sum_{n=1}^{19} n = 2 \cdot \frac{19 \cdot 20 \cdot 39}{6} + \frac{19 \cdot 20}{2} = \frac{19 \cdot 20 \cdot 39}{3} + 190.
\]

计算:
\[
\frac{19 \cdot 20 \cdot 39}{3} + 190 = 19 \cdot 20 \cdot 13 + 190 = 4940 + 190 = 5130.
\]

\[
\therefore \sum_{m=1}^{19} \sum_{n=m}^{19} (2m+n) = 5130.
\]
\end{solution}


    \question 令
    \[
    S(x) = \sum_{k=1}^\infty \frac{(-1)^k x^{k+1}}{3^k(k+1)},
    \]
    \begin{parts}
    \part 试证
    \[
    S'(x) = \frac{3}{3 + x} - 1。
    \]
    \begin{solution}
        对各项微分,有
        \[
         S'(x) = \sum_{k=1}^\infty \frac{(-1)^k x^k}{3^k} = \sum_{k=1}^\infty \left(-\frac{x}{3}\right)^k= \frac{-\frac{x}{3}}{1+\frac{x}{3}}= \frac{-x}{3+x}= \frac{3}{3+x} - 1
        \]
    \end{solution}
    \part 据此,证
    \[
    S(x) = 3 \ln(3 + x) - 3 \ln 3 - x。
    \]
    \begin{solution}
        \[
        S(x) = \int \left( \frac{3}{3+x} - 1 \right) dx 
        = 3\ln(3+x) - x + C
        \]
        由 \( S(0)=0 \)可得 \( C = -3\ln 3 \),故
        \[
        S(x) = 3\ln(3+x) - 3\ln 3 - x
        \]
    \end{solution}
    \part 求
    \[
    \sum_{k=1}^\infty \frac{(-1)^k}{k \cdot 3^k}
    \]
    \begin{solution}改写求和,
        \begin{align*}
            \sum_{k=1}^\infty \frac{(-1)^k}{k \cdot 3^k}
            &= \sum_{k=0}^\infty \frac{(-1)^{k+1}}{(k+1)3^{k+1}}
            = -\frac{1}{3}\sum_{k=0}^\infty \frac{(-1)^{k}}{(k+1)3^{k}}\\
            &= -\frac{1}{3} (1+S(1)) 
            = -\frac{1}{3}(1+3 \ln4 - 3 \ln 3 - 1)
            =\ln\frac{3}{4}
        \end{align*}
    \end{solution}
    \end{parts}

    \question 求
\[
\sum_{n=1}^\infty 
\ln\left(1+\frac{1}{n}\right)
\ln\left(1+\frac{1}{2n}\right)
\ln\left(1+\frac{1}{2n+1}\right)
\]
的值。

\begin{solution}
定义
\[
f(n)=\ln\left(\frac{n+1}{n}\right),\quad n\ge1.
\]
注意到
\[
f(2n)+f(2n+1)=f(n).
\]
由不等式 $\ln(1+x)\le x$ 可得
\[
f(n)\le\frac{1}{n}.
\]
再定义
\[
g(n)=\sum_{k=n}^{2n-1} f^3(k).
\]
于是
\[
g(n)<n f^3(n)\le\frac{1}{n^2}.
\]

接下来计算
\begin{align*}
g(n)-g(n+1)
&=f^3(n)-f^3(2n)-f^3(2n+1) \\
&=(f(2n)+f(2n+1))^3-f^3(2n)-f^3(2n+1) \\
&=3(f(2n)+f(2n+1))f(2n)f(2n+1) \\
&=3f(n)f(2n)f(2n+1).
\end{align*}
因此
\[
\sum_{n=1}^N f(n)f(2n)f(2n+1)
=\frac{1}{3}\sum_{n=1}^N\bigl(g(n)-g(n+1)\bigr)
=\frac{1}{3}\bigl(g(1)-g(N+1)\bigr).
\]

由于当 $N\to\infty$ 时 $g(N+1)\to0$,于是
\[
\sum_{n=1}^\infty f(n)f(2n)f(2n+1)
=\frac{1}{3}g(1)
=\frac{1}{3}\ln^3 2.
\]
这正是所求级数的值。
\end{solution}


    \question 考虑正整数 $m, n$,且 $m \ge 2$。$(m,n)$-锯齿数列是从 1 开始的连续整数序列,有 $n$ 个“齿”,每个齿从 2 上升到 $m$ 再下降到 1,如$(3,4)$-锯齿数列为
    \[
    \begin{array}{ccccccccccccccccccccc}
    &&3&&&&3&&&&3&&&&3&& \\
    &2&&2&&2&&2&&2&&2&&2&&2& \\
    1&&&&1&&&&1&&&&1&&&&1 \\
    \end{array}
    \]
    该序列共有 17 项,平均数为 $\dfrac{33}{17}$。
    \begin{parts}
    \part 求 $(4,2)$-锯齿数列的项和。
    \begin{solution}
        $(4,2)$-锯齿数列的项为
        \[
        1,2,3,4,3,2,1,2,3,4,3,2,1
        \]
        其和为 $31$。
    \end{solution}
    \part 对任意正整数 $m \ge 2$,求 $(m,3)$-锯齿数列中所有数字之和的通项。
    \begin{solution}
        $(m,3)$-锯齿数列由初始 1 和 3 个齿组成,每个齿为 $2,3,\dots,m-1,m,m-1,\dots,2,1$,项和为
        \[
        2 + 3 + \dots + m + (m-1) + \dots + 1 = 2(1+2+\dots+m)-1-m = m^2-1.
        \]
        因此序列和为 $1 + 3(m^2-1) = 3m^2 - 2$。
    \end{solution}
    \part 求所有使 $(m,n)$-锯齿数列的项和为 145 的 $(m,n)$ 序对。
    \begin{solution}
        每个齿的和为 $m^2-1$,整个序列和为
        \[
        1 + n(m^2-1) = 145 \Rightarrow n(m^2-1) = 144.
        \]
        考虑$144$的因数分解,当$m \ge 2$ 时,$m^2-1$可能值为
        \[
        3,8,15,24,35,48,63,80,99,120,143,
        \]
        其中 $3,8,24,48$ 能整除 144,因此符合条件的序对为
        \[
        (m,n) = (2,48), (3,18), (5,6), (7,3).
        \]
    \end{solution}
    \part 证明对于所有正整数对 $(m,n)$ 且 $m \ge 2,(m,n)$-锯齿数列的平均数不是整数。
    \begin{solution}
        平均数为
        \[
        \frac{1 + n(m^2-1)}{1 + n(2m-2)}
        \]
        假设平均数为整数 $k$,可得一关于$m$的二次方程式
        \[
        \frac{1 + n(m^2-1)}{1 + n(2m-2)} = k \Rightarrow m^2 n - 2mn k + (2nk - n - k + 1) = 0
        \]
        由于$m$为整数,故判别式
        \[
        \Delta = (-2nk)^2 - 4n(2nk - n - k +1) = (2n(k-1)+1)^2 - 1.
        \]
        必须为完全平方数,且两完全平方数差为 1 的只有当 0 和 1,因此 
        \[
        2n(k-1)+1 = 1 \Rightarrow k=1
        \]
        但若平均数为 1,则
        \[
        n(m^2-1)+1 = n(2m-2)+1 \Rightarrow n(m-1)^2 = 0,
        \]
        得 $n>0$ 且 $m \ge 2$,矛盾,因此$(m,n)$-锯齿数列的平均数不可能为整数。
    \end{solution}
    \end{parts}

    \question 已知正整数 $n$,考虑一个边长为 $n$ 、顶点朝上的等边三角形,将其划分为单位等边三角形。对于每个 $n$,记 $f(n)$ 为所有大小的顶点朝下等边三角形的总数,例如$f(3)=3,f(4)=7$。
    %2008 euclid P10
    \begin{figure}[H]
        \centering        
        \includegraphics[width=0.6\textwidth]{images/image197.png}
    \end{figure}
    \begin{figure}[H]
        \centering        
        \includegraphics[width=0.6\textwidth]{images/image198.png}
    \end{figure}
    \begin{parts}
    \part 求 $f(5)$ 和 $f(6)$。
    \begin{solution}
        当 $n=5$ 时,有$1+2+3+4=10$个边长为1、$1+2=3$个边长为 2 的三角形,所以
        \[
        f(5) = 10 + 3 = 13
        \]
        当 $n=6$ 时,有$1+2+3+4+5=15$个边长为 1、$1+2+3=6$个边长为2、$1$个边长为 3 的三角形 ,所以
        \[
        f(6) = 15 + 6 + 1 = 22
        \]
    \end{solution}
    \part 证明对每个正整数 $k\ge 1$,有 $f(2k) = f(2k-1) + k^2$。
    \ifprintanswers
    \begin{figure}[H]
        \centering        
        \includegraphics[width=0.4\textwidth]{images/image199.png}
    \end{figure}
    \fi
    \begin{solution}
        设第 $i$ 行有 $i+1$ 个单位点,从左至右记为$0,1,\cdots,i$,考虑边长为 $m$ 、顶点朝下的等边三角形,且其底顶点位于行 $i$ 时,注意到每个这样的三角形都被其底顶点确定,故可能的底点数为
        \[
        (i-m)-m+1 = i+1-2m,
        \]
        要求 $2m \le i \le n$ 才能形成三角形,因此边长为 $m$ 、顶点朝下的三角形总数为
        \[
        \sum_{i=2m}^{n} (i+1-2m) = \frac{(n+1-2m)(n+2-2m)}{2}.
        \]
        若 $n=2k$,则 $m=1,2,\dots,k$,所以
        \begin{align*}
        f(2k) &= \sum_{m=1}^{k} \frac{(2k+1-2m)(2k+2-2m)}{2} \\
        &= \sum_{l=1}^{k} (2l-1)l \quad (\text{令 } l=k+1-m) \\
        &= 2\sum_{l=1}^{k} l^2 - \sum_{l=1}^{k} l \\
        &= 2\cdot \frac{k(k+1)(2k+1)}{6} - \frac{k(k+1)}{2} \\
        &= \frac{k(k+1)(4k-1)}{6}.
        \end{align*}
        若 $n=2k-1$,则 $m=1,2,\dots,k-1$,所以
        \begin{align*}
        f(2k-1) &= \sum_{m=1}^{k-1} \frac{(2k-2m)(2k+1-2m)}{2} \\
        &= \sum_{l=1}^{k-1} l(2l+1) \quad (\text{令 } l=k-m) \\
        &= 2\sum_{l=1}^{k-1} l^2 + \sum_{l=1}^{k-1} l \\
        &= 2\cdot \frac{(k-1)k(2k-1)}{6} + \frac{(k-1)k}{2} \\
        &= \frac{k(k-1)(4k+1)}{6}.
        \end{align*}
        因此,
        \[
        f(2k) - f(2k-1) = \frac{k(k+1)(4k-1)}{6} - \frac{k(k-1)(4k+1)}{6} = k^2
        \]
        证毕。
    \end{solution}
    \end{parts}
    
    \question 如下图,$O(0,0),A_1(8,0)$, $\overline{A_1A_2}$ 与 $x$ 轴正向夹 $45^\circ$ 角, 又 $\overline{A_1A_2} \parallels \overline{A_3A_4} \parallels \overline{A_5A_6} \parallels \dots$, 且 $\overline{OA_1} \parallels \overline{A_2A_3} \parallels \overline{A_4A_5} \parallels \dots$。 已知$\overline{A_1A_2} = 8,$且对所有$k \in \mathbb{N},$有$\overline{A_kA_{k+1}} = 2\overline{A_{k+1}A_{k+2}}$。若点 $A_n$ 的坐标为 $(x_n, y_n)$, 求 $\displaystyle \lim_{n \to \infty} (x_n+y_n)$。
    \begin{figure}[H]
        \centering        
        \includegraphics[width=0.4\textwidth]{images/image90.png}
    \end{figure}
    \begin{solution}
        令 $a_n = \overline{A_nA_{n+1}} = 2^{4-n}, n \in \mathbb{N},$有
        \begin{align*}
        x_\infty &= 8 + \frac{a_1}{\sqrt 2} - a_2 - \frac{a_3}{\sqrt 2} + a_4 + \frac{a_5}{\sqrt 2} - a_6 - \frac{a_7}{\sqrt 2} + a_8 + \cdots \\
        &= 8 + (a_4 + a_8 + a_{12} + \cdots) + \frac{1}{\sqrt 2}(a_1 + a_5 + a_9 + \cdots) \\
        &\quad - \frac{1}{\sqrt 2}(a_3 + a_7 + a_{11} + \cdots) - (a_2 + a_6 + a_{10} + \cdots) \\
        &= 8 + \frac{1}{1 - \frac{1}{16}} + \frac{1}{\sqrt 2} \cdot \frac{8}{1 - \frac{1}{16}} - \frac{1}{\sqrt 2} \cdot \frac{2}{1 - \frac{1}{16}} - \frac{4}{1 - \frac{1}{16}} = 8 + \frac{16}{5} (\sqrt 2 - 1) \\
        y_\infty &= \frac{a_1}{\sqrt 2} - \frac{a_3}{\sqrt 2} + \frac{a_5}{\sqrt 2} - \frac{a_7}{\sqrt 2} + \cdots \\
        &= \frac{1}{\sqrt 2} (a_1 + a_5 + a_9 + \cdots) - \frac{1}{\sqrt 2} (a_3 + a_7 + a_{11} + \cdots) \\
        &= \frac{1}{\sqrt 2} \cdot \frac{8}{1 - \frac{1}{16}} - \frac{1}{\sqrt 2} \cdot \frac{2}{1 - \frac{1}{16}} = \frac{16 \sqrt 2}{5} 
        \end{align*}
        所以
        \[
        x_\infty + y_\infty = \frac{24 + 32 \sqrt 2}{5}
        \]
    \end{solution}
\end{questions}
\pagebreak

\begin{center}
  {\fontsize{30pt}{26pt}\selectfont
    \hypertarget{泰勒展开式}{泰勒展开式} \label{泰勒展开式}
  }
\end{center}
\separator
\vspace{1pt}
\begin{questions}
    \question
设 $y = e^{\tan x}$,$x \in \mathbb{R}$。

\textbf{a)} 求 $\frac{d^2y}{dx^2}$ 并证明
\[
\frac{d^2y}{dx^2} = (1+\tan^2 x + 2 \tan x) \frac{dy}{dx}.
\]

\begin{solution}
首先对 $y$ 求导:
\[
\frac{dy}{dx} = e^{\tan x} \sec^2 x = y \sec^2 x.
\]

再次求导:
\[
\frac{d^2y}{dx^2} = \frac{d}{dx}(y \sec^2 x) = \frac{dy}{dx} \sec^2 x + y \cdot 2 \sec^2 x \tan x.
\]

将 $y \sec^2 x$ 替换为 $\frac{dy}{dx}$:
\[
\frac{d^2y}{dx^2} = \frac{dy}{dx} \sec^2 x + 2 \frac{dy}{dx} \tan x = \frac{dy}{dx} ( \sec^2 x + 2 \tan x ) = \frac{dy}{dx} (1 + \tan^2 x + 2 \tan x ).
\]
\end{solution}

\textbf{b)} 求 $e^{\tan x}$ 的 $x^3$ 展开式。

\begin{solution}
使用 Taylor 展开:
\[
y = y_0 + y'_0 x + \frac{y''_0}{2!} x^2 + \frac{y'''_0}{3!} x^3 + O(x^4),
\]
在 $x=0$ 时,$y_0 = 1$,$y'_0 = 1$,$y''_0 = 1$。

计算三阶导数:
\[
\frac{d^3y}{dx^3} = 2(1+\tan x) \sec^2 x \frac{dy}{dx} + (1+\tan x)^2 \frac{d^2y}{dx^2},
\]
代入 $x=0$ 得 $y'''_0 = 3$。

于是
\[
e^{\tan x} = 1 + x + \frac{1}{2}x^2 + \frac{1}{2}x^3 + O(x^4).
\]
\end{solution}

\question
设 $f(x) \equiv \sin[\ln(1+x)]$,$x \in \mathbb{R}$,$x > -1$。

\textbf{a)} 求 $f(x)$ 满足的微分方程。

\begin{solution}
首先求导:
\[
f'(x) = \cos[\ln(1+x)] \frac{1}{1+x}.
\]
\[
(1+x) f'(x) = \cos[\ln(1+x)].
\]

再次求导:
\[
(1+x) f''(x) + f'(x) = - \sin[\ln(1+x)] \frac{1}{1+x}.
\]

两边乘以 $(1+x)$:
\[
(1+x)^2 f''(x) + (1+x) f'(x) = - \sin[\ln(1+x)].
\]

移项得微分方程:
\[
(1+x)^2 f''(x) + (1+x) f'(x) + f(x) = 0.
\]
\end{solution}

\textbf{b)} 求 Maclaurin 展开前 3 个非零项。

\begin{solution}
利用微分方程可求各阶导数在 $x=0$ 的值:
\[
f(0) = \sin(\ln 1) = 0, \quad f'(0) = \cos(\ln 1) = 1.
\]
由微分方程 $(1+x)^2 f'' + (1+x) f' + f = 0$ 得:
\[
f''(0) + f'(0) + f(0) = 0 \implies f''(0) = -1.
\]

三阶导数:
\[
(1+x)^2 f''' + 3(1+x) f'' + 2 f' = 0 \implies f'''(0) + 3 f''(0) + 2 f'(0) = 0 \implies f'''(0) = 1.
\]

因此 Maclaurin 展开:
\[
f(x) = f(0) + f'(0) x + \frac{f''(0)}{2!} x^2 + \frac{f'''(0)}{3!} x^3 + \dots
= x - \frac{1}{2} x^2 + \frac{1}{6} x^3 + \dots
\]
\end{solution}

\textbf{c)} 用结果求 $\sin[\ln(1+x)]$ 的前 2 个非零项的 Maclaurin 展开。

\begin{solution}
从 (b) 的展开式可直接得到前 2 个非零项:
\[
\sin[\ln(1+x)] = x - \frac{1}{2} x^2 + \dots
\]
\end{solution}

\question
设 $y = \tan x$, $0 \le x < \frac{\pi}{2}$。

\textbf{a) 证明下列公式:}

\textbf{i)}
\[
\frac{d^2y}{dx^2} = 2y \frac{dy}{dx}
\]

\textbf{ii)}
\[
\frac{d^5y}{dx^5} = 6 \left(\frac{d^2y}{dx^2}\right)^2 + 8 \frac{dy}{dx} \frac{d^3y}{dx^3} + 2y \frac{d^4y}{dx^4}
\]

\begin{solution}
\textbf{(i) 求二阶导数}:
\[
y = \tan x \implies \frac{dy}{dx} = \sec^2 x = 1 + \tan^2 x = 1+y^2
\]
\[
y' = 1+y^2
\]
对 $x$ 求导:
\[
y'' = 2y y'
\]

\textbf{(ii) 高阶导数}:
\[
y''' = 2y'y' + 2yy'' = 2(y')^2 + 2yy''
\]
\[
y^{(4)} = 6y'y'' + 2yy'''
\]
\[
y^{(5)} = 6(y'')^2 + 8y'y''' + 2y y^{(4)}
\]
\end{solution}

\textbf{b) 求 $y = \tan x$ 的 Maclaurin 展开前 3 个非零项}

\begin{solution}
在 $x=0$ 处求各阶导数:
\[
y_0 = \tan(0) = 0
\]
\[
y'_0 = 1+y_0^2 = 1
\]
\[
y''_0 = 2y_0y'_0 = 0
\]
\[
y'''_0 = 2y_0y''_0 + 2(y'_0)^2 = 2
\]
\[
y^{(4)}_0 = 6y'_0y''_0 + 2y_0y'''_0 = 0
\]
\[
y^{(5)}_0 = 6(y''_0)^2 + 8y'_0y'''_0 + 2y_0y^{(4)}_0 = 16
\]

Maclaurin 展开式:
\[
y = y_0 + xy'_0 + \frac{x^2}{2!}y''_0 + \frac{x^3}{3!}y'''_0 + \frac{x^4}{4!}y^{(4)}_0 + \frac{x^5}{5!}y^{(5)}_0 + O(x^6)
\]
代入各阶导数值:
\[
y = 0 + x + \frac{x^3}{3} + \frac{2}{15}x^5 + O(x^6)
\]

因此,
\[
\tan x = x + \frac{1}{3}x^3 + \frac{2}{15}x^5 + O(x^6)
\]
\end{solution}

\question
设 $y = \ln(2-e^x)$, $x<\ln 2$。

\textbf{a) 证明下列公式:}
\[
e^y \left[ \frac{d^3 y}{dx^3} + 3 \frac{dy}{dx} \frac{d^2 y}{dx^2} + \left(\frac{dy}{dx}\right)^3 \right] + e^x = 0
\]

\begin{solution}
由 $y = \ln(2-e^x)$ 得
\[
e^y = 2-e^x
\]
两边对 $x$ 求导:
\[
\frac{d}{dx}(e^y) = \frac{d}{dx}(2-e^x) \implies e^y y' = -e^x \implies e^y y' + e^x = 0
\]

再次求导:
\[
\frac{d}{dx}(e^y y' + e^x) = 0 \implies e^y[(y')^2 + y''] + e^x = 0
\]

再求一次导数:
\[
\frac{d}{dx}\big(e^y[(y')^2 + y''] + e^x\big) = 0 \implies e^y[y''' + 3y'y'' + (y')^3] + e^x = 0
\]
\end{solution}

\textbf{b) 求 Maclaurin 展开式的前三个非零项}

\begin{solution}
在 $x=0$ 处求各阶导数:

\[
y_0 = \ln(2-e^0) = \ln 1 = 0
\]
\[
e^y y' + e^x = 0 \implies y'_0 + 1 = 0 \implies y'_0 = -1
\]
\[
e^y[(y')^2 + y''] + e^x = 0 \implies (-1)^2 + y''_0 + 1 = 0 \implies y''_0 = -2
\]
\[
e^y[(y')^3 + 3y'y'' + y'''] + e^x = 0 \implies (-1)^3 + 3(-1)(-2) + y'''_0 + 1 = 0 \implies y'''_0 = -6
\]

Maclaurin 展开式:
\[
y = y_0 + xy'_0 + \frac{x^2}{2!}y''_0 + \frac{x^3}{3!}y'''_0 + O(x^4)
\]
代入各阶导数值:
\[
y = 0 - x - x^2 - x^3 + O(x^4)
\]

因此
\[
\ln(2-e^x) = -x - x^2 - x^3 + O(x^4)
\]
\end{solution}

\question
函数 $f$ 和 $g$ 定义如下:
\[
f(x) = \arctan\left(\frac{2}{3}x\right), \quad x\in \mathbb{R},
\]
\[
g(y) = \frac{1}{1+y}, \quad y\in \mathbb{R}, \quad -1<y<1.
\]

\textbf{a)} 将 $g(y)$ 展开为二项级数,并保留 $y^3$ 项。

\begin{solution}
使用二项式定理:
\[
(1+y)^{-1} = 1 + (-1)y + \frac{(-1)(-2)}{2!}y^2 + \frac{(-1)(-2)(-3)}{3!}y^3 + O(y^4)
\]
\[
(1+y)^{-1} = 1 - y + y^2 - y^3 + O(y^4)
\]
\end{solution}

\textbf{b)} 利用 $f'(x)$ 和 (a) 的结果,求 $\arctan\left(\frac{2}{3}x\right)$ 的级数展开。

\begin{solution}
\[
f(x) = \arctan\left(\frac{2}{3}x\right)
\]
\[
f'(x) = \frac{2/3}{1+(\frac{2}{3}x)^2} = \frac{2}{3}\left(1+\frac{4}{9}x^2\right)^{-1}
\]

将 (a) 的结果代入,令 $y = \frac{4}{9}x^2$:
\[
f'(x) \approx \frac{2}{3}\left[1 - \frac{4}{9}x^2 + \left(\frac{4}{9}x^2\right)^2 - \left(\frac{4}{9}x^2\right)^3 + O(x^8)\right]
\]
\[
f'(x) \approx \frac{2}{3} - \frac{8}{27}x^2 + \frac{32}{243}x^4 - \frac{128}{2187}x^6 + O(x^8)
\]

对 $x$ 积分:
\[
f(x) \approx \int \left[\frac{2}{3} - \frac{8}{27}x^2 + \frac{32}{243}x^4 - \frac{128}{2187}x^6 + O(x^8) \right] dx
\]
\[
\arctan\left(\frac{2}{3}x\right) \approx \frac{2}{3}x - \frac{8}{81}x^3 + \frac{32}{1215}x^5 - \frac{128}{15309}x^7 + C
\]

由 $x=0$ 可得 $C=0$,因此:
\[
\arctan\left(\frac{2}{3}x\right) \approx \frac{2}{3}x - \frac{8}{81}x^3 + \frac{32}{1215}x^5 - \frac{128}{15309}x^7.
\]
\end{solution}

\question
\noindent
已知 $y = \tan x$。

\noindent
\textbf{a)} 证明
\[
\frac{d^3 y}{dx^3} = 2y \frac{d^2 y}{dx^2} + 2 \left(\frac{dy}{dx}\right)^2
\]

\noindent
\textbf{b)} 求 $\tan x$ 在 $x=\frac{\pi}{4}$ 附近的前四项泰勒展开式

\noindent
\textbf{c)} 从上式近似计算
\[
\tan \frac{5\pi}{18} \approx 1+\frac{\pi}{18}+\frac{\pi^2}{648}+\frac{\pi^3}{17496}
\]

\begin{solution}
\noindent
\textbf{a) 计算高阶导数}

\[
y = \tan x
\]
\[
\frac{dy}{dx} = \sec^2 x = 1 + \tan^2 x = 1 + y^2
\]

\noindent
对 $x$ 再求两次导数:
\[
\frac{d^2 y}{dx^2} = \frac{d}{dx}\left(\frac{dy}{dx}\right) = \frac{d}{dx}(1+y^2) = 2y \frac{dy}{dx}
\]

\[
\frac{d^3 y}{dx^3} = \frac{d}{dx}\left(2y \frac{dy}{dx}\right) = 2y \frac{d^2 y}{dx^2} + 2\left(\frac{dy}{dx}\right)^2
\]

\noindent
\textbf{b) 泰勒展开式}

在 $x=a=\frac{\pi}{4}$ 处计算各阶导数:
\[
y = \tan a = 1
\]
\[
\frac{dy}{dx} = 1 + y^2 = 2
\]
\[
\frac{d^2 y}{dx^2} = 2 y \frac{dy}{dx} = 2 \cdot 1 \cdot 2 = 4
\]
\[
\frac{d^3 y}{dx^3} = 2 y \frac{d^2 y}{dx^2} + 2\left(\frac{dy}{dx}\right)^2 = 2\cdot1\cdot4 + 2\cdot2^2 = 16
\]

\noindent
泰勒展开公式:
\[
\tan x \approx f(a) + (x-a)f'(a) + \frac{(x-a)^2}{2!}f''(a) + \frac{(x-a)^3}{3!}f'''(a) + \dots
\]

代入各阶导数:
\[
\tan x \approx 1 + 2(x-\frac{\pi}{4}) + 2(x-\frac{\pi}{4})^2 + \frac{8}{3}(x-\frac{\pi}{4})^3 + \dots
\]

\noindent
\textbf{c) 近似计算 $\tan\frac{5\pi}{18}$}

\[
x-\frac{\pi}{4} = \frac{5\pi}{18} - \frac{\pi}{4} = \frac{\pi}{36}
\]

代入泰勒展开:
\[
\tan \frac{5\pi}{18} \approx 1 + 2\cdot \frac{\pi}{36} + 2\cdot \left(\frac{\pi}{36}\right)^2 + \frac{8}{3} \cdot \left(\frac{\pi}{36}\right)^3
\]

\[
\tan \frac{5\pi}{18} \approx 1 + \frac{\pi}{18} + \frac{2\pi^2}{1296} + \frac{8\pi^3}{139968}
\]

\[
\tan \frac{5\pi}{18} \approx 1 + \frac{\pi}{18} + \frac{\pi^2}{648} + \frac{\pi^3}{17496}
\]
\end{solution}


\question
\noindent
求 $\arctan x$ 的麦克劳林展开式,并利用该展开证明
\[
\pi=\sum_{n=0}^{\infty} f(n)
\]
其中 $f(n)$ 为适当的函数

\begin{solution}
\noindent
先对 $\arctan x$ 求导
\[
\frac{d}{dx}(\arctan x)=\frac{1}{1+x^{2}}
\]

当 $|x|<1$ 时,有几何级数展开
\[
\frac{1}{1+x^{2}}=1-x^{2}+x^{4}-x^{6}+x^{8}-\cdots
\]
\[
=\sum_{n=0}^{\infty}(-1)^{n}x^{2n}
\]

对上式逐项积分,且取积分常数
\[
\arctan x=\sum_{n=0}^{\infty}\int(-1)^{n}x^{2n}\,dx
\]
\[
=\sum_{n=0}^{\infty}\frac{(-1)^{n}x^{2n+1}}{2n+1}+C
\]

令 $x=0$,则 $\arctan 0=0$,从而 $C=0$,于是
\[
\arctan x=\sum_{n=0}^{\infty}\frac{(-1)^{n}x^{2n+1}}{2n+1}
\]

代入 $x=1$ 得
\[
\arctan 1=\sum_{n=0}^{\infty}\frac{(-1)^{n}}{2n+1}
\]
而 $\arctan 1=\frac{\pi}{4}$,故
\[
\frac{\pi}{4}=\sum_{n=0}^{\infty}\frac{(-1)^{n}}{2n+1}
\]

两边同乘 $4$,得
\[
\pi=\sum_{n=0}^{\infty}\frac{4(-1)^{n}}{2n+1}
\]

因此可取
\[
f(n)=\frac{4(-1)^{n}}{2n+1}
\]
\end{solution}

\question
\noindent
利用适当的二项式展开式,证明
\[
\arcsin x = \sum_{r=0}^{\infty} \left[ \frac{\binom{2r}{r}}{2r+1} \frac{2}{4^r} \left(\frac{x}{2}\right)^{2r+1} \right]
\]

\begin{solution}
\noindent
从二项式展开 $(1-x^2)^{-1/2}$ 出发
\[
\frac{1}{\sqrt{1-x^2}} = (1-x^2)^{-1/2} = 1 + \frac{-1/2}{1}(-x^2) + \frac{-1/2(-3/2)}{1\cdot 2}(-x^2)^2 + \frac{-1/2(-3/2)(-5/2)}{1\cdot 2\cdot 3}(-x^2)^3 + O(x^8)
\]

化简符号得
\[
\frac{1}{\sqrt{1-x^2}} = 1 + \frac{1}{2}x^2 + \frac{1/2\cdot 3/2}{1\cdot 2}x^4 + \frac{1/2\cdot 3/2\cdot 5/2}{1\cdot 2\cdot 3}x^6 + O(x^{8})
\]

整理成阶乘形式
\[
\frac{1}{\sqrt{1-x^2}} = 1 + \frac{1\cdot 3}{2\cdot 2}x^4 + \frac{1\cdot 3\cdot 5}{3!} \frac{x^6}{8} + O(x^8)
\]

进一步写成二项式系数
\[
\frac{1}{\sqrt{1-x^2}} = \sum_{r=0}^{\infty} \frac{\binom{2r}{r}}{4^r} x^{2r}
\]

对两边在收敛半径内积分
\[
\int \frac{1}{\sqrt{1-x^2}}\,dx = \int \sum_{r=0}^{\infty} \frac{\binom{2r}{r}}{4^r} x^{2r}\,dx
\]

逐项积分得
\[
\arcsin x = \sum_{r=0}^{\infty} \frac{\binom{2r}{r}}{4^r} \frac{x^{2r+1}}{2r+1} + C
\]

代入 $x=0$ 得 $C=0$,于是
\[
\arcsin x = \sum_{r=0}^{\infty} \frac{\binom{2r}{r}}{2r+1} \frac{x^{2r+1}}{4^r}
\]

重写为所需形式
\[
\arcsin x = \sum_{r=0}^{\infty} \frac{\binom{2r}{r}}{2r+1} \frac{2}{4^r} \left(\frac{x}{2}\right)^{2r+1}
\]
\end{solution}

\question
求下列无穷级数的和:
\[
1+\frac{1}{3\times4}+\frac{1}{5\times4^2}+\frac{1}{7\times4^3}+\frac{1}{9\times4^4}+\cdots
\]

\begin{solution}
考虑对数的幂级数展开:
\[
\ln(1+x) = x - \frac{1}{2}x^2 + \frac{1}{3}x^3 - \frac{1}{4}x^4 + \dots
\]
\[
\ln(1-x) = -x - \frac{1}{2}x^2 - \frac{1}{3}x^3 - \frac{1}{4}x^4 + \dots
\]

两式相减:
\[
\ln(1+x) - \ln(1-x) = 2\left( x + \frac{x^3}{3} + \frac{x^5}{5} + \frac{x^7}{7} + \dots \right)
\]
\[
\ln\left(\frac{1+x}{1-x}\right) = 2 \sum_{k=0}^{\infty} \frac{x^{2k+1}}{2k+1}
\]

在收敛半径内令 $x = \frac{1}{2}$:
\[
\ln\left(\frac{1+\frac{1}{2}}{1-\frac{1}{2}}\right) = 2 \sum_{k=0}^{\infty} \frac{(1/2)^{2k+1}}{2k+1}
\]
\[
\ln 3 = 2 \sum_{k=0}^{\infty} \frac{1}{(2k+1)2^{2k+1}} = \sum_{k=0}^{\infty} \frac{1}{(2k+1)4^k}
\]

注意到原级数的形式与此一致:
\[
1 + \frac{1}{3\cdot 4} + \frac{1}{5\cdot 4^2} + \frac{1}{7\cdot 4^3} + \cdots = \sum_{k=0}^{\infty} \frac{1}{(2k+1)4^k}
\]

\[
\therefore 1+\frac{1}{3\times4}+\frac{1}{5\times4^2}+\frac{1}{7\times4^3}+\cdots = \ln 3
\]
\end{solution}

\question
求下列交错级数的和:
\[
\frac{1}{1} - \frac{1}{1+4} + \frac{1}{1+4+9} - \frac{1}{1+4+9+16} + \frac{1}{1+4+9+16+25} - \cdots
\]

\begin{solution}
首先写成紧凑形式:
\[
\sum_{n=1}^{\infty} \frac{(-1)^{n+1}}{1^2 + 2^2 + \cdots + n^2} = \sum_{n=1}^{\infty} \frac{(-1)^{n+1}}{\frac{1}{6}n(n+1)(2n+1)} = \sum_{n=1}^{\infty} \frac{6(-1)^{n+1}}{n(n+1)(2n+1)}
\]

对 $\frac{1}{n(n+1)(2n+1)}$ 做部分分式分解:
\[
\frac{1}{n(n+1)(2n+1)} = \frac{1}{n} + \frac{1}{n+1} - \frac{4}{2n+1}
\]

因此级数可写为:
\[
\sum_{n=1}^{\infty} 6(-1)^{n+1} \left( \frac{1}{n} + \frac{1}{n+1} - \frac{4}{2n+1} \right)
= 6\sum_{n=1}^{\infty} \frac{(-1)^{n+1}}{n} + 6\sum_{n=1}^{\infty} \frac{(-1)^{n+1}}{n+1} - 24\sum_{n=1}^{\infty} \frac{(-1)^{n+1}}{2n+1}
\]

分别处理各个级数:
\[
\sum_{n=1}^{\infty} \frac{(-1)^{n+1}}{n} = \ln 2
\]

\[
6\sum_{n=1}^{\infty} \frac{(-1)^{n+1}}{n+1} = 6 - 6 \ln 2
\]

再利用 $\arctan x$ 展开:
\[
\arctan x = \sum_{n=0}^{\infty} \frac{(-1)^n x^{2n+1}}{2n+1} \implies \arctan 1 = \sum_{n=0}^{\infty} \frac{(-1)^n}{2n+1} = \frac{\pi}{4}
\]

\[
24 \sum_{n=1}^{\infty} \frac{(-1)^{n+1}}{2n+1} = 24 \left(\frac{\pi}{4} - 1\right) = 6\pi - 24
\]

将所有部分相加:
\[
6\ln 2 + (6 - 6\ln 2) - (6\pi - 24) = 6 - 6\pi + 24 = 30 - 6\pi
\]

因此原级数的和为:
\[
\frac{1}{1} - \frac{1}{1+4} + \frac{1}{1+4+9} - \cdots = 6(\pi - 3)
\]
\end{solution}

\question
证明
\[
\sum_{r=1}^{\infty}\left[\frac{2r+3}{(r+1)3^r}\right]=3\ln\left(\frac{3}{2}\right).
\]

\begin{solution}
先对通项进行代数变形:
\[
\frac{2r+3}{(r+1)3^r}
=\frac{2(r+1)+1}{r+1}\left(\frac{1}{3}\right)^r
=2\left(\frac{1}{3}\right)^r+\frac{1}{r+1}\left(\frac{1}{3}\right)^r
\]

因此
\[
\sum_{r=1}^{\infty}\left[\frac{2r+3}{(r+1)3^r}\right]
=\sum_{r=1}^{\infty}2\left(\frac{1}{3}\right)^r
+\sum_{r=1}^{\infty}\frac{1}{r+1}\left(\frac{1}{3}\right)^r
\]

第一项为等比级数,其首项为 $\frac{2}{3}$,公比为 $\frac{1}{3}$,故
\[
\sum_{r=1}^{\infty}2\left(\frac{1}{3}\right)^r
=\frac{\frac{2}{3}}{1-\frac{1}{3}}=1
\]

对第二项,考虑
\[
\ln(1-x)=-x-\frac{1}{2}x^2-\frac{1}{3}x^3-\frac{1}{4}x^4-\cdots
\]
两边同除以 $-x$,得
\[
-\frac{1}{x}\ln(1-x)
=1+\frac{1}{2}x+\frac{1}{3}x^2+\frac{1}{4}x^3+\cdots
=1+\sum_{r=1}^{\infty}\frac{1}{r+1}x^r
\]

取 $x=\frac{1}{3}$,由于 $|x|<1$,级数收敛,于是
\[
-3\ln\left(\frac{2}{3}\right)
=1+\sum_{r=1}^{\infty}\frac{1}{r+1}\left(\frac{1}{3}\right)^r
\]
从而
\[
\sum_{r=1}^{\infty}\frac{1}{r+1}\left(\frac{1}{3}\right)^r
=-3\ln\left(\frac{2}{3}\right)-1
\]

将两部分相加:
\[
\sum_{r=1}^{\infty}\left[\frac{2r+3}{(r+1)3^r}\right]
=1-3\ln\left(\frac{2}{3}\right)-1
=-3\ln\left(\frac{2}{3}\right)
\]

整理得
\[
\sum_{r=1}^{\infty}\left[\frac{2r+3}{(r+1)3^r}\right]
=3\ln\left(\frac{3}{2}\right)
\]
\end{solution}

\question
通过考虑 $\ln(1-x^2)$ 与 $\ln\left(\frac{1+x}{1-x}\right)$ 的级数展开,或用其他方法,求下列级数的精确值:
\[
\sum_{r=1}^{\infty} \left[\left(\frac{1}{2r}+\frac{1}{2r+1}\right)\left(\frac{1}{4}\right)^r\right].
\]

\begin{solution}
由
\[
\ln(1+x)=x-\frac{1}{2}x^2+\frac{1}{3}x^3-\cdots,\quad |x|<1
\]
\[
\ln(1-x)=-x-\frac{1}{2}x^2-\frac{1}{3}x^3-\cdots,\quad |x|<1
\]
可得
\[
\ln(1-x^2)= -x^2-\frac{1}{2}x^4-\frac{1}{3}x^6-\cdots
\]

因此
\[
-\frac{1}{2}\ln(1-x^2)=\frac{1}{2}x^2+\frac{1}{4}x^4+\frac{1}{6}x^6+\cdots
\]

令 $x=\frac{1}{2}$,得
\[
\frac{1}{2}\cdot\frac{1}{4}+\frac{1}{4}\cdot\frac{1}{16}+\frac{1}{6}\cdot\frac{1}{64}+\cdots
=-\frac{1}{2}\ln\frac{3}{4}
\]

再考虑
\[
\ln\left(\frac{1+x}{1-x}\right)
=2\left(x+\frac{x^3}{3}+\frac{x^5}{5}+\cdots\right)
\]
从而
\[
\frac{1}{2x}\ln\left(\frac{1+x}{1-x}\right)
=1+\frac{x^2}{3}+\frac{x^4}{5}+\cdots
\]

令 $x=\frac{1}{2}$,得
\[
1+\frac{1}{3}\frac{1}{2^2}+\frac{1}{5}\frac{1}{2^4}+\frac{1}{7}\frac{1}{2^6}+\cdots
=\ln 3
\]

因此
\[
\frac{1}{3}\cdot\frac{1}{4}+\frac{1}{5}\cdot\frac{1}{16}+\frac{1}{7}\cdot\frac{1}{64}+\cdots
=\ln 3-1
\]

将两部分相加:
\[
\sum_{r=1}^{\infty} \left[\left(\frac{1}{2r}+\frac{1}{2r+1}\right)\left(\frac{1}{4}\right)^r\right]
=-\frac{1}{2}\ln\frac{3}{4}+\ln 3-1
\]

化简得
\[
\sum_{r=1}^{\infty} \left[\left(\frac{1}{2r}+\frac{1}{2r+1}\right)\left(\frac{1}{4}\right)^r\right]
=\frac{1}{2}\ln 12-1
\]
\end{solution}

\question
\noindent
\textbf{a)} 用适当的积分方法计算
\[
\int_{0}^{1} x^{3}\arctan x\,dx
\]

\noindent
\textbf{b)} 求 $\arctan x$ 的无穷级数展开,并利用该展开验证 a) 的结果

\begin{solution}
\noindent
\textbf{a)} 先用分部积分法  
取
\[
u=\arctan x,\quad dv=x^{3}\,dx
\]
则
\[
du=\frac{1}{1+x^{2}}\,dx,\quad v=\frac{x^{4}}{4}
\]

于是
\[
\int_{0}^{1} x^{3}\arctan x\,dx
=\left[\frac{x^{4}}{4}\arctan x\right]_{0}^{1}
-\frac{1}{4}\int_{0}^{1}\frac{x^{4}}{1+x^{2}}\,dx
\]

边界项为
\[
\left[\frac{x^{4}}{4}\arctan x\right]_{0}^{1}
=\frac{\pi}{16}
\]

化简被积函数
\[
\frac{x^{4}}{1+x^{2}}=x^{2}-1+\frac{1}{1+x^{2}}
\]

因此
\[
\int_{0}^{1}\frac{x^{4}}{1+x^{2}}\,dx
=\int_{0}^{1}x^{2}\,dx-\int_{0}^{1}1\,dx+\int_{0}^{1}\frac{1}{1+x^{2}}\,dx
\]
\[
=\frac{1}{3}-1+\frac{\pi}{4}
\]

代回得
\[
\int_{0}^{1} x^{3}\arctan x\,dx
=\frac{\pi}{16}-\frac{1}{4}\left(\frac{1}{3}-1+\frac{\pi}{4}\right)
\]
\[
=\frac{1}{6}
\]

\noindent
\textbf{b)} 由
\[
\frac{1}{1+x^{2}}=1-x^{2}+x^{4}-x^{6}+\cdots
\]
两边积分得
\[
\arctan x
=x-\frac{x^{3}}{3}+\frac{x^{5}}{5}-\frac{x^{7}}{7}+\cdots
=\sum_{n=0}^{\infty}\frac{(-1)^{n}x^{2n+1}}{2n+1}
\]

代入积分
\[
\int_{0}^{1} x^{3}\arctan x\,dx
=\int_{0}^{1}x^{3}\sum_{n=0}^{\infty}\frac{(-1)^{n}x^{2n+1}}{2n+1}\,dx
\]
\[
=\sum_{n=0}^{\infty}\frac{(-1)^{n}}{2n+1}
\int_{0}^{1}x^{2n+4}\,dx
\]
\[
=\sum_{n=0}^{\infty}\frac{(-1)^{n}}{(2n+1)(2n+5)}
\]

作部分分式分解
\[
\frac{1}{(2n+1)(2n+5)}
=\frac{1}{4}\left(\frac{1}{2n+1}-\frac{1}{2n+5}\right)
\]

于是
\[
\int_{0}^{1} x^{3}\arctan x\,dx
=\frac{1}{4}\sum_{n=0}^{\infty}(-1)^{n}
\left(\frac{1}{2n+1}-\frac{1}{2n+5}\right)
\]

该级数为望远镜型,化简得
\[
=\frac{1}{4}\left(1-\frac{1}{3}\right)
=\frac{1}{6}
\]

与 a) 的结果一致
\end{solution}

\question
已知
\[
1-\frac{1}{3}+\frac{1}{5}-\frac{1}{7}+\cdots=\frac{\pi}{4},
\quad
1-\frac{1}{4}+\frac{1}{9}-\frac{1}{16}+\cdots=\frac{\pi^2}{12},
\quad
1-\frac{1}{2}+\frac{1}{3}-\frac{1}{4}+\cdots=\ln 2.
\]
在假设下列积分收敛的前提下,求
\[
\int_0^1 (\ln x)(\arctan x)\,dx
\]
的精确值。

\begin{solution}
先利用反正切函数的幂级数展开。由
\[
\frac{d}{dx}(\arctan x)=\frac{1}{1+x^2}=1-x^2+x^4-x^6+\cdots
\]
逐项积分得
\[
\arctan x=x-\frac{1}{3}x^3+\frac{1}{5}x^5-\frac{1}{7}x^7+\cdots
=\sum_{n=0}^{\infty}\frac{(-1)^n x^{2n+1}}{2n+1},
\quad 0\le x\le1.
\]

将其代入积分,并交换求和与积分次序:
\[
\int_0^1 (\arctan x)(\ln x)\,dx
=\sum_{n=0}^{\infty}\frac{(-1)^n}{2n+1}
\int_0^1 x^{2n+1}\ln x\,dx.
\]

对积分作分部积分,取
$u=\ln x$,$dv=x^{2n+1}dx$,则
\[
\int_0^1 x^{2n+1}\ln x\,dx
=\left[\frac{x^{2n+2}\ln x}{2n+2}\right]_0^1
-\frac{1}{2n+2}\int_0^1 x^{2n+1}\,dx.
\]
端点项为零,因此
\[
\int_0^1 x^{2n+1}\ln x\,dx
=-\frac{1}{(2n+2)^2}.
\]

于是
\[
\int_0^1 (\arctan x)(\ln x)\,dx
=-\sum_{n=0}^{\infty}\frac{(-1)^n}{(2n+1)(2n+2)^2}.
\]

对通项作部分分式分解:
\[
\frac{1}{(2n+1)(2n+2)^2}
=\frac{1}{2n+1}-\frac{1}{2n+2}-\frac{1}{(2n+2)^2}.
\]

代回得
\[
\int_0^1 (\arctan x)(\ln x)\,dx
=-\sum_{n=0}^{\infty}(-1)^n
\left[
\frac{1}{2n+1}
-\frac{1}{2n+2}
-\frac{1}{(2n+2)^2}
\right].
\]

将其拆分为三个已知级数:
\[
\sum_{n=0}^{\infty}\frac{(-1)^n}{2n+1}=\frac{\pi}{4},
\quad
\sum_{n=0}^{\infty}\frac{(-1)^n}{n+1}=\ln 2,
\quad
\sum_{n=0}^{\infty}\frac{(-1)^n}{(n+1)^2}=\frac{\pi^2}{12}.
\]

因此
\[
\int_0^1 (\arctan x)(\ln x)\,dx
=-\left[
\frac{\pi}{4}
-\frac{1}{2}\ln 2
-\frac{1}{4}\cdot\frac{\pi^2}{12}
\right].
\]

化简得
\[
\int_0^1 (\arctan x)(\ln x)\,dx
=\frac{\pi^2}{48}+\frac{1}{2}\ln 2-\frac{\pi}{4}
=\frac{\pi^2-12\pi+24\ln 2}{48}.
\]
\end{solution}
\begin{solution}
设
\[
I(a)=\int_0^1 (\ln x)\arctan(ax)\,dx,
\quad a\ge0.
\]
所求积分为 $I(1)$。

对参数 $a$ 求导,有
\[
I'(a)=\int_0^1 (\ln x)\frac{x}{1+a^2x^2}\,dx.
\]

交换积分顺序,令 $u=ax$,得
\[
I'(a)=\int_0^1 x\ln x
\int_0^{a}\frac{du}{1+u^2x^2}\,dx.
\]

改为先对 $x$ 积分。注意到
\[
\int_0^1 \frac{x\ln x}{1+u^2x^2}\,dx
=-\frac{1}{2u^2}\ln(1+u^2),
\]
因此
\[
I'(a)
=-\frac12\int_0^{a}\frac{\ln(1+u^2)}{u^2}\,du.
\]

对该积分作分部积分,取
$U=\ln(1+u^2)$,$dV=\frac{du}{u^2}$,
则
\[
I'(a)
=-\frac12
\left[
-\frac{\ln(1+u^2)}{u}
+\int_0^{a}\frac{2u}{1+u^2}\cdot\frac{1}{u}\,du
\right].
\]

化简得
\[
I'(a)
=\frac12\frac{\ln(1+a^2)}{a}
-\int_0^{a}\frac{du}{1+u^2}.
\]

于是
\[
I'(a)=\frac12\frac{\ln(1+a^2)}{a}-\arctan a.
\]

再对 $a$ 从 $0$ 积分到 $1$,注意 $I(0)=0$,得到
\[
I(1)=\int_0^1
\left(
\frac12\frac{\ln(1+a^2)}{a}-\arctan a
\right)da.
\]

分项计算。首先
\[
\int_0^1 \arctan a\,da
=\frac{\pi}{4}-\frac12\ln2.
\]

其次
\[
\int_0^1 \frac{\ln(1+a^2)}{a}\,da
=\int_0^1\int_0^{a^2}\frac{1}{a(1+t)}\,dt\,da
=\int_0^1\frac{1-a^2}{a(1+a^2)}\,da
=\frac{\pi^2}{12}.
\]

综上
\[
I(1)
=\frac12\cdot\frac{\pi^2}{12}
-\left(\frac{\pi}{4}-\frac12\ln2\right).
\]

因此
\[
\int_0^1 (\ln x)(\arctan x)\,dx
=\frac{\pi^2}{48}+\frac12\ln2-\frac{\pi}{4}
=\frac{\pi^2-12\pi+24\ln2}{48}.
\]
\end{solution}

\end{questions}
\pagebreak

\begin{center}
  {\fontsize{30pt}{26pt}\selectfont
    \hypertarget{二项展开式}{二项展开式} \label{二项展开式}
  }
\end{center}
\separator
\vspace{1pt}

\begin{questions}
    \question 求\((x^2 + x + y)^5\) 展开式中\( x^5 y^2 \) 的系数。  
    \begin{solution}
        系数为
        \[
        \comb{5}{2}\comb{3}{1}\comb{2}{2}=30
        \]
    \end{solution}

    \question 求\(\left(x \sqrt{y} - y \sqrt{x}\right)^4\) 的展开式中\( xy \) 的系数。
    \begin{solution}
        观察发现$$(x\sqrt{y}-y\sqrt{x})^4 = x^2y^2(\sqrt{x}-\sqrt{y})^4$$,
        即只需求$(\sqrt{x}-\sqrt{y})^4$展开式中的含 $xy$ 项的系数 $\comb{4}{2}=6$
    \end{solution}

    \question 求在 $\left(x+\dfrac{4}{x}+4\right)^6$ 的展开式中$x^4$ 的系数。
    \begin{solution}
        原式可化为
        \[
        \left(x + \dfrac{4}{x} + 4\right)^6 = \left(\dfrac{x^2 + 4x + 4}{x}\right)^6 = \dfrac{(x+2)^{12}}{x^6}
        \]
        即求 $(x+2)^{12}$ 展开式中$x^{10}$ 的系数
        \[
        \comb{12}{2} \cdot 2^2 = 264
        \]
    \end{solution}

    \question 求 \( (1 + x^2) + (1 + x^2)^2 + \cdots + (1 + x^2)^{10} \) 展开式中 \( x^6 \) 的系数。  
    \begin{solution}
        原式为$$\frac{(1+x^2) \left[ (1+x^2)^{10}-1 \right]}{(1+x^2)-1} = \frac{(1+x^2)^{11}-(1+x^2)}{x^2}
        $$ 
        
        展开式$(1+x^2)^{11}$中 $x^8$ 系数为 $\comb{10}{4} = \dfrac{11 \cdot 10 \cdot 9 \cdot 8}{4 \cdot 3 \cdot 2} = 330$,故原式中 $x^6$ 系数为 $330$.
    \end{solution}

    \question
求下列多项式中 $x^{50}$ 的系数:

(a) $(1+x)^{1000} + x(1+x)^{999} + x^2(1+x)^{998} + \dots + x^{1000}$

(b) $(1+x) + 2(1+x)^2 + 3(1+x)^3 + \dots + 1000(1+x)^{1000}$

\begin{solution}
    (a) 利用等比数列求和公式以及二项式定理,我们得到:
    \begin{align*}
    (1+x)^{1000} + x(1+x)^{999} + \dots + x^{1000} &= \frac{(1+x)^{1001} - x^{1001}}{(1+x) - x} \\
    &= \frac{(1+x)^{1001} - x^{1001}}{1} \\
    &= (1+x)^{1001} - x^{1001} \\
    &= 1 + \comb{1001}{1}x + \comb{1001}{2}x^2 + \dots + \comb{1001}{1001}x^{1001} - x^{1001}
    \end{align*}
    因此,所求的系数等于:
    \[ \comb{1001}{50} = \frac{1001!}{50!\,951!} \]

    (b) 设所给多项式为 $P(x)$。我们可以写出:
    \begin{align*}
    x P(x) &= (1+x)P(x) - P(x) \\
    &= \left[(1+x)^2 + 2(1+x)^3 + \dots + 999(1+x)^{1000} + 1000(1+x)^{1001}\right] \\
    &\quad - \left[(1+x) + 2(1+x)^2 + 3(1+x)^3 + \dots + 1000(1+x)^{1000}\right] \\
    &= 1000(1+x)^{1001} - \left[(1+x) + (1+x)^2 + (1+x)^3 + \dots + (1+x)^{1000}\right] \\
    &= 1000(1+x)^{1001} - \frac{(1+x)[(1+x)^{1000} - 1]}{(1+x) - 1} \\
    &= 1000(1+x)^{1001} - \frac{(1+x)^{1001} - (1+x)}{x}
    \end{align*}
    从而得出:
    \[ P(x) = \frac{1000(1+x)^{1001}}{x} - \frac{(1+x)^{1001} - (1+x)}{x^2} \]
    
    为了找到 $P(x)$ 中 $x^{50}$ 的系数,我们需要找到 $\frac{1000(1+x)^{1001}}{x}$ 中 $x^{50}$ 的系数(即 $(1+x)^{1001}$ 中 $x^{51}$ 的系数),以及 $\frac{(1+x)^{1001}}{x^2}$ 中 $x^{50}$ 的系数(即 $(1+x)^{1001}$ 中 $x^{52}$ 的系数)。
    
    展开式如下:
    \begin{align*}
    P(x) &= 1000 \sum_{k=0}^{1001} \comb{1001}{k} x^{k-1} - \sum_{k=0}^{1001} \comb{1001}{k} x^{k-2} + \frac{1+x}{x^2} \\
    \end{align*}
    
    因此,所求的系数为:
    \begin{align*}
    1000 \comb{1001}{51} - \comb{1001}{52} &= \frac{1000 \cdot 1001!}{51!\,950!} - \frac{1001!}{52!\,950!} \\
    &= \frac{1001!}{52!\,950!} \left[52 \cdot 1000 - 950\right] \\
    &= \frac{51050 \cdot 1001!}{52!\,950!}
    \end{align*}
\end{solution}

    \question 求\(\left[a + (b + c)^2\right]^8\) 展开式中 \( a^5 b^4 c^2 \) 的系数。
    \begin{solution}
        令$(b+c)^2=x,$展开式为
        $$(a+x)^8 = \sum_{k=0}^{8} \frac{8!}{k!(8-k)!} a^k x^{8-k} $$
        欲求$a^5b^4c^2$ 系数,于是$k=5$,有
        $$\frac{8!}{5!3!} a^5 [(b+c)^2]^3 = 56a^5 (b+c)^6 $$
        观察
        $$(b+c)^6 = \sum_{t=0}^{6} \frac{6!}{t!(6-t)!} b^t c^{6-t}$$
        欲求$b^4c^2$ 系数,于是 $t=4$,有
        $$\frac{6!}{4!2!} b^4 c^2 = 15b^4 c^2 $$
        $\therefore a^5b^4c^2$ 系数为 $56 \cdot 15 = 840$
    \end{solution}

    \question 求 $(-xy+2x+3y-6)^{6}$的展开式中$x^3y^3$的系数。
    \begin{solution}
        考虑
        \[
        \frac{6!}{n_1!n_2!n_3!n_4!} (-xy)^{n_1}(2x)^{n_2}(3y)^{n_3}(-6)^{n_4}=Ax^3y^3,
        \]
        其中$n_1+n_2+n_3+n_4=6$
        \begin{itemize}
        \item 若 \(n_1=0\),则 \(n_2=3, n_3=3, n_4=0\),
        \[
        \frac{6!}{0!3!3!0!}(-xy)^0(2x)^3(3y)^3(-6)^0 = 4320x^3y^3
        \]
        \item 若 \(n_1=1\),则 \(n_2=2, n_3=2, n_4=1\),
        \[
        \frac{6!}{1!2!2!1!}(-xy)^1(2x)^2(3y)^2(-6)^1 = 38880x^3y^3
        \]
        \item 若 \(n_1=2\),则 \(n_2=1, n_3=1, n_4=2\),
        \[
        \frac{6!}{2!1!1!2!}(-xy)^2(2x)^1(3y)^1(-6)^2 = 38880x^3y^3
        \]
        \item 若 \(n_1=3\),则 \(n_2=0, n_3=0, n_4=3\),
        \[
        \frac{6!}{3!0!0!3!}(-xy)^3(2x)^0(3y)^0(-6)^3  = 4320x^3y^3
        \]
        \end{itemize}
        因此$x^3y^3$的系数为$4320 + 38880 + 38880 + 4320 = 86400$
    \end{solution}

    \question \(\left(x + \dfrac{a}{x}\right) \left(2x - \dfrac{1}{x}\right)^5\) 的展开式中各项系数的和为 2,则该展开式中的常数项为  
    \begin{solution}
        \textbf{解法一}
        
        令$x=1$ 得 $a=1$.故原式为 $$(x+\frac{1}{x})(2x-\frac{1}{x})^5$$ 
        通项
        $$T_{r+1}=\comb{5}{r}(2x)^{5-r}(-x^{-1})^r=\comb{5}{r}(-1)^r 2^{5-r}x^{5-2r}$$
        由$5-2r=1$得$r=2$,对应的常数项$=80$,由$5-2r=-1$得
        $r=3$,对应的常数项$=-40$,故所求的常数项为$40$

        \textbf{解法二}
        
        用组合提取法,把原式看做$6$个因式相乘,若第$1$个括号提出$x$,从余下的$5$个括号中选$2$个提出$x$,选$3$个提出$\dfrac{1}{x}$,
        
        若第$1$个括号提出$\dfrac{1}{x}$,从余下的$5$个括号中选$2$个提出$\dfrac{1}{x}$,选$3$个提出$x$.
        
        故常数项$$x \cdot \comb{5}{2}(2x)^2 \cdot \comb{3}{3}(-\frac{1}{x})^3 + \frac{1}{x} \cdot \comb{5}{2}(-\frac{1}{x})^2 \cdot \comb{3}{3}(2x)^3 =-40+80=40$$
    \end{solution}

    \question 试求 
    \[
    (\comb{2}{2}+\comb{3}{2}x+\comb{4}{2}x^2+\comb{5}{2}x^3+\comb{6}{2}x^4+\comb{7}{2}x^5)^4
    \] 
    展开式中 \(x^5\) 的系数。
    \begin{solution}
        考虑 
        \[
        f(x) = \left(\sum_{n=0}^\infty \comb{n+2}{2} x^n \right)^4
        \]
        设 
        \[
        g(x)=\frac{x^2}{1-x} = \sum_{n=0}^\infty x^{n+2}
        \]
        则
        \[
        g'(x) = \frac{x(2-x)}{(x-1)^2}= \sum_{n=0}^\infty (n+2)x^{n+1}
        \]
        \[
        g''(x) = -\frac{2}{(x-1)^3}=\sum_{n=0}^\infty (n+2)(n+1)x^n = 2 \sum_{n=0}^\infty \comb{n+2}{2} x^n
        \]
        于是
        \[
        f(x) = \left( \sum_{n=0}^\infty \comb{n+2}{2} x^n \right)^4 = \left(- \frac{1}{(1 - x)^3} \right)^4 = \frac{1}{(1 - x)^{12}}
        \]
        故\((\comb{2}{2}+\comb{3}{2}x+\comb{4}{2}x^2+\comb{5}{2}x^3+\comb{6}{2}x^4+\comb{7}{2}x^5)^4\)  展开式中 \(x^5\) 的系数为
        \[
        \comb{12 + 5 - 1}{5} = 4368
        \]
    \end{solution}

    \question 若 
    \[
    (1 + x + x^2)^6 = a_0 + a_1 x + a_2 x^2 + \cdots + a_{12} x^{12},
    \]
    求\(a_2 + a_4 + \cdots + a_{12}\)的值。
    \begin{solution}
        设$f(x)=(1 + x + x^2)^6 = a_0 + a_1 x + a_2 x^2 + \cdots + a_{12} x^{12}$,则
            $$a_2 + a_4 + \cdots + a_{12}=\frac{f(1)+f(-1)}{2}-a_0=\frac{3^6+1}{2}-1=364$$
    \end{solution}

    \question 已知非零数列$\{a_n\}$满足$a_{1}=1,a_{2}=3,a_{n}(a_{n+2}-1)=a_{n+1}(a_{n+1}-1),n\in N,$求 $$\comb{2023}{0}a_{1}+\comb{2023}{1}a_{2}+\comb{2023}{2}a_{3}+\cdot\cdot\cdot+\comb{2023}{2023}a_{2024}$$ 
    \begin{solution}
        由递推关系得
        \[
        \frac{a_{n+2}-1}{a_{n+1}} = \frac{a_{n+1}-1}{a_n}= \frac{a_2-1}{a_1} = \frac{3-1}{1} = 2
        \]
        即
        \[
         a_{n+1}-1 = 2a_n \Rightarrow a_{n+1}+1 = 2(a_n+1) 
        \]
        数列 $\{a_n+1\}$ 是首项为$2$,公比为$2$的等比数列, 故$a_n = 2^n - 1 \quad$,则
        \begin{align*}
        &\comb{2023}{0}a_{1}+\comb{2023}{1}a_{2}+\comb{2023}{2}a_{3}+\comb{2023}{3}a_{4}+\cdots+\comb{2023}{2023}a_{2024} \\
        &=\ \comb{2023}{0}(2^{1}-1)+\comb{2023}{1}(2^{2}-1)+\comb{2023}{2}(2^{3}-1)+\cdots+\comb{2023}{2023}(2^{2024}-1) \\
        &=\ 2\left(\comb{2023}{0}2^{0}+\comb{2023}{1}2^{1}+\comb{2023}{2}2^{2}+\cdots+\comb{2023}{2023}2^{2023}\right) \\
        &\qquad- \left(\comb{2023}{0}+\comb{2023}{1}+\comb{2023}{2}+\cdots+\comb{2023}{2023}\right) \\
        &=\ 2(1+2)^{2023} - 2^{2023} =\ 2 \cdot 3^{2023} - 2^{2023}
        \end{align*}
    \end{solution}

    \question 计算级数
\[
2 + \frac{2 \cdot 5}{3!} + \frac{2 \cdot 5 \cdot 7}{5!} + \frac{2 \cdot 5 \cdot 7 \cdot 9}{7!} + \dots
\]
的值,并以根式形式表示。

\begin{solution}
注意到该级数的分子是连续奇数的乘积,而分母是阶乘。我们可以尝试与二项式展开式的广义形式联系起来:
\[
(1+x)^{-\frac{1}{2}} = 1 - \frac{1}{2}x + \frac{(-\frac{1}{2})(-\frac{3}{2})}{2!} x^2 - \frac{(-\frac{1}{2})(-\frac{3}{2})(-\frac{5}{2})}{3!} x^3 + \dots
\]

通过调整系数,可以将原级数写成:
\[
2 \left[ 1 + \frac{3 \cdot 5}{2!} \left(\frac{2}{3}\right)^2 + \frac{3 \cdot 5 \cdot 7}{3!} \left(\frac{2}{3}\right)^3 + \dots \right].
\]

这个形式与广义二项式展开 \((1 - x)^{-1/2}\) 对应,其中 \(x = \frac{2}{3}\)。于是:
\[
\sum = 2 \cdot \left(1 - \frac{2}{3}\right)^{-1/2} = 2 \cdot \left(\frac{1}{3}\right)^{-1/2} = 2 \cdot \sqrt{3} = 2\sqrt{3}.
\]

因此,原级数的值为
\[
\boxed{2\sqrt{3}}.
\]
\end{solution}

\question 求下列无穷级数的和
\[
S=\frac{3}{8}+\frac{3\times9}{8\times16}+\frac{3\times9\times15}{8\times16\times24}
+\frac{3\times9\times15\times21}{8\times16\times24\times32}+\cdots
\]

\begin{solution}
将一般项改写为
\[
S=\sum_{k=1}^{\infty}
\frac{3^k(1\cdot3\cdot5\cdots(2k-1))}{8^{k+1}(1\cdot2\cdot3\cdots k)}.
\]

整理得
\[
S=\sum_{k=1}^{\infty}
\frac{\left(\frac12\right)\left(\frac32\right)\cdots\left(\frac{2k-1}{2}\right)}{k!}
\left(\frac34\right)^k.
\]

注意到
\[
\left(\frac12\right)\left(\frac32\right)\cdots\left(\frac{2k-1}{2}\right)
=(-1)^k\left(-\frac12\right)\left(-\frac32\right)\cdots\left(-\frac{2k-1}{2}\right),
\]
于是
\[
S=\sum_{k=1}^{\infty}
\frac{\left(-\frac12\right)\left(-\frac32\right)\cdots\left(-\frac{2k-1}{2}\right)}{k!}
\left(-\frac34\right)^k.
\]

两边同时加 $1$,得到
\[
1+S=\sum_{k=0}^{\infty}
\frac{\left(-\frac12\right)\left(-\frac32\right)\cdots\left(-\frac{2k-1}{2}\right)}{k!}
\left(-\frac34\right)^k.
\]

这是二项式展开,因此
\[
1+S=\left(1-\frac34\right)^{-\frac12}.
\]

计算得
\[
1+S=\left(\frac14\right)^{-\frac12}=2,
\]
从而
\[
S=1.
\]
\end{solution}

\question
求下列表达式的无穷和
\[
\sum_{k=1}^{\infty}\left[\prod_{r=1}^{k}\left(\frac{8r-7}{40r}\right)\right].
\]

\begin{solution}
先将前几项写出以观察其结构:
\begin{align*}
\sum_{k=1}^{\infty}\left[\prod_{r=1}^{k}\left(\frac{8r-7}{40r}\right)\right]
&= \prod_{r=1}^{1}\left(\frac{8r-7}{40r}\right)
 + \prod_{r=1}^{2}\left(\frac{8r-7}{40r}\right)
 + \prod_{r=1}^{3}\left(\frac{8r-7}{40r}\right)
 + \cdots \\
&= \frac{1}{40}
 + \frac{1}{40}\cdot\frac{9}{80}
 + \frac{1}{40}\cdot\frac{9}{80}\cdot\frac{17}{120}
 + \frac{1}{40}\cdot\frac{9}{80}\cdot\frac{17}{120}\cdot\frac{25}{160}
 + \cdots .
\end{align*}

将每一项写成阶乘形式:
\begin{align*}
&= \frac{1}{40\cdot 1!}
 + \frac{1\cdot 9}{40^2\cdot(1\cdot 2)}
 + \frac{1\cdot 9\cdot 17}{40^3\cdot(1\cdot 2\cdot 3)}
 + \frac{1\cdot 9\cdot 17\cdot 25}{40^4\cdot(1\cdot 2\cdot 3\cdot 4)}
 + \cdots .
\end{align*}

注意到
\[
40 = (-8)(-5),
\]
于是上式可改写为
\begin{align*}
&= \frac{1}{(-8)(-5)\,1!}
 + \frac{1\cdot 9}{(-8)^2(-5)^2\,2!}
 + \frac{1\cdot 9\cdot 17}{(-8)^3(-5)^3\,3!}
 + \frac{1\cdot 9\cdot 17\cdot 25}{(-8)^4(-5)^4\,4!}
 + \cdots .
\end{align*}

进一步整理得
\begin{align*}
\sum_{k=1}^{\infty}\left[\prod_{r=1}^{k}\left(\frac{8r-7}{40r}\right)\right]
&= \frac{-\frac{1}{8}}{1!}\left(-\frac{1}{5}\right)
 + \frac{\left(-\frac{1}{8}\right)\left(-\frac{9}{8}\right)}{2!}\left(-\frac{1}{5}\right)^2 \\
&\quad + \frac{\left(-\frac{1}{8}\right)\left(-\frac{9}{8}\right)\left(-\frac{17}{8}\right)}{3!}\left(-\frac{1}{5}\right)^3
 + \cdots .
\end{align*}

这是二项式级数展开式
\[
(1+x)^{\alpha}
= \sum_{k=0}^{\infty}\frac{\alpha(\alpha-1)\cdots(\alpha-k+1)}{k!}x^k
\]
在 \(\alpha=-\frac{1}{8}\)、\(x=-\frac{1}{5}\) 情形下去掉首项后的结果,因此
\begin{align*}
\sum_{k=1}^{\infty}\left[\prod_{r=1}^{k}\left(\frac{8r-7}{40r}\right)\right]
&= \left(1-\frac{1}{5}\right)^{-\frac{1}{8}} - 1 \\
&= \left(\frac{4}{5}\right)^{-\frac{1}{8}} - 1 \\
&= \sqrt[8]{\frac{5}{4}} - 1.
\end{align*}
\end{solution}

\question
已知级数
\[
S=\frac{5}{18}-\frac{5\times8}{18\times24}
+\frac{5\times8\times11}{18\times24\times30}
-\frac{5\times8\times11\times14}{18\times24\times30\times36}
+\cdots
\]
收敛,求其和。

\begin{solution}
原级数可写为
\begin{align*}
S
&=\frac{5}{18}-\frac{5\times8}{18\times24}
+\frac{5\times8\times11}{18\times24\times30}
-\frac{5\times8\times11\times14}{18\times24\times30\times36}
+\cdots
\end{align*}

将分母拆分为 $18=6\times3,\;24=6\times4,\;30=6\times5,\dots$,则
\begin{align*}
S
&=\frac{5}{6\times3}
-\frac{5\times8}{6^2(3\times4)}
+\frac{5\times8\times11}{6^3(3\times4\times5)}
-\frac{5\times8\times11\times14}{6^4(3\times4\times5\times6)}
+\cdots
\end{align*}

整理得
\begin{align*}
S
&=\frac{5}{3}\left(\frac{1}{6}\right)
-\frac{5}{3}\left(\frac{8}{4}\right)\left(\frac{1}{6}\right)^2
+\frac{5}{3}\left(\frac{8}{4}\right)\left(\frac{11}{5}\right)\left(\frac{1}{6}\right)^3 \\
&\quad
-\frac{5}{3}\left(\frac{8}{4}\right)\left(\frac{11}{5}\right)\left(\frac{14}{6}\right)\left(\frac{1}{6}\right)^4
+\cdots
\end{align*}

进一步变形并提取常数,可得
\begin{align*}
\frac{1}{36}S
&=\frac{-\frac{1}{3}\left(-\frac{4}{3}\right)\left(-\frac{8}{3}\right)}{3!}
\left(\frac{1}{6}\right)^3
+\frac{-\frac{1}{3}\left(-\frac{4}{3}\right)\left(-\frac{8}{3}\right)\left(-\frac{11}{3}\right)}{4!}
\left(\frac{1}{6}\right)^4 \\
&\quad
+\frac{-\frac{1}{3}\left(-\frac{4}{3}\right)\left(-\frac{8}{3}\right)\left(-\frac{11}{3}\right)\left(-\frac{14}{3}\right)}{5!}
\left(\frac{1}{6}\right)^5
+\cdots
\end{align*}

在级数前补上缺失项,可构成完整的二项式展开:
\begin{align*}
\frac{1}{36}S
&=\left[1+\frac{-\frac{1}{3}}{1!}\left(\frac{1}{6}\right)
+\frac{-\frac{1}{3}\left(-\frac{4}{3}\right)}{2!}\left(\frac{1}{6}\right)^2
+\cdots\right] \\
&\quad
-\left[1+\frac{-\frac{1}{3}}{1!}\left(\frac{1}{6}\right)
+\frac{-\frac{1}{3}\left(-\frac{4}{3}\right)}{2!}\left(\frac{1}{6}\right)^2\right].
\end{align*}

由二项式定理,
\[
1+\frac{-\frac{1}{3}}{1!}x+\frac{-\frac{1}{3}\left(-\frac{4}{3}\right)}{2!}x^2+\cdots
=(1+x)^{-\frac{1}{3}},
\]
取 $x=\frac{1}{6}$,得
\begin{align*}
\frac{1}{36}S
&=\left(1+\frac{1}{2}\right)^{-\frac{1}{3}}
-\left(1+\frac{1}{6}-\frac{1}{36}\right).
\end{align*}

因此
\begin{align*}
S
&=36\left(\frac{3}{2}\right)^{\frac{1}{3}}-(36+6-1) \\
&=36\sqrt[3]{\frac{3}{2}}-41.
\end{align*}
\end{solution}

\question
2) 证明恒等式
\[
k \comb{n}{k} = n \comb{n-1}{k-1}
\]

\begin{solution}
从组合数的定义出发:
\begin{align*}
k \comb{n}{k} &= k \cdot \frac{n!}{k!(n-k)!} \\
&= \frac{n!}{(k-1)!(n-k)!} \\
&= n \cdot \frac{(n-1)!}{(k-1)!\,[ (n-1)-(k-1) ]!} \\
&= n \comb{n-1}{k-1}
\end{align*}
\end{solution}

\question 证明恒等式
\[\binom{n}{r} = \binom{n}{n-r}\]

\begin{solution}
\begin{align*}
\binom{n}{n-r} &= \frac{n!}{(n-r)!(n-(n-r))!} \\
&= \frac{n!}{(n-r)!r!} \\
&= \frac{n!}{r!(n-r)!} \\
&= \binom{n}{r}
\end{align*}
\end{solution}

\question 证明恒等式 (Pascal's Identity)
\[\binom{n}{r-1} + \binom{n}{r} = \binom{n+1}{r}\]

\begin{solution}
\begin{align*}
\binom{n}{r-1} + \binom{n}{r} &= \frac{n!}{(r-1)!(n-(r-1))!} + \frac{n!}{r!(n-r)!} \\
&= \frac{n!}{(r-1)!(n-r+1)!} + \frac{n!}{r!(n-r)!} \\
&= \frac{n! \cdot r}{r!(n-r+1)!} + \frac{n! \cdot (n-r+1)}{r!(n-r+1)!} \\
&= \frac{n! [r + (n-r+1)]}{r!(n-r+1)!} \\
&= \frac{n! (n+1)}{r!(n-r+1)!} \\
&= \frac{(n+1)!}{r!(n+1-r)!} \\
&= \binom{n+1}{r}
\end{align*}
\end{solution}

\question 证明恒等式
\[\binom{n}{m}\binom{m}{k} = \binom{n}{k}\binom{n-k}{m-k}, \quad n \in \mathbb{R}, k, m \in \mathbb{Z}\]

\begin{solution}
\begin{align*}
\binom{n}{m}\binom{m}{k} &= \frac{n!}{m!(n-m)!} \cdot \frac{m!}{k!(m-k)!} \\
&= \frac{n!}{k!(m-k)!(n-m)!} \\
&= \frac{n!}{k!(n-k)!} \cdot \frac{(n-k)!}{(m-k)!(n-m)!} \\
&= \binom{n}{k}\binom{n-k}{m-k}
\end{align*}
\end{solution}

\question 设 $n$ 是任意实数, $k$ 是任意整数, 均有:
\[\binom{n}{0} + \binom{n+1}{1} + \dots + \binom{n+k}{k} = \binom{n+k+1}{k}\]

\noindent 又有:
\[\binom{k}{k} + \binom{k+1}{k} + \dots + \binom{n}{k} = \binom{n+1}{k+1}, \quad n,k \in \mathbb{N}\]

\noindent 注: 考虑帕斯卡公式 $\binom{n}{k} = \binom{n-1}{k} + \binom{n-1}{k-1}$

\begin{solution}
(1) 反复对上式中最后一个二项式系数运用帕斯卡公式得:
\begin{align*}
\binom{n}{k} &= \binom{n-1}{k} + \binom{n-1}{k-1} \\
&= \binom{n-1}{k} + \binom{n-2}{k-1} + \binom{n-2}{k-2} \\
&\dots \\
\therefore \binom{n}{k} &= \binom{n-1}{k} + \binom{n-2}{k-1} + \dots + \binom{n-k-1}{0} + \binom{n-k-1}{-1}
\end{align*}
而 $\binom{n-k-1}{-1} = 0$, 用 $r+k+1$ 取代 $n$, 移项, 便得到该式.

\vspace{0.5cm}

(2) 反复对上式中第一个二项式系数运用帕斯卡公式得:
\begin{align*}
\binom{n}{k} &= \binom{n-1}{k} + \binom{n-1}{k-1} \\
&= \binom{n-2}{k} + \binom{n-2}{k-1} + \binom{n-1}{k-1} \\
&\dots \\
\therefore \binom{n}{k} &= \binom{k}{k} + \binom{k}{k-1} + \binom{k+1}{k-1} + \dots + \binom{n-2}{k-1} + \binom{n-1}{k-1}
\end{align*}
而 $\binom{k-1}{k} = 0$, 用 $r$ 取代 $k$, $k+1$ 取代 $n$, 便得到该式.
\end{solution}

\question 证明恒等式
\[\sum_{r=0}^{n} \binom{n}{r}^2 = \binom{2n}{n}, \quad n \ge 0\]

\begin{solution}
设 $S$ 是一个有 $2n$ 个元素的集合,$\binom{2n}{n}$ 为计数 $S$ 的 $n$ 子集的数目。

又把 $S$ 划分成 $A$ 和 $B$ 两个子集,每一个子集都有 $n$ 个元素。利用 $S$ 的这个划分去划分 $S$ 的 $n$ 子集。$S$ 的每一个 $n$ 子集包含 $k$ 个 $A$ 的元素和 $n-k$ 个 $B$ 的元素,$k$ 是介于 $0$ 和 $n$ 之间的任意整数。我们把 $S$ 的 $n$ 子集划分成 $n+1$ 个部分:
\[C_0, C_1, \dots, C_n\]
其中 $C_k$ 是由包含 $k$ 个 $A$ 的元素和 $n-k$ 个 $B$ 的元素组成的 $n$ 子集。根据加法原理,有
\[\binom{2n}{n} = |C_0| + |C_1| + \dots + |C_n|\]

$C_k$ 中的 $n$ 子集可以通过从 $A$ 中选择 $k$ 个元素,再从 $B$ 中选择 $n-k$ 个元素而得到。根据乘法原理,
\[|C_k| = \binom{n}{k} \binom{n}{n-k} = \binom{n}{k}^2, \quad k \ge 0.\]

把上式代入得:
\[\sum_{k=0}^{n} \binom{n}{k}^2 = \binom{n}{0}^2 + \binom{n}{1}^2 + \dots + \binom{n}{n}^2 = \binom{2n}{n}\]
\end{solution}

    \question 证明下列恒等式:
    \begin{parts}
    \part 
\[
\comb{n}{0} + \comb{n}{2} + \dots = \comb{n}{1} + \comb{n}{3} + \dots = 2^{\,n-1}
\]
\begin{solution}
使用二项式定理的交错和:
\[
\comb{n}{0} - \comb{n}{1} + \comb{n}{2} - \dots + (-1)^n \comb{n}{n} = 0
\]

由此可得:
\[
\comb{n}{0} + \comb{n}{2} + \dots = \comb{n}{1} + \comb{n}{3} + \dots
\]

又因为
\[
\comb{n}{0} + \comb{n}{1} + \comb{n}{2} + \dots + \comb{n}{n} = 2^n
\]

因此:
\[
\comb{n}{0} + \comb{n}{2} + \dots = \comb{n}{1} + \comb{n}{3} + \dots = \frac{1}{2}(2^n) = 2^{\,n-1}
\]
\end{solution}
    \part 
\[
\binom{n}{0} - \binom{n}{1} + \binom{n}{2} - \dots + (-1)^n \binom{n}{n} = 0, \quad n \ge 1
\]  
\begin{solution}  
由二项式定理  
\[(x+y)^n = \binom{n}{0}x^n + \binom{n}{1}x^{n-1}y + \binom{n}{2}x^{n-2}y^2 + \dots + \binom{n}{n}y^n\]  
令 $x=1, y=-1$,得到  
\begin{align*}
(1+(-1))^n &= \binom{n}{0} - \binom{n}{1} + \binom{n}{2} - \dots + (-1)^n \binom{n}{n} \\
0 &= \binom{n}{0} - \binom{n}{1} + \binom{n}{2} - \dots + (-1)^n \binom{n}{n}
\end{align*}  
因此恒等式成立。  
\end{solution}
    \part 
    \[
    1 \cdot \comb{n}{1} + 2 \cdot \comb{n}{2} + \dots + n \cdot \comb{n}{n}=n 2^{n-1}
    \]
    \begin{solution}
\begin{align*}
\sum_{k=1}^{n} k \binom{n}{k} &= 1\binom{n}{1} + 2\binom{n}{2} + \dots + n\binom{n}{n} \\
&= n \binom{n-1}{0} + n \binom{n-1}{1} + \dots + n \binom{n-1}{n-1} && \text{(利用 } k\binom{n}{k} = n \binom{n-1}{k-1} \text{)} \\
&= n \left[ \binom{n-1}{0} + \binom{n-1}{1} + \dots + \binom{n-1}{n-1} \right] \\
&= n \cdot 2^{n-1} && \text{(利用 } \sum_{k=0}^{m} \binom{m}{k} = 2^m \text{, } m=n-1) 
\end{align*}
\end{solution}
    \begin{solution}
考虑 $(1+x)^n = \sum_{k=0}^n \comb{n}{k} x^k$,两边对 $x$ 求导:
\[
n(1+x)^{n-1} = \sum_{k=1}^n k \comb{n}{k} x^{k-1}.
\]
令 $x=1$:
\[
n 2^{n-1} = \sum_{k=1}^n k \comb{n}{k}.
\]
    \end{solution}
    \part 
    \[
    \sum_{k=1}^n k^2 \comb{n}{k} = n(n+1) 2^{n-2}
    \]
\begin{solution}
由二项式定理:
\[(1+x)^n = \binom{n}{0} + \binom{n}{1}x + \binom{n}{2}x^2 + \dots + \binom{n}{n}x^n\]

两边求导,并乘以 $x$ 得:
\begin{align*}
x \frac{d}{dx}(1+x)^n &= x \sum_{k=1}^{n} k \binom{n}{k} x^{k-1} \\
x n (1+x)^{n-1} &= \sum_{k=1}^{n} k \binom{n}{k} x^k
\end{align*}

再次对两边求导:
\[
\frac{d}{dx} \left[ x n (1+x)^{n-1} \right] = \sum_{k=1}^{n} k^2 \binom{n}{k} x^{k-1}
\]

计算左边导数:
\[
n (1+x)^{n-1} + x n (n-1) (1+x)^{n-2} = \sum_{k=1}^{n} k^2 \binom{n}{k} x^{k-1}
\]

代入 $x=1$:
\begin{align*}
n \cdot 2^{n-1} + n(n-1) \cdot 2^{n-2} &= \sum_{k=1}^{n} k^2 \binom{n}{k} \\
n \left[ 2^{n-1} + (n-1) 2^{n-2} \right] &= \sum_{k=1}^{n} k^2 \binom{n}{k} \\
n \left[ (2 + n - 1) 2^{n-2} \right] &= \sum_{k=1}^{n} k^2 \binom{n}{k} \\
n(n+1) 2^{n-2} &= \sum_{k=1}^{n} k^2 \binom{n}{k}
\end{align*}
\end{solution}
    \part 
    \[
    \comb{n}{0} + \frac{1}{2} \comb{n}{1} + \frac{1}{3} \comb{n}{2} + \cdots + \frac{1}{n+1} \comb{n}{n} = \frac{1}{n+1} \left(2^{n+1} - 1\right)
    \]
    \begin{solution}
        \begin{align*}
        \sum_{k=0}^{n} \frac{\comb{n}{k}}{k+1} 
        &= \sum_{k=0}^{n} \frac{1}{k+1} \cdot \frac{n!}{k!(n-k)!} \\
        &= \sum_{k=0}^{n} \frac{n!}{(k+1)!(n-k)!} \\
        &= \frac{1}{n+1} \sum_{k=0}^{n} \frac{(n+1)!}{(k+1)!(n-k)!} \\
        &= \frac{1}{n+1} \sum_{k=0}^{n} \comb{n+1}{k+1} \\
        &= \frac{1}{n+1} \left( \comb{n+1}{1} + \comb{n+1}{2} + \dots + \comb{n+1}{n+1} \right) \\
        &= \frac{1}{n+1} (2^{n+1}-1)
        \end{align*}
        于是得证。
    \end{solution}
    \part 
\[\sum_{k \text{ even}} k \binom{n}{k} = n 2^{n-2}\]

\begin{solution}
由二项式定理:
\[(1+x)^n = \binom{n}{0} + \binom{n}{1}x + \binom{n}{2}x^2 + \dots + \binom{n}{n}x^n\]

两边求导:
\[
n(1+x)^{n-1} = \binom{n}{1} + 2\binom{n}{2}x + 3\binom{n}{3}x^2 + \dots + n\binom{n}{n}x^{n-1}
\]

同理,将 $x$ 替换为 $-x$:
\[
n(1-x)^{n-1} = \binom{n}{1} - 2\binom{n}{2}x + 3\binom{n}{3}x^2 - \dots + (-1)^{n-1} n \binom{n}{n}x^{n-1}
\]

两式相加:
\[
n(1+x)^{n-1} + n(1-x)^{n-1} = 2 \sum_{k \text{ even}} k \binom{n}{k} x^{k-1}
\]

代入 $x=1$:
\[
n 2^{n-1} + n 0^{n-1} = 2 \sum_{k \text{ even}} k \binom{n}{k}
\]

因此:
\[
\sum_{k \text{ even}} k \binom{n}{k} = n 2^{n-2}
\]
\end{solution}

    \end{parts}
\section*{Vandermonde's Identity}

\question 证明 Vandermonde 恒等式:
\[\binom{n+m}{k} = \sum_{i=0}^{k} \binom{n}{i}\binom{m}{k-i}\]

\begin{solution}
注意到:
\[(1+x)^n (1+x)^m = (1+x)^{n+m}\]

等号左边展开为:
\begin{align*}
(1+x)^n (1+x)^m &= \left(\sum_{i=0}^{n} \binom{n}{i}x^i\right)\left(\sum_{j=0}^{m} \binom{m}{j}x^j\right) \\
&= \sum_{k=0}^{m+n} \left(\sum_{i=0}^{k} \binom{n}{i}\binom{m}{k-i}\right)x^k
\end{align*}

等号右边根据定义展开为:
\[(1+x)^{n+m} = \sum_{k=0}^{n+m} \binom{n+m}{k} x^k\]

比较 $x^k$ 系数, 得到:
\[\binom{n+m}{k} = \sum_{i=0}^{k} \binom{n}{i}\binom{m}{k-i}\]
\end{solution}

    \question 证明  
    \[
    \left(\comb{n}{0}-\comb{n}{2}+\comb{n}{4}-\cdots\right)^{2}+\left(\comb{n}{1}-\comb{n}{3}+\comb{n}{5}-\cdots\right)^{2}=2^{n}
    \]
    \begin{solution}
        设 \(f(x)=(x+1)^n = \comb{n}{0} + \comb{n}{1} x + \comb{n}{2} x^2 + \cdots + \comb{n}{n} x^n\),取 \(x=i\) 得 
        \[
        f(i) = (i+1)^n = (\sqrt{2}\, e^{\frac{\pi i}{4}})^n = 2^{\frac{n}{2}} e^{\frac{\pi i}{4}}
        \]
        且有
        \[
        f(i) = (\comb{n}{0} - \comb{n}{2} + \comb{n}{4} - \cdots) + i (\comb{n}{1} - \comb{n}{3} + \comb{n}{5} - \cdots)
        \]
        比较实部与虚部得 
        \[
        \comb{n}{0} - \comb{n}{2} + \comb{n}{4} - \cdots = 2^{\frac{n}{2}} \cos\frac{\pi i}{4},\quad \comb{n}{1} - \comb{n}{3} + \comb{n}{5} - \cdots = 2^{\frac{n}{2}} \sin\frac{\pi i}{4}
        \]
        因此得证
        \[
        (\comb{n}{0} - \comb{n}{2} + \cdots)^2 + (\comb{n}{1} - \comb{n}{3} + \cdots)^2 
        = 2^n\cos^2\left(\frac{\pi i}{4}+\sin^2\frac{\pi i}{4}\right) = 2^n
        \]
    \end{solution}

    \question 若将 $(x+1)^{2000}$ 展开,问有多少个系数是奇数?
    \begin{solution}
        一般地,若 $(x+1)^n$ 展开式中奇数系数的个数为 $2^{a}$,其中 $a$ 是 $n$ 的二进制表示中 1 的个数,将 $2000$ 转换为二进制:
        \[
        2000 = 11111010000_2
        \]
        其中有 6 个 1,因此 $(x+1)^{2000}$ 中奇数系数的个数为
        \[
        2^6 = 64.
        \]
        又解:对 2 取模,有
        \begin{align*}
        (x+1)^{2000} 
        &= (x+1)^{1024}(x+1)^{512}(x+1)^{256}(x+1)^{128}(x+1)^{64}(x+1)^{16} \\
        &\equiv (x^{1024}+1)(x^{512}+1)\cdots(x^{16}+1) \pmod{2},
        \end{align*}
        展开后共有 $2^6=64$ 项系数为奇数。
    \end{solution}

    \question 设 \( n \) 为大于或等于 1 的正整数,考虑数列 \( a_n = \left(1 + \dfrac{1}{n} \right)^n\),试用二项式定理证明对所有 \( n \) 皆满足  
    \[
    a_n < e
    \]
    \begin{solution}
        设 \( a_n = \left(1 + \dfrac{1}{n} \right)^n \),考虑对 \( a_n \) 进行二项式展开
        \[
        \left(1 + \frac{1}{n} \right)^n = \sum_{k=0}^{n} \binom{n}{k} \cdot \left(\frac{1}{n}\right)^k
        = \sum_{k=0}^{n} \frac{n(n-1)\cdots(n-k+1)}{k! \cdot n^k}
        \]
        注意到
        \[
        \binom{n}{k} \cdot \left(\frac{1}{n}\right)^k \le \frac{1}{k!}
        \Rightarrow a_n=\left(1 + \frac{1}{n} \right)^n < \sum_{k=0}^{n} \frac{1}{k!} < \sum_{k=0}^{\infty} \frac{1}{k!} = e 
        \]
    \end{solution}

    \question 定义 \( a_n \) 为 \( (3 - \sqrt{x})^n \) 展开式中 \( x \) 项的系数,其中 \( n \) 是正整数。求极限  
    \[
    \lim_{n \to \infty} \left( \frac{3^2}{a_2} + \frac{3^3}{a_3} + \cdots + \frac{3^n}{a_n} \right)
    \]
    \begin{solution}
        观察到 \( (3 - \sqrt{x})^n \) 的二项展开式为:
        \[
        (3 - \sqrt{x})^n = \sum_{k=0}^n \binom{n}{k} 3^{n - k} (-\sqrt{x})^k
        \]
        \( x \) 项的系数对应 \( k = 2 \):
        \[
        a_n = \binom{n}{2} 3^{n - 2}
        \]
        所以
        \[
        \frac{3^n}{a_n} = \frac{3^n}{\binom{n}{2} 3^{n - 2}} = \frac{3^2}{\binom{n}{2}} = \frac{9}{\binom{n}{2}}
        \]
        原式变为
        \[
        9 \sum_{n=2}^{\infty} \frac{1}{\binom{n}{2}}=
        18\sum_{n=2}^{\infty} \frac{2}{n(n-1)} = 18\sum_{n=2}^{\infty} \left( \frac{1}{n-1} - \frac{1}{n} \right)
        \]
        这是一个裂项求和,且
        \[
        \sum_{n=2}^{\infty} \left( \frac{1}{n-1} - \frac{1}{n} \right) =  1 - \frac{1}{2} + \frac{1}{2} - \frac{1}{3} + \cdots = 1
        \]
        因此原极限为 $18$。
    \end{solution}

    \question 用 $|S|$ 表示集合 $S$ 中元素的个数。已知集合
    \[
    S = \left\{ \frac{1}{2}, \frac{1}{3}, \frac{1}{4}, \ldots, \frac{1}{113} \right\},\quad
    T = \left\{ A \subseteq S \,\middle|\, |A| = 2n,\, n \in \mathbb{N} \right\},
    \]
    试回答下列问题:
    \begin{parts}
    \part $|T| = \;?$
    \begin{solution}
        集合 $S$ 中的元素个数为 $|S| = 112$,所以:
        \[
        |T| = \comb{112}{2} + \comb{112}{4} + \cdots + \comb{112}{112}
        \]
        由二项式定理,
        \begin{align*}
        (1 + 1)^{112} &= \comb{112}{0} + \comb{112}{1} + \cdots + \comb{112}{112} = 2^{112} \\
        (1 - 1)^{112} &= \comb{112}{0} - \comb{112}{1} + \comb{112}{2} - \cdots + (-1)^{112} \cdot \comb{112}{112} = 0
        \end{align*}
        两式相加得
        \[
        2^{112} = 2 \left( \comb{112}{0} + \comb{112}{2} + \comb{112}{4} + \cdots + \comb{112}{112} \right)
        \Rightarrow \sum_{k=0}^{56} \comb{112}{2k} = 2^{111}
        \]
        于是
        \[
        |T| = 2^{111} - 1
        \]
    \end{solution}
    \part 对任意 $A_i \in T$,将 $A_i$ 中所有元素相乘的乘积记为 $m_i$,再将所有的 $m_i$ 相加,其和为 $M$,求 $M$ 的值。
    \end{parts}
    \begin{solution}
        设
        \[
        f(x) = \left(x + \frac{1}{2}\right)\left(x + \frac{1}{3}\right)\cdots\left(x + \frac{1}{113}\right)
        \]
        展开 $f(x)$ 得
        \begin{itemize}
            \item $x^{110}$ 的系数为所有 $|A_i| = 2$ 的子集乘积之和;
            \item $x^{108}$ 的系数为所有 $|A_i| = 4$ 的子集乘积之和;
            \item $\cdots$;
            \item 常数项为所有 $|A_i| = 112$ 的子集乘积之和。
        \end{itemize}
        故所有偶数个元素子集的乘积之和为
        \[
        M = \frac{1}{2} \left( f(1) + f(-1) \right) - 1
        \]
        而
        \[
        f(1) = \left(1 + \frac{1}{2}\right)\left(1 + \frac{1}{3}\right)\cdots\left(1 + \frac{1}{113}\right)
        = \frac{3}{2} \cdot \frac{4}{3} \cdot \frac{5}{4} \cdots \frac{114}{113} = 57
        \]
        \[
        f(-1) = \left(-1 + \frac{1}{2}\right)\left(-1 + \frac{1}{3}\right)\cdots\left(-1 + \frac{1}{113}\right)
        = \left(-\frac{1}{2}\right)\left(-\frac{2}{3}\right)\cdots\left(-\frac{112}{113}\right) = \frac{1}{113}
        \]
        因此
        \[
        M = \frac{1}{2} \left( 57 + \frac{1}{113} \right) - 1 = \frac{3108}{113}
        \]
    \end{solution}

    \question
函数 $f$ 由实常数 $a,b,c$ 定义为
\[
f(x)=(a+bx+cx^2)(1-x)^{-3},\quad x\in\mathbb{R},\ |x|<1.
\]

\textbf{a)} 证明
\[
f(x)=a+(3a+b)x+\frac{1}{2}\sum_{n=2}^{\infty}\left[a(n+1)(n+2)+bn(n+1)+cn(n-1)\right]x^n.
\]

\textbf{b)} 利用第(a)问的结果求
\[
\sum_{n=1}^{\infty}\frac{n^2}{2^n}.
\]

\begin{solution}
\textbf{(a)}  
先写出二项式展开式
\[
(1-x)^{-3}
= \sum_{n=0}^{\infty}\frac{1}{2}(n+1)(n+2)x^n.
\]

于是
\begin{align*}
f(x)
&=(a+bx+cx^2)(1-x)^{-3} \\
&=(a+bx+cx^2)\sum_{n=0}^{\infty}\frac{1}{2}(n+1)(n+2)x^n \\
&=\frac{1}{2}a\sum_{n=0}^{\infty}(n+1)(n+2)x^n
 +\frac{1}{2}b\sum_{n=0}^{\infty}(n+1)(n+2)x^{n+1} \\
&\quad +\frac{1}{2}c\sum_{n=0}^{\infty}(n+1)(n+2)x^{n+2}.
\end{align*}

将前两项单独取出并把求和指标统一为从 $n=2$ 开始,可得
\[
f(x)
= a+(3a+b)x
 +\frac{1}{2}\sum_{n=2}^{\infty}\left[(n+1)(n+2)
 +n(n+1)
 +(n-1)n\right]x^n.
\]

整理得
\[
f(x)
= a+(3a+b)x
 +\frac{1}{2}\sum_{n=2}^{\infty}
 \left[a(n+1)(n+2)+bn(n+1)+cn(n-1)\right]x^n.
\]

\textbf{(b)}  
展开求和中 $x^n$ 的系数:
\[
\frac{1}{2}\left[(a+b+c)n^2+(3a+b-c)n+2a\right].
\]

要求该系数等于 $n^2$,比较系数得
\[
\frac{1}{2}(a+b+c)=1,\quad
\frac{1}{2}(3a+b-c)=0,\quad
a=0.
\]

解得
\[
a=0,\quad b=1,\quad c=1.
\]

于是
\[
f(x)=(x+x^2)(1-x)^{-3}
= x+\sum_{n=2}^{\infty}n^2x^n.
\]

令 $x=\frac{1}{2}$,有
\[
f\left(\frac{1}{2}\right)
=\left(\frac{1}{2}+\frac{1}{4}\right)\left(1-\frac{1}{2}\right)^{-3}
=\frac{3}{4}\cdot 8=6.
\]

另一方面,
\[
f\left(\frac{1}{2}\right)
=\frac{1}{2}+\sum_{n=2}^{\infty}\frac{n^2}{2^n}.
\]

因此
\[
\sum_{n=2}^{\infty}\frac{n^2}{2^n}
=6-\frac{1}{2}=\frac{11}{2}.
\]

再加上 $n=1$ 的项 $\frac{1}{2}$,得到
\[
\sum_{n=1}^{\infty}\frac{n^2}{2^n}=6.
\]
\end{solution}

\question
计算:
\[ \cos \alpha + \comb{n}{1} \cos 2\alpha + \comb{n}{2} \cos 3\alpha + \dots + \comb{n}{n-1} \cos n\alpha + \cos (n + 1)\alpha \]
以及
\[ \sin \alpha + \comb{n}{1} \sin 2\alpha + \comb{n}{2} \sin 3\alpha + \dots + \comb{n}{n-1} \sin n\alpha + \sin (n + 1)\alpha \]

\begin{solution}
    我们需要计算下列复数和的实部与虚部系数:
    \[ (\cos \alpha + i \sin \alpha) + \comb{n}{1}(\cos 2\alpha + i \sin 2\alpha) + \comb{n}{2}(\cos 3\alpha + i \sin 3\alpha) + \dots + [\cos (n + 1)\alpha + i \sin (n + 1)\alpha] \]
    令 $x = \cos \alpha + i \sin \alpha$。利用棣莫弗公式(De Moivre's formula)和二项式定理,我们可以将该和式转化为以下形式:
    \begin{align*}
    x + \comb{n}{1} x^2 + \comb{n}{2} x^3 + \dots + x^{n+1} &= x(1 + x)^n \\
    &= (\cos \alpha + i \sin \alpha)(1 + \cos \alpha + i \sin \alpha)^n \\
    &= (\cos \alpha + i \sin \alpha)\left(2\cos^2 \frac{\alpha}{2} + 2i \sin \frac{\alpha}{2} \cos \frac{\alpha}{2}\right)^n \\
    &= (\cos \alpha + i \sin \alpha) \left[ 2\cos \frac{\alpha}{2} \left(\cos \frac{\alpha}{2} + i \sin \frac{\alpha}{2}\right) \right]^n \\
    &= 2^n \cos^n \frac{\alpha}{2} (\cos \alpha + i \sin \alpha)\left(\cos \frac{n\alpha}{2} + i \sin \frac{n\alpha}{2}\right) \\
    &= 2^n \cos^n \frac{\alpha}{2} \left[ \cos \left(\alpha + \frac{n\alpha}{2}\right) + i \sin \left(\alpha + \frac{n\alpha}{2}\right) \right] \\
    &= 2^n \cos^n \frac{\alpha}{2} \left(\cos \frac{n+2}{2} \alpha + i \sin \frac{n+2}{2} \alpha\right)
    \end{align*}
    由此可得:
    \begin{align*}
    \cos \alpha + \comb{n}{1} \cos 2\alpha + \comb{n}{2} \cos 3\alpha + \dots + \cos (n + 1)\alpha &= 2^n \cos^n \frac{\alpha}{2} \cos \frac{n+2}{2} \alpha \\
    \sin \alpha + \comb{n}{1} \sin 2\alpha + \comb{n}{2} \sin 3\alpha + \dots + \sin (n + 1)\alpha &= 2^n \cos^n \frac{\alpha}{2} \sin \frac{n+2}{2} \alpha
    \end{align*}
\end{solution}

    \question
(a) 证明对于任意正整数 $n$ 有
\begin{align*}
\sin n\theta &= \binom{n}{1}\sin\theta \cos^{n-1}\theta - \binom{n}{3}\sin^{3}\theta \cos^{n-3}\theta + \binom{n}{5}\sin^{5}\theta \cos^{n-5}\theta - \dots, \\
\cos n\theta &= \cos^{n}\theta - \binom{n}{2}\sin^{2}\theta \cos^{n-2}\theta + \binom{n}{4}\sin^{4}\theta \cos^{n-4}\theta - \dots
\end{align*}

(b) 证明,对于区间 $[-1,1]$ 上的所有 $x$ 和任意正整数 $n$,函数
\[
T_n(x) = \cos(n \cos^{-1} x)
\]
是 $x$ 的次数为 $n$ 的多项式,且首项系数为 $2^{\,n-1}$。

\begin{solution}
(a) 由德·摩根定理 (De Moivre's Theorem):
\[
(\cos\theta + i \sin\theta)^n = \cos n\theta + i \sin n\theta,
\]
对左边使用二项式定理展开,并分别比较实部和虚部,即得所述公式。

(b) 取 $\theta = \cos^{-1} x$,则 $\cos\theta = x$,$\sin\theta = \sqrt{1-x^2}$,应用(a)可得
\[
\cos(n \cos^{-1} x) = x^n - \binom{n}{2}(1-x^2)x^{\,n-2} + \binom{n}{4}(1-x^2)^2 x^{\,n-4} - \dots
\]
显然 $T_n(x)$ 是 $x$ 的次数为 $n$ 的多项式。其首项系数为
\[
1 + \binom{n}{2} + \binom{n}{4} + \dots + \binom{n}{2k}.
\]

由于
\[
2^n = (1+1)^n = 1 + \binom{n}{1} + \binom{n}{2} + \dots + \binom{n}{n}, \quad
0 = (1-1)^n = 1 - \binom{n}{1} + \binom{n}{2} - \dots + (-1)^n \binom{n}{n},
\]
两式相加,得到
\[
1 + \binom{n}{2} + \binom{n}{4} + \dots + \binom{n}{2k} = 2^{\,n-1}.
\]
因此首项系数为 $2^{\,n-1}$。
\end{solution}

\question 设函数
\[
f(x)=\frac{1}{\sqrt{1-x}},\quad -1<x<1。
\]

\noindent a) 通过操作二项展开的一般项,证明
\[
f(x)=\sum_{r=0}^{\infty}\binom{2r}{r}\left(\frac{1}{4}x\right)^r。
\]

\noindent b) 求 $\frac{1}{\sqrt{16-x^2}}$ 的类似展开式,并进一步证明
\[
\frac{x}{(16-x^2)^{\frac{3}{2}}}
=\sum_{r=1}^{\infty}\binom{2r}{r}\left(\frac{1}{16}\right)^r\left(\frac{1}{8}x\right)^{2r-1}。
\]

\noindent c) 求下列级数的精确值
\[
\sum_{r=1}^{\infty}\binom{2r}{r}\left(\frac{5}{32}\right)^r\left(\frac{4}{25}\right)^r。
\]

\begin{solution}
\textbf{a)}

由二项定理,
\[
(1-x)^{-\frac12}
=1+\frac{-\frac12}{1!}(-x)
+\frac{-\frac12\left(-\frac32\right)}{2!}(-x)^2
+\cdots
\]

一般项可写为
\[
(1-x)^{-\frac12}
=1+\sum_{r=1}^{\infty}
\frac{\left(-\frac12\right)\left(-\frac32\right)\cdots\left(-\frac{2r-1}{2}\right)}{r!}(-x)^r。
\]

整理符号与系数得
\[
(1-x)^{-\frac12}
=1+\sum_{r=1}^{\infty}
\frac{1\cdot3\cdot5\cdots(2r-1)}{r!2^r}x^r。
\]

利用
\[
1\cdot3\cdot5\cdots(2r-1)
=\frac{(2r)!}{2^r r!},
\]
可得
\[
(1-x)^{-\frac12}
=\sum_{r=0}^{\infty}\frac{(2r)!}{r!r!}\left(\frac{x}{4}\right)^r
=\sum_{r=0}^{\infty}\binom{2r}{r}\left(\frac{x}{4}\right)^r。
\]

\textbf{b)}

\[
(16-x^2)^{-\frac12}
=16^{-\frac12}\left(1-\frac{x^2}{16}\right)^{-\frac12}
=\frac14\sum_{r=0}^{\infty}\binom{2r}{r}\left(\frac{x^2}{64}\right)^r
=\frac14\sum_{r=0}^{\infty}\binom{2r}{r}\left(\frac{x}{8}\right)^{2r}。
\]

对两边求导,
\[
\frac{d}{dx}(16-x^2)^{-\frac12}
=\sum_{r=1}^{\infty}\frac14\binom{2r}{r}(2r)\left(\frac{x}{8}\right)^{2r-1}\frac18。
\]

左边化简得
\[
\frac{x}{(16-x^2)^{\frac32}}
=\sum_{r=1}^{\infty}\binom{2r}{r}\frac{r}{16}\left(\frac{x}{8}\right)^{2r-1},
\]
即证所需结果。

\textbf{c)}

原级数化简为
\[
\sum_{r=1}^{\infty}\binom{2r}{r}\left(\frac{1}{16}\right)^r\left(\frac{2}{5}\right)^{2r-1}。
\]

由 b) 的结果,令
\[
\frac{x}{8}=\frac{2}{5}\Rightarrow x=\frac{16}{5},
\]
则
\[
\sum_{r=1}^{\infty}\binom{2r}{r}\left(\frac{1}{16}\right)^r\left(\frac{2}{5}\right)^{2r-1}
=\frac{\frac{16}{5}}{\left(16-\left(\frac{16}{5}\right)^2\right)^{\frac32}}。
\]

计算得
\[
16-\left(\frac{16}{5}\right)^2
=16\left(1-\frac{16}{25}\right)
=16\cdot\frac{9}{25},
\]
因此
\[
\frac{\frac{16}{5}}{\left(16-\left(\frac{16}{5}\right)^2\right)^{\frac32}}
=\frac{16}{5}\cdot\frac{25}{64\cdot27}
=\frac{25}{108}。
\]

\end{solution}


\end{questions}

\pagebreak
\begin{center}
  {\fontsize{30pt}{26pt}\selectfont
    \hypertarget{数学归纳法}{数学归纳法} \label{数学归纳法}
  }
\end{center}
\separator
\vspace{1pt}

\begin{questions}
    \question 证明对任意正整数 $n$, 
\[
(n + 1)(n + 2)(n + 3) \cdots (2n - 1)(2n) = 2^n \cdot 1 \cdot 3 \cdot 5 \cdots (2n - 1)
\]

\begin{solution}
    定义命题:
    \[
    P_n: (n + 1)(n + 2) \cdots (2n) = 2^n \cdot 1 \cdot 3 \cdots (2n - 1)
    \]
    其中 $n \in \mathbb{Z}^{+}$。观察当 $n=1$ 时,
    \[
    \text{左边} = 1+1=2, \quad \text{右边} = 2^1 \cdot 1 = 2
    \]
    左边等于右边,即 $P_1$ 成立。现假设 $P_k$ 成立,即当 $n=k$ 时,
    \[
    (k + 1)(k + 2) \cdots (2k) = 2^k \cdot 1 \cdot 3 \cdots (2k - 1)
    \]
    观察当 $n=k+1$ 时,左边为:
    \begin{align*}
    (k + 2)(k + 3) \cdots (2k)(2k + 1)(2k + 2) &= \frac{(k + 1)(k + 2) \cdots (2k) \cdot (2k + 1) \cdot 2(k + 1)}{k + 1} \\
    &= \frac{2^k \cdot 1 \cdot 3 \cdots (2k - 1) \cdot (2k + 1) \cdot 2(k + 1)}{k + 1} \\
    &= 2^{k + 1} \cdot 1 \cdot 3 \cdots (2k - 1)(2k + 1)
    \end{align*}
    此时左边等于 $P_{k+1}$ 的右边,因此 $P_{k+1}$ 亦成立。
    
    即 $P_k \implies P_{k+1}$。由数学归纳法,$P_n$ 对任意正整数 $n$ 都成立。
\end{solution}

\question 证明对任意正整数 $n \ge p$,
\[
\sum_{i=1}^{n-p+1} \frac{1}{i(i+1)\cdots(i+p-1)} = \frac{1}{p-1} \left[ \frac{1}{(p-1)!} - \frac{1}{(n-p+2)(n-p+3)\cdots n} \right]
\]

\begin{solution}
    定义命题 $P_n$ 为上述等式。其中 $n, p \in \mathbb{Z}^{+}$ 且 $p > 1$。
    
    第一步:当 $n = p$ 时,
    左边 $= \frac{1}{1 \cdot 2 \cdots p} = \frac{1}{p!}$。
    右边 $= \frac{1}{p-1} \left[ \frac{1}{(p-1)!} - \frac{1}{2 \cdot 3 \cdots p} \right] = \frac{1}{p-1} \cdot \frac{p-1}{p!} = \frac{1}{p!}$。
    左边等于右边,即 $P_p$ 成立。

    第二步:假设当 $n = k$ ($k \ge p$) 时结论成立,即:
    \[
    \sum_{i=1}^{k-p+1} \frac{1}{i \cdots (i+p-1)} = \frac{1}{p-1} \left[ \frac{1}{(p-1)!} - \frac{1}{(k-p+2) \cdots k} \right]
    \]

    第三步:考虑当 $n = k + 1$ 时的情况,左边为:
    \begin{align*}
    \sum_{i=1}^{k-p+2} \frac{1}{i \cdots (i+p-1)} &= \sum_{i=1}^{k-p+1} \frac{1}{i \cdots (i+p-1)} + \frac{1}{(k-p+2) \cdots (k+1)} \\
    &= \frac{1}{p-1} \left[ \frac{1}{(p-1)!} - \frac{1}{(k-p+2) \cdots k} \right] + \frac{1}{(k-p+2) \cdots (k+1)} \\
    &= \frac{1}{p-1} \left[ \frac{1}{(p-1)!} - \left( \frac{1}{(k-p+2) \cdots k} - \frac{p-1}{(k-p+2) \cdots (k+1)} \right) \right]
    \end{align*}
    对括号内进行通分:
    \[
    \frac{(k+1) - (p-1)}{(k-p+2) \cdots (k+1)} = \frac{k-p+2}{(k-p+2)(k-p+3) \cdots (k+1)} = \frac{1}{(k-p+3) \cdots (k+1)}
    \]
    由此可得:
    \[
    \text{左边} = \frac{1}{p-1} \left[ \frac{1}{(p-1)!} - \frac{1}{(k-p+3) \cdots (k+1)} \right]
    \]
    此时左边等于 $P_{k+1}$ 的右边,因此 $P_{k+1}$ 亦成立。
    
    即 $P_k \implies P_{k+1}$。由数学归纳法,$P_n$ 对任意正整数 $n \ge p$ 均成立。
\end{solution}

\question
证明:对于任意整数 $n > 1$,有
\[ \frac{1}{2} \cdot \frac{3}{4} \cdot \frac{5}{6} \cdots \frac{2n - 1}{2n} < \frac{1}{\sqrt{3n + 1}} \]
据此,试证
\[
\frac{1}{2} \cdot \frac{3}{4} \cdot \frac{5}{6} \cdots \frac{99}{100} < \frac{1}{12}
\]
\begin{solution}
    我们使用数学归纳法进行证明。

    第一步:当 $n = 1$ 时,
    \[ \text{左边} = \frac{1}{2}, \quad \text{右边} = \frac{1}{\sqrt{3(1) + 1}} = \frac{1}{2} \]
    此时左边等于右边(注:题目要求证明 $n>1$ 时的严格不等式,此处作为归纳基础)。

    第二步:假设当 $n = k$ 时不等式(或等式)成立,即:
    \[ \frac{1}{2} \cdot \frac{3}{4} \cdots \frac{2k - 1}{2k} \le \frac{1}{\sqrt{3k + 1}} \]

    第三步:考虑当 $n = k + 1$ 时,左边乘积项变为:
    \[ \text{左边} = \left( \frac{1}{2} \cdots \frac{2k - 1}{2k} \right) \cdot \frac{2k + 1}{2k + 2} \le \frac{1}{\sqrt{3k + 1}} \cdot \frac{2k + 1}{2k + 2} \]
    为了证明结论成立,我们需要证明 $\frac{2k + 1}{(2k + 2)\sqrt{3k + 1}} < \frac{1}{\sqrt{3(k + 1) + 1}}$。
    我们对不等式左边项的平方进行分析:
    \begin{align*}
    \left[ \frac{2k + 1}{(2k + 2)\sqrt{3k + 1}} \right]^2 &= \frac{(2k + 1)^2}{(4k^2 + 8k + 4)(3k + 1)} \\
    &= \frac{(2k + 1)^2}{12k^3 + 28k^2 + 20k + 4} \\
    &= \frac{(2k + 1)^2}{(12k^3 + 28k^2 + 19k + 4) + k}
    \end{align*}
    注意到上式分母中括号内的部分恰好等于 $(2k + 1)^2(3k + 4)$。因此:
    \[ \frac{(2k + 1)^2}{(2k + 1)^2(3k + 4) + k} < \frac{(2k + 1)^2}{(2k + 1)^2(3k + 4)} = \frac{1}{3k + 4} \]
    这表明:
    \[ \frac{2k + 1}{(2k + 2)\sqrt{3k + 1}} < \frac{1}{\sqrt{3k + 4}} \]
    即 $P_{k+1}$ 成立。

    综上所述,由数学归纳法可知,原不等式对所有 $n > 1$ 均成立。

    据此,令$n=50$,得
\end{solution}

\question 已知
\[
f(n)=3^{2n+4}-2^{2n},\quad n\in\mathbb{N},
\]
用数学归纳法证明 $f(n)$ 能被 $5$ 整除。

\begin{solution}
先验证基础情况。

当 $n=1$ 时,
\begin{align*}
f(1) &= 3^6-2^2 \\
&=729-4 \\
&=725,
\end{align*}
显然 $725$ 能被 $5$ 整除,因此结论对 $n=1$ 成立。

假设当 $n=k$ 时结论成立,即存在 $m\in\mathbb{N}$,使得
\[
f(k)=3^{2k+4}-2^{2k}=5m.
\]

下面证明结论对 $n=k+1$ 也成立。

\begin{align*}
f(k+1)-f(k)
&=\left(3^{2(k+1)+4}-2^{2(k+1)}\right)-\left(3^{2k+4}-2^{2k}\right)\\
&=3^{2k+6}-2^{2k+2}-3^{2k+4}+2^{2k}\\
&=3^{2k+4}(3^2-1)-2^{2k}(2^2-1)\\
&=8\cdot3^{2k+4}-3\cdot2^{2k}.
\end{align*}

由此得
\begin{align*}
f(k+1)
&=f(k)+8\cdot3^{2k+4}-3\cdot2^{2k}\\
&=5m+8\cdot3^{2k+4}-3(3^{2k+4}-5m)\\
&=5m+8\cdot3^{2k+4}-3\cdot3^{2k+4}+15m\\
&=20m+5\cdot3^{2k+4}\\
&=5(4m+3^{2k+4}).
\end{align*}

因此 $f(k+1)$ 也能被 $5$ 整除。

由数学归纳法可知,对所有 $n\in\mathbb{N}$,
\[
f(n)=3^{2n+4}-2^{2n}
\]
均能被 $5$ 整除。
\end{solution}

    \question 证明对所有正整数 $n$,都有
\[
\frac{1}{\sqrt{1}} + \frac{1}{\sqrt{2}} + \dots + \frac{1}{\sqrt{n}} \ge \sqrt{n}.
\]

\begin{solution}
\textbf{用数学归纳法证明:}

\textit{当 $n=1$ 时:}
\begin{align*}
\text{L.H.S.} &= \frac{1}{\sqrt{1}} = 1,\\
\text{R.H.S.} &= \sqrt{1} = 1.
\end{align*}
所以当 $n=1$ 时命题成立。

假设当 $n=k$ 时命题成立,即
\[
\frac{1}{\sqrt{1}} + \frac{1}{\sqrt{2}} + \dots + \frac{1}{\sqrt{k}} \ge \sqrt{k}.
\]

\textit{当 $n=k+1$ 时:}
\begin{align*}
\text{L.H.S.} &= \frac{1}{\sqrt{1}} + \frac{1}{\sqrt{2}} + \dots + \frac{1}{\sqrt{k}} + \frac{1}{\sqrt{k+1}} \\
&\ge \sqrt{k} + \frac{1}{\sqrt{k+1}}.
\end{align*}

只需证明
\[
\sqrt{k} + \frac{1}{\sqrt{k+1}} \ge \sqrt{k+1}.
\]

两边同减 $\sqrt{k+1}$:
\begin{align*}
\sqrt{k} + \frac{1}{\sqrt{k+1}} - \sqrt{k+1} 
&= \frac{\sqrt{k(k+1)} + 1 - (k+1)}{\sqrt{k+1}} \\
&= \frac{\sqrt{k(k+1)} - k}{\sqrt{k+1}} \\
&= \frac{k(k+1) - k^2}{(\sqrt{k(k+1)} + k)\sqrt{k+1}} \\
&= \frac{k}{\sqrt{k+1}(\sqrt{k^2+k} + k)} > 0 \quad (\text{因为 } k>0)
\end{align*}

因此
\[
\sqrt{k} + \frac{1}{\sqrt{k+1}} > \sqrt{k+1},
\]
即当 $n=k+1$ 时命题成立。

由数学归纳法,命题对所有正整数 $n$ 成立。
\end{solution}

\question 用数学归纳法证明:若 $n \in \mathbb{N}, n \geq 3$,则
\[
n^{n+1} > (n+1)^n,
\]
并由此推导出若 $n \in \mathbb{N}, n \geq 3$,则
\[
\sqrt{n} > \sqrt[n+1]{n+1}.
\]

\begin{solution}
首先进行归纳证明。

\textbf{基础情况:} $n=3$  
\[
3^{3+1} = 81 > 64 = 4^3,
\]  
因此结果对 $n=3$ 成立。

\textbf{归纳假设:} 假设对于 $n=k \ge 3$,不等式成立:
\[
k^{k+1} > (k+1)^k.
\]

\textbf{归纳步骤:} 需要证明若归纳假设成立,则
\[
(k+1)^{(k+1)+1} = (k+1)^{k+2} > (k+2)^{k+1}.
\]

根据归纳假设:
\[
k^{k+1} > (k+1)^k \implies k^{k+1} (k+1)^2 > (k+1)^k (k+2)^{k+1} ?
\]

考虑简单的方法:对任意 $k \ge 3$,有
\[
(k+1)^2 > k(k+2) \implies k^2 + 2k + 1 > k^2 + 2k,
\]  
这显然成立,因此归纳步骤成立。

因此,如果不等式对 $n=k$ 成立,则对 $n=k+1$ 也成立。结合基础情况 $n=3$,则对所有 $n \ge 3$,不等式成立:
\[
n^{n+1} > (n+1)^n.
\]

\textbf{推导 $\sqrt{n} > \sqrt[n+1]{n+1}$:}

两边同时取 $(n(n+1))$ 次根:
\begin{align*}
\left(\frac{n^{n+1}}{(n+1)^{n+1}}\right)^{\frac{1}{n(n+1)}} &> \left(\frac{(n+1)^n}{(n+1)^{n+1}}\right)^{\frac{1}{n(n+1)}} \\
\frac{n^{\frac{1}{n}}}{(n+1)^{\frac{1}{n+1}}} &> \frac{1}{(n+1)^{\frac{1}{n+1}}} \\
n^{\frac{1}{n}} &> (n+1)^{\frac{1}{n+1}} \\
\sqrt{n} &> \sqrt[n+1]{n+1}.
\end{align*}

由此完成证明。
\end{solution}

    \question 设 $x_0=5$,且 $x_{n+1}=x_n+\dfrac{1}{x_n}$。证明对一切 $n\ge1$ 都有
\[
2n < x_n^2 - 25 < \frac{47n}{23}.
\]

\begin{solution}
先验算 $n=1$ 的情形。$x_1=5+\dfrac{1}{5}=\dfrac{26}{5}$,所以
\[
x_1^2-25=\left(\frac{26}{5}\right)^2-25=\frac{51}{25}=2.04,
\]
显然 $2<\dfrac{51}{25}<\dfrac{47}{23}\approx2.04348$,因此不等式对 $n=1$ 成立。

下面用数学归纳法。假设不等式对某个正整数 $k$ 成立,即
\[
2k < x_k^2-25 < \frac{47k}{23}.
\]
由递推关系得
\[
x_{k+1}^2-25=\Big(x_k+\frac{1}{x_k}\Big)^2-25=(x_k^2-25)+2+\frac{1}{x_k^2}.
\]

首先给出下界:由于 $\dfrac{1}{x_k^2}>0$,有
\[
x_{k+1}^2-25>(x_k^2-25)+2>2k+2=2(k+1).
\]

再给出上界:由归纳假设 $x_k^2-25<\dfrac{47k}{23}$,所以
\[
x_{k+1}^2-25<( \tfrac{47k}{23} )+2+\frac{1}{x_k^2}.
\]
注意到对 $k\ge1$ 有 $x_k\ge x_1=\tfrac{26}{5}>5$,因此 $x_k^2>25>23$,从而
\[
\frac{1}{x_k^2}<\frac{1}{23}.
\]
因此
\[
x_{k+1}^2-25<\frac{47k}{23}+2+\frac{1}{23}=\frac{47k}{23}+\frac{47}{23}=\frac{47(k+1)}{23}.
\]

综上,若不等式对 $n=k$ 成立,则对 $n=k+1$ 也成立。结合基例 $n=1$,由归纳法可知对所有 $n\ge1$ 都有
\[
2n < x_n^2-25 < \frac{47n}{23}.
\]
证毕。
\end{solution}

\question 设数列 $\{a_n\}$ 定义为 $a_1 = 1$, $a_2 = 1$, 且 $a_{m+1} = a_m + a_{m-1}$ 对所有 $m \ge 2$。证明
\[
a_n = \frac{1}{\sqrt{5}}\left[\left(\frac{1+\sqrt{5}}{2}\right)^n - \left(\frac{1-\sqrt{5}}{2}\right)^n\right] \quad \text{对所有正整数 } n \text{成立}.
\]

\begin{solution}
\textbf{用归纳法证明:}

\textit{当 $n=1$ 时:}
\begin{align*}
\text{L.H.S.} &= a_1 = 1,\\
\text{R.H.S.} &= \frac{1}{\sqrt{5}}\left[\frac{1+\sqrt{5}}{2} - \frac{1-\sqrt{5}}{2}\right] 
= \frac{1}{\sqrt{5}}\cdot \sqrt{5} = 1 = \text{L.H.S.}
\end{align*}

\textit{当 $n=2$ 时:}
\begin{align*}
\text{L.H.S.} &= a_2 = 1,\\
\text{R.H.S.} &= \frac{1}{\sqrt{5}}\left[\left(\frac{1+\sqrt{5}}{2}\right)^2 - \left(\frac{1-\sqrt{5}}{2}\right)^2\right] 
= \frac{1}{\sqrt{5}}\left[(\frac{6+2\sqrt{5}}{4}) - (\frac{6-2\sqrt{5}}{4})\right] 
= 1 = \text{L.H.S.}
\end{align*}

假设当 $n=k$ 和 $n=k+1$ 时命题成立,即
\[
a_k = \frac{1}{\sqrt{5}}\left[\left(\frac{1+\sqrt{5}}{2}\right)^k - \left(\frac{1-\sqrt{5}}{2}\right)^k\right],\quad
a_{k+1} = \frac{1}{\sqrt{5}}\left[\left(\frac{1+\sqrt{5}}{2}\right)^{k+1} - \left(\frac{1-\sqrt{5}}{2}\right)^{k+1}\right].
\]

\textit{当 $n=k+2$ 时:}
\begin{align*}
a_{k+2} &= a_{k+1} + a_k \\
&= \frac{1}{\sqrt{5}}\Big[\left(\frac{1+\sqrt{5}}{2}\right)^{k+1} - \left(\frac{1-\sqrt{5}}{2}\right)^{k+1} + \left(\frac{1+\sqrt{5}}{2}\right)^k - \left(\frac{1-\sqrt{5}}{2}\right)^k\Big] \\
&= \frac{1}{\sqrt{5}}\Big[\left(\frac{1+\sqrt{5}}{2}\right)^k\left(\frac{1+\sqrt{5}}{2}+1\right) - \left(\frac{1-\sqrt{5}}{2}\right)^k\left(\frac{1-\sqrt{5}}{2}+1\right)\Big] \\
&= \frac{1}{\sqrt{5}}\Big[\left(\frac{1+\sqrt{5}}{2}\right)^k \left(\frac{3+\sqrt{5}}{2}\right) - \left(\frac{1-\sqrt{5}}{2}\right)^k \left(\frac{3-\sqrt{5}}{2}\right)\Big].
\end{align*}

注意到
\[
\frac{3+\sqrt{5}}{2} = \left(\frac{1+\sqrt{5}}{2}\right)^2,\quad \frac{3-\sqrt{5}}{2} = \left(\frac{1-\sqrt{5}}{2}\right)^2,
\]
所以
\[
a_{k+2} = \frac{1}{\sqrt{5}}\left[\left(\frac{1+\sqrt{5}}{2}\right)^{k+2} - \left(\frac{1-\sqrt{5}}{2}\right)^{k+2}\right].
\]

由数学归纳法得证,命题对所有正整数 $n$ 成立。
\end{solution}

\question 已知数列由递推关系
\[
a_{n+1}=\frac{a_n-5}{3a_n-7},\quad a_1=-1,\quad n\in\mathbb{N},\ n\ge1
\]
定义,用数学归纳法证明
\[
a_n=\frac{2^{\,n+1}-5}{2^{\,n+1}-3}.
\]

\begin{solution}
先改写递推关系。
\[
a_{n+1}=\frac{a_n-5}{3a_n-7}
=\frac{1}{3}\frac{a_n-5}{a_n-\frac{7}{3}}
=\frac{1}{3}\left(1-\frac{\frac{8}{3}}{a_n-\frac{7}{3}}\right)
\]
\[
a_{n+1}=\frac{1}{3}-\frac{8}{9a_n-21}.
\]

先验证基础情况。
当 $n=1$ 时,
\[
a_1=\frac{2^{1+1}-5}{2^{1+1}-3}
=\frac{4-5}{4-3}
=-1,
\]
与已知 $a_1=-1$ 一致,因此结论对 $n=1$ 成立。

假设当 $n=k$ 时结论成立,即
\[
a_k=\frac{2^{k+1}-5}{2^{k+1}-3}.
\]

则
\[
9a_k-21
=9\left(\frac{2^{k+1}-5}{2^{k+1}-3}\right)-21
=\frac{9\cdot2^{k+1}-45-21(2^{k+1}-3)}{2^{k+1}-3}
\]
\[
=\frac{-12\cdot2^{k+1}+18}{2^{k+1}-3}.
\]

因此
\[
\frac{8}{9a_k-21}
=\frac{8(2^{k+1}-3)}{-12\cdot2^{k+1}+18}.
\]

代回递推关系得
\[
a_{k+1}
=\frac{1}{3}-\frac{8}{9a_k-21}
=\frac{1}{3}-\frac{8(2^{k+1}-3)}{12\cdot2^{k+1}-18}.
\]

通分化简得
\[
a_{k+1}
=\frac{6\cdot2^{k+1}-9+12\cdot2^{k+1}-36}{3(6\cdot2^{k+1}-9)}
=\frac{18\cdot2^{k+1}-45}{9(2\cdot2^{k+1}-3)}.
\]

约分后
\[
a_{k+1}=\frac{2\cdot2^{k+1}-5}{2\cdot2^{k+1}-3}
=\frac{2^{k+2}-5}{2^{k+2}-3}.
\]

因此若结论对 $n=k$ 成立,则对 $n=k+1$ 也成立。

由于结论对 $n=1$ 成立,根据数学归纳法,结论对所有 $n\in\mathbb{N}$ 成立。
\end{solution}

\question 证明对所有整数 $n \ge 1$,有
\[
(a+b)^n = \prescript{n}{}{\text{C}}_0 a^n + \prescript{n}{}{\text{C}}_1 a^{n-1}b + \prescript{n}{}{\text{C}}_2 a^{n-2}b^2 + \dots + \prescript{n}{}{\text{C}}_n b^n.
\]

\begin{solution}
\textbf{用数学归纳法证明:}

\textit{当 $n=1$ 时:}
\begin{align*}
\text{L.H.S.} &= (a+b)^1 = a+b,\\
\text{R.H.S.} &= \prescript{1}{}{\text{C}}_0 a^1 + \prescript{1}{}{\text{C}}_1 b^1 = a+b = \text{L.H.S.}
\end{align*}
所以当 $n=1$ 时命题成立。

假设当 $n=k$ 时命题成立,即
\[
(a+b)^k = \prescript{k}{}{\text{C}}_0 a^k + \prescript{k}{}{\text{C}}_1 a^{k-1}b + \dots + \prescript{k}{}{\text{C}}_k b^k.
\]

\textit{当 $n=k+1$ 时:}
\begin{align*}
(a+b)^{k+1} &= (a+b)(a+b)^k \\
&= (a+b)(\prescript{k}{}{\text{C}}_0 a^k + \prescript{k}{}{\text{C}}_1 a^{k-1}b + \dots + \prescript{k}{}{\text{C}}_k b^k) \\
&= a(\prescript{k}{}{\text{C}}_0 a^k + \dots + \prescript{k}{}{\text{C}}_k b^k) + b(\prescript{k}{}{\text{C}}_0 a^k + \dots + \prescript{k}{}{\text{C}}_k b^k) \\
&= \prescript{k}{}{\text{C}}_0 a^{k+1} + (\prescript{k}{}{\text{C}}_1 + \prescript{k}{}{\text{C}}_0)a^k b + (\prescript{k}{}{\text{C}}_2 + \prescript{k}{}{\text{C}}_1)a^{k-1} b^2 + \dots \\
&\quad + (\prescript{k}{}{\text{C}}_k + \prescript{k}{}{\text{C}}_{k-1})a b^k + \prescript{k}{}{\text{C}}_k b^{k+1}.
\end{align*}

利用组合数恒等式
\[
\prescript{k}{}{\text{C}}_r + \prescript{k}{}{\text{C}}_{r-1} = \prescript{k+1}{}{\text{C}}_r,
\]
得到
\[
(a+b)^{k+1} = \prescript{k+1}{}{\text{C}}_0 a^{k+1} + \prescript{k+1}{}{\text{C}}_1 a^k b + \dots + \prescript{k+1}{}{\text{C}}_{k+1} b^{k+1} = \text{R.H.S.}
\]

因此命题对 $n=k+1$ 成立。由数学归纳法,命题对所有正整数 $n$ 成立。
\end{solution}



\question 求极限
\[
\lim_{n \to \infty} \sum_{k=1}^{n} \Bigg(\sin \left(\frac{\pi/2}{k}\right) - \cos \left(\frac{\pi/2}{k}\right) - \sin \left(\frac{\pi/2}{k+2}\right) + \cos \left(\frac{\pi/2}{k+2}\right)\Bigg).
\]

\begin{solution}
设部分和为
\[
S_n = \sum_{k=1}^{n} \Bigg(\sin \left(\frac{\pi/2}{k}\right) - \cos \left(\frac{\pi/2}{k}\right) - \sin \left(\frac{\pi/2}{k+2}\right) + \cos \left(\frac{\pi/2}{k+2}\right)\Bigg).
\]

观察前几项的望远镜消去规律:
\begin{align*}
S_1 &= 1 - \sin(\pi/6) + \cos(\pi/6), \\
S_2 &= S_1 + \sin(\pi/4) - \cos(\pi/4) - \sin(\pi/8) + \cos(\pi/8) = 1 - \sin(\pi/6) + \cos(\pi/6) - \sin(\pi/8) + \cos(\pi/8), \\
S_3 &= S_2 + \sin(\pi/6) - \cos(\pi/6) - \sin(\pi/10) + \cos(\pi/10) = 1 - \sin(\pi/8) + \cos(\pi/8) - \sin(\pi/10) + \cos(\pi/10), \\
S_4 &= S_3 + \sin(\pi/8) - \cos(\pi/8) - \sin(\pi/12) + \cos(\pi/12) = 1 - \sin(\pi/10) + \cos(\pi/10) - \sin(\pi/12) + \cos(\pi/12).
\end{align*}

由归纳法可得一般形式:
\[
S_n = 1 - \sin \frac{\pi/2}{n+1} - \sin \frac{\pi/2}{n+2} + \cos \frac{\pi/2}{n+1} + \cos \frac{\pi/2}{n+2}.
\]

当 $n \to \infty$ 时,
\[
\lim_{n \to \infty} S_n = 1 - 0 - 0 + 1 + 1 = 3.
\]
\end{solution}

\question 设数列 $(a_n)_{n=0}^\infty$ 满足对所有 $n\ge0$ 都有
\[
\frac{1}{2}<a_n<1.
\]
定义数列 $(x_n)$ 为
\[
x_0=a_0,\quad x_{n+1}=\frac{a_{n+1}+x_n}{1+a_{n+1}x_n}\quad(n\ge0).
\]
求 $\lim_{n\to\infty}x_n$ 的所有可能取值,并判断该数列是否可能发散。

\begin{solution}
我们用归纳法证明
\[
0<1-x_n<\frac{1}{2^{n+1}}.
\]
由此可得 $(1-x_n)\to0$,从而 $x_n\to1$。

当 $n=0$ 时,由 $\frac{1}{2}<x_0=a_0<1$,结论成立。

假设结论对某个 $n$ 成立,由递推关系得
\begin{align*}
1-x_{n+1}
&=1-\frac{a_{n+1}+x_n}{1+a_{n+1}x_n}
=\frac{1-a_{n+1}}{1+a_{n+1}x_n}(1-x_n).
\end{align*}
由于
\[
0<\frac{1-a_{n+1}}{1+a_{n+1}x_n}
<\frac{1-\frac{1}{2}}{1+0}
=\frac{1}{2},
\]
于是
\[
0<1-x_{n+1}
<\frac{1}{2}(1-x_n)
<\frac{1}{2}\cdot\frac{1}{2^{n+1}}
=\frac{1}{2^{n+2}}.
\]
因此结论对 $n+1$ 也成立。

综上,$1-x_n\to0$,故
\[
\lim_{n\to\infty}x_n=1.
\]
因此该数列在所有情况下都收敛,其极限唯一为 $1$,不可能发散。
\end{solution}


\question 对于实数 $a>0$,定义数列 $\{x_n\}$ 为
\[
x_{n+1}=a(x_n^2+4),\quad x_0=0.
\]
求 $\lim_{n\to\infty}x_n$ 存在且有限的充要条件。

\begin{solution}
假设 $\lim_{n\to\infty}x_n=x$,且 $x$ 有限,则
\[
x=a(x^2+4),
\]
从而
\[
x=\frac{1\pm\sqrt{1-16a^2}}{2a}.
\]
要使 $x$ 为实数且有限,必须有 $1-16a^2\ge0$。又已知 $a>0$,于是
\[
0<a\le\frac14.
\]

下面假设 $0<a\le\frac14$。用归纳法可证明 $\{x_n\}$ 为单调不减数列。
有
\[
x_1=4a\ge x_0=0.
\]
若假设 $x_n\ge x_{n-1}$,则
\[
x_{n+1}-x_n=a(x_n^2-x_{n-1}^2),
\]
从而 $x_{n+1}\ge x_n$。

注意到 $x_1=4a<2$,再用归纳法可证明 $\{x_n\}$ 有上界。若 $x_n<2$,则
\[
x_{n+1}=a(x_n^2+4)<\frac14(4+4)=2.
\]
因此 $\{x_n\}$ 是一个有上界的单调不减数列,必然收敛于一个有限实数。

综上所述,所求的充要条件为
\[
0<a\le\frac14.
\]
\end{solution}

\question 设 $n$ 为正整数,且 $a_k,b_k$ 满足
\[
a_{k}=\frac{1}{\binom{n}{k}}, \quad b_{k}=2^{k-n}, \quad \text{for } k=1,2,...,n.
\]
证明
\[
\frac{a_1-b_1}{1}+\frac{a_2-b_2}{2}+\cdots+\frac{a_n-b_n}{n}=0。
\tag{1}
\]

\begin{solution}
由于对所有 $k\ge1$ 都有
\[
k\binom{n}{k}=n\binom{n-1}{k-1},
\]
式 (1) 等价于
\[
\frac{2^n}{n}\left[
\frac{1}{\binom{n-1}{0}}+\frac{1}{\binom{n-1}{1}}+\cdots+\frac{1}{\binom{n-1}{n-1}}
\right]
=
\frac{2^1}{1}+\frac{2^2}{2}+\cdots+\frac{2^n}{n}。
\tag{2}
\]

下面用数学归纳法证明 (2)。

当 $n=1$ 时,等式两边均等于 $2$,结论成立。

假设 (2) 对某个 $n$ 成立。设
\[
X_n=
\frac{2^n}{n}\left[
\frac{1}{\binom{n-1}{0}}+\frac{1}{\binom{n-1}{1}}+\cdots+\frac{1}{\binom{n-1}{n-1}}
\right]。
\]
则
\begin{align*}
X_{n+1}
&=
\frac{2^{n+1}}{n+1}\sum_{k=0}^{n}\frac{1}{\binom{n}{k}} \\
&=
\frac{2^n}{n+1}\left(
1+\sum_{k=0}^{n-1}\left(
\frac{1}{\binom{n}{k}}+\frac{1}{\binom{n}{k+1}}
\right)+1
\right)。
\end{align*}

利用
\[
\frac{1}{\binom{n}{k}}+\frac{1}{\binom{n}{k+1}}
=\frac{n+1}{k+1}\cdot\frac{1}{\binom{n-1}{k}},
\]
得到
\begin{align*}
X_{n+1}
&=
\frac{2^n}{n+1}\left(
\sum_{k=0}^{n-1}\frac{n+1}{k+1}\cdot\frac{1}{\binom{n-1}{k}}
\right)
+\frac{2^{n+1}}{n+1} \\
&=
\frac{2^n}{n}\sum_{k=0}^{n-1}\frac{1}{\binom{n-1}{k}}
+\frac{2^{n+1}}{n+1} \\
&=
X_n+\frac{2^{n+1}}{n+1}。
\end{align*}

由归纳假设可得
\[
X_{n+1}=\frac{2^1}{1}+\frac{2^2}{2}+\cdots+\frac{2^{n+1}}{n+1},
\]
从而 (2) 对 $n+1$ 成立。

因此 (2) 对所有正整数 $n$ 成立,从而 (1) 也成立。
\end{solution}

\question 定义数列 $x_1, x_2, \dots$ 递推为 $x_1 = \sqrt{5}$ 且 $x_{n+1} = x_n^2 - 2$ 对每个 $n \ge 1$. 求
\[
\lim_{n \to \infty} \frac{x_1 \cdot x_2 \cdot \cdots \cdot x_n}{x_{n+1}}.
\]

\begin{solution}
令 $y_n = x_n^2$,则递推关系变为 $y_{n+1} = (y_n - 2)^2$,并且
\[
y_{n+1} - 4 = y_n (y_n - 4).
\]
由于 $y_2 > 5$,可以归纳得到 $y_n > 5$ 对所有 $n \ge 2$,从而
\[
y_{n+1} - y_n = y_n^2 - 5y_n + 4 > 4,
\]
所以 $y_n \to \infty$。

利用关系 $y_{n+1} - 4 = y_n (y_n - 4)$,有
\begin{align*}
\left(\frac{x_1 x_2 \cdots x_n}{x_{n+1}}\right)^2 &= \frac{y_1 y_2 \cdots y_n}{y_{n+1}} 
= \frac{y_1 y_2 \cdots y_n}{y_{n+1}-4} \cdot \frac{y_{n+1}-4}{y_{n+1}} 
= \frac{y_1 y_2 \cdots y_{n-1}}{y_n - 4} \cdot \frac{y_{n+1}-4}{y_{n+1}} \\
&= \cdots = \frac{y_1}{y_2 - 4} \cdot \frac{y_2 - 4}{y_3 - 4} \cdots \frac{y_n - 4}{y_{n+1} - 4} \cdot \frac{y_{n+1}-4}{y_{n+1}} 
= \frac{y_1}{y_{n+1}} \cdot \frac{y_{n+1}-4}{y_1 - 4} \to 1.
\end{align*}

因此,
\[
\lim_{n \to \infty} \frac{x_1 x_2 \cdots x_n}{x_{n+1}} = 1.
\]
\end{solution}


\question 设数列 $\{a_n\}_{n=1}^{\infty}$ 定义为
\[
a_1=1,\quad a_2=1,\quad a_3=2,
\]
并且当 $n\ge3$ 时,
\[
a_{n+1}=\frac{n a_n a_{n-2}}{a_{n-1}}。
\]
\begin{parts}
\item[(a)] 证明对所有 $n\ge1$,$a_n$ 都是正整数。
\item[(b)] 定义数列 $\{b_n\}_{n=1}^{\infty}$ 为
\[
b_n=\frac{a_n}{\sqrt{(n+1)!}}。
\]
证明 $\{b_n\}$ 有界。
\end{parts}

\begin{solution}
(a) 对于 $n\ge1$,有
\[
a_{n+3}=\frac{(n+2)a_{n+2}a_n}{a_{n+1}},
\]
从而
\begin{align*}
a_{n+4}
&=\frac{(n+3)a_{n+3}a_{n+1}}{a_{n+2}} \\
&=\frac{(n+3)(n+2)a_{n+2}a_n a_{n+1}}{a_{n+1}a_{n+2}} \\
&=(n+3)(n+2)a_n。
\end{align*}
因此
\[
a_{n+4}=(n+3)(n+2)a_n。 \tag{3}
\]

由于
\[
a_4=\frac{3a_3a_1}{a_2}=6,
\]
可知 $a_1,a_2,a_3,a_4$ 均为正整数。假设对所有 $k<n+4$,$a_k$ 都是正整数,则由强归纳法和公式 (3) 可得 $a_{n+4}$ 也是正整数。于是对所有 $n\ge1$,$a_n$ 都是正整数。

(b) 由 (3) 式,有
\begin{align*}
b_{n+4}
&=\frac{a_{n+4}}{\sqrt{(n+5)!}} \\
&=\frac{(n+3)(n+2)a_n}{\sqrt{(n+1)!(n+2)(n+3)(n+4)(n+5)}} \\
&=\frac{a_n}{\sqrt{(n+1)!}}
\sqrt{\frac{(n+3)(n+2)}{(n+4)(n+5)}} \\
&\le \frac{a_n}{\sqrt{(n+1)!}} \\
&=b_n。
\end{align*}

由归纳法,并注意到 $b_1,b_2,b_3,b_4\le1$,可得对所有 $n\ge1$,
\[
b_n\le1。
\]
因此数列 $\{b_n\}$ 有界。
\end{solution}

\question 设 $x_1=1$,且对 $n\ge 1$,
\[
x_{n+1}=\frac{1}{x_n}\left(\sqrt{1+x_n^2}-1\right)。
\]
证明数列 $\{2^n x_n\}$ 收敛,并求其极限。

\begin{solution}
我们先用数学归纳法证明
\[
x_n=\tan(\pi/2^{\,n+1}) \tag{3}
\]
对所有正整数 $n$ 都成立。由于 $x_1=1=\tan(\pi/4)$,(3) 对 $n=1$ 成立。假设 (3) 对某个正整数 $n$ 成立,则
\begin{align*}
x_{n+1} &= \frac{1}{\tan(\pi/2^{\,n+1})}\Bigl(\sqrt{1+\tan^2(\pi/2^{\,n+1})}-1\Bigr) \\
&= \frac{\sec(\pi/2^{\,n+1})-1}{\tan(\pi/2^{\,n+1})} 
= \frac{1-\cos(\pi/2^{\,n+1})}{\sin(\pi/2^{\,n+1})} \\
&= \tan\Bigl(\frac{1}{2}\cdot \pi/2^{\,n+1}\Bigr) 
= \tan(\pi/2^{\,n+2})。
\end{align*}
因此,由归纳法可知 (3) 对所有正整数 $n$ 都成立。

令 $y=\pi/2^n$,则当 $n\to\infty$ 时,$y\to 0^+$,由洛必达法则得到
\[
\lim_{n\to\infty} 2^n x_n = \lim_{y\to 0^+} \frac{\tan(\pi y/2)}{y} 
= \lim_{y\to 0^+} \frac{\pi \sec^2(\pi y/2)}{2} = \frac{\pi}{2}。
\]
\end{solution}




    \question 设无穷数列$\{a_n\}$符合 $a_0=0$ 且当 $n \ge 1$ 时,
    \[
    a_n - a_{n-1} =
    \begin{cases}
    \left(\dfrac{1}{5}\right)^n, & n \: \text{为偶数} \\[2mm]
    \left(\dfrac{1}{5}\right)^n - \left(\dfrac{1}{3}\right)^n, & n \: \text{为奇数}
    \end{cases}
    \]
    \begin{parts}
    \part 求 $\lim_{n \to \infty} a_{2n}$ 的值。
    \begin{solution}
        由于
        \begin{align*}
        a_{2n} - a_0 
        &= a_{2n} - a_{2n-1}+a_{2n-1} - a_{2n-2}+ \cdots +a_2 - a_1+a_1 - a_0 \\
        &= \left(\frac15 + \frac1{5^2} + \cdots + \frac1{5^{2n}}\right)
        - \left(\frac13 + \frac1{3^3} + \cdots + \frac1{3^{2n-1}}\right)
        \end{align*}
        所以
        \[
        \lim_{n\to\infty} a_{2n} = \frac{\frac15}{1 - \frac15} - \frac{\frac13}{1 - \frac19}
        = -\frac18
        \]
    \end{solution}

    \part 证明当 $n \ge 0,a_{2n+2} - a_{2n} < 0$,并依此证明对于所有正整数 $n$,不等式 $$-\frac{1}{8} \le a_{2n} < 0$$ 恒成立。
    \begin{solution}
        \begin{align*} 
        a_{2n+2} - a_{2n} &=a_{2n+2} - a_{2n+1}+a_{2n+1} - a_{2n}\\
        &=\left(\frac15\right)^{2n+2}+\left(\frac15\right)^{2n+1} - \left(\frac13\right)^{2n+1}\\
        &= \frac{\frac65 \cdot 3^{2n+1} - 5^{2n+1}}{15^{2n+1}}
        \end{align*}
        当 $n=0$ 时,$\: a_2 - a_0 = \frac{18/5 - 5}{15} < 0$,成立;假设 $n=k$ 时成立,即
        \[
        \frac65 \cdot 3^{2k+1} < 5^{2k+1}
        \]
        观察当 $n=k+1$ 时,
        \[
        a_{2k+4} - a_{2k+2} = \frac{\frac65 \cdot 3^{2k+3} - 5^{2k+3}}{15^{2k+3}}
        \]
        由归纳假设,
        \[
        \frac65 \cdot 3^{2k+3} < 3^2 \cdot 5^{2k+1} < 5^2 \cdot 5^{2k+1} = 5^{2k+3}
        \]
        因此 $a_{2k+4} - a_{2k+2} < 0$,即当 $n=k+1$ 时也成立;由数学归纳法,$\: a_{2n+2} - a_{2n} < 0$;又 $$a_{2n} < a_{2n-2} < \cdots < a_2 < a_0 = 0,$$故 $a_{2n} < 0$,结合 $\displaystyle \lim_{n\to\infty} a_{2n} = -\frac18$得
        \[
        -\frac18 \le a_{2n} < 0,\quad n \in \mathbb{N}
        \]
    \end{solution}
    \end{parts}

    \question 设正实数 $a_0, a_1, \ldots, a_n$ 满足 $a_{k+1} - a_k \ge 1$ 对所有 $k = 0,1,\ldots,n-1$ 成立。证明
\[
1 + \left(1+\frac{1}{a_0}\right)\cdots \left(1+\frac{1}{a_n-a_0}\right) \le \left(1+\frac{1}{a_0}\right)\left(1+\frac{1}{a_1}\right)\cdots \left(1+\frac{1}{a_n}\right).
\]

\begin{solution}
用归纳法对 $n$ 证明。空积视作 1,则 $n=0$ 时显然成立。

假设对某个 $n$ 结论成立,考虑 $n+1$。不等式可以拆为两部分:
\[
1+\left(1+\frac{1}{a_0}\right)\cdots\left(1+\frac{1}{a_n-a_0}\right) \le \left(1+\frac{1}{a_0}\right)\cdots\left(1+\frac{1}{a_n}\right)
\]
(这是归纳假设),以及
\[
\frac{1}{a_0}\left(1+\frac{1}{a_1-a_0}\right)\cdots\left(1+\frac{1}{a_n-a_0}\right)\cdot \frac{1}{a_{n+1}-a_0} \le \left(1+\frac{1}{a_0}\right)\cdots \left(1+\frac{1}{a_n}\right)\cdot \frac{1}{a_{n+1}}. \tag{*}
\]

只需证明不等式 $(*)$。

对 $n$ 再用归纳法。$n=0$ 时,需要验证
\[
\frac{1}{a_0}\cdot \frac{1}{a_1-a_0} \le \left(1+\frac{1}{a_0}\right)\cdot \frac{1}{a_1}.
\]

两边乘以 $a_0 a_1 (a_1 - a_0)$,化简为
\[
a_1 \le (a_0+1)(a_1 - a_0) \implies a_0 \le a_0 a_1 - a_0^2 \implies 1 \le a_1 - a_0,
\]
成立。

归纳步骤只需证明
\[
\left(1 + \frac{1}{a_{n+1}-a_0}\right) \cdot \frac{a_{n+1}-a_0}{a_{n+2}-a_0} \le \left(1+\frac{1}{a_{n+1}}\right) \cdot \frac{a_{n+1}}{a_{n+2}}.
\]

两边乘以 $(a_{n+2}-a_0)a_{n+2}$,得到
\[
(a_{n+1}-a_0+1)a_{n+2} \le (a_{n+1}+1)(a_{n+2}-a_0) \implies a_0 \le a_0 a_{n+2} - a_0 a_{n+1} \implies 1 \le a_{n+2} - a_{n+1},
\]
成立。因此归纳完成。
\end{solution}

\question 用数学归纳法证明
\[
\sum_{r=1}^{n}\left[r(r+1)\left(\frac{1}{2}\right)^{r-1}\right]
=16-\left(\frac{1}{2}\right)^{\,n-1}(n^2+5n+8),
\quad n\ge1,\ n\in\mathbb{N}.
\]

\begin{solution}
先验证基础情况。

当 $n=1$ 时,
\begin{align*}
\sum_{r=1}^{1}\left[r(r+1)\left(\frac{1}{2}\right)^{r-1}\right]
&=1\cdot2\cdot\left(\frac{1}{2}\right)^0\\
&=2,
\end{align*}
而
\begin{align*}
16-\left(\frac{1}{2}\right)^{1-1}(1^2+5\cdot1+8)
&=16-(1)(14)\\
&=2.
\end{align*}
因此结论对 $n=1$ 成立。

假设当 $n=k$ 时结论成立,即
\[
\sum_{r=1}^{k}\left[r(r+1)\left(\frac{1}{2}\right)^{r-1}\right]
=16-\left(\frac{1}{2}\right)^{k-1}(k^2+5k+8).
\]

下面证明结论对 $n=k+1$ 成立。

\begin{align*}
\sum_{r=1}^{k+1}\left[r(r+1)\left(\frac{1}{2}\right)^{r-1}\right]
&=\sum_{r=1}^{k}\left[r(r+1)\left(\frac{1}{2}\right)^{r-1}\right]
+(k+1)(k+2)\left(\frac{1}{2}\right)^k\\
&=16-\left(\frac{1}{2}\right)^{k-1}(k^2+5k+8)
+(k+1)(k+2)\left(\frac{1}{2}\right)^k\\
&=16+\left(\frac{1}{2}\right)^k\left[(k+1)(k+2)
-2(k^2+5k+8)\right]\\
&=16+\left(\frac{1}{2}\right)^k
\left[k^2+3k+2-2k^2-10k-16\right]\\
&=16-\left(\frac{1}{2}\right)^k
\left[k^2+7k+14\right]\\
&=16-\left(\frac{1}{2}\right)^k
\left[(k+1)^2+5(k+1)+8\right]\\
&=16-\left(\frac{1}{2}\right)^{(k+1)-1}
\left[(k+1)^2+5(k+1)+8\right].
\end{align*}

因此若结论对 $n=k$ 成立,则对 $n=k+1$ 也成立。

由数学归纳法可知,该等式对所有 $n\in\mathbb{N}$ 成立。
\end{solution}

    \question 对每个 $n\ge 1$,令
\[
a_n=\sum_{k=0}^\infty \frac{k^n}{k!}, \quad b_n=\sum_{k=0}^\infty \frac{(-1)^k k^n}{k!}.
\]
证明 $a_n\cdot b_n$ 是整数。

\begin{solution}
我们用归纳法证明:对所有 $n\ge 0$,$a_n/e$ 与 $b_n e$ 都是整数。这里也包含 $n=0$ 的情形(在定义中约定 $0^0=1$)。

由指数函数的幂级数展开可知
\[
a_0=\sum_{k=0}^\infty \frac{1}{k!}=e,\quad 
b_0=\sum_{k=0}^\infty \frac{(-1)^k}{k!}=e^{-1},
\]
因此结论对 $n=0$ 成立。

假设对某个 $n\ge 0$,$a_0,a_1,\ldots,a_n$ 都是 $e$ 的整数倍,$b_0,b_1,\ldots,b_n$ 都是 $e^{-1}$ 的整数倍。下面证明结论对 $n+1$ 也成立。

由二项式定理,有
\begin{align*}
a_{n+1}
&=\sum_{k=0}^\infty \frac{(k+1)^{\,n+1}}{(k+1)!}
=\sum_{k=0}^\infty \frac{(k+1)^n}{k!} \\
&=\sum_{k=0}^\infty \frac{1}{k!}\sum_{m=0}^n \binom{n}{m}k^m
=\sum_{m=0}^n \binom{n}{m}\sum_{k=0}^\infty \frac{k^m}{k!} \\
&=\sum_{m=0}^n \binom{n}{m}a_m.
\end{align*}
同理,
\begin{align*}
b_{n+1}
&=\sum_{k=0}^\infty \frac{(-1)^{k+1}(k+1)^{\,n+1}}{(k+1)!}
=-\sum_{k=0}^\infty \frac{(-1)^k (k+1)^n}{k!} \\
&=-\sum_{k=0}^\infty \frac{(-1)^k}{k!}\sum_{m=0}^n \binom{n}{m}k^m
=-\sum_{m=0}^n \binom{n}{m}\sum_{k=0}^\infty \frac{(-1)^k k^m}{k!} \\
&=-\sum_{m=0}^n \binom{n}{m} b_m.
\end{align*}

由归纳假设,$a_m$ 是 $e$ 的整数倍,$b_m$ 是 $e^{-1}$ 的整数倍,而上式中系数均为整数,因此 $a_{n+1}$ 仍是 $e$ 的整数倍,$b_{n+1}$ 仍是 $e^{-1}$ 的整数倍。

由归纳法可知,对所有 $n\ge 0$,$a_n/e$ 与 $b_n e$ 均为整数,从而
\[
a_n b_n=\left(\frac{a_n}{e}\right)\left(b_n e\right)
\]
是整数。
\end{solution}

    \question 试证对任意正整数 $n$,
    \[
    \floor*{\sqrt{n}} + \sum_{k=1}^{n} \;\floor*{\frac{n}{k}}
    \]
    必为偶数,其中 $\floor*{x}$ 为高斯函数。
    \begin{solution}
        定义命题:
        \[
        P_n: a_n = \floor*{\sqrt{n}} + \sum_{k=1}^n \;\floor*{\frac{n}{k}} \text{ 为偶数},
        \]
        其中 $n \in \mathbb{Z}^{+}$。观察当 $n=1$ 时,
        \[
        a_1 = \floor*{\sqrt{1}} + \floor*{\frac{1}{1}} = 1+1=2
        \]
        为偶数,即 $P_1$ 成立。现假设 $P_n$ 成立,即当 $n=\ell$ 时,
        \[
        a_{\ell} = \floor*{\sqrt{\ell}} + \sum_{k=1}^{\ell} \;\floor*{\frac{\ell}{k}}
        \]
        为偶数。观察当 $n=\ell+1$ 时:
        \begin{itemize}
        \item 若 $\ell+1$ 为完全平方数,则
        \[
        \floor*{\sqrt{\ell+1}} = \floor*{\sqrt{\ell}} + 1,
        \]
        且
        \[
        \sum_{k=1}^{\ell+1} \;\floor*{\frac{\ell+1}{k}} = \sum_{k=1}^{\ell} \;\floor*{\frac{\ell}{k}} + m,
        \]
        其中 $m$ 为 $\ell+1$ 的正因数个数(奇数)。于是
        \[
        a_{\ell+1} = a_\ell + m + 1 = \text{偶} + \text{奇} + \text{奇} = \text{偶}.
        \]
        \item 若 $\ell+1$ 不是完全平方数,则
        \[
        \floor*{\sqrt{\ell+1}} = \floor*{\sqrt{\ell}},
        \]
        且
        \[
        \sum_{k=1}^{\ell+1} \;\floor*{\frac{\ell+1}{k}} = \sum_{k=1}^{\ell} \;\floor*{\frac{\ell}{k}} + m,
        \]
        此时 $m$ 为偶数,于是
        \[
        a_{\ell+1} = a_\ell + m = \text{偶} + \text{偶} = \text{偶}.
        \]
        \end{itemize}
        因此 $P_{\ell+1}$ 亦成立。即 $P_n \implies P_{n+1}$。由数学归纳法,$P_n$ 对任意正整数 $n$ 都成立。
    \end{solution}

    \question 票箱中有甲、乙两人的选票分别为 $m$ 张和 $n$ 张且 $m>n$。令 $P_{m,n}$ 表示开票的过程中甲的选票会一路领先乙的选票的概率,
    \begin{parts}
    \part 计算 $P_{m,1}$ 和 $P_{m,2}$。
    \begin{solution}
        视甲得$1$票为向上走一步,乙得$1$票为向右走一步,
        则 $P_{m,n}$ 表示从原点 $O(0,0)$ 走格点至 $P(n,m)$ 但不经过 $(a,a)$ 的概率。

        $m$ 个上、1 个右的排列数为 $\dfrac{(m+1)!}{m!}=m+1$,其中
        \[
        \begin{cases}
        \text{开头为右的数列有 1 个}\\
        \text{开头为上右的数列也只有 1 个}
        \end{cases}
        \]
        因此符合条件的数列有 $m+1-2=m-1$ 个,故$$P_{m,1}=\frac{m-1}{m+1}$$

        $m$ 个上、2 个右的排列数为 $\dfrac{(m+2)!}{m!2!}=\dfrac{(m+2)(m+1)}{2}$,其中
        \[
        \begin{cases}
        \text{开头为右的数列有 $m+1$ 个}\\
        \text{开头为上右的数列有 $m$ 个}\\
        \text{开头为上上右右的数列有 1 个}
        \end{cases}
        \]
        因此符合条件的数列有
        \[
        \frac{(m+2)(m+1)}{2}-(m+1)-m-1=\frac{(m+1)(m-2)}{2}
        \]
        所以 $$P_{m,2}=\frac{m-2}{m+2}$$
    \end{solution}
    \part 证明 $P_{m,n} = \dfrac{m}{m+n} P_{m-1,n} + \dfrac{n}{m+n} P_{m,n-1}$。
    \begin{solution}
        从 $(0,0)$ 至 $(n,m)$ 的方法数等于从 $(0,0)$ 至 $(n-1,m)$ 的方法数加上从 $(0,0)$ 至 $(n,m-1)$ 的方法数,因此
        \[
        P_{m,n}=\frac{\frac{(m+n-1)!}{(m-1)!n!}P_{m-1,n}+\frac{(m+n-1)!}{m!(n-1)!}P_{m,n-1}}{\frac{(m+n)!}{m!n!}}=\frac{m}{m+n}P_{m-1,n}+\frac{n}{m+n}P_{m,n-1}
        \]
    \end{solution}
    \part 先猜测 $P_{m,n}$ 的答案,再利用(b)用归纳法证明你的猜测。
    \begin{solution}
        由(a)可猜 $$P_{m,n}=\frac{m-n}{m+n},\ m\ge n$$
        令 $k=m+n$,观察当 $k=2$ 时,
        \[
        \begin{cases}
        m=2,n=0 \Rightarrow P_{2,0}=\frac{2-0}{2+0}=1\\
        m=n=1 \Rightarrow P_{1,1}=\frac{1-1}{1+1}=0
        \end{cases}
        \]
        显然猜测成立。现假设 $k=N$ 时猜测成立,观察当 $k=N+1$ 时,
        \begin{align*} 
        P_{m,n}&=\frac{m}{m+n}P_{m-1,n}+\frac{n}{m+n}P_{m,n-1}\\
        &=\frac{m}{m+n}\cdot\frac{m-n-1}{m+n-1}+\frac{n}{m+n}\cdot\frac{m-n+1}{m+n-1}\\
        &=\frac{(m-n)(m+n-1)}{(m+n)(m+n-1)}=\frac{m-n}{m+n}
        \end{align*}
        猜测亦成立,故猜测对所有$m,n\in \mathbb{N}\cup\{0\}$且$m\ge n$皆成立。
    \end{solution}
    \end{parts}
\question 用数学归纳法证明:若 $n \ge 1,\; n \in \mathbb{N}$,则
\[
\prod_{r=1}^{n} \cos\left(2^{r-1}x\right)
= \frac{\sin\left(2^{n}x\right)}{2^{n}\sin x}.
\]

\begin{solution}
先将乘积符号展开:
\[
\prod_{r=1}^{n} \cos\left(2^{r-1}x\right)
= \cos x \cos 2x \cos 4x \cos 8x \cdots \cos\left(2^{n-1}x\right).
\]

先验证基础情况。

当 $n=1$ 时,
\begin{align*}
\text{L.H.S.}
&= \prod_{r=1}^{1} \cos\left(2^{r-1}x\right)
= \cos\left(2^{0}x\right)
= \cos x, \\
\text{R.H.S.}
&= \frac{\sin\left(2^{1}x\right)}{2^{1}\sin x}
= \frac{\sin 2x}{2\sin x}
= \frac{2\sin x \cos x}{2\sin x}
= \cos x.
\end{align*}

左右两边相等,结论对 $n=1$ 成立。

假设当 $n=k$ 时结论成立,即
\[
\prod_{r=1}^{k} \cos\left(2^{r-1}x\right)
= \frac{\sin\left(2^{k}x\right)}{2^{k}\sin x}.
\]

考虑 $n=k+1$ 的情形,
\begin{align*}
\prod_{r=1}^{k+1} \cos\left(2^{r-1}x\right)
&= \left(\prod_{r=1}^{k} \cos\left(2^{r-1}x\right)\right)
\cos\left(2^{k}x\right) \\
&= \frac{\sin\left(2^{k}x\right)}{2^{k}\sin x}
\cos\left(2^{k}x\right) \\
&= \frac{\sin\left(2^{k}x\right)\cos\left(2^{k}x\right)}{2^{k}\sin x} \\
&= \frac{2\sin\left(2^{k}x\right)\cos\left(2^{k}x\right)}{2^{k+1}\sin x} \\
&= \frac{\sin\left(2^{k+1}x\right)}{2^{k+1}\sin x}.
\end{align*}

因此若结论对 $n=k$ 成立,则对 $n=k+1$ 也成立。

由于结论对 $n=1$ 成立,根据数学归纳法,结论对所有 $n \ge 1,\; n \in \mathbb{N}$ 成立。
\end{solution}

\question 用数学归纳法证明
\[
\cos x + \cos 3x + \cos 5x + \cdots + \cos[(2n-1)x]
= \frac{\sin(2nx)}{2\sin x}.
\]

\begin{solution}
先验证基础情况。

当 $n=1$ 时,
\begin{align*}
\text{L.H.S.} &= \cos(2\times 1-1)x = \cos x, \\
\text{R.H.S.} &= \frac{\sin(2\times 1 \, x)}{2\sin x}
= \frac{\sin(2x)}{2\sin x}
= \frac{2\sin x \cos x}{2\sin x}
= \cos x.
\end{align*}
因此结论对 $n=1$ 成立。

假设当 $n=k$ 时结论成立,即
\[
\sum_{r=1}^{k} \cos[(2r-1)x]
= \frac{\sin(2kx)}{2\sin x}.
\]

为进行归纳步骤,先推导一个恒等式。
由
\begin{align*}
\sin(A+B) &= \sin A \cos B + \cos A \sin B, \\
\sin(A-B) &= \sin A \cos B - \cos A \sin B,
\end{align*}
两式相加得
\[
2\sin A \cos B = \sin(A+B) + \sin(A-B).
\]

取 $A=x,\; B=(2k+1)x$,则
\begin{align*}
2\sin x \cos[(2k+1)x]
&= \sin[(2k+2)x] + \sin[-2kx] \\
&= \sin[(2k+2)x] - \sin(2kx).
\end{align*}

现在考虑 $n=k+1$:
\begin{align*}
\sum_{r=1}^{k+1} \cos[(2r-1)x]
&= \left(\sum_{r=1}^{k} \cos[(2r-1)x]\right)
+ \cos[(2k+1)x] \\
&= \frac{\sin(2kx)}{2\sin x} + \cos[(2k+1)x] \\
&= \frac{\sin(2kx) + 2\sin x \cos[(2k+1)x]}{2\sin x} \\
&= \frac{\sin(2kx) + \sin[(2k+2)x] - \sin(2kx)}{2\sin x} \\
&= \frac{\sin[(2k+2)x]}{2\sin x} \\
&= \frac{\sin[2(k+1)x]}{2\sin x}.
\end{align*}

因此若结论对 $n=k$ 成立,则对 $n=k+1$ 也成立。

由于结论对 $n=1$ 成立,根据数学归纳法,结论对所有 $n\in\mathbb{N}$ 成立。
\end{solution}

    \question 用数学归纳法证明:
\[
\frac{d^n}{dx^n}[e^x \cos x]
= 2^{\frac{n}{2}} e^x \cos\left(x + \frac{n\pi}{4}\right),
\quad n \ge 1,\; n \in \mathbb{N}.
\]

\begin{solution}
先验证基础情况。

当 $n=1$ 时,
\begin{align*}
\frac{d}{dx}[e^x \cos x]
&= e^x \cos x + e^x(-\sin x) \\
&= e^x(\cos x - \sin x).
\end{align*}

而右边为
\begin{align*}
2^{\frac{1}{2}} e^x \cos\left(x + \frac{\pi}{4}\right)
&= \sqrt{2} e^x \left(\cos x \cos\frac{\pi}{4} - \sin x \sin\frac{\pi}{4}\right) \\
&= \sqrt{2} e^x \left(\cos x \cdot \frac{1}{\sqrt{2}} - \sin x \cdot \frac{1}{\sqrt{2}}\right) \\
&= e^x(\cos x - \sin x).
\end{align*}

左右两边相等,结论对 $n=1$ 成立。

假设当 $n=k$ 时结论成立,即
\[
\frac{d^k}{dx^k}[e^x \cos x]
= 2^{\frac{k}{2}} e^x \cos\left(x + \frac{k\pi}{4}\right).
\]

考虑 $n=k+1$ 的情形,
\begin{align*}
\frac{d^{k+1}}{dx^{k+1}}[e^x \cos x]
&= \frac{d}{dx}\left[2^{\frac{k}{2}} e^x \cos\left(x + \frac{k\pi}{4}\right)\right] \\
&= 2^{\frac{k}{2}} \left(
e^x \cos\left(x + \frac{k\pi}{4}\right)
- e^x \sin\left(x + \frac{k\pi}{4}\right)
\right) \\
&= 2^{\frac{k}{2}} e^x
\left[
\cos\left(x + \frac{k\pi}{4}\right)
- \sin\left(x + \frac{k\pi}{4}\right)
\right].
\end{align*}

利用恒等式
\[
\cos A - \sin A = \sqrt{2}\cos\left(A + \frac{\pi}{4}\right),
\]
可得
\begin{align*}
\frac{d^{k+1}}{dx^{k+1}}[e^x \cos x]
&= 2^{\frac{k}{2}} e^x
\cdot \sqrt{2}
\cos\left(x + \frac{k\pi}{4} + \frac{\pi}{4}\right) \\
&= 2^{\frac{k+1}{2}} e^x
\cos\left(x + \frac{(k+1)\pi}{4}\right).
\end{align*}

因此若结论对 $n=k$ 成立,则对 $n=k+1$ 也成立。

由于结论对 $n=1$ 成立,根据数学归纳法,
结论对所有 $n \ge 1,\; n \in \mathbb{N}$ 成立。
\end{solution}

    \question 设 $n$ 为正整数。计算
\[
\int_{0}^{\pi/2} \frac{\sin [(2n + 1) x]}{\sin x} \,dx。
\]

\begin{solution}
令
\[
I_n = \int_{0}^{\pi/2} \frac{\sin [(2n+1)x]}{\sin x} \,dx。
\]

当 $n=1$ 时,
\begin{align*}
I_1 &= \int_0^{\pi/2} \frac{\sin 3x}{\sin x} \,dx \\
&= \int_0^{\pi/2} \frac{3 \sin x - 4 \sin^3 x}{\sin x} \,dx \\
&= \int_0^{\pi/2} (3 - 4 \sin^2 x) \,dx \\
&= \int_0^{\pi/2} [1 + 2 \cos(2x)] \,dx \\
&= [x + \sin(2x)]_0^{\pi/2} \\
&= \frac{\pi}{2}.
\end{align*}

假设结论对 $n-1$ 成立,考虑 $n$ 的情况。利用正弦的加角公式,
\begin{align*}
\int_0^{\pi/2} \frac{\sin (2n+1)x}{\sin x} \,dx 
&= \int_0^{\pi/2} \frac{\sin 2n x \cos x + \cos 2n x \sin x}{\sin x} \,dx \\
&= \int_0^{\pi/2} \frac{\sin 2n x \cos x}{\sin x} \,dx + \int_0^{\pi/2} \cos 2n x \,dx \\
&= \int_0^{\pi/2} \frac{\sin [(2n-1)x] \cos^2 x}{\sin x} \,dx + \int_0^{\pi/2} \cos x \cos [(2n-1)x] \,dx \\
&= \int_0^{\pi/2} \frac{\sin [(2n-1)x] (1 - \sin^2 x)}{\sin x} \,dx + \int_0^{\pi/2} \cos x \cos [(2n-1)x] \,dx \\
&= \int_0^{\pi/2} \frac{\sin [(2n-1)x]}{\sin x} \,dx - \int_0^{\pi/2} \sin x \sin [(2n-1)x] \,dx + \int_0^{\pi/2} \cos x \cos [(2n-1)x] \,dx \\
&= I_{n-1} - \frac{1}{2} \int_0^{\pi/2} (\cos [(2n-2)x] - \cos 2n x) \,dx + \frac{1}{2} \int_0^{\pi/2} (\cos [(2n-2)x] + \cos 2n x) \,dx \\
&= I_{n-1} + \int_0^{\pi/2} \cos (2n x) \,dx \\
&= \frac{\pi}{2}.
\end{align*}

由归纳法得
\[
\int_0^{\pi/2} \frac{\sin [(2n+1)x]}{\sin x} \,dx = \frac{\pi}{2} \quad \text{对所有正整数 $n$ 成立。}
\]
\end{solution}

\question 已知矩阵
\[
\mathbf{A}=\begin{pmatrix}1&0\\2&1\end{pmatrix},
\]
用数学归纳法证明
\[
\mathbf{A}^n=n\mathbf{A}-(n-1)\mathbf{I},\quad n\ge1,\ n\in\mathbb{N}.
\]

\begin{solution}
先验证基础情况。

当 $n=1$ 时,
\[
\mathbf{A}^1=1\cdot\mathbf{A}-(1-1)\mathbf{I}=\mathbf{A},
\]
结论成立。

假设当 $n=k$ 时结论成立,即
\[
\mathbf{A}^k=k\mathbf{A}-(k-1)\mathbf{I}.
\]

则
\begin{align*}
\mathbf{A}^{k+1}
&=\mathbf{A}^k\mathbf{A} \\
&=[k\mathbf{A}-(k-1)\mathbf{I}]\mathbf{A} \\
&=k\mathbf{A}^2-(k-1)\mathbf{A}.
\end{align*}

为继续化简,先将 $\mathbf{A}^2$ 表示成 $\mathbf{A}$ 与 $\mathbf{I}$ 的线性组合。
设
\[
\mathbf{A}^2=\lambda\mathbf{A}+\mu\mathbf{I}.
\]
则
\[
\begin{pmatrix}1&0\\2&1\end{pmatrix}
\begin{pmatrix}1&0\\2&1\end{pmatrix}
=
\lambda\begin{pmatrix}1&0\\2&1\end{pmatrix}
+\mu\begin{pmatrix}1&0\\0&1\end{pmatrix}.
\]
即
\[
\begin{pmatrix}1&0\\4&1\end{pmatrix}
=
\begin{pmatrix}\lambda+\mu&0\\2\lambda&\lambda+\mu\end{pmatrix}.
\]

比较对应元素得
\begin{align*}
2\lambda&=4 \Rightarrow \lambda=2,\\
\lambda+\mu&=1 \Rightarrow \mu=-1.
\end{align*}
因此
\[
\mathbf{A}^2=2\mathbf{A}-\mathbf{I}.
\]

代回得
\begin{align*}
\mathbf{A}^{k+1}
&=k(2\mathbf{A}-\mathbf{I})-(k-1)\mathbf{A} \\
&=2k\mathbf{A}-k\mathbf{I}-k\mathbf{A}+\mathbf{A} \\
&=(k+1)\mathbf{A}-k\mathbf{I} \\
&=(k+1)\mathbf{A}-[(k+1)-1]\mathbf{I}.
\end{align*}

因此若结论对 $n=k$ 成立,则对 $n=k+1$ 也成立。

由于结论对 $n=1$ 成立,根据数学归纳法,结论对所有 $n\ge1,\ n\in\mathbb{N}$ 成立。
\end{solution}

    \question 设 $V_n$ 为 $n$ 阶范德蒙行列式,定义如下:
    \[
    V_n = \begin{vmatrix}
    1 & x_1 & x_1^2 & \dots & x_1^{n-1} \\
    1 & x_2 & x_2^2 & \dots & x_2^{n-1} \\
    \vdots & \vdots & \vdots & \ddots & \vdots \\
    1 & x_n & x_n^2 & \dots & x_n^{n-1}
    \end{vmatrix}
    \]
    试以数学归纳法证明
    \[
    V_n = \prod_{1 \le i < j \le n} (x_j - x_i).
    \]
    \begin{solution}
        定义命题
        \[
        P_n:V_n = 
        \begin{vmatrix}
        1 & x_1 & x_1^2 & \dots & x_1^{n-1} \\
        1 & x_2 & x_2^2 & \dots & x_2^{n-1} \\
        \vdots & \vdots & \vdots & \ddots & \vdots \\
        1 & x_n & x_n^2 & \dots & x_n^{n-1}
        \end{vmatrix}= \prod_{1 \le i < j \le n} (x_j - x_i)\quad,n \in \mathbb{N}
        \]
        基础步骤:显然$P_1$成立,因$V=1=|1|=1$,且显然$P_2$亦成立,因为
        \[
        V_2=
        \begin{vmatrix} 
        1 & x_1 \\
        1 & x_2 
        \end{vmatrix} 
        =x_2-x_1
        \]
        归纳步骤:假设命题$P_k$成立,即
        \[
        V_k = 
        \begin{vmatrix}
        1 & x_1 & x_1^2 & \dots & x_1^{k-1} \\
        1 & x_2 & x_2^2 & \dots & x_2^{k-1} \\
        \vdots & \vdots & \vdots & \ddots & \vdots \\
        1 & x_k & x_k^2 & \dots & x_k^{k-1}
        \end{vmatrix}= \prod_{1 \le i < j \le k} (x_j - x_i)
        \]
        观察$V_{k+1}$,将 $x_1$写成变量$x$,即
        \[
        V_{k+1} = \begin{vmatrix}
        1 & x & x^2 & \dots & x^{k} \\
        1 & x_2 & x_2^2 & \dots & x_2^{k} \\
        \vdots & \vdots & \vdots & \ddots & \vdots \\
        1 & x_k & x_k^2 & \dots & x_k^{k} \\
        1 & x_{k+1} & x_{k+1}^2 & \dots & x_{k+1}^{k}
        \end{vmatrix}
        \]
        $V_{k+1}$按$R_1$展开后是次数不大于$k$的多项式,设$f(x)=V_{k+1}$,发现当$x=x_2,x_3,\cdots,x_{k+1}$,行列式$V_{k+1}$有相同的两行,故
        \[
        f(x_2)=f(x_3)=\;\dots\;=f(x_{k+1})=0 
        \]
        故 $f(x)$ 是一个次数为 $k$且以 $x_2,x_3,\cdot,x_{k+1}$ 为根的多项式,由余氏定理,
        \[
        f(x) = C(x-x_2)(x-x_3)\cdots(x-x_{k+1}) \tag{1}
        \]
        且$C$是一常数。又$V_{k+1}$按$R_1$展开式中$x^k$的系数为
        \[
        \begin{vmatrix}
        1 & x_1 & x_1^2 & \dots & x_1^{k-1} \\
        1 & x_2 & x_2^2 & \dots & x_2^{k-1} \\
        \vdots & \vdots & \vdots & \ddots & \vdots \\
        1 & x_k & x_k^2 & \dots & x_k^{k-1}
        \end{vmatrix}
        \]
        由归纳假设知此系数为
        \[
        \prod_{2 \le i < j \le k+1} (x_j - x_i)
        \]
        故由$(1)$知
        \[
        C=\prod_{2 \le i < j \le k+1} (x_j - x_i)
        \]
        即
        \[
        f(x)=(x-x_2)(x-x_3)\cdots(x-x_{k+1})\prod_{2 \le i < j \le k+1} (x_j - x_i)
        \]
        将$x$置换回$x_1$,即得命题$P_{k+1}$成立,此时蕴含$P_k \implies P_{k+1}$成立。
        
        由数学归纳法,命题$P_n$对所有正整数$n$皆成立。
        \textcolor{red}{验证是$x_j - x_i$还是$x_i - x_j$}
    \end{solution}

    \question 试证莱布尼茨公式:设函数 $f,g$ 定义在开区间 $I$ 上,$n$ 为正整数,$x \in I$ 为$I$内一点使得 $f,g$ 均可导 $n$ 次,则有
    \[
    (f(x)g(x))^{(n)} = \sum_{k=0}^{n} \binom{n}{k} f^{(k)}(x) g^{(n-k)}(x),
    \]
    其中 $(n)$ 表示导数的阶数。
    \begin{solution}
        定义命题
        \[
        P_n: (f(x)g(x))^{(n)} = \sum_{k=0}^{n} \binom{n}{k} f^{(k)}(x) g^{(n-k)}(x) \quad,n \in \mathbb{N}
        \]
        基础步骤:观察当$n=1$时,由乘积求导法则,左式为
        \[
        (f(x)g(x))' = f(x)g'(x) + f'(x)g(x)
        \]
        而右式为
        \[
        \sum_{k=0}^{1} \binom{1}{k} f^{(k)}(x) g^{(1-k)}(x)
        = \binom{1}{0} f^{(0)}(x) g^{(1)}(x) + \binom{1}{1} f^{(1)}(x) g^{(0)}(x)
        = f(x)g'(x) + f'(x)g(x),
        \]
        此时左式等于右式,即$P_1$成立。

        归纳步骤:设 $n \in \mathbb{N}$,假设归纳假设$P_n$成立,有
        \[
        (f(x)g(x))^{(n)} = \sum_{k=0}^{n} \binom{n}{k} f^{(k)}(x) g^{(n-k)}(x)
        \]
        欲证
        \[
        (f(x)g(x))^{(n+1)} = \sum_{k=0}^{n+1} \binom{n+1}{k} f^{(k)}(x) g^{(n+1-k)}(x)
        \]
        由归纳假设,
        \begin{align*}
        (f(x)g(x))^{(n+1)}
        &= \left( \sum_{k=0}^{n} \binom{n}{k} f^{(k)}(x) g^{(n-k)}(x) \right)' \\
        &= \sum_{k=0}^{n} \binom{n}{k} \left(f^{(k)}(x) g^{(n-k)}(x) \right)' \\
        &= \sum_{k=0}^{n} \binom{n}{k} \left( f^{(k+1)}(x) g^{(n-k)}(x) + f^{(k)}(x) g^{(n-k+1)}(x) \right) \\
        &= \sum_{k=0}^{n} \binom{n}{k} f^{(k+1)}(x) g^{(n-k)}(x)
        + \sum_{k=0}^{n} \binom{n}{k} f^{(k)}(x) g^{(n-k+1)}(x)
        \end{align*}
        将第一个求和中 $k=n$ 的项分离,第二个求和中 $k=0$ 的项分离,得
        \begin{align*}
        (f(x)g(x))^{(n+1)} 
        &= \sum_{k=0}^{n-1} \binom{n}{k} f^{(k+1)}(x) g^{(n-k)}(x)
        + \sum_{k=1}^{n} \binom{n}{k} f^{(k)}(x) g^{(n-k+1)}(x) \\
        &\quad + \binom{n}{n} f^{(n+1)}(x) g^{(0)}(x)
        + \binom{n}{0} f^{(0)}(x) g^{(n+1)}(x) \\
        &= \sum_{k=1}^{n-1} \binom{n}{k-1} f^{(k)}(x) g^{(n+1-k)}(x)
        + \sum_{k=1}^{n} \binom{n}{k} f^{(k)}(x) g^{(n-k+1)}(x) \\
        &\quad + \binom{n}{n} f^{(n+1)}(x) g^{(0)}(x)
        + \binom{n}{0} f^{(0)}(x) g^{(n+1)}(x)
        \end{align*}
        由帕斯卡恒等式,
        \begin{align*}
        &(f(x) g(x))^{(n+1)} \\
        &= \sum_{k=1}^{n} \binom{n+1}{k} f^{(k)}(x) g^{(n+1-k)}(x) + \binom{n}{0} f^{(0)}(x) g^{(n+1)}(x) + \binom{n}{n} f^{(n+1)}(x) g^{(0)}(x) \\
        &= \sum_{k=1}^{n} \binom{n+1}{k} f^{(k)}(x) g^{(n+1-k)}(x) + \binom{n+1}{0} f(x) g^{(n+1)}(x) + \binom{n+1}{n+1} f^{(n+1)}(x) g(x) \\
        &= \sum_{k=0}^{n+1} \binom{n+1}{k} f^{(k)}(x) g^{(n+1-k)}(x)
        \end{align*}
        即命题$P_{n+1}$成立,此时蕴含$P_n \implies P_{n+1}$成立。
        
        由数学归纳法,命题$P_n$对所有正整数$n$皆成立。
    \end{solution}

    \question 设 $a_0, a_1, a_2, \dots$ 为一无限实数数列, 且满足 $$\frac{a_{n-1}+a_{n+1}}{2} \ge a_n, \forall \,n\in \mathbb{N}$$ 证明 $$\frac{a_0+a_{n+1}}{2} \ge \frac{a_1+a_2+\dots+a_n}{n},\forall \,n\in \mathbb{N}$$
    \begin{solution}
        \textbf{解法一}
        
        先证明:
        \[
        a_{0}+na_{n+1}\ge(n+1)a_{n}\tag{*}
        \]
        当$n=1$时, $a_{0}+a_{2}\ge2a_{1}$,显然(*)成立;假设$n=k$时(*)成立, 即
        \[
        a_{0}+ka_{k+1}\ge(k+1)a_{k}\tag{1}
        \]
        又
        \[
        a_{k+2}+a_{k}\ge2a_{k+1} \tag{2}
        \]
        由$(1),(2)$得$$a_{0}+(k+1)a_{k+2}\ge (k+1)(a_{k+2}+a_k)-ka_{k+1}\ge2(k+1)a_{k+1}-ka_{k+1}=(k+2)a_{k+1}$$即$n=k+1$时(*)也成立,由数学归纳法知(*)对所有$n\in \mathbb{N}$均成立。
        
        现证:
        \[
        n(a_{n}+a_{n+1})\ge2\sum_{i=1}^{n}a_{i}\tag{**}
        \]
        当$n=1$时,$\ a_{0}+a_{2}\ge2a_{1}$,(**)成立;假设$n=k$时(**)成立,即 
        \[
        k(a_{0}+a_{k+1})\ge2\sum_{i=1}^{k}a_{i}\tag{3}
        \]
        由$(*)$知
        \[
        a_{0}+(k+1)a_{k+2}\ge(k+2)a_{k+1}\tag{4}
        \]
        由$(3),(4)$得$$(k+1)(a_{0}+a_{k+2})\ge(k+2)a_{k+1}+2\sum_{i=1}^{k}a_{i}-ka_{k+1}=2\sum_{i=1}^{k+1}a_{i}$$ 即$n=k+1$时(**)也成立,由数学归纳法知(**)对所有$n\in \mathbb{N}$均成立
        即$$\frac{a_{0}+a_{n+1}}{2}\ge\frac{a_{1}+a_{2}+\cdot\cdot\cdot+a_{n}}{n},\forall \,n\in \mathbb{N}$$
        证毕。
    \end{solution}
    \begin{solution}
        \textbf{解法二}
        
        设\(a_{n+1}-a_{n}=d_{n}\),由题意可知
        \[
        a_{n+1}-a_{n}\ge a_{n}-a_{n-1}\Rightarrow d_{n}\ge d_{n-1}
        \]
        且 $a_{n}=a_{0}+d_{1}+d_{2}+\cdot\cdot\cdot+d_{n-1}$,欲证等价于
        \begin{align*}
        &\frac{a_{0}+a_{n+1}}{2}\ge\frac{a_{1}+a_{2}+\cdot\cdot\cdot+a_{n}}{n}\\
        \Longleftrightarrow \ &n(a_{0}+a_{n+1})\ge2(a_{1}+a_{2}+\cdot\cdot\cdot+a_{n})\\
        \Longleftrightarrow \ &na_{0}+n(a_{0}+d_{0}+d_{1}+\cdot\cdot\cdot+d_{n-1})\\
        &\ge 2[(a_{0}+d_{0})+(a_{0}+d_{0}+d_{1})+\cdot\cdot\cdot+(a_{0}+d_{0}+d_{1}+\cdot\cdot\cdot+d_{n-1})]\\
        \Longleftrightarrow \ &n d_{n}\ge n d_{0}+(n-2)d_{1}+(n-4)d_{2}+(n-6)d_{3}+\cdot\cdot\cdot+(4-n)d_{n-2}+(2-n)d_{n-1}
        \end{align*} 
        当$n$为奇数,即证 \[n(d_{0}-d_{n})+(n-2)(d_{1}-d_{n-1})+(n-4)(d_{2}-d_{n-2})+\cdot\cdot\cdot+(d_{\frac{n-1}{2}}-d_{\frac{n+1}{2}})\le0 \tag{1}\]
        而 $d_{n}\ge d_{n-1}$, 则$$d_{0}-d_{n}\le0, d_{1}-d_{n-1}\le0, \cdot\cdot\cdot, d_{\frac{n-1}{2}}-d_{\frac{n+1}{2}}\le0$$ 此时(1)显然成立。
        
        当$n$为偶数,即证 \[n(d_{0}-d_{n})+(n-2)(d_{1}-d_{n-1})+(n-4)(d_{2}-d_{n-2})+\cdot\cdot\cdot+2(d_{\frac{n-2}{2}}-d_{\frac{n+2}{2}})\le0\tag{2}\]
        上式中$d_{\frac{n}{2}}$这一项没有,而$d_{n}\ge d_{n-1}$
        则$$d_{0}-d_{n}\le0, d_{1}-d_{n-1}\le0,\cdot\cdot\cdot,d_{\frac{n-2}{2}}-d_{\frac{n+2}{2}}\le0$$ 此时(2)也成立。
        
        故得证 $$\frac{a_0+a_{n+1}}{2} \ge \frac{a_1+a_2+\dots+a_n}{n},\forall \,n\in \mathbb{N}$$
    \end{solution}

    \question 已知多项式 $P(x) = x^2 + 2019x + 1$,试证对任意正整数 $n$,函数 $P^{(n)}(x) = 0$ 至少有一个实数根,其中 $P^{(n)}(x)$ 表示 $P$ 自身复合 $n$ 次。
    \begin{solution}
        定义命题:对任意正整数 $n,P^{(n)}(x)$ 有一个负实根 $-c_n$,且 $0 < c_n < 2$。
        
        当 $n=1$ 时,
        \[
        P^{(1)}(x) = P(x) = x^2 + 2019x + 1
        \]
        其判别式为 $\Delta_1 = 2019^2 - 4 > 0$,所以 $P^{(1)}(x)$ 有两个实数根,设为 $-b_1, -c_1$,其中 $c_1 \le b_1$,由韦达定理,
        \[
        b_1 + c_1 = 2019,\quad b_1 c_1 = 1
        \]
        因此 $b_1, c_1$ 都为正数,且因为乘积是 $1$,可得 $c_1 \le 1 < 2$,所以 $-c_1$ 是一个负实根,满足 $0 < c_1 < 2$,故命题在$n=1$ 时成立。
        
        假设命题在$n=k$ 时成立,即 $P^{(k)}(x)$ 有一负实根 $-c_k$,其中 $0 < c_k < 2$,即
        \[
        P^{(k)}(x) = (x + c_k) Q_k(x)
        \]
        其中$Q_k(x)$为某多项式 ,则有
        \[
        P^{(k+1)}(x) = P^{(k)}(P(x)) = (P(x) + c_k) Q_k(P(x))
        \]
        其中
        \[
        P(x) + c_k = x^2 + 2019x + 1 + c_k
        \]
        的判别式为
        \[
        \Delta_{k+1} = 2019^2 - 4(1 + c_k) > 0
        \]
        即$P(x) + c_k$有两个实根,设为 $-b_{k+1}, -c_{k+1}$,其中 $c_{k+1} \le b_{k+1}$,由韦达定理,
        \[
        b_{k+1} + c_{k+1} = 2019,\quad b_{k+1} c_{k+1} = 1 + c_k > 0
        \]
        可知 $b_{k+1}, c_{k+1}$ 都为正数,且
        \[
        c_{k+1} \le \sqrt{1 + c_k} < \sqrt{3} < 2
        \]
        所以 $-c_{k+1}$ 是 $P^{(k+1)}(x)$ 的负实根,满足 $0 < c_{k+1} < 2$,故命题在$n=k+1$ 时也成立。
        
        由数学归纳法知,对任意正整数 $n,P^{(n)}(x)$ 至少有一个实数根。
    \end{solution}
\end{questions}
\pagebreak

\begin{center}
  {\fontsize{30pt}{26pt}\selectfont
    \hypertarget{行列式、矩阵}{行列式、矩阵} \label{行列式、矩阵}
  }
\end{center}
\separator
\vspace{1pt}

\begin{questions}
    \question 求下列行列式的立方根:
    \[ 
    \begin{vmatrix}
    -\dfrac{2019^2}{\sqrt[5]{80}} & \dfrac{2020^2}{\sqrt[5]{80^2}} & -\dfrac{2021^2}{\sqrt[5]{80^3}} \\[8pt]
    \dfrac{2022^2}{\sqrt[5]{80^4}} & -\dfrac{2023^2}{\sqrt[5]{80^5}} & \dfrac{2024^2}{\sqrt[5]{80^6}} \\[8pt]
    -\dfrac{2025^2}{\sqrt[5]{80^7}} & \dfrac{2026^2}{\sqrt[5]{80^8}} & -\dfrac{2027^2}{\sqrt[5]{80^9}}
    \end{vmatrix}
    \]
    \begin{solution}
        设$a=2019,r=-\dfrac{1}{\sqrt[5]{80}}$,原行列式为
        \[
        D=
        \begin{vmatrix}
        a^2 r & (a+1)^2 r^2 & (a+2)^2 r^3 \\
        (a+3)^2 r^4 & (a+4)^2 r^5 & (a+5)^2 r^6 \\
        (a+6)^2 r^7 & (a+7)^2 r^8 & (a+8)^2 r^9
        \end{vmatrix}
        \]
        进行列变换$C_2 \rightarrow C_2-C_1,C_3 \rightarrow C_3-C_2$,
        \[
        D=\begin{vmatrix}
        a^2 r & (2a+1) r^2 & (2a+3) r^3 \\
        (a+3)^2 r^4 & (2a+7) r^5 & (2a+9) r^6 \\
        (a+6)^2 r^7 & (2a+13) r^8 & (2a+15) r^9
        \end{vmatrix}
        \]
        进行列变换$C_3 \rightarrow C_3-C_2$,
        \[
        D=\begin{vmatrix}
        a^2 r & (2a+1) r^2 & 2r^3 \\
        (a+3)^2 r^4 & (2a+7) r^5 & 2r^6 \\
        (a+6)^2 r^7 & (2a+13) r^8 & 2r^9
        \end{vmatrix}
        \]
        进行行变换$R_3 \rightarrow R_3-r^3R_2,R_2 \rightarrow R_2-r^3R_1$
        \[
        D=\begin{vmatrix}
        a^2 r & (2a+1) r^2 & 2r^3 \\
        (6a+9) r^4 & 6 r^5 & 0 \\
        (6a+27) r^7 & 6 r^8 & 0
        \end{vmatrix}
        \]
        进行行变换$R_3 \rightarrow R_3-r^3R_2$,
        \[
        D=\begin{vmatrix}
        a^2 r & (2a+1) r^2 & 2r^3 \\
        (6a+9) r^4 & 6 r^5 & 0 \\
        18 r^7 & 0 & 0
        \end{vmatrix}
        \]
        进行列变换$C_1 \leftrightarrow C_3$,
        \[
        D=-\begin{vmatrix}
        2r^3 & (2a+1) r^2 & a^2 r \\
        0 & 6 r^5 & (6a+9) r^4 \\
        0 & 0 & 18 r^7
        \end{vmatrix}
        =-(2r^3)(6r^5)(18r^7)=-216r^{15}=\frac{216}{80^3}
        \]
        故原行列式的立方根为$\dfrac{3}{40}$
    \end{solution}

    \question 若 $a+b+c=x+y+z=0$,求行列式 
    \[
    \begin{vmatrix}
    xa & yb & zc \\
    yc & za & xb \\
    zb & xc & ya
    \end{vmatrix}
    \]
    \begin{solution}
        展开行列式得
        \[
        \begin{vmatrix}
        xa & yb & zc \\
        yc & za & xb \\
        zb & xc & ya
        \end{vmatrix}=xyz(a^3+b^3+c^3)+abc(x^3+y^3+z^3)
        \]由于
        \[
        a^3+b^3+c^3-3abc=(a+b+c)(a^2+b^2+c^2-ab-bc-ca)=0
        \] 
        \[
        x^3+y^3+z^3-3xyz=(x+y+z)(x^2+y^2+z^2-xy-yz-zx)=0
        \]
        故
        \[
        \begin{vmatrix}
        xa & yb & zc \\
        yc & za & xb \\
        zb & xc & ya
        \end{vmatrix}=xyz(3abc)-abc(3xyz)=0\]
    \end{solution}

    \question 计算行列式
\[
\begin{vmatrix} 
\log a & \log b & \log \frac{1}{ab} \\ 
\log b & \log c & \log \frac{1}{bc} \\ 
\log c & \log a & \log \frac{1}{ca} 
\end{vmatrix}.
\]

\begin{solution}
将乘积写成和:
\[
\begin{vmatrix} 
\log a & \log b & -\log a - \log b \\ 
\log b & \log c & -\log b - \log c \\ 
\log c & \log a & -\log c - \log a 
\end{vmatrix}.
\]

对第三列进行变换 $C_3 \leftarrow C_3 - C_1 - C_2$,得到
\[
\begin{vmatrix} 
\log a & \log b & 0 \\ 
\log b & \log c & 0 \\ 
\log c & \log a & 0 
\end{vmatrix} = 0.
\]
因此该行列式的值为 0。
\end{solution}


    \question  若 $m$为正整数,求行列式 \[
        \begin{vmatrix}
        1 & 1 & 1 \\
        \comb{m}{1} & \comb{m+1}{1} & \comb{m+2}{1} \\
        \comb{m}{2} & \comb{m+1}{2} & \comb{m+2}{2}
        \end{vmatrix}
        \]
    \begin{solution}
        令\[
        D=\begin{vmatrix}
        1 & 1 & 1 \\
        \comb{m}{1} & \comb{m+1}{1} & \comb{m+2}{1} \\
        \comb{m}{2} & \comb{m+1}{2} & \comb{m+2}{2}
        \end{vmatrix}
        \]
        由性质 $\comb{n-1}{k-1}+\comb{n-1}{k}=\comb{n}{k}$,有\[
        D=\begin{vmatrix}
        1 & 1 & 1 \\
        \comb{m}{1} & \comb{m+1}{1} & \comb{m+1}{0}+\comb{m+1}{1} \\
        \comb{m}{2} & \comb{m+1}{2} & \comb{m+1}{1}+\comb{m+1}{2}
        \end{vmatrix} 
        \]
        进行列变换$C_3\rightarrow C_3-C_2,$\[
        D=\begin{vmatrix}
        1 & 1 & 0 \\
        \comb{m}{1} & \comb{m+1}{1} & \comb{m+1}{0} \\
        \comb{m}{2} & \comb{m+1}{2} & \comb{m+1}{1}
        \end{vmatrix} 
        \]
        再由性质 $\comb{n-1}{k-1}+\comb{n-1}{k}=\comb{n}{k}$,\[
        D=\begin{vmatrix}
        1 & 1 & 0 \\
        \comb{m}{1} & \comb{m}{0}+\comb{m}{1} & \comb{m+1}{0} \\
        \comb{m}{2} & \comb{m}{1}+\comb{m}{2} & \comb{m+1}{1}
        \end{vmatrix} 
        \]
        进行列变换$C_2\rightarrow C_2-C_1,$\[
        D=\begin{vmatrix}
        1 & 0 & 0 \\
        \comb{m}{1} & \comb{m}{0} & \comb{m+1}{0} \\
        \comb{m}{2} & \comb{m}{1} & \comb{m+1}{1}
        \end{vmatrix} 
        \]
        按第一行展开得\[
        D= \comb{m}{0}\comb{m+1}{1}-\comb{m}{1}\comb{m+1}{0}=1\cdot(m+1)-m\cdot1=1
        \]
    \end{solution}

    \question 已知 $\triangle ABC$,求  
\[
\begin{vmatrix} 
\tan A & 1 & 1 \\ 
1 & \tan B & 1 \\ 
1 & 1 & \tan C 
\end{vmatrix}.
\]

\begin{solution}
展开行列式:
\[
\begin{vmatrix} 
\tan A & 1 & 1 \\ 
1 & \tan B & 1 \\ 
1 & 1 & \tan C 
\end{vmatrix} = \tan A \tan B \tan C - (\tan A + \tan B + \tan C) + 2.
\]

由于 $A + B + C = \pi$,有
\[
\tan(A + B) = \tan(\pi - C) = -\tan C,
\]
又
\[
\tan(A + B) = \frac{\tan A + \tan B}{1 - \tan A \tan B}.
\]

因此
\[
\frac{\tan A + \tan B}{1 - \tan A \tan B} = -\tan C \implies \tan A + \tan B = -\tan C + \tan A \tan B \tan C \implies \tan A + \tan B + \tan C = \tan A \tan B \tan C.
\]

将其代入行列式表达式,得到
\[
\begin{vmatrix} 
\tan A & 1 & 1 \\ 
1 & \tan B & 1 \\ 
1 & 1 & \tan C 
\end{vmatrix} = \tan A \tan B \tan C - (\tan A + \tan B + \tan C) + 2 = 2.
\]
\end{solution}

    \question 计算行列式
    \[
    \begin{vmatrix}
    \tan 40^{\circ} & \tan 10^{\circ} & \tan 50^{\circ} \\
    \tan 20^{\circ} & \tan 50^{\circ} & \tan 70^{\circ} \\
    \tan 10^{\circ} & \tan 70^{\circ} & \tan 80^{\circ}
    \end{vmatrix}
    \]
    \begin{solution}
        令
        \[
        D=
        \begin{vmatrix}
        \tan 40^\circ & \tan 10^\circ & \tan 50^\circ \\
        \tan 20^\circ & \tan 50^\circ & \tan 70^\circ \\
        \tan 10^\circ & \tan 70^\circ & \tan 80^\circ
        \end{vmatrix}
        \]
        进行列变换 $C_3 \to -C_1 - C_2 + C_3 $,
        \[
        D=
        \begin{vmatrix}
        \tan 40^\circ & \tan 10^\circ & -\tan 10^\circ - \tan 40^\circ + \tan 50^\circ \\
        \tan 20^\circ & \tan 50^\circ & -\tan 20^\circ - \tan 50^\circ + \tan 70^\circ \\
        \tan 10^\circ & \tan 70^\circ & -\tan 10^\circ - \tan 70^\circ + \tan 80^\circ
        \end{vmatrix}
        \]
        由
        \[
        \tan(A+B) = \frac{\tan A + \tan B}{1 - \tan A \tan B} 
        \]
        可得
        \[
        \tan A \tan B \tan (A+B) = \tan(A+B) - \tan A - \tan B
        \]
        且若 $A + B = 90^\circ$,则 
        \[
        \tan A \tan B = 1
        \]
        因此
        \[
        D=\begin{vmatrix}
        \tan 40^\circ & \tan 10^\circ & \tan 10^\circ \cdot \tan 40^\circ \cdot \tan 50^\circ \\
        \tan 20^\circ & \tan 50^\circ & \tan 20^\circ \cdot \tan 50^\circ \cdot \tan 70^\circ \\
        \tan 10^\circ & \tan 70^\circ & \tan 10^\circ \cdot \tan 70^\circ \cdot \tan 80^\circ
        \end{vmatrix}
        =
        \begin{vmatrix}
        \tan 40^\circ & \tan 10^\circ & \tan 10^\circ \\
        \tan 20^\circ & \tan 50^\circ & \tan 50^\circ \\
        \tan 10^\circ & \tan 70^\circ & \tan 70^\circ
        \end{vmatrix}
        =0
        \]
    \end{solution}

    \question 已知 $\triangle ABC$,计算行列式
\[
\begin{vmatrix} 
a & a^2 & \cos A \\ 
b & b^2 & \cos B \\ 
c & c^2 & \cos C 
\end{vmatrix}.
\]

\begin{solution}
利用余弦定理 \(\cos A = \frac{b^2+c^2-a^2}{2bc}\) 等,代入行列式:
\[
\begin{vmatrix} 
a & a^2 & \frac{b^2+c^2-a^2}{2bc} \\ 
b & b^2 & \frac{a^2+c^2-b^2}{2ac} \\ 
c & c^2 & \frac{a^2+b^2-c^2}{2ab} 
\end{vmatrix}.
\]

提取公因子 \(\frac{1}{2abc}\):
\[
= \frac{1}{2abc} 
\begin{vmatrix} 
a & a^2 & a(b^2+c^2-a^2) \\ 
b & b^2 & b(a^2+c^2-b^2) \\ 
c & c^2 & c(a^2+b^2-c^2) 
\end{vmatrix}.
\]

提取 \(a,b,c\):
\[
= \frac{1}{2} 
\begin{vmatrix} 
1 & a & b^2+c^2-a^2 \\ 
1 & b & a^2+c^2-b^2 \\ 
1 & c & a^2+b^2-c^2 
\end{vmatrix}.
\]

行变换:
\[
R_2 \leftarrow R_2-R_1, \quad R_3 \leftarrow R_3-R_1
\implies
\frac{1}{2} 
\begin{vmatrix} 
1 & a & b^2+c^2-a^2 \\ 
0 & b-a & 2a^2-2b^2 \\ 
0 & c-a & 2a^2-2c^2 
\end{vmatrix}.
\]

按第一列展开:
\[
= \frac{1}{2} 
\begin{vmatrix} 
b-a & 2a^2-2b^2 \\ 
c-a & 2a^2-2c^2 
\end{vmatrix}
= 
\begin{vmatrix} 
b-a & (a-b)(a+b) \\ 
c-a & (a-c)(a+c) 
\end{vmatrix}.
\]

提取公因子 \((b-a)(c-a)\):
\[
= (b-a)(c-a) 
\begin{vmatrix} 
1 & -(a+b) \\ 
1 & -(a+c) 
\end{vmatrix} 
= -(b-a)(c-a) 
\begin{vmatrix} 
1 & a+b \\ 
1 & a+c 
\end{vmatrix}.
\]

计算 2×2 行列式:
\[
= -(b-a)(c-a) ((a+c)-(a+b)) = -(b-a)(c-a)(c-b) = (a-b)(a-c)(b-c).
\]

最终结果:
\[
\begin{vmatrix} 
a & a^2 & \cos A \\ 
b & b^2 & \cos B \\ 
c & c^2 & \cos C 
\end{vmatrix} = (a-b)(a-c)(b-c).
\]
\end{solution}


    \question 在坐标平面上,设 $\triangle ABC$ 经二阶方阵 $M = \begin{bmatrix} a & b \\ c & d \end{bmatrix}$ 作线性变换后成 $\triangle A'B'C'$。若 $\triangle ABC$ 的面积为 $\triangle,\triangle A'B'C'$ 的面积为 $\triangle'$,试证$$\triangle' =  \begin{vmatrix} a & b \\ c & d \end{vmatrix} \cdot \triangle$$
    \begin{solution}
        设$A(x_a, y_a),B(x_b, y_b),C(x_c, y_c)$
        经变换后为
        \[
        A'(a x_a + b y_a,\; c x_a + d y_a),B'(a x_b + b y_b,\; c x_b + d y_b),C'(a x_c + b y_c,\; c x_c + d y_c)
        \]
        则
        \[
        \Delta' = \frac{1}{2}
        \begin{vmatrix}
        a x_a + b y_a & c x_a + d y_a & 1 \\
        a x_b + b y_b & c x_b + d y_b & 1 \\
        a x_c + b y_c & c x_c + d y_c & 1
        \end{vmatrix}
        \]
        变为
        \[
        \Delta' = \frac{1}{2} \begin{vmatrix}
        a x_a & d y_a & 1 \\
        a x_b & d y_b & 1 \\
        a x_c & d y_c & 1
        \end{vmatrix}
        +
        \frac{1}{2} \begin{vmatrix}
        b y_a & c x_a & 1 \\
        b y_b & c x_b & 1 \\
        b y_c & c x_c & 1
        \end{vmatrix} +0+0
        = \frac{1}{2}(ad-bc)
        \begin{vmatrix}
        x_a & y_a & 1 \\
        x_b & y_b & 1 \\
        x_c & y_c & 1
        \end{vmatrix}
        = \begin{vmatrix} a & b \\ c & d \end{vmatrix} \cdot \triangle
        \]
    \end{solution}

    \question 若 $a^2+b^2+c^2=47$, 求
    \[
    \begin{vmatrix}
    a^2-1 & ab & ca \\
    ab & b^2-1 & bc \\
    ca & bc & c^2-1
    \end{vmatrix}
    \]
    \begin{solution}
        令
        \[
        D=\begin{vmatrix}
        a^2-1 & ab & ca \\
        ab & b^2-1 & bc \\
        ca & bc & c^2-1
        \end{vmatrix}
        \]
        进行行变换$R_1\to aR_1, R_2\to bR_2, R_3\to cR_3$,
        \[
        D=\frac{1}{abc}\begin{vmatrix}
        a(a^2-1) & a^2b & ca^2 \\
        ab^2 & b(b^2-1) & b^2c \\
        c^2a & bc^2 & c(c^2-1)
        \end{vmatrix}
        \]
        进行列变换$C_1\to \dfrac{C_1}{a}, C_2\to \dfrac{C_2}{b}, C_3\to \dfrac{C_3}{c}$,
        \[
        D=\begin{vmatrix}
        a^2-1 & a^2 & a^2 \\
        b^2 & b^2-1 & b^2 \\
        c^2 & c^2 & c^2-1
        \end{vmatrix}
        \]
        进行行变换 $R_1\to R_1+R_2+R_3$,
        \begin{align*}
            D&=\begin{vmatrix}
            a^2+b^2+c^2-1 & a^2+b^2+c^2-1 & a^2+b^2+c^2-1 \\
            b^2 & b^2-1 & b^2 \\
            c^2 & c^2 & c^2-1
            \end{vmatrix}\\
            &=(a^2+b^2+c^2-1)
            \begin{vmatrix}
            1 & 1 & 1 \\
            b^2 & b^2-1 & b^2 \\
            c^2 & c^2 & c^2-1
            \end{vmatrix}
        \end{align*}
        进行列变换 $C_2\to -C_1+C_2,C_3\to -C_1+C_3$,
        \[
        D=(a^2+b^2+c^2-1) 
        \begin{vmatrix}
        1 & 0 & 0 \\
        b^2 & -1 & 0 \\
        c^2 & 0 & -1
        \end{vmatrix}
        =(47-1)\cdot1=46
        \]
    \end{solution}

    \question 已知
\[
\begin{vmatrix} 
a & b & c \\ 
r & q & p \\ 
x & y & z 
\end{vmatrix} = 5,
\]
求
\[
\begin{vmatrix} 
b+c & c+a & a+b \\ 
q+r & r+p & p+q \\ 
y+z & z+x & x+y 
\end{vmatrix}.
\]

\begin{solution}
先对行和列进行初等变换:
\begin{align*}
\begin{vmatrix} 
b+c & c+a & a+b \\ 
q+r & r+p & p+q \\ 
y+z & z+x & x+y 
\end{vmatrix}
&= \begin{vmatrix} 
b+c & c+a & b-c \\ 
q+r & r+p & q-r \\ 
y+z & z+x & y-z 
\end{vmatrix} \\
&= 2 \begin{vmatrix} 
b & c+a & b-c \\ 
q & r+p & q-r \\ 
y & z+x & y-z 
\end{vmatrix} \\
&= 2 \begin{vmatrix} 
b & c+a & -c \\ 
q & r+p & -r \\ 
y & z+x & -z 
\end{vmatrix} \\
&= 2 \begin{vmatrix} 
b & a & -c \\ 
q & p & -r \\ 
y & x & -z 
\end{vmatrix} \\
&= 2 \begin{vmatrix} 
b & a & c \\ 
q & p & r \\ 
y & x & z 
\end{vmatrix} \\
&= 2 \begin{vmatrix} 
a & b & c \\ 
p & q & r \\ 
x & y & z 
\end{vmatrix} \\
&= 2 \times 5 = 10.
\end{align*}

因此,该行列式的值为
\[
\boxed{10}.
\]
\end{solution}

\question 计算行列式
\[
\begin{vmatrix} 
b+c & a & a \\ 
b & c+a & b \\ 
c & c & a+b 
\end{vmatrix}.
\]

\begin{solution}
先对第三列进行变换:
\[
C_3 \leftarrow C_3 - C_2 \implies 
\begin{vmatrix} 
b+c & a & -b \\ 
b & c+a & -c \\ 
c & c & a 
\end{vmatrix}.
\]

对第一行和第二行进行行变换:
\[
R_1 \leftarrow R_1 + R_2, \quad R_2 \leftarrow R_2 + R_3 \implies
\begin{vmatrix} 
2b & a & -b \\ 
2b & c+a & -c \\ 
2c & c & a 
\end{vmatrix}.
\]

提取公因子 2:
\[
= 2 \begin{vmatrix} 
b & a & -b \\ 
b & c+a & -c \\ 
c & c & a 
\end{vmatrix}.
\]

对第三列进行列变换:
\[
C_3 \leftarrow C_3 + C_1 \implies
2 \begin{vmatrix} 
b & a & 0 \\ 
b & c+a & 0 \\ 
c & c & a+c 
\end{vmatrix}.
\]

对第二行进行行变换:
\[
R_2 \leftarrow R_2 - R_1 \implies
2 \begin{vmatrix} 
b & a & 0 \\ 
0 & c & 0 \\ 
c & c & a+c 
\end{vmatrix}.
\]

展开计算:
\[
2 \big( b(c(a+c)) - a(0) + 0 \big) = 2(abc + bc^2).
\]

最后整理得到:
\[
\begin{vmatrix} 
b+c & a & a \\ 
b & c+a & b \\ 
c & c & a+b 
\end{vmatrix} = 4abc.
\]
\end{solution}



    \question 证明$$ \begin{vmatrix}
        a^2 & b^2 + c^2 & bc \\
        b^2 & c^2 + a^2 & ca \\
        c^2 & a^2 + b^2 & ab
        \end{vmatrix}= -(a - b)(b - c)(c - a)(a + b + c)(a^2 + b^2 + c^2)$$
    \begin{solution}
        令
        \[
        D = \begin{vmatrix}
        a^2 & b^2 + c^2 & bc \\
        b^2 & c^2 + a^2 & ca \\
        c^2 & a^2 + b^2 & ab
        \end{vmatrix}
        \]
        进行行变换 $R_2\rightarrow R_2-R_1,\;R_3\rightarrow R_3-R_1,$
        \[D=
        \begin{vmatrix}
        a^2 & b^2 + c^2 & bc \\
        b^2 - a^2 & a^2 - b^2 & ca - bc \\
        c^2 - a^2 & a^2 - c^2 & ab - bc
        \end{vmatrix}
        \]
        化简得
        \begin{align*}
            D&=\begin{vmatrix}
            a^2 & b^2 + c^2 & bc \\
            (b+a)(b-a)& (a+b)(a-b) & c(a - b) \\
            (c+a)(c-a) & (a+c)(a-c) & b(a - c)
            \end{vmatrix} \\
            &= (a-b)(a-c)\begin{vmatrix}
            a^2 & b^2 + c^2 & bc \\
            -a-b& a+b & c \\
            -a-c & a+c & b
            \end{vmatrix} 
        \end{align*}
        进行列变换 $C_1 \rightarrow C_1+C_2,$
        \[
            D=(a-b)(a-c)\begin{vmatrix}
            a^2+b^2+c^2 & b^2 + c^2 & bc \\
            0& a+b & c \\
            0 & a+c & b
            \end{vmatrix}         
        \]
        按第一列展开得
        \begin{align*}
            D&=(a-b)(a-c)(a^2+b^2+c^2)(ab+b^2-ac-c^2)\\
            &=(a-b)(a-c)(a^2+b^2+c^2)(a(b-c)+(b+c)(b-c))\\
            &=-(a - b)(b - c)(c - a)(a + b + c)(a^2 + b^2 + c^2)
        \end{align*}
    \end{solution}
        
    \question 证明
        \[
        \begin{vmatrix}
        (b+c)^2 & a^2 & a^2 \\
        b^2 & (c+a)^2 & b^2 \\
        c^2 & c^2 & (a+b)^2 \\
        \end{vmatrix}
        = 2abc(a+b+c)^3
        \]
    \begin{solution}
        令
        \[
        D = \begin{vmatrix}
        (b+c)^2 & a^2 & a^2 \\
        b^2 & (c+a)^2 & b^2 \\
        c^2 & c^2 & (a+b)^2 \\
        \end{vmatrix}
        \]
        进行列变换 \( C_2 \rightarrow C_2 - C_1,\; C_3 \rightarrow C_3 - C_1 \),得
        \[
        D = \begin{vmatrix}
        (b+c)^2 & a^2 - (b+c)^2 & a^2 - (b+c)^2 \\
        b^2 & (c+a)^2 - b^2 & 0 \\
        c^2 & 0 & (a+b)^2 - c^2 \\
        \end{vmatrix}
        \]
        化简得
        \begin{align*}
            D &= \begin{vmatrix}
            (b+c)^2 & (a+b+c)(a-b-c) & (a+b+c)(a-b-c) \\
            b^2 & (a+b+c)(a-b+c) & 0 \\
            c^2 & 0 & (a+b+c)(a+b-c) \\
            \end{vmatrix} \\
            &= (a+b+c)^2 \begin{vmatrix}
            (b+c)^2 & a-b-c & a-b-c \\
            b^2 & a-b+c & 0 \\
            c^2 & 0 & a+b-c \\
            \end{vmatrix}
        \end{align*}
        进行行变换 \( R_1 \rightarrow R_1 - R_2 - R_3 \),得
        \[
        D = (a+b+c)^2 \begin{vmatrix}
        2bc & -2c & -2b \\
        b^2 & a-b+c & 0 \\
        c^2 & 0 & a+b-c \\
        \end{vmatrix}
        \]
        再进行列变换 \( C_2 \rightarrow C_2 + \dfrac{1}{b}C_1,\; C_3 \rightarrow C_3 + \dfrac{1}{c}C_1 \),得
        \[
        D = (a+b+c)^2 \begin{vmatrix}
        2bc & 0 & 0 \\
        b^2 & a+c & \dfrac{b^2}{c} \\
        c^2 & \dfrac{c^2}{b} & a+b \\
        \end{vmatrix}
        \]
        按第一行展开行列式得
        \begin{align*}
        D &= (a+b+c)^2 \cdot 2bc \left[(a+c)(a+b) - \frac{b^2}{c} \cdot \frac{c^2}{b} \right] \\
        &= 2bc(a+b+c)^2 \left(a^2 + ab + ac + bc - bc \right) \\
        &= 2abc(a+b+c)^3
        \end{align*}
    \end{solution}

    \question 证明
        \[
        \left|
        \begin{matrix}
        ax + by & ay + bz & az + bx \\
        ay + bz & az + bx & ax + by \\
        az + bx & ax + by & ay + bz \\
        \end{matrix}
        \right|
        =
        \left(a^3 + b^3\right)
        \left|
        \begin{matrix}
        x & y & z \\
        y & z & x \\
        z & x & y \\
        \end{matrix}
        \right|
        \]
    \begin{solution}
        令\[
        D=\begin{vmatrix}
        ax + by & ay + bz & az + bx \\
        ay + bz & az + bx & ax + by \\
        az + bx & ax + by & ay + bz \\
        \end{vmatrix}
        \]
        按$C_1$拆分
        \begin{align*}
            D&=\begin{vmatrix}
            ax  & ay + bz & az + bx \\
            ay  & az + bx & ax + by \\
            az  & ax + by & ay + bz \\
            \end{vmatrix} \;+\;       
            \begin{vmatrix}
            by & ay + bz & az + bx \\
            bz & az + bx & ax + by \\
            bx & ax + by & ay + bz \\
            \end{vmatrix}\\
            &= a\begin{vmatrix}
            x  & ay + bz & az + bx \\
            y  & az + bx & ax + by \\
            z  & ax + by & ay + bz \\
            \end{vmatrix} \;+\;b       
            \begin{vmatrix}
            y & ay + bz & az + bx \\
            z & az + bx & ax + by \\
            x & ax + by & ay + bz \\
            \end{vmatrix}
        \end{align*}
        第一个行列式进行$C_3 \rightarrow C_3-bC_1$,第二个行列式进行$C_2 \rightarrow C_2-aC_1$得
        \begin{align*}
            D&=a\begin{vmatrix}
            x & ay + bz & az \\
            y & az + bx & ax \\
            z & ax + by & ay \\
            \end{vmatrix}
            +b\begin{vmatrix}
            y & bz & az + bx \\
            z & bx & ax + by \\
            x & by & ay + bz \\
            \end{vmatrix}\\
            &=a^2\begin{vmatrix}
            x & ay + bz & z \\
            y & az + bx & x \\
            z & ax + by & y \\
            \end{vmatrix}
            +b^2\begin{vmatrix}
            y & z & az + bx \\
            z & x & ax + by \\
            x & y & ay + bz \\
            \end{vmatrix}
        \end{align*}第一个行列式进行$C_2 \rightarrow C_2-bC_3$,第二个行列式进行$C_3 \rightarrow C_3-aC_2$得
        \begin{align*}
            D&=a^2\begin{vmatrix}
            x & ay & z \\
            y & az & x \\
            z & ax & y \\
            \end{vmatrix}
            +b^2\begin{vmatrix}
            y & z & bx \\
            z & x & by \\
            x & y & bz \\
            \end{vmatrix}\\
            &=(a^3+b^3)\begin{vmatrix}
                x & y & z \\
                y & z & x \\
                z & x & y \\
                \end{vmatrix}
        \end{align*}
    \end{solution}
    
    \question 证明
    \[
    \begin{vmatrix} 
    a+bx & c+dx & p+qx \\ 
    ax+b & cx+d & px+q \\ 
    u & v & w 
    \end{vmatrix} 
    = (1-x^2) 
    \begin{vmatrix} 
    a & c & p \\ 
    b & d & q \\ 
    u & v & w 
    \end{vmatrix}
    \]
    \begin{solution}
        令
        \[
        D=
        \begin{vmatrix} 
        a+bx & c+dx & p+qx \\ 
        ax+b & cx+d & px+q \\ 
        u & v & w 
        \end{vmatrix} 
        \]
        进行列行变换 \( R_1 \rightarrow R_1 - R_2 \),得
        \begin{align*}
        D &=
        \begin{vmatrix} 
        a+bx-x(ax+b) & c+dx-x(cx+d) & p+qx-x(px+q) \\ 
        ax+b & cx+d & px+q \\ 
        u & v & w 
        \end{vmatrix} \\
        &= 
        \begin{vmatrix} 
        a-ax^2 & c-cx^2 & p-px^2 \\ 
        ax+b & cx+d & px+q \\ 
        u & v & w 
        \end{vmatrix} \\
        &= (1-x^2) 
        \begin{vmatrix} 
        a & c & p \\ 
        ax+b & cx+d & px+q \\
        u & v & w 
        \end{vmatrix}
        \end{align*}
        按$R_2$拆分,得
        \[
        D= 0+(1-x^2) 
        \begin{vmatrix} 
        a & c & p \\ 
        b & d & q \\
        u & v & w 
        \end{vmatrix}
        = (1-x^2)
        \begin{vmatrix} 
        a & c & p \\ 
        b & d & q \\
        u & v & w 
        \end{vmatrix}
        \]
    \end{solution}

    \question 证明
    \[
    \begin{vmatrix} 
    a & a^2 & bc \\ 
    b & b^2 & ca \\ 
    c & c^2 & ab 
    \end{vmatrix} 
    = 
    \begin{vmatrix} 
    1 & a^2 & a^3 \\ 
    1 & b^2 & b^3 \\ 
    1 & c^2 & c^3 
    \end{vmatrix}
    \]
    \begin{solution}
        令
        \[
        D=
        \begin{vmatrix} 
        a & a^2 & bc \\ 
        b & b^2 & ca \\ 
        c & c^2 & ab 
        \end{vmatrix} 
        \]
        将$a,b,c$分别乘入$R_1,R_2,R_3$
        \[
        D= \frac{1}{abc} 
        \begin{vmatrix} 
        a^2 & a^3 & abc \\ 
        b^2 & b^3 & abc \\ 
        c^2 & c^3 & abc 
        \end{vmatrix} 
        = \frac{abc}{abc} 
        \begin{vmatrix} 
        a^2 & a^3 & 1 \\ 
        b^2 & b^3 & 1 \\ 
        c^2 & c^3 & 1 
        \end{vmatrix} 
        \]
        进行$C_2 \leftrightarrow C_3$,再$C_1 \leftrightarrow C_2$得
        \[
        D= - 
        \begin{vmatrix} 
        a^2 & 1 & a^3 \\ 
        b^2 & 1 & b^3 \\ 
        c^2 & 1 & c^3 
        \end{vmatrix} 
        = 
        \begin{vmatrix} 
        1 & a^2 & a^3 \\ 
        1 & b^2 & b^3 \\ 
        1 & c^2 & c^3 
        \end{vmatrix}
        \]
    \end{solution}

    \question 证明
    \[
    \begin{vmatrix} 
    \sin\alpha & \cos\alpha & \cos(\alpha+\delta) \\ 
    \sin\beta & \cos\beta & \cos(\beta+\delta) \\ 
    \sin\gamma & \cos\gamma & \cos(\gamma+\delta) 
    \end{vmatrix} = 0
    \]
    \begin{solution}
        令\[
        D=    
        \begin{vmatrix} 
        \sin\alpha & \cos\alpha & \cos(\alpha+\delta) \\ 
        \sin\beta & \cos\beta & \cos(\beta+\delta) \\ 
        \sin\gamma & \cos\gamma & \cos(\gamma+\delta) 
        \end{vmatrix}
        \]
        由三角公式 $\cos(\theta+\delta) = \cos\theta\cos\delta - \sin\theta\sin\delta$,将$C_3$展开得
        \[
        D=\begin{vmatrix} \sin\alpha & \cos\alpha & \cos(\alpha+\delta) \\ \sin\beta & \cos\beta & \cos(\beta+\delta) \\ \sin\gamma & \cos\gamma & \cos(\gamma+\delta) \end{vmatrix} 
        = \cos\delta \begin{vmatrix} \sin\alpha & \cos\alpha & \cos\alpha \\ \sin\beta & \cos\beta & \cos\beta \\ \sin\gamma & \cos\gamma & \cos\gamma \end{vmatrix} 
        - \sin\delta \begin{vmatrix} \sin\alpha & \cos\alpha & \sin\alpha \\ \sin\beta & \cos\beta & \sin\beta \\ \sin\gamma & \cos\gamma & \sin\gamma \end{vmatrix} 
        \]
        其中两个行列式都有两列成比例,因此
        \[
        D=0
        \]
    \end{solution}
    
    \question 证明
    \[
    \begin{vmatrix}
    \cos^2 a & \sin a & \sin^2 a \\
    -\cos 2a & \cos a - \sin a & \cos 2a \\
    -\sin 2a - \cos 2a & 2 \cos a & 1 + \sin 2a + \cos 2a
    \end{vmatrix} = \sin a \cos a (\cos a - \sin a)
    \]
    \begin{solution}
        令
        \[
        D=\begin{vmatrix}
        \cos^2 a & \sin a & \sin^2 a \\
        -\cos 2a & \cos a - \sin a & \cos 2a \\
        -\sin 2a - \cos 2a & 2 \cos a & 1 + \sin 2a + \cos 2a
        \end{vmatrix}
        \]
        进行列变换 $C_1 \rightarrow C_1 - C_3$,
        \[
        D=\begin{vmatrix}
        1 & \sin a & \sin^2 a \\
        0 & \cos a - \sin a & \cos 2a \\
        1 & 2 \cos a & 1 + \sin 2a + \cos 2a
        \end{vmatrix}
        \]
        进行行变换 $R_3 \rightarrow R_3 - R_2$,
        \[
        D= \begin{vmatrix}
        1 & \sin a & \sin^2 a \\
        0 & \cos a - \sin a & \cos 2a \\
        1 & \cos a + \sin a & 1 + \sin 2a
        \end{vmatrix}
        \]
        进行行变换 $R_2 \rightarrow R_2 + R_1$,化为一范德蒙行列式,
        \[
        D= \begin{vmatrix}
        1 & \sin a & \sin^2 a \\
        1 & \cos a & \cos^2 a \\
        1 & \cos a + \sin a & (\cos a + \sin a)^2
        \end{vmatrix} 
        = \sin a \cos a (\cos a - \sin a)
        \]  
    \end{solution}

    \question 证明
        \[
        \begin{vmatrix}
        x-1 & x-2 & x-3 & x-4 \\
        x^2 - 1 & x^2 - 2^2 & x^2 - 3^2 & x^2 - 4^2 \\
        x^3 - 1 & x^3 - 2^3 & x^3 - 3^3 & x^3 - 4^3 \\
        x^4 - 1 & x^4 - 2^4 & x^4 - 3^4 & x^4 - 4^4 \\
        \end{vmatrix}
        =12(x - 1)(x - 2)(x - 3)(x - 4)
        \]
    \begin{solution}
        令
        \[
        D=\begin{vmatrix}
        x-1 & x-2 & x-3 & x-4 \\
        x^2 - 1 & x^2 - 2^2 & x^2 - 3^2 & x^2 - 4^2 \\
        x^3 - 1 & x^3 - 2^3 & x^3 - 3^3 & x^3 - 4^3 \\
        x^4 - 1 & x^4 - 2^4 & x^4 - 3^4 & x^4 - 4^4 \\
        \end{vmatrix}        
        \]
        提取公因式得$D=(x - 1)(x - 2)(x - 3)(x - 4)M$,其中
        \[
        M=
        \begin{vmatrix}
        1 & 2 & 3 & 4 \\
        x+1 & x+2 & x+3 & x+4 & \\
        x^2+x+1 & x^2+2x+2^2 & x^2+3x+3^2 & x^2+4x+4^2 & \\
        x^3+x^2+x+1 & x^3+2x^2+2^2x+2^3 & x^3+3x^2+3^2x+3^3 & x^3+4x^2+4^2x+4^3 & \\
        \end{vmatrix}        
        \]
        依次进行行变换$R_4 \rightarrow R_4-xR_3,R_3 \rightarrow R_3-xR_2,R_2 \rightarrow R_2-xR_1,$
        \[
        M=
        \begin{vmatrix}
        1 & 2 & 3 & 4 \\
        1^2 & 2^2 & 3^2 & 4^2 \\
        1^3 & 2^3 & 3^3 & 4^3 \\
        1^4 & 2^4 & 3^4 & 4^4 \\
        \end{vmatrix}
        \]
        是一范德蒙行列式,
        \[
        M= \prod_{1\leq i<j\leq4}(j−i)=(2−1)(3-1)(3−2)(4−1)(4−2)(4−3)=1\cdot2\cdot1\cdot3\cdot2\cdot1= 12
        \]
        于是得证
        \[
        D=12(x - 1)(x - 2)(x - 3)(x - 4)
        \]
    \end{solution}

    \question 设 $V_n$ 为 $n$ 阶范德蒙行列式,定义如下:
    \[
    V_n = \begin{vmatrix}
    1 & x_1 & x_1^2 & \dots & x_1^{n-1} \\
    1 & x_2 & x_2^2 & \dots & x_2^{n-1} \\
    \vdots & \vdots & \vdots & \ddots & \vdots \\
    1 & x_n & x_n^2 & \dots & x_n^{n-1}
    \end{vmatrix}
    \]
    证明
    \[
    V_n = \prod_{1 \le i < j \le n} (x_j - x_i).
    \]
    \begin{solution}
        设
        \[
        V_n = \begin{vmatrix}
        1 & x_1 & x_1^2 & \dots & x_1^{n-1} \\
        1 & x_2 & x_2^2 & \dots & x_2^{n-1} \\
        1 & x_3 & x_3^2 & \dots & x_3^{n-1} \\
        \vdots & \vdots & \vdots & \ddots & \vdots \\
        1 & x_{n-1} & x_{n-1}^2 & \dots & x_{n-1}^{n-1} \\
        1 & x_n & x_n^2 & \dots & x_n^{n-1}
        \end{vmatrix}
        \]
        进行行变换$R_2\rightarrow R_2-R_1,R_3\rightarrow R_3-R_1,\cdots R_n\rightarrow R_n-R_1$,得
        \[
        V_n = \begin{vmatrix}
        1 & x_1 & x_1^2 & \dots & x_1^{n-2} & x_1^{n-1} \\
        0 & x_2-x_1 & x_2^2-x_1^2 & \dots & x_2^{n-2}-x_1^{n-2} & x_2^{n-1}-x_1^{n-1} \\
        0 & x_3-x_1 & x_3^2-x_1^2 & \dots & x_3^{n-2}-x_1^{n-2} & x_3^{n-1}-x_1^{n-1} \\
        \vdots & \vdots & \vdots & \ddots & \vdots & \vdots \\
        0 & x_{n-1}-x_1 & x_{n-1}^2-x_1^2 & \dots & x_{n-1}^{n-2}-x_1^{n-2} & x_{n-1}^{n-1}-x_1^{n-1} \\
        0 & x_n-x_1 & x_n^2-x_1^2 & \dots & x_n^{n-2}-x_1^{n-2} & x_n^{n-1}-x_1^{n-1}
        \end{vmatrix}
        \]
        进行列变换$C_2\rightarrow C_2-C_1,C_3\rightarrow C_3-C_2,\cdots C_n\rightarrow C_n-C_{n-1}$,得
        \[
        V_n =
        \begin{vmatrix}
        1 & 0 & 0 & \cdots & 0 & 0 \\
        0 & x_2-x_1 & (x_2-x_1)x_2 & \cdots & (x_2-x_1)x_2^{n-3} & (x_2-x_1)x_2^{n-2} \\
        0 & x_3-x_1 & (x_3-x_1)x_3 & \cdots & (x_3-x_1)x_3^{n-3} & (x_3-x_1)x_3^{n-2} \\
        \vdots & \vdots & \vdots & \ddots & \vdots & \vdots \\
        0 & x_{n-1}-x_1 & (x_{n-1}-x_1)x_{n-1} & \cdots & (x_{n-1}-x_1)x_{n-1}^{n-3} & (x_{n-1}-x_1)x_{n-1}^{n-2} \\
        0 & x_n-x_1 & (x_n-x_1)x_n & \cdots & (x_n-x_1)x_n^{n-3} & (x_n-x_1)x_n^{n-2}
        \end{vmatrix} 
        \]
        提取公因式,
        \[
        V_n= \prod_{k=2}^{n} (x_k-x_1)
        \begin{vmatrix}
        1 & 0 & 0 & \cdots & 0 & 0 \\
        0 & 1 & x_2 & \cdots & x_2^{n-3} & x_2^{n-2} \\
        0 & 1 & x_3 & \cdots & x_3^{n-3} & x_3^{n-2} \\
        \vdots & \vdots & \vdots & \ddots & \vdots & \vdots \\
        0 & 1 & x_{n-1} & \cdots & x_{n-1}^{n-3} & x_{n-1}^{n-2} \\
        0 & 1 & x_n & \cdots & x_n^{n-3} & x_n^{n-2}
        \end{vmatrix}
        \]
        按$R_1$展开得
        \[
        V_n = \prod_{k=2}^{n} (x_k-x_1)
        \begin{vmatrix}
        1 & x_2 & \cdots & x_2^{n-3} & x_2^{n-2} \\
        1 & x_3 & \cdots & x_3^{n-3} & x_3^{n-2} \\
        \vdots & \vdots & \ddots & \vdots & \vdots \\
        1 & x_{n-1} & \cdots & x_{n-1}^{n-3} & x_{n-1}^{n-2} \\
        1 & x_n & \cdots & x_n^{n-3} & x_n^{n-2}
        \end{vmatrix}
        = \prod_{k=2}^{n}(x_k-x_1)\,V_{n-1}
        \]
        依次递推得证
        \[
        V_n = \prod_{1 \le i < j \le n} (x_j - x_i)
        \]
        其中
        \[
        V_2=\begin{vmatrix}1 & x_{n-1}\\1 & x_n\end{vmatrix}=x_n-x_{n-1}
        \]
    \end{solution}

    \question 写出帕斯卡三角形的无限数组如下:
%\begin{tabular}{cccccc}
%1 & 1 & 1 & 1 & 1 & \dots \\
%1 & 2 & 3 & 4 & 5 & \dots \\
%1 & 3 & 6 & 10 & 15 & \dots \\
%1 & 4 & 10 & 20 & 35 & \dots \\
%1 & 5 & 15 & 35 & 70 & \dots \\
%\vdots & \vdots & \vdots & \vdots & \vdots & \ddots
%\end{tabular}
其中首行和首列全为 $1$,其余每一项为左方和上方两项之和。对每个正整数 $n$,设 $D_n$ 为取该数组前 $n$ 行和前 $n$ 列形成的 $n\times n$ 矩阵。求 $\det(D_n)$ 并证明。

\begin{solution}
对小的 $n$ 直接计算表明 $\det(D_n) = 1$ 对所有 $n$ 成立。

\textbf{证明思路:}

对 $D_n$ 依次对行做初等变换,从第 $n$ 行到第 $2$ 行:
\begin{enumerate}
    \item 第 $n$ 行减去第 $n-1$ 行,
    \item 第 $n-1$ 行减去第 $n-2$ 行,
    \item \dots
    \item 第 2 行减去第 1 行。
\end{enumerate}

根据数组的构造方式,这将使每列向右平移一位,并使第一列变为首项为 $1$,其余 $n-1$ 项为 $0$。对行 $n$ 到 $3$,再到 $n$ 到 $4$ 等重复此过程,可以得到上三角矩阵,且对角线上全为 $1$,显然行列式为 $1$。

另一种方法:在第一次行变换后展开第一列的余子式,可得到 $\det(D_n)=\det(D_{n-1})$。结合小 $n$ 的值即可得出结果。
\end{solution}

\question 证明
    \[
    \left|
    \begin{matrix}
    a & b & c & d \\
    -b & a & -d & c \\
    -c & d & a & -b \\
    -d & -c & b & a \\
    \end{matrix}
    \right|
    =
    \left(a^2 + b^2 + c^2 + d^2\right)^2
    \]    

\question 求行列式$D_n = \begin{vmatrix}
    1 & 1 & 1 & \cdots & 1 & 1 & -n \\
    1 & 1 & 1 & \cdots & 1 & -n & 1 \\
    1 & 1 & 1 & \cdots & -n & 1 & 1 \\
    \vdots & \vdots & \vdots & \ddots & \vdots & \vdots & \vdots \\
    -n & 1 & 1 & \cdots & 1 & 1 & 1
    \end{vmatrix}$
    \begin{solution}
    这是一个 $n\times n$ 的行列式,每一行有 $n{-}1$ 个 $1$ 和一个 $-n$,且 $-n$ 分别出现在每行的不同列。
    
    设该行列式为 $D_n$,我们将其简化。
    
    \textbf{第一步:行变换}。
    
    令第 $i$ 行减去第 $n$ 行,$i = 1,2,\ldots,n-1$,则得到一个新行列式 $D_n'$,变换后前 $n{-}1$ 行只有两个非零元素($1{-}1=0$,$-n{-}1=-(n{+}1)$,$1{-}(-n)=n{+}1$ 等等),便于处理。
    
    经此变换,前 $n{-}1$ 行在对角线上为 $n{+}1$,其余为 $0$。
    
    \textbf{第二步:观察结构}。
    
    经过上述行变换,$D_n'$ 为一个上三角矩阵,其前 $n{-}1$ 个对角元为 $n+1$,最后一行为原来的第 $n$ 行,未变动。
    
    于是可得:
    \[
    \begin{aligned}
    D_n &= \begin{vmatrix}
    n+1 & 0   & 0   & \cdots & 0   & a_1 \\
    0   & n+1 & 0   & \cdots & 0   & a_2 \\
    0   & 0   & n+1 & \cdots & 0   & a_3 \\
    \vdots & \vdots & \vdots & \ddots & \vdots & \vdots \\
    0   & 0   & 0   & \cdots & n+1 & a_{n-1} \\
    - n & 1   & 1   & \cdots & 1   & 1
    \end{vmatrix}
    \end{aligned}
    \]
    
    将该行列式按最后一列展开,只需考虑对角线乘积(因为其余部分为 0),于是有:
    \[
    D_n = (n+1)^{n-1} \cdot \text{余子式项}.
    \]
    
    由于我们做了 $(n-1)$ 次行变换,每次减去第 $n$ 行,所以符号为 $(-1)^{\frac{n(n+1)}{2}}$(由置换中逆序数判断)。
    
    \textbf{最终结果为}:
    \[
    \boxed{D_n = (-1)^{\frac{n(n+1)}{2}} (n+1)^{n-1}}.
    \]
    \textcolor{red}{(待验证)}
\end{solution}

\question 计算 $n\times n$ 矩阵 $A=[a_{ij}]$ 的行列式。

\begin{solution}
将第二行加到第一行,再将第三行加到第二行,依此类推,将第 $n$ 行加到第 $n-1$ 行,行列式的值不变,于是得到
\[
\det(A)=
\begin{vmatrix}
2 & -1 & +1 & \cdots & \pm1 & \mp1 \\
-1 & 2 & -1 & \cdots & \pm1 & \mp1 \\
+1 & -1 & 2 & \cdots & \pm1 & \mp1 \\
\vdots & \vdots & \vdots & \ddots & \vdots & \vdots \\
\mp1 & \pm1 & \mp1 & \cdots & 2 & -1 \\
\pm1 & \mp1 & \pm1 & \cdots & -1 & 2
\end{vmatrix}
=
\begin{vmatrix}
1 & 1 & 0 & 0 & \cdots & 0 & 0 \\
0 & 1 & 1 & 0 & \cdots & 0 & 0 \\
0 & 0 & 1 & 1 & \cdots & 0 & 0 \\
\vdots & \vdots & \vdots & \vdots & \ddots & \vdots & \vdots \\
0 & 0 & 0 & 0 & \cdots & 1 & 1 \\
\pm1 & \mp1 & \pm1 & \mp1 & \cdots & -1 & 2
\end{vmatrix}.
\]

接着,用第一列减去第二列,再用所得的第二列减去第三列,依此类推,最后用第 $n-1$ 列减去第 $n$ 列,得到
\[
\det(A)=
\begin{vmatrix}
1 & 0 & 0 & \cdots & 0 & 0 \\
0 & 1 & 0 & \cdots & 0 & 0 \\
0 & 0 & 0 & \cdots & 1 & 0 \\
0 & 0 & 0 & \cdots & 0 & n+1
\end{vmatrix}
= n+1。
\]
\end{solution}

        
    \question 设
        \[
        \begin{cases}
        k x_1 + x_2 + x_3 = 5, \\
        3 x_1 + 2 x_2 + k x_3 = 18 - 5k, \\
        x_2 + 2 x_3 = 2,
        \end{cases}
        \]
        问 $k$ 取何值时,方程组无解、有唯一解、有无穷解?在有无穷解时,求全部解。
        
    \begin{solution}
        原方程组为$ \mathbf{A} \mathbf{x}=\mathbf{b}$,其中
        \[
        \mathbf{A} = \begin{bmatrix}
        k & 1 & 1 \\
        3 & 2 & k \\
        0 & 1 & 2
        \end{bmatrix},
        \quad
        \mathbf{b} = \begin{bmatrix} 5 \\ 18 - 5k \\ 2 \end{bmatrix}.
        \]
        解
        \[
        \det(\mathbf{A}) = 4k-3-k^2-6= -k^2 + 4k - 3 = 0
        \]
        得$k=1,3$, 因此:
        \begin{itemize}
        \item 当 $k \ne 1, 3$ 时,$\det(\mathbf{A}) \ne 0$,方程组有唯一解;
        \item 当 $k = 1$ 时,代入方程组得:
        \[
        \begin{cases}
        x_1 + x_2 + x_3 = 5, \\
        3x_1 + 2x_2 + x_3 = 13, \\
        x_2 + 2x_3 = 2.
        \end{cases}
        \]
        用第三式解出 $x_2 = 2 - 2x_3$,代入前两式得
        \[
        \begin{aligned}
        x_1 + (2 - 2x_3) + x_3 &= 5 \Rightarrow x_1 = 3 + x_3, \\
        3x_1 + 2x_2 + x_3 &= 13 \Rightarrow 3(3 + x_3) + 2(2 - 2x_3) + x_3 = 13.
        \end{aligned}
        \]
        检验等式成立:
        \[
        9 + 3x_3 + 4 - 4x_3 + x_3 = 13 \Rightarrow 13 = 13.
        \]
        成立,说明方程组有无穷多解,设 $x_3 = \lambda$,则
        \[
        x_1 = 3 + \lambda, \quad
        x_2 = 2 - 2\lambda, \quad
        x_3 = \lambda \quad (\lambda \in \mathbb{R}).
        \]
        \item 当 $k = 3$ 时,代入原方程组得
        \[
        \begin{cases}
        3x_1 + x_2 + x_3 = 5, \\
        3x_1 + 2x_2 + 3x_3 = 3, \\
        x_2 + 2x_3 = 2.
        \end{cases}
        \]
        由前两式相减得:
        \[
        (3x_1 + 2x_2 + 3x_3) - (3x_1 + x_2 + x_3) = -2 \Rightarrow x_2 + 2x_3 = -2,
        \]
        与第三式 $x_2 + 2x_3 = 2$ 矛盾,因此方程组无解。
        \end{itemize}
    \end{solution}

    \question 已知
    \[
    A=\begin{bmatrix}1&0&0\\ -2&3&0\\ 0&-4&5\end{bmatrix},\quad
    I=\begin{bmatrix}1&0&0\\ 0&1&0\\ 0&0&1\end{bmatrix},\quad
    B=(I+A)^{-1}(I-A),
    \]
    求矩阵 $(I+B)^{-1}$。
    \begin{solution}
        由 $B=(I+A)^{-1}(I-A)$ 可得
        \[
        (I+A)B=I-A
        \]
        两边加上 $I+A$ 得
        \[
        I+A+B+AB =(I+A)(I+B)=2I
        \]
        因此
        \[
        (I+B)^{-1}=\frac{1}{2}(I+A)=\frac{1}{2}\begin{bmatrix}2&0&0\\ -2&4&0\\ 0&-4&6\end{bmatrix}
        =\begin{bmatrix}1&0&0\\ -1&2&0\\ 0&-2&3\end{bmatrix}
        \]
    \end{solution}

    \question 已知
    \[
    \begin{bmatrix} a & b \\ c & d \end{bmatrix} 
    \begin{bmatrix} p & q \\ r & s \end{bmatrix} 
    = \begin{bmatrix} 5 & 3 \\ 3 & 2 \end{bmatrix}
    \]
    且联立方程
    \[
    \begin{cases} ax+by=5\\ cx+dy=-3 \end{cases}
    \]
    恰有一组解 $(x,y)=(1,2)$,求联立方程
    \[
    \begin{cases} pu+qv=1\\ ru+sv=2 \end{cases}
    \]
    的解 $(u,v)$。
    \begin{solution}
        已知
        \[
        \begin{bmatrix} a & b \\ c & d \end{bmatrix} \begin{bmatrix} 1 \\ 2 \end{bmatrix} = \begin{bmatrix} 5 \\ -3 \end{bmatrix} 
        \]
        且
        \[
        \begin{bmatrix} a & b \\ c & d \end{bmatrix} 
        \begin{bmatrix} p & q \\ r & s \end{bmatrix} 
        = \begin{bmatrix} 5 & 3 \\ 3 & 2 \end{bmatrix} 
        \Rightarrow 
        \begin{bmatrix} p & q \\ r & s \end{bmatrix}^{-1} = 
        \begin{bmatrix} 2 & -3 \\ -3 & 5 \end{bmatrix} 
        \begin{bmatrix} a & b \\ c & d \end{bmatrix}
        \]
        因此
        \[
        \begin{bmatrix} p & q \\ r & s \end{bmatrix}^{-1} \begin{bmatrix} 1 \\ 2 \end{bmatrix} 
        = \begin{bmatrix} 2 & -3 \\ -3 & 5 \end{bmatrix} \begin{bmatrix} 5 \\ -3 \end{bmatrix} 
        = \begin{bmatrix} 19 \\ -30 \end{bmatrix}
        \]
        即
        \[
        \begin{bmatrix} p & q \\ r & s \end{bmatrix} \begin{bmatrix} 19 \\ -30 \end{bmatrix} = \begin{bmatrix} 1 \\ 2 \end{bmatrix}
        \]
        故联立方程的解为
        \[
        (u,v) = (19,-30)
        \]
    \end{solution}

    \question 设三阶方阵 
    \[
    A = \begin{bmatrix}1 & a & b \\ 0 & 1 & a \\ 0 & 0 & 1\end{bmatrix},\quad 
    A^{10} = \begin{bmatrix}1 & ka & pa^2 + qb \\ 0 & 1 & ka \\ 0 & 0 & 1\end{bmatrix},
    \]
    其中$k,p,q$为常数,试求 $k+p+q$。
    \begin{solution}
        将 $A$ 分解为 $A = I + B$,其中
        \[
        I = \begin{bmatrix}1 & 0 & 0 \\ 0 & 1 & 0 \\ 0 & 0 & 1\end{bmatrix},\quad
        B = \begin{bmatrix}0 & a & b \\ 0 & 0 & a \\ 0 & 0 & 0\end{bmatrix}.
        \]
        注意到 $B^3 = 0$,且
        \[
        B^2 = \begin{bmatrix}0 & 0 & a^2 \\ 0 & 0 & 0 \\ 0 & 0 & 0\end{bmatrix}.
        \]
        由二项式定理,
        \[
        A^{10} = (I + B)^{10} = \sum_{k=0}^{10} \comb{10}{k} B^k I^{10-k} = I + 10B + 45 B^2=\begin{bmatrix}1 & 10a & 45a^2+10b \\ 0 & 1 & 10a \\ 0 & 0 & 1\end{bmatrix}
        \]
        所以
        \[
        k = 10,\quad p = 45,\quad q = 10 \quad \Rightarrow \quad k+p+q = 65.
        \]
    \end{solution}

    \question 设实数 $a>b$,且有二阶方阵 $X,Y$ 满足
    \[
    X + Y = I, \quad XY = O,
    \]
    其中
    \[
    I = \begin{bmatrix}1 & 0 \\ 0 & 1 \end{bmatrix}, \quad O = \begin{bmatrix}0 & 0 \\ 0 & 0 \end{bmatrix},
    \]
    且
    \[
    A = \begin{bmatrix} 2 & 4 \\ 1 & -1 \end{bmatrix} = aX + bY,
    \]
    求 $a,b$ 的值。据此,求$X^{2025}-Y^{2025}$。
    \begin{solution}
        首先有
        \[
        A = aX + bY = aX + b(I - X) = bI + (a - b)X,
        \]
        故
        \[
        X = \frac{A - bI}{a - b}, \quad Y = \frac{aI - A}{a - b}.
        \]
        所以
        \[
        XY = \frac{1}{(a - b)^2} (A - bI)(aI - A) = O,
        \]
        即
        \[
        \begin{bmatrix} 2 - b & 4 \\ 1 & -1 - b \end{bmatrix}
        \begin{bmatrix} a - 2 & -4 \\ -1 & a + 1 \end{bmatrix}
        = \begin{bmatrix} 0 & 0 \\ 0 & 0 \end{bmatrix}.
        \]
        得到方程组
        \[
        \begin{cases}
        (2 - b)(a - 2) + 4(-1) = 0, \\
        (2 - b)(-4) + 4(a + 1) = 0, \\
        1 \cdot (a - 2) + (-1 - b)(-1) = 0, \\
        1 \cdot (-4) + (-1 - b)(a + 1) = 0.
        \end{cases}
        \]
        由$a>b$,解得
        \[
        a = 3 ,b = -2 
        \]
        因此
        \[
        X=\frac{1}{5}\begin{bmatrix}4&4\\1&1\end{bmatrix},\quad
        Y=I-X=\frac{1}{5}\begin{bmatrix}1&-4\\-1&4\end{bmatrix}
        \]
        且发现 $X^2=X,\ Y^2=Y$,即$X,Y$是幂等矩阵,故对任意正整数 $n$,
        \[
        X^n=X,\quad Y^n=Y
        \]
        因此
        \[
        X^{2025}-Y^{2025}=X-Y=\frac{1}{5}\begin{bmatrix}3&8\\2&-3\end{bmatrix}
        \]
    \end{solution}

    \question 设 $$A=\begin{bmatrix} 4 & 4 \\ -1 & -1 \end{bmatrix}$$满足 $$A+A^2+\cdots+A^n = \begin{bmatrix} 2(3^n-1) & a \\ b & c \end{bmatrix},$$求 $b+c$。
    \begin{solution}
        发现
        \[
        A^2=\begin{bmatrix}4&4\\-1&-1\end{bmatrix}^2
        =\begin{bmatrix}12&12\\-3&-3\end{bmatrix}=3A,
        \]
        于是对任意 $k\ge1$ 有 $A^k=3^{\,k-1}A$,因此
        \[
        A+\cdots+A^n
        =A\sum_{k=0}^{n-1}3^k
        =\frac{3^n-1}{2}\,A
        =\begin{bmatrix} 2(3^n-1) & 2(3^n-1) \\ \dfrac12(1-3^n) & \dfrac12(1-3^n) \end{bmatrix}
        \]
        故
        \[
        b+c = \frac12(1-3^n) + \frac12(1-3^n) = 1 - 3^n
        \]
    \end{solution}
    \begin{solution}
        又解:对角化$A$可得
        \[
        A=\begin{bmatrix}4 & 4 \\ -1 & -1 \end{bmatrix} 
        =\begin{bmatrix}-1 & -4 \\ 1 & 1 \end{bmatrix} 
        \begin{bmatrix}0 & 0 \\ 0 & 3 \end{bmatrix} 
        \begin{bmatrix} \frac13 & \frac43 \\ -\frac13 & -\frac13 \end{bmatrix} \equiv P D P^{-1}
        \]
        由$A^n = P D^n P^{-1}$,
        \[
        A + A^2 + \cdots + A^n = P(D + D^2 + \cdots + D^n)P^{-1}
        \]
        \[
        = P \begin{bmatrix}0 & 0 \\ 0 & 3+3^2+\cdots+3^n \end{bmatrix} P^{-1} 
        = P \begin{bmatrix}0 & 0 \\ 0 & \dfrac{3^{n+1}-3}{2} \end{bmatrix} P^{-1}
        = \begin{bmatrix} 2(3^n-1) & 2(3^n-1) \\ \dfrac12(1-3^n) & \dfrac12(1-3^n) \end{bmatrix}
        \]
        同上。
    \end{solution}

    \question 设矩阵 $$A=\begin{pmatrix} 7 & -8 \\ -7 & 8 \end{pmatrix}$$ 满足 $$(I+A)^n = I + a_n A,$$其中 $n$ 为自然数,$$I=\begin{pmatrix} 1 & 0 \\ 0 & 1 \end{pmatrix},$$且 $\{a_n\}$ 为一个数列,求 $a_n$ 的通项公式。
    \begin{solution}
        求 $A$ 的特征值:
        \[
        \det(A - \lambda I) = \begin{vmatrix} 7 - \lambda & -8 \\ -7 & 8 - \lambda \end{vmatrix} = \lambda^2 - 15\lambda = 0 \Rightarrow \lambda = 0,\,15
        \]
        设 
        \[
        f(\lambda) = (1+\lambda)^n = (\lambda^2 - 15\lambda)p(\lambda) + a\lambda + b,
        \]令 $\lambda=0,15$,解得
        \[
        a = \frac{16^n - 1}{15},\quad b = 1
        \]
        由凯莱-哈密顿定理,
        \[
        (I + A)^n = aA + bI = \frac{16^n - 1}{15}A + I
        \Rightarrow a_n = \frac{1}{15}(16^n - 1)
        \]
    \end{solution}

    \question 设矩阵 
    \[
    A = \begin{bmatrix} -5 & -4 \\ 9 & 7 \end{bmatrix}, 
    \] 
    求 
    \[
    A^{51} - A^{50} + A^3 - 3A^2 - 2A + 4I_2,
    \] 
    其中 \(I_2 = \begin{bmatrix} 1 & 0 \\ 0 & 1 \end{bmatrix}\)。
    \begin{solution}
        发现 
        \[
        (A-I)^2 = {\begin{bmatrix}-6 & -4 \\ 9 & 6\end{bmatrix}}^2=\begin{bmatrix}0 & 0 \\ 0 & 0\end{bmatrix} \Rightarrow A^2=2A-I
        \] 
        且考虑方程
        \[
        \lambda^{50} = q(\lambda)(\lambda-1)^2 + c_1 \lambda + c_0 \tag{1}
        \]
        对 \(\lambda\) 求导,可得
        \[
        50\lambda^{49}=q'(\lambda)(\lambda-1)^2+2q(\lambda)(\lambda-1)+c_1 \tag{2}
        \]
        由$(1),(2),$代入 \(\lambda=1\) 得 
        \[
        c_0=-49, \ c_1=50
        \]
        由凯莱-哈密顿定理,
        \[
        A^{50} = 50A - 49I
        \] 
        因此 
        \[
        A^{51} - A^{50} = (50A^2-49A) - (50A - 49I) = 100A-50I -99A - 49I=A-I
        \]
        又 
        \[
        A^3 - 3A^2 - 2A + 4I = (A-I)^3-5(A-I)=-5(A-I)
        \] 
        故
        \[
        A^{51} - A^{50} + A^3 - 3A^2 - 2A + 4I = -4(A-I) = \begin{bmatrix} 24 & 16 \\ -36 & -24 \end{bmatrix}
        \]
    \end{solution}

    \question 已知
    \[
    \begin{bmatrix}
    a & b & c \\
    d & e & f \\
    g & h & i
    \end{bmatrix}^{101}
    =
    \begin{bmatrix}
    -1 & -2 & -2 \\
    1 & 2 & 1 \\
    -1 & -1 & 0
    \end{bmatrix}\\[2mm]
    \]
    求 $a + b + c + d + e + f + g + h + i$。
    \begin{solution}
        设
        \[
        A = \begin{bmatrix}
        -1 & -2 & -2 \\
        1 & 2 & 1 \\
        -1 & -1 & 0
        \end{bmatrix}
        \]
        解 $\det(A - \lambda I) = 0$,即
        \[
        \begin{vmatrix}
        -1-\lambda & -2 & -2 \\
        1 & 2-\lambda & 1 \\
        -1 & -1 & -\lambda
        \end{vmatrix}
        = -(\lambda - 1)^2(\lambda + 1) =0 \Rightarrow \lambda = 1 \ (m=2), \ \lambda = -1
        \]
        对于 $\lambda = 1$,解 $(A - I)\textbf{x} = \textbf{0}$:
        \[
        A - I =
        \begin{bmatrix}
        -2 & -2 & -2 \\
        1 & 1 & 1 \\
        -1 & -1 & -1
        \end{bmatrix}
        \Rightarrow
        \begin{bmatrix}
        1 & 1 & 1 \\
        0 & 0 & 0 \\
        0 & 0 & 0
        \end{bmatrix}
        \Rightarrow x_1 = -x_2 - x_3,\ x_2,\ x_3 \in \mathbb{R}
        \]
        取两个线性无关解:
        \[
        \vec{v}_1 =
        \begin{bmatrix}
        -1 \\ 1 \\ 0
        \end{bmatrix},\quad
        \vec{v}_2 =
        \begin{bmatrix}
        -1 \\ 0 \\ 1
        \end{bmatrix}
        \]
        对于 $\lambda = -1$,解 $(A + I)\textbf{x} = \textbf{0}$:
        \[
        A + I =
        \begin{bmatrix}
        0 & -2 & -2 \\
        1 & 3 & 1 \\
        -1 & -1 & 1
        \end{bmatrix}
        \Rightarrow
        \begin{bmatrix}
        1 & 3 & 1 \\
        0 & 1 & 1 \\
        0 & 0 & 0
        \end{bmatrix} \Rightarrow x_1 = x_3,\ x_2 = -x_3
        \]
        取
        \[
        \vec{v}_3 =
        \begin{bmatrix}
        2 \\ -1 \\ 1
        \end{bmatrix}
        \]
        于是
        \[
        P =
        \begin{bmatrix}
        -1 & -1 & 2 \\
        1 & 0 & -1 \\
        0 & 1 & 1
        \end{bmatrix}, \quad
        D =
        \begin{bmatrix}
        1 & 0 & 0 \\
        0 & 1 & 0 \\
        0 & 0 & -1
        \end{bmatrix}
        \]
        现计算逆矩阵 \(P^{-1}\),由$|P|=2$得
        \[
        P^{-1} 
        =\frac{1}{2}
        \begin{bmatrix}
        \begin{vmatrix} 0 & -1 \\ 1 & 1 \end{vmatrix} & -\begin{vmatrix} 1 & -1 \\ 0 & 1 \end{vmatrix} & \begin{vmatrix} 1 & 0 \\ 0 & 1 \end{vmatrix} \\
        -\begin{vmatrix} -1 & 2 \\ 1 & 1 \end{vmatrix} & \begin{vmatrix} -1 & 2 \\ 0 & 1 \end{vmatrix} & -\begin{vmatrix} -1 & -1 \\ 0 & 1 \end{vmatrix} \\
        \begin{vmatrix} -1 & 2 \\ 0 & -1 \end{vmatrix} & -\begin{vmatrix} -1 & 2 \\ 1 & -1 \end{vmatrix} & \begin{vmatrix} -1 & -1 \\ 1 & 0 \end{vmatrix}
        \end{bmatrix}^T
        =\begin{bmatrix}
        \frac{1}{2} & \frac{3}{2} & \frac{1}{2} \\
        -\frac{1}{2} & -\frac{1}{2} & \frac{1}{2} \\
        \frac{1}{2} & \frac{1}{2} & \frac{1}{2}
        \end{bmatrix}
        \]
        于是
        \[
        A = P D P^{-1} =
        \begin{bmatrix}
        -1 & -1 & 1 \\
        1 & 0 & -1 \\
        0 & 1 & 1
        \end{bmatrix}
        \begin{bmatrix}
        1 & 0 & 0 \\
        0 & 1 & 0 \\
        0 & 0 & -1
        \end{bmatrix}
        \begin{bmatrix}
        \frac{1}{2} & \frac{3}{2} & \frac{1}{2} \\
        -\frac{1}{2} & -\frac{1}{2} & \frac{1}{2} \\
        \frac{1}{2} & \frac{1}{2} & \frac{1}{2}
        \end{bmatrix}
        \]
        所以
        \[
        \begin{bmatrix}
        a & b & c \\
        d & e & f \\
        g & h & i
        \end{bmatrix}
        =
        \begin{bmatrix}
        -1 & -1 & 2 \\
        1 & 0 & -1 \\
        0 & 1 & 1
        \end{bmatrix}
        \begin{bmatrix}
        1^{\frac{1}{101}} & 0 & 0 \\
        0 & 1^{\frac{1}{101}} & 0 \\
        0 & 0 & (-1)^{\frac{1}{101}}
        \end{bmatrix}
        \begin{bmatrix}
        \frac{1}{2} & \frac{3}{2} & \frac{1}{2} \\
        -\frac{1}{2} & -\frac{1}{2} & \frac{1}{2} \\
        \frac{1}{2} & \frac{1}{2} & \frac{1}{2}
        \end{bmatrix}
        =
        \begin{bmatrix}
        -1 & -2 & -2 \\
        1 & 2 & 1 \\
        -1 & -1 & 0
        \end{bmatrix}
        \]
        恰好也是原矩阵,因此
        \[
        a + b + c + d + e + f + g + h + i = -3
        \]
    \end{solution}

\question 设 $A$ 和 $B$ 为 $n\times n$ 实矩阵,且满足
\[
AB + A + B = 0.
\]
证明 $AB = BA$。

\begin{solution}
注意到
\[
(A+I)(B+I) = AB + A + B + I = I,
\]
其中 $I$ 为单位矩阵。因此 $A+I$ 和 $B+I$ 互为逆矩阵。

由于逆矩阵的性质,有
\[
(A+I)(B+I) = (B+I)(A+I) = I.
\]
展开等式得到
\[
AB + A + B + I = BA + B + A + I \implies AB = BA.
\]

\end{solution}

\question 已知矩阵 $A,B,A+B$ 都是可逆矩阵,证明 $A^{-1}+B^{-1}$ 也是可逆矩阵。

\begin{solution}
我们需要证明存在一个矩阵 $C$,使得
\[
(A^{-1}+B^{-1})C=I.
\]
设
\[
C=B(A+B)^{-1}A.
\]

下面验证该选择是可行的。计算
\begin{align*}
(A^{-1}+B^{-1})C
&=(A^{-1}+B^{-1})B(A+B)^{-1}A \\
&=A^{-1}B(A+B)^{-1}A+B^{-1}B(A+B)^{-1}A \\
&=A^{-1}B(A+B)^{-1}A+I(A+B)^{-1}A \\
&=A^{-1}B(A+B)^{-1}A+(A+B)^{-1}A.
\end{align*}

利用恒等式 $B=(A+B)-A$,代入上式得
\begin{align*}
(A^{-1}+B^{-1})C
&=A^{-1}[(A+B)-A](A+B)^{-1}A+(A+B)^{-1}A \\
&=[A^{-1}(A+B)-A^{-1}A](A+B)^{-1}A+(A+B)^{-1}A \\
&=[A^{-1}(A+B)-I](A+B)^{-1}A+(A+B)^{-1}A \\
&=A^{-1}(A+B)(A+B)^{-1}A-I(A+B)^{-1}A+(A+B)^{-1}A \\
&=A^{-1}A-(A+B)^{-1}A+(A+B)^{-1}A \\
&=I.
\end{align*}

因此存在矩阵 $C=B(A+B)^{-1}A$ 使得 $(A^{-1}+B^{-1})C=I$,从而 $A^{-1}+B^{-1}$ 可逆,其逆矩阵为
\[
(A^{-1}+B^{-1})^{-1}=B(A+B)^{-1}A.
\]
\end{solution}

\question 设 $A,B$ 均为 $n$ 阶方阵,且 $I+AB$ 可逆,化简
\[
(I+BA)\bigl[I - B(I+AB)^{-1}A\bigr].
\]

\begin{solution}
\[
\begin{aligned}
(I+BA)\bigl[I - B(I+AB)^{-1}A\bigr] 
&= I - B(I+AB)^{-1}A + \bigl[BA - BAB(I+AB)^{-1}A\bigr] \\
&= I - B(I+AB)^{-1}A + B\bigl(I+AB - AB\bigr)(I+AB)^{-1}A \\
&= I - B(I+AB)^{-1}A + B(I+AB)(I+AB)^{-1}A \\
&= I.
\end{aligned}
\]
\end{solution}


\question 设 $A$, $B$ 和 $C$ 为同阶实方阵,且 $A$ 可逆。证明如果
\[
(A-B)C = BA^{-1},
\]
则有
\[
C(A-B) = A^{-1}B.
\]

\begin{solution}
由假设
\[
(A-B)C = BA^{-1}.
\]

在两边加上 $A^{-1}(A-B)$,得到
\[
(A-B)C + A^{-1}(A-B) = BA^{-1} + A^{-1}(A-B) = I.
\]

于是
\[
(A-B)(C + A^{-1}) = I \implies (A-B)^{-1} = C + A^{-1}.
\]

两边左乘 $(A-B)$ 得
\[
(C + A^{-1})(A-B) = I.
\]

展开即可得到
\[
C(A-B) = A^{-1}B.
\]
\end{solution}

\question 对任意整数 $n\ge 2$,设 $A,B$ 为两个 $n\times n$ 的实矩阵,且满足
\[
A^{-1}+B^{-1}=(A+B)^{-1}.
\]
证明 $\det A=\det B$。问:若 $A,B$ 为复矩阵,结论是否仍成立?

\begin{solution}
两边同乘以 $(A+B)$,得
\begin{align*}
I&=(A+B)(A+B)^{-1} \\
&=(A+B)(A^{-1}+B^{-1}) \\
&=AA^{-1}+AB^{-1}+BA^{-1}+BB^{-1} \\
&=I+AB^{-1}+BA^{-1}+I.
\end{align*}
于是
\[
AB^{-1}+BA^{-1}+I=0.
\]

令 $X=AB^{-1}$,则 $A=XB$,且
\[
BA^{-1}=X^{-1}.
\]
因此
\[
X+X^{-1}+I=0.
\]
两边左乘 $(X-I)X$,得
\begin{align*}
0&=(X-I)X(X+X^{-1}+I) \\
&=(X-I)(X^2+X+I) \\
&=X^3-I.
\end{align*}
从而
\[
X^3=I.
\]
取行列式可得
\[
(\det X)^3=\det(X^3)=\det I=1,
\]
由于 $X$ 为实矩阵,故
\[
\det X=1.
\]
又因为
\[
\det A=\det(XB)=\det X\det B=\det B,
\]
从而 $\det A=\det B$。

下面说明在复矩阵情形下结论不成立。设
\[
\omega=\frac{1}{2}(-1+i\sqrt{3}),
\]
则 $\omega\notin\mathbb{R}$,且 $\omega^3=1$,并有
\[
1+\omega+\omega^2=0.
\]
取 $A=I$,$B$ 为对角矩阵,其对角元均取为 $\omega$ 或 $\omega^2$,并使得 $\det B=1$。若 $n$ 不是 $3$ 的倍数,可直接取 $B=\omega I$。

此时
\[
A^{-1}=I,\quad B^{-1}=\overline{B},
\]
且
\[
I+B+\overline{B}=0.
\]
因此
\[
(A+B)^{-1}=(-\overline{B})^{-1}=-\overline{B}^{-1}=-B=I+\overline{B}
=A^{-1}+B^{-1}.
\]
但由构造可知
\[
\det A=1\ne\det B.
\]
因此在复矩阵情形下结论不成立。
\end{solution}


    \question 设 $A,B$ 为实 $n\times n$ 矩阵,满足
\[
A^2+B^2=AB。
\]
证明:若 $BA-AB$ 为可逆矩阵,则 $n$ 能被 $3$ 整除。

\begin{solution}
设
\[
S=A+\omega B,
\]
其中
\[
\omega=-\frac12+i\frac{\sqrt3}{2}。
\]
则
\begin{align*}
S\bar S
&=(A+\omega B)(A+\bar\omega B) \\
&=A^2+\omega BA+\bar\omega AB+B^2。
\end{align*}
由已知条件 $A^2+B^2=AB$,可得
\[
S\bar S
=AB+\omega BA+\bar\omega AB。
\]
注意到 $\bar\omega+1=-\omega$,于是
\[
S\bar S=\omega(BA-AB)。
\]

对两边取行列式,有
\[
\det(S\bar S)=\det S\cdot\det\bar S,
\]
这是一个实数。同时,
\[
\det(S\bar S)=\det(\omega(BA-AB))=\omega^n\det(BA-AB)。
\]
由于 $BA-AB$ 可逆,故 $\det(BA-AB)\ne0$,从而 $\omega^n$ 必须为实数。

而 $\omega$ 是三次单位根,$\omega^n$ 为实数当且仅当 $n$ 能被 $3$ 整除。因此 $n$ 必须是 $3$ 的倍数。
\end{solution}

\question 求所有复数 $\Lambda$,使得存在正整数 $n$ 以及实 $n\times n$ 矩阵 $A$,满足
\[
A^2=A^T,
\]
并且 $\Lambda$ 是 $A$ 的一个特征值。

\begin{solution}
由条件 $A^2=A^T$,两边平方可得
\begin{align*}
A^4=(A^2)^2=(A^T)^2=(A^2)^T=(A^T)^T=A.
\end{align*}
因此
\[
A^4-A=0.
\]
于是 $A$ 的任一特征值 $\Lambda$ 必须满足多项式方程
\[
\Lambda^4-\Lambda=0.
\]
即
\[
\Lambda(\Lambda^3-1)=0.
\]
其全部根为
\[
\Lambda=0,\quad \Lambda=1,\quad \Lambda=\frac{-1\pm i\sqrt{3}}{2}.
\]

下面验证这些值均可实现。考虑如下矩阵:
\[
A_0=(0),\qquad
A_1=(1),
\]
\[
A_2=
\begin{pmatrix}
-\frac{1}{2} & -\frac{\sqrt{3}}{2} \\
\frac{\sqrt{3}}{2} & -\frac{1}{2}
\end{pmatrix},
\]
\[
A_4=
\begin{pmatrix}
0 & 0 & 0 & 0 \\
0 & 1 & 0 & 0 \\
0 & 0 & -\frac{1}{2} & -\frac{\sqrt{3}}{2} \\
0 & 0 & \frac{\sqrt{3}}{2} & -\frac{1}{2}
\end{pmatrix}.
\]

显然,$0$ 和 $1$ 分别是 $1\times1$ 矩阵 $A_0$ 与 $A_1$ 的特征值。
矩阵 $A_2$ 的特征值为
\[
\frac{-1\pm i\sqrt{3}}{2},
\]
并且可以直接计算验证
\[
A_2^2=
\begin{pmatrix}
-\frac{1}{2} & \frac{\sqrt{3}}{2} \\
-\frac{\sqrt{3}}{2} & -\frac{1}{2}
\end{pmatrix}
=A_2^T.
\]

矩阵 $A_4$ 则在同一矩阵中同时包含上述四个可能的特征值。

综上,满足条件的全部复数 $\Lambda$ 为
\[
\Lambda\in\left\{0,\;1,\;\frac{-1\pm i\sqrt{3}}{2}\right\}.
\]
\end{solution}

\question 已知 $A$ 是 $4 \times 2$ 实矩阵,$B$ 是 $2 \times 4$ 实矩阵,且
\[
AB = \begin{pmatrix}
1 & 0 & -1 & 0 \\
0 & 1 & 0 & -1 \\
-1 & 0 & 1 & 0 \\
0 & -1 & 0 & 1
\end{pmatrix}.
\]

求 $BA$。

\begin{solution}
将矩阵分块表示:
\[
A = \begin{pmatrix} A_1 \\ A_2 \end{pmatrix}, \quad B = \begin{pmatrix} B_1 & B_2 \end{pmatrix},
\]
其中 $A_1, A_2, B_1, B_2$ 为 $2 \times 2$ 矩阵。于是
\[
AB = \begin{pmatrix} A_1 \\ A_2 \end{pmatrix} \begin{pmatrix} B_1 & B_2 \end{pmatrix} = 
\begin{pmatrix} A_1B_1 & A_1B_2 \\ A_2B_1 & A_2B_2 \end{pmatrix}.
\]

比较分块得到:
\[
A_1B_1 = A_2B_2 = I_2, \quad A_1B_2 = A_2B_1 = -I_2.
\]

于是可解得:
\[
B_1 = A_1^{-1}, \quad B_2 = -A_1^{-1}, \quad A_2 = -A_1.
\]

最后
\[
BA = \begin{pmatrix} B_1 & B_2 \end{pmatrix} \begin{pmatrix} A_1 \\ A_2 \end{pmatrix} 
= B_1A_1 + B_2A_2 = A_1^{-1}A_1 + (-A_1^{-1})(-A_1) = 2 I_2.
\]

\end{solution}

\question 已知 
\[
A=\begin{bmatrix} 
2 & 4 & 0 & 0 \\ 
1 & 2 & 0 & 0 \\ 
0 & 0 & 2 & 4 \\ 
0 & 0 & 2 & 2 
\end{bmatrix}, 
\] 
求 $A^n$。

\begin{solution}
将 $A$ 分块为 
\[
A=\begin{bmatrix} B & 0 \\ 0 & C \end{bmatrix}, \quad 
B=\begin{bmatrix} 2 & 4 \\ 1 & 2 \end{bmatrix}, \quad 
C=\begin{bmatrix} 2 & 4 \\ 0 & 2 \end{bmatrix}.
\]

\noindent 对 $B$,可写为 $B = \begin{bmatrix} 2 \\ 1 \end{bmatrix} \begin{bmatrix} 1 & 2 \end{bmatrix}$,所以
\[
B^n = 4^{\,n-1} B.
\]

\noindent 对 $C$,注意 $\begin{bmatrix} 0 & 4 \\ 0 & 0 \end{bmatrix}^2 = 0$,则
\[
C^n = \bigl(2E + \begin{bmatrix} 0 & 4 \\ 0 & 0 \end{bmatrix}\bigr)^n = 2^n E + n2^{\,n-1} \begin{bmatrix} 0 & 4 \\ 0 & 0 \end{bmatrix} = \begin{bmatrix} 2^n & 4n2^{\,n-1} \\ 0 & 2^n \end{bmatrix}.
\]

\noindent 因此
\[
A^n = \begin{bmatrix} B^n & 0 \\ 0 & C^n \end{bmatrix} 
= \begin{bmatrix}
2 \cdot 4^{\,n-1} & 4 \cdot 4^{\,n-1} & 0 & 0 \\
1 \cdot 4^{\,n-1} & 2 \cdot 4^{\,n-1} & 0 & 0 \\
0 & 0 & 2^n & n 2^{\,n+1} \\
0 & 0 & 0 & 2^n
\end{bmatrix} 
= \begin{bmatrix}
2^{\,2n-1} & 2^{\,2n+1} & 0 & 0 \\
2^{\,2n-2} & 2^{\,2n} & 0 & 0 \\
0 & 0 & 2^n & n 2^{\,n+1} \\
0 & 0 & 0 & 2^n
\end{bmatrix}.
\]
\end{solution}


\question 设 $A$ 和 $B$ 是 $n \times n$ 实矩阵,且满足
\[
\mathrm{rk}(AB - BA + I) = 1,
\]
其中 $I$ 是 $n\times n$ 单位矩阵。证明
\[
\mathrm{trace}(ABAB) - \mathrm{trace}(A^2B^2) = \frac{1}{2}n(n-1).
\]

\begin{solution}
令 $X = AB - BA$。注意
\begin{align*}
\mathrm{trace}(X^2) &= \mathrm{trace}(ABAB - ABBA - BAAB + BABA) \\
&= 2\mathrm{trace}(ABAB) - 2\mathrm{trace}(A^2B^2),
\end{align*}
因为迹具有循环性。因此只需证明 $\mathrm{trace}(X^2) = n(n-1)$。

由假设,$X+I$ 的秩为 $1$,所以可以写成
\[
X+I = v^t w
\]
对某些向量 $v,w$。于是
\[
X^2 = (v^t w - I)^2 = I - 2 v^t w + v^t w v^t w = I + (w v^t - 2)v^t w.
\]

由 $X$ 的定义,有 $\mathrm{trace}(X) = 0$,因此 $\mathrm{trace}(w v^t) = \mathrm{trace}(v w) = n$,于是
\[
\mathrm{trace}(X^2) = n + (n-2)n = n(n-1).
\]

另一种方法是利用秩为 $1$ 的条件:因为 $X+I$ 的秩为 $1$,它有 $0$ 的特征值,重数为 $n-1$。因此 $X$ 有 $-1$ 的特征值,重数为 $n-1$。由于 $\mathrm{trace}(X) = 0$,剩余的特征值为 $n-1$。于是
\[
\mathrm{trace}(X^2) = (n-1)^2 + (n-1)\cdot 1^2 = n(n-1).
\]

由 $\mathrm{trace}(X^2) = 2(\mathrm{trace}(ABAB) - \mathrm{trace}(A^2B^2))$,得到
\[
\mathrm{trace}(ABAB) - \mathrm{trace}(A^2B^2) = \frac{1}{2}n(n-1).
\]
\end{solution}

\question 设 $A$ 是 $n \times n$ 实矩阵,且 $A^3 = 0$。
\begin{enumerate}
\item[(a)] 证明存在唯一的 $n \times n$ 实矩阵 $X$ 满足
\[
X + AX + XA^2 = A.
\]
\item[(b)] 用 $A$ 表示 $X$。
\end{enumerate}

\begin{solution}
首先假设某矩阵 $X$ 满足方程。考虑通过左右乘以 $A$ 的幂得到的新方程。例如,
\[
A^2(X + AX + XA^2)A^2 = A^2 X A^2 + A^3 X A^2 + A^2 X A^4 = A^2 X A^2.
\]
右边为零,因为 $A^3 = 0$,所以
\[
A^2 X A^2 = 0.
\]

同理可得:
\begin{align*}
A^2 X &= A^2(X + AX + XA^2) = A^3 = 0, \\
AXA &= A(X + AX + XA^2)A = A^3 = 0, \\
X A^2 &= (X + AX + XA^2)A^2 = A^3 = 0.
\end{align*}

此外,
\[
AX = A(X + AX + XA^2) = A^2,
\]
于是
\[
X = A - AX - XA^2 = A - A^2.
\]

因此,唯一可能的解为 $X = A - A^2$。为了验证其确实满足方程:
\[
X + AX + X A^2 = (A - A^2) + A(A - A^2) + (A - A^2) A^2 = A - A^4 = A.
\]

\noindent 综上,$X = A - A^2$ 是方程的唯一解。
\end{solution}

\question 设 $A$, $B$, $C$ 为 $n \times n$ 复矩阵,满足
\[
A^2 = B^2 = C^2, \quad B^3 = ABC + 2I.
\]
证明 $A^6 = I$。

\begin{solution}
由 $A^2=B^2=C^2$,可得
\[
B^3 - ABC = B^2B - ABC = A^2B - ABC = 2I.
\]

于是
\[
A(AB - BC) = 2I.
\]

同理,由 $B^2 = C^2$,可得
\[
BC^2 - ABC = 2I \implies (BC - AB)C = 2I.
\]

这说明 $A$ 是 $\frac{AB - BC}{2}$ 的左逆矩阵,而 $-C$ 是右逆矩阵,所以 $A = -C$。因此
\[
ABA = A^2B = B^3.
\]

由于 $ABA$ 与 $B$ 交换,得到 $(AB)^2 = (BA)^2$。计算
\begin{align*}
(AB - BA)(AB + BA) &= (AB)^2 + AB^2A - BA^2B - (BA)^2 \\
&= (AB)^2 + A^4 - B^4 - (BA)^2 = 0.
\end{align*}

又因为 $AB - BC = AB + BA$ 可逆,故 $AB = BA$,于是 $ABA = A^2B = B^3$。由此得
\[
B^3 = I \implies A^6 = B^6 = I.
\]
\end{solution}


    \question 设$a=\begin{pmatrix} 1 \\ 2 \\ 1 \end{pmatrix},b=\begin{pmatrix} 1 \\ \frac12 \\ 0 \end{pmatrix}, c=\begin{pmatrix} 0 \\ 0 \\ 8 \end{pmatrix}$, 且$A=ab^T$, $B=ba^T$, 解矩阵方程$$2B^2 A^2 X = A^4 X + B^4 X + c$$\textcolor{red}{(题目有误,不符合矩阵乘法定义)}

\end{questions}

\pagebreak

\begin{center}
  {\fontsize{30pt}{26pt}\selectfont
    \hypertarget{复数}{复数} \label{复数}
  }
\end{center}
\separator
\vspace{1pt}

\begin{questions}
    \question 已知 $i = \sqrt{-1}$, 若 
    \[\frac{1}{i^{2025}} - \frac{2}{i^{2024}} + \frac{3}{i^{2023}} - \frac{4}{i^{2022}} + \dots - \frac{2024}{i^2} + \frac{2025}{i} = a+bi
    \]其中 $a,b$ 为实数, 求 $a-b$。
    \begin{solution}
        \begin{align*}
        &\frac{1}{i^{2025}} - \frac{2}{i^{2024}} + \frac{3}{i^{2023}} - \frac{4}{i^{2022}} + \cdots - \frac{2024}{i^2} + \frac{2025}{i} \\
        &= \frac{1}{i} - 2 - \frac{3}{i} + 4 + \frac{5}{i} - 6 - \frac{7}{i} + 8 + \cdots + 2024 + \frac{2025}{i} \\
        &= \frac{1}{i}\,(1 + 5 + \cdots + 2025) - (2 + 6 + \cdots + 2022) - \frac{1}{i}\,(3 + 7 + \cdots + 2023) + (4 + 8 + \cdots + 2024) \\
        &= \frac{1}{i} \cdot \frac{2026 \cdot 507}{2} - \frac{2024 \cdot 506}{2} - \frac{1}{i} \cdot \frac{2026 \cdot 506}{2} + \frac{2028 \cdot 506}{2} \\
        &= \frac{1013}{i} + 1012 \\
        &= 1012 - 1013 i \quad \Rightarrow \quad a - b = 2025
        \end{align*}
    \end{solution}

    \question 设 $z$ 是 $1$ 的七次方根,且 $z \neq 1$,试求 $z + z^2 + z^4$的值。
    \begin{solution}
        发现
        \[
        z^7 = 1 \Rightarrow 1 + z + \cdots + z^6 = 0
        \Rightarrow z + z^2 + \cdots + z^6 = -1
        \]
        设$\alpha = z + z^2 + z^4 ,\beta = z^3 + z^5 + z^6$,则
        \[
        \alpha + \beta = z + z^2 + \cdots + z^6 = -1
        \]
        又
        \[
        \alpha\beta = (z + z^2 + z^4)(z^3 + z^5 + z^6) = z^4(1 + z + z^2 + \cdots + z^6 + 2z^3)= z^4(0 + 2z^3) = 2z^7 = 2
        \]
        因此$\alpha, \beta$是方程$x^2 + x + 2 = 0$ 的两根,解得
        \[
        \alpha = z + z^2 + z^4 = \frac{-1 \pm \sqrt{7}i}{2}
        \]
    \end{solution}

    \question 若$z \in \mathbb{C}$满足$z+\dfrac1z=\sqrt 3$,求$z^{2025}+\dfrac{1}{z^{2025}}$的值。
    \begin{solution}
        先求 $z$,
        \[
        z + \frac{1}{z} = \sqrt{3} \quad\Rightarrow\quad z^2 - \sqrt{3} z + 1 = 0 \quad\Rightarrow\quad z = \frac{\sqrt{3}+i}{2} = e^{\frac{\pi i}{6}}
        \]
        于是
        \[
        z^{2025} = e^{\frac{2025 \pi i}{6}} = e^{\frac{3 \pi i}{2}} = -i \quad\Rightarrow\quad z^{2025} + \frac{1}{z^{2025}} = -i + \frac{1}{-i} = 0
        \]
    \end{solution}

    \question 求
    \[
    \left(\frac{\sqrt{3}+1}{2\sqrt{2}} + \frac{\sqrt{3}-1}{2\sqrt{2}} i \right)^6
    \]
    \begin{solution}
    观察
    \[
    \left(\frac{\sqrt{3}+1}{2\sqrt{2}} + \frac{\sqrt{3}-1}{2\sqrt{2}} i \right)^2 
    = \frac{\sqrt{3}}{2} + \frac{1}{2} i 
    = \cos 30^\circ + i \sin 30^\circ.
    \]
    由棣莫弗定理,原式为
    \[
    (\cos 30^\circ + i \sin 30^\circ)^3 = \cos 90^\circ + i \sin 90^\circ = i.
    \]
    \end{solution}

    \question 已知 $a,b,c \in \mathbb{C}$, 且 $a+b+c=a^2+b^2+c^2=3, a^3+b^3+c^3=6$, 试求 \[(a-1)^{2023}+(b-1)^{2023}+(c-1)^{2023}\]的值。
    \begin{solution}
        由
        \[
        3=3^{2}-2(ab+bc+ca), \quad 6-3abc=3(3-3)
        \]
        解得
        \[
        ab+bc+ca=3, abc=2
        \]
        因此 $a,b,c$ 是方程
        \[
        x^{3}-3x^{2}+3x-2=0
        \]
        即
        \[
        (x-1)^{3}=1
        \]
        的根,因此
        \[
        a-1,b-1,c-1 \in \{1,\ \omega,\ \omega^{2}\}, \omega=e^{i\frac{2\pi}{3}}
        \]
        不失一般性,
        \[
        (a-1)^{2023}+(b-1)^{2023}+(c-1)^{2023}
        =1^{2023}+\omega^{2023}+(\omega^{2})^{2023}=1+\omega+\omega^{2}=0
        \]
    \end{solution}

    \question 证明
\[
1 + 2i + 3i^2 + 4i^3 + \dots + (4n+1)i^{4n}=(2n+1) - 2n i, \quad n \in \mathbb{N}.
\]

\begin{solution}
设
\[
S = 1 + 2i + 3i^2 + 4i^3 + \dots + (4n+1)i^{4n}.
\]

两边乘以 $i$:
\[
iS = i + 2i^2 + 3i^3 + 4i^4 + \dots + (4n)i^{4n} + (4n+1)i^{4n+1}.
\]

原式减去该式:
\[
(1-i)S = 1 + i + i^2 + i^3 + \dots + i^{4n} - (4n+1)i^{4n+1}.
\]

右边的几何级数求和(注意 $i^{4n+1} = i$):
\[
(1-i)S = \frac{i^{4n+1}-1}{i-1} - (4n+1)i = 1 - (4n+1)i.
\]

解 $S$:
\begin{align*}
S &= \frac{1 - (4n+1)i}{1-i} = \frac{(1 - (4n+1)i)(1+i)}{(1-i)(1+i)} \\
&= \frac{1 + i - (4n+1)i - (4n+1)i^2}{1 - i^2} \\
&= \frac{1 + i - 4n i - i + 4n + 1}{2} \\
&= \frac{2 + 4n - 4n i}{2} \\
&= 1 + 2n - 2n i \\
&= (2n+1) - 2n i.
\end{align*}
\end{solution}


    \question 考虑极坐标作图 \(z=\sqrt{2}(1+i),\,w=\sqrt{3}-i\),证
    \[
    \tan\frac{\pi}{24}=\sqrt{6}-\sqrt{3}+\sqrt{2}-2
    \]
    \ifprintanswers
    \begin{figure}[H]
    \centering
    \includegraphics[width=0.4\textwidth]{images/image14.png}
    \end{figure}
    \fi
    \begin{solution} 
        如图作$z=\sqrt{2}(1+i),\,w=\sqrt{3}-i,$则$$\arg(z)=\arctan \frac{\sqrt2}{\sqrt2}=\dfrac{\pi}{4},\arg(w)=\arctan \frac{\sqrt3}{-1}=-\dfrac{\pi}{6}$$
        于是 \begin{align*}
        \arg(z+w)&=\angle QOR-|\arg w|\\
        &=\frac{1}{2}\angle POR-|\arg w|\\
        &=\frac{1}{2}(\angle POQ+|\arg w|)-|\arg w|\\
        &=\frac{1}{2}(\frac{\pi}{4}+\frac{\pi}{6})-\frac{\pi}{6}=\frac{\pi}{24}
        \end{align*}
        又$z+w=\sqrt2+\sqrt3+(\sqrt2-1)i$,所以有
        \[
        \tan\frac{\pi}{24}=\frac{\sqrt2-1}{\sqrt2+\sqrt3}=\frac{\sqrt2-1}{\sqrt2+\sqrt3}\cdot\frac{\sqrt3-\sqrt2}{\sqrt3-\sqrt2}=\sqrt{6}-\sqrt{3}+\sqrt{2}-2
        \]
    \end{solution}

    \question 复平面上有三点 $P, Q, R$ 对应三个复数 $z_1, z_2, z_3$,且 $|z_1| = \sqrt{2},|z_2| = \sqrt{5},|z_3| = 3$。若原点 $O$ 为 $\triangle PQR$ 的重心,求 $\Re(\overline{z_1} z_2)$ 。
    \begin{solution}
        设
        \[
        z_1 = a_1 + b_1i, \quad z_2 = a_2 + b_2i, \quad z_3 = a_3 + b_3i
        \]
        由于点 $P, Q, R$ 的重心在原点,故有
        \[
        a_1 + a_2 + a_3 = 0, \quad b_1 + b_2 + b_3 = 0
        \]
        且$a_1^2 + b_1^2 = 2,a_2^2 + b_2^2 = 5,a_3^2 + b_3^2 = 9$,又因为$a_3 = -a_1 - a_2, \ b_3 = -b_1 - b_2,$
        \begin{align*}
        a_3^2 + b_3^2 &= (a_1 + a_2)^2 + (b_1 + b_2)^2\\
        &= a_1^2 + a_2^2 + 2a_1a_2 + b_1^2 + b_2^2 + 2b_1b_2\\
        &= 7 + 2(a_1a_2 + b_1b_2)=9
        \end{align*}
        故
        \[
        \Re(\overline{z_1}z_2)= a_1a_2 + b_1b_2=1
        \]
    \end{solution}

    \question 设 $\omega$ 为复数,且 $|\omega| = 5$。存在一正实数 $\lambda > 1$,使得 $\omega, \omega^2, \lambda\omega$ 这三个复数在复平面上构成一个正三角形,试求 $\lambda$ 的值。
    \ifprintanswers
        \begin{figure}[H]
        \centering
        \includegraphics[width=0.4\textwidth]{images/image19.png}
        \end{figure}
    \fi
    \begin{solution}
        设$O(0, 0),A(\omega),B(\omega^2),C(\lambda \omega)$,则
        \[
        \begin{aligned}
        OA &= |\omega| = 5 \\
        OB &= |\omega^2| = |\omega|^2 = 25 \\
        AB &= AC = |\lambda\omega - \omega| = 5(\lambda - 1) \\
        \end{aligned}
        \]
        又由于 $\triangle ABC$ 是正三角形,因此 $\angle OAB = 180^\circ - 60^\circ = 120^\circ$。
        
        在 $\triangle OAB$中,由余弦定理,
        \[
        \cos \angle OAB = \frac{OA^2 + AB^2 - OB^2}{2 \cdot OA \cdot AB}
        = \frac{5^2 + (5(\lambda - 1))^2 - 25^2}{2 \cdot 5 \cdot 5(\lambda - 1)}=-\frac{1}{2}
        \]
        解得
        \[
        \lambda = \frac{1 + \sqrt{97}}{2}>1
        \]
    \end{solution}

    \question 如图,在坐标平面上有一个半径为 2 的圆,其圆心 $O$ 为原点,且正七边形 $ABCDEFG$ 内接于此圆。若 $A(2,0),P(1,1)$,求 $\overline{PA}\cdot \overline{PB}\cdot\overline{PC}\cdot\overline{PD}\cdot\overline{PE}\cdot\overline{PF}\cdot\overline{PG}$。
    \begin{figure}[H]
        \centering
        \includegraphics[width=0.4\textwidth]{images/image30.png}
    \end{figure}
    \begin{solution}
        设$x^7=2^7$ 的七根分别为 $2,\omega,\omega^2,\dots,\omega^6,$其中 $\omega =2e^{\frac{2\pi i}{7}}$,则
        \[ 
        f(x)=x^7-2^7=(x-2)(x-\omega)(x-\omega^2)\cdots(x-\omega^6)
        \]
        令$x=1+i$,可得
        \[
        \overline{PA}\cdot \overline{PB}\cdot\overline{PC}\cdot\overline{PD}\cdot\overline{PE}\cdot\overline{PF}\cdot\overline{PG}=|f(1+i)|=|(1+i)^7-2^7|=|2i(1+i)-128| =8\sqrt{226}
        \]
    \end{solution}

    \question 已知 $\omega=\cos\dfrac{2\pi}{9}+i\sin\dfrac{2\pi}{9}$,求 $|2-\omega|^2+|2-\omega^2|^2+\cdots+|2-\omega^8|^2$。
    \ifprintanswers
    \begin{figure}[H]
        \centering
        \includegraphics[width=0.5\linewidth]{images/image77.png}
    \end{figure}
    \fi
    \begin{solution}
        即求点 $P(2,0)$ 至单位圆上正九边形各顶点距离的平方和:
        \[
        |2-\omega|^2+|2-\omega^2|^2+\cdots+|2-\omega^8|^2
        =\sum_{n=1}^8 A_nP^2
        \]
        在$\triangle OA_nP$中,由余弦定理,
        \[
        A_nP^2=1^2+2^2-2\cdot 2\cdot 1\cdot \cos\angle A_nOP=5-4\cos(40^\circ\cdot n)
        \]
        于是
        \begin{align*}
        \sum_{n=1}^8 A_nP^2
        &=\sum_{n=1}^8\left(5-4\cos(40^\circ\cdot n)\right)\\
        &=40-4(\cos40^\circ+\cos80^\circ+\cdots+\cos320^\circ)\\
        &=40-8(\cos40^\circ+\cos80^\circ+\cos120^\circ+\cos160^\circ)\\
        &=40-8\left(2\cos60^\circ\cos20^\circ-\frac12+\cos160^\circ\right)\\
        &=40-8\left(\cos20^\circ+\cos160^\circ-\frac12\right)\\
        &=40-8\left(2\cos90^\circ\cos70^\circ-\frac12\right)\\
        &=40-8\left(-\frac12\right)=44
        \end{align*}
    \end{solution}

    \question $\omega^{503}=1,\ \omega \neq 1$,求
        $$
        \frac{\omega^2}{\omega-1}+\frac{\omega^4}{\omega^2-1}+\frac{\omega^6}{\omega^3-1}+\cdots+\frac{\omega^{1004}}{\omega^{502}-1}
        $$

        \begin{solution}
        由 $\omega^{503}=1$ 得
        \[
        (1-\omega)(1+\omega+\omega^2+\cdots+\omega^{502})=0
        \]
        所以
        \[
        \sum_{k=0}^{502}\omega^k=0
        \]
        又
        \[
        \frac{1}{\omega^k-1}+\frac{1}{\omega^{503-k}-1}
        =\frac{1}{\omega^k-1}+\frac{1}{\omega^{-k}-1}
        =\frac{1}{\omega^k-1}+\frac{\omega^k}{1-\omega^k}=-1
        \]
        原式
        \[
        =\sum_{k=1}^{502}\frac{\omega^{2k}}{\omega^k-1}
        =\sum_{k=1}^{502}\left( \frac{\omega^{2k}-1}{\omega^k-1}+\frac{1}{\omega^k-1} \right)
        =\sum_{k=1}^{502} \left( \omega^k+1+\frac{1}{\omega^k-1} \right)
        \]
        \[
        =-1+502+\sum_{k=1}^{502}\frac{1}{\omega^k-1}
        \]
        \[
        =501+\left(\frac{1}{\omega-1}+\frac{1}{\omega^{502}-1}\right)
        +\left(\frac{1}{\omega^2-1}+\frac{1}{\omega^{501}-1}\right)
        +\cdots
        +\left(\frac{1}{\omega^{251}-1}+\frac{1}{\omega^{252}-1}\right)
        \]
        \[
        =501+(-1)\times 251=250
        \]
    \end{solution}

    \question 证明
        \[
        |z+w|^2-|z+\overline{w}|^2=4\Re(z)\Re(w)
        \]
    \begin{solution}
        \textbf{解法一}
        
        设$z = x + iy,\;w = u + iv,$其中$x,y,u,i\in \mathbb{R}$,则
        \begin{align*}
        |z+w|^2 - |z-\overline{w}|^2 
        &= |(x+iy) + (u+iv)|^2 - |(x+iy) - (u-iv)|^2 \\
        &= |(x+u) + i(y+v)|^2 - |(x-u) + i(y+v)|^2 \\
        &= (x+u)^2 + (y+v)^2 - (x-u)^2 - (y+v)^2 \\
        &= 4xu \\
        &= 4 \Re(z) \Re(w)
        \end{align*}
    \end{solution}
    
    \begin{solution}
        \textbf{解法二}
        \begin{align*} 
        |z + w|^2 - |z - w|^2 
        &= (z + w)(\overline{z+w}) - (z - \overline{w})(\overline{z - \overline{w}}) \\
        &= (z + w)(\overline{z}+\overline{w}) - (z - \overline{w})(\overline{z} - w)\\
        &= z\overline{z} + z\overline{w} + w\overline{z} + w\overline{w} - \left( z\overline{z} - zw - \overline{w}\:\overline{z} + w\overline{w} \right) \\
        &= zw+z\overline{w}+w\overline{z}+\overline{w}\:\overline{z}\\
        &=(z+\overline{z})(w+\overline{w})\\
        &=2\Re(z)\cdot 2\Re(w)\\
        &=4 \Re(z) \Re(w)
        \end{align*}
    \end{solution}    
    
    \question 已知 \(z=\cos\theta+i\sin\theta,\ 0\leq\theta<2\pi\),证明:
        \[
        \frac{2}{1+z}=1-i\tan\frac{\theta}{2}
        \]
    \begin{solution}
        运用各位三角恒等式化简
        \begin{align*}
        \frac{2}{1+z} 
        &= \frac{2}{1+\cos\theta+i\sin\theta} \\
        &= \frac{2(1+\cos\theta - i\sin\theta)}{(1+\cos\theta)^2 + \sin^2\theta} \\
        &= \frac{2(1+\cos\theta - i\sin\theta)}{1 + 2\cos\theta + \cos^2\theta + \sin^2\theta} \\
        &= \frac{2(1+\cos\theta - i\sin\theta)}{2 + 2\cos\theta} \\
        &= \frac{1+\cos\theta}{1+\cos\theta} - i\cdot\frac{\sin\theta}{1+\cos\theta} \\
        &= 1 - i \cdot \frac{2\sin\frac{\theta}{2}\cos\frac{\theta}{2}}{2\cos^2\frac{\theta}{2}} \\
        &= 1 - i \tan\frac{\theta}{2}
        \end{align*}
    \end{solution}  
    
    % \question 若 \(m,n\in\mathbb{N}\),证明方程 \(z^m=2-i\) 与 \(z^n=1+i\) 无公共复根。  
    % \begin{solution}
    %     $\forall\;m,n\in\mathbb{N}$,假设存在复数 \(z\),使得
    %     \[
    %     z^m = 2 - i,\quad z^n = 1 + i
    %     \]
    %     取模得到
    %     \[
    %     |z^m| = |2 - i| = \sqrt{2^2 + 1^2} = \sqrt{5},\quad |z^n| = |1 + i| = \sqrt{1^2 + 1^2} = \sqrt{2}
    %     \]
    %     因此,
    %     \[
    %     |z|^m = \sqrt{5} \Rightarrow |z| = 5^{\frac{1}{2m}},\quad |z|^n = \sqrt{2} \Rightarrow |z| = 2^{\frac{1}{2n}}
    %     \]
    %     于是,
    %     \[
    %     5^{\frac{1}{2m}} = 2^{\frac{1}{2n}}
    %     \Rightarrow \left(5^{\frac{1}{2m}}\right)^{2mn} = \left(2^{\frac{1}{2n}}\right)^{2mn}
    %     \Rightarrow 5^n = 2^m
    %     \]
    %     左边是奇数,右边是偶数,矛盾!
    %     \end{solution}
\question
设复数 $z$ 满足 $|z|=1$ 且 $z = e^{i\theta}$,求
\[
\Re \left[ \frac{z(1-\bar{z})}{\bar{z}(1+z)} \right].
\]

\begin{solution}
先将表达式化简:
\begin{align*}
\Re \left[ \frac{z(1-\bar{z})}{\bar{z}(1+z)} \right] 
&= \Re \left[ \frac{z - z\bar{z}}{\bar{z} + z\bar{z}} \right] 
= \Re \left[ \frac{z - |z|^2}{\bar{z} + |z|^2} \right] \\
&= \Re \left[ \frac{z - 1}{\bar{z} + 1} \right] 
= \Re \left[ \frac{e^{i\theta} - 1}{e^{-i\theta} + 1} \right].
\end{align*}

乘以共轭以简化:
\begin{align*}
\Re \left[ \frac{e^{i\theta}-1}{e^{-i\theta}+1} \right] 
&= \Re \left[ \frac{(e^{i\theta}-1)(e^{i\theta}+1)}{(e^{-i\theta}+1)(e^{i\theta}+1)} \right] \\
&= \Re \left[ \frac{e^{i2\theta}-1}{2 + e^{i\theta}+e^{-i\theta}} \right] 
= \Re \left[ \frac{e^{i\theta} (e^{i\theta}-e^{-i\theta})}{2 + 2\cos\theta} \right] \\
&= \Re \left[ \frac{e^{i\theta} \cdot 2i\sin\theta}{2(1+\cos\theta)} \right] 
= \Re \left[ \frac{e^{i\theta} \cdot i\sin\theta}{1+\cos\theta} \right] \\
&= \frac{1}{1+\cos\theta} \Re \left[ i \sin\theta (\cos\theta + i\sin\theta) \right] \\
&= \frac{1}{1+\cos\theta} \Re \left[ i \sin\theta \cos\theta - \sin^2\theta \right] \\
&= -\frac{\sin^2\theta}{1+\cos\theta}.
\end{align*}

使用半角公式化简:
\[
-\frac{\sin^2\theta}{1+\cos\theta} 
= -\frac{(2\sin\frac{\theta}{2}\cos\frac{\theta}{2})^2}{2\cos^2\frac{\theta}{2}} 
= -2 \sin^2 \frac{\theta}{2}.
\]

\[
\therefore \Re \left[ \frac{z(1-\bar{z})}{\bar{z}(1+z)} \right] = -2 \sin^2 \frac{\theta}{2}.
\]
\end{solution}


    
    \question 已知两个相异复数 \(z_1,z_2\),且 \(|z_1|=|z_2|\ne 0\),证明
        \[
        \frac{z_1+z_2}{z_1-z_2}
        \]
        是纯虚数。  
    \begin{solution}
        设$w = \dfrac{z_1 + z_2}{z_1 - z_2}$ ,欲证 \(\overline{w} = -w\)。
        
        因为 \(|z_1| = |z_2| = r\),且 \(z_1, z_2 \ne 0\),根据性质
        \[
        \overline{z} = \frac{r^2}{z}
        \Rightarrow \overline{z_1} = \frac{r^2}{z_1},\quad \overline{z_2} = \frac{r^2}{z_2}
        \]
        
        因此
        \begin{align*}
        \overline{w} 
        &= \overline{\left(\frac{z_1 + z_2}{z_1 - z_2}\right)} 
        = \frac{\overline{z_1} + \overline{z_2}}{\overline{z_1} - \overline{z_2}} 
        = \frac{\dfrac{r^2}{z_1} + \dfrac{r^2}{z_2}}{\dfrac{r^2}{z_1} - \dfrac{r^2}{z_2}} 
        = \frac{\dfrac{1}{z_1} + \dfrac{1}{z_2}}{\dfrac{1}{z_1} - \dfrac{1}{z_2}} 
        = \frac{z_2 + z_1}{z_2 - z_1} = -\frac{z_1 + z_2}{z_1 - z_2} = -w
        \end{align*}
        \end{solution}
    
    % \question 复数 \(z\) 满足关系式:
    %     \[
    %     5(z+i)^n = (4+3i)(1+iz)^n, \quad n \in \mathbb{R}
    %     \]
    %     试证 \(z \in \mathbb{R}\)。
        
    % \begin{solution}
    %     原式写成
    %     \[
    %     5(z+i)^n = i^n (4+3i)(z - i)^n
    %     \]
    %     两边取模有
    %     \begin{align*}
    %     5 |z+i|^n &= |i^n| \cdot |4+3i| \cdot |z-i|^n \\
    %     5 |z+i|^n &= 1 \cdot \sqrt{4^2 + 3^2} \cdot |z - i|^n = 5 |z - i|^n\\
    %     |z+i|^n &= |z-i|^n
    %     \end{align*}
    %     设 \(z = x + iy\),代入得
    %     \begin{align*}
    %     |x + i(y+1)|^n &= |x + i(y-1)|^n \\
    %     \left(\sqrt{x^2 + (y+1)^2}\right)^n &= \left(\sqrt{x^2 + (y-1)^2}\right)^n
    %     \end{align*}
    %     故
    %     \[
    %     x^2 + (y+1)^2 = x^2 + (y-1)^2 \Rightarrow y = 0
    %     \]
    %     $\therefore \;z \in \mathbb{R}$
    % \end{solution}
    
    % \question 已知复数 \(z, w\) 满足 \(|z| = |w| = 1\),证明:
    %     \[
    %     \frac{z + w}{1 + zw} \in \mathbb{R}
    %     \]
        
    % \begin{solution}
    %     设
    %     \[
    %     v = \frac{z + w}{1 + zw}
    %     \]
    %     即证$v = \overline{v}$:
    %     \begin{align*}
    %     \overline{v} &= \overline{\left(\frac{z + w}{1 + zw}\right)} = \frac{\overline{z} + \overline{w}}{1 + \overline{z} \cdot \overline{w}}
    %     \end{align*}
        
    %     由于 \[|z| = |w| = 1 \Rightarrow \overline{z} = \frac{1}{z},\; \overline{w} = \frac{1}{w}
    %     \]
    %     则
    %     \begin{align*}
    %     \overline{v} &= \frac{\frac{1}{z} + \frac{1}{w}}{1 + \frac{1}{zw}} 
    %     = \frac{w + z}{zw + 1} = v
    %     \end{align*}
        
    %     \(\therefore v \in \mathbb{R}\)。
        
    % \end{solution}

    % \question 若 \(z,w\in\mathbb{C},\,|w|>1,\,z+w=|z|w\),试以 \(|w|\) 表示 \(|z|\)。  

    % \begin{solution}
    %     由题意
    %     \[
    %     z + w = |z|w \Rightarrow z = (|z| - 1)w
    %     \]
    %     设 \(w = a + bi,\)
    %     \[
    %     z = (|z| - 1)(a + bi) = a(|z| - 1) + b(|z| - 1)i
    %     \]
    %     取模得
    %     \[
    %     |z| = \left|a(|z| - 1) + b(|z| - 1)i\right| = \sqrt{a^2(|z| - 1)^2 + b^2(|z| - 1)^2}
    %     \]
    %     故
    %     \[
    %     |z| = \pm |w|(|z| - 1)
    %     \Rightarrow |z| = \pm |w||z| \mp |w|
    %     \]
    %     即
    %     \[
    %     |z| = \frac{\mp |w|}{1 \mp |w|}
    %     \]
    % \end{solution}

    \question 设复数 $z_1, z_2$ 满足 $|z_1|=2$, $|z_2|=3$, $|z_1+z_2|=4$, 求 $\dfrac{z_1}{z_2}$。 
    \begin{solution}
        设 $w = \dfrac{z_1}{z_2}$,则 $z_1 = w z_2$。由
        \[
        |z_1 + z_2|^2 = (z_1 + z_2)(\overline{z_1} + \overline{z_2}) = |z_1|^2 + |z_2|^2 + z_1\overline{z_2} + z_2\overline{z_1}
        \]
        代入已知得
        \[
        z_1\overline{z_2} + z_2\overline{z_1} = 3
        \]
        将 $z_1 = w z_2$ 代入得
        \[
        wz_2 \cdot \overline{z_2} + z_2 \cdot \overline{w z_2} = w |z_2|^2 + \overline{w} |z_2|^2 = (w + \overline{w}) \cdot 9=3 \Rightarrow w + \overline{w} = \frac{1}{3}
        \]
        设 $w = x + iy,x,y\in \mathbb{R}$,则 $w + \overline{w} = 2x = \dfrac{1}{3} \Rightarrow x = \dfrac{1}{6}$;又因 $|w| = \dfrac{|z_1|}{|z_2|} = \dfrac{2}{3}$,所以
        \[
        |w|^2 = \frac{1}{36} + y^2 = \frac{4}{9}
        \Rightarrow y = \pm \frac{\sqrt{15}}{6} \Rightarrow   \frac{z_1}{z_2} = \frac{1}{6} \pm i \frac{\sqrt{15}}{6}
        \]
    \end{solution}

    \question 若复数 $z$ 使得$\dfrac{z-3i}{z+i}$ 为负实数, $\dfrac{z-3}{z+1}$ 为纯虚数,求 $z$。
    \begin{solution}
        设 $z=a+bi,a,b\in \mathbb{R}$,由于 $$\frac{z-3i}{z+i}=\frac{a+(b-3)i}{a+(b+1)i}$$ 为负实数,则 $$(a+(b-3)i)(a-(b+1)i)<0$$
        所以 $a^{2}+(b-3)(b+1)<0$ 且 $-4a=0$,即 $a=0$ 且 $b-3,b+1$ 异号; 又此时
        $$\frac{z-3}{z+1}=\frac{-3+bi}{1+bi}$$为纯虚数,故 
        $$\Re((-3+bi)(1-bi))=b^{2}-3=0$$又 $b-3,b+1$异号知 $b=\sqrt{3}$,所以 $z=\sqrt{3}i$.
    \end{solution}
    
    \question  已知虚数 $z$ 使得 $z_1 = \dfrac{z}{1+z^2}$ 和 $z_2 = \dfrac{z^2}{1+z}$ 都是实数,求 $z$。
    
    \begin{solution}
        由已知$(z^2 + 1)z_1 = z, (1 + z)z_2 = z^2$,联立两式
        \[
        z = (z^2 + 1)z_1 = ((1+z)z_2 + 1)z_1
        \]
        整理得
        \[
        z_1 + z_1 z_2 = z(1 - z_1 z_2)
        \]
        由于 $z$ 为虚数, $z_1 + z_1 z_2$ 与 $1 - z_1 z_2$ 为实数,故
        \[
        1 - z_1 z_2 = 0 \Rightarrow z_1 z_2 = 1
        \]
        代入 $z_1 = \dfrac{z}{1 + z^2},z_2 = \dfrac{z^2}{1 + z}$ 得
        \[
        z_1 z_2 = \frac{z}{1+z^2} \cdot \frac{z^2}{1+z} = \frac{z^3}{(1+z)(1+z^2)} = 1
        \]
        给出
        \[
        z^3 = (1+z)(1+z^2)
        \Rightarrow z^2 + z + 1 = 0
        \]
        即
        \[
        z = -\frac{1}{2} \pm \frac{\sqrt{3}}{2}i
        \]
    \end{solution}
    
    \question 已知复数 \(z\ne 0\),满足方程 \(z^2=z+i|z|\),求 \(|z|\) 的值。
    \begin{solution}
    设 $z = x + iy,\;x,y\in \mathbb{R},$则
    \[
    z^2 = z + i|z| \Rightarrow x^2 - y^2 + 2ixy  = x + i\left(y + \sqrt{x^2 + y^2}\right)
    \]
    比较实部与虚部得
    \[
    x^2 - y^2 = x \tag{1}
    \]
    \[
    2xy = y + \sqrt{x^2 + y^2} \tag{2}
    \]
    将(2)移项并平方
    \[
    y^2(2x - 1)^2 = x^2 + y^2 
    \]
    由(1)得
    \[
    (x^2 - x)(2x - 1)^2 = 2x^2 - x \Rightarrow x^2(2x-3)(2x-1)=0
    \]
    解得 $x=0,\dfrac{1}{2},\dfrac{3}{2}$,其中$x=0$不合理$(z\neq 0)$,且$x=\dfrac{1}{2}$不合理$(y^2=-\dfrac{1}{4}<0)$,故
    \[
    x = \dfrac{3}{2},\; y^2 = \dfrac{3}{4}\Rightarrow|z| = \sqrt{\frac{9}{4} + \frac{3}{4}} = \sqrt3
    \]
    \end{solution}

    \question 确定所有满足 $|a|=|b|=1$ 且 $a + b + a\bar{b} \in \mathbb{R}$ 的复数对 $(a,b)$。

\begin{solution}
方法一:设 $a = e^{ix}$,$b = e^{iy}$,其中 $x, y \in [0, 2\pi)$。利用欧拉公式以及恒等式
\[
\sin x + \sin y = 2 \sin \frac{x+y}{2} \cos \frac{x-y}{2}, \quad \sin(x-y) = 2 \sin \frac{x-y}{2} \cos \frac{x-y}{2},
\]
得到
\begin{align*}
\operatorname{Im}(a+b+a\bar{b}) &= (\sin x + \sin y) + \sin(x-y) \\
&= 2 \sin \frac{x+y}{2} \cos \frac{x-y}{2} + 2 \sin \frac{x-y}{2} \cos \frac{x-y}{2} \\
&= 2 \left(\sin \frac{x+y}{2} + \sin \frac{x-y}{2}\right) \cos \frac{x-y}{2} \\
&= 4 \sin \frac{x}{2} \cos \frac{y}{2} \cos \frac{x-y}{2}.
\end{align*}

因此,$a+b+a\bar{b}$ 为实数当且仅当
\[
\sin \frac{x}{2} = 0, \quad \cos \frac{y}{2} = 0, \quad \text{或} \quad \cos \frac{x-y}{2} = 0,
\]
分别对应
\[
x = 2k\pi, \quad y = (2k+1)\pi, \quad x = y + (2k+1)\pi, \quad k \in \mathbb{Z}.
\]

所以解集为
\[
(a,b) = (1, b), \quad (a,-1), \quad (a,-a), \quad \text{其中 } |a|=|b|=1.
\]

方法二:注意到
\[
a+b+a\bar{b}\in\mathbb{R} \iff 1+a+b+a\bar{b} \in \mathbb{R}.
\]

设 $c \in \mathbb{C}$ 且 $a=c^2$,则
\[
\overline{c}(1+a+b+a\bar{b}) = \overline{c} + \overline{c}c^2 + \overline{c}b + \overline{c}c^2\overline{b} = \overline{c}+c+\overline{c}b+c\overline{b} \in \mathbb{R},
\]
利用 $\overline{c}c = 1$ 以及 $z+\overline{z} \in \mathbb{R}$。因此要么 $c \in \mathbb{R}$,要么 $1+a+b+a\overline{b}=0$。第一种情况 $c=\pm 1$,于是 $a=1$。第二种情况将等式因式分解为
\[
(a+b)(1+\overline{b}) = 1+a+b+a\overline{b}=0,
\]
得到 $a=-b$ 或 $b=-1$。所以得到的三族解为:
\[
(a,b) = (1,b), \quad (a,-1), \quad (a,-a), \quad |a|=|b|=1.
\]
\end{solution}


    \question 已知两复数 $z_1, z_2$ 满足 $|z_1 - (3 + 3i)| = 2,|iz_2 - 1| = 1$,求 $|z_1 - z_2|$ 的最小值。
    \begin{solution}
        由$|z_1 - (3 + 3i)| = 2,P(z_1)$在圆
        \[
        C_1 : (x - 3)^2 + (y - 3)^2 = 2^2 
        \]
        上,设$z_2 = a + bi$,则$iz_2 - 1 = (-1 - b) + ai$,故$ |iz_2 - 1| = 1 $意味$Q(z_2)$在圆 
        \[
        C_2 : x^2 + (y + 1)^2 = 1 
        \]
        上,故$|z_1 - z_2|=PQ$ 的最小值为
        \[
        \text{两圆心距离} - \text{两圆半径和} = \sqrt{3^2 + 4^2} - (2 + 1) = 2
        \]
    \end{solution}

    \question 设 $\alpha$ 为 $\left|(z-1)\left(z+\dfrac{1}{2}\right)\right|$ 在圆盘上 $\{z \in \mathbb{C}:|z| \leq 1\}$ 的最大值,则 $\alpha^2=$ ?
    \ifprintanswers
    \begin{figure}[H]
        \centering
        \includegraphics[width=0.5\linewidth]{images/image75.png}
    \end{figure}
    \fi
    \begin{solution}
        如图,$P$ 在圆周上,$A(1,0),B\left(-\dfrac12,0\right)$,则
        \[
        PA \cdot PB = \left|(z-1)\left(z+\frac{1}{2}\right)\right|
        \]
        设$\theta=\angle POA$,在$\triangle POA$及$\triangle POB$中,由余弦定理,
        \[
        PA^2 = 1^2+1^2-2\cdot 1 \cdot 1 \cdot \cos \theta= 2-2\cos\theta
        \]
        \[
        PB^2 = 1^2+ \left(\frac{1}{2}\right)^2 -2\cdot 1 \cdot \left(\frac{1}{2}\right) \cdot \cos(\pi-\theta)=\cos\theta+\frac54
        \]
        因此
        \[
        f(\theta) = (2-2\cos\theta)\left(\cos\theta+\frac54\right) = -2\cos^2\theta - \frac12\cos\theta + \frac52  
        \]
        且
        \[
        \alpha^2 = f\left(-\frac18\right) = \frac{81}{32}
        \]
    \end{solution}

    \question 设有一虚部不为零的复数 $z$, 其长度为 $2$, 且在复数平面上与 $-2$ 及 $z^2$ 刚好在同一直线上, 与 $1$ 及 $z^3$ 也同在另一直线上, 试求以 $z,z^2,z^3$ 所围成的三角形面积。
    \begin{solution}
        由 $1,z,z^3$ 共线可得
        \[
        \frac{z^3-z}{z-1}=k \in \mathbb R \Rightarrow z^2+z-k=0 \Rightarrow z=\frac{-1\pm \sqrt{\,1+4k\,}}{2}
        \]
        其中$z$实部为$-\dfrac{1}{2}$,由 $-2,z,z^2$ 共线可得
        \[
        \frac{z^2-z}{z+2}=t \in \mathbb R
        \Rightarrow z^2-(t+1)z-2t=0
        \Rightarrow z=\frac{t+1\pm \sqrt{(t+1)^2+8t}}{2}
        \]
        因此
        \[
        \frac{t+1}{2}=-\frac{1}{2}\Rightarrow t=-2
        \]
        代入得
        \[
        z=\frac{-1\pm \sqrt{15}\, i}{2}.
        \]
        取$z=\dfrac{-1+\sqrt{15}\, i}{2}$,则
        \[
        z^2=\frac{-7-\sqrt{15}\, i}{2},\quad
        z^3=\frac{11-3\sqrt{15}\, i}{2}.
        \]
        $z,z^2,z^3$在复平面上的坐标为
        \[
        A(z)=\left(-\frac12,\frac{\sqrt{15}}{2}\right),\quad
        B(z^2)=\left(-\frac72,-\frac{\sqrt{15}}{2}\right),\quad
        C(z^3)=\left(\frac{11}{2},-\frac{3\sqrt{15}}{2}\right).
        \]
        故所求面积为
        \[
        \frac{1}{2}
        \begin{vmatrix}
        -\dfrac12 & \dfrac{\sqrt{15}}{2} & 1\\[6pt]
        -\dfrac72 & -\dfrac{\sqrt{15}}{2} & 1\\[6pt]
        \dfrac{11}{2} & -\dfrac{3\sqrt{15}}{2} & 1
        \end{vmatrix}
        =6\sqrt{15}
        \]
    \end{solution}

    \question 若实数 $m,n$ 使得关于 $x$ 的方程 $x^3 + mx + n = 0$ 有模为 $3$ 的虚根,求 $m+n$ 的取值范围。
    \begin{solution}
        由于系数皆为实数,设三根为 $a + bi,\ a - bi,\ -2a$,其中 $a,b \in \mathbb{R},\,a^2 + b^2 = 9,\ b \ne 0$。
        
        由韦达定理,
        \[
        m = (a + bi)(a - bi) + (-2a)(a + bi + a - bi)
        = 9 - 4a^2
        \]
        \[
        n=(a + bi)(a - bi)(-2a)= -18a 
        \]
        所以
        \[
        m + n = 9 - 4a^2 + 18a = -4\left(a - \dfrac{9}{4} \right)^2 + \dfrac{117}{4}
        \]
        又$a^2<9,-3<a<3$,则\[
        m+n \in \left(-81,\ \dfrac{117}{4}\right]
        \]
    \end{solution}

    \question 关于 $z$ 的方程 $z^{n+1} - \sqrt{3}z^n - 1 = 0$ 存在一个模为 $1$ 的虚根,求正整数 $n$ 的最小值。
    \begin{solution}
        设该虚根为 $z_0= e^{i\theta} = \cos\theta + i\sin\theta$,有        
        \[
        z_0^n(z_0 - \sqrt{3}) = 1
        \]
        两边取模得
        \[
         |z_0^n||z_0 - \sqrt{3}| = 1
        \]
        由于 $|z_0^n| = 1$,从而
        \[
        |z_0 - \sqrt{3}| = 1
        \]
        将$z_0= e^{i\theta} = \cos\theta + i\sin\theta$代入得
        \[
        |(\cos\theta - \sqrt{3}) + i\sin\theta| = 1
        \]
        于是
        \[
        (\cos\theta - \sqrt{3})^2 + \sin^2\theta = 1 \Rightarrow \cos\theta = \frac{\sqrt{3}}{2}
        \]
        得 $\theta = \dfrac{\pi}{6}$,即 $z_0 = e^{i\frac{\pi}{6}}$,代入 $z_0^n(z_0 - \sqrt{3}) = 1$ 得:
        \[
        \left(e^{i\frac{\pi}{6}}\right)^n \left(e^{i\frac{\pi}{6}} - \sqrt{3}\right) =
        e^{i\frac{n\pi}{6}} \left( \frac{\sqrt{3}}{2} + \frac{1}{2}i - \sqrt{3} \right) = e^{i\frac{n\pi}{6}} \left( -\frac{\sqrt{3}}{2} + \frac{1}{2}i \right) = 1
        \]
        注意到$-\frac{\sqrt{3}}{2} + \frac{1}{2}i = e^{i\frac{5\pi}{6}},$变为
        \[
        e^{i\frac{n\pi}{6}} \cdot e^{i\frac{5\pi}{6}}= e^{i\frac{(n+5)\pi}{6}}= 1 \Rightarrow \frac{(n+5)\pi}{6} = 2k\pi \Rightarrow n = 12k - 5
        \]
        令 $k=1$ 得最小正整数解为 $n = 7$.
    \end{solution}

    \question 已知复数 \(z_1, z_2, z_3\) 满足
    \[
    \begin{cases}
    |z_1| = |z_2| = |z_3| = 1 \\[1mm]
    \displaystyle \frac{z_1}{z_2} + \frac{z_2}{z_3} + \frac{z_3}{z_1} = 1
    \end{cases}
    \]
    求 \(|z_1 + 2 z_2 + 3 z_3|\) 最大可能值。
    \begin{solution}
        设存在一三次多项式,根为 \(\dfrac{z_1}{z_2}, \dfrac{z_2}{z_3}, \dfrac{z_3}{z_1}\),则
        \[
        \frac{z_1}{z_2} + \frac{z_2}{z_3} + \frac{z_3}{z_1} = 1 
        \]
        \[
        \frac{z_1}{z_2} \cdot \frac{z_2}{z_3} \cdot \frac{z_3}{z_1} = 1
        \]
        \begin{align*}
        &\frac{z_1}{z_2} \cdot \frac{z_2}{z_3} + \frac{z_2}{z_3} \cdot \frac{z_3}{z_1} + \frac{z_3}{z_1} \cdot \frac{z_1}{z_2} = \frac{z_1}{z_3} + \frac{z_2}{z_1} + \frac{z_3}{z_2} \\
        &= \overline{\left(\frac{z_3}{z_1}\right)} + \overline{\left(\frac{z_1}{z_2}\right)} + \overline{\left(\frac{z_2}{z_3}\right)} = \overline{\left(\frac{z_1}{z_2} + \frac{z_2}{z_3} + \frac{z_3}{z_1}\right)} = 1
        \end{align*}
        因此该多项式为
        \[
        f(x) = x^3 - x^2 + x - 1 = (x^2 + 1)(x - 1),
        \]
        根为 \(x = 1, \pm i\);为使 \(|z_1 + 2 z_2 + 3 z_3|\) 最大,取
        \[
        \frac{z_2}{z_3} = 1, \quad \frac{z_3}{z_1} = i, \quad \frac{z_1}{z_2} = -i,
        \]
        则
        \[
        z_2 = z_3 = 1, \quad z_1 = -i,
        \]
        所以
        \[
        |z_1 + 2 z_2 + 3 z_3| = |-i + 5| = \sqrt{26}
        \]
    \end{solution}
\question 设 $z$ 为复数,且满足 $|z+1|>2$。证明
\[
|z^3+1|>1.
\]

\begin{solution}
注意到
\[
z^3+1=(z+1)(z^2-z+1),
\]
因此只需证明
\[
|z^2-z+1|\ge \frac{1}{2}.
\]

设
\[
z+1=re^{i\varphi},
\]
其中 $r=|z+1|>2$,$\varphi$ 为某个实数。于是
\begin{align*}
z^2-z+1
&=(re^{i\varphi}-1)^2-(re^{i\varphi}-1)+1 \\
&=r^2e^{2i\varphi}-3re^{i\varphi}+3.
\end{align*}

于是
\begin{align*}
|z^2-z+1|^2
&=(r^2e^{2i\varphi}-3re^{i\varphi}+3)
(r^2e^{-2i\varphi}-3re^{-i\varphi}+3) \\
&=r^4+9r^2+9-(6r^3+18r)\cos\varphi+6r^2\cos 2\varphi \\
&=r^4+9r^2+9-(6r^3+18r)\cos\varphi+6r^2(2\cos^2\varphi-1) \\
&=12\left(r\cos\varphi-\frac{r^2+3}{4}\right)^2
+\frac{1}{4}(r^2-3)^2.
\end{align*}

由于 $r>2$,可得
\[
|z^2-z+1|^2>\frac{1}{4},
\]
从而
\[
|z^2-z+1|>\frac{1}{2}.
\]

因此
\[
|z^3+1|=|z+1|\,|z^2-z+1|>2\cdot\frac{1}{2}=1.
\]
证毕。
\end{solution}

    \question 复数 $z_1,z_2,z_3,z_4,z_5$ 满足
        \begin{equation*}
        \begin{cases}
        \;|z_1| \leq 1 \\
        \;|z_2| \leq 1 \\
        \;|2z_3 - (z_1 + z_2)| \leq |z_1 - z_2| \\
        \;|2z_4 - (z_1 + z_2)| \leq |z_1 - z_2| \\
        \;|2z_5 - (z_3 + z_4)| \leq |z_3 - z_4|
        \end{cases}
        \end{equation*}
        求 $|z_5|$ 的最大值.
    \begin{solution}
        由\[
        |z_1 - z_2| \geq |2z_3 - (z_1 + z_2)| \geq |2|z_3| - |z_1 + z_2||
        \] 
        我们有
        \[
        |z_1 + z_2| - |z_1 - z_2| \leq 2|z_3| \leq |z_1 + z_2| + |z_1 - z_2|
        \] 
        由AM-QM不等式,
        \[
        |z_3| \leq \frac{|z_1 + z_2| + |z_1 - z_2|}{2} \leq \sqrt{\frac{|z_1 + z_2|^2 + |z_1 - z_2|^2}{2}} = \sqrt{\frac{2|z_1|^2 + 2|z_2|^2}{2}} \leq \sqrt{1^2 + 1^2} = \sqrt{2}
        \] 
        同理可得 $|z_4| \leq \sqrt{2}$,故
        $$ |z_5| \leq \frac{|z_3 + z_4|}{2} \leq \frac{|z_3| + |z_4|}{2} \leq \frac{\sqrt{2} + \sqrt{2}}{2} = \sqrt{2}$$ 
        考虑等号何时成立:当$z_1=1, z_2=i$时, \(z_{3}=1+i\);当$z_1=-1,z_2=i$时, \(z_{4}=-1+i\); 当$z_3=1+i, z_4=-1+i$时, \(z_{5}=2i\),这时 \(|z_{5}|=2\). 故$|z_5|_{\max}=2$.
    \end{solution}
    
    \question 设复数 $z$ 满足 $|z|=1$,则 $|z^7 + \bar{z}^5 - 3z^3 - 3\bar{z}|$ 的最大值为
    \begin{solution}
        试 $z = \frac{1+i}{\sqrt{2}}$,算得值为:
        \[
        |z^7 + \bar{z}^5 - 3z^3 - 3\bar{z}| = |2i + 2 - 6i - 6| = |-4i - 4| = \sqrt{16 + 16} = \boxed{4\sqrt{2}}
        \]
        \textcolor{red}{(待解)(设$z=\cos \theta + i \sin \theta,...$感觉是尝试因式分解再AM-GM)}

    \end{solution}

    \question 证明:
    \begin{parts}
        \part 如果 $z = \cos \theta + i \sin \theta$,则
\[
z^n + z^{-n} = 2\cos n\theta, \quad z^n - z^{-n} = 2i \sin n\theta.
\]

\begin{solution}
由 De Moivre 定理:
\[
z^n = \cos n\theta + i \sin n\theta, \quad z^{-n} = \cos(-n\theta) + i \sin(-n\theta) = \cos n\theta - i \sin n\theta.
\]

相加得到:
\[
z^n + z^{-n} = (\cos n\theta + i \sin n\theta) + (\cos n\theta - i \sin n\theta) = 2\cos n\theta.
\]

相减得到:
\[
z^n - z^{-n} = (\cos n\theta + i \sin n\theta) - (\cos n\theta - i \sin n\theta) = 2i \sin n\theta.
\]
\end{solution}

\part 证明
\[
16 \sin^5\theta = \sin 5\theta - 5 \sin 3\theta + 10 \sin \theta, \quad 32 \cos^6\theta = \cos 6\theta + 6 \cos 4\theta + 15 \cos 2\theta + 10.
\]

\begin{solution}
对于第一个恒等式,使用 $2i \sin \theta = z - z^{-1}$:
\begin{align*}
(2i \sin \theta)^5 &= (z - z^{-1})^5 \\
32 i^5 \sin^5 \theta &= z^5 - 5 z^3 z^{-2} + 10 z z^{-4} - 10 z^2 z^{-3} + 5 z z^{-4} - z^{-5} \\
32 i \sin^5 \theta &= (z^5 - z^{-5}) - 5(z^3 - z^{-3}) + 10 (z - z^{-1}) \\
32 i \sin^5 \theta &= 2i \sin 5\theta - 10 i \sin 3\theta + 20 i \sin \theta \\
16 \sin^5 \theta &= \sin 5\theta - 5 \sin 3\theta + 10 \sin \theta.
\end{align*}

对于第二个恒等式,使用 $2 \cos \theta = z + z^{-1}$:
\begin{align*}
(2 \cos \theta)^6 &= (z + z^{-1})^6 \\
64 \cos^6 \theta &= (z^6 + z^{-6}) + 6(z^4 + z^{-4}) + 15 (z^2 + z^{-2}) + 20 \\
64 \cos^6 \theta &= 2 \cos 6\theta + 12 \cos 4\theta + 30 \cos 2\theta + 20 \\
32 \cos^6 \theta &= \cos 6\theta + 6 \cos 4\theta + 15 \cos 2\theta + 10.
\end{align*}
\end{solution}
\end{parts}

\question 证明:
\[
\cos^5\theta \sin^3\theta = \frac{1}{128}\Bigl(6\sin 2\theta + 2\sin 4\theta - 2\sin 6\theta - \sin 8\theta\Bigr)
\]

\begin{solution}
设 $z = \cos\theta + i \sin\theta$,则
\[
z^n = \cos n\theta + i\sin n\theta, \quad z^{-n} = \cos n\theta - i\sin n\theta,
\]
并且
\[
z^n + z^{-n} = 2\cos n\theta, \quad z^n - z^{-n} = 2i \sin n\theta.
\]

使用 $2\cos\theta = z + z^{-1}$ 得:
\begin{align*}
(2\cos\theta)^5 &= (z + z^{-1})^5 \\
32 \cos^5\theta &= z^5 + 5 z^4 z^{-1} + 10 z^3 z^{-2} + 10 z^2 z^{-3} + 5 z z^{-4} + z^{-5} \\
32 \cos^5\theta &= (z^5 + z^{-5}) + 5(z^3 + z^{-3}) + 10(z + z^{-1}) \\
16 \cos^5\theta &= \cos 5\theta + 5\cos 3\theta + 10\cos\theta.
\end{align*}

利用 $4\sin^3\theta = 3\sin\theta - \sin 3\theta$,得到
\[
16\cos^5\theta \cdot 4 \sin^3\theta = (\cos 5\theta + 5\cos 3\theta + 10\cos\theta)(3\sin\theta - \sin 3\theta).
\]

展开并使用和差化积公式:
\begin{align*}
64 \cos^5\theta \sin^3\theta &= 3\cos 5\theta \sin\theta - \cos 5\theta \sin 3\theta + 15 \cos 3\theta \sin\theta - 5 \cos 3\theta \sin 3\theta \\
&\quad + 30 \cos\theta \sin\theta - 10 \cos\theta \sin 3\theta \\
&= \frac{3}{2}(\sin 6\theta - \sin 4\theta) - \frac{1}{2}(\sin 8\theta - \sin 2\theta) + \frac{15}{2}(\sin 4\theta - \sin 2\theta) \\
&\quad - \frac{5}{2} \sin 6\theta + \frac{30}{2} \sin 2\theta - \frac{10}{2} \sin 4\theta \\
128 \cos^5\theta \sin^3\theta &= 6 \sin 2\theta + 2 \sin 4\theta - 2 \sin 6\theta - \sin 8\theta.
\end{align*}

因此
\[
\cos^5\theta \sin^3\theta = \frac{1}{128}\Bigl(6\sin 2\theta + 2\sin 4\theta - 2\sin 6\theta - \sin 8\theta\Bigr).
\]
\end{solution}

    \question 定义无穷级数
        \[
        C=\cos\theta+\frac{1}{2}\cos5\theta+\frac{1}{4}\cos9\theta+\cdots
        \]
        \[
        S=\sin\theta+\frac{1}{2}\sin5\theta+\frac{1}{4}\sin9\theta+\cdots
        \]
        证明
        \[
        C+iS = \frac{2e^{i\theta}}{2 - e^{4i\theta}}, \quad 
        S = \frac{4\sin\theta + 2\sin3\theta}{5 - 4\cos4\theta}
        \]
        
    \begin{solution}
        记
        \begin{align*}
        C+iS &= \cos\theta + i\sin\theta + \frac{1}{2}(\cos5\theta + i\sin5\theta) + \frac{1}{4}(\cos9\theta + i\sin9\theta) + \cdots\\
        &= e^{i\theta} + \frac{1}{2}e^{i5\theta} + \frac{1}{4}e^{i9\theta} + \cdots
        = \sum_{k=0}^{\infty} \frac{1}{2^k} e^{i(4k+1)\theta}
        = e^{i\theta} \sum_{k=0}^{\infty} \left(\frac{e^{4i\theta}}{2}\right)^k        \end{align*}
        这是一个公比为 \(\dfrac{e^{4i\theta}}{2}\)的等比级数,其模为 \(\dfrac{1}{2}<1\),因此级数收敛:
        \[
        C+iS = \frac{e^{i\theta}}{1 - \frac{1}{2}e^{4i\theta}} = \frac{2e^{i\theta}}{2 - e^{4i\theta}}
        \]
        又
        \[
        C+iS =\frac{2e^{i\theta}}{2 - e^{4i\theta}} \cdot \frac{2 - e^{-4i\theta}}{2 - e^{-4i\theta}} = \frac{4e^{i\theta} - 2e^{-3i\theta}}{|2 - e^{4i\theta}|^2}=\frac{(4\cos\theta - 2\cos3\theta) + i(4\sin\theta + 2\sin3\theta)}{(2 - \cos4\theta)^2 + \sin^2 4\theta}
        \]
        因此
        \[
        S = \Im(C + iS) = \frac{4\sin\theta + 2\sin3\theta}{5 - 4\cos4\theta}
        \]
       
    \end{solution}

    \question
证明下列无穷级数的和:
\[
\sum_{n=1}^{\infty} (-1)^{n+1} \frac{\sin(n\theta)}{n!} = e^{-\cos\theta}\sin(\sin\theta).
\]

\begin{solution}
\textbf{步骤 1:定义两个级数 $C$ 与 $S$}

\[
C = \sum_{n=1}^{\infty} (-1)^{n+1} \frac{\cos(n\theta)}{n!},\quad 
S = \sum_{n=1}^{\infty} (-1)^{n+1} \frac{\sin(n\theta)}{n!}.
\]

\medskip
\textbf{步骤 2:使用复数形式合并}

\[
C+iS = \sum_{n=1}^{\infty} (-1)^{n+1} \frac{\cos(n\theta)+i\sin(n\theta)}{n!} 
= \sum_{n=1}^{\infty} (-1)^{n+1} \frac{(e^{i\theta})^n}{n!}.
\]

\medskip
\textbf{步骤 3:识别指数级数}

\[
C+iS = e^{i\theta} - \frac{(e^{i\theta})^2}{2!} + \frac{(e^{i\theta})^3}{3!} - \cdots = \sum_{n=1}^{\infty} \frac{(-1)^{n+1} (e^{i\theta})^n}{n!}.
\]

利用指数函数展开式:
\[
e^{-z} = 1 - z + \frac{z^2}{2!} - \frac{z^3}{3!} + \cdots \implies \sum_{n=1}^{\infty} \frac{(-1)^{n+1} z^n}{n!} = 1 - e^{-z}.
\]

取 $z = e^{i\theta}$,得到:
\[
C+iS = 1 - e^{-e^{i\theta}} = 1 - e^{-(\cos\theta + i\sin\theta)} = 1 - e^{-\cos\theta} e^{-i\sin\theta}.
\]

\medskip
\textbf{步骤 4:分离实部和虚部}

\[
C+iS = 1 - e^{-\cos\theta} [\cos(\sin\theta) - i \sin(\sin\theta)] 
= [1 - e^{-\cos\theta}\cos(\sin\theta)] + i [e^{-\cos\theta}\sin(\sin\theta)].
\]

\medskip
\textbf{步骤 5:取虚部得到原级数和}

\[
\sum_{n=1}^{\infty} (-1)^{n+1} \frac{\sin(n\theta)}{n!} = \Im(C+iS) = e^{-\cos\theta}\sin(\sin\theta).
\]

\end{solution}

\question
求下列多项式方程的所有实数解,并在适当情况下用精确的三角形式表示:
\[
x^7-7x^6-21x^5+35x^4+35x^3-21x^2-7x+1=0.
\]

\begin{solution}
注意到系数呈现正负交替的规律,并且与二项式系数有关,因此考虑利用三角恒等式进行处理。

设
\[
\cos\theta=C,\quad \sin\theta=S,
\]
则
\[
C+iS=\cos\theta+i\sin\theta.
\]
由棣莫弗定理,
\[
(C+iS)^7=\cos7\theta+i\sin7\theta.
\]
另一方面,展开得
\begin{align*}
(C+iS)^7
&=C^7+7iC^6S-21C^5S^2-35iC^4S^3 \\
&\quad+35C^3S^4+21iC^2S^5-7CS^6-iS^7.
\end{align*}

比较实部与虚部,得到
\begin{align*}
\cos7\theta&=C^7-21C^5S^2+35C^3S^4-7CS^6,\\
\sin7\theta&=7C^6S-35C^4S^3+21C^2S^5-S^7.
\end{align*}

于是
\[
\tan7\theta=\frac{\sin7\theta}{\cos7\theta}.
\]
令
\[
T=\tan\theta=\frac{S}{C},
\]
则
\begin{align*}
\tan7\theta
&=\frac{7T-35T^3+21T^5-T^7}{1-21T^2+35T^4-7T^6}.
\end{align*}

令 $\tan7\theta=1$,得到
\[
\frac{7T-35T^3+21T^5-T^7}{1-21T^2+35T^4-7T^6}=1,
\]
整理可得
\[
T^7-7T^6-21T^5+35T^4+35T^3-21T^2-7T+1=0.
\]
这正是题目所给的多项式方程,其中 $T=x$。

由 $\tan7\theta=1$,
\[
7\theta=\frac{\pi}{4}+n\pi,
\]
从而
\[
\theta=\frac{(4n+1)\pi}{28},\quad n=0,1,2,3,4,5,6.
\]

因此方程的所有实数解为
\begin{align*}
x_0&=\tan\frac{\pi}{28},\\
x_1&=\tan\frac{5\pi}{28},\\
x_2&=\tan\frac{9\pi}{28},\\
x_3&=\tan\frac{13\pi}{28},\\
x_4&=\tan\frac{17\pi}{28},\\
x_5&=\tan\frac{21\pi}{28}=-1,\\
x_6&=\tan\frac{25\pi}{28}.
\end{align*}
\end{solution}

    \question 
    \begin{parts}
    \part 证明:
    \[
    \sin7\theta = 7\sin\theta - 56\sin^3\theta + 112\sin^5\theta - 64\sin^7\theta
    \]
    \begin{solution}
        从棣莫弗定理出发,
        \[
        (\cos\theta + i\sin\theta)^7 = \cos 7\theta + i\sin 7\theta
        \]
        由二项式定理展开左式,
        \[
        (\cos\theta + i\sin\theta)^7 = \sum_{k=0}^{7} \comb{7}{k} \cos^{7-k} \theta (i\sin\theta)^k
        \]
        比较虚部项系数得
        \[
        \sin 7\theta =  \comb{7}{1} \cos^6\theta \sin\theta - \comb{7}{3} \cos^4\theta \sin^3\theta + \comb{7}{5} \cos^2\theta \sin^5\theta - \comb{7}{7} \sin^7\theta
        \]
        由恒等式 \(\cos^2\theta = 1 - \sin^2\theta\),令 \(s = \sin\theta\),则有
        \[
        \sin 7\theta = 7(1 - s^2)^3 s - 35(1 - s^2)^2 s^3 + 21(1 - s^2) s^5 - s^7=7s - 56s^3 + 112s^5 - 64s^7
        \]
        故得证。
    \end{solution}
    \part 据此,解方程
    \[
    1 + 7x - 56x^3 + 112x^5 - 64x^7 = 0
    \]
    \begin{solution}
        原方程
        \[
        1 + 7x - 56x^3 + 112x^5 - 64x^7 = 0
        \]
        可写为
        \[
        1 + \sin 7\theta = 0 \Rightarrow \sin 7\theta = -1
        \]
        解得
        \[
        7\theta = \dots,-\frac{5\pi}{2},-\frac{\pi}{2},\frac{3\pi}{2},\frac{7\pi}{2},\dots
        \Rightarrow  \theta = \dots,-\frac{5\pi}{14},-\frac{\pi}{14},\frac{3\pi}{14},\frac{7\pi}{14},\dots
        \]
        所以\[
        x=\sin \theta = \sin\left(-\frac{5\pi}{14}\right),\sin\left(-\frac{\pi}{14}\right),\sin \frac{3\pi}{14},\sin \frac{7\pi}{14}=-0.901, -0.223, 0.623, 1
        \]
    \end{solution}
    \part 通过构造合适的多项式方程式,证明
    \[
    \csc^2\left(\frac{\pi}{7}\right)+\csc^2\left(\frac{2\pi}{7}\right)+\csc^2\left(\frac{3\pi}{7}\right)=8.
    \]
\begin{solution}
由
\[
\sin 7\theta=0
\]
可得解
\[
\theta=0,\ \frac{\pi}{7},\ \frac{2\pi}{7},\ \frac{3\pi}{7},\ \dots
\]
利用三角恒等式
\[
\sin7\theta
=7\sin\theta-56\sin^3\theta+112\sin^5\theta-64\sin^7\theta,
\]
得
\[
-\sin\theta\bigl(64\sin^6\theta-112\sin^4\theta+56\sin^2\theta-7\bigr)=0.
\]

对非零解 $\theta\neq0$,令
\[
Z=\sin^2\theta,
\]
则
\[
64Z^3-112Z^2+56Z-7=0.
\]
设该三次方程的三个正根为
\[
\alpha=\sin^2\frac{\pi}{7},\quad
\beta=\sin^2\frac{2\pi}{7},\quad
\gamma=\sin^2\frac{3\pi}{7}.
\]

由韦达定理,
\begin{align*}
\alpha+\beta+\gamma &= \frac{112}{64}=\frac{7}{4},\\
\alpha\beta+\beta\gamma+\gamma\alpha &= \frac{56}{64}=\frac{7}{8},\\
\alpha\beta\gamma &= \frac{7}{64}.
\end{align*}

注意到
\[
\sin\frac{\pi}{7}=\sin\frac{6\pi}{7},\quad
\sin\frac{2\pi}{7}=\sin\frac{5\pi}{7},\quad
\sin\frac{3\pi}{7}=\sin\frac{4\pi}{7},
\]
因此上述三根正好对应所需角度。

于是
\begin{align*}
\csc^2\frac{\pi}{7}+\csc^2\frac{2\pi}{7}+\csc^2\frac{3\pi}{7}
&=\frac{1}{\alpha}+\frac{1}{\beta}+\frac{1}{\gamma}\\
&=\frac{\alpha\beta+\beta\gamma+\gamma\alpha}{\alpha\beta\gamma}\\
&=\frac{\frac{7}{8}}{\frac{7}{64}}\\
&=8.
\end{align*}
证毕。
\end{solution}
    \end{parts}

\question
\begin{enumerate}
\item[(a)] 证明
\[
(1+i\tan\theta)^4+(1-i\tan\theta)^4=\frac{2\cos4\theta}{\cos^4\theta}.
\]

\item[(b)] 利用(a)的结果,通过构造合适的多项式,进一步证明:
\begin{enumerate}
\item[i.] $\tan^2\left(\frac{\pi}{8}\right)\tan^2\left(\frac{3\pi}{8}\right)=1$
\item[ii.] $\tan^2\left(\frac{\pi}{8}\right)+\tan^2\left(\frac{3\pi}{8}\right)=6$
\end{enumerate}
\end{enumerate}

\begin{solution}
\textbf{(a)}  
从左边开始:
\begin{align*}
(1+i\tan\theta)^4+(1-i\tan\theta)^4
&=\left(1+i\frac{\sin\theta}{\cos\theta}\right)^4+\left(1-i\frac{\sin\theta}{\cos\theta}\right)^4 \\
&=\left(\frac{\cos\theta+i\sin\theta}{\cos\theta}\right)^4
 +\left(\frac{\cos\theta-i\sin\theta}{\cos\theta}\right)^4 \\
&=\frac{(\cos\theta+i\sin\theta)^4+(\cos\theta-i\sin\theta)^4}{\cos^4\theta}.
\end{align*}

由棣莫弗定理,
\[
(\cos\theta\pm i\sin\theta)^4=\cos4\theta\pm i\sin4\theta.
\]
因此
\begin{align*}
(1+i\tan\theta)^4+(1-i\tan\theta)^4
&=\frac{\cos4\theta+i\sin4\theta+\cos4\theta-i\sin4\theta}{\cos^4\theta} \\
&=\frac{2\cos4\theta}{\cos^4\theta},
\end{align*}
结论得证。

\medskip
\textbf{(b)}  
当 $\cos4\theta=0$ 时,由(a)可得
\[
(1+i\tan\theta)^4+(1-i\tan\theta)^4=0.
\]
令 $z=i\tan\theta$,则
\[
(1+z)^4+(1-z)^4=0.
\]
展开得
\begin{align*}
(1+4z+6z^2+4z^3+z^4)+(1-4z+6z^2-4z^3+z^4)&=0 \\
2+12z^2+2z^4&=0,
\end{align*}
即
\[
z^4+6z^2+1=0.
\]

设其四个根为
\[
i\tan\frac{\pi}{8},\quad i\tan\frac{3\pi}{8},\quad
i\tan\frac{5\pi}{8},\quad i\tan\frac{7\pi}{8}.
\]

\textbf{(i)}  
由常数项与首项系数之比,
\[
\alpha\beta\gamma\delta=1.
\]
利用 $\tan(\pi-\theta)=-\tan\theta$,
\[
\tan^2\left(\frac{\pi}{8}\right)\tan^2\left(\frac{3\pi}{8}\right)=1.
\]

\textbf{(ii)}  
由二次项系数,
\[
\alpha\beta+\alpha\gamma+\cdots+\gamma\delta=6,
\]
同样配对并利用对称性,可得
\[
\tan^2\left(\frac{\pi}{8}\right)+\tan^2\left(\frac{3\pi}{8}\right)=6.
\]
\end{solution}

    \question 设 $w = \cos\frac{2\pi}{5} + i\sin\frac{2\pi}{5}$,
(a) 证明 \(1+w+w^2+w^3+w^4=0\). 
(b) 求 \((1-w)(1-w^2)(1-w^3)(1-w^4)\). 
(c) 证明 \((1-w)(1-w^4) = 4\sin^2\frac{\pi}{5}\). 
(d) 证明 \((1-w^2)(1-w^3) = 4\sin^2\frac{2\pi}{5}\). 
(e) 证明 \(\sin\frac{\pi}{5}\sin\frac{2\pi}{5} = \frac{\sqrt{5}}{4}\).
\begin{solution}
(a) 因为 $w^5 = (\cos\frac{2\pi}{5} + i\sin\frac{2\pi}{5})^5 = \cos 2\pi + i \sin 2\pi = 1$,且 $w \neq 1$,所以
\[
w^5 - 1 = 0 \implies (w-1)(1+w+w^2+w^3+w^4) = 0 \implies 1+w+w^2+w^3+w^4 = 0.
\]

(b) 考虑乘积 $P = (1-w)(1-w^2)(1-w^3)(1-w^4)$。使用共轭与代数关系:
\begin{align*}
P &= (1-w)(1-w^2)(1-w^3)(1-w^4) \\
&= (1-w)(1-w^4)(1-w^2)(1-w^3) \\
&= [2-(w+w^4)][2-(w^2+w^3)] \\
&= 4 - 2(w+w^4+w^2+w^3) + (w+w^4)(w^2+w^3) \\
&= 4 - 2(-1) + (w+w^4)(w^2+w^3) \quad (\text{由 } 1+w+\cdots+w^4=0)\\
&= 6 + (w^3+w^4+w^6+w^7) \\
&= 6 + (w^3+w^4+w+w^2) \quad (w^5=1 \implies w^6=w, w^7=w^2)\\
&= 6 + (-1) \\
&= 5.
\end{align*}

(c) $(1-w)(1-w^4)$:
\begin{align*}
(1-w)(1-w^4) &= 1 - w - w^4 + w^5 = 2 - (w + w^4) \\
&= 2 - \left(\cos\frac{2\pi}{5} + i\sin\frac{2\pi}{5} + \cos\frac{8\pi}{5} + i\sin\frac{8\pi}{5}\right) \\
&= 2 - 2\cos\frac{2\pi}{5} \\
&= 2\left(1 - \cos\frac{2\pi}{5}\right) = 2\cdot 2 \sin^2\frac{\pi}{5} = 4\sin^2\frac{\pi}{5}.
\end{align*}

(d) $(1-w^2)(1-w^3)$:
\begin{align*}
(1-w^2)(1-w^3) &= 1 - w^2 - w^3 + w^5 = 2 - (w^2 + w^3) \\
&= 2 - (\cos\frac{4\pi}{5} + \cos\frac{6\pi}{5}) \\
&= 2 - 2 \cos\frac{4\pi}{5} \\
&= 2\left(1 - \cos\frac{4\pi}{5}\right) = 2\cdot 2 \sin^2\frac{2\pi}{5} = 4\sin^2\frac{2\pi}{5}.
\end{align*}

(e) 由前面结果可知:
\begin{align*}
16 \sin^2\frac{\pi}{5} \sin^2\frac{2\pi}{5} &= (1-w)(1-w^2)(1-w^3)(1-w^4) = 5, \\
\sin^2\frac{\pi}{5} \sin^2\frac{2\pi}{5} &= \frac{5}{16}, \\
\sin\frac{\pi}{5} \sin\frac{2\pi}{5} &= \pm \frac{\sqrt{5}}{4}.
\end{align*}

因为 $0 < \frac{\pi}{5} < \frac{\pi}{2}$ 且 $0 < \frac{2\pi}{5} < \frac{\pi}{2}$,所以
\[
\sin\frac{\pi}{5} > 0, \quad \sin\frac{2\pi}{5} > 0,
\]
于是
\[
\sin\frac{\pi}{5} \sin\frac{2\pi}{5} > 0 \implies \sin\frac{\pi}{5} \sin\frac{2\pi}{5} = \frac{\sqrt{5}}{4}.
\]
\end{solution}


    \question 设 \(S_n=a^n+b^n\),其中 \(a,b\) 是方程 \(x^2+x+1=0\) 的根,求
        \[
        \sum_{n=0}^{1729}(-1)^n S_n
        \]

    \begin{solution}
        方程根为 \(a = \omega,\ b = \omega^2\),其中 \(\omega = e^{2\pi i/3}\),满足 \(\omega^3 = 1\)。
        
        设 \(S_n = \omega^n + \omega^{2n}\),发现 \(S_n\) 是周期为 $3$ 的数列:
        \[
        S_0 = 2,\quad S_1 = -1,\quad S_2 = -1
        \]
        将原式写成
        \[
        \sum_{k=0}^{576} \left[(-1)^{3k} S_{3k} + (-1)^{3k+1} S_{3k+1} + (-1)^{3k+2} S_{3k+2}\right] + (-1)^{1728} S_{1728} + (-1)^{1729} S_{1729}
        \]
        前面 $576$ 组和为 $0,$剩余
        \[
        (-1)^{1728} S_{1728} + (-1)^{1729} S_{1729} = 1 \cdot 2 + (-1) \cdot (-1) = 3
        \]
    \end{solution}

    \question 设 \(\omega \in \mathbb{C}\)、\(\omega \ne 1\)、且 \(\omega^7 = 1\),计算:
    \[
    \prod_{k=0}^{6} (\omega^{2k} + 2\omega^k + 4)
    \]
    \begin{solution}
        设方程 \(x^7 - 1 = 0\) 的根为 \(1,\omega,\omega^2,\dots,\omega^6\),则
        \[
        \prod_{k=0}^{6} \left((\omega^k)^2 + 2\omega^k + 4\right)
        =\prod_{k=0}^{6} \frac{(\omega^k)^3 - 8}{\omega^k - 2}
        \]
        由于 $(\omega^0,\omega^3,\omega^6,\omega^9,\omega^{12},\omega^{15},\omega^{18})=(\omega^0,\omega^3,\omega^6,\omega^2,\omega^{5},\omega^{1},\omega^{4})$,因此:
        \[
        \prod_{k=0}^{6} ((\omega^k)^3 - 8) = \prod_{j=0}^{6} (\omega^j - 8)
        \]
        又因为 \(x^7 - 1 = \prod_{j=0}^{6} (x - \omega^j)\),令 \(x = 8\),得
        \[
        \prod_{j=0}^{6} (\omega^j - 8) = -(8^7 - 1) = 1 - 8^7
        \]
        同理,令 \(x = 2\)得
        \[
        \prod_{j=0}^{6} (\omega^j - 2) = -(2^7 - 1) = 1 - 2^7
        \]
        因此原式为:
        \[
        \frac{1 - 8^7}{1 - 2^7}=16513
        \]
    \end{solution}

    \question 令 $\omega = \cos\dfrac{2\pi}{111} + i \sin\dfrac{2\pi}{111}$, 其中 $i = \sqrt{-1}$, 求 \[ \sum_{k=1}^{110} \frac{\omega^{2k}}{\omega^k - 1}\]
    \begin{solution}
        由$\omega = \cos\dfrac{2\pi}{111} + i \sin\dfrac{2\pi}{111}$可知
        \[
        \omega^{111}-1 = 0 \Rightarrow \sum_{k=0}^{110} \omega^k = 0
        \]
        因此
        \[
        \sum_{k=1}^{110} \frac{\omega^{2k}}{\omega^k - 1} = \sum_{k=1}^{110} \left(\omega^k + 1 + \frac{1}{\omega^k - 1}\right) = -1 + 110 + \sum_{k=1}^{110} \frac{1}{\omega^k - 1} \tag{1}
        \]
        设 
        \[f(x) = \sum_{k=0}^{110} x^k = \prod_{k=1}^{110} (x - \omega^k)
        \]
        则
        \[ f'(x) = \sum_{k=1}^{110} k x^{k-1} = \sum_{m=1}^{110} \prod_{k=1, k\ne m}^{110} (x - \omega^k)
        \]
        于是
        \[
        g(x) = \frac{f'(x)}{f(x)} = \sum_{k=1}^{110} \frac{1}{x - \omega^k}
        \]
        令$x=1$,
        \[
        g(1) = \frac{111 \cdot \frac{110}{2}}{111} = \sum_{k=1}^{110} \frac{1}{1 - \omega^k}\Rightarrow \sum_{k=1}^{110} \frac{1}{\omega^k - 1} = -55
        \]
        代入 (1) 得
        \[
        \sum_{k=1}^{110} \frac{\omega^{2k}}{\omega^k - 1} = -1 + 110 - 55 = 54
        \]
    \end{solution}

    \question 设 \(z=\cos\dfrac{\pi}{6}+i\sin\dfrac{\pi}{6}\),求
        \[
        \left|1+2z+3z^2+\cdots+2016z^{2015}\right|
        \]
    \begin{solution}
        记
        \[
        S = 1 + 2z + 3z^2 + \cdots + 2016z^{2015} 
        \]
        考虑函数
        \[
        f(x) = x + x^2 + \cdots + x^{2016} = \frac{x(x^{2016} - 1)}{x - 1},\quad 
        \]
        则有$$S = f'(z)$$
        求导:
        \[
        f'(x) = \frac{[(x^{2016} - 1) + 2016x^{2016}](x - 1) - x(x^{2016} - 1)}{(x - 1)^2}
        \]
        设 \(z = \cos\dfrac{\pi}{6} + i\sin\dfrac{\pi}{6} = e^{\frac{i\pi}{6}}\),由棣莫弗定理得
        \[
        z^{2016} = \left(e^{i\pi/6}\right)^{2016} = e^{i\cdot336\pi} = 1
        \]
        于是
        \[
        f'(z) = \frac{2016(z - 1)}{(z - 1)^2} = \frac{2016}{z - 1}
        \Rightarrow |S| = \frac{2016}{|z - 1|}
        \]
        而
        \[
        z - 1 = \left(\frac{\sqrt{3}}{2} - 1\right) + \frac{1}{2}i
        \]
        \[
        |z - 1| = \sqrt{\left(\frac{\sqrt{3}}{2} - 1\right)^2 + \left(\frac{1}{2}\right)^2} = \sqrt{2 - \sqrt{3}} = \frac{\sqrt6+\sqrt2}{2}
        \]
        所以
        \[
        |S| = \frac{2016\cdot2}{\sqrt6+\sqrt2}\cdot \frac{\sqrt6-\sqrt2}{\sqrt6-\sqrt2} = 1008\sqrt{2} + 1008\sqrt{6}
        \]
    \end{solution}

    \question 设 $\omega = e^{\frac{2\pi i}{n}}$,求
    \[ 
    -\sum_{k=1}^{n-1} \prod_{\substack{j=1 \\ j \neq k}}^{n-1} (\omega^k - \omega^j). 
    \]
    \begin{solution}
        设
        \[
        P(x)=\prod_{k=1}^{n-1}(x-\omega^k)=\frac{x^n-1}{x-1}=x^{n-1}+x^{n-2}+\cdots +x+1
        \]
        则
        \[
        P'(x)=(n-1)x^{n-2}+(n-2)x^{n-3}+\cdots + 1
        \]
        于是
        \[
        \sum_{k=1}^{n-1} \prod_{\substack{j=1 \\ j \neq k}}^{n-1} (\omega^k - \omega^j)=-\sum_{k=1}^{n-1} P'(\omega^k)
        \]
        观察到由于$\omega^n=1$,
        \[
        \sum_{k=0}^{n-1} \omega^{(n-1)k}
        =\begin{cases}
        n, \quad k=0 \\[2pt]
        \dfrac{1-\omega^{nk}}{1-\omega^k}=0,\quad k \neq 0 \\
        \end{cases}
        \]
        因此
        \[
        \sum_{k=0}^{n-1} P'(\omega^k)=0+\cdots+0+n=n
        \]
        故
        \[
        \sum_{k=1}^{n-1} \prod_{\substack{j=1 \\ j \neq k}}^{n-1} (\omega^k - \omega^j)=-\sum_{k=1}^{n-1} P'(\omega^k)=-(n-P'(1))=\frac{n(n-1)}{2}-n=\frac{n(n-3)}{2}
        \]
    \end{solution}

    \question 设复数平面上三点 $A(\alpha)$, $B(\beta)$, $C(\gamma)$ 可连成正三角形 $ABC$,  
已知 $\alpha$, $\beta$, $\gamma$ 满足
\[
\alpha^4 - 2\alpha^3\beta + (\beta^2 - 4)\alpha^2 + 8\alpha\gamma - 4\gamma^2 = 0
\]
且 $\alpha$ 的实部和虚部均为正数,  
当 $\triangle ABC$ 的重心 $G$ 为 $\frac{\alpha}{2^{110}}$ 时,求 $\beta$ 及 $\gamma$ 各为何?

\begin{solution}
由 $G = \frac{\alpha+\beta+\gamma}{3} = \frac{\alpha}{2^{110}}$,可得
\[
\beta + \gamma = \frac{3\alpha}{2^{110}} - \alpha.
\]
又因 $ABC$ 是正三角形,满足
\[
\beta = \alpha + (\gamma - \alpha)\omega,
\]
其中 $\omega = -\frac12 + \frac{\sqrt{3}}{2}i$。

将 $\beta$ 与 $\gamma$ 表示式代入原方程,化简可得
\[
\alpha^4 - 2\alpha^3\left[\alpha + (\gamma - \alpha)\omega\right] + \left[\left(\alpha + (\gamma - \alpha)\omega\right)^2 - 4\right]\alpha^2 + 8\alpha\gamma - 4\gamma^2 = 0.
\]
经过整理与代数运算,可解得
\[
\alpha = 2 + 2i.
\]
代回 $\beta+\gamma$ 与正三角形条件,可得
\[
\beta = 1 + (2+\sqrt{3})i, \quad \gamma = 1 + (2-\sqrt{3})i
\]
\textcolor{red}{(待验证)}
\end{solution}

\question 设 $P(z)$ 是一个 $n$ 次复系数多项式,其所有零点都位于复平面的单位圆上。证明多项式
\[
\tilde{P}(z)=2zP'(z)-nP(z)
\]
的所有零点也都位于同一单位圆上。

\begin{solution}
不妨只考虑首项系数为 $1$ 的多项式。设
\[
P(z)=(z-\alpha_1)(z-\alpha_2)\cdots(z-\alpha_n),
\]
其中 $|\alpha_j|=1$,$j=1,2,\ldots,n$,并允许这些复数 $\alpha_1,\alpha_2,\ldots,\alpha_n$ 相同。

计算得
\[
\tilde{P}(z)=2zP'(z)-nP(z)
=(z+\alpha_1)(z-\alpha_2)\cdots(z-\alpha_n)+\cdots+(z-\alpha_1)(z-\alpha_2)\cdots(z+\alpha_n)。
\]
因此
\[
\frac{\tilde{P}(z)}{P(z)}=\sum_{k=1}^n \frac{z+\alpha_k}{z-\alpha_k}。
\]

注意到对任意复数 $z,\alpha$ 且 $z\neq\alpha$,都有
\[
\operatorname{Re}\frac{z+\alpha}{z-\alpha}
=\frac{|z|^2-|\alpha|^2}{|z-\alpha|^2}。
\]
于是,在当前情形下,
\[
\operatorname{Re}\frac{\tilde{P}(z)}{P(z)}
=\sum_{k=1}^n \frac{|z|^2-|\alpha_k|^2}{|z-\alpha_k|^2}。
\]
由于 $|\alpha_k|=1$,可知当 $|z|\neq1$ 时,上式不为零。

因此,若 $\tilde{P}(z)=0$,则必有
\[
\operatorname{Re}\frac{\tilde{P}(z)}{P(z)}=0,
\]
从而推出 $|z|=1$。这说明 $\tilde{P}(z)$ 的所有零点也都位于单位圆上。
\end{solution}


    \question 复数平面上以原点为中心的单位圆中,有一内接四边形,其顶点为 $z_1, z_2, z_3, z_4$,设
    \[
    S_n = z_1^n + z_2^n + z_3^n + z_4^n
    \]
    且 $S_1 = 0$ 且 $S_2 = 1$,
    \begin{parts}
    \part 证明:若
    \[
    |z_1|=|z_2|=|z_3|=|z_4|=1,\quad z_1+z_2+z_3+z_4=0
    \]
    则 $z_1,z_2,z_3,z_4$ 为一个矩形的顶点。
    \begin{solution}
        设
        \[
        f(z) = (z-z_1)(z-z_2)(z-z_3)(z-z_4) = z^4 + a z^3 + b z^2 + c z + d
        \]
        由 $S_1 = z_1+z_2+z_3+z_4 = 0$ 得 $a=0$,又由于 $|z_k|=1$,有 $\bar z_k = \dfrac{1}{z_k}$,于是
        \[
        \overline{z_1} + \overline{z_2} + \overline{z_3} + \overline{z_4} = 0
        \quad\Rightarrow\quad
        \frac1{z_1}+\frac1{z_2}+\frac1{z_3}+\frac1{z_4} = 0
        \]
        即
        \[
        z_2 z_3 z_4 + z_1 z_3 z_4 + z_1 z_2 z_4 + z_1 z_2 z_3 = 0
        \]
        所以 $c=0$,因此
        \[
        f(z) = z^4 + b z^2 + d
        \]
        为偶函数,四根关于原点成对对称,于是可设
        \[
        z_1+z_3=0,z_2+z_4=0
        \]
        此时两对顶点互为相反数,对角线过原点,且 $|z_1|=|z_2|=1$,故为内接矩形。
    \end{solution}
    \part 计算该矩形的面积。
    \begin{solution}
        设
        \[
        z_1 = a+bi,\quad z_2 = -a+bi,\quad z_3 = -a-bi,\quad z_4 = a-bi
        \]
        由 $S_2 = z_1^2+z_2^2+z_3^2+z_4^2 = 4(a^2-b^2) = 1$ 得
        \[
        a^2 - b^2 = \frac14
        \]
        又因 $|z_1|=1$,有 $a^2+b^2=1$,解得
        \[
        a = \frac{\sqrt5}{2\sqrt2},\quad b = \frac{\sqrt3}{2\sqrt2}
        \]
        故矩形面积为
        \[
        2a \cdot 2b = \sqrt{\frac52} \cdot \sqrt{\frac32} = \frac{\sqrt{15}}{2}
        \]
    \end{solution}
    \end{parts}

\end{questions}


